\documentclass[11pt]{article}
%用ctex显示中文并用fandol主题
\usepackage[fontset=fandol]{ctex}
\setmainfont{CMU Serif} %能显示大量外文字体
\xeCJKsetup{CJKmath=true} %数学模式中可以输入中文

%AMS全家桶,\DeclareMathOperator依赖之
\usepackage{amsmath,amssymb,amsthm,amsfonts,amscd}
\usepackage{pgfplots,tikz,tikz-cd} %用来画交换图
\usepackage{bm} %粗体字母(含希腊字母)
\everymath{\displaystyle} %全体公式为行间形式

%纸张上下左右页边距
\usepackage[a4paper,left=1cm,right=1cm,top=1.5cm,bottom=1.5cm]{geometry}
%生成书签和目录上的超链接
\usepackage[colorlinks=true,linkcolor=blue,filecolor=blue,urlcolor=blue,citecolor=cyan]{hyperref}
%各种列表环境的行距
\usepackage{enumitem}
\setenumerate[1]{itemsep=0pt,partopsep=0pt,parsep=\parskip,topsep=0pt}
\setenumerate[2]{itemsep=0pt,partopsep=0pt,parsep=\parskip,topsep=0pt}
\setenumerate[3]{itemsep=0pt,partopsep=0pt,parsep=\parskip,topsep=0pt}
\setitemize[1]{itemsep=0pt,partopsep=0pt,parsep=\parskip,topsep=5pt}
\setdescription{itemsep=0pt,partopsep=0pt,parsep=\parskip,topsep=5pt}
\setlength\belowdisplayskip{2pt}
\setlength\abovedisplayskip{2pt}

%左右配对符号
\newcommand{\br}[1]{\!\left(#1\right)} %括号
\newcommand{\cbr}[1]{\left\{#1\right\}} %大括号
\newcommand{\abr}[1]{\left<#1\right>} %尖括号(内积)
\newcommand{\bbr}[1]{\left[#1\right]} %中括号
\newcommand{\abbr}[1]{\left(#1\right]} %左开右闭区间
\newcommand{\babr}[1]{\left[#1\right)} %左闭右开区间
\newcommand{\abs}[1]{\left|#1\right|} %绝对值
\newcommand{\norm}[1]{\left\|#1\right\|} %范数
\newcommand{\floor}[1]{\left\lfloor#1\right\rfloor} %下取整
\newcommand{\ceil}[1]{\left\lceil#1\right\rceil} %上取整
%常用数集简写
\newcommand{\R}{\mathbb{R}} %实数域
\newcommand{\N}{\mathbb{N}} %自然数集
\newcommand{\Z}{\mathbb{Z}} %整数集
\newcommand{\C}{\mathbb{C}} %复数域
\newcommand{\F}{\mathbb{F}} %一般数域
\newcommand{\kfield}{\Bbbk} %域
\newcommand{\K}{\mathbb{K}} %域
\newcommand{\Q}{\mathbb{Q}} %有理数域
\newcommand{\Pprime}{\mathbb{P}} %全体素数,或概率
%范畴记号
\newcommand{\Ccat}{\mathsf{C}}
\newcommand{\Grp}{\mathsf{Grp}} %群范畴
\newcommand{\Ab}{\mathsf{Ab}} %交换群范畴
\newcommand{\Ring}{\mathsf{Ring}} %(含幺)环范畴
\newcommand{\Set}{\mathsf{Set}} %集合范畴
\newcommand{\Mod}{\mathsf{Mod}} %模范畴
\newcommand{\Vect}{\mathsf{Vect}} %向量空间范畴
\newcommand{\Alg}{\mathsf{Alg}} %代数范畴
\newcommand{\Comm}{\mathsf{Comm}} %交换
%代数集合
\DeclareMathOperator{\Hom}{Hom} %同态
\DeclareMathOperator{\End}{End} %自同态
\DeclareMathOperator{\Iso}{Iso} %同构
\DeclareMathOperator{\Aut}{Aut} %自同构
\DeclareMathOperator{\Inn}{Inn} %内自同构
% \DeclareMathOperator{\inv}{Inv}
\DeclareMathOperator{\GL}{GL} %一般线性群
\DeclareMathOperator{\SL}{SL} %特殊线性群
\DeclareMathOperator{\GF}{GF} %Galois域
%正体符号
\renewcommand{\i}{\mathrm{i}} %本产生无点i
\newcommand{\id}{\mathrm{id}} %恒等映射
\newcommand{\e}{\mathrm{e}} %自然常数e
\renewcommand{\d}{\mathrm{d}} %微分符号,本产生重音符号
\newcommand{\D}{\partial} %偏导符号
\newcommand{\diff}[2]{\frac{\d #1}{\d #2}}
\newcommand{\Diff}[2]{\frac{\D #1}{\D #2}}
%运算符(分析)
\DeclareMathOperator{\Arg}{Arg} %辐角
\DeclareMathOperator{\re}{Re} %实部
\DeclareMathOperator{\im}{im} %像,虚部
\DeclareMathOperator{\grad}{grad} %梯度
\DeclareMathOperator{\lcm}{lcm} %最小公倍数
\DeclareMathOperator{\sgn}{sgn} %符号函数
\DeclareMathOperator{\conv}{conv} %凸包
\DeclareMathOperator{\supp}{supp} %支撑
\DeclareMathOperator{\Log}{Log} %广义对数函数
\DeclareMathOperator{\card}{card} %集合的势
\DeclareMathOperator{\Res}{Res} %留数
%运算符(代数,几何,数论)
\newcommand{\Span}{\mathrm{span}} %张成空间
\DeclareMathOperator{\tr}{tr} %迹
\DeclareMathOperator{\rank}{rank} %秩
\DeclareMathOperator{\charfield}{char} %域的特征
\DeclareMathOperator{\codim}{codim} %余维度
\DeclareMathOperator{\coim}{coim} %余维度
\DeclareMathOperator{\coker}{coker} %余维度
\DeclareMathOperator{\Spec}{Spec} %留数
\newcommand{\Obj}{\mathrm{Obj}} %对象类
\newcommand{\Mor}{\mathrm{Mor}} %态射类
\newcommand{\Cen}{C} %群/环的中心 或记\mathrm{Cen}
\newcommand{\opcat}{^{\mathrm{op}}}
%简写
\newcommand{\hyphen}{\textrm{-}}
\newcommand{\ds}{\displaystyle} %行间公式形式
\newcommand{\ve}{\varepsilon} %手写体ε
\newcommand{\rev}{^{-1}\!} %逆
\newcommand{\T}{^{\mathsf{T}}} %转置
\renewcommand{\H}{^{\mathsf{H}}} %共轭转置
\newcommand{\adj}{^\lor} %伴随
\newcommand{\dual}{^\vee} %对偶
\DeclareMathOperator{\lhs}{LHS}
\DeclareMathOperator{\rhs}{RHS}
\newcommand{\hint}[1]{{\small (#1)}} %提示
\newcommand{\why}{\textcolor{red}{(Why?)}}
\newcommand{\tbc}{\textcolor{red}{(To be continued...)}} %未完待续

%定理环境(随笔记形式更改)
\newtheorem{definition}{定义}
\newtheorem{remark}{注}
\newtheorem{example}{例}
\makeatletter
\@ifclassloaded{article}{
    \newtheorem{theorem}{定理}[section]
}{
    \newtheorem{theorem}{定理}[chapter]
}
\makeatother
\newtheorem{lemma}[theorem]{引理}
\newtheorem{proposition}[theorem]{命题}
\newtheorem{corollary}[theorem]{推论}
\newtheorem{property}[theorem]{性质}

\title{极值图论笔记 (编纂中)}
\author{授课人: 彭兴, 窦春阳; 编辑: 章亦流}
\date{\today}

\begin{document}
\maketitle

\section{Lecture 1: Tur\'an Theorem and Erd\"os-Stone-Simonovits Theorem}
\begin{definition}[Tur\'an数]
    
\end{definition}
\begin{theorem}[Mantel, 1902]
    
\end{theorem}
\begin{proof}
    
\end{proof}
\begin{proof}
    
\end{proof}
\begin{proof}
    
\end{proof}
\begin{theorem}[Tur\'an, 1941]
    
\end{theorem}
\begin{proof}
    
\end{proof}
\begin{proof}[Zykov]
    
\end{proof}
\begin{theorem}[Erd\"os-Stone-Simonovits]\label{ESSThm}
    
\end{theorem}
\begin{lemma}
    
\end{lemma}
\begin{proof}
    
\end{proof}
\begin{proof}[定理\ref{ESSThm}的证明]
    
\end{proof}

\newpage
\section{Lecture 2: Erd\"os-Stone-Simonovits Theorem and Szemer\'edi's Regularity Lemma}
\begin{definition}[边密度]
    顶点子集$A,B\subset V$间的边密度$d(A,B):=\frac{e(A,B)}{\abs{A}\abs{B}}$.
\end{definition}
\begin{definition}[$\varepsilon$-正则]
    对于顶点子集$A,B\subset V$,称$(A,B)$是$\varepsilon$-正则对,若$\forall A'\subset A, B'\subset B$,其中$\abs{A'}\geq \varepsilon\abs{A}, \abs{B'}\geq \varepsilon\abs{B}$,都有$\abs{d(A',B')-d(A,B)}\leq \varepsilon$.称分划$P: V=\bigsqcup_{i=0}^k V_i$是$\varepsilon$-正则的,若$\sum_{(V_i, V_j)\text{不正则}}\frac{\abs{V_i}\abs{V_j}}{n^2}\leq \varepsilon$.

    称分划$P: V=\bigsqcup_{i=0}^k V_i$为$\varepsilon$-正则的均分,若$\abs{V_1}=\dots=\abs{V_k},\abs{V_0}\leq \varepsilon n$,且$\cbr{(V_i,V_j)|V_i,V_j\in P}$中的非$\varepsilon$-正则对有$\leq\varepsilon k^2$个.
\end{definition}
\begin{theorem}[Szemer\'edi正则性引理]\label{SRL}
    $\forall \varepsilon>0\forall m\in \N_+\exists n_0(\varepsilon,m)\exists M\in \N_+$使得对于任意顶点数$n\geq n_0$的图$G$,都有分划$V=\bigsqcup_{i=0}^k V_i$,满足
    \begin{enumerate}[label=(\arabic*)]
        \item $\abs{V_0}\leq \varepsilon n$
        \item $\abs{V_1}=\dots=\abs{V_k}$
        \item $m\leq k\leq M$,其中$M=\left.2^{2^{{\dots}^2}}\right\}\approx\varepsilon^{-5}$个$2$.
        \item $\cbr{(V_i,V_j)|V_i,V_j\in P}$中的非$\varepsilon$-正则对有至少一个,至多$\varepsilon k^2$个.
    \end{enumerate}
    即$P$是一个$\varepsilon$-正则的均分.
\end{theorem}
\begin{definition}[缩略图和爆破图]
    给定$d\in \abbr{0,1}$,对于$\varepsilon$-正则均分$V=\bigsqcup_{i=0}^k V_i$,记$\ell=\abs{V_k}$,构造图$R, V(R)=\cbr{v_1,\dots,v_k}, E(R)=\cbr{v_iv_j|(V_i,V_j) \varepsilon\text{-正则且}d(V_i,V_j)\geq d}$,则称$R$是$G$的参数$\varepsilon,\ell,d$的缩略图.

    对于$s\in \N_+, R$的爆破图$R(s)$即$k$部$s$个点$A_1,\dots,A_k$,其中$A_i,A_j$之间完备$\iff v_iv_j\in E(R)$.
\end{definition}
\begin{lemma}[图嵌入引理]\label{GEL}
    $\forall d\in \abbr{0,1},\Delta\geq 1, \exists \varepsilon_0(d,\Delta)$使得:若图$G,H$满足(1)$\Delta(H)\leq \Delta$;\\
    (2)$G$有参数为$\varepsilon\leq \varepsilon_0, \ell\geq 2s\Delta/d$和$d$的缩略图$R$;(3)$H\subset R(s)$,则$H\subset G$.
\end{lemma}
\begin{proof}[定理\ref{ESSThm}的另一证明]
    
\end{proof}
\begin{definition}
    
\end{definition}
\begin{lemma}
    
\end{lemma}
\begin{lemma}
    
\end{lemma}
\begin{proof}
    
\end{proof}
\begin{proof}[定理\ref{SRL}的证明]
    
\end{proof}
\begin{theorem}
    
\end{theorem}

\newpage
\section{Lecture 3: Embedding Lemma, Counting Lemma and Removal Lemma}
\begin{lemma}
    $(A,B)$是$\varepsilon$-正则对,令$d=d(A,B)$, 若有$Y\subset B, \abs{Y}\geq \varepsilon \abs{B}$,则$A$中有$>(1-\varepsilon)\abs{A}$个顶点都在$Y$中有$\geq (d-\varepsilon)\abs{Y}$个邻点.
\end{lemma}
\begin{proof}
    取$X\subset A, \abs{X}\geq \varepsilon\abs{A}$则由$\varepsilon$-正则知$\abs{d(X,Y)-d}\leq \varepsilon$,故$e(X,Y)\geq (d-\varepsilon)\abs{X}\abs{Y}$
\end{proof}

\end{document}