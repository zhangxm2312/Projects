\documentclass[UTF8]{book}
    \usepackage{ctex,verbatim,amsfonts,geometry,titlesec}
    \geometry{bottom=1cm,left=1cm,right=1cm}
\begin{document}
\renewcommand{\chaptername}{第 {\thechapter}章}
\newcommand{\sectionname}{节}
\renewcommand{\figurename}{图}
\renewcommand{\tablename}{表}
\renewcommand{\bibname}{参考文献}
\renewcommand{\contentsname}{目~录}
\renewcommand{\listfigurename}{图~目~录}
\renewcommand{\listtablename}{表~目~录}
\renewcommand{\indexname}{索~引}
\renewcommand{\abstractname}{\Large{摘~要}}
\renewcommand{\partname}{第  {\thepart} 部分}
\newcommand{\keywords}[1]{\\ \\ \textbf{关~键~词}:#1}
\titleformat{\chapter}[block]{\center\Large\bf}{\chaptername}{20pt}{}
\titleformat{\section}[block]{\large\bf}{\thesection}{10pt}{}

\title{导读《浮生取义》}
\author{zhangxm2312@gmail.com}
\date{2019年9月28日}
\maketitle
%\setcounter{chapter}{-1}

\chapter{死或生}
\section{田野 \protect\footnote{出于书的内容结构考虑,笔记中部分章节分划与原书并不符合。}}
\begin{enumerate}
    \item 
    作者首先展示了两个例子:跳井自杀的少妇与上吊的小学生,由关于他们死亡的争端引入对社会的观察,并临时具体地定义了:\\
        \textbf{
        \begin{itemize}
            \item 冤枉:在公共生活中(尤指在家庭外)遭受到的不公正待遇。
            \item 委屈:在亲密关系者之间遭受到的不公正待遇。
        \end{itemize}}
    实际上,在两个事例中,无论是医院还是公安局都没有做出过多的反应——所谓“公安”,所负责的便是在公共领域内的“冤枉”。“学校——家”关系还是少妇家庭,两者均不在公共领域内。\\
    对于大多数人来说,不只是公共生活的安稳就能过好日子——家庭秩序应该更重要。自杀提出了这样的问题:
    \begin{center}
        \textbf{如何理解当代中国家庭秩序中的公正?}
    \end{center}
    
    \item 中国人的家庭纠纷外人一般不会,也无法插手其中,所谓“清官难断家务事”,这就反映了中国家庭秩序的排外性。然而在对自杀的讨论上,这成为了作者调查研究的一个阻碍。很显然,一位自杀者很有可能是遭受很大部分的“冤枉”而自杀,但被想当然的归结为家庭纠纷也是不可避免的(作者的调查证明了这一点)。\\
    一位诊所的医师也验证了这个说法:他也从不会去关注自杀者(即病人)的受伤成因,而且在县医院很近的精神病院的医师同样如此。即使是负责去解决病人精神问题的医生也不会去关心病人的内在成因,这和医生是否玩忽职守无关——因为,\textbf{委屈不是一个医学问题。}\\
    无论是公安局还是医院,在社会上离自杀现象最近的两个组织却也对其持有如此态度,正是因为社会公正和家庭秩序的界限分明。我们可以用归谬法如此推出:和每个人的精神卫生息息相关的自杀现象也得不到足够的公共关注,说明人民的日常生活既不是一个公共安全问题,也不是一个公共卫生问题——甚至不是一个社会公共问题。\\
    很多自杀的研究者认为,在中国,自杀是一个社会问题而非医学问题。之所以两者如此矛盾,是因为,自杀现象常常不是因为疾病,而是由于委屈和冤枉:它们常常不发生在任何公共的社会空间中,从而不涉及社会公正,不需要一个公共的权威来插手解决。\\
    因此,自杀是社会空间之外的一个社会问题,是公共政治领域之外的政治问题。自杀总与私人的公正与冤屈有关,那么,理解自杀问题的关键便变成了:
    \begin{center}
        \textbf{人们在日常生活中最在意的公正是什么?}
    \end{center}
    \newpage
    \item {调查经历和方法}\\
        作者调查的“华北某县”叫做孟陬县,2002年人口32万,作者在2000和2001年的六到八月,2002.9-2003.8均在此做田野调查。简单来说,作者在遭遇碰壁后,以当年其母亲在当地留下的人脉开始调查,并逐渐建立对全县各镇的联络网:通过每个镇的一位德高望重的“向导”对各个村调查。\\
        统计结果显示,2002年县内有61起个案,33女28男,其中女性平均37.7岁,男性平均40.1岁。根据作者估计,孟陬贤自杀率约为20人/10万人,比回龙观医院估计率(23人/10万人)\footnote{Micheal Phillips,Xianyun Li,Yanping ZHang,\emph{"Suicide Rate in China,1995-1999"},in The Lancet ,March 2002,vol 359.,issue 3909,pp.835-840}低。需要注意的是,统计结果有如下特点:
        \begin{itemize}
            \item 在61人内,只有一人死于“冤枉”。
            \item 女性自杀人数高于男性。这与西方国家男性自杀数更高非常不同。
            \item 男性自杀年龄比女性高,虽然并不明显。
        \end{itemize}
        作者对自杀的理解遵循了Douglas的方法:\textbf{尽量揭示社会和不同的人对自杀的文化意义的理解}\footnote{Jack Douglas,\emph{The Social Meaning of Suicide},Princeton University Press,1967}。\\
        作者的研究中心在于:
        \begin{itemize}
            \item 研究关于自杀的社会和文化话语
            \item 寻找导致自杀的真正原因
        \end{itemize}
\end{enumerate}

\section{自杀研究史概要}
    \subsection{文献}
        \subsubsection{中国大陆的自杀研究}
            \begin{enumerate}
                \item 1949-1976:几乎没有
                \item 1977-1989:随着现象增加而间接发展
                \item 1990-1999:发展迅速。1997年,何兆雄《自杀病学》,翟书涛《危机干预和自杀预防》。
                \item 1999-今:1999年,谢丽华《中国农村妇女自杀报告》《农家女》,中国农村妇女自杀问题的严重性,费力鹏《自杀与中国的社会变迁》。2002年,回龙观医院“北京生命危机研究和干预中心”。
            \end{enumerate}
        \subsubsection{中国自杀问题}
            \begin{enumerate}
                \item 中国年轻妇女的自杀率为什么比男子高?
                \item 中国的主要自杀方式喝农药意味着什么?
                \item 中国农村的自杀率为什么比城市高?
                \item 中国的自杀与精神疾病之间是什么关系?
            \end{enumerate}
            其中对第四个问题,费力鹏等人算出中国自杀者最多63\%有精神疾病,少于西方的90\%以上,因此其关系
\end{document}