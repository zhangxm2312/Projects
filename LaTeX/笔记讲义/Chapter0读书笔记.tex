\documentclass[11pt]{article} % 字号可更改
% 用ctex显示中文并用fandol主题
\usepackage[fontset=fandol]{ctex}
\setmainfont{CMU Serif} % 能显示大量外文字体
\xeCJKsetup{CJKmath=true} % 数学模式中可以输入中文

% AMS全家桶,\DeclareMathOperator依赖之
\usepackage{amsmath,amssymb,amsthm,amsfonts,amscd}
\usepackage{pgfplots,tikz,tikz-cd} % 用来画交换图
\usepackage{bm,mathrsfs} % 粗体字母(含希腊字母)和\mathscr字体
\everymath{\displaystyle} % 全体公式为行间形式

% 纸张上下左右页边距
\usepackage[a4paper,left=1cm,right=1cm,top=1.5cm,bottom=1.5cm]{geometry}
% 生成书签和目录上的超链接
\usepackage[colorlinks=true,linkcolor=blue,filecolor=blue,urlcolor=blue,citecolor=cyan]{hyperref}
% 各种列表环境的行距
\usepackage{enumitem}
\setenumerate[1]{itemsep=0pt,partopsep=0pt,parsep=\parskip,topsep=0pt}
\setenumerate[2]{itemsep=0pt,partopsep=0pt,parsep=\parskip,topsep=0pt}
\setenumerate[3]{itemsep=0pt,partopsep=0pt,parsep=\parskip,topsep=0pt}
\setitemize[1]{itemsep=0pt,partopsep=0pt,parsep=\parskip,topsep=5pt}
\setdescription{itemsep=0pt,partopsep=0pt,parsep=\parskip,topsep=5pt}
\setlength\belowdisplayskip{2pt}
\setlength\abovedisplayskip{2pt}

% 左右配对符号
\newcommand{\br}[1]{\!\left(#1\right)} % 括号
\newcommand{\cbr}[1]{\left\{#1\right\}} % 大括号
\newcommand{\abr}[1]{\left<#1\right>} % 尖括号(内积)
\newcommand{\bbr}[1]{\left[#1\right]} % 中括号
\newcommand{\abbr}[1]{\left(#1\right]} % 左开右闭区间
\newcommand{\babr}[1]{\left[#1\right)} % 左闭右开区间
\newcommand{\abs}[1]{\left|#1\right|} % 绝对值
\newcommand{\norm}[1]{\left\|#1\right\|} % 范数
\newcommand{\floor}[1]{\left\lfloor#1\right\rfloor} % 下取整
\newcommand{\ceil}[1]{\left\lceil#1\right\rceil} % 上取整
% 常用数集简写
\newcommand{\R}{\mathbb{R}} % 实数域
\newcommand{\N}{\mathbb{N}} % 自然数集
\newcommand{\Z}{\mathbb{Z}} % 整数集
\newcommand{\C}{\mathbb{C}} % 复数域
\newcommand{\F}{\mathbb{F}} % 一般数域
\newcommand{\kfield}{\Bbbk} % 域
\newcommand{\K}{\mathbb{K}} % 域
\newcommand{\Q}{\mathbb{Q}} % 有理数域
\newcommand{\Pprime}{\mathbb{P}} % 全体素数,或概率
% 范畴记号
\newcommand{\Ccat}{\mathsf{C}}
\newcommand{\Grp}{\mathsf{Grp}} % 群范畴
\newcommand{\Ab}{\mathsf{Ab}} % 交换群范畴
\newcommand{\Ring}{\mathsf{Ring}} % (含幺)环范畴
\newcommand{\Set}{\mathsf{Set}} % 集合范畴
\newcommand{\Mod}{\mathsf{Mod}} % 模范畴
\newcommand{\Vect}{\mathsf{Vect}} % 向量空间范畴
\newcommand{\Alg}{\mathsf{Alg}} % 代数范畴
\newcommand{\Comm}{\mathsf{Comm}} % 交换
% 代数集合
\DeclareMathOperator{\Hom}{Hom} % 同态
\DeclareMathOperator{\End}{End} % 自同态
\DeclareMathOperator{\Iso}{Iso} % 同构
\DeclareMathOperator{\Aut}{Aut} % 自同构
\DeclareMathOperator{\Inn}{Inn} % 内自同构
% \DeclareMathOperator{\inv}{Inv}
\DeclareMathOperator{\GL}{GL} % 一般线性群
\DeclareMathOperator{\SL}{SL} % 特殊线性群
\DeclareMathOperator{\GF}{GF} % Galois域
% 正体符号
\renewcommand{\i}{\mathrm{i}} % 本产生无点i
\newcommand{\id}{\mathrm{id}} % 恒等映射
\newcommand{\e}{\mathrm{e}} % 自然常数e
\renewcommand{\d}{\mathrm{d}} % 微分符号,本产生重音符号
\newcommand{\D}{\partial} % 偏导符号
\newcommand{\diff}[2]{\frac{\d #1}{\d #2}}
\newcommand{\Diff}[2]{\frac{\D #1}{\D #2}}
% 运算符(分析)
\DeclareMathOperator{\Arg}{Arg} % 辐角
\DeclareMathOperator{\re}{Re} % 实部
\DeclareMathOperator{\im}{im} % 像,虚部
\DeclareMathOperator{\grad}{grad} % 梯度
\DeclareMathOperator{\lcm}{lcm} % 最小公倍数
\DeclareMathOperator{\sgn}{sgn} % 符号函数
\DeclareMathOperator{\conv}{conv} % 凸包
\DeclareMathOperator{\supp}{supp} % 支撑
\DeclareMathOperator{\Log}{Log} % 广义对数函数
\DeclareMathOperator{\card}{card} % 集合的势
\DeclareMathOperator{\Res}{Res} % 留数
% 运算符(代数,几何,数论)
\newcommand{\Span}{\mathrm{span}} % 张成空间
\DeclareMathOperator{\tr}{tr} % 迹
\DeclareMathOperator{\rank}{rank} % 秩
\DeclareMathOperator{\charfield}{char} % 域的特征
\DeclareMathOperator{\codim}{codim} % 余维度
\DeclareMathOperator{\coim}{coim} % 余维度
\DeclareMathOperator{\coker}{coker} % 余维度
\DeclareMathOperator{\Spec}{Spec} % 谱
\DeclareMathOperator{\diag}{diag} % 谱
\newcommand{\Obj}{\mathrm{Obj}} % 对象类
\newcommand{\Mor}{\mathrm{Mor}} % 态射类
\newcommand{\Cen}{C} % 群/环的中心 或记\mathrm{Cen}
\newcommand{\opcat}{^{\mathrm{op}}}
% 简写
\newcommand{\hyphen}{\textrm{-}}
\newcommand{\ds}{\displaystyle} % 行间公式形式
\newcommand{\ve}{\varepsilon} % 手写体ε
\newcommand{\rev}{^{-1}\!} % 逆
\newcommand{\T}{^{\mathsf{T}}} % 转置
\renewcommand{\H}{^{\mathsf{H}}} % 共轭转置
\newcommand{\adj}{^\lor} % 伴随
\newcommand{\dual}{^\vee} % 对偶
\DeclareMathOperator{\lhs}{LHS}
\DeclareMathOperator{\rhs}{RHS}
\newcommand{\hint}[1]{{\small (#1)}} % 提示
\newcommand{\why}{\textcolor{red}{(Why?)}}
\newcommand{\tbc}{\textcolor{red}{(To be continued...)}} % 未完待续

% 定理环境(随笔记形式更改)
\newtheorem{definition}{定义}
\newtheorem{remark}{注}
\newtheorem{example}{例}
\makeatletter
\@ifclassloaded{article}{
    \newtheorem{theorem}{定理}[section]
}{
    \newtheorem{theorem}{定理}[chapter]
}
\makeatother
\newtheorem{lemma}[theorem]{引理}
\newtheorem{proposition}[theorem]{命题}
\newtheorem{corollary}[theorem]{推论}
\newtheorem{property}[theorem]{性质}
\usepackage{wrapfig}
\title{Algebra: Chapter 0 读书笔记}
\author{章小明}
\date{更新日期: \today}

\begin{document}
\maketitle
\tableofcontents
% Hungerford 内容:
% II.2,3,7

\newpage
\section*{前言}
\addcontentsline{toc}{section}{前言}
本笔记以Paolo Aluffi的Algebra: Chapter 0为蓝本,并参考了一些其它教材,基于本人的手写自学笔记总结而成,并不蕴含书的全部内容,其中对我而言较为简明或显然的部分被略过.
% 未重构部分:0范畴论(未添加),1.1,1.2,1.3,1.4,1.5
% 范畴论内容暂废弃,需等到未来完成。

\setcounter{section}{-1}
\section{范畴论基础}
本章旨在记录学习后面的代数内容所必要的范畴论基础内容,并不包含过多的范畴论知识.

% 第一节省略了各种范畴的具体例子以及切片与余切片范畴.

\subsection{基础概念}
一个范畴$\Ccat$包含对象类$\Obj(\Ccat)$与态射类$\Mor(\Ccat)$.$\forall X,Y\in\Obj(\Ccat)$存在态射集合$\Hom_\Ccat(X,Y)$,其全体并即为$\Mor(\Ccat)$.态射满足:
\begin{enumerate}[label=(\arabic*)]
    \item $\Hom(X,X)=\End(X)$中均存在恒等元素$\id_X$.
    \item 态射间存在复合,即存在复合映射$\circ:\Hom(X,Y)\times\Hom(Y,Z)\to\Hom(X,Z), (f,g)\mapsto gf$.
    \item 复合运算满足结合律.
    \item 恒等元素(关于复合)为态射的左/右幺元.
\end{enumerate}

若态射$f$(关于复合)满足左消去律$f\alpha=f\alpha'\implies \alpha=\alpha'$,则称$f$为单态射.相应的右消去律则称为满态射.若$f\in\Hom(X,Y)$可逆(即存在$g\in\Hom(Y,X), fg=\id_Y, gf=\id_X$)则称$f$为同构.需要注意的是,在$\Set$中有``单+满=同构'',但这在一般的范畴中并不成立.

另外由定义可知,$\End(X)$为幺半群且$\Aut(X)$为群.称所有态射都是同构的范畴为群胚/广群.

范畴$\Ccat$有子范畴$\Ccat'$,若$\Obj(\Ccat')\subset\Obj(\Ccat)$且对于$X,Y\in\Obj(\Ccat')$都有$\Hom_{\Ccat'}(X,Y)\subset\Hom_{\Ccat}(X,Y)$.若后者为等号则为全子范畴.

最后我们给出反范畴$\Ccat\opcat$,其对象与$\Ccat$相同而$\Hom_{\Ccat\opcat}(X,Y)=\Hom_\Ccat(Y,X)$,且$f\circ\opcat g=gf$.

% \paragraph{范畴实例}
% \begin{enumerate}
%     \item 
% \end{enumerate}

\subsection{函子与自然变换}
函子$F:\Ccat_1\to\Ccat_2$表现为对象间的映射$F:\Obj(\Ccat_1)\to\Obj(\Ccat_2)$与态射间的映射,后者表现为总有映射$F:\Hom_{\Ccat_1}(X,Y)\to\Hom_{\Ccat_2}(FX,FY)$,其保持恒等$\id_X\mapsto \id_{FX}$且保持态射间的复合运算(即关于复合运算$F$成为同态).另外有其反函子$F\opcat:\Ccat_1\opcat\to\Ccat_2\opcat$,其保持对象间映射不变,而$F\opcat:\Hom_{\Ccat_1\opcat}(X,Y)\to\Hom_{\Ccat_2\opcat}(FX,FY), $.

\section{群论(第一部分)}
\subsection{$\Grp$范畴}
\paragraph{群的基本概念}
让我们先从一句抽象废话来描述群的定义:
\begin{itemize}
    \item 仅含一个对象的群胚(groupoid)(其所有态射连带态射复合)构成群.更进一步的,$\Aut_{\Ccat}(X)$是群.
\end{itemize}
而群之间的同态等价于$\Grp$上的态射,即使
\begin{tikzcd}
    G\times G \arrow[d, "m_G"] \arrow[r, "\varphi\times\varphi"] & H\times H \arrow[d, "m_H"] \\
    G \arrow[r, "\varphi"]                                       & H                         
\end{tikzcd}
交换的$\varphi:G\to H$,这样的$\varphi$将群上的二元运算$m_G$平移到另一个群上的$m_H$,是很自然的性质.而这样的交换图给定了$\Grp$中的态射,$\Grp$自然成为范畴.

\paragraph{积与余积}
$\Grp$中的积(切片范畴$\Grp_{G,H}$的终对象)和余积(余切片范畴$\Grp^{G,H}$的始对象)成为群之间的直积$G\times H$和自由积$G*H$.对于前者我们可以直接构造分量(componentwise)积运算$(g,h)(g',h')=(gg',hh')$的形式,后者在自由群中给出构造.

而$\Ab$中的积和余积是等价的,即成为群的直和$G\oplus H$.等价性的直接原因是对于任意$\varphi:G\times H\to A$,余积定义要求$\varphi(g,h)=\varphi_G(g)\varphi_H(h)$,它成为同态需要交换.但我暂且不知道更深刻的内涵.

最后,$\Hom_{\Ab}(G,H)$构成交换群(态射的加法逐点定义),此处交换性来源于$G$的交换性,但构成群的良定性源于$H$的交换性,因此在$H$是交换群时,$\Hom_{\Grp}(G,H)$和$\Hom_{\Set}(A,H)=H^A$都构成群.

\paragraph{范畴中的群对象}
范畴$\Ccat$若具有有限积和终对象$1$,则其中群对象指$\Ccat$中对象$G$连带态射(二元运算)$m:G\times G\to G$, (幺元)$e:1\to G$和(取逆)$\iota:G\to G$,其满足结合律,(双边)幺和(双边)逆:

\begin{tikzcd}
    (G\times G)\times G \arrow[d, "\cong"] \arrow[r, "m\times \id"] & G\times G \arrow[d, "m"] & 1\times G \arrow[rd, "\cong"'] \arrow[r, "e\times\id"] & G\times G \arrow[d, "m"] & G\times 1 \arrow[ld, "\cong"] \arrow[l, "\id\times e"'] & G \arrow[r, "\id\times\id"] \arrow[d] & G\times G \arrow[r, "\id\times\iota", shift left] \arrow[r, "\iota\times\id"', shift right] & G\times G \\
    G\times(G\times G) \arrow[r, "\id\times m"] & G & & G & & 1 \arrow[rr, "e"] & & G \arrow[u, "m"]
\end{tikzcd}

从而可以看出,群是$\Set$中的群对象.

\subsection{自由群}
对于某集合$A$,考虑由集合函数$j:A\to G$作为对象的范畴$\mathcal{F}^A$,其中$G$为任意群,态射由其自然诱导,即$j_1\to j_2$为满足$j_2=j_1\circ\varphi$的群同态$\varphi:G_1\to G_2$.我们定义集合$A$生成的自由群$F(A)$为$\mathcal{F}^A$的始对象,即
\begin{tikzcd}
    F(A) \arrow[r, "\exists!\varphi"]         & G \\
    A \arrow[ru, "\forall f"'] \arrow[u, "j"] &  
\end{tikzcd}
将定义中的$G$改为交换群,所得到的群即自由交换群$F^{ab}(A)$.

从范畴论的观点来看,自由群的泛性质使其构造仅差一个同构,而$A\mapsto F(A), \Set\to\Grp$给定了一个自由函子,其是遗忘函子$\mathrm{For}:\Grp\to\Set$的左伴随.\footnote{伴随我还没学过,待补充.}从这一观点(或直接从泛性质和自由群的构造)出发,我们可以得到结果
$$F(A\sqcup B)=F(A)*F(B), F^{ab}(A\sqcup B)=F^{ab}(A)\oplus F^{ab}(B).$$

为了具体刻画$F(A)$,我们给出其构造:用$A$与$A^{-1}$($A$的复制)的无交并构造有限序列(被称为词word),其全集为$W(A)$,再用化简/消去的函数$R$将其相邻的互逆元素消去得到化简词,$F(A)=R(W(A))$.以词的连接并化简作为$F(A)$上的运算,因此可构造出群$F(A)$的具体形式,而此时$j:A\to F(A), a\mapsto (a)$成为$\mathcal{F}^A$的始对象.

容易看出$F(\cbr{*})=F^{ab}(\cbr{*})=\Z$,因此对于$[n]=\cbr{1,2,\cdots,n}, F^{ab}([n])=\Z^{\oplus n}:=\Z\oplus\cdots\oplus\Z=\Z^n$.而对于任意集合$A$和交换群$H$,我们定义$H^A$的子群$H^{\oplus A}:=\cbr{\alpha:A\to H\big| \text{仅有限个}\alpha(a)\neq e_H}$.事实上我们有$F^{ab}(A)\cong\Z^{\oplus A}$,这是因为前者的每个词仅含有限个元素,而后者也仅有限个元素/分量非零.可以考虑$j:A\to \Z^{\oplus A}, a\mapsto \chi_a$,其中$\chi_a\in \Z^{\oplus A}: x\mapsto [x=a]$是示性函数,因此$\sum_{a\in A}m_a j(a)\mapsto \prod_{a\in A}a^{m_a}$成为上述同构.

\subsection{子群与商群} % 需要重写
\paragraph{子群与单同态}
我们给出子群一个比较新奇的定义:群$G$的子集$H$的嵌入映射$i_H:H\to G$是群同态.这与其一般的定义或$ab^{-1}\in H$等价.子群的任意交,直和,同态像与原像都是子群.

在范畴论视角下,群同态$\varphi:G\to G'$的核具有某种泛性质:考虑$\Grp$的子范畴$\Ccat_\varphi$,其对象为满足$\varphi\circ\alpha=0$的群同态$\alpha:K\to G$,则嵌入$i:\ker\varphi\to G$是其终对象,即
\begin{tikzcd}[sep=large]
    \ker \varphi \arrow[d, "i"', hook'] & K \arrow[ld, "\alpha"'] \arrow[d, "\varphi\circ\alpha=0"] \arrow[l, "\exists!\bar{\alpha}"'] \\
    G \arrow[r, "\varphi"] & G'
\end{tikzcd}
这其实是``使同态像平凡的子群中$\ker \varphi$是最大的''的抽象废话式描述.

另外,群的单同态$\iff$核平凡$\iff$同态作为集合函数是单射.但$\Grp$与$\Set$中的单同态(前者是群单同态,后者是单射)不同,尽管存在左逆蕴含单同态,但$\Set$中反之亦然,而$\Grp$中单同态不一定存在左逆.\footnote{有反例$\Z/3\to S_3, k\mapsto (123)^k$.}

\paragraph{子集生成的子群}
考虑群$G$中的$A\subset G$,由泛性质可取出唯一的$\varphi_A:F(A)\to G$,我们可以定义$A$生成的子群$\abr{A}:=\im \varphi_A<G$.$G$交换时取$F^{ab}(A)$.\footnote{这其中有什么差异呢?我想交换应当成为所有普通的群的特例.但先前自由交换群是较为平凡的情形,而与一般的自由群有一些区别.我想这里可能会有一些问题.}该定义与其它定义等价,如$\abr{A}=\bigcap_{A\subset H<G}H$或$\abr{A}=\cbr{\prod_{a'\in A'}a'\bigg| A'\subset A}$.若$A$有限,则称$\abr{A}$为有限生成群,而由定义,这等价于存在满同态$F([n])\to G$.这一结论也可迁移至交换群上.

\paragraph{商群} % 正规子群呢?正规子群与子群的积为子群,而正规子群之间的积为正规子群
对于群$G$商去其上等价关系$\sim$所得到的结构,我们有结论
\begin{itemize}
    \item $[a]_{\sim}[b]_{\sim}=[ab]_{\sim}$给定了$G/\sim$上的群结构,等价于$a\sim a'\implies ag\sim a'g, ga\sim ga'$.而此时在满足$a\sim a'\implies\varphi(a)=\varphi(a')$的同态$\varphi:G\to G'$构成的范畴中,典范投影$\pi:G\to G/\sim$成为其始对象.
    \item 若$H\lhd G$且$H<\ker\varphi$,则(由上泛性质)有
    \begin{tikzcd}
        G/H \arrow[r, "\exists!\tilde{\varphi}"]                      & G' \\
        G \arrow[u, "\pi_H", two heads] \arrow[ru, "\forall\varphi"'] &   
    \end{tikzcd}
\end{itemize}
实际上商去(同态给定的)等价关系与商去(正规)子群并没有本质区别,只有记号差异$H=[e_G]_\sim$,尽管同态给定的等价关系内蕴的要求了$[e_G]_{\sim}<\ker\varphi$.而$G/H$的泛性质表明,它是商去后最普适的,最小的.另外,去掉$H<\ker\varphi$则会导致等价关系不成立,最小的商群成为$G/\ker\varphi$.

这一结果也表明,对于每个同态都能取其核为正规子群,而对于每个正规子群都存在以其为核的同态,这表明核与正规子群之间有一定的等价联系.

\paragraph{特征子群}
我们可以推广正规子群这一概念:考虑群$G$的子群$H$,若$\forall\varphi\in \Inn(G)$(或$\Aut(G), \End(G)$),$\varphi(H)<H$,则称$H$是$G$的正规子群(或特征子群,全特征子群).全特征$\implies$特征$\implies$正规.由上显然正规子群不具有传递性,但特征与全特征子群具有传递性.此外,正规子群的特征子群仍为正规子群,即$K<H\lhd G, K$是$H$的特征子群,则$K\lhd G$.\footnote{$\varphi_g:a\mapsto gag\rev, \varphi_g|_H\in \Inn(H)\subset \Aut(H), \varphi_g|_H(K)=gKg\rev\subset K$,反向同理,故$gKg\rev=K$.}而$H\lhd G$且$G$有限,$\abs{H}$与$\abs{G/H}$互素,则$H$是$G$的特征子群.\footnote{考虑$\pi_H:G\to G/H, \forall \varphi\in \Aut(G), \abs{\pi_H\circ \varphi(H)}$整除$\abs{\varphi(H)}=\abs{H}$以及$\abs{\im\varphi}=\abs{G/H}$,故$\pi_H\circ\varphi(H)=H,\varphi(H)\subset H$.}

\paragraph{导群}
群$G$的导群是以其全体交换子$[g,h]=ghg\rev h\rev$生成的子群,记为$G',[G,G]$或$G_{\text{der}}$.导群保序,即$H<G$则$H'<G'$.对于群同态$\varphi:G_1\to G_2$,显然有$\varphi([g,h])=[\varphi(g),\varphi(h)]$且$\varphi(G'_1)<G'_2$.由此可见$G'$是特征子群,从而$G'\lhd G$.若令$\varphi=\pi_{G'}$从而可见$G/G'$交换,称之为$G$的Abel化.对于$N\lhd G$,$G/N$交换等价于$G'<N$.

考虑群$G$到任意交换群$A$的群同态$\alpha:G\to A$,其构成范畴并自然诱导态射.由$\alpha(G')<A'=\cbr{e}$知$G'<\ker\alpha$,从而由商群的泛性质可见$\pi:G\to G/G'$为该范畴的始对象.

在自由群上我们有如下结果:$F(A)/F(A)'\cong F^{ab}(A)$.\footnote{考虑任意交换群$G$及集合函数$j:A\to G$,由$F(A)$的泛性质可得唯一群同态$\varphi:F(A)\to G, a\mapsto j(a)$,而核即$F(A)$中次数和为零的词,显然$F(A)'<\ker\varphi$,故由商群泛性质知存在唯一同态$F(A)/F(A)'\to G$,故可构造$j:A\to F(A)/F(A)', a\mapsto aF(A)'$为始对象.}\footnote{这表明包含函子$\Ab\to\Grp$的左伴随是Abel化函子$G\mapsto G/G'$.此处需要更多说明.}

\subsection{典范分解}
考虑在$\Grp$中的典范分解\footnote{典范分解的更多内容应在补充范畴论内容后继续补充.参考\href{https://chaoli.club/index.php/10345}{什么样的范畴具有典范分解?}}
\begin{tikzcd}[sep=large]
    G \arrow[r, "\varphi"] \arrow[d, "\pi", two heads] & G'\\
    G/\ker\varphi \arrow[r, "\tilde{\varphi}"] & \im\varphi \arrow[u, "i", hook]
\end{tikzcd}
其中$\tilde{\varphi}$是商群关于$\varphi$诱导的映射,可以证明其为同构.此即本节的出发点.该同构也被称为群同构第一定理.进一步我们有如下推论:

\begin{itemize}
    \item 若$H_1\lhd G_1,H_2\lhd G_2$则$\frac{G_1\times G_2}{H_1\times H_2}\cong \frac{G_1}{H_1}\times \frac{G_2}{H_2}$.\footnote{考虑$\pi=\pi_1\times \pi_2$即可.}特例是$(G_1\times G_2)/G_1\cong G_2$.
    \item 对于满同态$\varphi:G\to H$,有保序双射$f:\cbr{K<G|\ker\varphi<K}\to \cbr{J|J<H}, K\mapsto \varphi(K)$,在此对应下正规子群映成正规子群.\footnote{容易验证$K\mapsto \varphi\rev(K)$是$f$的逆,保序显然.此处应当注意到$\ker\varphi<K\iff \varphi\rev(\varphi(K))=K$.最后$K\lhd G\implies \varphi(K)<H$且$J\lhd H\implies \varphi\rev(J)\lhd G$.}若令$\varphi=\pi_N:G\to G/N, N\lhd G$,则可见$G/N$的子群均为商群形式,且$K\lhd G\iff K/N\lhd G/N$.
    \item (群同构第三定理)若$H<N<G$且$H\lhd G$,则$N\lhd G\iff \frac{N}{H}\lhd\frac{G}{H}$,且此时有$\frac{G/H}{N/H}\cong\frac{G}{N}$.\footnote{$\implies$由商群泛性质考虑$G/H\to G/N$,$\impliedby$与同构考虑$\pi_{N/H}\circ\pi_H:G\to G/H\to\frac{G/H}{N/H}$即可.}
    \item (群同构第二定理)若$K<G$且$H\lhd G$,则$H\lhd HK<G,H\cap K<K$且$\frac{HK}{H}\cong\frac{K}{H\cap K}$.\footnote{包含关系由考虑$\pi_H\rev$内的包含可得,再取$\varphi:K\to HK/H, k\mapsto Hk$可证.}
\end{itemize}
%Lagrange定理;H和K为正规子群则HK仍为正规子群

\paragraph{自由群的关系}
群$G$的表示是指同构$G\cong F(A)/R$或$\rho:F(A)\twoheadrightarrow G, \ker\rho=R$是(词的集合)``关系''$\mathcal{R}\subset F(A)$在$F(A)$中生成的正规子群.记$G=(A|\mathcal{R})$.我们有$F(A)=(A|\varnothing), D_{2n}=(x,y|x^2,y^2,(xy)^n)$.最后,由``自由是遗忘的左伴随''同样可知$(A|\mathcal{R})*(A'|\mathcal{R}')=(A\sqcup A'|\mathcal{R}\sqcup\mathcal{R}')$.

\begin{wrapfigure}[7]{r}{0.25\textwidth}
\begin{tikzcd}[sep=large]
    \ker\varphi \arrow[d, "i"', hook] & H \arrow[l, "\exists!\bar{\alpha}"'] \arrow[d, "\varphi\circ\alpha=0"] \arrow[ld, "\alpha"'] \\
    G \arrow[r, "\varphi"] \arrow[d, "\beta\circ\varphi=0"'] & G' \arrow[ld, "\beta"'] \arrow[d, "\pi"', two heads] \\
    K & \coker\varphi \arrow[l, "\exists!\bar{\beta}"']
\end{tikzcd}
\end{wrapfigure}

\paragraph{满同态与$\coker$}
在考虑子群与单同态时我们注意到嵌入$i:\ker\varphi\to G$是$\Grp$中范畴$\Ccat_\varphi$的终对象.我们考虑其对偶形式$\Ccat^\varphi$:$\Obj(\Ccat^\varphi)=\cbr{\alpha:G'\to L|\alpha\circ\varphi=0}$,其始对象即为$\pi:G'\twoheadrightarrow \coker\varphi$.综合两者我们有交换图:

若在$\Ab$中由商群泛性质直接可得$G'/\im\varphi\cong\coker\varphi$,而在$\Grp$中$\im\varphi$不一定正规于$G'$,因此由泛性质仅可得到$\coker\varphi\cong G'/N, N$是$\im\varphi$在$G'$中生成的正规子群,即$G'$中包含$\im\varphi$最小的正规子群.

最后,$\varphi$是满射同态$\iff \varphi$是满态射$\implies \coker\varphi$平凡.最后一个箭头的反向需在$\Ab$中成立.\footnote{反例是嵌入$H=\cbr{(1),(12)}\hookrightarrow S_3$的$\coker$平凡但非满射.}

\subsection{群的作用}
\paragraph{基本概念与$G\hyphen\Set$范畴}
群$G$在范畴$\Ccat$的对象$A$上的作用即态射$\sigma:G\to \Aut_\Ccat(A)$.\footnote{此处以及下面的$G$-$\Set$应当有函子背景,未来需补全.}称作用是忠实的,若$\sigma$是单射;称作用是自由的,若$\ker\sigma$平凡.当$\Ccat=\Set$时我们有记号$\rho:G\times A\to A, (g,a)\mapsto \sigma(g)(a)$,记作$ga$.我们有Cayley定理:所有群$G$总能忠实地作用于某集合$A$上,换言之,$G$均同构于某对称群$S_A$的子群.\footnote{其在群作用视角下显然,因为$G$到自身的左乘作用$G\to \Aut_{\Set}(G)=S_G, a\mapsto \lambda_a$显然是忠实的.}记轨道$O_G(a)=\cbr{ga|g\in G}$,稳定子$G_a=\mathrm{Stab}_G(a)=\cbr{g\in G|ga=a}$.称作用$\rho$可迁,若$O_G(a)=A$,即总能将一个元素迁移为任意另一个.显然$G_{ga}=gG_a g\rev$,且作用的核$\ker\sigma=\bigcap_{a\in A}G_a$.

群$G$在集合上的作用构成范畴$G$-$\Set$,其对象是$(\rho,A)$,态射为函数$\varphi:A\to A'$,其满足
\begin{tikzcd}
    G\times A \arrow[d, "\rho"] \arrow[r, "\id_{G}\times\varphi"] & G\times A' \arrow[d, "\rho'"] \\
    A \arrow[r, "\varphi"] & A'
\end{tikzcd}
称$G$-$\Set$中态射为($G$-)等价的,即$g\varphi(a)=\varphi(ga)$,而其中同构即等价双射.我们有如下结果:$G$在$A$上的可迁左作用在$G$-$\Set$中同构于$G$在$G/_L G_a(\forall a\in A)$上的左乘作用.\footnote{即考虑函数$\varphi:G/H\to A, gH\mapsto ga$,易证其良定等价双射.}换言之在一般情形下,在每个轨道$O_G(a)$上都有$O_G(a)\simeq G/_L G_a$.因此在$G$和轨道$O(a)$有限时,我们有$|G|=|O(a)|\cdot |G_a|$,此即轨道-稳定子定理.另外若将$G$视为左乘作用下的$G$-集合,则显然有$\Aut_{G\hyphen\Set}(G)\cong G$.

\paragraph{共轭作用}
考虑群$G$作用在有限集$S$上,定义作用的不动点集$Z=\cbr{a\in S | \forall g\in G, ga=a}, a\in Z\iff G_a=G\iff O_a=\cbr{a}$.我们有$\abs{S}=\abs{Z}+\sum_{a\in A}[G:G_a]$,其中$A\subset S$是$S$的轨道等价类中非平凡轨道的代表元,即$a\in A, \abs{O_a}>1$.这是通过计数得到的:前项为平凡轨道的数量,后项为所有非平凡轨道中轨道元素数量之和.

考虑群$G$在自身上的共轭作用,此时的作用不动点集即中心$\Cen(G)$,而稳定子即中心化子$\Cen_G(a)$,轨道即共轭类$[a]=\cbr{gag\rev | g\in G}$.显然$\Cen(G)=\bigcap_{a\in G}\Cen_G(a), a\in\Cen(G)\iff \Cen_G(a)=G\iff [a]=\cbr{a}$.实际上有$G/\Cen(G)\cong\Inn(G)$,且其循环时$G$交换.\footnote{前者考虑$g\mapsto (\varphi_g: a\mapsto gag\rev)$,后者考虑$\varphi_g=\varphi_a^n\implies gag\rev=a, a\in\Cen(G)$,故内自同构均平凡.}对共轭作用考虑上述公式可得类数公式:$\abs{G}=\abs{\Cen(G)}+\sum_{a\in A}[G:\Cen_G(a)]$,$A$定义同上.另外注意到$H\lhd G$有$a\in H$则$[a]\subset H$,故$H=\bigcup_{a\in H}[a]$(实则共轭类的不交并).

再考虑群$G$在其幂集上的共轭作用,即$\rho(g,A)=gAg\rev=\cbr{gag\rev | a\in A}$,应当注意到$A\to gAg\rev, a\mapsto gag\rev$是双射.此作用的稳定化子即正规化子$N_G(A)=\cbr{g\in G|gAg\rev=A}$,而$\Cen_G(A)\subset N_G(A)$的每个点在作用下不变.对于子群$H<G$有$H\lhd N_G(H)<G$,且$N_G(H)$是$G$中最大的使$H$正规于之的子群,即$H\lhd K$则$K<N_G(H)$,显然$H\lhd G\iff N_G(H)=G$.且由轨道-稳定子定理,$H$的共轭子群数量为$[G:N_G(H)]$.

事实上$\Cen_G(A)\lhd N_G(A)$,且$N_G(A)$在$A$上有共轭作用\footnote{即群同态$N_G(A)\to S_A, n\mapsto (a\mapsto nan\rev)$},作用核为$\Cen_G(A)$,故$N_G(A)/\Cen_G(A)$同构于$S_A$的某子群.对于$H<G$,该作用成为同态$N_G(H)\to \Aut_{\Grp}(H)$,即$N_G(H)/\Cen_G(H)$同构于$\Aut(H)$的某子群,此即N/C定理.其在$H=G$时即为$G/\Cen(G)\cong \Inn(G)$,而对于$N<N_G(H)$在$H$上的共轭作用,作用核为$N\cap C_G(H)$.

\paragraph{$p$-群}
$p$-群即阶为素数$p$幂的群.由类数公式可知,其在有限集$S$上作用时有$\abs{Z}\equiv\abs{S}\bmod p$,故对于$G$在自身上的共轭作用有$\abs{G}\equiv \abs{\Cen(G)}\bmod p$,从而由$\abs{\Cen(G)}\geq 1$可知,非平凡$p$-群必有非平凡中心.若有限群$G$有$p$-子群$H$,则考虑$H$在$G/_LH$上的左乘作用,可见$Z=N_G(H)/H$,即$[N_G(H):H]\equiv [G:H]\bmod p$.

$p^2$阶群总交换,而$p^n$阶$p$-群对$m\in [n]$均有$p^m$阶正规子群.\footnote{显然$\Cen(G)$的阶为$p^s, s\leq n$,故其$p^k$阶子群($0\leq k\leq s$)均正规.而考虑$p^m$阶正规子群$H$,$G/H$仍为$p$-群,故其有非平凡中心.取其中$p$阶子群$N$,可见$\pi_H^{-1}(N)$是$p^{m+1}$阶正规子群,归纳可证.}另外,对于有限群$G$及$\abs{G}$的最小素因子$p$,指数$p$的子群均正规,而正规$p$阶子群均含于$\Cen(G)$.\footnote{(1)即$[G:H]=p$,$G$在$G/_L H$上的左乘作用即$\sigma:G\to S_p$,而$\ker\sigma\subset H, \abs{G/\ker\sigma}=[G:H][H:\ker\sigma]$,又有$\abs{G/\ker\sigma}|\gcd(\abs{G},\abs{S_p})=p$,故$[H:\ker\sigma]=1, H=\ker\sigma\lhd G$.(2)即有共轭作用$\sigma:G\to\Aut_{\Grp}(H)\cong \Z/(p-1)$,其中$\ker\sigma=C_G(H)$,故$\abs{G/\ker\sigma}|\gcd(\abs{G},p-1)=1$,即$G=C_G(H), H<C(G)$.}

\subsection{Sylow定理}
\paragraph{Cauchy定理}
对于有限群$G$及$\abs{G}$的素因子$p$,$G$中总有$p$阶元.

\begin{proof}[证明(James McKay)]
    考虑$S=\cbr{(a_1,\cdots,a_p)|a_i\in G, a_1\cdots a_p=e}$,由$a_p$由前元素唯一决定,故$\abs{S}=\abs{G}^{p-1}\equiv 0\bmod p$.令$\Z/p$循环作用于$S$上,即$m(a_1,\cdots,a_p)=(a_{m+1},\cdots,a_p,a_1,\cdots,a_m)\in S$(容易验证$ab=e$则$ba=e$),显然该作用的稳定点$Z=\cbr{(a,\cdots,a)|a\in G}$,且有$\abs{Z}\equiv\abs{S}\equiv 0\bmod p$,故有$a^p=e, a\neq e$.
\end{proof}

故由此有推论:$G$中$p$阶子群的数量$N\equiv 1\bmod p$.\footnote{通过计数有$\abs{Z}=1+(p-1)N$,模$p$即得.}此处需要插入看似无关的引理:若$G$中仅有一个子群$H$同构于某群$K$,则$H\lhd G$.(共轭不变)

\paragraph{Sylow第一定理}
$G$为有限群,则对$\abs{G}$的任意素因子$p$, $G$总含Sylow $p$-子群.$G$中的Sylow $p$-子群$P$即$P<G$为$p$-群且$p$与$[G:P]$互素,即$G$中的极大$p$-子群.换言之,$\abs{G}=p^rm, \abs{P}=p^r, \gcd(p,m)=1$.

定理的等价(由$p$-群性质)描述为:$G$为$p^nm$阶群($p$为素数且与$m$互素),则对$k\in [n]$总有$p^k$阶子群,且该子群是某个$p^{k+1}$阶子群的正规子群.

\begin{proof}
    首先由Cauchy定理,$G$中总含$p$阶子群.下对$k$归纳证明:若$H$是$G$的$p^k$阶子群($k<n$),则$0\equiv [G:H]\equiv [N_G(H):H]\bmod p$,后项非0故$N_G(H)\neq H$,且$N_G(H)/H$也含$p$阶子群,记为$H_1/H$,其中$H<H_1<N_G(H)$,因此$H\lhd H_1, \abs{H_1}=\abs{H}\abs{H_1/H}=p^{k+1}$.
\end{proof}

事实上对于有限交换群有更强的结论:若$G$是有限交换群且$d|\abs{G}$则$G$有$d$阶子群.\footnote{任取元素生成循环子群再不断商去可得任意素因子,再如上考虑商群子群,可得任意因子.关键在于交换群的子群均正规.}

\paragraph{Sylow第二定理}
$P$是有限群$G$中的Sylow $p$-子群,$H<G$是$p$-子群,则$H$在$P$的某个共轭中,即$H<gPg\rev$.特别的,$G$的Sylow $p$-子群间互相共轭.
\begin{proof}
    令$H$左乘作用于$G/_LP$上,作用不动点集的势$\abs{Z}\equiv [G:P]\not\equiv 0\bmod p$,故有作用不动点$aP, a\rev ha\in P(\forall h\in H), H<aPa\rev$.
\end{proof}

事实上有限群$G$的全体Sylow $p$-子群之交$N=\bigcap_{g\in G}gPg\rev$是$G$中的极大正规$p$-子群,即任意$G$的正规$p$-子群均含于$N$中.换言之,在$G$的$\abs{G}/p^\alpha$阶同态像中$G/N$是终对象.

\paragraph{Sylow第三定理}
$G$为有限群,$\abs{G}=p^nm$,其中$p$为素数且与$m$互素,则$G$中的Sylow $p$-子群数量$N_p|m$且$N_p\equiv 1\bmod p$.
\begin{proof}
    对任意Sylow $p$-子群$P$有$N_p=[G:N_G(P)]$,且$m=[G:P]=N_p[N_G(P):P]$,而$m=[G:P]\equiv [N_G(P):P]\bmod p$,因此$mN_p\equiv m\bmod p$,而$m$与$p$互素,故$N_p\equiv 1\bmod p$.
\end{proof}

\paragraph{Sylow定理的应用}
\begin{enumerate}
    \item 单群即正规子群平凡(仅有单位元或本身)的群.我们有如下命题:
    \begin{itemize}
        \item $mp^r$阶群($1<m<p, p$为素数)不为单群\footnote{由Sylow第三定理,$1+kp=N_p|m<p$,因此$k=0, N_p=1$,故Sylow $p$-子群唯一,故其正规,从而非单.}.若$m$的模$p$余1因子仅有1,则该命题同样成立.
        \item 非平凡交换单群有且仅有$\Z/p, p$为素数.
        \item $G$为非平凡有限单群$\iff$其同态像平凡或同构于本身\footnote{由定义,同态核仅有两种选择,因此同态像也即如此.},从而非平凡群同态$\varphi:G\to G'$总为单射.
        \item 无平方因子群(即素数平方不能整除阶)均不是单群.\footnote{参考\href{https://math.stackexchange.com/a/242811}{Given 3 distinct primes $p,q,r$, then $|G|=pqr\implies G$ is not simple}及\href{https://math.stackexchange.com/questions/2842126}{Burnside's transfer theorem in group theory}.}
        \item $G$是有限单群且$H$是其指数为$N$的真子群,则$\abs{G}|N!$.\footnote{令$G$左乘作用于$G/_L H\cong S_N$上,由$G$单知作用核平凡,故$\abs{G}|N!$.}特别的,令$P$是其Sylow $p$-子群,$H=N_G(P)$,则$\abs{G}|N_p!$.
        \item 延申阅读:\href{https://zhuanlan.zhihu.com/p/25301636}{Cole与有限单群(I)}
    \end{itemize}
    \item 对于$pq$阶群$G$,其中$p,q$均为素数且$p<q$, 若$q\not\equiv 1\bmod p$,则$G$为循环群.\footnote{由Sylow第三定理及条件有$N_p=1$,即$p$阶子群$H$唯一(正规).考虑$G$在$H$上的共轭作用$\gamma:G\to \Aut(H)\cong\Z/(p-1)$, $\abs{\gamma(G)}|\gcd(pq,p-1)=1$,故$\gamma$平凡,$H<\Cen(G)$,考虑$G/\Cen(G)$的阶从而循环,即$G$交换.取其中$p,q$阶元,乘积必为$pq$阶元,从而循环.}反之,$G$不交换则有$N_p=q\equiv 1\bmod p$
    \item 若$p$为奇素数,则$2p$阶非交换群$G\cong D_{2p}$.\footnote{显然$N_p=1$,令其为正规$p$阶子群$\abr{y}$,可见$G-\abr{y}$的元素均为2阶,任取之为$x$,令$xyx\rev=y^r$,而$(y^r)^r=xy^rx\rev=x^2yx^{-2}=y$,从而$y^{r^2-1}=e, p|(r^2-1)$即$p|(r-1)$或$p|(r+1)$,即$r=1$或$p-1$.前者则$xy=yx$可得$xy$阶$2p$,矛盾.故可得$x^2=y^p=xyxy=e$,此即$D_{2p}$.}
\end{enumerate}

\subsection{合成列与可解群}
\paragraph{合成列}
群$G$的次正规列(subnormal series)是指一列降序子群:
$$G=G_0\rhd G_1 \rhd G_2 \rhd \cdots$$
其中$G_i\neq G_{i+1}, G_i\rhd G_{i+1}$,称$G_i/G_{i+1}$为因子(群).若总有$G_i\lhd G$,则称之为正规列.\footnote{该定义在不同资料中略有不同.此处降序与升序没有本质区别,仅有下标差别, Chapter 0与Hungerford均使用降序,而互联网中多见升序.此外, Chapter 0中用正规列指Hungerford与该笔记中的次正规列,而没有提到本文的正规列概念.在Hungerford和维基百科中并没有要求$G_i\neq G_{i+1}$,但在使用该概念时常需令$G_{i+1}$严格嵌入$G_i$,故在此令两者不等没有本质区别.最后,在维基百科中要求子群链结束/开始于$\cbr{e}$,但同样没有本质问题.}列的长度即严格嵌入$G_i\hookleftarrow G_{i+1}$的数量,也即非平凡因子的数量.记群$G$的次正规列的最大长度(若有限)为$\ell(G)$,即$G_{\ell(G)}=\cbr{e}$. $\ell(G)=0$即$G$平凡,而$\ell(G)=1$即$G$为单群.

次正规列的一步细化(one-step refinement)是指比原列仅多一项的次正规列,次正规列的细化(refinement)即有限次一步细化所得次正规列,若细化比原列长则称为真(proper)细化\footnote{在此定义中所有细化均为真细化,因此我们将混用两概念.}.换言之,称某列是另一列的细化,若后者的项均出现在前者中.此外,次正规列之间等价是指次正规列的因子之间仅差一个置换相同(同构).

对于$G$的有限长次正规列
$$G=G_0\rhd G_1 \rhd G_2 \rhd \cdots \rhd G_n=\cbr{e}$$
若因子均为单群,则称之为合成列(composition series);若因子均交换(循环),则称之为可解列(solvable series)(循环列).其中,$G/N$为单群等价于$N$在$G$的所有真正规子群中极大,故称$N$为$G$的极大正规子群.显然有限群均有合成列,且次正规列长$\ell(G)$时为合成列.

\paragraph{Jordan-H\"older定理}
若群$G$有二合成列
$$G=G_0\rhd \underbrace{G_1\rhd\cdots\rhd G_n=\cbr{e}}_{G_\bullet},\qquad G=G'_0\rhd \underbrace{G'_1\rhd\cdots\rhd G'_m=\cbr{e}}_{G'_\bullet}$$
则$m=n$,且两者等价.换言之合成因子与合成列的选取无关,即$G_i/G_{i+1}$与$G'_i/G'_{i+1}$间仅差一个置换.
\begin{proof}
    我们对较短合成列的长度$n$归纳证明之.显然$n=0,1$时命题已成立.若命题在$<n$时成立,且$G_1=G'_1$,则两合成列可化为$G_\bullet$和$G'_\bullet$,其长度满足归纳假设,从而命题得证.故下设$G_1\neq G'_1$.
    
    易证$G_1\lneq G_1G'_1\lhd G$,而$G_1$是$G$的极大正规子群,故可知$G=G_1G'_1$.设$K=G_1\cap G'_1$,考虑其任意合成列(由下定理知存在)$K_\bullet$.由$G_1/K\cong G/G'_1, G'_1/K\cong G/G_1$知两者均为单群,故可得二合成列$G\rhd G_1(G'_1)\rhd K_\bullet$,其长度相同且因子间只差一个置换.
    
    最后,合成列$G_1\rhd K_\bullet$与$G_\bullet$中较短列的长度满足归纳假设,从而命题成立,故$K_\bullet$长$n-2$,从而$G'_1\rhd K_\bullet$与$G'_\bullet$也满足归纳假设,因此命题成立,综上得证.
\end{proof}
该定理表明合成列若存在则几乎相同且互相等价,从而可知$G$的合成列均长$\ell(G)$,但我们缺乏对存在性的证明.另外,上述证明中提到若$K$存在正合列,则会有$\ell(G)=n=2+(n-2)=\ell(G/K)+\ell(K)$.为证明其存在性并推广该结论,我们有如下定理

\paragraph{Schreier定理}% 可推广
% 我们首先引入Zassenhaus引理(或称蝴蝶引理):若$U,V$均为某群的子群,且有$u\lhd U, v\lhd V$,则有
% $$u(U\cap v)\lhd u(U\cap V),\quad (u\cap V)v\lhd (U\cap V)v,\quad \frac{u(U\cap V)}{u(U\cap v)}\cong \frac{(U\cap V)v}{(u\cap V)v}.$$
% \begin{proof}
%     $U\cap v=(U\cap V)\cap v\lhd U\cap V$,令$D=(u\cap V)(U\cap v)\lhd $
% \end{proof}

考虑群$G$及其正规子群$N$,$G$有合成列等价于$N$与$G/N$有合成列,此时$\ell(G)=\ell(N)+\ell(G/N)$,且$G$的合成因子包含$N$与$G/N$的合成因子.
\begin{proof}
    一方面,若$N$与$G/N$均有合成列
    $$N\rhd N_1 \rhd \cdots \rhd N_{\ell(N)}=\cbr{e},\qquad G/N\rhd G_1/N\rhd \cdots \rhd G_{\ell(G/N)}/N=\cbr{N}$$
    则每个$G_i/N$均可对应到$G$的子群$G_i$上,且同样有$G_i\rhd G_{i+1}, G_i/G_{i+1}\cong (G_i/N)/(G_{i+1}/N)$为单群,从而可以构造$G$的合成列
    $$G\rhd G_1\rhd \cdots \rhd G_{\ell(G/N)}=N\rhd N_1 \rhd \cdots \rhd N_{\ell(N)}=\cbr{e}$$
    并且长度与因子的命题同样得证.

    另一方面,若$G$有合成列
    $$G_\bullet: G=G_0\rhd G_1\rhd\cdots\rhd G_n=\cbr{e}$$
    则可以构造关于$N$的子群链
    $$N\cap G_\bullet: N=N\cap G\rhd N\cap G_1\rhd\cdots\rhd N\cap \cbr{e}=\cbr{e}$$
    且$G_i\cap N\rhd G_{i+1}\cap N$.考虑$\pi:G_i\cap N\to G_i/G_{i+1}, a\mapsto aG_{i+1}, \ker\pi=G_{i+1}\cap N$,可以证明$\im\pi\lhd G_i/G_{i+1}$\footnote{即证$\forall g\in G_i\forall a\in G_i\cap N, gag\rev G_{i+1}\in \im\pi$,即$gag\rev\in G_i\cap N$.显然$gag\rev\in G_i$,而$N\lhd G$从而$gag\rev\in N$,故得证.},从而由$G_i/G_{i+1}$是单群知$\frac{G_i\cap N}{G_{i+1}\cap N}$平凡或同构于$G_i/G_{i+1}$.由此可见在删去子群链$N\cap G_\bullet$中的重复项后,其成为$N$的合成列.

    对于$G/N$的子群链可以类似构造:
    $$\frac{G_\bullet}{N}: \frac{G}{N}\rhd \frac{G_1N}{N}\rhd\cdots\rhd \frac{\cbr{e}N}{N}=\cbr{e_{G/N}}$$
    且$G_iN/N\rhd G_{i+1}N/N$.同样考虑$\pi:G_i\to (G_iN)/(G_{i+1}N), a\mapsto aG_{i+1}N$,易见其为满射,且$G_{i+1}<\ker\pi$,故可由商群的泛性质知有满射$\varphi:G_i/G_{i+1}\to (G_iN)/(G_{i+1}N), \pi=\varphi\circ\pi_{G_{i+1}}$.而$G_i/G_{i+1}$为单群,故$\frac{G_iN/N}{G_{i+1}N/N}\cong \frac{G_iN}{G_{i+1}N}$平凡或同构于之.同上可见也有$G/N$的合成列.最后由$N$与$G/N$的合成列同第一部分可证剩下命题.
\end{proof}

\paragraph{导出列与可解群}
回忆导群相关知识,群$G$的导出列(derived series)即子群链$G\rhd G'\rhd G''\rhd \cdots$.\footnote{对于一般资料而言,次正规列不要求相邻项不等,故导出列是次正规列.}若$G$交换,则导出列仅为$G\rhd G'=\cbr{e}$;若$G$为非交换单群,则$G=G'=G''=\cdots$.可解群(solvable group)即导出列终止于$\cbr{e}$的群,其等价于具有可解列.\footnote{可解群的导出列显然邻项不等,故即可解列;若群有可解列$G\rhd G_1\rhd \cdots \rhd G_n=\cbr{e}$,则$G/G_1$循环可得$G_1>G'$,同理$G_2>G'_1>G''$,故$G_i>G^{(i)}, \cbr{e}>G^{(n)}$,从而有可解列.}

对于有限群$G$,其可解$\iff$合成因子均循环$\iff$有循环列$\iff$有可解列.\footnote{已知$(1)\iff (4)$,显然$(2)\implies (3)\implies (4)$,仅证$(4)\implies (2)$:将可解列细化为合成列,注意到交换单群仅有素数阶循环群.}由此知所有$p$-群均可解.结合Schreier定理知,对于$N\lhd G$, 有限群$G$可解$\iff N$与$G/N$可解.此外,可解群的子群也可解.

\paragraph{幂零群}
% Exer III.3.10,11 + Hungerford 相应内容

\newpage
\section{环与模}
\subsection{$\Ring$范畴}
\paragraph{环的基本概念}
环$(R,+,\cdot)$是交换群$(R,+,0)$与幺半群$(R,\cdot,1)$\footnote{大部分书都以后者为半群作环的定义,但由于关于环的大部分讨论都在含幺环上进行讨论,因此本书中环的定义使乘法半群含幺.另外,不一定含幺的环构成的范畴有时记作$\mathsf{Rng}$.}关于分配律构成的代数结构.

整环是指无零因子\footnote{需要注意的是很多书以零因子为非零元素,但也有很多书认为0也是零因子.但零因子=左零因子$\cup$右零因子.}交换(含幺)环,除环是指非零元均可逆的环,即乘法群$R*=R-\cbr{0}$,域是交换除环.直观上可以认为域去除可逆性即为整环,而去除交换性即为除环.它们之间有一定关系:
\begin{tikzcd}
    整环 \arrow[r, "有限交换", Rightarrow] & 域 & 除环 \arrow[l, "有限或交换"', Rightarrow]
\end{tikzcd}.\footnote{整环有限交换时为域:任意元素非零因子则左乘作用为单射,而有限情况下容易看出其为满射,这等价于任意元素为正则元(即可逆).\\ 有限除环为域:即Wedderburn小定理,后面待证.}

关于零因子(非零元之积为0)和正则元(乘法可逆元)\footnote{$a$是左零因子$\iff \exists b\neq 0, ab=0$; $u$是左正则元$\iff \exists v, uv=1$.}我们有如下结论(类似对另一边也有):

\begin{tikzcd}
    a不是左零因子 \arrow[r, Leftrightarrow] & a的左乘作用是 R\to R的单射 & \\
    a是右正则元 \arrow[r, Leftrightarrow] \arrow[u, Rightarrow] & a的右乘作用是 R\to R 的满射 \arrow[r, Leftrightarrow] \arrow[u, Rightarrow] & R=Ra
\end{tikzcd}

环$R$的中心$\Cen(R)=\cbr{r\in R|\forall a\in R, ar=ra}$是$R$的交换子环,而$a$的中心化子$\Cen_R(a)=\cbr{r\in R|ar=ra}$是子环,且$\Cen(R)=\bigcap_{a\in R}\Cen_R(a)$.除环的中心化子也是除环,从而其中心是域.

\paragraph{幺半群环}
给定环$R$和幺半群$M$,幺半群环$R[M]$的元素为$r\cdot m(r\in R, m\in M)$的有限线性组合,其间加法与乘法类似多项式环定义.可见$R[x]=R[\N],R[x,x\rev]=R[\Z]$.

\paragraph{$\Ring$的泛对象}
$\Ring$以全体(含幺)环为对象,态射为保持加法交换群与乘法幺半群结构的映射/同态(需$1_R\mapsto 1_{R'}$).其中终对象为零环$\cbr{0}$,始对象为$\Z$(对每个环都有$n\mapsto n1_{R}$).

由于对任意环$R$有唯一环同态$\iota:\Z\to R, n\mapsto n1_R$,其核$\ker \iota=(\charfield R)\Z$,我们据此定义环$R$的特征$\charfield R\geq 0$.换言之,$\charfield R$是$1_R$在加法群中的阶(若阶无限则特征为0).\footnote{对于不一定含幺的环$R, \charfield R=\min\cbr{n\geq 1|\forall a\in R-\cbr{0}, na=0}=R$中非零元的最大加法阶.}整环的特征仅有零或质数.

\paragraph{多项式环的泛性质}
给定$A=\cbr{a_1,\cdots,a_n}$与任意交换环$R$,在以$(j:A\to R, R)$为对象的范畴(态射为诱导的环同态)中,$(i:a_i\mapsto x_i,\Z[x_1,\cdots,x_n])$为始对象.详细来说,对于任意$j:A\to R$,存在唯一的环同态$\varphi:\Z[x_1,\cdots,x_n]\to R, n\mapsto n1_{R}, x_i\mapsto j(x_i)$.此处要求交换环是因为$R$中的乘法与多项式乘法不一致,因此需要令不定元的像$\varphi(x_i)=j(a_i)$与任意同态像元素$\varphi(n)=n1_R$交换,以使$\varphi$仍保持乘法运算从而成为环同态.

$n=1$时,这一泛性质在任意(含幺)环上都存在\footnote{任取$s$作为$j(a)$,其总与其它像元素交换,即$s\varphi(n)=s(n1_R)=s(1+\cdots+1)=ns=\varphi(n)s$.}.更进一步的,对于给定环同态$\alpha:R\to S$,若有$s\in S$与$\alpha(r) (\forall r\in R)$交换,则$\alpha$有唯一环同态延拓$\bar{\alpha}:R[x]\to S, x\mapsto s$.\footnote{存在性容易构造,唯一性的重点在于$\bar{\alpha}|_R$是否唯一,以使其与$x\mapsto s$相容.在$\Z[x]$情形下其唯一性由$\Z$的泛性质保证,而在此处已固定$\alpha=\bar{\alpha}|_R$,故仍唯一.}由这一结果,我们对交换环上的多项式总有取值映射$\bar{\alpha}:R[x]\mapsto R, f(x)\mapsto \sum_{i\geq 0}a_ir^i=f(r)$,其由上述$\alpha=\id_R$导出.换言之,多项式决定了一个多项式函数$f:r\mapsto f(r)$.

\paragraph{单满态射与积}
单态射的情形与$\Grp$中相同:对于环同态,单同态$\iff$核平凡$\iff$单射.同样也有类似的子环定义(嵌入映射是单同态).但对于满同态,我们仅有:满射$\implies$满同态,其反例是嵌入$\iota:\Z\to\Q$是满同态\footnote{$\Z$到任意环的同态唯一,故使$\Q$到任意环的同态唯一,从而有右消去律.}而非满射.从而在$\Ring$中``单+满$\impliedby$同构''但反向不成立.

这一问题其实提示我们:在一般的范畴中,态射可能过多或过少,因此可能导致满射态射比满态射要更多.进一步的,在具有泛对象的范畴中,泛对象的性质也可能导致这一结果.我们看到了$\Set$和$\Grp$中泛对象都是平凡的,但$\Ring$中的始对象不平凡导致满态射要更少.

对于环的积,其与群的情形相同,考虑分量运算的构造即可.但对于余积,我们需要在未来考虑张量积运算.

\paragraph{$\End_{\Ab}(G)$}
考虑交换群的自同态集,其关于逐点加法和复合运算构成环.事实上,我们有环同构$\End_{\Ab}(\Z)\cong\Z$\footnote{考虑$\varphi:\Z\to\End_{\Ab}(\Z),n\mapsto n\id_{\Z}, n\id_{\Z}:k\mapsto nk$,其有逆$\alpha\mapsto \alpha(1)$.},以及类似的$\End_{\Ab}(\Z/n)\cong \Z/n$.更进一步,我们有Cayley定理的环版本:对于任意环$R$中元素的左乘作用$\lambda_r:a\mapsto ra, \lambda:R\to \End_{\Ab}(R), r\mapsto \lambda_r$给出了环单同态.

由于$\Z$上的环自同态与左乘作用完全一致,因此由$\lambda: (\Z,+,\circ)\overset{\sim}{\to} \End_{\Ab}(\Z)\cong (\Z,+,\cdot)$知,$(\Z,+)$上的环(在同构意义下)仅有一种.对于任意环$R$,我们仅有更弱的结论:$\Cen(\End_{\Ab}(R))\cong \Cen(R)$\footnote{$\alpha\in \Cen(\End_{\Ab}(R))\implies \alpha\circ\mu_r=\mu_r\circ\alpha\implies\alpha(r)=\alpha(1)r,\alpha=\lambda_{\alpha(1)}$,因此可以验证$\Cen(\End_{\Ab}(R))\to \Cen(R),\alpha\mapsto\alpha(1)$成为$\lambda|_{\Cen(R)}$的逆.},从而对于交换环有$\Cen(\End_{\Ab}(R))\cong R$.

\subsection{理想与商环}
环$R$的左理想是指$I<(R,+)$满足$RI\subset I$,即$\forall r\in R\forall a\in I: ra\in I$.类似定义右理想.(双边)理想即同时为左理想与右理想.含幺理想仅有$R$本身,在不要求含幺的环定义中理想是子环,而在本书定义中仅为子模.称$R$的平凡理想为$\cbr{0}$与$R$本身.
\begin{itemize}
    \item 对于环同态$\varphi:R\to S, I\lhd R, J\lhd S$,则$\varphi\rev(J)\lhd R$, 因此$\ker\varphi\lhd R$. 而$\varphi(I)\lhd \im\varphi$但非$S$的理想.
    \item 理想$I_\alpha$的(有限)和$\sum_{\alpha\in A}=\cbr{\sum_{\alpha\in A}r_\alpha\big|r_\alpha\in I_\alpha 仅有限非0}$和任意交$\bigcap_{\alpha\in A}I_\alpha$仍为理想.
    \item 除环等价于仅有平凡(即零环和本身)的左/右理想,而仅有平凡(双边)理想的环称为单环,交换单环即域.
    \item 交换环$R$中全体幂零元构成理想$N$,称之为诣零根(nilradical).非交换环中其不存在.$R/N$中不含非零幂零元,称之为约化环.
\end{itemize}

若环$R$有理想$I$,其有商环$R/I$\footnote{商环的基础是加法交换群的商群,其良定下自动成为环,但为使商良定,换言之为使$\pi$成为环同态,$I$作为理想是充要的.},其运算直接由$R$的运算导出,且有环满同态$\pi:R\to R/I, r\mapsto r+I$.换言之,环同态的核与理想的关系是与群的情况一致的:每个核都是理想,而每个理想都有自然投影使其成为核.

\paragraph{典范分解}
基于上述内容,我们有其与群的情形完全类似的泛性质与典范分解:
\begin{itemize}
    \item 商环的泛性质:$I\lhd R, \varphi:R\to S, I\subset \ker\varphi$,则有交换图
    \begin{tikzcd}
        & R \arrow[ld, "\pi"', two heads] \arrow[rd, "\varphi"] &    \\
        R/I \arrow[rr, "\exists!\tilde{\varphi}"] & & S
    \end{tikzcd}
    \item 任意环同态$\varphi:R\to S$总有典范分解
    \begin{tikzcd}%[sep=large]
        G \arrow[r, "\varphi"] \arrow[d, "\pi", two heads] & G'\\
        G/\ker\varphi \arrow[r, "\tilde{\varphi}"] & \im\varphi \arrow[u, "i", hook]
    \end{tikzcd}
    其中$\tilde{\varphi}$是由$\varphi$诱导的同构.
\end{itemize}

因此我们也有完全类似的同构定理:
\begin{itemize}
    \item $\im \varphi\cong R/\ker \varphi$.
    \item $u:\cbr{J\lhd R|I<J}\to \cbr{J'\lhd R/I}, J\mapsto J/I$是保(包含)序双射.
    \item $I\lhd R, I\subset J\lhd R$,则$J/I\lhd R/I$且$\frac{R/I}{J/I}\cong \frac{R}{J}$.考虑$\varphi|_{I+\ker\varphi}$的典范分解,有特例$\frac{R/\ker\varphi}{\varphi(I)}\cong\frac{R}{I+\ker\varphi}$.
    \item 第二同构定理略有不同:若$S<R,I\lhd R$,则$I\lhd S+I<R, S\cap I\lhd S$,且有$\frac{S}{S\cap I}\cong \frac{S+I}{I}$.
\end{itemize}

\subsection{主理想,素理想与极大理想}
\paragraph{主理想}
$a$的左右主理想为$Ra$和$aR$,定义$a$生成的主理想$(a)=RaR=\cbr{\sum_{i=1}^n r_ias_i\bigg| r_i,s_i\in R}$,其等价定义为$R$中含$a$的最小理想(或所有理想的交),这一定义也可类似推广至子集上.考虑$R$交换的情形,此时$(a)=Ra=aR$,由于总有$(a_{\alpha})_{\alpha\in A}=\sum_{\alpha\in A}(a_{\alpha})$,故有有限生成理想$(a_1,\cdots,a_n)=(a_1)+\cdots+(a_n)=\cbr{\sum_{i=1}^n r_ia_i\bigg| r_i\in R}$,基于此我们有同构$\frac{R/(a)}{(\bar{b})}\cong \frac{R}{(a,b)}$,其中$\bar{b}=b+(a)\in R/(a)$.另外我们对理想$I_i(i\in [n])$有乘积理想$I_1\cdots I_n=\sum_{a_i\in I_i}(a_1\cdots a_n)\subset \bigcap_{1\leq i\leq n}I_i$.若$R$交换时$\sum_{i=1}^n I_i=R$,或$R/(I_1\cdots I_n)$是约化环,则乘积理想$=$交理想.

称交换环$R$是Noether环(Noetherian ring),若每个理想均有限生成,称整环是主理想整环(PID),若每个理想都是主理想.容易看出,$\Z$是PID,且$(n)=n\Z, (m,n)=(\gcd(m,n))$,但$\Z[x]$不是,因为$(2,x)$不能由一个元素生成.

\paragraph{多项式环的商}
任意环$R$上的多项式环$R[x]$中的首一多项式$f$对任意多项式有能良定的带余除法.\footnote{即$\forall g\in R[x]\exists! q,r\in R[x]: g=fq+r$且$\deg r<\deg g$.存在性:记$d=\deg f$,对于$\deg g=n>d$可构造性的通过$g=ax^{n-d}f+h, \deg h<\deg g$说明这样的操作可以降次,再归纳的用$f$除$h$可以最终得到余项$r, \deg r<\deg f$,因此存在性得证.唯一性:$fq_1+r_1=fq_2+r_2\implies f(q_1-q_2)=r_1-r_2$,比较次数说明两个差都是零,因此唯一.}而对于$R$交换的情形,带余除法表明对于首一多项式$f, \forall g\exists! r: g+(f)=r+(f)$且$\deg r<\deg f$.记$d=\deg f$,通过将次数$<d$的多项式$r(x)=r_0+\cdots r_{d-1}x^{d-1}$对应到其系数$(r_0,\cdots,r_{d-1})\in R^{\oplus d}$,带余除法给定了关于$f$的群满同态$\varphi:R[x]\to R^{\oplus d}, g\mapsto (r_0,\cdots ,r_{d-1})$以及群同构$R[x]/(f)\cong R^{\oplus \deg f}$.

对某些特别情况我们有更强的结果:
\begin{itemize}
    \item $\varphi:R[x]\to R,g\mapsto g(a)$给出的$R[x]/(x-a)\cong R$是环同构.换言之即带余除法$f(x)=q(x)(x-a)+f(a)$.
    \item 对于$f(x)=x^2+1$,通过在$R\oplus R$中定义乘法结构$(a_0,a_1)\cdot (b_0,b_1)=(a_0b_0-a_1b_1,a_0b_1+a_1b_0)$,这使$R\oplus R$成为环且上述加法群同态$\varphi$成为环同态,故$R[x]/(x^2+1)\cong R\oplus R$成为环同构.
    \item 对于$R=\R$,由此有$\R[x]/(x^2+1)\cong \C$,由下可说明$(x^2+1)$是$\R[x]$的素(极大)理想.
    \item $\Q[t]/(t^2-d)\cong \Q(\sqrt{d})$.
    \item 交换环$R$上的多项式$f_1,\cdots,f_r\in R[x]$以及$a\in R$有$(f_1(x),\cdots,f_r(x),x-a)=(f_1(a),\cdots,f_r(a),x-a)$\footnote{$f_i(x)=q_i(x)(x-a)+f_i(a)\implies (f_i(x))\subset (f_i(a),x-a)$以及$(f_i(a))\subset (f_i(x),x-a)$,同理易证等式,后面同构由上节定理.},因此有$\frac{R[x]}{(f_1(x),\cdots,f_r(x),x-a)}\cong \frac{R}{(f_1(a),\cdots,f_r(a))}$.多元情况下也有$\frac{R[x_1,\cdots,x_n]}{(x_1-a_1,\cdots,x_n-a_n)}\cong R$.
\end{itemize}

\paragraph{素理想与极大理想}
考虑环$R$中的真理想$I$,若$R$交换且$R/I$为整环(即$ab\in I\implies a\in I 或 b\in I$)则称$I$为素理想,其全集即素谱$\Spec R$;若$R/I$是单环($R$交换时即域)则称$I$为极大理想,其等价于$R$中没有真包含$I$的真理想.由定义,$R$交换时有极大理想$\implies$素理想,若此时$R/I$有限则两者等价.$R$是整环$\iff (0)$是素理想,而$R$是PID时,极大理想$=$非零素理想.\footnote{前者由有限交换整环为域.对于后者,考虑理想$I=(a)\subset J=(b)$,由存在$c$使$a=bc$,而由素知$b\in (a)\implies I=J$或$c\in (a)\implies c=da,a=bc=bda\implies bd=1, J=R$.}另外对于交换环,素理想的原像仍为素理想,但极大理想没有该性质.\footnote{反例为$\iota:\Z\hookrightarrow \Q$,后者的极大理想仅有$(0)$.}

对于$R[x]$,上小节同构表明理想$(x-a)$是其素(极大)理想等价于$R$是整环(域),而$(2,x)\lhd\Z[x]$是素理想,因为其商环为$\Z/2$.对于PID $\Z$中理想$(n)=n\Z, n$为素数$\iff (n)$为非零素(即极大)理想.换言之,$\Spec\Z=\cbr{(p)|p 是质数或 0}$

对于域$\Bbbk$,由$\Bbbk[x]$是PID\footnote{考虑$I\lhd \Bbbk[x]$中次数最小的首一多项式$f$,其唯一,由带余除法的余项次数小于$\deg f$可知$I$中多项式整除$f$,故$I=Rf=(f)$.}知其中非零素理想等价于极大理想.而对于代数闭域$\Bbbk$(即$f\in \Bbbk[x]$的根均在$\Bbbk$内),$\Bbbk[x]$的极大理想有且仅有$(x-c), c\in \Bbbk$\footnote{若$I=(x-c)$则$\Bbbk[x]/I\cong\Bbbk$,即$I$是极大理想;若$I=(f)$是极大理想,则由代数闭域知$f(x)=q(x)(x-c), I\subset (x-c)$,由极大知$I=(x-c)$.},故可见$\Spec \Bbbk[x]=\cbr{(x-c)|c\in\Bbbk}\cup \cbr{(0)}$.对于$\Bbbk=\C$,由此可见$\C[x]$的非零素理想分布在一条复``直线''上,这表明其Krull维数为1.交换环$R$的(Krull)维数$\dim R$即素谱中的最大(包含)链长.

\subsection{$R$-模}
\paragraph{$R$-模}
(左)$R$-模就是环$R$在交换群$M$上的(左)环作用,即环同态$\sigma:R\to \End_{\Ab}(M)$,记$\sigma(r)(m)=rm$.\footnote{需要注意的是,对于同样的$R$和$M$也可以有不同的环作用使之成为不同的模,因此应当认识到模本质上是一个环作用/环同态$\sigma$或$\rho$,其凭依的$R$和$M$都不是本质的模本身.但为简便言还是通常称$M$为模,此时默认其上有一个$R$-作用,而对不同的模也其上的作用不同.}以此言之,左$R$-模即加法交换群$M$连带环$R$与$M$间的运算$\rho:R\times M\to M, (r,m)\mapsto rm$,其满足$M$-线性$r(m+n)=rm+rn$, $R$-线性$(r+s)m=rm+sm$, (作用)结合$(rs)m=r(sm)$, (作用)含幺$1m=m$.所有交换群$M$都能对应到唯一的$\Z$-模$M$上,由作用$\sigma$唯一.

$R$-模间的态射即保持交换群运算和$R$-作用不变的同态\footnote{即$\varphi(m+n)=\varphi(m)+\varphi(n), \varphi(rm)=r\varphi(m)$,其与$G\hyphen\Set$中态射一致.},由此全体$R$-模构成范畴$R\hyphen\Mod$,其中有零对象平凡模.另外,$R\hyphen\Mod$中的双射态射自然成为同构.易见$\Z\hyphen\Mod$与$\Ab$等价,且$\Hom_{R\hyphen\Mod}(M,N)\subset\Hom_{\Ab}(M,N)$\footnote{有时记$\Hom_{R\hyphen\Mod}$为$\Hom_R$.}.$R$交换时$R\hyphen\Mod$与$\Ab$类似:$\Hom_{\Ab}(M,N)$(关于复合)构成交换群,同样的$\Hom_{R\hyphen\Mod}(M,N)$成为$R$-模,且此时有模同构$\Hom_{R}(R,M)\cong M$.\footnote{交换性源于要求$r\varphi(r'm)=r'[r\varphi(m)]$.同构可以取$m\mapsto (\lambda_m:1_R\mapsto m)$.}最后,若$R=\Bbbk$为域,称模为$\Bbbk$-向量空间,其构成范畴$\Bbbk\hyphen\Vect$,态射即线性映射.

\paragraph{$R$-代数}
对于给定的环同态$\alpha:R\to S$,可用同态$\rho:R\times S\to S, (r,s)\mapsto \alpha(r)s$定义(左)$R$-模$S$.容易看出,若令$S=R,\alpha=\id_R$,则$R$同样可以成为自身的$R$-模.若$R$交换且$\alpha(R)<\Cen(S)$,称此时的环同态$\alpha$是一个$R$-代数\footnote{出于语言简便,有时也称$S$是一个$R$-代数.},其交换即$S$交换.此时左右模相同,且可在模上附加环运算$\rho:(s_1,s_2)\mapsto s_1s_2$,进而有$(r_1s_1)(r_2s_2)=(r_1r_2)(s_1s_2)$,因此称$S$中的乘法是$R$-双线性的.对于域$\K$及其子域$\Bbbk$,前者可被视为$\kfield$-代数,此时称$\K$为$\kfield$的扩张.

由上可见,$R$-代数$\alpha$(或$S$)即带有$R$-模结构的环$S$,$R$-代数间的态射即同时保持环与模结构的同态\footnote{保持加法和乘法运算$\varphi(s_1s_2)=\varphi(s_1)\varphi(s_2),\varphi(s_1+s_2)=\varphi(s_1)+\varphi(s_2)$,保持幺元$\varphi(1)=1$,保持$R$-作用$\varphi(rs)=r\varphi(s)$.可见代数同态相当于保持$R$-作用的环同态.},由此构成范畴$R\hyphen\Alg$,其始对象即$R$.可见$\Z\hyphen\Alg$与$\Ring$等价\footnote{上见交换群与附加$\Z$-模结构的模等价,而此处仅同时增加了环结构.}.环$R$交换时,复合运算使$\iota:R\to \End_{R}(M),r\mapsto r\id_M$成为$R$-代数.另外,交换$R$-代数同样构成范畴$R\hyphen\Comm\Alg$,其为交换环范畴上的余切片范畴$\Comm\Ring^{R}$.交换环$R$上的多项式环$R[x_1,\cdots,x_n]$\footnote{准确地说,是嵌入$\iota:R\to R[x_1,\cdots,x_n]$.}是一个交换$R$-代数.

\paragraph{子模与商模}
$R$-模$M$的子模$N$也被自然定义:$N$是$R$-模且嵌入$\iota:N\to M$是模同态.换言之,$N$是$M$的子群且在$R$-作用下封闭:$\forall r\in R\forall a\in N: ra\in N$.可见$R$若作为自身的$R$-模,则其(左)子模即自身的(左)理想.模同态的核与像均为子模,且子模的和与交均为子模.若$r\in \Cen(R), I\lhd R$,则$rM$与$IM=\cbr{\sum_{i=1}^n r_im_i\bigg| r_i\in I,m_i\in M}$均为$M$的子模.

对于$M$的子模$N$,容易定义其商模$M/N$,其作为交换群的商且保持$R$-作用故同样成为$R$-模,且典范投影成为模同态.考虑环$R$上的情形,$I\lhd R<$环$S$,则$S,R,I,R/I$均为$R$-模,且$R$交换时$R$与$R/I$均成为$R$-代数.而$R$不交换且$I$为其左理想(即左子模)时,商群$R/I$成为左$R$-模.

类似群与环,商模也有泛性质与典范分解:
\begin{itemize}
    \item $N$是$R$-模$M$的子模,$R$-模同态$\varphi:M\to P$满足$N\subset \ker\varphi$,则有唯一模同态$\tilde{\varphi}:M/N\to P$使\\
    \begin{tikzcd}
        & M \arrow[ld, "\pi"', two heads] \arrow[rd, "\varphi"] &    \\
        M/N \arrow[rr, "\exists!\tilde{\varphi}"] & & P
    \end{tikzcd}
    交换.可见与正规子群或理想的情况类似,模同态的核与子模等价.但有所不同的是,核并不为某个子结构赋予更强的限制,在未来我们会看到,这是$R\hyphen\Mod$作为$\Ab$-范畴所特有的性质.
    \item $R$-模同态也可以被典范分解为满射,双射与单射的复合,即
    \begin{tikzcd}%[sep=large]
        M \arrow[r, "\varphi"] \arrow[d, "\pi", two heads] & M'\\
        M/\ker\varphi \arrow[r, "\tilde{\varphi}"] & \im\varphi \arrow[u, "i", hook]
    \end{tikzcd}
\end{itemize}
以及由典范分解得来的模同构定理
\begin{itemize}
    \item $\im\varphi\cong M/\ker\varphi$.
    \item $u:\cbr{P<M|N<P}\to \cbr{P'\lhd M/N}, P\mapsto P/N$是保(包含)序双射.
    \item $N<M, N\subset P<M$,则$P/N<M/N$且$\frac{M/N}{P/N}\cong \frac{M}{P}$.
    \item $N<M, P<M$,则$N+P<M, N\cap P<P$且$\frac{N+P}{N}\cong \frac{P}{N\cap P}$.
    \item 对于交换环$R, I,J\lhd R$,则有$R$-模同构$I\cdot (R/J)\cong (I+J)/J$.
\end{itemize}

\subsection{$R\hyphen\Mod$中的基础概念}
\paragraph{积与纤维积}
% 此处关于范畴论的讨论需要最终迁移至第零章.
在$\Ab$-范畴(如$\Ab$或$R\hyphen\Mod$)中积(切片范畴的终对象)与余积(余切片范畴的始对象)在任意情形下均存在,其中积总为直积(分量积),而余积总为直和(或称弱直积,即仅有限分量非零的积).两者在有限情形下等价,而在无限情形下余积为积的子结构.$\Grp$不是$\Ab$-范畴,因此其中余积为自由积.

对于以指标集$A$构造的$R$-模$M$的积与余积,其分别为$M^A=\prod_{a\in A}M$与$M^{\oplus A}$.尽管$M^{\oplus A}<M^A$,但在$R$交换时有$\Hom_{R}(R^{\oplus A},M)\cong M^A, \varphi\mapsto \cbr{\varphi(a)}_{a\in A}$,其有限情形下即$\Hom_R(R^n,R)\cong R^n$,此即对偶间的同构.

将上述(余)切片范畴改为纤维(fibered)形式,我们有纤维积(或拉回, pull-back)与纤维余积(或推出, push-out),如下交换图所示:给定$R$-模$M,N,A$及模同态$\mu,\nu$,存在$R$-模$M\times_A N$与模同态$\pi_M, \pi_N$($M\oplus_A N, i_M, i_N$)使得对任意$R$-模$P$及任意模同态$\varphi_M, \varphi_N$($f_M, f_N$)都有唯一模同态满足如下交换图.

\begin{tikzcd}
    P \arrow[rrd, "\varphi_N"', bend left] \arrow[rd, "\exists!" , dashed] \arrow[rdd, "\varphi_M", bend right] & & & A \arrow[r, "\nu"] \arrow[d, "\mu"'] \arrow[rd, "\boxplus", phantom] & N \arrow[d, "i_N"] \arrow[rdd, "f_N", bend left] & \\
    & M\times_A N \arrow[d, "\pi_M"] \arrow[r, "\pi_N"] \arrow[rd, "\Box", phantom] & N \arrow[d, "\nu"] & M \arrow[r, "i_M"] \arrow[rrd, "f_M", bend right] & M\oplus_A N \arrow[rd, "\exists!" description, dashed] &   \\
    & M \arrow[r, "\mu"'] & A & & & P
\end{tikzcd}
\\可知$M\times_A N=\cbr{(m,n)\in M\times N | \mu(m)=\nu(n)}, M\oplus_A N=(M\oplus N)/\cbr{(\mu(a),-\nu(a))\in M\oplus N | a\in A}$.

\paragraph{核}
\begin{wrapfigure}[7]{r}{0.25\textwidth}
\begin{tikzcd}[sep=large]
    \ker\varphi \arrow[d, "i"', hook] & P \arrow[l, "\exists!\bar{\alpha}"'] \arrow[d, "\varphi\circ\alpha=0"] \arrow[ld, "\alpha"'] \\
    M \arrow[r, "\varphi"] \arrow[d, "\beta\circ\varphi=0"'] & N \arrow[ld, "\beta"'] \arrow[d, "\pi"', two heads] \\
    Q & \coker\varphi \arrow[l, "\exists!\bar{\beta}"']
\end{tikzcd}
\end{wrapfigure}
$R\hyphen\Mod$中的核与余核也存在:考虑模同态$\varphi:M\to N$,对于满足$\varphi\circ\alpha=0$的模同态$\alpha:P\to M$为对象的范畴,$\ker\varphi$为其终对象;对于满足$\beta\circ\varphi=0$的模同态$\beta:N\to Q$为对象的范畴,$\coker\varphi\cong N/\im\varphi$为其始对象.需要注意的是,这一定义模式可以直接推广到更多范畴中.对核与余核类似也有交换图:

在$R\hyphen\Mod$中也有关于单满态射的等价关系:单态射$\iff$核平凡$\iff$单射态射;满态射$\iff$余核平凡$\iff$满射态射.这样的等价关系与$\Ab$中完全一致,这也是$\Ab$-范畴的一般性质.另外,尽管存在左(右)逆$\implies$单(满)态射,但反之不一定对.在$R\hyphen\Mod$中可以仅考虑$\Z\to \Z, n\mapsto 2n$即可.

\paragraph{自由模与自由代数}
类似自由群的定义,考虑集合$A$射到任意$R$-模$M$上的集合函数$f:A\to M$构成的范畴(态射容易诱导),其中始对象$j:A\to F^R(A)$即$A$生成的自由模,即交换图
\begin{tikzcd}
    F^R(A) \arrow[r, "\exists!\varphi"] & M \\
    A \arrow[ru, "f"'] \arrow[u, "j"]   &
\end{tikzcd}
.为了详细构造自由模的结构,我们对$R$-模$N$与集合$A$定义直和$N^{\oplus A}:=\cbr{\alpha:A\to N| 仅有限 a\in A 有 \alpha(a)\neq 0}$\footnote{实际上此处$A$即指标集,也可将该直和中的元素记作$\cbr{n_a}_{a\in A}$或$\sum_{a\in A}n_a a$,其中仅有限个$n_a\neq 0$,而这与映射定义完全相同.这也是上节余积定义的构造形式.},其容易赋有$R$-模结构,且有$(R^{\oplus A_1})^{\oplus A_2}\cong R^{\oplus A_1\times A_2}$.考虑$j:A\to R^{\oplus A}, a\mapsto \chi_a$.可以验证$j$即上述始对象,故$F^R(A)\cong R^{\oplus A}$.特别地,即$F^R([n])\cong R^{\oplus n}=R^n$.

可类似定义自由交换$R$-代数,仅考虑$A=[n]$有限情形,记$R[A]=R[x_1,\cdots,x_n]$,考虑函数$j:A\to R[A], i\mapsto x_i$,其同样成为所定义范畴的始对象,换言之$R[A]$即$A$生成的自由交换$R$-代数.\footnote{始对象中的唯一性可直接验证,也可考虑多项式环的泛性质进行唯一延拓.}换言之,(有限不定元的)多项式环即有限集生成的交换$R$-代数,也因此可见$\Z[x_1,\cdots,x_n]$的泛性质实际是其在交换环范畴$\Comm\Ring$(即$\Z\hyphen\Comm\Alg$)中的自由对象.

上述之所以要求代数交换,是为了不定元之间互相交换,而$R\hyphen\Alg$中的自由对象包含不交换多项式环$R\abr{A}$,其同构于一个(基于$A$生成的自由幺半群的)幺半群环,由$A$中所有有限长字符串构造.

{\small 综上我有一个问题:
\begin{itemize}
    \item 为什么同样对于一个集合$A$,其生成的自由$R$-模仅是以$A$中元素为下标的直和(即弱直积)$R^{\oplus A}$,而生成的自由交换$R$-代数则为以$A$中元素为不定元的多项式环?准确的说,为什么是多项式环结构?
\end{itemize}
在此我给出我的回答:
\begin{itemize}
    \item 对于自由群与自由交换群,它们都是被$A$中元素``生成''的群,且生成的方式是取元素构造字符串,只是交换情形下字符串退化成元素的直和,字符串连接也成为分量运算.
    \item 模结构实际上即向量空间的退化,即数域退化成环,因此模仍保留大量向量空间的性质.另一方面,$R$-模与(加法)交换群的本质区别在于$R$-作用,即$r\in R$作用于$m\in M$可以得到$rm\in M$,这可以被看做某种``数乘''.
    \item 综上可见,自由模是环的直和并不令人意外:一方面,模是带有$R$-作用的交换群,因此自由交换群$\Z^{\oplus A}$(元素为$\sum_{a\in A}m_a a$, $m_a\in \Z$)加上$R$-作用自然可以成为$R^{\oplus A}$(元素为$\sum_{a\in A}r_a a, r_a\in R$).另一方面,集合``生成''的向量空间即以其为基底的向量空间,其退化为模时自然带有其分量结构,即$R^{\oplus A}$.
    \item 交换$R$-代数可以视为具有交换环结构的$R$-模.限于所学,下仅讨论$A$有限情形.$A$中元素在带有环(乘法)的结构中生成,其可被视作某种不定元,且应当自然具有幂次与元素间的(交换)积.而加法与$R$-作用能自然定义加法和$R$系数,这些已经自然地给出了多项式,且其次数总有限(否则不良定).
\end{itemize}}

\paragraph{子集生成的子模和子代数}
生成子模可类似群定义:考虑$R$-模$M$及其子集$A$,上节诱导了唯一模同态$\varphi_A:R^{\oplus A}\to M$,其像即$A$(作为$R$-模)生成的子模$\abr{A}=\im \varphi_A=\cbr{\sum_{a\in A}r_a a \bigg| 仅有限 a\in A 有 r_a\neq 0}$,它也是$M$中含$A$最小子模.有限生成模即模可由此被有限集生成,有限生成模$M$的子模$N$不一定有限生成\footnote{如无穷不定元多项式环$\Z[x_1,x_2,\cdots]$,其是自身的有限生成子模,但其理想(即子模)$(x_1,x_2,\cdots)$并非有限生成.},但$N$与$M/N$为有限生成模时$M$也是(证明同下).另外,有限生成模的同态像也是有限生成模.

可类似定义生成子代数及有限生成代数.对于$R$-代数$S$,其可被视为作为模有限生成或作为代数有限生成,我们分别称之为$S$ is finite与$S$ is of finite type.作为有限生成模时$S\cong R^{\oplus n}/M$,而作为有限生成代数时$S\cong R[x_1,\cdots,x_n]/I$.$S$作为有限生成$R$-模$\implies S$作为有限生成$R$-代数.\footnote{$R[x]$是有限生成$R$-代数,但非有限生成$R$-模.}

\paragraph{Noether模}
称一个$R$-模是Noether模(Noetherian module),若其所有子模均作为$R$-模有限生成.\footnote{需要注意的是,考虑Noether模时需要注意其所凭依的环$R$,我们认为一个$R$-模是Noether模时,同样将与其相关的模看作$R$-模.}若$R$是Noether环,可见其即为Noether $R$-模. Noether模的同态像也是Noether模.对于子模$N<M$, $M$是Noether模$\iff N$和$M/N$都是Noether模.\footnote{$\implies: M/N=\pi_N(M)$显然同样是Noether模,因$N$的子模也是$M$的子模,故$N$也是Noether模. $\impliedby:$考虑$P<M$,注意到$P\cap N$有限生成且$P/(P\cap N)\cong (P+N)/N<M/N$,因此$P/(P\cap N)$有限生成,故$P$有限生成,即得证.}若$R$是Noether环,则有限生成$R$-模$M$是Noether $R$-模.\footnote{注意到$M$是某个$R^{\oplus n}$的同态像,而$R^{\oplus n}$是Noether模,可由上句归纳证明.}

\paragraph{单模与循环模}
称$R$-模$M$为单模(或不可约模),若$M$仅有平凡子模. 我们有Schur引理:单模间的非零同态仅有同构,因此单模的自同态环$\End_{R}(M)$是除环.\footnote{注意到同态核与像均为子模,同态非零则仅有同构情形,从而自同态均为自同构.}称(左)$R$-模$M$是循环模,若$M=\abr{m}=Rm, m\in M$.单模都是循环模.\footnote{若单模有多个生成元,可由此给出非平凡子模.}循环模的商模也都是循环模.\footnote{$N<M=Rm, \pi_N:M\to N, N=R\pi_N(m)$.}模$M$是循环模等价于$M\cong R/I$,其中$I$是$R$的(左)理想\footnote{$\implies: \varphi_M:R\to \abr{m}, 1\mapsto m$是满模同态,故可取$I=\ker\varphi_M$.$\impliedby:$若$\varphi:R/I\to M$是同构,取$m_0=\varphi(1+I), \forall m\in M\exists! r+I: m=\varphi(r+I)=rm_0$,故$M=\abr{m_0}$.},且此时对$R$-模$N$有$\Hom_R(M,N)=\cbr{n\in N | In=0}$\footnote{注意到$\varphi\in \Hom_R(R/I,N)$由$n_0=\varphi(1+I)\in \rhs$确定,且需满足$in_0=0, i\in I$以确保良定.},由此可知$\Hom_{\Ab}(\Z/a, \Z/b)\cong \Z/\gcd(a,b)$.

\subsection{链复形与同调}
\paragraph{链复形与正合列}
$R$-模的链复形(chain complex)是指一列$R$-模与$R$-模同态:
$$\cdots \overset{d_{i+2}}{\longrightarrow}M_{i+1}\overset{d_{i+1}}{\longrightarrow}M_{i}\overset{d_{i}}{\longrightarrow}M_{i-1}\overset{d_{i-1}}{\longrightarrow}\cdots$$
其满足$d_i\circ d_{i+1}=0$,换言之即$\im d_{i+1}\subset \ker d_i$.常记一列链复形为$(M_{\bullet},d_{\bullet})$(或仅$M_{\bullet}$),且下标随箭头减小.称同态$d_i$为边界或微分,其也常被记为$d^n, \partial_n$.称链复形在$M_i$处正合,若$\im d_{i+1}=\ker d_i$,正合列(exact sequence)即每处正合的链复形.短正合列即形如
$$0\longrightarrow L \overset{\alpha}{\longrightarrow} M \overset{\beta}{\longrightarrow} N \longrightarrow 0$$
的正合列,其等价于$\alpha$是单同态且$\beta$是满同态.由$\im \alpha=\ker \beta$可见$N\cong M/\im \alpha\cong M/L$.由此可见,对于每个模同态$\varphi:M\to M'$可以诱导短正合列
$$0\longrightarrow \ker\varphi \overset{i}{\longrightarrow} M \overset{\varphi}{\longrightarrow} \im \varphi \longrightarrow 0 \qquad\text{或}\qquad 0\longrightarrow \ker\varphi \overset{i}{\longrightarrow} M \overset{\varphi}{\longrightarrow} M' \overset{\pi}{\longrightarrow} \coker\varphi \longrightarrow 0$$
应当注意到,我们可以将每条正合列视作一系列短正合列:

\begin{center}
\begin{tikzcd}
    & M_{i+2} \arrow[rd, "d_{i+2}"] \arrow[d]      &                                          & 0 \arrow[d]                                &         \\
    0 \arrow[r] & {\scriptstyle \im d_{i+2}=\ker d_{i+1}} \arrow[r] \arrow[d] & M_{i+1} \arrow[rd, "d_{i+1}"'] \arrow[r] & {\scriptstyle \im d_{i+1}=\ker d_i} \arrow[d] \arrow[r]   & 0       \\
    & 0                                            &                                          & M_i \arrow[rd, "d_i"] \arrow[d]            &         \\
    &                                              & 0 \arrow[r]                              & {\scriptstyle \im d_{i}=\ker d_{i-1}} \arrow[r] \arrow[d] & M_{i-1} \\
    &                                              &                                          & 0                                          &        
\end{tikzcd}
\end{center}

\paragraph{分裂正合列}
称一个正合列是分裂(split)的,若其满足以下交换图(即一系列同构):
\begin{center}
\begin{tikzcd}
    0 \arrow[r] & L \arrow[r, "\alpha"] \arrow[d, "\sim"] & M \arrow[r, "\beta"] \arrow[d, "\sim"] & N \arrow[r] \arrow[d, "\sim"] & 0 \\
    0 \arrow[r] & M_1 \arrow[r, "i_1"]                    & M_1\oplus M_2 \arrow[r, "\pi_2"]       & M_2 \arrow[r]                 & 0
\end{tikzcd}
\end{center}

分裂正合列可以解答问题``什么样的同态有左(右)逆?''考虑$R$-模同态$\varphi:M\to N$, 我们有:
\begin{itemize}
    \item $\varphi$有左逆$\iff$有分裂正合列$0\longrightarrow M \overset{\varphi}{\longrightarrow} N {\longrightarrow} \coker\varphi \longrightarrow 0$, 此时称$\varphi$为分裂单同态.
    \item $\varphi$有右逆$\iff$有分裂正合列$0\longrightarrow \ker\varphi {\longrightarrow} M \overset{\varphi}{\longrightarrow} N \longrightarrow 0$, 此时称$\varphi$为分裂满同态.
\end{itemize}
{\small
仅证第一个命题,第二个类似.\begin{itemize}
    \item[$\impliedby$:]正合列分裂即$\varphi$是$M$到$M\oplus M'$的嵌入,因此有左逆$\psi=\pi_{M}, \psi\varphi=\id_M$.
    \item[$\implies:$] 令$\varphi$左逆为$\psi:N\to M$,考虑同态$M\oplus \ker\psi\to N, (m,k)\mapsto \varphi(m)+k$,其逆为$n\mapsto (\psi(n), n-\varphi\psi(n))$,故其为同构,即$M\oplus \ker\psi\cong N$,且由此有$\varphi:m\mapsto \varphi(m)\leftrightarrow (m,0)$,此即嵌入$i_M:M\mapsto M\oplus \ker\psi$.最后,$\coker\varphi\to \ker\psi$也有同构$n+\im\varphi\mapsto n-\varphi\psi(n)$,因此正合列分裂.有下图所示:
\begin{center}
\begin{tikzcd}
    0 \arrow[r] & \underset{m}{M} \arrow[r, "\varphi"] \arrow[d, "\id_{M}"] & \underset{n=\varphi(m)}{N} \arrow[r] \arrow[d, "\sim"] & \underset{n+\im\varphi}{\coker\varphi} \arrow[r] \arrow[d, "\sim"] & 0 \\
    0 \arrow[r] & \underset{m}{M} \arrow[r, "i_M"] & \underset{(\psi(n),n-\varphi\psi(n))=(m,0)}{M\oplus \ker\psi} \arrow[r] & \underset{n-\varphi\psi(n)}{\ker\psi} \arrow[r] & 0 \\
\end{tikzcd}
\end{center}
\end{itemize}}

\paragraph{同调与蛇形引理}
对链复形
$$M_{\bullet}:\cdots \overset{d_{i+2}}{\longrightarrow}M_{i+1}\overset{d_{i+1}}{\longrightarrow}M_{i}\overset{d_{i}}{\longrightarrow}M_{i-1}\overset{d_{i-1}}{\longrightarrow}\cdots$$
可定义其同调(模)$H_i(M_{\bullet})=\ker d_i/ \im d_{i+1}$,可见$H_i(M_{\bullet})=0\iff M_i$处正合.对于链复形
$$M_{\bullet}: 0\longrightarrow M_1 \overset{\varphi}{\longrightarrow} M_0 \longrightarrow 0$$
可见$H_1(M_{\bullet})\cong \ker\varphi, H_0(M_{\bullet})\cong \coker\varphi$.若其正合则同调平凡,即$\varphi$是同构.

考虑两正合列,其间有对应同态:
\begin{center}
\begin{tikzcd}
    0 \arrow[r] & L_1 \arrow[r, "\alpha_1"] \arrow[d, "\lambda"] & M_1 \arrow[r, "\beta_1"] \arrow[d, "\mu"] & N_1 \arrow[r] \arrow[d, "\nu"] & 0 \\
    0 \arrow[r] & L_0 \arrow[r, "\alpha_0"]                      & M_0 \arrow[r, "\beta_0"]                  & N_0 \arrow[r]                  & 0
\end{tikzcd}
\end{center}
我们有蛇形引理,即存在正合列\footnote{若不令$\alpha_1, \beta_0$分别为单射与满射,即去掉交换图左上角与右下角的0,则结论也需去掉正合列两端的0.这是蛇形引理更一般的版本.}
$$0\longrightarrow \ker\lambda \longrightarrow \ker\mu \longrightarrow \ker\nu \overset{\delta}{\longrightarrow} \coker\lambda \longrightarrow \coker\mu \longrightarrow \coker\nu \longrightarrow 0$$
或
\begin{center}
\begin{tikzcd}
    0 \arrow[r] & H_1(L_{\bullet}) \arrow[r] & H_1(M_{\bullet}) \arrow[r] & H_1(N_{\bullet}) \arrow[lld, out=-30, in=150, "\delta"'] &   \\
    & H_0(L_{\bullet}) \arrow[r] & H_0(M_{\bullet}) \arrow[r] & H_0(N_{\bullet}) \arrow[r]              & 0
\end{tikzcd}
\end{center}

\begin{center}
\begin{tikzcd}
    & 0 \arrow[d]                                    & 0 \arrow[d]                               & 0 \arrow[d]                                             &   \\
    0 \arrow[r] & \ker\lambda \arrow[d] \arrow[r]                & \ker\mu \arrow[d] \arrow[r]               & \underset{a}{\ker\nu} \arrow[d] \arrow[llddd, out=-30, in=150, red, "\delta"'] &   \\
    0 \arrow[r] & L_1 \arrow[r, "\alpha_1"] \arrow[d, "\lambda"] & \underset{b}{M_1} \arrow[r, "\beta_1"] \arrow[d, "\mu"] & \underset{a}{N_1} \arrow[r] \arrow[d, "\nu"]                          & 0 \\
    0 \arrow[r] & \underset{c}{L_0} \arrow[r, "\alpha_0"] \arrow[d]            & \underset{\mu(b)}{M_0} \arrow[r, "\beta_0"] \arrow[d]        & N_0 \arrow[r] \arrow[d]                                 & 0 \\
    & \underset{c+\im\lambda}{\coker\lambda} \arrow[d] \arrow[r]              & \coker\mu \arrow[d] \arrow[r]             & \coker\nu \arrow[d] \arrow[r]                           & 0 \\
    & 0                                              & 0                                         & 0                                                       &  
\end{tikzcd}
\end{center}

\small
蛇形引理的证明即对如上交换图的阐释.
\begin{enumerate}
    \item 首先应注意到图中横纵列均为正合列.仅需考虑第一行与第四行的正合性.由$\alpha_1$单与$\beta_0$满知$\ker\lambda$与$\coker\nu$处正合.
    \begin{itemize}
        \item 对于$\ker\mu$处,即证$\alpha_1(\ker\lambda)=\ker\mu\cap \ker\beta_1=\ker\mu\cap \im \alpha_1$.$\subset: \forall a\in \ker\lambda, \alpha_1(a)\in \im\alpha_1$, 且$\mu\alpha_1(a)=\alpha_0 \lambda(a)=0, \alpha_1(a)\in \ker\mu$.$\supset: \forall b\in \ker\mu\cap \im\alpha_1 \exists a\in L_1, b=\alpha_1(a), \mu(b)=\mu\alpha_1(a)=\alpha_0\lambda(a)=0$,由$\alpha_0$单知$a\in\ker\lambda$.
        \item 对于$\coker\mu$处,由$\alpha_0(\im\lambda)\subset \im\mu$知$\bar{\alpha}_0:a+\im\lambda\mapsto \alpha_0(a)+\im\mu$良定,$\bar{\beta}_0$同理.即证$\bar{\alpha}_0(\coker\lambda)=\ker\bar{\beta}_0$.$\subset: \forall a+\im\lambda\in\coker\lambda, \bar{\beta}_0\bar{\alpha}_0(a+\im\lambda)=\beta_0\alpha_0(a)+\im\nu=\im\nu$,因此$\bar{\alpha}_0(a+\im\lambda)\in \ker\bar{\beta}_0$.$\supset: \forall b+\im\mu\in\ker\bar{\beta}_0$, $\beta_0(b)\in\im\nu$, 故$\exists c\in N_1, \nu(c)=\beta_0(b)$,又由$\beta_1$满知$\exists d\in M_1, c=\beta_1(d)$, 因此$\beta_0(b)=\nu\beta_1(d)=\beta_0\mu(d), b-\mu(d)\in\ker\beta_0=\im\alpha_0$.因此$\exists a\in L_0, \alpha_0(a)=b-\mu(d), \bar{\alpha}_0(a+\im\lambda)=b+\im\mu$.
    \end{itemize}
    故该两处正合,因此得证.
    \item $\delta:\ker\nu\to\coker\lambda, a\mapsto \alpha_0\rev \mu\beta_1\rev(a)+\im\lambda$的定义.如图:$\forall a\in\ker\nu$,其在嵌入至$N_1$中后由$\beta_1$满知有原像$b\in M_1, a=\beta_1(b)$.由$\beta_0\mu(b)=\nu\beta_1(b)=\nu(a)=0$知$\mu(b)\in\ker\beta_0=\im\alpha_0$,故有$c\in L_0, \mu(b)=\alpha_0(c)$.最后令$\delta(a)=c+\im\lambda$即可.对于其良定性,首先由$\alpha_0$单知关于$b$的$c$唯一,而考虑$a$的不同原像$b,b'\in M_1, b-b'\in\ker\beta_1=\im\alpha_1$, 即有$g\in L_1, b-b'=\alpha_1(g), \mu(b-b')=\mu\alpha_1(g)=\alpha_0\lambda(g)$,即$c-c'=\lambda(g)$,故可知不同的原像$b$仍使$\delta$的像$c+\im\lambda$不变.
    \item 最后说明$\ker\nu$与$\coker\lambda$处正合.
    \begin{itemize}
        \item 对于$\ker\nu$, 即证$\beta_1(\ker\mu)=\ker\delta$.$\subset: \forall a\in \ker\mu, \delta\beta_1(a)=\alpha_0\rev\mu(a)+\im\lambda=\im \lambda$.$\supset: \forall a\in \ker\delta, \delta(a)=\im\lambda$, $\alpha_0\rev \mu\beta_1\rev(a)\in \im\lambda$, 故$\exists b\in L_1, \mu\beta_1\rev(a)=\alpha_0\lambda(b)=\mu\alpha_1(b), \beta_1\rev(a)-\alpha_1(b)\in\ker\mu, \beta_1(\beta_1\rev(a)-\alpha_1(b))=a$.
        \item 对于$\coker\lambda$, 即证$\im\delta=\ker\bar{\alpha}_0$.$\subset: \forall a\in\ker\nu, \bar{\alpha}_0\delta(a)=\mu\beta_1\rev(a)+\im\mu=\im\mu$.$\supset: \forall c+\im\lambda\in \ker\bar{\alpha}_0, \alpha_0(c)\in \im\mu$, 即$\exists b\in M_1, \alpha_0(c)=\mu(b)$,故有$a=\beta_1(b), \nu(a)=\beta_0\mu(b)=\beta_0\alpha_0(c)=0, a\in\ker\nu$,由$\delta$定义知$\delta(a)=c+\im\lambda$.
    \end{itemize}
\end{enumerate}
\normalsize

蛇形引理有直接推论:若$\mu$满且$\nu$单,则$\lambda$满且$\nu$是同构.\footnote{此条件下即仅有$0\longrightarrow \ker\lambda \longrightarrow \ker\mu \longrightarrow \underset{\ker\nu}{0} \overset{\delta}{\longrightarrow} \coker\lambda \longrightarrow \underset{\coker\mu}{0} \longrightarrow \coker\nu \longrightarrow 0$,从而结论显然.}以及短五引理:$\lambda,\nu$均为同构,则$\mu$也是同构.由此可见分裂正合列的定义中,$M\cong M_1\oplus M_2$的同构可由其它两同构推出.

\paragraph{正合列的应用}
\begin{itemize}
    \item 若有正合列
    $$\cdots\longrightarrow L\longrightarrow M\longrightarrow N\longrightarrow\cdots$$
    且$L,N$是Noether模,则$M$也是.\footnote{注意到$N\cong M/L$及Noether模的等价条件.}
    \item 对于$R$-模$L,M,N,P$有正合列
    $$0\longrightarrow L\overset{\alpha}{\longrightarrow} M\overset{\beta}{\longrightarrow} N\longrightarrow 0$$
    其可诱导(交换群,$R$交换时则为模)正合列\footnote{首先证明$a:\varphi\mapsto \varphi\beta$是单射,仅需考虑$\varphi\beta=0=0\beta\implies \varphi=0$.对于$\Hom_R(M,P)$处的正合,即证$\im a=\ker b$,一方面$\psi\in\im a, b(\psi)=\psi\alpha=\varphi\beta\alpha=0$,另一方面由$N\cong\coker\alpha$的泛性质,对任意$\psi\in\ker b$有唯一$\varphi, a(\varphi)=\varphi\beta=\psi$.}
    $$0\longrightarrow\Hom_R(N,P)\overset{\varphi\mapsto\varphi\beta}{\longrightarrow}\Hom_R(M,P)\overset{\psi\mapsto\psi\alpha}{\longrightarrow}\Hom_R(L,P)$$
    而原正合列分裂时,该诱导的正合列末尾有$\longrightarrow 0$,即最右端同态为满射.而$N$为自由模时,原正合列分裂.\footnote{即证此时$\beta$有右逆即可使正合列分裂.由$N=R^{\oplus A}$考虑$\beta(m_a)=a$, $m_a$总存在(但不唯一),故可构造$\sigma:N\to M, n=\sum_{a\in A}r_a a\mapsto \sum_{a\in A}r_a m_a, \beta\circ\sigma=\id_N$,故有右逆.$\sigma$的良定源于$N$自由,即$n\in N$可被分解为$a$的唯一线性组合.}
    \item 四引理(four-lemma)与五引理(five-lemma):考虑如下交换图,其中横行均为正合列.
    \begin{center}
    \begin{tikzcd}
        A_1 \arrow[r, "f_4"] \arrow[d, "\alpha"] & B_1 \arrow[r, "f_3"] \arrow[d, "\beta"] & C_1 \arrow[r, "f_2"] \arrow[d, "\gamma"] & D_1 \arrow[d, "\delta"] \arrow[r, "f_1"] & E_1 \arrow[d, "\epsilon"] \\
        A_0 \arrow[r, "g_4"]                     & B_0 \arrow[r, "g_3"]                    & C_0 \arrow[r, "g_2"]                     & D_0 \arrow[r, "g_1"]                     & E_0                      
    \end{tikzcd}
    \end{center}
    四引理即(1)$\alpha$满且$\beta,\delta$单,则$\gamma$单;(2)$\epsilon$单且$\beta,\delta$满,则$\gamma$满.\footnote{(1)考虑$c_1\in C_1, \gamma(c_1)=0$,则$g_2\gamma(c_1)=\delta f_2(c_1)=0$,由$\delta$单知$c_1\in \ker f_2=\im f_3$,故有$b_1\in B_1, c_1=f_3(b_1), g_3\beta(b_1)=0$.故$\beta(b_1)\in \ker g_3=\im g_4$,故有$a_1\in A_1, \beta f_4(a_1)=g_4\alpha(a_1)=\beta(b_1)$, 由$\beta$单知$f_4(a_1)=b_1, c_1=f_4f_3(a_1)=0$,故$\gamma$单.\\ (2)对$c_0\in C_0$有$d_1\in D_1, \delta(d_1)=g_2(c_0)$,而$\epsilon f_1(d_1)=g_1\delta(d_1)=g_1g_2(c_0)=0$,由$\epsilon$单知$d_1\in \ker f_1=\im f_2$,故有$c_1\in C_1, f_2(c_1)=d_1$,即$g_2(c_0)=\delta(d_1)=\delta f_2(c_1)=g_2\gamma(c_1), c_0-\gamma(c_1)\in \ker g_2=\im g_3$,故有$b_1\in B_1, c_0-\gamma(c_1)=g_3\beta(b_1)=\gamma f_3(b_1), c_0=\gamma(c_1+f_3(b_1))\in \im \gamma$.}(其均不涉及第五个态射)\\
    五引理即其直接推论:$\beta,\delta$同构,$\alpha$满且$\epsilon$单,则$\gamma$同构.
    \item 九引理(nine-lemma):考虑如下交换图,其中横行均为正合列.
    \begin{center}
    \begin{tikzcd}
        & 0 \arrow[d]             & 0 \arrow[d]                       & 0 \arrow[d]             &   \\
        0 \arrow[r] & L_2 \arrow[r] \arrow[d] & M_2 \arrow[r] \arrow[d, "\alpha"] & N_2 \arrow[r] \arrow[d] & 0 \\
        0 \arrow[r] & L_1 \arrow[r] \arrow[d] & M_1 \arrow[r] \arrow[d, "\beta"]  & N_1 \arrow[r] \arrow[d] & 0 \\
        0 \arrow[r] & L_0 \arrow[r] \arrow[d] & M_0 \arrow[r] \arrow[d]           & N_0 \arrow[r] \arrow[d] & 0 \\
        & 0                       & 0                                 & 0                       &  
    \end{tikzcd}
    \end{center}
    则(1)左端两列或右端两列正合,则剩下一列也正合;(2)若左右两列正合,则$\alpha$单且$\beta$满,而中间列为链复形时则同样正合;(3)中间列为链复形时其它列也是,此时任意两列正合时剩下一列也正合.其对一般链复形也成立.
\end{itemize}
\end{document}