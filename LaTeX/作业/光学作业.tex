\documentclass{article}
\title{光学作业}
\author{章亦流}
\input{../newcommand.tex}
\newcommand{\cm}{\mathrm{cm}}

\begin{document}
\maketitle
\paragraph{题1.10}顶角为$50^\circ$的三棱镜的最小偏向角为$35^\circ$,如果把它浸入水中,最小偏向角是多少?(水的折射率为1.33.)

\paragraph{题1.13}设光导纤维玻璃芯与外套的折射率分别为$n_1,n_2(n_1>n_2)$,垂直端面外介质的折射率为$n_0$.试证明,能使光线在纤维内发生全反射的入射光束的最大孔径角满足$n_1\sin\theta_1=\sqrt{n_1^2-n_2^2}$.

\paragraph{题1.15}极限法测液体折射率的装置如本题图所示,$ABC$是直角棱镜,已知其折射率$n_g$.将待测液体涂一薄层于其上表面$AB$,覆盖一块毛玻璃,用扩展u昂元在掠入射的方向照明.从棱镜$AC$面出射的光线的折射角将有一下限$i'$(用望远镜观察,则在视场中出现有明显分界线的半明半暗区).试证明,待测液体的折射率$n=\sqrt{n_g^2-\sin^2 i'}$.用这种方法测液体折射率,测量范围受到什么限制?

\paragraph{题2.5}凹面镜半径40cm,物体放在何处成放大两倍的实像?放在何处成放大两倍的虚像?
\begin{proof}
    联立$$\frac{1}{s}+\frac{1}{s'}=-\frac{2}{r}=\frac{1}{f},\qquad V=-\frac{s'}{s}$$
    其中实像和虚像分别对应$V=-2$和$V=2$.代入$r=-40\cm$,分别得到$s=30\cm$和$s=10\cm$.因此物体放在距镜顶点30cm处成放大两倍的实像,放在距镜顶点10cm处成放大两倍的虚像.
\end{proof}

\paragraph{题2.10}以平行平面玻璃板的折射率为$n$,厚度为$h$,点光源$Q$发出的傍轴光束(即接近于正入射的光束)经上表面反射,成像于$Q_1'$;而折射先穿过上表面后在下表面反射,再从上表面折射的光束成像于$Q_2'$.证明$Q_1'$与$Q_2'$间的距离为$\frac{2h}{n}$.
\begin{proof}
    注意到$\frac{n}{s'}+\frac{1}{s}=\frac{n'-n}{r}=0$.设$Q$到玻璃板的距离为$s_1$,有$s'_1=-ns_1$.
\end{proof}

\paragraph{题2.15}某透镜用$n=1.500$的玻璃制成,它在空气中的焦距为10.0cm,它在水中的焦距为多少?(水的折射率为4/3.)

\paragraph{题2.24}$L_1,L_2$为凸透镜和凹透镜,前放一小物,移动屏幕到$L_2$后20cm的$S_1$处接到像.现将凹透镜$L_2$撤去,将屏移前5cm至$S_2$处,重新接收到像.求凹透镜$L_2$的焦距.

\paragraph{题2.40}某人对2.5m以外的物看不起,需配多少度的眼镜?另一人对1m以内的物看不清,需配怎样的眼镜?

\end{document}