\documentclass[UTF8]{article}
\title{泛函分析作业}
\author{章亦流 V21914009}
\date{}
\newcommand{\parablue}[1]{\paragraph*{\textcolor{blue}{#1}}}
\newcommand{\unsolved}{\textcolor{red}{\#Unsolved}}
\newcommand{\inspired}{\textcolor{red}{\#Inspired}}
%用ctex显示中文并用fandol主题
\usepackage[fontset=fandol]{ctex}
\setmainfont{CMU Serif} %能显示大量外文字体
\xeCJKsetup{CJKmath=true} %数学模式中可以输入中文

%AMS全家桶,\DeclareMathOperator依赖之
\usepackage{amsmath,amssymb,amsthm,amsfonts,amscd}
\usepackage{pgfplots,tikz,tikz-cd} %用来画交换图
\usepackage{bm} %粗体字母(含希腊字母)
\everymath{\displaystyle} %全体公式为行间形式

%纸张上下左右页边距
\usepackage[a4paper,left=1cm,right=1cm,top=1.5cm,bottom=1.5cm]{geometry}
%生成书签和目录上的超链接
\usepackage[colorlinks=true,linkcolor=blue,filecolor=blue,urlcolor=blue,citecolor=cyan]{hyperref}
%各种列表环境的行距
\usepackage{enumitem}
\setenumerate[1]{itemsep=0pt,partopsep=0pt,parsep=\parskip,topsep=0pt}
\setenumerate[2]{itemsep=0pt,partopsep=0pt,parsep=\parskip,topsep=0pt}
\setenumerate[3]{itemsep=0pt,partopsep=0pt,parsep=\parskip,topsep=0pt}
\setitemize[1]{itemsep=0pt,partopsep=0pt,parsep=\parskip,topsep=5pt}
\setdescription{itemsep=0pt,partopsep=0pt,parsep=\parskip,topsep=5pt}
\setlength\belowdisplayskip{2pt}
\setlength\abovedisplayskip{2pt}

%左右配对符号
\newcommand{\br}[1]{\!\left(#1\right)} %括号
\newcommand{\cbr}[1]{\left\{#1\right\}} %大括号
\newcommand{\abr}[1]{\left<#1\right>} %尖括号(内积)
\newcommand{\bbr}[1]{\left[#1\right]} %中括号
\newcommand{\abbr}[1]{\left(#1\right]} %左开右闭区间
\newcommand{\babr}[1]{\left[#1\right)} %左闭右开区间
\newcommand{\abs}[1]{\left|#1\right|} %绝对值
\newcommand{\norm}[1]{\left\|#1\right\|} %范数
\newcommand{\floor}[1]{\left\lfloor#1\right\rfloor} %下取整
\newcommand{\ceil}[1]{\left\lceil#1\right\rceil} %上取整
%常用数集简写
\newcommand{\R}{\mathbb{R}} %实数域
\newcommand{\N}{\mathbb{N}} %自然数集
\newcommand{\Z}{\mathbb{Z}} %整数集
\newcommand{\C}{\mathbb{C}} %复数域
\newcommand{\F}{\mathbb{F}} %一般数域
\newcommand{\kfield}{\Bbbk} %域
\newcommand{\K}{\mathbb{K}} %域
\newcommand{\Q}{\mathbb{Q}} %有理数域
\newcommand{\Pprime}{\mathbb{P}} %全体素数,或概率
%范畴记号
\newcommand{\Ccat}{\mathsf{C}}
\newcommand{\Grp}{\mathsf{Grp}} %群范畴
\newcommand{\Ab}{\mathsf{Ab}} %交换群范畴
\newcommand{\Ring}{\mathsf{Ring}} %(含幺)环范畴
\newcommand{\Set}{\mathsf{Set}} %集合范畴
\newcommand{\Mod}{\mathsf{Mod}} %模范畴
\newcommand{\Vect}{\mathsf{Vect}} %向量空间范畴
\newcommand{\Alg}{\mathsf{Alg}} %代数范畴
\newcommand{\Comm}{\mathsf{Comm}} %交换
%代数集合
\DeclareMathOperator{\Hom}{Hom} %同态
\DeclareMathOperator{\End}{End} %自同态
\DeclareMathOperator{\Iso}{Iso} %同构
\DeclareMathOperator{\Aut}{Aut} %自同构
\DeclareMathOperator{\Inn}{Inn} %内自同构
% \DeclareMathOperator{\inv}{Inv}
\DeclareMathOperator{\GL}{GL} %一般线性群
\DeclareMathOperator{\SL}{SL} %特殊线性群
\DeclareMathOperator{\GF}{GF} %Galois域
%正体符号
\renewcommand{\i}{\mathrm{i}} %本产生无点i
\newcommand{\id}{\mathrm{id}} %恒等映射
\newcommand{\e}{\mathrm{e}} %自然常数e
\renewcommand{\d}{\mathrm{d}} %微分符号,本产生重音符号
\newcommand{\D}{\partial} %偏导符号
\newcommand{\diff}[2]{\frac{\d #1}{\d #2}}
\newcommand{\Diff}[2]{\frac{\D #1}{\D #2}}
%运算符(分析)
\DeclareMathOperator{\Arg}{Arg} %辐角
\DeclareMathOperator{\re}{Re} %实部
\DeclareMathOperator{\im}{im} %像,虚部
\DeclareMathOperator{\grad}{grad} %梯度
\DeclareMathOperator{\lcm}{lcm} %最小公倍数
\DeclareMathOperator{\sgn}{sgn} %符号函数
\DeclareMathOperator{\conv}{conv} %凸包
\DeclareMathOperator{\supp}{supp} %支撑
\DeclareMathOperator{\Log}{Log} %广义对数函数
\DeclareMathOperator{\card}{card} %集合的势
\DeclareMathOperator{\Res}{Res} %留数
%运算符(代数,几何,数论)
\newcommand{\Span}{\mathrm{span}} %张成空间
\DeclareMathOperator{\tr}{tr} %迹
\DeclareMathOperator{\rank}{rank} %秩
\DeclareMathOperator{\charfield}{char} %域的特征
\DeclareMathOperator{\codim}{codim} %余维度
\DeclareMathOperator{\coim}{coim} %余维度
\DeclareMathOperator{\coker}{coker} %余维度
\DeclareMathOperator{\Spec}{Spec} %留数
\newcommand{\Obj}{\mathrm{Obj}} %对象类
\newcommand{\Mor}{\mathrm{Mor}} %态射类
\newcommand{\Cen}{C} %群/环的中心 或记\mathrm{Cen}
\newcommand{\opcat}{^{\mathrm{op}}}
%简写
\newcommand{\hyphen}{\textrm{-}}
\newcommand{\ds}{\displaystyle} %行间公式形式
\newcommand{\ve}{\varepsilon} %手写体ε
\newcommand{\rev}{^{-1}\!} %逆
\newcommand{\T}{^{\mathsf{T}}} %转置
\renewcommand{\H}{^{\mathsf{H}}} %共轭转置
\newcommand{\adj}{^\lor} %伴随
\newcommand{\dual}{^\vee} %对偶
\DeclareMathOperator{\lhs}{LHS}
\DeclareMathOperator{\rhs}{RHS}
\newcommand{\hint}[1]{{\small (#1)}} %提示
\newcommand{\why}{\textcolor{red}{(Why?)}}
\newcommand{\tbc}{\textcolor{red}{(To be continued...)}} %未完待续

%定理环境(随笔记形式更改)
\newtheorem{definition}{定义}
\newtheorem{remark}{注}
\newtheorem{example}{例}
\makeatletter
\@ifclassloaded{article}{
    \newtheorem{theorem}{定理}[section]
}{
    \newtheorem{theorem}{定理}[chapter]
}
\makeatother
\newtheorem{lemma}[theorem]{引理}
\newtheorem{proposition}[theorem]{命题}
\newtheorem{corollary}[theorem]{推论}
\newtheorem{property}[theorem]{性质}

\begin{document}
\maketitle
{\small \textcolor{blue}{题号为黑色的题为丁超老师所布置的,题号为蓝色的为本人自行添加的.证明中蓝色部分通常为附注,红色部分为尚未解决的部分.课本如有错漏部分将直接在题干上更改,不作另行标注.}}
\tableofcontents

\section{第一章}
\paragraph*{习题1(例1.1.25)}设$(E,d)$是一个度量空间,$F\subset E$,$d_F$是距离$d$在$F$上的限制,那么当$E$和$F$分别赋予距离$d$和$d_F$诱导的拓扑,则$F$是$E$的拓扑子空间.
\begin{proof}
    首先有$$\forall x\in F,\qquad F\cap \cbr{y:d(x,y)<\delta}=\cbr{y:d(x,y)<\delta\land y\in F}=\cbr{y\in F:d_F(x,y)<\delta}$$
    故$$\forall \text{开集}U\subset F:U=\bigcup_{i\in I}\cbr{y\in F:d_F(x_i,y)<\delta_i}=\bigcup_{i\in I}F\cap \cbr{y:d(x_i,y)<\delta_i}=F\cap \bigcup_{i\in I}B(x_i,\delta_i)$$后者是$E$中开集,即$F$中开集为$E$中开集与$F$的交,得证,
\end{proof}

\paragraph*{习题2(注1.2.7)}用连续映射定义证明:$f$在点$x$连续且$x_n\to x$,则$f(x_n)\to f(x)$.
\begin{proof}
    首先有$$\forall O(f(x))\in N(f(x))\exists O(x)\in N(x):f(O(x))\subset O(f(x)),\qquad \forall O(x)\exists N\forall n\geq N:x_n\in O(x)$$
    因此$$\forall O(f(x))\exists O(x)\exists N\forall n\geq N:f(x_n)\in f(O(x))\subset O(f(x))$$
    即$f(x_n)\to f(x)$,得证.
\end{proof}

\paragraph*{习题3(注1.2.10)}$E$上有两拓扑$\tau,\tau'$.$\tau$是$\tau'$的强拓扑$\iff \id_E:(E,\tau)\to (E,\tau'), x\mapsto x$连续.
\begin{proof}
    若$\id_E$连续,则$\forall U\in \tau':f\rev(U)=U\in \tau$.此即强拓扑的定义,得证.
\end{proof}

\paragraph*{1.3}设$E$是$\R^*=\R-\cbr{0}$和另外两个不同的点构成的并集,如$E=\R^*\cup\cbr{-\infty,+\infty}$.并设$\tau$是$E$中满足如下条件的子集$U$构成的集族:(i)在$\R^*$中的拓扑下,$U\cap \R^*$开于$\R^*$;(ii)若$-\infty\in U\lor +\infty\in U$,则$U$包含一个形如$\R^*\cap V$的集合,其中$V$是$\R$中零点的一个邻域.\\
证明:1.$\tau$是$E$上的拓扑;2.$\tau$不是Hausdorff空间;3.$\forall a\in E$的所有邻域的交集为$\cbr{a}$.
\begin{proof}
    1.即证(1)对$\tau$中任意个元素$\cbr{U_i}_{i\in I}$有$\R^*\cap\bigcup_{i\in I}U_i=\bigcup_{i\in I}\R^*\cap U$开于$\R^*$.若$-\infty\in U\lor +\infty\in U$,则$$U_i\supset \R^*\cap V\implies \bigcup_{i\in I}U_i\supset \bigcup_{i\in I}\R^*\cap V_i=\R^*\cap\bigcup_{i\in I}V_i$$
    后者依然是0的一个邻域,故$\bigcup_{i\in I}U_i\in \tau$.\\
    (2)对$\tau$中有限元素$\cbr{U_i}_{i\in [n]}$同理,故$\bigcap_{i\in [n]}U_i\in \tau$.\\
    (3)$\emptyset$开于$\R^*$且$\pm\infty\notin\emptyset$,故$\emptyset\in \tau$;$E\cap \R^*=\R^*$开于$\R^*$,$E\supset \R^*\supset \R^*\cap V$,故$E\in \tau$.综上,$\tau$是一个拓扑.

    2.考虑$E=\R^*\cup\cbr{-\infty,+\infty}$,则$$\forall O(+\infty),O(-\infty)\exists V_-,V_+\in N_\R(0):\R^*\cap V_-\subset O(a),\R^*\cap V_+\subset O(-\infty)\implies O(a)\cap O(-\infty)=\R^*\cap V_-\cap V_+\neq \emptyset$$
    故$(E,\tau)$不是Hausdorff空间.

    3.$\forall x\in \R^*: \bigcap_{O(x)\in N_{\R^*}(x)}O(x)=\cbr{x}$,否则$\forall O(x)\in N_{\R^*}\exists x'\in O(x)$,但$x'\notin B(x,|x'-x|)$.\\
    而$\forall a\in E-\R^*\cup \pm\infty: a\in\bigcap_{O(a)\in N(a)} O(a)\subset\cbr{a}\cup \br{\bigcap_{U\in \tau}U\cap \R^*}=\cbr{a}$.\\
    最后$\forall a=\pm\infty:\bigcap_{O(a)\in N(a)} O(a)=\cbr{a}\cup \br{\bigcap_{V\in N_\R(0)}\cap \R^*}=\cbr{a}\cup \br{\cbr{0}\cap \R^*}=\cbr{a}$.
\end{proof}

\paragraph*{1.4}证明紧空间中的任意序列有粘着点.
\begin{proof}
    序列$\cbr{x_n}$的粘着点$x$定义为$\forall O(x)\in N(x)\forall N\in\N\exists n\geq N:x_n\in O(x)$.\\
    有引理(课本定理1.2.4):对序列$\cbr{x_n}$定义$A_n=\cbr{x_m:m\geq n}$,$\cbr{x_n}$的粘着点全体为$\bigcap_{n\in \N}\overline{A_n}$.

    由空间是紧的,故取闭集族$\cbr{\overline{A_n}}$,对任意一个有限指标集$J\subset \N$有$\bigcap_{i\in J}\overline{A_i}\neq \emptyset$,这是因为$$\forall n\geq \max_{j\in J} j:x_n\in \bigcap_{i\in J}A_i\subset \bigcap_{i\in J}\overline{A_i}$$
    因此$\bigcap_{i\in \N}\overline{A_i}$非空,即存在粘着点.

    下证引理.$$x\in \overline{A_n}\iff \forall O(x)\exists x_m\in A_n:x_m\in O(x)$$
    因此$$x\in \bigcap_{n\in \N}\overline{A_n}\iff \forall O(x)\forall n\in \N\exists x_m\in A_n:x_m\in O(x)$$
    而$x_m\in A_n\iff m\geq n$,故得证.
\end{proof}

\newpage
\section{第二章}
\parablue{2.1}1.设函数$ \phi(x)=\frac{x}{1+|x|},x\in \R$并定义$d:(x,y)\mapsto \abs{\phi(x)-\phi(y)}$.证明由此定义的$d$是$\R$上距离并与$\R$上通常的拓扑一致,但$d$不完备.\\
2.更一般地,设$O\subsetneq E$为完备度量空间$(E,d)$上的开子集,定义$$\phi:O\to E\times \R, x\mapsto \br{x,\frac{1}{d(x,O^c)}}:=(x,\rho(x)),\qquad \forall x\in O$$
证明$\phi$是从$O$到$E\times \R$上一个闭子集的同胚,并由此导出$O$上存在一个完备的距离,由其诱导的拓扑和$d$在$O$中诱导的拓扑一致.
\begin{proof}
    1.首先证明$d(\cdot,\cdot)$是一个度量.其非负性与对称性显然,正定性由$$d(x,y)=\abs{\phi(x)-\phi(y)}=0\iff \phi(x)=\phi(y)\iff x=y$$得到.最后由
    $$d(x,z)=\abs{\phi(x)-\phi(z)}\leq \abs{\phi(x)-\phi(y)}+\abs{\phi(y)-\phi(z)}=d(x,y)+d(y,z)$$可知三角不等式成立.

    其次取$ B_d(x,\delta)=\cbr{y:\abs{\phi(x)-\phi(y)}<\delta}, B(x,\ve)=\cbr{y:\abs{x-y}<\ve}$.由$\phi(x)$严格单调可知其有反函数$\phi\rev(x)$,$$y\in B_d(x,\delta)\iff \abs{\phi(x)-\phi(y)}<\delta\iff y\in \br{\phi\rev\br{\phi(x)-\delta},\phi\rev\br{\phi(x)+\delta}}$$
    取$m_x,M_x$为$x-\phi\rev\br{\phi(x)-\delta}$和$\phi\rev\br{\phi(x)+\delta}-x$的较小值和较大值,则有$$\forall x\in \R \forall \delta>0 \exists m_x,M_x: B(x,m_x)\subset B_d(x,\delta)\subset B(x,M_x)$$
    而$\cbr{B(x,\delta):x\in \R,\delta>0}$是$\R$上的一个拓扑基,因此$\cbr{B_d(x,\delta):x\in \R,\delta>0}$也是,故其生成同一个拓扑.

    最后,取$a_n=n$,则$$\forall \ve>0\exists N=\floor{1/\ve}\forall n\geq N\forall p\in \N:\abs{\phi(a_n)-\phi(a_{n+p})}=\frac{n+p}{1+n+p}-\frac{n}{1+n}<1-\frac{n}{1+n}=\dfrac{1}{n+1}<\ve$$
    而$a_n\to +\infty$不收敛于$\R$中.
\end{proof}

\paragraph*{2.2} $(E,d)$完备$\iff \forall\cbr{x_n}\forall n\in \N:d(x_n,x_{n+1})\leq 2^{-n}$则$\cbr{x_n}$收敛.
\begin{proof}
    $\implies:\cbr{x_n}$是Cauchy列,因为$$\forall \ve\in (0,1) \exists N=\ceil{-\log_2 \ve}\forall n,m\geq N: d(x_n,x_m)\leq \sum_{k=n}^m 2^{-k}< 2^{1-n}<2^{-N}<\ve$$
    $\impliedby:$取$(E,d)$中Cauchy列$\cbr{x_n},\forall \ve=2^{-k}\exists N_k\forall n,m\geq N_k:d(x_n,x_m)<2^{-k}$,取子列使得$$n_1\leq N_1\leq n_2\leq \cdots \leq n_k\leq N_k\leq n_{k+1}\leq \cdots$$
    有$\forall k\in \N:d(x_{n_k},x_{n_{k+1}})<2^{-k}$.其收敛,即Cauchy列的收敛子列,故Cauchy列收敛于同极限,故空间完备.
\end{proof}

\paragraph*{2.3}度量空间$(E,d)$中有Cauchy列$\cbr{x_n}$.$A\subset E, \overline{A}$完备,$d(x_n,A)\to 0$,求证$x_n$在$E$中收敛.
\begin{proof}
    取$\cbr{x_i}$的子列$\cbr{y_i},y_i=x_{n_i}$,使得$d(y_n,A)<1/n$.取点列$\cbr{a_n}$使得$a_n\in B(y_n,1/n)\cap A$.这是Cauchy列,因为
    $$\forall \ve \exists N\forall n,m\geq N:d(y_n,y_m)<\ve$$
    $$\forall \ve\exists N'=\max\cbr{\ceil{\frac{2}{\ve}},N}\forall n,m\geq N': d(a_n,a_m)<\frac{1}{n}+\frac{1}{m}+d(y_n,y_m)<\frac{2}{N'}+\ve<2\ve$$
    故$a_n$收敛,设极限为$a\in \overline{A}$.而$d(a_n,y_n)\to 0$故$y_n\to a$.由Cauchy列的有收敛子列则其收敛,故得证.
\end{proof}

\paragraph*{2.4}$(E,d)$是度量空间,$A\subset E,\alpha>0.\forall x,y\in A:x\neq y\implies d(x,y)\geq \alpha$.求证$A$完备.
\begin{proof}
    $A$中任意Cauchy列在某项后必为同一元素.否则,在每项后都有不同的元素,即$$\forall N\exists n,m\geq N:x_n\neq x_m\implies d(x_n,x_m)>\frac{\alpha}{2}$$与Cauchy列定义矛盾.

    而这样的序列必然收敛,故$A$中任意Cauchy列收敛,即得证.
\end{proof}

\paragraph*{2.5}$(E,d)$是度量空间,$A\subset E$.若$A$中任意Cauchy列收敛于$E$,则$\overline{A}$完备.
\begin{proof}
    $A$中收敛列收敛于$\overline{A}$(由闭包性质),故$A$中任意Cauchy列收敛于$\overline{A}$.考虑$\overline{A}$中的Cauchy列$\cbr{x_n}$,则可以构造$A$中Cauchy列$\cbr{y_n}$:$$y_n=\begin{cases}
        x_n&x_n\in A\\ x'_n&x_n\in \overline{A}, d(x_n,x'_n)<1/n, x'_n\in A
    \end{cases}$$
    有$d(x_n,y_n)\to 0$且$y_n$收敛(证明方法同2.3),故$x_n$收敛.因此$\overline{A}$完备.
\end{proof}

\paragraph*{2.6}$(E,d)$是度量空间,$\cbr{x_n}$是$E$中发散Cauchy列.\\
求证(1)$\forall x\in E:d(x,x_n)$收敛于一个正数,记为$g(x)$;(2)$x\mapsto 1/g(x)$连续;(3)$1/g$无界.
\begin{proof}
    1.仅需证$\cbr{d(x,x_n)}$是Cauchy列.$$\forall \ve \exists N\forall n,m\geq N: d(x_n,x_m)<\ve\quad \forall \ve \exists N\forall n,m\geq N: \abs{d(x,x_n)-d(x,x_m)}\leq d(x_n,x_m)<\ve$$
    而$d(x,x_n)>0\implies g(x)\geq 0$,而$g(x)=0\iff d(x,x_n)\to 0\iff x_n\to x$,这是不可能的.

    2.仅需证$g$连续.由$$\abs{g(x)-g(y)}=\abs{\lim_{n\to \infty}d(x,x_n)-d(y,x_n)}\leq \lim_{n\to\infty}\abs{d(x,y)}=d(x,y)$$故$$\forall \ve\exists \delta=\ve\forall y:d(x,y)<\delta\implies \abs{g(x)-g(y)}<d(x,y)<\ve$$因此$g$连续,$1/g$连续.

    3.取$\cbr{g(x_n)}$,有$ \lim_{n\to \infty}g(x_n)=\lim_{n,m\to\infty}d(x_n,x_m)=0$,故$g(x_n)\to 0$,相应的$1/g(x_n)\to +\infty$.
\end{proof}

\paragraph*{2.7}$(E,d_E),(F,d_F)$是度量空间,$f:E\to F$和$f\rev$均为一致连续双射,求证$\forall A\subset E:A$完备$\iff f(A)$完备.
\begin{proof}
    仅需证$\implies$,反方向仅需考虑$B=f(A)$完备$\implies f\rev(B)=A$完备即可.即证任意Cauchy列$\cbr{y_n}\subset f(A)$收敛于$f(A)$中.考虑$\cbr{x_n}\subset A$,其中$x_n=f\rev(y_n)\in A$.首先证明这是一个Cauchy列:
    
    由$\cbr{y_n}$是Cauchy列,有$$\forall \ve\exists N_\ve\forall n,m\geq N_\ve:\abs{y_n-y_m}<\ve$$
    由$f\rev$是一致连续函数,因此有
    $$\forall \ve\exists\delta\exists N_\delta\forall n,m\geq N_\delta:\abs{y_n-y_m}<\delta\implies \abs{f\rev(y_n)-f\rev(y_m)}=\abs{x_n-x_m}<\ve$$
    即$$\forall \ve\exists N=N_\delta\forall n,m\geq N:\abs{x_n-x_m}<\ve$$
    因此$\cbr{x_n}$是Cauchy列.而由$A$的完备性,$x_n\to x\in A$,下证$y_n\to y=f(x)\in f(A)$.
    
    同上,取$N_\ve$为使$\forall n\geq N_\ve:\abs{x_n-x}<\ve$的数,由$f$的连续性有:
    $$\forall \ve\exists \delta \exists N_\delta \forall n\geq N_\delta:\abs{x_n-x}<\delta\implies \abs{y_n-y}<\ve$$
    因此$\cbr{y_n}$收敛于$f(A)$中,故得证.
\end{proof}

\paragraph*{2.8}$f:\R^n\to \R$一致连续,证明$\exists a,b\geq 0:\abs{f(x)}\leq a\norm{x}+b$.其中$\norm{x}$为$x$的Euclidean范数.
\begin{proof}
    首先取$\ve=1,\forall x,y\in \R^n \exists \delta:d(x,y)<\delta\implies \abs{f(x)-f(y)}<1$.再考虑0与点$x$的连线上有点$a_x$满足$$\norm{x}=\frac{\delta}{2}n_x+\norm{x-a_x}, t_x=\norm{x-a_x}\in \left[0,\frac{\delta}{2}\right),\qquad n_x=\frac{2}{\delta}(\norm{x}-t_x)\leq \frac{2\norm{x}}{\delta}$$
    而$$t_x=\norm{x-a_x}<\delta\implies \abs{f(x)-f(a_x)}<1$$
    对$a_x$与0间$n_x$段距离$<\dfrac{\delta}{2}$的线段应用这一不等式,因此有$$|f(x)|\leq \abs{f(a_x)-f(0)}+\abs{f(x)-f(a_x)}+\abs{f(0)}\leq n_x+1+|f(0)|\leq \frac{2}{\delta}\norm{x}+1+\abs{f(0)}$$
    取$ a=\frac{2}{\delta},b=1+\abs{f(0)}$即可.
\end{proof}

\paragraph*{2.9}$f:E\to F$是两度量空间间的连续映射,且$f$在$E$的每个有界子集上一致连续.\\
1.证明$f$将$E$中Cauchy列映为$F$中Cauchy列;\\
2.设$E$在度量空间$\tilde{E}$中稠密且$F$完备,证明$f$可唯一延拓为连续映射$\tilde{f}:\tilde{E}\to F$.
\begin{proof}
    1.考虑$E$中Cauchy列$\cbr{x_n}$,则首先考虑$\forall \ve \exists N\forall n,\geq N:d_E(x_n,x_m)<\ve$,则$ \cbr{x_n}\subset B_E(x_N,\ve)\cup\bigcup_{k\in [N]}\cbr{x_k}$,而后者有界,故$f$在其上一致连续.

    记$\cbr{y_n}\subset F, y_n=f(x_n)$.有$$\forall \ve \exists \delta \exists N_\delta \forall n,m\geq N_\delta:d_E(x_n,x_m)<\delta\implies d_F(y_n,y_m)<\ve$$
    故$\cbr{y_n}$也是Cauchy列.

    2.考虑$\tilde{f}(x)=\begin{cases}
        f(x)&x\in E\\ \lim f(x_n)&x\in \tilde{E}-E, x_n\to x
    \end{cases}.$由于$\forall x\in \tilde{E}-E$,一定有收敛列$\cbr{x_n}$使$x_n\to x$,而收敛列是Cauchy列.由1.的结论,$\cbr{f(x_n)}$也是,故有极限,即$\tilde{f}$在$\tilde{E}-E$上有定义.

    首先考虑这一定义是否良定,即$\lim x_n=\lim x'_n=x\implies \lim f(x_n)=\lim f(x'_n)=f(x)$.由$d_E(x_n,x'_n)\to 0$和$f$的一致连续性,有$$\forall\ve \exists \delta\exists N_\delta\forall n\geq N_\delta:d_E(x_n,x'_n)<\delta\implies d_F(f(x_n),f(x'_n))<\ve$$
    故$d_F(f(x_n),f(x'_n))\to 0$,而$d(\cdot,\cdot)$连续,故$\lim d_F(f(x_n),f(x'_n))=d_F\br{\lim f(x_n),\lim f(x'_n)}=0$,故极限相同,即良定.

    再证明其连续.考虑$$\begin{gathered}
        \forall x,x'\in \tilde{E}\exists \cbr{x_n},\cbr{x'_n}\subset E\exists N_\ve\forall n\geq N_\ve:d_E(x_n,x)<\ve\land d_E(x',x'_n)<\ve\\
        \forall \ve \exists \delta:d_E(x_n,x'_n)<\delta\implies d_F(f(x_n),f(x'_n))<\ve
    \end{gathered}$$
    因此
    $$\begin{aligned}
        &\forall x\in \tilde{E}\forall \ve\exists \delta\forall x'\in B_{\tilde{E}}\br{x,\frac{\delta}{3}}\exists\cbr{x_n},\cbr{x'_n}\subset E\exists N_{\delta/3}\forall n\geq N_{\delta/3}:\\
        &d_E(x_n,x'_n)\leq d_E(x_n,x)+d_E(x,x')+d_E(x',x'_n)\leq \delta\implies d_F(f(x_n),f(x'_n))<\ve
    \end{aligned}$$
    由$d(\cdot,\cdot)$连续,则取$n\to +\infty, \tilde{f}(x)=\lim f(x_n), \tilde{f}(x')=\lim f(x'_n)$,有$$d_F(\tilde{f}(x),\tilde{f}(x'))=\lim d_F(f(x_n),f(x'_n))\leq \ve$$
    即$$\forall x\in \tilde{E}\forall\ve \exists \delta\forall x'\in \tilde{E}:d_{\tilde{E}}(x,x')<\frac{\delta}{3}\implies d_F(\tilde{f}(x),\tilde{f}(x'))\leq \ve<2\ve$$
    故连续性得证.最后证明其唯一性.假设连续映射$\tilde{f}':\tilde{E}\to F$,且同样有$\tilde{f}'|_E=\tilde{f}|_E=f$,则由连续性有
    $$\forall x\in\tilde{E}\exists \cbr{x_n}\subset E:x_n\to x,\qquad \tilde{f}'(x)=\lim \tilde{f}'(x_n)=\lim f(x_n)=\tilde{f}(x)$$
    故得证.
\end{proof}

\paragraph*{2.10}构造反例说明,在不动点定理中,若减弱$f$条件为$$d\br{f(x),f(y)}<d(x,y),\qquad \forall x,y\in E\land x\neq y$$则结论不成立.(HINT:$f(x)=\sqrt{x^2+1},x\in [0,+\infty)$.)
\begin{proof}
    由HINT,考虑在$\R$上$\forall y>x\geq 0$:
        $$\sqrt{y^2+1}-\sqrt{x^2+1}<y-x\iff \sqrt{y^2+1}-y<\sqrt{x^2+1}-x$$
        即证$g(x)=\sqrt{x^2+1}-x$严格单调递减.而$g'(x)=\dfrac{x}{\sqrt{x^2+1}}-1<1-1=0$,故$f$满足$d\br{f(x),f(y)}<d(x,y)$,但$f(x)=x$无解.
\end{proof}

\paragraph*{2.11}完备度量空间$(E,d)$上映射$f$满足$f^n=\underbrace{f\circ \cdots\circ f}_{n\text{次}}$为压缩映射($n$为常数).证明$f$有唯一不动点,并举一$f$不连续的例子.
\begin{proof}
    设$a$为$f^n$的不动点,$f(a)=f(f^n(a))=f^{n+1}(a)=f^n(f(a))$,而$f^n$仅有唯一不动点$a$,故$f(a)=a$,$a$是$f$的一个不动点.若有另一个不动点$a'$,则$f^n(a')=a'$,而$f^n$仅有唯一不动点$a$,则$a=a'$.

    $f(x)=\begin{cases}
        0&x\in [0,1]\\1&x\in (1,2]
    \end{cases}$,则$f^n=0(n\geq 2)$,其有唯一不动点$x=0$,但$f$不连续.
\end{proof}

\paragraph*{2.12}记$I=(0,+\infty)$上通常拓扑为$\tau$.\\ 1.证明$\tau$可被完备距离$d:(x,y)\mapsto \abs{\ln x-\ln y}$诱导;\\
2.设$f\in C^1(I)$满足$\exists \lambda<1\forall x\in I:x\abs{f'(x)}\leq \lambda f(x)$,证明$f$在$I$上有唯一不动点.
\begin{proof}
    1.容易验证$d(\cdot,\cdot)$是一个距离.设$B(x,\ve)=\cbr{y>0:|x-y|<\ve},B_d(x,\ve)=\cbr{y>0:\abs{\ln x-\ln y}<\ve}$.而$$y\in B_d(x,\delta)\iff \abs{\ln x-\ln y}<\delta\implies y\in \br{\e^{\ln x-\delta},\e^{\ln x+\delta}}=\br{\e^{-\delta}x,\e^\delta x}$$
    而$\e^\delta-x\geq x-\e^{-\delta}$(由$(\e^\delta-1)^2\geq 0$),故$B(x,\e^{-\delta} x)\subset B_d(x,\delta)\subset B(x,\e^\delta x)$,因此它们诱导同一度量拓扑,下证$d(\cdot,\cdot)$是完备的距离.考虑$\cbr{x_n}$是$d$下的Cauchy列,即$$\forall \ve\exists N\forall n,m\geq N:\abs{\ln x_n-\ln x_m}<\ve,x_m\in B_d(x_n,\ve)\subset B(x_n,\e^\ve x_n)$$
    即$$\forall \ve\exists N\forall n,m\geq N:|x_n-x_m|<\e^\ve x_n\leq M\e^\ve,\qquad M=\sup_{n\in \N}x_n\in B_d(x_N,2\ve)<+\infty$$
    因此$\cbr{x_n}$是Cauchy列,其收敛,故$d$完备.

    2.$$\forall x,y\in I:\frac{d(f(x),f(y))}{d(x,y)}=\abs{\frac{\ln f(x)-\ln f(y)}{\ln x-\ln y}}=\abs{\frac{f'(\xi)/f(\xi)}{1/\xi}}=\frac{\xi\abs{f'(\xi)}}{f(\xi)}\leq \lambda$$
    故$f$是压缩映射,故在$I$上有唯一不动点.
\end{proof}

\paragraph*{2.13}对可数集$E=\cbr{a_1,a_2,\cdots}$定义$ d(a_p,a_q)=\begin{cases}
    0&p=q,\\ 10+\frac{1}{p}+\frac{1}{q}&p\neq q
\end{cases}$.\\
1.证明$d$是$E$上距离,且$(E,d)$为完备度量空间;\\
2.$f:E\to E, a_p\mapsto a_{p+1}$,证明$p\neq q$时$d\br{f(a_p),f(a_q)}<d(a_p,a_q)$,但$f$无不动点.
\begin{proof}
    1.显然$d$非负,正定,对称,下证三角不等式:$$d(a_p,a_r)=10+\frac{1}{p}+\frac{1}{q}\leq \br{10+\frac{1}{p}+\frac{1}{q}}+\br{10+\frac{1}{q}+\frac{1}{r}}=d(a_p,a_q)+d(a_q,a_r)$$
    再证其完备.$\forall a_p,a_q\in E:a_p\neq a_q\implies d(a_p,a_q)>10$,根据2.4,$(E,d)$完备.

    2.有$$d(f(a_p),f(a_q))=d(a_{p+1},a_{q+1})=10+\frac{1}{p+1}+\frac{1}{q+1}<10+\frac{1}{p}+\frac{1}{q}=d(a_p,a_q)$$
    而$f(a_p)=a_{p+1}\neq a_p$,即无不动点.
\end{proof}

\parablue{2.14}本题是为了给出压缩映射原理的另一个证明,故默认不使用定理的结论.\\ 设$(E,d)$为非空完备度量空间,$f:E\to E$为压缩映射.$\forall R\geq 0:A_R:=\cbr{x\in E:d(x,f(x))\leq R}$.\\
证明:1.$f(A_R)\subset A_{\lambda R}$;\\
2.$R>0$时$A_R$是$E$中非空闭子集;\\
3.$\forall x,y\in A_R:d(x,y)\leq 2R+d(f(x),f(y))$,并导出$\mathrm{diam}(A_R)\leq \dfrac{2R}{1-\lambda}$;\\
4.$A_0$非空.
\begin{proof}
    1.即$x\in A_R\implies f(x)\in A_{\lambda R}$.$d(x,f(x))\leq R\implies d(f(x),f^2(x))\leq \lambda d(x,f(x))<\lambda R$,其中$\lambda$为$f$的Lipschitz常数,$\lambda<1$.

    2.即证:对$A_R$中任意收敛列$\cbr{x_n}$有$x_n\to x,d(x,f(x))\leq R$.由$$\forall \ve\exists N\forall n\geq N:d(x_n,x)<\ve, d(f(x_n),f(x))<\lambda \ve$$
    因此$d(x,f(x))\leq d(x,x_n)+d(x_n,f(x_n))+d(f(x_n),f(x))<(1+\lambda)\ve+R$.取$\ve\to 0$,有$d(x,f(x))\leq R$.

    3.$d(x,y)\leq d(x,f(x))+d(f(x),f(y))+d(f(y),y)=2R+d(f(x),f(y))\leq 2R+\lambda d(x,y)\implies d(x,y)\leq \dfrac{2R}{1-\lambda}$,因此$\mathrm{diam}(A_R)=\sup_{x,y\in A_R}d(x,y)\leq \dfrac{2R}{1-\lambda}$.

    4.首先显然$R\geq r\implies A_R\supset A_r$.而$R\to 0,\mathrm{diam}(A_R)\to 0$.因此由闭集套定理,$\bigcap_{n\in \N}A_{\frac{1}{n}}$非空且仅有一个元素$a$,其$\forall n\in \N:d(a,f(a))\leq n\rev$,故$d(a,f(a))=0,a\in A_0$.
\end{proof}

\parablue{2.15}设$(E,d)$为完备度量空间,$f,g$为$E$上可交换的压缩映射.证明$f,g$有唯一且共同的不动点.并通过反例说明,去掉可交换条件则结论不成立.
\begin{proof}
    首先$f,g$均有唯一不动点,记为$a,b$.而$g(a)=g(f(a))=f(g(a))\implies g(a)=a\implies a=b$,故其不动点相同.不交换的反例如$\R$上$f:x\mapsto 2$与$g:x\mapsto 3$,$f\circ g=f\neq g=g\circ f$,其不动点分别为2和3.
\end{proof}

\parablue{2.16}设$(E,d)$为完备度量空间,定义$A\subset E$的距离函数$d_A(x):=d(x,A)$,并设$\mathcal{C}$为$E$的所有紧子集构成的集族,且定义$ \forall A,B\in \mathcal{C}:h(A,B)=\sup_{x\in E}\abs{d_A(x)-d_B(x)}$.\\
1.证明$h$为$\mathcal{C}$上的一个距离;\\
2.$\forall F\subset E:F_\ve:=\cbr{x:d_F(x)\leq \ve}$.证明$h(A,B)=\inf\cbr{\ve\geq 0:A\subset B_\ve,B\subset A_\ve}$;\\
3.证明$(\mathcal{C},h)$完备;\\
4.取$E$上$n$个压缩映射$\cbr{f_i}_{i=1}^n$,定义$(\mathcal{C},h)$上映射$$T:A\mapsto \bigcup_{k=1}^n f_k(A),\qquad A\in \mathcal{C}$$证明$T$是压缩映射,并由此导出存在唯一的紧子集$K$使$T(K)=K$.
\begin{proof}
    1.非负与对称性显然,正定性有$\sup_{x\in E}\abs{d_A(x)-d_B(x)}=0\implies \forall x\in E:d_A(x)=d_B(x)\implies A=B$.三角不等式有:
    $$\begin{aligned}
        h(A,C)&=\sup_{x\in E}\abs{d_A(x)-d_C(x)}\leq \sup_{x\in E}\br{\abs{d_A(x)-d_B(x)}+\abs{d_B(x)-d_C(x)}}\\
        &\leq \sup_{x\in E}\abs{d_A(x)-d_B(x)}+\sup_{x\in E}\abs{d_B(x)-d_C(x)}=h(A,B)+h(B,C)
    \end{aligned}$$
    故$h$为$\mathcal{C}$上的距离.

    2.\unsolved
    
    % 记$ h_1=\sup_{x\in E}\abs{d_A(x)-d_B(x)}, h_2=\inf\cbr{\ve\geq 0:A\subset B_\ve,B\subset A_\ve}$,一个直接的想法就是证明其互相大于等于.\\
    % 考虑$$\begin{aligned}
    %     h_1&=\max\cbr{\sup_{x\in A}\abs{d_A(x)-d_B(x)},\sup_{x\in B}\abs{d_A(x)-d_B(x)},\sup_{x\in E-A\cup B}\abs{d_A(x)-d_B(x)}}\\
    %     &=\max\cbr{\sup_{x\in A}d_B(x),\sup_{x\in B}d_A(x),\sup_{x\in E-A\cup B}\abs{d_A(x)-d_B(x)}}%=\max\cbr{a,b,c}
    % \end{aligned}$$
    % 而$\forall x\in E-A\cup B\forall \delta>0\exists a\in A\exists b\in B:d(x,A)+\delta\geq d(x,a),d(a,B)+\delta\geq d(a,b)$.因此$$d(x,B)\leq d(x,b)\leq d(x,a)+d(a,b)\leq d(x,A)+d(a,B)+2\delta\implies d(x,B)-d(x,A)\leq d(a,B)+2\delta\leq \sup_{x\in A}d(x,B)+2\delta$$
    % 故$\delta\to 0$时有$\forall x\in E-A\cup B:\abs{d_A(x)-d_B(x)}\leq \sup_{x\in A}d_B(x)$,故$\sup_{x\in E-A\cup B}\abs{d_A(x)-d_B(x)}\leq \sup_{x\in A}d_B(x)$.故$ h_1=\max\cbr{\sup_{x\in A}d_B(x),\sup_{x\in B}d_A(x)}$.

    % 另外,$\cbr{\ve\geq 0:A\subset B_\ve}=\cbr{\ve\geq 0:\forall x\in A:d(x,B)\leq \ve}$,但$\forall x\in A:d(x,B)\leq \ve\iff \ve\geq \sup_{x\in A}d(x,B)$,因此$ h_2=\min\cbr{\sup_{x\in A}d_B(x),\sup_{x\in B}d_A(x)}$.下证$ \sup_{x\in A}d_B(x)=\sup_{x\in B}d_A(x)$.\textcolor{red}{这是错的!但整个证明哪里错了?}

    % 另外,记$L^{AB}=\cbr{\ve\geq 0:A\subset B_\ve}=\cbr{\ve\geq 0:\forall x\in A:d(x,B)\leq \ve}$,类似定义$L^{BA}$.$ h_2=\min\cbr{\inf L^{AB},\inf L^{BA}}$.

    % 首先,$ h_1\geq h_2\impliedby \sup_{x\in A}d_B(x)\geq \inf L^{AB}$,由$\forall x\in A:d_B(x)\leq \sup_{x\in A}d_B(x)$有$\sup_{x\in A}d_B(x)\in L^{AB}$,故不等式成立,得证.而$ h_1\leq h_2$需要$ \sup_{x\in A}d_B(x),\sup_{x\in B}d_A(x)$分别小于等于$L^{AB},L^{BA}$.仅需证前者第一个小于等于后两个,其余同理.

    % 首先,$\br{\sup_{x\in A}d_B(x)\leq \inf L^{AB}}\iff \br{\forall \ve\in L^{AB}\forall x\in A:\ve\geq d_B(x)}$,而后者显然成立,故得证.其次,
    % $$\br{\sup_{x\in A}d_B(x)\leq \inf L^{BA}}\iff \br{\forall \ve\in L^{BA}\forall x\in A:\ve \geq d_B(x)}\iff \br{\forall \ve:\br{\forall x\in B:d_A(x)\leq \ve}\implies\br{\forall x\in A:\ve \geq d_B(x)}}$$

    % 3.考虑Cauchy列$\cbr{A_n}\subset\mathcal{C}$,$\forall \ve\exists N\forall n,m\geq N:h(A_n,A_m)<\ve$.\\
    % 首先考虑$\sup_{x\in E}\abs{d(x,A)-d(x,B)}<\ve\implies \forall x\in E:\abs{d(x,A)-d(x,B)}<\ve$.而$x\in A\implies d(x,B)<\ve$,反之依然.若$x\in E-A\cup B\implies \abs{d(x,A)-d(x,B)}=\abs{d(x,a)-d(x,b)}\leq d(a,b)\leq d(A,B)<\ve$.

    % 4.即证明$\exists \lambda<1\forall A,B\in\mathcal{C}:h(T(A),T(B))\leq \lambda h(A,B)$,而$h(T(A),T(B))=T\br{\bigcup_{k\in [n]}f_k(A),\bigcup_{k\in [n]}f_k(B)}=\max\cbr{\sup\cbr{d\br{x,\bigcup_{k\in [n]}f_k(A)}:x\in \bigcup_{k\in [n]}f_k(B)},\sup\cbr{d\br{x,\bigcup_{k\in [n]}f_k(B)}:x\in \bigcup_{k\in [n]}f_k(A)}}$
\end{proof}

\newpage
\section{第三章}
\paragraph*{3.1}$\forall f\in C[0,1]:\norm{f}_\infty=\sup_{t\in [0,1]}\abs{f(t)}, \norm{f}_1=\int_0^1\abs{f(t)}\d t$.\\
证明1.$\norm{\cdot}_\infty,\norm{\cdot}_1$都是$C[0,1]$的范数;2.$C[0,1]$关于$\norm{\cdot}_\infty$完备;3.$C[0,1]$关于$\norm{\cdot}_1$不完备.
\begin{proof}
    1.显然有正定,正齐,三角不等式.

    2.考虑Cauchy列$\cbr{f_n}$,$$\forall \ve\exists N\forall n,m\geq N:\norm{f_n-f_m}_\infty=\sup_{t\in [0,1]}\abs{f_n(t)-f_m(t)}<\ve$$
    此即一致收敛定义.考虑$f(x)=\lim_{n\to\infty}f_n(x)$,则首先$\forall x\in [0,1]:f(x)$存在.其次证明其连续性.$$\forall x\in [0,1]\forall \ve\exists \delta\exists N\forall n\geq N\forall y\in B(x,\delta):\abs{f_n(x)-f(x)}<\ve, \abs{f_n(x)-f_n(y)}<\ve, \abs{f_n(y)-f(y)}<\ve$$
    因此$\forall x\in [0,1]\forall \ve\exists \delta\forall y\in B(x,\delta):\abs{f(x)-f(y)}<3\ve$,即$f\in C[0,1]$.

    3.考虑Cauchy列$\cbr{f_n}$,$$\forall \ve\exists N\forall n,m\geq N:\norm{f_n-f_m}_1=\int_0^1\abs{f_n(t)-f_m(t)}\d t<\ve$$
    取$f_n(x)=x^{n}$,则$\int_0^1\abs{f_n(t)-f_m(t)}\d t\leq \frac{1}{1+\max\cbr{n,m}}\to 0$,但$f_n(x)\to 1_{\cbr{1}}(x)$不连续.

    \textcolor{blue}{题解认为$\cbr{x^n}$虽然逐点收敛到不连续函数,但在$L^1$范数下收敛到$f=0$,因此函数列极限在$C[0,1]$中存在.题解给出的函列为$f_n(x)=\begin{cases}
        -1&x\in \babr{0,\frac{1}{2}-\frac{1}{n}}\\
        n\br{t-\frac{1}{2}}&x\in \bbr{\frac{1}{2}-\frac{1}{n},\frac{1}{2}+\frac{1}{n}}\\
        1&\abbr{\frac{1}{2}+\frac{1}{n},1}
    \end{cases}$.}
\end{proof}

\paragraph*{3.2}$\forall P\in \R[x]:\norm{P}_\infty:=\max_{x\in [0,1]}\abs{P(x)}$.\\
1.证明$\norm{\cdot}_\infty$是$\R[x]$上范数;\\
2.$\forall a\in \R$定义$L_a:\R[x]\to \R, P\mapsto P(a)$.证明$L_a$连续$\iff a\in [0,1]$,且给出$\norm{L_a}$;\\
3.设$a<b$,定义$L_{a,b}:\R[x]\to \R, P\mapsto \int_a^b P(x)\d x$.求$A\subset \R$使$a,b\in A\iff L_{a,b}$连续,然后确定$\norm{L_{a,b}}$.
\begin{proof}
    1.首先正定性显然,正齐性由$\max_{x\in [0,1]}\abs{P(x)}=0\implies \abs{P(x)}\leq 0\implies P(x)=0$得到.$\norm{\lambda P}_\infty=\abs{\lambda}\norm{P}_\infty$显然.三角不等式由$$\norm{P+Q}_\infty=\max_{x\in [0,1]}\abs{P(x)+Q(x)}\leq \max_{x\in [0,1]}\br{\abs{P(x)}+\abs{Q(x)}}\leq \max_{x\in [0,1]}\abs{P(x)}+\max_{x\in [0,1]}\abs{Q(x)}=\norm{P}_\infty+\norm{Q}_\infty$$
    故$\norm{\cdot}_\infty$是$\R[x]$上的一个范数.

    2.容易证明$L_a$是一个线性函数,因此$L_a$连续$\iff \exists C\geq 0\forall P\in \R[x]:\abs{P(a)}\leq C\norm{P}_\infty$.\\ 考虑$P_n(x)=(2x-1)^n$,则$\exists C\geq 0\forall n\in \N:\abs{P_n(a)}=\abs{2a-1}^n\leq C\cdot \norm{P_n}_\infty=C\iff \abs{2a-1}\leq 1\iff a\in [0,1]$.\\ 另一方面,$a\in [0,1]$时必然成立$\abs{P(a)}\leq \norm{P}_\infty$,因此$L_a$连续$\iff a\in [0,1]$.

    最后,取$P=1$则$\norm{L_a}\geq 1$,而$\norm{L_a}=\sup_{P\in \R[x]-\cbr{0}}\frac{\abs{P(a)}}{\norm{P}_\infty}\leq \sup_{P\in \R[x]-\cbr{0}}\frac{\norm{P}_\infty}{\norm{P}_\infty}=1$,故$\norm{L_a}=1$.

    3.容易证明$L_{a,b}$是一个线性函数,则$L_{a,b}$连续$\iff \exists C\geq 0\forall P\in \R[x]:\abs{\int_a^b P(x)\d x}\leq C\norm{P}_\infty$.\\
    仍考虑$P_n(x)=(2x-1)^n$,则$\abs{\int_a^b P_n(x)\d x}=\frac{\abs{(2b-1)^{n+1}-(2a-1)^{n+1}}}{2(n+1)}$有界$\iff 2a-1,2b-1\in [0,1]$,即$a,b\in [0,1]$.而$a,b\in [-1,1]$时$\abs{\int_a^b P(x)\d x}\leq (b-a)\norm{P}_\infty$,故$a,b\in [0,1]\iff L_{a,b}$连续.

    最后有$$b-a=\frac{\abs{\int_a^b 1 \d x}}{1}\leq\norm{L_{a,b}}=\sup_{P\in \R[x]-\cbr{0}}\frac{\abs{\int_a^bP(x)\d x}}{\norm{P}_\infty}\leq \sup_{P\in \R[x]-\cbr{0}}\frac{\int_a^b\norm{P}_\infty\d x}{\norm{P}\infty}=b-a$$故$\norm{L_{a,b}}=b-a$.
\end{proof}

\parablue{3.3}设$(\R[x],\norm{\cdot}_\infty)$为上题定义的赋范空间,设$E_0\subset \R[x]$为全体无常数项多项式构成的向量子空间.\\
1.证明$N(P):=\norm{P'}_\infty$定义$E_0$上一个范数,且$\forall P\in E_0:\norm{P}_\infty\leq N(P)$.\\
2.证明$L(P)=\int_0^1 \frac{P(x)}{x}\d x$定义了$E_0$关于$N$的连续线性泛函,并求$\norm{N}$.\\
3.上述$L$是否关于$\norm{\cdot}_\infty$连续?\\
4.$\norm{\cdot}_\infty$和$N$在$E_0$上是否等价?

\begin{proof}
    1.首先证明$N(P)$是一个范数:$$\begin{gathered}
        N(P)\geq 0; N(P)=\norm{P'}_\infty=0\iff P'=0\land P\in E_0\iff P=0;N(\lambda P)=\norm{\lambda P'}_\infty=\abs{\lambda}\norm{P}_\infty\\
        N(P+Q)=\norm{P'+Q'}_\infty\leq \norm{P'}_\infty+\norm{Q'}_\infty=N(P)+N(Q)
    \end{gathered}$$
    而由Lagrange中值定理有$\forall P\in E_0\forall x\in [0,1]:\abs{\frac{P(x)-P(0)}{x-0}}=\abs{P'(\xi)}\leq \max_{x\in [0,1]}\abs{P'(x)}=N(P)$,故$\abs{P(x)}\leq \abs{P'(\xi)}\abs{x}\leq \abs{P'(\xi)}\leq N(P)$,故$\norm{P}_\infty\leq N(P)$.

    2.线性:$L(\lambda P+\mu Q)=\int_0^1\frac{\lambda P(x)+\mu Q(x)}{x}\d x=\lambda\int_0^1 \frac{P(x)}{x}\d x+\mu\int_0^1\frac{Q(x)}{x}\d x=\lambda L(P)+\mu L(Q)$.

    因此$L$连续$\iff \exists C\geq 0\forall P\in E_0:\abs{\int_0^1 \frac{P(x)}{x}\d x}\leq C\cdot N(P)$.而同上使用Lagrange中值定理,有$$\abs{\int_0^1 \frac{P(x)}{x}\d x}\leq \max_{x\in [0,1]}\abs{\frac{P(x)}{x}}\leq \max_{x\in [0,1]}\abs{P'(x)}=N(P)$$

    故$L$连续.最后$$1=\frac{\int_0^1 1\d x}{1}=\norm{N}=\sup_{P\in E_0-\cbr{0}}\frac{\abs{\int_0^1 \frac{P(x)}{x}\d x}}{\norm{P'}_\infty}\leq \sup_{P\in E_0-\cbr{0}}\frac{\norm{P(x)/x}_\infty}{\norm{P'}_\infty}\leq 1$$
    因此$\norm{N}=1$.

    3.考虑$P_n(x)=nx^n$,则$L(P_n)=\abs{\int_0^1 \frac{P_n(x)}{x}\d x}=1, \norm{P_n}_\infty=n$,因此$\frac{L(P)}{\norm{P}_\infty}$无界,即$L$关于$\norm{\cdot}_\infty$不连续.

    4.考虑$P_n(x)=x^n$,则$\norm{P_n}_\infty=1,N(P_n)=n$,因此不存在常数使得范数等价.
\end{proof}

\paragraph*{3.4}在$C[0,1]$上定义范数$\norm{f}_1=\int_0^1 \abs{f(x)}\d x, N(f)=\int_0^1 x\abs{f(x)}\d x$.\\
1.验证$N$是$C[0,1]$上范数且$N\leq \norm{\cdot}_1$.\\
2.$f_n(x)=\begin{cases}
    n-n^2x&x\in [0,n\rev\,]\\
    0& \text{else}
\end{cases}$.证明在$(C[0,1],N)$中$f_n\to 0$,并问:$\cbr{f_n}$在$(C[0,1],\norm{\cdot}_1)$中是否收敛?这两个范数在$C[0,1]$上诱导的拓扑是否相同?\\
3.取$a\in \abbr{0,1}$,设$B=\cbr{f\in C[0,1]:\forall x\in [0,a]:f(x)=0}$,证明这两个范数在$B$上诱导相同拓扑.

\begin{proof}
    1.首先证明$N$是一个范数:$$\begin{gathered}
        0\leq N(f)=0\iff f=0,\qquad N(\lambda f)=\lambda\int_0^1 x\abs{f(x)}\d x=\lambda N(f),\\ 
        N(f+g)=\int_0^1 x\abs{f(x)+g(x)}\d x\leq \int_0^1 x\abs{f(x)}\d x+\int_0^1 x\abs{g(x)}\d x=N(f)+N(g)
    \end{gathered}$$
    其次有$\forall f\in C[0,1]\exists \xi\in (0,1):N(f)=\int_0^1 x\abs{f(x)}\d x=\xi\int_0^1\abs{f(x)}\d x\leq \norm{f}_1$,因此$N\leq \norm{\cdot}_1$.

    2.$N(f_n)=\int_0^{1/n}x(n-n^2x)\d x=\frac{1}{6n}\to 0$,因此$\cbr{f_n}$在$N$下收敛.\\
    而$\norm{f_n-f}_1=\int_0^{1/n}\abs{n-n^2x-f(x)}\d x+\int_{1/n}^1\abs{f(x)}\d x\geq \int_{1/n}^1\abs{f(x)}\d x\to \int_0^1\abs{f(x)}\d x$.因此$\norm{f_n-f}_1\to 0\implies f=0$,但$\norm{f_n}_1=\int_0^{1/n}n-n^2x\d x=\frac{1}{2}$,因此在$\norm{\cdot}_1$下不收敛.\\
    也因此$\forall \ve>0\exists n:f_n\in B_N(0,\ve)$,但$f_n\notin B_1(0,1/2)$,故$\exists \ve>0:B_N(0,\ve)\not\subset B_1(0,1/2)$,即拓扑不等价.

    3.仅需证明两范数等价.有$\forall f\in B\exists \xi\in (a,1):N(f)=\int_a^1 x\abs{f(x)}\d x=\xi\int_a^1\abs{f(x)}\d x\in \bbr{a\norm{f}_1,\norm{f}_1}$,得证.
\end{proof}

\paragraph*{3.5}$\varphi:[0,1]\to [0,1]$连续且不恒为1.取$\alpha\in \R$,定义$T:C[0,1]\to C[0,1], f(x)\mapsto \alpha+\int_0^x f(\varphi(t))\d t$.证明\textcolor{red}{$T^2$}是压缩映射.由此证明$$f(0)=\alpha,\qquad f'(x)=f(\varphi(x)),\qquad x\in [0,1]$$有唯一解.

\begin{proof}
    首先$T^2(f)(x)=\alpha+\int_0^x\br{\alpha+\int_0^{\varphi(t)}f(\varphi(s))\d s}\d t$,$\br{T^2(f)-T^2(g)}(x)=\int_0^x\d t \int_0^{\varphi(t)}(f-g)(\varphi(s))\d s$.而$$\begin{aligned}
        \int_0^{\varphi(t)}(f-g)(\varphi(s))\d s&=(f-g)(\varphi(\xi))\varphi(t)\leq \norm{f-g}_\infty\varphi(t),\qquad \xi\in (0,\varphi(t))\\
        \norm{T^2(f)-T^2(g)}_\infty&=\max_{x\in [0,1]}\abs{\int_0^x \d t \int_0^{\varphi(t)}(f-g)(\varphi(s))\d s}\leq \max_{x\in [0,1]}\abs{\norm{f-g}_\infty\int_0^x \varphi(t)\d t}\leq \lambda\norm{f-g}_\infty
    \end{aligned}$$
    其中$\lambda=\abs{\int _0^1 \varphi(t)\d t}<1$,因此$T^2$为压缩映射.又由2.11,$T$有唯一不动点,即$f(x)=\alpha+\int_0^x f(\varphi(t))\d t$有唯一解,而此方程等价于题干中方程组.

    需要注意的是,$T$不是压缩映射,如取$f(x)=b,g(x)=a$.
\end{proof}

\paragraph*{3.6}设$\alpha\in \R, a>0,b>1$.考察微分方程$$f(0)=\alpha,\qquad f'(x)=af(x^b),\qquad x\in [0,1]$$
1.取$M>0$,验证$(C[0,1],\norm{\cdot})$是Banach空间,其中$\norm{f}=\sup_{x\in [0,1]}\abs{f(x)}\e^{-Mx}$.\\
2.定义$T:C[0,1]\to C[0,1], f\mapsto \alpha+\int_0^x af(t^b)\d t$,证明选择合适的$M$可使$T$为压缩映射.\\
3.证明此微分方程有唯一解.

\begin{proof}
    1.考虑$(C[0,1],\norm{\cdot})$中Cauchy列$\cbr{f_n}_{n\in \N}$,则$\e^{-M}\norm{f_n-f_m}_\infty\leq \norm{f_n-f_m}< \ve$,故$\forall \ve\exists N\forall n,m\geq N:\norm{f_n-f_m}<\e^M\ve$,即$\cbr{f_n}$也是$\norm{\cdot}_\infty$下的Cauchy列.因此其收敛,即$(C[0,1],\norm{\cdot})$完备.

    2.$$\begin{aligned}
        \norm{T(f)-T(g)}&=\max_{x\in [0,1]}\abs{\int_0^x a(f-g)(t^b)\d t}\e^{-Mx}=a\max_{x\in [0,1]}\abs{f(\xi^b)-g(\xi^b)}x\e^{-Mx}\\
        &\leq a\br{\max_{x\in [0,1]}\abs{f(\xi^b)-g(\xi^b)}\e^{-M\xi^b}}\br{\max_{x\in [0,1]}x\e^{M(\xi^b-x)}}\leq a\norm{f-g}\max_{x\in [0,1]}\e^{M(\xi^b-x)}
    \end{aligned}$$
    若希望使得$\norm{T(f)-T(g)}\leq \lambda\norm{f-g}$,则需$\lambda=a\max_{x\in [0,1]}\e^{M(\xi^b-x)}<1$.取$x_0=\arg\max \e^{M(\xi^b-x)}$,此时$\xi=\xi_0$,则$M>\frac{\ln a}{x_0-\xi_0^b}$时成立.

    3.由$T$在适当情况下为压缩映射,故$T(f)=f$有唯一解,而此式等价于上述微分方程,故得证.
\end{proof}

\parablue{3.7}设$E$是数域$\F$上的无限维向量空间,设$\cbr{e_i}_{i\in I}$是其中一组向量.若$E$中任一向量可被$\cbr{e_i}_{i\in I}$中有限个向量唯一线性表示,则称$\cbr{e_i}_{i\in I}$是$E$中的一组Hamel基.\\
1.由Zorn引理证明$E$中有一组Hamel基.\\
2.若$E$是赋范空间,则$E$上必存在不连续的线性泛函.\\
3.证明在任一无限维赋范空间上,一定存在一个比原来范数严格强的范数.由此证明,若向量空间$E$上任意两个范数诱导同一拓扑,则$E$必为有限维空间.

\begin{proof}
    1.考虑$E$中线性无关组全体$E'$中的偏序关系$\subset$.对$E'$中任意链$e^1\subset e^2\subset \cdots$考虑$e=\bigcup_{n\geq 1}e^n$为其一个上界.这是因为$\forall n\geq 1:e^n\subset e$,且$e$中任意有限子集$\cbr{e_j}_{j\in J}$含于某一$e^n$,故$\cbr{e_j}_{j\in J}$线性无关,故$e$线性无关,$e\subset E'$.而由Zorn引理,$E'$中存在一个极大元$\epsilon$.\\
    下证$E=\Span(\epsilon)$.若否,则$\exists v\in E$不能被$\epsilon$表出,即$\epsilon\cup \cbr{v}$线性无关.这与$\epsilon$在$E'$中的极大性矛盾.\\
    最后证明$\epsilon$是$E$的Hamel基.若有$v\in E$不能被$\epsilon$中有限向量线性表示,则考虑$\epsilon'=\epsilon\cup \cbr{v}$,其中有限子集若不含$v$则含于$\epsilon$,线性无关;若含$v$则$v$不能被剩下向量线性表示,线性无关.因此$\epsilon'$线性无关,而这与$\epsilon$的极大性矛盾.

    2.对$E$中Hamel基$\epsilon$取一可数子集$\cbr{e_i}_{i\geq 1}$及线性泛函$f:(E,\norm{\cdot})\to \F,\begin{cases}
        e_n\mapsto n,& e_n\in \cbr{e_i}_{i\geq 1}\\
        e\mapsto 1,&e\in \epsilon-\cbr{e_i}_{i\geq 1}
    \end{cases}$,因此$\forall C\exists n>C:\abs{f(e_n)}=n>C$.因此$f$不连续.

    3.考虑$\norm{\cdot}_1:x\mapsto \norm{x}+\abs{f(x)}$,容易证明这是一个范数,且$\norm{x}\leq \norm{x}_1$.而由上可知$\forall C>0\exists e_n:\norm{e_n}_1=\norm{e_n}+n>\norm{e_n}$,因此$\norm{\cdot}_1$严格强于$\norm{\cdot}$.
\end{proof}

\paragraph*{3.8}设$E$是数域$\F$上$n$维向量空间($n<+\infty$),$e=\cbr{e_i}_{i\in [n]}$为$E$上的一组基.记$[u]$为$u\in \hom(E)$在基$e$下对应的矩阵.\\
1.证明$\varphi:u\mapsto [u]$给出$\hom(E)\to M_n(\F)$间的一个同构映射.\\
2.若$E=\F^n,e$是经典基,其上取范数$\norm{\cdot}_2$.证明若$u$(或等价地$[u]$)可\textcolor{red}{正交相似}对角化,则$\norm{u}=\max_{\lambda\text{是}u\text{的特征值}} \abs{\lambda}$.\\
3.基$e$如上,试由$[u]$中元素分别确定$p=1,\infty$时$u:(\F^n,\norm{\cdot}_p)\to (\F^n,\norm{\cdot}_p)$的范数$\norm{u}$.

\begin{proof}
    1.仅需证明$\varphi$为线性双射.首先记$u(e_j)=\sum_{i\in [n]}u_{ij}e_i,x=\sum_{i\in [n]}a_ie_i,[u]x=u(x)=\sum_{i\in [n]}a_iu(e_i)=\sum_{i\in [n]}a_i\sum_{j\in [n]}u_{ji}e_j=\sum_{i,j\in [n]}a_ju_{ij}e_i$.因此$[\lambda u+\mu v]x=\sum_{i,j\in [n]}a_j(\lambda u_{ij}+\mu v_{ij})e_i=\lambda\sum_{i,j\in [n]}a_ju_{ij}e_i+\mu\sum_{i,j\in [n]}a_jv_{ij}e_i=(\lambda [u]+\mu [v])x$.\\
    单射:$[u]=[v]\iff \forall \cbr{a_j}_{j\in [n]}\subset \F:\sum_{i,j\in [n]}a_ju_{ij}e_i=\sum_{i,j\in [n]}a_jv_{ij}e_i\iff \forall x\in E:u(x)=v(x)\iff u=v$,\\
    满射:$\forall [u]\in M_n(\F)\exists \cbr{u_{ij}}\subset \F\forall x\in E:u(x)=[u]x$.因此$\varphi$为线性双射,即同构.

    2.$u$可正交相似对角化,即可在$E$中取正交基$\cbr{\epsilon_i}_{i\in [n]}$有$[u]\epsilon_i=\lambda_i\epsilon_i$,其中$\lambda_i$为$[u]$的特征值.因此$$\forall x\in E:x=\sum_{i\in [n]}a_i\epsilon_i, \norm{u(x)}_2=\norm{\sum_{i\in [n]}a_iu(\epsilon_i)}_2=\norm{\sum_{i\in [n]}a_i\lambda_i\epsilon_i}_2$$
    而由$\cbr{\epsilon_i}_{i\in [n]}$是正交基,即$\abr{\epsilon_i,\epsilon_j}=\delta_{ij}$.
    故$$\begin{aligned}
        \norm{\sum_{i\in [n]}c_i\epsilon_i}^2_2&=\norm{\sum_{i\in [n-1]}c_i\epsilon_i}^2_2+\norm{c_n\epsilon_n}^2_n+2\abr{\sum_{i\in [n-1]}c_i\epsilon_i,c_n\epsilon_n}=\norm{\sum_{i\in [n-1]}c_i\epsilon_i}^2_2+c_n^2+2\sum_{i\in [n-1]}c_ic_n\abr{\epsilon_i,\epsilon_n}\\
        &=\norm{\sum_{i\in [n-1]}c_i\epsilon_i}^2_2+c_n^2=\sum_{i\in [n]}c_n^2
    \end{aligned}$$
    因此$$\norm{u}=\sup_{x\in E-\cbr{0}}\frac{\norm{u(x)}}{\norm{x}}=\sqrt{\frac{\sum_{i\in [n]}a_i^2\lambda_i^2}{\sum_{i\in [n]}a_i^2}}\leq \max_{i\in [n]}\abs{\lambda_i}$$
    设上式取到极大值时的指标为$i'$,则$\norm{u}\geq \frac{\norm{u(\epsilon_{i'})}}{\norm{\epsilon_{i'}}}=\lambda_{i'}=\max_{i\in [n]}\lambda_i$.因此$\norm{u}=\max_{i\in [n]}\lambda_i$.

    {\small 答案给出一个$u$仅能对角化而不能正交相似对角化时的反例.考虑$[u]=\begin{pmatrix}1&1\\0&0\end{pmatrix},P=\begin{pmatrix}1&-1\\0&1\end{pmatrix}$,此时$P\rev [u]P=\begin{pmatrix}1&0\\0&0\end{pmatrix}$,即$[u]$的特征值为$\lambda_1=1,\lambda_2=0$.而$x=2P^{(1)}+P^{(2)}=\begin{pmatrix}1\\1\end{pmatrix}$时,$\norm{u}\geq \frac{\norm{[u](1,1)^T}^2_2}{\norm{(1,1)^T}^2_2}=\frac{2}{\sqrt{2}}=\sqrt{2}>1=\max\cbr{\lambda_1,\lambda_2}$.}

    3.$p=1$:由$x=\sum_{i\in [n]}a_ie_i,u(x)=\sum_{i\in [n]}a_iu(e_i)=\sum_{i\in [n]}a_i\sum_{j\in [n]}u_{ji}e_j=\sum_{j\in [n]}e_j\sum_{i\in [n]}a_iu_{ji}$,故$$\norm{u(x)}_1=\sum_{j\in [n]}\abs{\sum_{i\in [n]}a_iu_{ji}}\leq \sum_{i,j\in [n]}\abs{a_i}\abs{u_{ji}}=\sum_{i\in [n]}\abs{a_i}\sum_{j\in [n]}\abs{u_{ji}}\leq \max_{i\in [n]}\sum_{j\in [n]}\abs{u_{ji}}\sum_{i\in [n]}\abs{a_i}=\max_{i\in [n]}\sum_{j\in [n]}\abs{u_{ji}}\norm{x}_1$$
    因此$\norm{u}\leq \max_{i\in [n]}\sum_{j\in [n]}\abs{u_{ji}}$.设上式取最大值时的指标为$i'$,则$x=e_{i'}$时,$\norm{u}\geq \frac{\norm{u(e_{i'})}_1}{\norm{e_{i'}}_1}=\sum_{j\in [n]}\abs{u_{ji'}}=\max_{i\in [n]}\sum_{j\in [n]}\abs{u_{ji}}$.因此$\norm{u}=\max_{i\in [n]}\sum_{j\in [n]}\abs{u_{ji}}$.

    $p=\infty$:$$\norm{u(x)}_\infty=\max_{j\in [n]}\abs{\sum_{i\in [n]}a_iu_{ji}}\leq \max_{j\in [n]}\sum_{i\in [n]}\abs{a_i}\abs{u_{ji}}\leq \max_{j\in [n]}\sum_{i\in [n]}\abs{u_{ji}}\max_{i\in [n]}\abs{a_i}=\max_{j\in [n]}\sum_{i\in [n]}\abs{u_{ji}}\norm{x}_\infty$$
    因此$\norm{u}\leq \max_{j\in [n]}\sum_{i\in [n]}\abs{u_{ji}}$.设上式取最大值时的指标为$j'$,则$x=(\sgn u_{j'1},\sgn u_{j'2},\cdots,\sgn u_{j'n})^T,\norm{x}_\infty=1$,$$\norm{u(x)}_\infty=\abs{\sum_{i\in [n]}(\sgn u_{j'i})u_{j'i}}=\abs{\sum_{i\in [n]}\abs{u_{j'i}}}=\sum_{i\in [n]}\abs{u_{j'i}}=\max_{j\in [n]}\sum_{i\in [n]}\abs{u_{ji}}\implies \norm{u}\geq \frac{\norm{u(x)}_\infty}{\norm{x}_\infty}=\max_{j\in [n]}\sum_{i\in [n]}\abs{u_{ji}}$$
    因此$\norm{u}=\max_{i\in [n]}\sum_{j\in [n]}\abs{u_{ij}}$.
\end{proof}

\parablue{3.9}$E$是Banach空间.\\
1.设$u\in \mathcal{B}(E),\norm{u}<1$,证明$\id_E-u$在$\mathcal{B}(E)$中可逆.HINT:考虑$\mathcal{B}(E)$中级数$\sum u^n$.\\
2.记$\GL(E)$为$\mathcal{B}(E)$中全体可逆元构成的集合,证明$\GL(E)$关于复合运算成群,且为$\mathcal{B}(E)$上开集.\\
3.证明$u\mapsto u\rev$是$\GL(E)$上的同胚映射.

\begin{proof}
    1.$\sum_{n\geq 0}\norm{u^n}\leq \sum_{n\geq 0}\norm{u}^n=\frac{1}{1-\norm{u}}$收敛,即$\sum_{n\geq 1}u^n$绝对收敛.由$E$是Banach空间知其收敛,即$\sum_{n\geq 0}u^n\in \mathcal{B}(E)$.而$(\id_E-u)\circ\br{\sum_{n\geq 0}u^n}=\br{\sum_{n\geq 0}u^n}\circ(\id_E-u)=\id_E$,因此$\id_E-u$在$\mathcal{B}(E)$中可逆.

    2.$\forall u,v,w\in \GL(E):$$$\begin{gathered}
        \br{(u\circ v)\circ (v\rev \circ u\rev)}(x)=(u\circ v)(v\rev(u\rev(x)))=u(v(v\rev(u\rev(x))))=u(u\rev(x))=x\implies u\circ v\in \GL(E)\\
        u\circ \id_E=\id_E\circ u=u,\qquad \exists u\rev:u\circ u\rev=u\rev\circ u=\id_E\\
        \br{(u\circ v)\circ w}(x)=u(v(w(x)))=\br{u\circ (v\circ w)}(x)
    \end{gathered}$$
    因此$(\GL(E),\circ)$是群.而$\forall u\in \GL(E)\exists \ve=\frac{1}{\norm{u\rev}}\forall v\in \mathcal{B}(E):\norm{u-v}<\frac{1}{\norm{u\rev}}\implies \norm{\id_E-u\rev v}\leq \norm{u\rev}\norm{u-v}<1\implies u\rev v$可逆$\implies v$可逆.因此$\GL(E)$是$\mathcal{B}(E)$中开集.

    3.由于$\varphi:u\mapsto u\rev$为双射且$\varphi=\varphi\rev$,故$\varphi$连续则$\varphi\rev$连续,故仅需证明$\varphi$连续:
    $$\forall u\in \GL(E)\forall \ve\exists \delta<\frac{\ve}{\norm{u\rev}^2+\norm{u\rev}\ve}\forall v\in \GL(E):\norm{u-v}<\delta\implies \norm{u\rev-v\rev}=\norm{u\rev v\rev (u-v)}<\norm{u\rev}\norm{v\rev}\delta$$
    而$\norm{v\rev}=\norm{u\rev\br{\id_E-(u-v)u\rev}\rev}\leq \norm{u\rev}\norm{\br{\id_E-(u-v)u\rev}\rev}$,其中$\norm{(u-v)u\rev}<\norm{u\rev}\delta=\frac{\ve}{\norm{u\rev}+\ve}<1$,因此$$\norm{\br{\id_E-(u-v)u\rev}\rev}=\norm{\sum_{n\geq 0}\br{(u-v)u\rev}^n}\leq \sum_{n\geq 0}\norm{(u-v)u\rev}^n=\frac{1}{1-\norm{(u-v)u\rev}}<\frac{1}{1-\norm{u\rev}\delta}$$
    代入可得:$\norm{u\rev-v\rev}<\frac{\norm{u\rev}\delta}{1-\norm{u\rev}\delta}=\frac{1}{1-\norm{u\rev}\delta}-1<\frac{\ve}{\norm{u\rev}}$,故$\varphi$连续.
\end{proof}

\paragraph*{3.10}$f\in L^2(\R),g(x)=\frac{1_{\babr{1,+\infty}}(x)}{x}$,证明$fg\in L^1(\R)$.举例说明$f_1,f_2\in L^1(\R),f_1 f_2\notin L^1(\R)$.

\begin{proof}
    由H\"{o}lder不等式,$$\norm{fg}_1\leq \norm{f}_2\norm{g}_2=\int_1^{+\infty}\frac{\d x}{x^2}\norm{f}_2=\norm{f}_2<\infty$$
    因此$fg\in L^1(\R)$.

    考虑$f(x)=x^{-\frac{1}{2}}1_{\abbr{0,1}}(x),\norm{f}_1=2,\norm{f^2}_1=+\infty$.
\end{proof}

\paragraph*{3.11}$(\Omega,\mathcal{A},\mu)$为有限测度空间.\\
1.证明$0<p<q\leq \infty$则$L^q(\Omega)\subset L^p(\Omega)$.举反例说明$\mu(\Omega)=\infty$时结论不成立.\\
2.证明若$f\in L^\infty(\Omega)$则$f\in\bigcap_{p<\infty}L^p(\Omega)$且$\norm{f}_\infty=\lim_{p\to\infty}\norm{f}_p$.\\
3.设$f\in\bigcap_{p<\infty}L^p(\Omega)$且$\varlimsup_{p\to \infty}\norm{f}_p<\infty$,证明$f\in L^\infty(\Omega)$.

\begin{proof}
    1.由H\"older不等式,考虑$p\rev=q\rev+s\rev$,其中$s=\br{p\rev-q\rev}\rev\in (0,+\infty)$,有$$\forall f\in L^q(\Omega):\norm{f}_p\leq \norm{f}_q\norm{1}_s=\mu(\Omega)^{\frac{1}{s}}\norm{f}_q<\infty\implies f\in L^p(\Omega)$$
    $\mu(\Omega)=\infty$时,$\norm{1}_\infty=1, \norm{1}_p=\mu(\Omega)=\infty$.

    2.由上,$\forall p\in (0,\infty):L^\infty(\Omega)\subset L^p(\Omega)$,因此$L^\infty(\Omega)\subset \bigcap_{p<\infty}L^p(\Omega)$.\\
    而$\norm{f}_p\leq \mu(\Omega)^{\frac{1}{s}}\norm{f}_\infty$,$q=\infty$时$s=p$.两端取$p\to \infty$有$\lim_{p\to \infty}\norm{f}_p\leq \norm{f}_\infty$.\\
    另一方面,设$S_\delta=\cbr{x\in\Omega:\abs{f(x)}\geq \norm{f}_\infty-\delta},\delta\in \br{0,\norm{f}_\infty}$.有$$\norm{f}_p\geq \br{\int_{S_\delta}\br{\norm{f}_\infty-\delta}^p\d \mu}^{1/p}=\br{\norm{f}_\infty-\delta}\mu(S_\delta)^{1/p}\implies \lim_{p\to \infty}\norm{f}_p\geq \norm{f}_\infty-\delta$$
    而$\delta>0$,因此有$\lim_{p\to \infty}\norm{f}_p\geq \norm{f}_\infty$.综上,$\lim_{p\to \infty}\norm{f}_p=\norm{f}_\infty$.

    3.若否,即对$E_M=\cbr{x\in \Omega:\abs{f(x)}\geq M},\forall M>0:\mu(E_M)>0$,则$\norm{f}_p\geq \int_{E_M}\abs{f}^p\d \mu\geq M\mu(E_M)^{1/p}$.两端取$p\to \infty$有$\infty>\lim_{p\to \infty}\norm{f}_p\geq M$.由$M$的任意性,$\lim_{p\to \infty}\norm{f}_p=\infty$,矛盾.
\end{proof}

\paragraph*{3.12}$0<p<q\leq \infty,\theta\in [0,1]$,且$\frac{1}{s}=\frac{\theta}{p}+\frac{1-\theta}{q}$.证明$\forall f\in L^p(\Omega)\cap L^q(\Omega):f\in L^s(\Omega),\norm{f}_s\leq \norm{f}_p^\theta\norm{f}_q^{1-\theta}$.

\begin{proof}
    若$\theta=0$则$s=q$,此时$f\in L^q(\Omega)=L^s(\Omega),\norm{f}_s=\norm{f}_q$.$\theta=1$时同理,换$q$为$p$即可.

    若$\theta\in (0,1)$则$\frac{s\theta}{p}+\frac{s(1-\theta)}{q}=1$.此时$\abs{f}^{s\theta}\in L^{\frac{p}{s\theta}}(\Omega),\abs{f}^{s(1-\theta)}\in L^{\frac{q}{s(1-\theta)}}(\Omega)$,故由H\"older不等式有:
    $$\int_{\Omega}\abs{f}^s=\norm{\abs{f}^{s\theta}\abs{f}^{s(1-\theta)}}_1\leq \norm{\abs{f}^{s\theta}}_{\frac{p}{s\theta}}\norm{\abs{f}^{s(1-\theta)}}_{\frac{q}{s(1-\theta)}}=\br{\int_{\Omega}\abs{f}^p}^{s\theta/p}\br{\int_{\Omega}\abs{f}^q}^{s(1-\theta)/q}=\norm{f}^{s\theta}_p\norm{f}^{s(1-\theta)}_q$$
    因此$\norm{f}_s\leq \norm{f}^{\theta}_p\norm{f}^{1-\theta}_q<\infty,f\in L^s(\Omega)$.
\end{proof}

\parablue{3.13 (广义Minkowski不等式)}设\textcolor{red}{$\sigma$-有限}测度空间$(\Omega_1,\mathcal{A}_1,\mu_1),(\Omega_2,\mathcal{A}_2,\mu_2),0<p<q<\infty$.\\
证明对任意可测函数$f:(\Omega_1\times \Omega_2,\mathcal{A}_1\otimes \mathcal{A}_2)\to \F$有:$$\br{\int_{\Omega_2}\br{\int_{\Omega_1}\abs{f(x_1,x_2)}^p\d \mu_1(x_1)}^{q/p}\d \mu_2(x_2)}^{1/q}\leq \br{\int_{\Omega_1}\br{\int_{\Omega_2}\abs{f(x_1,x_2)}^q\d \mu_2(x_2)}^{p/q}\d \mu_1(x_1)}^{1/p}$$

\begin{proof}[Proof in Folland Theorem 6.19 \& \href{https://math.stackexchange.com/questions/2355672/prove-minkowskis-inequality-for-integrals}{Here}]
    首先我们设$F(x_1,x_2)=\abs{f(x_1,x_2)}^p,s=q/p\in (1,\infty)$,则可改写不等式为:
    $$\br{\int_{\Omega_2}\br{\int_{\Omega_1}F(x_1,x_2)\d \mu_1(x_1)}^s\d \mu_2(x_2)}^{1/s}\leq \int_{\Omega_1}\br{\int_{\Omega_2}F(x_1,x_2)^s\d \mu_2(x_2)}^{1/s}\d \mu_1(x_1)$$
    考虑$s$的共轭数$r$及$g\in L^r(\Omega_2)$,有:
    $$\begin{gathered}
        \int_{\Omega_2}\br{\int_{\Omega_1}F(x_1,x_2)\d \mu_1(x_1)}\abs{g(x_2)}\d \mu_2(x_2)\stackrel{\text{Tonelli定理}}{=}\int_{\Omega_1\times\Omega_2}F(x_1,x_2)\abs{g(x_2)}\d \mu_1(x_1)\d \mu_2(x_2)\\
        \stackrel{\text{H\"older不等式}}{\leq} \int_{\Omega_1}\norm{F(x_1,\cdot)}_s\norm{g}_r\d\mu_1(x_1)=\norm{g}_r\int_{\Omega_1}\br{\int_{\Omega_2}F(x_1,x_2)^s\d \mu_2(x_2)}^{1/s}\d\mu_1(x_1)
    \end{gathered}$$
    由$L^s(\Omega_2)\to L^r(\Omega_2)^*$有一个同构$f\mapsto \varphi(f)$,其中$\varphi(f):g\mapsto \int_{\Omega_2}f(x_2)g(x_2)\d \mu_2(x_2)$,因此$$\begin{gathered}
        \br{\int_{\Omega_2}\br{\int_{\Omega_1}F(x_1,x_2)\d \mu_1(x_1)}^s\d \mu_2(x_2)}^{1/s}=\norm{\int_{\Omega_1}F(x_1,\cdot)\d \mu_1(x_1)}_s=\norm{\varphi\br{\int_{\Omega_1}F(x_1,\cdot)\d \mu_1(x_1)}}\\
        =\sup_{g\in L^r(\Omega_2)}\frac{1}{\norm{g}_r}\abs{\int_{\Omega_2}\br{\int_{\Omega_1}F(x_1,x_2)\d \mu_1(x_1)}\abs{g(x_2)}\d \mu_2(x_2)}\leq \int_{\Omega_1}\br{\int_{\Omega_2}F(x_1,x_2)^s\d \mu_2(x_2)}^{1/s}\d\mu_1(x_1)
    \end{gathered}$$
\end{proof}

\parablue{3.14}设$p\in (0,\infty)$.\\
1.对$\forall x=\cbr{x_n}_{n\in \N}\in \ell_p$定义$(0,1)$上函数$T(x)(t)=\sum_{n\geq 1}\br{n(n+1)}^{\frac{1}{p}}x_n1_{\br{\frac{1}{n+1},\frac{1}{n}}}(t)$.\\
证明$T:\ell_p\to \im(T)\subset L^p(0,1)$是线性等距同构映射.\\
2.若$p\geq 1, \frac{1}{p}+\frac{1}{q}=1$,对$\forall f\in L^p(0,1)\forall n\geq 1$定义$S(f)_n=\br{n(n+1)}^{\frac{1}{q}}\int_{\frac{1}{n+1}}^{\frac{1}{n}}f(t)\d t$.\\
证明$S:L^p(0,1)\to \ell_p,f\mapsto \cbr{S(f)_n}_{n\geq 1}$是线性映射,且$S\circ T=\id_{\ell_p}$.

\begin{proof}
    1.首先$$\norm{T(x)}_p=\br{\int_0^1\br{\sum_{n\geq 1}[n(n+1)]^{1/p}x_n1_{\br{\frac{1}{n+1},\frac{1}{n}}}(t)}^p\d t}^{1/p}=\br{\sum_{n\geq 1}\int_{\frac{1}{n+1}}^{\frac{1}{n}}\br{[n(n+1)]^{1/p}x_n}^p\d t}^{1/p}=\br{\sum_{n\geq 1}x_n^p}^{1/p}=\norm{x}_p$$
    线性性和单射性显然,由定义,$T$线性等距双射,即得证.

    2.线性性显然.$$\begin{aligned}
        S(T(x))_n&=S\br{\sum_{n\geq 1}[n(n+1)]^{1/p}x_n1_{\br{\frac{1}{n+1},\frac{1}{n}}}(\cdot)}_n=[n(n+1)]^{1/q}\int_{\frac{1}{n+1}}^{\frac{1}{n}}\sum_{n\geq 1}[n(n+1)]^{1/p}x_n1_{\br{\frac{1}{n+1},\frac{1}{n}}}(t)\d t\\
        &=[n(n+1)]^{1/q}\int_{\frac{1}{n+1}}^{\frac{1}{n}}[n(n+1)]^{1/p}x_n\d t=[n(n+1)]^{1/q}\br{n^{\frac{1}{p}-1}(n+1)^{\frac{1}{p}}-n^{\frac{1}{p}}(n+1)^{\frac{1}{p}-1}}x_n=x_n
    \end{aligned}$$
    因此$S\circ T=\id_{\ell_p}$.
\end{proof}

\parablue{3.15}1.证明:若$(E,d)$为可分度量空间,则$(F,d)$也是,$F\subset E$.\\
2.证明$\R^n,c_0,\ell_p(p\in \babr{1,\infty}),C([a,b],\R),C_0(\R,\R),L^p(0,1)(p\in \babr{1,\infty})$都是可分的.\\
3.设$C=\cbr{\pm 1}^{\N}\subset \ell_\infty$,验证$\forall x,y\in C:x\neq y\implies \norm{x-y}_\infty=2$,再证明$C$不可数,由此导出$\ell_\infty$不可分.并类似证明$L^\infty(0,1)$不可分.

\begin{proof}
    1.设$E$的可数稠密集为$A$,则$\forall x\in F\forall \ve>0\exists a\in A:x\in B(a,\ve)$,从中取$x_a\in F\cap B(a,\ve)$,则$B(a,\ve)\subset B(x_a,2\ve)$,因此$A_F=\cbr{x_a}_{a\in A}$是一个可数集,且$\forall x\in F\forall\ve>0\exists x_a\in A_F:x\in B(x_a,2\ve)$,故$\overline{A_F}=F$.

    2.(1)$\R^n=\overline{\Q^n}$;\\
    % $c_0$为($\R^{\infty}$中)收敛到0的全体序列,故$c_0=\bigcup_{n\geq 1}A_n^{\R}, A^{\R}_n=\cbr{\cbr{a_1,\cdots,a_n,0,\cdots}:a_i\in \R}$.而$\overline{A_n^{\Q}}=A^{\R}_n$,因此$c_0=\overline{\bigcup_{n\geq 1}A_n^{\Q}}$.\\

    (3)$\forall x\in \ell_p\forall \ve>0\exists N\in \N:\sum_{n>N}\abs{x_n}^p<\frac{\ve}{2}$,再取$q=\cbr{q_n}_{n\in [N]}\subset \Q$有$\abs{x_n-q_n}<\br{\frac{\ve}{2N}}^{1/p}(n\in [N])$.记$A_n=\cbr{\cbr{q_1,\cdots,q_n,0,\cdots}:q_i\in \Q}\subset \ell_p$,则$$\forall x\in \ell_p\forall \ve>0\exists N\in \N\exists q\in A_N:\norm{x-q}_p^{p}=\sum_{n>N}\abs{x_n}^p+\sum_{n\leq N}\abs{x_n-q_n}^p<\frac{\ve}{2}+N\cdot\frac{\ve}{2N}=\ve$$
    因此$\forall x\in \ell_p\forall \ve>0\exists q\in \bigcup_{n\geq 1}A_n:\norm{x-q}_p<\ve$,即$\overline{\bigcup_{n\geq 1}A_n}=\ell_p$.

    (4)首先由紧集上的(一致)连续性,$$\forall f\in C[a,b]\forall \ve\exists \delta\forall x,y\in [a,b]:\abs{x-y}<\delta\implies \abs{f(x)-f(y)}<\ve$$
    可以考虑分划$\cbr{s_n}_{n\in [N]}\subset [a,b]$,其中$s_0=a,s_N=b,s_{k+1}-s_k<\delta$.依次用折线连接$(s_k,f(s_k))$得到以折线连接的分段函数$g$,则$\forall x\in [s_k,s_{k+1}]:$ $$\abs{f(x)-g(x)}=\abs{
        \frac{f(s_{k+1})-f(s_k)}{s_{k+1}-s_k}(x-s_k)+f(s_k)-f(x)}\leq \abs{\frac{x-s_k}{s_{k+1}-s_k}}\abs{f(s_{k+1})-f(s_k)}+\abs{f(s_k)-f(x)}<2\ve$$
    因此$[a,b]$上全体折线函数稠密于$C[a,b]$中.而全体分段点$(s_k,f(s_k))\in \Q^2$的折线函数稠密于前者,该集合可数.因此得证.

    (5)由于$\forall f\in C_0(\R)\exists N>0:\norm{f|_{\R-[-N,N]}}<\ve$,因此$\bigcup_{N\geq 0}C[-N,N]$稠密于$C_0(\R)$.而$C[-N,N]$由(4)有一个可数稠密集$B_N$,因此$\overline{\bigcup_{N\geq 0}B_N}=C_0(\R)$.

    (6)由$L^1(0,1)$中阶梯函数族稠密于$L^p(0,1)$,而可以选取函数值为有理数和分段点为有理函数的阶梯函数稠密于前者,因此$L^p(0,1)$有可数稠密子集.

    3.$\forall x,y\in C:x\neq y\implies \norm{x-y}_\infty=\sup_{n\geq 1}\abs{x_n-y_n}=2$,这是由于$\exists N:x_N\neq y_N\implies \abs{x_N-y_N}=\abs{1-(-1)}=2$.\\
    而$C\to\mathcal{P}{\N}, x=\cbr{x_n}\mapsto A\subset \N,x_n=1\iff n\in A$给出一个双射,因此$\abs{C}=\abs{\mathcal{P}{\N}}>\N$,故不可数.\\
    因此,若$\ell_\infty$有可数稠密子集$A$,则$C$有可数稠密子集$A\cap C$,但$\exists x\in C-A\cap C\exists \ve<2\forall y\in A\cap C:\norm{x-y}>\ve$,因此$C$中不存在这样的稠密子集.
\end{proof}

\parablue{3.16 (卷积)}设$f,g\in L^1(\R)$.(下述积分为Lebesgue积分)\\
1.证明$\int_{\R^2}f(u)g(v)\d u\d v=\br{\int_\R f(u)\d u}\br{\int_\R g(v)\d v}=\int_\R\br{\int_\R f(x-y)g(y)\d y}\d x$,由此导出$x\mapsto \int_\R f(x-y)g(y)\d y$在$\R$上a.e.有定义.\\
2.定义卷积$f*g(x)=\begin{cases}
    \int_\R f(x-y)g(y)\d y, &\text{积分存在},\\ 0,& \text{else} .
\end{cases}$.证明$f*g\in L^1(\R)$且$\norm{f*g}_1\leq \norm{f}_1\norm{g}_1$.\\
3.取$f=1_{[0,1]}$,计算$f*f$.

\begin{proof}
    1.第一个等号由Fubini定理立得.第二个等号:$$\begin{aligned}
        \int_{\R}\br{\int_{\R}f(x-y)g(y)\d y}\d x&=\int_{\R}g(y)\br{\int_{\R}f(x-y)\d x}\d y=\int_{\R}g(y)\br{\int_{\R}f(x-y)\d (x-y)}\d y\\
        &=\int_{\R}g(v)\br{\int_{\R}f(u)\d u}\d v=\int_{\R^2}f(u)g(v)\d u\d v
    \end{aligned}$$
    而$\int_{\R}\br{\int_{\R}f(x-y)g(y)\d y}\d x=\br{\int_\R f(u)\d u}\br{\int_\R g(v)\d v}<\infty$,因此$\int_{\R}f(x-y)g(y)\d y$在$\R$上a.e.有限,故有定义.

    2.由上,$$\norm{f*g}_1=\int_{\R}\abs{f*g(x)}\d x\leq \int_{\R}\br{\int_{\R}\abs{f(x-y)}\abs{g(y)}\d y}\d x=\br{\int_{\R}\abs{f(x)}\d x}\br{\int_{\R}\abs{g(y)}\d y}=\norm{f}_1\norm{g}_1<\infty$$
    因此$f*g\in L^1(\R),\norm{f*g}_1\leq \norm{f}_1\norm{g}_1$.

    3.由$1_{[0,1]}(x-y)=1_{[x-1,x]}(y)$,因此$$f*f(x)=\int_{\R}1_{[0,1]}(x-y)1_{[0,1]}(y)\d y=\int_{\R} 1_{[x-1,x]}(y)1_{[0,1]}(y)\d y=m\br{[x-1,x]\cap [0,1]}=\begin{cases}
        x,&x\in [0,1]\\
        2-x,&x\in [1,2]\\
        0,&\text{else}
    \end{cases}$$
\end{proof}

\parablue{3.17 (Hardy不等式)}在$\R$上考虑Borel $\sigma$-代数和Lebesgue测度.设$p\in (1,\infty)$且$f\in L^p(0,+\infty)$.\\
在$(0,+\infty)$上定义$F(x)=\frac{1}{x}\int_0^x f(t)\d t$.本题的目标是证明Hardy不等式:$$\norm{F}_p\leq \frac{p}{p-1}\norm{f}_p,\qquad\forall f\in L^p(0,\infty)$$
1.说明$F$在$(0,+\infty)$上的定义是合理的,且$$\forall x_1,x_2>0:\abs{x_1F(x_1)-x_2F(x_2)}\leq \abs{x_1-x_2}^{\frac{1}{q}}\norm{f}_p$$
其中$\frac{1}{p}+\frac{1}{q}=1$,并由此证明$F$在$0,+\infty$上连续,故可测.\\
2.若$f$是有紧支撑的非负连续函数,证明$F$在$(0,+\infty)$上连续可导,且有$$(p-1)\int_0^{+\infty}F(x)^p\d x=p\int_0^{+\infty}F(x)^{p-1}f(x)\d x$$并由此导出Hardy不等式.\\
3.证明Hardy不等式对所有$f\in L^p(0,+\infty)$成立.\\
4.用反例说明$p=1$时不等式不成立,即不存在任何常数$C>0,\forall f\in L^p(0,+\infty):\norm{F}_p\leq C\norm{f}_p$.\\
5.证明$\frac{p}{p-1}$是使不等式成立的最优常数,即$\norm{F}_p\leq C\norm{f}_p\implies C\geq \frac{p}{p-1}$.\\
HINT:考虑$f(x)=x^{-\frac{1}{p}}1_{\bbr{1,n}}(x)$和极限$\lim_{n\to \infty}\norm{F1_{\bbr{1,n}}(x)}_p/\norm{f}_p$.

\begin{proof}
    1.
\end{proof}

\parablue{3.18}令$p\in \babr{2,+\infty}$.

1.首先证明Clarkson不等式:$$\norm{\frac{f+g}{2}}^p_p+\norm{\frac{f-g}{2}}^p_p\leq \frac{1}{2}\br{\norm{f}^p_p+\norm{g}^p_p},\qquad \forall f,g\in L^p(\R)$$
1.1.证明$\forall s,t\in \babr{0,+\infty}:s^p+t^p\leq \br{s^2+t^2}^{\frac{p}{2}}$.\\
1.2.证明$\forall a,b\in \R:\abs{\frac{a+b}{2}}^p+\abs{\frac{a-b}{2}}^p\leq \frac{1}{2}\br{\abs{a}^p+\abs{b}^p}$.\\
1.3.导出Clarkson不等式.

2.设$C$为$L^p(\R)$中非空闭凸集,且$f\in L^p(\R)$,记$d=d(f,C)$.下证:$\exists !g_0\in C:d=\norm{f-g_0}_p$.\\
2.1.解释为什么存在$C$中序列$\cbr{g_n}_{n\geq 1}$有$\norm{f-g_n}^p_p\leq d^p+\frac{1}{n}$.\\
2.2.用Clarkson不等式证明$\norm{\frac{g_n+g_m}{2}}^p_p\leq \frac{1}{2n}+\frac{1}{2m}$.\\
2.3.导出存在函数$g_0\in C$使得$d(f,C)=\norm{f-g_0}_p$.\\
2.4.证明上述$g_0\in C$唯一.

3.记上述$g_0$为$P_C(f)$,下证$P_C:L^p(\R)\to C$连续.\\
3.1.证明$\forall f,g\in L^p(\R):\norm{g-P_C(g)}_p\leq \norm{f-g}_p+\norm{f-P_C(f)}_p$.\\
3.2.用Clarkson不等式证明$$\forall f,g\in L^p(\R):\norm{\frac{P_C(f)-P_C(g)}{2}}^p_p\leq \frac{1}{2}\br{\norm{f-P_C(g)}^p_p-\norm{f-P_C(f)}^p_p}$$
3.3.最后导出$P_C$的连续性.

% \parablue{3.19}

\newpage
\addtocounter{section}{1}
\section{第五章}
\paragraph*{习题1}设$A=\cbr{x(t)\in C^1[a,b]:\abs{x(t)}\leq M, \abs{x'(t)}\leq M_1}$,则$A$是$C[a,b]$中的列紧集.
\begin{proof}
    仅需证明$A$等度连续,这样由$A$一致有界(即$\forall t\in [a,b]\forall x\in A: \abs{x(t)}\leq M$),再由Ascoli定理得到$A$相对紧(即列紧).

    首先由$\forall x\in A\forall t\in [a,b]:\abs{x'(t)}\leq M_1$可以给出$\forall t\in [a,b]\exists \delta_t>0:\abs{t-t_0}<\delta_t\implies \abs{x(t)-x(t_0)}\leq M_1\abs{t-t_0}$.用$B(t,\delta_t/2)$覆盖$[a,b]$,由紧性可以得到有限个开球$\cbr{B\br{t_i,\frac{\delta_i}{2}}}$覆盖$[a,b]$.令$t_{n_1},t_{n_2},\cdots,t_{n_m}$依次是$t$到$t_0$之间所有的开球中心$t_i$,因此有$$\begin{aligned}
        \forall t,t_0\in [a,b]:\abs{x(t)-x(t_0)}&\leq \abs{x(t)-x(t_{n_1})}+\abs{x(t_{n_1})-x(t_{n_2})}+\cdots+\abs{x(t_{n_m})-x(t_0)}\\
        &\leq M_1(\abs{t-t_{n_1}}+\abs{t_{n_1}-t_{n_2}}+\cdots+\abs{t_{n_m}-t_0})=M_1\abs{t-t_0}
    \end{aligned}$$
    由$x$的任意性,所以有$$\forall t_0\in [a,b]\forall \varepsilon>0\exists B\br{t_0,\frac{\varepsilon}{M_1}}\forall t\in B\br{t_0,\frac{\varepsilon}{M_1}}\forall x\in A:\abs{x(t)-x(t_0)}\leq M_1\abs{t-t_0}<\varepsilon$$
    故等度连续得证.
\end{proof}

\paragraph*{习题2}设$M$是$C[a,b]$中的有界集,证明集合$S=\cbr{F(x)=\int_a^x f(t)\d t:f\in M}$是列紧集.
\begin{proof}
    我们有$$\begin{gathered}
        \forall F\in S\forall x\in [a,b]:F(x)\leq \int_a^b\abs{f(t)}\d t=\norm{f}_1\leq (b-a)\norm{f}_\infty\\
        \abs{F(x)-F(x_0)}=\abs{\int_{x_0}^x f(t)\d t}\leq \int_{x_0}^x\abs{f(t)}\d t\leq \abs{x-x_0}\norm{f}_\infty
    \end{gathered}$$
    因此在$M$关于$\norm{\cdot}_\infty$有界时,$S$一致有界,且$F$是Lipschitz映射,故$S$等度连续.

    最后由Ascoli定理,$S$是列紧的.
\end{proof}

\paragraph*{习题3}证明集合$M=\cbr{\sin nx:n\in \Z_{\geq 0}}$在空间$C[0,\pi]$中是有界集,但不是列紧集.
\begin{proof}
    显然$\norm{\sin nx}_\infty=1$,但若$M$列紧,则$\exists \cbr{n_i}_{i\geq 1}\subset \Z_{\geq 0}\exists f\in C[0,\pi]:\sin n_ix\to f(x)$.\\
    而$\norm{f(x)-\sin n_ix}_\infty\geq \abs{f\br{\frac{k\pi}{n_i}}}\to 0, k\in [n_i]$,因此在$\cbr{\frac{k\pi}{n_i}:i\in \Z_{\geq 0},k\in [n_i]}$上$f$取0,而这是一个稠密集,且$f$连续,故$f=0$.但$\norm{\sin n_ix}_\infty=1$,矛盾.因此不存在这样的连续函数$f$,即$M$不列紧.
\end{proof}

\paragraph*{习题4}设$(M,d)$是一个列紧距离空间,$E\subset C(M)$,其中$C(M)$表示$M$上一切实值或复值连续函数全体,$E$中函数一致有界并满足下列不等式
$$|x(t_1)-x(t_2)|\leq c\cdot d(t_1,t_2)^\alpha,\qquad \forall x\in E, t_1,t_2\in M$$
其中$0<\alpha\leq 1,c>0$,求证$E$在$C(M)$中是列紧集.
\begin{proof}
    仅需证明$E$等度连续.$$\forall t_0\in M\forall\varepsilon\exists B\br{t_0,\sqrt[\alpha]{\frac{\varepsilon}{c}}}\forall t\in B\br{t_0,\sqrt[\alpha]{\frac{\varepsilon}{c}}}\forall x\in E:\abs{x(t)-x(t_0)}\leq c\cdot d(t,t_0)^\alpha<\varepsilon$$
\end{proof}

\paragraph*{5.3}拓扑空间$K$和度量空间$(E,d)$中,若$\cbr{f_n}$在$C(K,E)$中依一致范数收敛,则$\cbr{f_n}$等度连续.
\begin{proof}
    若$\norm{f}=\sup_{t\in K}\abs{f(t)}, \exists f\in C(K,E):\norm{f-f_n}\to 0$,则$\forall \varepsilon>0$,取$N\in \Z_{\geq 1}$,有$\forall n\geq N:\norm{f-f_n}<\varepsilon$.\\
    考虑$\forall x_0\in K\exists O(x)\forall x\in O(x_0):d(f(x),f(x_0))<\varepsilon$,则$\forall \varepsilon>0\forall x_0\in K\exists O(x_0)\forall x\in O(x_0)\forall n\geq N$时有$$d(f_n(x),f_n(x_0))\leq d(f_n(x),f(x))+d(f(x),f(x_0))+d(f(x_0),f_n(x_0))\leq 3\varepsilon$$
    因此$\cbr{f_n}_{n\geq N}$等度连续,故$\cbr{f_n}_{n\geq 1}=\cbr{f_n}_{1\leq n< N}\cup \cbr{f_n}_{n\geq N}$等度连续.
\end{proof}

\paragraph*{5.12}$[0,1]$上所有偶多项式$\mathcal{Q}$是否稠密于$C([0,1],\R)$?$[-1,1]$上所有偶多项式$\mathcal{R}$是否稠密于$C([-1,1],\R)$?
\begin{proof}
    首先$\mathcal{Q}$可分点,仅需注意到$x^2$在$[0,1]$上是双射.其次,$\forall x\in [0,1]:x^2+1\neq 0$.\\
    最后证明$\mathcal{Q}$是一个子代数:$\forall P,Q\in \mathcal{Q},\forall c\in \R$,记$P=\sum_{k= 0}^na_kx^{2k},Q=\sum_{k= 0}^mb_kx^{2k}$,
    $$cP=\sum_{k= 0}^nca_kx^{2k}\in\mathcal{Q},P+Q=\sum_{k= 0}^{\max\cbr{n,m}}(a_k+b_k)x^{2k}\in\mathcal{Q}, PQ=\sum_{k= 0}^{m+n}\br{\sum_{i+j=k}a_ib_j}x^{k}\in\mathcal{Q}$$
    因此由Stone-Weierstrass定理可知$\mathcal{Q}$稠密于$C([0,1],\R)$.

    另一方面,$\mathcal{R}$中的多项式都不是$[-1,1]$上的双射,因为$\forall P\in \mathcal{R}\forall x\in [0,1]:P(x)=P(-x)$.因此不能用Stone-Weierstrass定理.\\

\end{proof}

\newpage
\section{第六章}
\paragraph*{SJ 4.1}设$\sup_{n\geq 1}\abs{\alpha_n}<\infty$,在$\ell^1$上定义$T:\cbr{\xi_k}\mapsto \cbr{\alpha_k\xi_k}$.证明$T$有界线性且$\norm{T}=\sup_{n\geq 1}\abs{\alpha_n}$.
\begin{proof}
    $T$的线性性显然.设$a=\sup_{n\geq 1}\abs{\alpha_n},\xi=\cbr{\xi_k}_{k\geq 1}$,有$\norm{T\xi}_1=\sum_{k\geq 1}\abs{\alpha_k\xi_k}\leq a\sum_{k\geq 1}\abs{\xi_k}=a\norm{\xi}_1$,因此$\norm{T}\leq a$.\\
    另一方面对$\forall n\in \Z_{\geq 1}$,仅需考虑$\xi_k=\delta_{kn},\norm{\xi}_1=1,\norm{T\xi}_1=\abs{\alpha_n},\norm{T}\geq\sup_{n\geq 1}\abs{\alpha_n}$.故$\norm{T}=a$.
\end{proof}

\paragraph*{SJ 4.9}$X,Y$是Banach空间,$T\in \mathcal{B}(X,Y)$.若$T$是双射,证明$\exists a>0\exists b>0\forall x\in X:a\norm{x}\leq \norm{Tx}\leq b\norm{x}$.
\begin{proof}
    考虑双射$T\rev:Y\to X$,首先$\forall y_1,y_2\in Y\forall a_1,a_2\in \F\exists x_1,x_2\in X:$$$T\rev(a_1y_1+a_2y_2)=T\rev(a_1T(x_1)+a_2T(x_2))=T\rev\circ T(a_1x_1+a_2x_2)=a_1T\rev(y_1)+a_2T\rev(y_2)$$
    因此$T\rev$线性.其次由开映射定理,$\exists r>0:rB_Y\subset T(B_X)\implies rT\rev(B_Y)\subset B_X$,因此$\norm{T\rev}=\sup_{y\in B_Y}\norm{T\rev(y)}\leq r\rev$,因此$T\rev\in \mathcal{B}(X,Y)$,
    $$\forall x\in X\exists y\in Y:\norm{x}=\norm{T\rev(y)}\leq \norm{T\rev}\norm{y}\leq r\rev\norm{Tx}\implies r\norm{x}\leq \norm{Tx}$$
    因此仅需取$a=r,b=\norm{T}$即可.
\end{proof}

\paragraph*{SJ 4.13}考虑$T:C^1[-1,1]\to C[-1,1], x(t)\mapsto x'(t)$.\\
1.若$C^1[-1,1]$中范数是$\norm{x}_1=\max\cbr{\max_{t\in [-1,1]}\abs{x(t)},\max_{t\in [-1,1]}\abs{x'(t)}}$,则$T$是否有界?\\
2.若$C^1[-1,1]$中范数是$\norm{x}_2=\max_{t\in [-1,1]}\abs{x(t)}$,则$T$是否有界?
\begin{proof}
    1.$\frac{\norm{Tx}}{\norm{x}_1}=\frac{\norm{x'}_\infty}{\max\cbr{\norm{x}_\infty,\norm{x'}_\infty}}\leq 1$,因此$\norm{T}\leq 1$,有界.

    2.$\frac{\norm{Tx}}{\norm{x}_2}=\frac{\norm{x'}_\infty}{\norm{x}_\infty}$,因此取$x(t)=t^n$时,$\frac{\norm{Tx}}{\norm{x}_2}=n$,由$n$任意性,其无界.
\end{proof}

\paragraph*{SJ 4.14}定义$T(f)(x)=\int_a^x f(t)\d t,\forall f\in L^1[a,b]$.证明:\\
1.若$T:(L^1[a,b],\norm{\cdot}_1)\to (C[a,b],\norm{\cdot}_\infty)$,则$\norm{T}=1$;\\
2.若$T:(L^1[a,b],\norm{\cdot}_1)\to (L^1[a,b],\norm{\cdot}_1)$,则$\norm{T}=b-a$.
\begin{proof}
    1.$\norm{Tf}_\infty=\max_{x\in [a,b]}\abs{\int_a^x f(t)\d t}\leq \max_{x\in [a,b]}\int_a^x \abs{f(t)}\d t=\int_a^b \abs{f(t)}\d t=\norm{f}_1$,因此$\norm{T}\leq 1$.\\
    另一方面,$f(t)=\frac{1}{b-a}, \norm{Tf}_\infty=\max_{x\in [a,b]}\abs{\frac{x-a}{b-a}}=1$,因此$\norm{T}\geq 1$,得证.

    2.$\norm{Tf}_1=\int_a^b\abs{\int_a^xf(t)\d t}\d x\leq \int_a^b\int_a^x\abs{f(t)}\d t\d x\leq \int_a^b\int_a^b\abs{f(t)}\d t\d x=\int_a^b\norm{f}_1\d x=(b-a)\norm{f}_1$,因此$\norm{T}\leq b-a$.\\
    另一方面,取$f_n(t)=n\cdot 1_{[a,a+\frac{1}{n}]}(t),\norm{f_n}_1=1,\norm{Tf}_1=b-a-\frac{1}{2n}$,因此$\norm{T}\geq \sup_{n\geq 1}\frac{\norm{Tf_n}_1}{\norm{f_n}_1}=b-a$.因此$\norm{T}=b-a$.
\end{proof}

\paragraph*{SJ 4.32}$X$是Banach空间,$X_0$是$X$的闭子空间,定义$\Phi:X\to X/X_0, x\mapsto [x]$,其中$[x]$是含$x$的等价类,求证$\Phi$是开映射.
\begin{proof}
    在$X/X_0$上定义$\norm{[x]}=\inf_{x\in [x]}\norm{x}$,容易证明这是一个范数.
    
    其次,取$X/X_0$中的Cauchy列$\cbr{[x_n]}$,容易选取子列$\cbr{[x_{n_k}]}=\cbr{[u_k]}$使得$$\forall k\in \N^*\exists N_k\forall n,m\geq N_k:\norm{[u_n]-[u_m]}< 2^{-k}.$$
    考虑$u'_n\in [u_n],u'_{n+1}\in [u_{n+1}],v'_n\in X_0:\norm{u'_{n+1}-u'_n+v'_n}<2^{-n}$.记$w_n=u'_{n+1}-u'_n+v'_n,\sum_{n\geq 1}\norm{w_n}=1<\infty$,故$\cbr{w_n}$绝对收敛.由$X$完备,有$\sum_{n\geq 1}w_k=w$.令$[u]=[w]+[u_1]$,有$$\norm{[u_{n+1}]-[u]}=\norm{\bbr{u_{n+1}-w-u_1+\sum_{k=1}^n w_k}}\leq \norm{u_{n+1}-w-u_1+\sum_{k=1}^n w_k}=\norm{\sum_{k=1}^n w_k-w}\to 0$$
    因此$[u_n]\to [u]$.而$\cbr{[u_n]}$是$\cbr{[x_n]}$的收敛子列,因此$[x_n]\to [u]$.故其收敛,故$X/X_0$是Banach空间.

    最后,显然有$\norm{\Phi}\leq 1,\Phi\in \mathcal{B}(X,X/X_0)$且满,因此$\Phi$是开映射.
\end{proof}

\paragraph*{SJ 4.33}设$X$是$\ell^\infty$中只有有限个非0项的序列构成的子空间.定义$T:X\to X, \cbr{x_k}\mapsto \cbr{\frac{x_k}{k}}$,证明:\\
1.$T\in\mathcal{B}(X)$,并求出$\norm{T}$;2.$T\rev$无界;\\
3.这是否和Banach逆算子定理矛盾?
\begin{proof}
    $T$显然线性,$x=\cbr{x_1,x_2,\cdots,x_n,0,\cdots}\in X, Tx=\cbr{x_1,\frac{x_2}{2},\cdots,\frac{x_n}{n},0,\cdots}\in X$.$\norm{Tx}=\max_{k\in [n]}\frac{\abs{x_k}}{k}\leq \max_{k\in [n]}\frac{\norm{x}}{k}=\norm{x},\norm{T}\leq 1$.而$x'=\cbr{1,\cdots,1,0,\cdots}$时$\norm{Tx'}=1,\norm{T}\geq \frac{1}{1}=1$,故$\norm{T}=1$.

    $T\rev x=\cbr{x_1,2x_2,\cdots,nx_n,0,\cdots},\norm{T\rev x'}=n$,因此$\norm{T\rev}\geq n$,由$n$任意可知$T\rev$无界.

    这与Banach逆算子定理不矛盾,因为$X$不完备.如$x_n=\cbr{1,2\rev,3\rev,\cdots,n\rev,0,\cdots},x=\cbr{1,2\rev,3\rev,\cdots,n\rev,\cdots}$,\\
    $\norm{x_n-x}_\infty=\frac{1}{n+1}\to 0,x\notin X$.
\end{proof}

\paragraph*{SJ 4.36}令$\mathrm{Dom}(T)=\cbr{u\in L^2(\R):\int_{\R}t^2\abs{u(t)}^2\d t<\infty}$,且$\forall u\in \mathrm{Dom}(T):T(u)(t)=tu(t)$.说明$T$无界且闭.
\begin{proof}
    首先取$u(t)=\e^{-\abs{t}/n},n>0,\norm{u}^2_2=n,\norm{Tu}^2_2=\frac{n^3}{2},\norm{T}\geq \frac{n}{\sqrt{2}}$,由$n$任意可知$T$无界.
    
    要说明$G(T)=\cbr{(u,Tu)\in L^2(\R)\times L^2(\R):u\in \mathrm{Dom}(T)}$闭,需要说明$G(T)$中的收敛列$(u_n,Tu_n)\to (u,v)\in G(T)$. 而$T(u_n-u)=tu_n(t)-tu(t),\int_\R t^2(u_n-u)(t)^2\d t$
\end{proof}

\end{document}