% 用ctex显示中文并用fandol主题
\usepackage[fontset=fandol]{ctex}
\setmainfont{CMU Serif} % 能显示大量外文字体
\xeCJKsetup{CJKmath=true} % 数学模式中可以输入中文

% AMS全家桶,\DeclareMathOperator依赖之
\usepackage{amsmath,amssymb,amsthm,amsfonts,amscd}
\usepackage{pgfplots,tikz,tikz-cd} % 用来画交换图
\usepackage{bm,mathrsfs} % 粗体字母(含希腊字母)和\mathscr字体
\everymath{\displaystyle} % 全体公式为行间形式

% 纸张上下左右页边距
\usepackage[a4paper,left=1cm,right=1cm,top=1.5cm,bottom=1.5cm]{geometry}
% 生成书签和目录上的超链接
\usepackage[colorlinks=true,linkcolor=blue,filecolor=blue,urlcolor=blue,citecolor=cyan]{hyperref}
% 各种列表环境的行距
\usepackage{enumitem}
\setenumerate[1]{itemsep=0pt,partopsep=0pt,parsep=\parskip,topsep=0pt}
\setenumerate[2]{itemsep=0pt,partopsep=0pt,parsep=\parskip,topsep=0pt}
\setenumerate[3]{itemsep=0pt,partopsep=0pt,parsep=\parskip,topsep=0pt}
\setitemize[1]{itemsep=0pt,partopsep=0pt,parsep=\parskip,topsep=5pt}
\setdescription{itemsep=0pt,partopsep=0pt,parsep=\parskip,topsep=5pt}
\setlength\belowdisplayskip{2pt}
\setlength\abovedisplayskip{2pt}

% 左右配对符号
\newcommand{\br}[1]{\!\left(#1\right)} % 括号
\newcommand{\cbr}[1]{\left\{#1\right\}} % 大括号
\newcommand{\abr}[1]{\left<#1\right>} % 尖括号(内积)
\newcommand{\bbr}[1]{\left[#1\right]} % 中括号
\newcommand{\abbr}[1]{\left(#1\right]} % 左开右闭区间
\newcommand{\babr}[1]{\left[#1\right)} % 左闭右开区间
\newcommand{\abs}[1]{\left|#1\right|} % 绝对值
\newcommand{\norm}[1]{\left\|#1\right\|} % 范数
\newcommand{\floor}[1]{\left\lfloor#1\right\rfloor} % 下取整
\newcommand{\ceil}[1]{\left\lceil#1\right\rceil} % 上取整
% 常用数集简写
\newcommand{\R}{\mathbb{R}} % 实数域
\newcommand{\N}{\mathbb{N}} % 自然数集
\newcommand{\Z}{\mathbb{Z}} % 整数集
\newcommand{\C}{\mathbb{C}} % 复数域
\newcommand{\F}{\mathbb{F}} % 一般数域
\newcommand{\kfield}{\Bbbk} % 域
\newcommand{\K}{\mathbb{K}} % 域
\newcommand{\Q}{\mathbb{Q}} % 有理数域
\newcommand{\Pprime}{\mathbb{P}} % 全体素数,或概率
% 范畴记号
\newcommand{\Ccat}{\mathsf{C}}
\newcommand{\Grp}{\mathsf{Grp}} % 群范畴
\newcommand{\Ab}{\mathsf{Ab}} % 交换群范畴
\newcommand{\Ring}{\mathsf{Ring}} % (含幺)环范畴
\newcommand{\Set}{\mathsf{Set}} % 集合范畴
\newcommand{\Mod}{\mathsf{Mod}} % 模范畴
\newcommand{\Vect}{\mathsf{Vect}} % 向量空间范畴
\newcommand{\Alg}{\mathsf{Alg}} % 代数范畴
\newcommand{\Comm}{\mathsf{Comm}} % 交换
% 代数集合
\DeclareMathOperator{\Hom}{Hom} % 同态
\DeclareMathOperator{\End}{End} % 自同态
\DeclareMathOperator{\Iso}{Iso} % 同构
\DeclareMathOperator{\Aut}{Aut} % 自同构
\DeclareMathOperator{\Inn}{Inn} % 内自同构
% \DeclareMathOperator{\inv}{Inv}
\DeclareMathOperator{\GL}{GL} % 一般线性群
\DeclareMathOperator{\SL}{SL} % 特殊线性群
\DeclareMathOperator{\GF}{GF} % Galois域
% 正体符号
\renewcommand{\i}{\mathrm{i}} % 本产生无点i
\newcommand{\id}{\mathrm{id}} % 恒等映射
\newcommand{\e}{\mathrm{e}} % 自然常数e
\renewcommand{\d}{\mathrm{d}} % 微分符号,本产生重音符号
\newcommand{\D}{\partial} % 偏导符号
\newcommand{\diff}[2]{\frac{\d #1}{\d #2}}
\newcommand{\Diff}[2]{\frac{\D #1}{\D #2}}
% 运算符(分析)
\DeclareMathOperator{\Arg}{Arg} % 辐角
\DeclareMathOperator{\re}{Re} % 实部
\DeclareMathOperator{\im}{im} % 像,虚部
\DeclareMathOperator{\grad}{grad} % 梯度
\DeclareMathOperator{\lcm}{lcm} % 最小公倍数
\DeclareMathOperator{\sgn}{sgn} % 符号函数
\DeclareMathOperator{\conv}{conv} % 凸包
\DeclareMathOperator{\supp}{supp} % 支撑
\DeclareMathOperator{\Log}{Log} % 广义对数函数
\DeclareMathOperator{\card}{card} % 集合的势
\DeclareMathOperator{\Res}{Res} % 留数
% 运算符(代数,几何,数论)
\newcommand{\Span}{\mathrm{span}} % 张成空间
\DeclareMathOperator{\tr}{tr} % 迹
\DeclareMathOperator{\rank}{rank} % 秩
\DeclareMathOperator{\charfield}{char} % 域的特征
\DeclareMathOperator{\codim}{codim} % 余维度
\DeclareMathOperator{\coim}{coim} % 余维度
\DeclareMathOperator{\coker}{coker} % 余维度
\DeclareMathOperator{\Spec}{Spec} % 谱
\DeclareMathOperator{\diag}{diag} % 谱
\newcommand{\Obj}{\mathrm{Obj}} % 对象类
\newcommand{\Mor}{\mathrm{Mor}} % 态射类
\newcommand{\Cen}{C} % 群/环的中心 或记\mathrm{Cen}
\newcommand{\opcat}{^{\mathrm{op}}}
% 简写
\newcommand{\hyphen}{\textrm{-}}
\newcommand{\ds}{\displaystyle} % 行间公式形式
\newcommand{\ve}{\varepsilon} % 手写体ε
\newcommand{\rev}{^{-1}\!} % 逆
\newcommand{\T}{^{\mathsf{T}}} % 转置
\renewcommand{\H}{^{\mathsf{H}}} % 共轭转置
\newcommand{\adj}{^\lor} % 伴随
\newcommand{\dual}{^\vee} % 对偶
\DeclareMathOperator{\lhs}{LHS}
\DeclareMathOperator{\rhs}{RHS}
\newcommand{\hint}[1]{{\small (#1)}} % 提示
\newcommand{\why}{\textcolor{red}{(Why?)}}
\newcommand{\tbc}{\textcolor{red}{(To be continued...)}} % 未完待续

% 定理环境(随笔记形式更改)
\newtheorem{definition}{定义}
\newtheorem{remark}{注}
\newtheorem{example}{例}
\makeatletter
\@ifclassloaded{article}{
    \newtheorem{theorem}{定理}[section]
}{
    \newtheorem{theorem}{定理}[chapter]
}
\makeatother
\newtheorem{lemma}[theorem]{引理}
\newtheorem{proposition}[theorem]{命题}
\newtheorem{corollary}[theorem]{推论}
\newtheorem{property}[theorem]{性质}