\documentclass[11pt]{article}
% 用ctex显示中文并用fandol主题
\usepackage[fontset=fandol]{ctex}
\setmainfont{CMU Serif} % 能显示大量外文字体
\xeCJKsetup{CJKmath=true} % 数学模式中可以输入中文

% AMS全家桶,\DeclareMathOperator依赖之
\usepackage{amsmath,amssymb,amsthm,amsfonts,amscd}
\usepackage{pgfplots,tikz,tikz-cd} % 用来画交换图
\usepackage{bm,mathrsfs} % 粗体字母(含希腊字母)和\mathscr字体
\everymath{\displaystyle} % 全体公式为行间形式

% 纸张上下左右页边距
\usepackage[a4paper,left=1cm,right=1cm,top=1.5cm,bottom=1.5cm]{geometry}
% 生成书签和目录上的超链接
\usepackage[colorlinks=true,linkcolor=blue,filecolor=blue,urlcolor=blue,citecolor=cyan]{hyperref}
% 各种列表环境的行距
\usepackage{enumitem}
\setenumerate[1]{itemsep=0pt,partopsep=0pt,parsep=\parskip,topsep=0pt}
\setenumerate[2]{itemsep=0pt,partopsep=0pt,parsep=\parskip,topsep=0pt}
\setenumerate[3]{itemsep=0pt,partopsep=0pt,parsep=\parskip,topsep=0pt}
\setitemize[1]{itemsep=0pt,partopsep=0pt,parsep=\parskip,topsep=5pt}
\setdescription{itemsep=0pt,partopsep=0pt,parsep=\parskip,topsep=5pt}
\setlength\belowdisplayskip{2pt}
\setlength\abovedisplayskip{2pt}

% 左右配对符号
\newcommand{\br}[1]{\!\left(#1\right)} % 括号
\newcommand{\cbr}[1]{\left\{#1\right\}} % 大括号
\newcommand{\abr}[1]{\left<#1\right>} % 尖括号(内积)
\newcommand{\bbr}[1]{\left[#1\right]} % 中括号
\newcommand{\abbr}[1]{\left(#1\right]} % 左开右闭区间
\newcommand{\babr}[1]{\left[#1\right)} % 左闭右开区间
\newcommand{\abs}[1]{\left|#1\right|} % 绝对值
\newcommand{\norm}[1]{\left\|#1\right\|} % 范数
\newcommand{\floor}[1]{\left\lfloor#1\right\rfloor} % 下取整
\newcommand{\ceil}[1]{\left\lceil#1\right\rceil} % 上取整
% 常用数集简写
\newcommand{\R}{\mathbb{R}} % 实数域
\newcommand{\N}{\mathbb{N}} % 自然数集
\newcommand{\Z}{\mathbb{Z}} % 整数集
\newcommand{\C}{\mathbb{C}} % 复数域
\newcommand{\F}{\mathbb{F}} % 一般数域
\newcommand{\kfield}{\Bbbk} % 域
\newcommand{\K}{\mathbb{K}} % 域
\newcommand{\Q}{\mathbb{Q}} % 有理数域
\newcommand{\Pprime}{\mathbb{P}} % 全体素数,或概率
% 范畴记号
\newcommand{\Ccat}{\mathsf{C}}
\newcommand{\Grp}{\mathsf{Grp}} % 群范畴
\newcommand{\Ab}{\mathsf{Ab}} % 交换群范畴
\newcommand{\Ring}{\mathsf{Ring}} % (含幺)环范畴
\newcommand{\Set}{\mathsf{Set}} % 集合范畴
\newcommand{\Mod}{\mathsf{Mod}} % 模范畴
\newcommand{\Vect}{\mathsf{Vect}} % 向量空间范畴
\newcommand{\Alg}{\mathsf{Alg}} % 代数范畴
\newcommand{\Comm}{\mathsf{Comm}} % 交换
% 代数集合
\DeclareMathOperator{\Hom}{Hom} % 同态
\DeclareMathOperator{\End}{End} % 自同态
\DeclareMathOperator{\Iso}{Iso} % 同构
\DeclareMathOperator{\Aut}{Aut} % 自同构
\DeclareMathOperator{\Inn}{Inn} % 内自同构
% \DeclareMathOperator{\inv}{Inv}
\DeclareMathOperator{\GL}{GL} % 一般线性群
\DeclareMathOperator{\SL}{SL} % 特殊线性群
\DeclareMathOperator{\GF}{GF} % Galois域
% 正体符号
\renewcommand{\i}{\mathrm{i}} % 本产生无点i
\newcommand{\id}{\mathrm{id}} % 恒等映射
\newcommand{\e}{\mathrm{e}} % 自然常数e
\renewcommand{\d}{\mathrm{d}} % 微分符号,本产生重音符号
\newcommand{\D}{\partial} % 偏导符号
\newcommand{\diff}[2]{\frac{\d #1}{\d #2}}
\newcommand{\Diff}[2]{\frac{\D #1}{\D #2}}
% 运算符(分析)
\DeclareMathOperator{\Arg}{Arg} % 辐角
\DeclareMathOperator{\re}{Re} % 实部
\DeclareMathOperator{\im}{im} % 像,虚部
\DeclareMathOperator{\grad}{grad} % 梯度
\DeclareMathOperator{\lcm}{lcm} % 最小公倍数
\DeclareMathOperator{\sgn}{sgn} % 符号函数
\DeclareMathOperator{\conv}{conv} % 凸包
\DeclareMathOperator{\supp}{supp} % 支撑
\DeclareMathOperator{\Log}{Log} % 广义对数函数
\DeclareMathOperator{\card}{card} % 集合的势
\DeclareMathOperator{\Res}{Res} % 留数
% 运算符(代数,几何,数论)
\newcommand{\Span}{\mathrm{span}} % 张成空间
\DeclareMathOperator{\tr}{tr} % 迹
\DeclareMathOperator{\rank}{rank} % 秩
\DeclareMathOperator{\charfield}{char} % 域的特征
\DeclareMathOperator{\codim}{codim} % 余维度
\DeclareMathOperator{\coim}{coim} % 余维度
\DeclareMathOperator{\coker}{coker} % 余维度
\DeclareMathOperator{\Spec}{Spec} % 谱
\DeclareMathOperator{\diag}{diag} % 谱
\newcommand{\Obj}{\mathrm{Obj}} % 对象类
\newcommand{\Mor}{\mathrm{Mor}} % 态射类
\newcommand{\Cen}{C} % 群/环的中心 或记\mathrm{Cen}
\newcommand{\opcat}{^{\mathrm{op}}}
% 简写
\newcommand{\hyphen}{\textrm{-}}
\newcommand{\ds}{\displaystyle} % 行间公式形式
\newcommand{\ve}{\varepsilon} % 手写体ε
\newcommand{\rev}{^{-1}\!} % 逆
\newcommand{\T}{^{\mathsf{T}}} % 转置
\renewcommand{\H}{^{\mathsf{H}}} % 共轭转置
\newcommand{\adj}{^\lor} % 伴随
\newcommand{\dual}{^\vee} % 对偶
\DeclareMathOperator{\lhs}{LHS}
\DeclareMathOperator{\rhs}{RHS}
\newcommand{\hint}[1]{{\small (#1)}} % 提示
\newcommand{\why}{\textcolor{red}{(Why?)}}
\newcommand{\tbc}{\textcolor{red}{(To be continued...)}} % 未完待续

% 定理环境(随笔记形式更改)
\newtheorem{definition}{定义}
\newtheorem{remark}{注}
\newtheorem{example}{例}
\makeatletter
\@ifclassloaded{article}{
    \newtheorem{theorem}{定理}[section]
}{
    \newtheorem{theorem}{定理}[chapter]
}
\makeatother
\newtheorem{lemma}[theorem]{引理}
\newtheorem{proposition}[theorem]{命题}
\newtheorem{corollary}[theorem]{推论}
\newtheorem{property}[theorem]{性质}

\title{矩阵分析思考题及答案 (编纂中)}
\author{章亦流 A24201011}
\date{\today}

\begin{document}
\maketitle
\tableofcontents
\section{矩阵范数}
\paragraph{思考题1.1}$x\in \C^n$,则$\lim_{p\to\infty}\norm{x}_p=\norm{x}_\infty$.

\paragraph{思考题1.2}$A\in \C^{m\times n}, \norm{A}_{M_2}=\sqrt{\sum_{i,j} \abs{a_{ij}}}$是矩阵范数.

\paragraph{思考题1.3}$A\in M_n(\C), \norm{A}_{M'_\infty}=n\cdot \max_{i,j}\abs{a_{ij}}$是矩阵范数.

\paragraph{思考题1.4}$\norm{\cdot}_a$是向量范数,$\norm{\cdot}_b$是矩阵范数,若
$$\forall x\in \C^{n}, \forall A\in \C^{m\times n}: \norm{Ax}_a\leq \norm{A}_b\norm{x}_a,$$
则称范数$\norm{\cdot}_a,\norm{\cdot}_b$相容.证明向量2-范数$\norm{x}_{\ell_2}=\sqrt{\sum_{i=1}^{n}\abs{x_i}^2}$与矩阵范数$\norm{A}_{M_2}=\sqrt{\sum_{i,j} \abs{a_{ij}}}$相容.

\paragraph{思考题1.5}在$M_n(\C)$上证明$\norm{A}_\infty=\max_{x\neq 0}\frac{\norm{Ax}_{\infty}}{\norm{x}_\infty}$就是行和范数$\norm{A}_{r}=\max_{i}\sum_{j=1}^{n}\abs{a_{ij}}$.

\paragraph{思考题1.6}对于$A\in M_n(\C)$,$\lim_{k\to\infty}A^k=O \iff \rho(A)<1$,其中$\rho(A)=\max_{\lambda\in \Spec A}\abs{\lambda}$是谱半径.

\section{矩阵分解}
\paragraph{思考题2.1}从几何直观的角度,用寻找空间中基底的方式,给出奇异值分解的另一个证明.
\begin{quotation}
    \textbf{Theorem (奇异值分解, SVD)} $A\in \C^{m\times n}$,存在酉矩阵$U\in M_m(\C), V\in M_n(\C)$使得$UAV=\mathrm{diag}(s_1,\dots,s_p)$,其中$s_1\geq\dots\geq s_p$是$A$的全体奇异值,$p=\min(m,n)$.
\end{quotation}
\paragraph{思考题2.2}证明QR分解.
\begin{quotation}
    \textbf{Theorem (QR分解)} 对于任意$n$阶矩阵$A$,存在酉矩阵$Q$和上三角矩阵$R$,使得$A=QR$.
\end{quotation}

\section{Hermite矩阵}
\paragraph{思考题3.1}证明极大-极小原理的第二个等式.
\begin{quotation}
    \textbf{Theorem (极大-极小原理, Courant-Fischer定理)} $A$是Hermite矩阵,则$$\lambda_k(A)=\max_{\substack{S\subset \C^n\\ \dim S=k}}\min_{x\in S}R(x)=\max_{\substack{T\subset \C^n\\ \dim T=n-k+1}}\max_{x\in S}R(x), \text{where } R(x)=\frac{x^*Ax}{x^*x}.$$
\end{quotation}
\paragraph{思考题3.2}证明: $n$阶单圈图中谱半径最大的图是星图连接两个1度点形成的图.
\begin{proof}
    
\end{proof}
\paragraph{思考题3.3}给出所有满足$\rho(G)\leq \sqrt{2}$(或2)的图$G$.
\begin{proof}
    下仅考虑连通图,因为图的谱半径即所有连通分支的最大谱半径,因此若干个满足条件的连通图的不交并仍然满足条件.由于连通图总有$m\geq n-1$,故$\rho(G)\geq d(G)=2m/n\geq 2-2/n$.因此$\rho(G)\leq \sqrt{2}$时有$2-2/n\leq \sqrt{2}$,解得$n\leq 3$,可枚举出满足条件的图仅有$P_3,K_3,K_2,K_1$.

    $\rho(G)\leq 2$的图被称作Smith图\footnote{J.H. Smith, \textit{Some properties of the spectrum of a graph}, in: Combinatorial Structures and Their Applications, Gordon and Breach, New York -- London -- Paris, 1970, pp. 403--406.}\footnote{\href{https://doi.org/10.1016/j.laa.2016.11.020}{Dragoš Cvetković, Irena M. Jovanović, \textit{Constructing graphs with given spectrum and the spectral radius at most 2}, in: Linear Algebra and its Applications, Volume 515, 2017, pp. 255-274,}}.下分类讨论:
    \begin{enumerate}
        \item 带圈图$G$是Smith图,其含圈$C$,故$\rho(G)\geq\rho(C)=2$,其仅在$G=C$时取等,即带圈Smith图仅能为圈图.
        \item 树$T$是Smith图,且$\Delta(T)\leq 2$.此时$T$仅能为路图,而$\rho(P_n)=2\cos\frac{\pi}{n+1}<2$,故路图均为Smith图.
        \item 树$T$是Smith图,且$\Delta(T)\geq 4$.此时必然有$K_{1,4}\subset T$,即$\rho(T)\geq \rho(K_{1,4})=2$,由取等条件知$T=K_{1,4}$.
        \item 树$T$是Smith图,$\Delta(T)=3$,且存在两点$x,y$度数为3.由连通性知有$(x,y)$-路$P$, 从而记图$W_{n'}=T[V(P)\cup N(x)\cup N(y)]$,其为路$P_{n'}(n'\geq 2)$的首尾两点各与路外两点连边得到的$n'+4$点图. 再记$Z_{n'}$为$W_{n'}$删去一个末端点,从而由Heilbronner公式,对任意$n\geq 2$有
        $$\begin{aligned}
            \phi_{W_n}(x)&=x\phi_{Z_n}(x)-x\phi_{Z_{n-2}}(x)\\
            &=x(x\phi_{P_{n+2}}(x)-x\phi_{P_n}(x))-x(x\phi_{P_n}(x)-x\phi_{P_{n-2}}(x))\\
            &=x^2(\phi_{P_{n+2}}(x)+\phi_{P_{n-2}}(x)-2\phi_{P_n}(x))=x^2(x^2-4)\phi_{P_n}(x)
        \end{aligned}$$
        从而知$\rho(W_{n'})=2\leq \rho(T)$, 由取等条件知$T=W_{n-4}$.
        \item 树$T$是Smith图,$\Delta(T)=3$,且仅有一点$x$度数为3.视$T$为以$x$为根节点的根树,则$x$分别连接三条路$P_{n_1},P_{n_2},P_{n_3}$.记$T=T_{n_1,n_2,n_3}$,计算可得$\rho(T_{5,2,1})=\rho(T_{3,3,1})=\rho(T_{2,2,2})=2$,因此满足条件的$T$仅能为$T_{5,2,1},T_{3,3,1}$或$T_{2,2,2}$的子图.
    \end{enumerate}
    综上所述, Smith图有且仅有$C_n, W_n, K_{1,4}, T_{5,2,1}, T_{3,3,1}, T_{2,2,2}$及其子图.
\end{proof}
\paragraph{思考题3.4}定义图$G$的无符号Laplace矩阵$Q(G)=D(G)+A(G)$,其中$A(G)$是$G$的邻接矩阵,$D(G)=\mathrm{diag}(d(v_1),\dots,d(v_n))$是$G$的度数矩阵.下记$Q(G)$有特征值$\lambda_1(G)\geq\dots\geq\lambda_n(G)$,对于$\forall e\in E(G)$,证明:
\begin{enumerate}
    \item $\lambda_1(G)\geq \lambda_1(G-e)\geq \lambda_2(G)\geq \dots \geq \lambda_{n-1}(G)\geq \lambda_{n-1}(G-e)\geq \lambda_n(G)\geq \lambda_n(G-e)$.
    \item $\forall k\in [n-2], \lambda_k(G)\geq \lambda_k(G-e)\geq \lambda_{k+2}(G)$.
\end{enumerate}
\begin{proof}
    记$e=v_i v_j$,则$Q(G)=Q(G-e)+E_{ij}$,其中$E_{ij}$的第$(i,i), (j,j), (i,j), (j,i)$个元素为1,其余元素为0.容易计算得$\lambda_1(E_{ij})=2, \lambda_2(E_{ij})=\dots=\lambda_n(E_{ij})=0$.由Weyl不等式可知,$\forall k\in [n]$,
    $$\max_{r+s=k+n}\cbr{\lambda_r(G-e)+\lambda_s(E_{ij})}\leq \lambda_k(G)\leq \min_{r+s=k+1}\cbr{\lambda_r(G-e)+\lambda_s(E_{ij})}$$
    对第一个不等式取$r=k, s=n$,有$\lambda_k(G)\geq \lambda_k(G-e)+\lambda_n(E_{ij})=\lambda_k(G-e)$; 若$k\geq 2$, 对第二个不等式取$r=k-1, s=2$,则$\lambda_k(G)\leq \lambda_{k-1}(G-e)+\lambda_2(E_{ij})=\lambda_{k-1}(G-e)$.综上得证(1).而$\forall k\in [n-2], \lambda_k(G-e)\geq \lambda_{k+1}(G)\geq \lambda_{k+2}(G)$,因此得证(2).(怀疑(2)的题目有误.)
\end{proof}

\section{非负矩阵}

\end{document}

\paragraph{思考题1.1}
\paragraph{思考题1.2}
\paragraph{思考题1.3}
\paragraph{思考题1.4}
\paragraph{思考题1.5}
\paragraph{思考题1.6}