\documentclass[11pt]{article}
% 用ctex显示中文并用fandol主题
\usepackage[fontset=fandol]{ctex}
\setmainfont{CMU Serif} % 能显示大量外文字体
\xeCJKsetup{CJKmath=true} % 数学模式中可以输入中文

% AMS全家桶,\DeclareMathOperator依赖之
\usepackage{amsmath,amssymb,amsthm,amsfonts,amscd}
\usepackage{pgfplots,tikz,tikz-cd} % 用来画交换图
\usepackage{bm,mathrsfs} % 粗体字母(含希腊字母)和\mathscr字体
\everymath{\displaystyle} % 全体公式为行间形式

% 纸张上下左右页边距
\usepackage[a4paper,left=1cm,right=1cm,top=1.5cm,bottom=1.5cm]{geometry}
% 生成书签和目录上的超链接
\usepackage[colorlinks=true,linkcolor=blue,filecolor=blue,urlcolor=blue,citecolor=cyan]{hyperref}
% 各种列表环境的行距
\usepackage{enumitem}
\setenumerate[1]{itemsep=0pt,partopsep=0pt,parsep=\parskip,topsep=0pt}
\setenumerate[2]{itemsep=0pt,partopsep=0pt,parsep=\parskip,topsep=0pt}
\setenumerate[3]{itemsep=0pt,partopsep=0pt,parsep=\parskip,topsep=0pt}
\setitemize[1]{itemsep=0pt,partopsep=0pt,parsep=\parskip,topsep=5pt}
\setdescription{itemsep=0pt,partopsep=0pt,parsep=\parskip,topsep=5pt}
\setlength\belowdisplayskip{2pt}
\setlength\abovedisplayskip{2pt}

% 左右配对符号
\newcommand{\br}[1]{\!\left(#1\right)} % 括号
\newcommand{\cbr}[1]{\left\{#1\right\}} % 大括号
\newcommand{\abr}[1]{\left<#1\right>} % 尖括号(内积)
\newcommand{\bbr}[1]{\left[#1\right]} % 中括号
\newcommand{\abbr}[1]{\left(#1\right]} % 左开右闭区间
\newcommand{\babr}[1]{\left[#1\right)} % 左闭右开区间
\newcommand{\abs}[1]{\left|#1\right|} % 绝对值
\newcommand{\norm}[1]{\left\|#1\right\|} % 范数
\newcommand{\floor}[1]{\left\lfloor#1\right\rfloor} % 下取整
\newcommand{\ceil}[1]{\left\lceil#1\right\rceil} % 上取整
% 常用数集简写
\newcommand{\R}{\mathbb{R}} % 实数域
\newcommand{\N}{\mathbb{N}} % 自然数集
\newcommand{\Z}{\mathbb{Z}} % 整数集
\newcommand{\C}{\mathbb{C}} % 复数域
\newcommand{\F}{\mathbb{F}} % 一般数域
\newcommand{\kfield}{\Bbbk} % 域
\newcommand{\K}{\mathbb{K}} % 域
\newcommand{\Q}{\mathbb{Q}} % 有理数域
\newcommand{\Pprime}{\mathbb{P}} % 全体素数,或概率
% 范畴记号
\newcommand{\Ccat}{\mathsf{C}}
\newcommand{\Grp}{\mathsf{Grp}} % 群范畴
\newcommand{\Ab}{\mathsf{Ab}} % 交换群范畴
\newcommand{\Ring}{\mathsf{Ring}} % (含幺)环范畴
\newcommand{\Set}{\mathsf{Set}} % 集合范畴
\newcommand{\Mod}{\mathsf{Mod}} % 模范畴
\newcommand{\Vect}{\mathsf{Vect}} % 向量空间范畴
\newcommand{\Alg}{\mathsf{Alg}} % 代数范畴
\newcommand{\Comm}{\mathsf{Comm}} % 交换
% 代数集合
\DeclareMathOperator{\Hom}{Hom} % 同态
\DeclareMathOperator{\End}{End} % 自同态
\DeclareMathOperator{\Iso}{Iso} % 同构
\DeclareMathOperator{\Aut}{Aut} % 自同构
\DeclareMathOperator{\Inn}{Inn} % 内自同构
% \DeclareMathOperator{\inv}{Inv}
\DeclareMathOperator{\GL}{GL} % 一般线性群
\DeclareMathOperator{\SL}{SL} % 特殊线性群
\DeclareMathOperator{\GF}{GF} % Galois域
% 正体符号
\renewcommand{\i}{\mathrm{i}} % 本产生无点i
\newcommand{\id}{\mathrm{id}} % 恒等映射
\newcommand{\e}{\mathrm{e}} % 自然常数e
\renewcommand{\d}{\mathrm{d}} % 微分符号,本产生重音符号
\newcommand{\D}{\partial} % 偏导符号
\newcommand{\diff}[2]{\frac{\d #1}{\d #2}}
\newcommand{\Diff}[2]{\frac{\D #1}{\D #2}}
% 运算符(分析)
\DeclareMathOperator{\Arg}{Arg} % 辐角
\DeclareMathOperator{\re}{Re} % 实部
\DeclareMathOperator{\im}{im} % 像,虚部
\DeclareMathOperator{\grad}{grad} % 梯度
\DeclareMathOperator{\lcm}{lcm} % 最小公倍数
\DeclareMathOperator{\sgn}{sgn} % 符号函数
\DeclareMathOperator{\conv}{conv} % 凸包
\DeclareMathOperator{\supp}{supp} % 支撑
\DeclareMathOperator{\Log}{Log} % 广义对数函数
\DeclareMathOperator{\card}{card} % 集合的势
\DeclareMathOperator{\Res}{Res} % 留数
% 运算符(代数,几何,数论)
\newcommand{\Span}{\mathrm{span}} % 张成空间
\DeclareMathOperator{\tr}{tr} % 迹
\DeclareMathOperator{\rank}{rank} % 秩
\DeclareMathOperator{\charfield}{char} % 域的特征
\DeclareMathOperator{\codim}{codim} % 余维度
\DeclareMathOperator{\coim}{coim} % 余维度
\DeclareMathOperator{\coker}{coker} % 余维度
\DeclareMathOperator{\Spec}{Spec} % 谱
\DeclareMathOperator{\diag}{diag} % 谱
\newcommand{\Obj}{\mathrm{Obj}} % 对象类
\newcommand{\Mor}{\mathrm{Mor}} % 态射类
\newcommand{\Cen}{C} % 群/环的中心 或记\mathrm{Cen}
\newcommand{\opcat}{^{\mathrm{op}}}
% 简写
\newcommand{\hyphen}{\textrm{-}}
\newcommand{\ds}{\displaystyle} % 行间公式形式
\newcommand{\ve}{\varepsilon} % 手写体ε
\newcommand{\rev}{^{-1}\!} % 逆
\newcommand{\T}{^{\mathsf{T}}} % 转置
\renewcommand{\H}{^{\mathsf{H}}} % 共轭转置
\newcommand{\adj}{^\lor} % 伴随
\newcommand{\dual}{^\vee} % 对偶
\DeclareMathOperator{\lhs}{LHS}
\DeclareMathOperator{\rhs}{RHS}
\newcommand{\hint}[1]{{\small (#1)}} % 提示
\newcommand{\why}{\textcolor{red}{(Why?)}}
\newcommand{\tbc}{\textcolor{red}{(To be continued...)}} % 未完待续

% 定理环境(随笔记形式更改)
\newtheorem{definition}{定义}
\newtheorem{remark}{注}
\newtheorem{example}{例}
\makeatletter
\@ifclassloaded{article}{
    \newtheorem{theorem}{定理}[section]
}{
    \newtheorem{theorem}{定理}[chapter]
}
\makeatother
\newtheorem{lemma}[theorem]{引理}
\newtheorem{proposition}[theorem]{命题}
\newtheorem{corollary}[theorem]{推论}
\newtheorem{property}[theorem]{性质}

\title{高等代数作业答案 (编纂中)}
\author{章亦流 A24201011}
\date{\today}

\begin{document}
\maketitle
\tableofcontents

% \section{第一章 行列式}
% \subsection{数域}
% \subsection{排列}
% \subsection{$n$阶行列式}
% \subsection{行列式的性质与展开}
% \subsection{行列式的计算}
% \subsection{Cramer法则}
% \subsection{复习题1}

% \section{第二章 矩阵}
% \subsection{矩阵的定义与运算}
% \subsection{矩阵的初等变换与秩}
% \subsection{可逆矩阵}
% \subsection{分块矩阵}
% \subsection{Gauss消元法}
% \subsection{复习题2}

% \section{第三章 线性空间}
% \subsection{线性空间的定义和性质}
% \subsection{向量组的线性相关性}
% \subsection{基与坐标}
% \subsection{线性子空间}
% \subsection{子空间的交、和与直和}
% \subsection{商空间}
% \subsection{复习题3}

% \section{第四章 线性映射}
% \subsection{线性映射的定义与矩阵}
% \subsection{线性空间的同构}
% \subsection{线性映射的像与核}
% \subsection{线性变换及其矩阵}
% \subsection{不变子空间}
% \subsection{复习题 4}

\addtocounter{section}{4}
\section{第五章 多项式}
\subsection{一元多项式}
\paragraph{5.1.1}$f,g\in\F[x]$,证明$fg=0\iff f$和$g$中至少一个是0.
\begin{proof}[证明一]
    $\impliedby$显然.$\implies:$若$f,g$均非零,则两者的首项系数之积非零,从而$fg\neq 0$.
\end{proof}
\begin{proof}[证明二]
    $\implies:$按逐项系数递推,记
    $$f(x)=\sum_{i=0}^{m}a_ix^i, g(x)=\sum_{j=0}^{n}b_jx^j, f(x)g(x)=\sum_{t=0}^{m+n}\br{\sum_{i+j=t}a_ib_j}x^t=0$$
    从而$c_t=\sum_{i+j=t}a_ib_j=0, \forall t=0,\cdots,m+n$.若$f=0$则命题得证; 若$f\neq 0$,则$a_m\neq 0$,而$c_{m+n}=a_mb_n=0, b_n=0$.

    下证明$b_{n-r}=0, r=0,\cdots,n$. 对$r$归纳, $r=0$时已证.若$<r$的情形已证,即
    $$b_{n-0}=b_{n-1}=\cdots=b_{n-r+1}=0$$
    则
    $$c_{m+n-r}=a_mb_{n-r}+a_{m-1}b_{n-r+1}+\cdots+a_{m-r}b_{n}=a_mb_{n-r}=0$$
    从而$b_{n-r}=0$,得证.
\end{proof}
\paragraph{5.1.2}$f,g,h\in \F[x]$,若$f\neq 0$,则$fg=fh\iff g=h$.
\begin{proof}
    $fg=fh\iff f(g-h)=0$,而$f\neq 0$,由上题知$g-h=0, g=h$.
\end{proof}
\paragraph{5.1.3}对于$f\in\R[x], f\neq 0$满足$f(x^2)=f^2(x)$,求多项式$f(x)$.
\begin{proof}[证明一]
    记$f(x)=\sum_{i=0}^{n}a_ix^i, a_i\in \R, a_n\neq 0$. 取$m=\max\cbr{k\mid a_k\neq 0, k=0,\cdots, n-1}$, 即除$a_nx^n$外最高非零项的次数,则
    $$f(x^2)=\sum_{i=0}^{n}a_i x^{2i}=a_nx^{2n}+a_mx^{2m}+\cdots, f^2(x)=\sum_{t=0}^{2n}\br{\sum_{i+j=t}a_ia_j}x^t=a_n^2 x^{2n}+2a_na_m x^{n+m}+\cdots$$
    比较系数,$a_n=a_n^2, a_n=1$.而$n+m>2m$,故$x^{n+m}$项系数$2a_na_m=0, a_m=0$.这与$m$定义矛盾,故$m$不存在,即$a_0=\cdots=a_{n-1}=0, f(x)=x^n$.
\end{proof}
\begin{proof}[证明二]
    同上记号且易证$a_n=1$.对$f(x^2),f^2(x)$展开有:
    $$f(x^2)=\sum_{i=0}^{n}a_i x^{2i}, f^2(x)=\sum_{t=0}^{2n}\br{\sum_{i+j=t}a_ia_j}x^t$$
    逐项比较系数,可得:
    $$a_k=\sum_{i+j=2k}a_ia_j,\qquad 0=\sum_{i+j=2k+1}a_ia_j,\qquad \forall k=0,\cdots,n$$
    从而$0=2a_na_{n-1}, a_{n-1}=0$.下证$a_{n-r}=0, r=1,\cdots,n$. 对$r$归纳, $r=1$时已证.假设$<r$的情形已证,即$a_{n-1}=\cdots=a_{n-r+1}=0$时:若$r$为偶数,则
    $$0=a_{n-r/2}=\sum_{i+j=2n-r}a_ia_j=2a_na_{n-r}$$
    若$r$为奇数,则取$k=n-(r+1)/2$,
    $$0=\sum_{i+j=2n-r}a_ia_j=2a_na_{n-r}$$
    可知总有$a_{n-r}=0$,从而得证,即$f(x)=x^n$.
\end{proof}
\paragraph{5.1.4}$f,g,h\in \R[x]$,证明若$f^2(x)=xg^2(x)+xh^2(x)$,则$f=g=h=0$.
\begin{proof}[证明一]
    若$f\neq 0$则$\deg f^2=2\deg f$为偶数,且此时$g^2+h^2\neq 0$.由于$g^2$与$h^2$的首项系数均为正数,故两者和也为正数,故$\deg(g^2+h^2)=\max(\deg g^2,\deg h^2)$,从而有
    $$2\deg f=\deg f^2=\deg\br{xg^2(x)+xh^2(x)}=2\max(\deg g,\deg h)+1$$
    左端为偶数,右端为奇数,矛盾,从而$f=0,g^2+h^2=0,g=h=0$.
\end{proof}
\begin{proof}[证明二]
    若$g,h$中至少有一个非零,取$g\neq 0$,则$\exists c\in\R, g(c)\neq 0$,故$g^2(c)+h^2(c)>0, g^2+h^2\neq 0$.而$\deg f^2$为偶数,$\deg\br{xg^2(x)+xh^2(x)}$为奇数,矛盾.故$g=h=0,f=0$.
\end{proof}
\paragraph{5.1.5}在$\C[x]$中找一组不全为0的多项式$f,g,h$使得$f^2(x)=xg^2(x)+xh^2(x)$.
\begin{proof}
    $f(x)=0, g(x)=\i, h(x)=1$.
\end{proof}
\begin{proof}[通解]
    由于$x\mid f^2(x)$,则$x\mid f(x)$. 记$f(x)=xq(x)$, 有
    $$xq^2(x)=g^2(x)+h^2(x)=(g(x)+\i h(x))(g(x)-\i h(x))$$
    不失一般性地认为$g,h$互素,因为上式等价于
    $$x\br{\frac{q(x)}{(g(x),h(x))}}^2=\br{\frac{g(x)}{(g(x),h(x))}}^2+\br{\frac{h(x)}{(g(x),h(x))}}^2$$
    另一方面,$x\mid (g(x)+\i h(x))(g(x)-\i h(x))$,不失一般性地认为$x\mid g(x)+\i h(x)$.
    
    将$q(x)$分解为不可约多项式的乘积,即$q=p_1p_2\cdots p_m$,则
    $$g(x)+\i h(x)=xp_1^2(x)\cdots p_{s}^2(x) p_{s+1}(x)\cdots p_{t}(x), g(x)-\i h(x)=p_{s+1}(x)\cdots p_{t}(x)p_{t+1}^2(x)\cdots p_{m}^2(x)$$
    记
    $$a(x)=p_1(x)\cdots p_{s}(x), b(x)=p_{t+1}(x)\cdots p_{m}(x), d(x)=p_{s+1}(x)\cdots p_{t}(x)$$
    则$q=abd, g+\i h=x a^2d, g-\i h=db^2, (g+\i h, g-\i h)=d(a,b)^2$.而
    $(g+\i h, g-\i h)=(g+\i h, 2g)=(g,h)=1$,因此$d=(a,b)=1, g+\i h=x a^2, g-\i h=b^2$,解得:
    $$f=xq=xab, g=\frac{x a^2+b^2}{2}, h=\frac{x a^2-b^2}{2\i}$$
    最后回代$g,h$不互素的情况,得到通解:对于$\forall a,b\in \C[x],$上式为通解.
\end{proof}

\newpage
\subsection{整除}
\paragraph{5.2.1}求下列$f(x)$除以$g(x)$的商式$q(x)$与余式$r(x)$:
\begin{enumerate}
    \item $f(x)=5x^4+3x^3+2x^2+x-1, g(x)=x^2+2x-2$;
    \item $f(x)=6x+3x^4-4x^3, g(x)=x+2$.
\end{enumerate}
\begin{proof}
    \begin{enumerate}
        \item $q(x)=5 x^2-7x+26, r(x)=-65x+51$.
        \item $q(x)=3 x^3-10  x^2+20  x-34, r(x)=68$.
    \end{enumerate}
\end{proof}
\paragraph{5.2.2}求$f(x)$按$x-c$幂的展开式,即写成$f(x)=\sum_{k=0}^n a_k (x-c)^k$的形式:
\begin{enumerate}
    \item $f(x)=x^5, c=1$;
    \item $f(x)=x^3-10x^2+13, c=-2$.
\end{enumerate}
\begin{proof}
    \begin{enumerate}
        \item $(x-1)^5+5(x-1)^4+10(x-1)^3+10(x-1)^2+5(x-1)+1$.
        \item $(x+2)^3-16(x+2)^2+52(x+2)-35$.
    \end{enumerate}
\end{proof}
\paragraph{5.2.3}问参数$m,n,p$满足什么条件时有
\begin{enumerate}
    \item $x^2-2x+1\mid x^4-5x^3+11x^2+mx+n$;
    \item $x^2-2mx+2\mid x^4+3x^2+mx+n$;
    \item $x^2+m-1\mid x^3+nx+p$;
    \item $x^2+mx+1\mid x^4+nx^2+p$.
\end{enumerate}
\begin{proof}
    \begin{enumerate}
        \item 要求除法余式$r(x)=(m+11)x+(n-4)=0$,即$m=-11, n=4$.
        \item 要求除法余式$r(x)=(8 m^3-m) x-8 m^2+n-2=0$,解得$m=0, n=2$或$m=\pm\frac{\sqrt{2}}{4}, n=3$.
        \item 要求除法余式$r(x)=(1-m+n)x+p=0$,即$m=n+1, p=0$.
        \item 要求除法余式$r(x)=(-m^3-m n+2 m)x -m^2-n+p+1=0$,解得$m=0, n=p+1$或$m^2+n=2, p=1$.
    \end{enumerate}
\end{proof}
\paragraph{5.2.4}求$u(x),v(x)$使得$uf+vg=(f,g)$.
\begin{enumerate}
    \item $f(x)=x^4+3x^3-x^2-4x-3, g(x)=3x^3+10x^2+2x-3$;
    \item $f(x)=x^4-10x^2+1, g(x)=x^4-4\sqrt{2}x^3+6x^2+4\sqrt{2}x+1$;
    \item $f(x)=x^4-x^3-4x^2+4x+1, g(x)=x^2-x-1$.
\end{enumerate}
\begin{proof}
    \begin{enumerate}
        \item $u(x)=\frac{3}{5}x-1, v(x)=-\frac{1}{5}x^2+\frac{2}{5}x$.
        \item $u(x)=-\frac{\sqrt{2}}{8}x+\frac{1}{2}, v(x)=\frac{\sqrt{2}}{8}x+\frac{1}{2}$.
        \item $u(x)=-x-1, v(x)=x^3+x^2-3x-2$.
    \end{enumerate}
    (答案均不唯一.)
\end{proof}
\paragraph{5.2.5}设$f(x)=x^3+(t+1)x^2+2x+2u$与$g(x)=x^3+tx^2+u$的最大公因式为二次多项式,求$t,u$.
\begin{proof}
    考虑带余除法$f=qg+r$,比较次数与系数可知$q(x)=1$,故$r(x)=f(x)-g(x)=x^2+2x+u$.继续辗转相除得到$g=q_1r+r_1$,其中$\deg r_1<\deg r=2$,而$(f,g)\mid r_1$,因此$r_1=0, g=q_1r$.比较系数知$q_1$为首项系数为1的一次多项式$(x-a)$,因此有
    $$g(x)=x^3+tx^2+u=(x-a)(x^2+2x+u)=x^3+(2-a)x^2+(u-2a)x-au$$
    比较系数可得
    $$t=2-a,\quad 0=u-2a,\quad u=-au$$
    解得$t=2, u=0, a=0$或$t=3, u=-2, a=-1$.
\end{proof}
\paragraph{5.2.6}对于多项式$f,g,d$,若$d\mid f, d\mid g$且存在多项式$u,v$使得$d=uf+vg$,证明$d=(f,g)$.
\begin{proof}
    由$d\mid f, d\mid g$知$d\mid (f,g)$,而$(f,g)\mid uf+vg=d$,因此$d$与$(f,g)$间差一个非零常数,即$d$是$f,g$的一个最大公因数.
\end{proof}
\paragraph{5.2.7}设$f,g\in\F[x]$,证明:
\begin{enumerate}
    \item 若$a,b,c,d\in \F$满足$ad-bc\neq 0$,则$(af+bg,cf+dg)=(f,g)$;
    \item $(f^2,g^2)=(f,g)^2$;
    \item $(f,f+g)=1\iff (f,g)=1$.
\end{enumerate}
\begin{proof}
    首先证明引理:对于任意多项式$q\in \F[x], (f,g)=(f+qg,g)$. 证:由于$(f,g)$整除$f+qg$和$g$,因此$(f,g)\mid (f+qg,g)$,同理$(f+qg,g)\mid (f+qg-qg,g)=(f,g)$,从而两者相等.

    1. 
    $$\br{af+bg,cf+dg}=\br{af+bg,cf+dg-\frac{c}{a}(af+bg)}=\br{af+bg,\frac{ad-bc}{a}g}=(f,g)$$

    2.记$d=(f,g)$,有$f=df_1, g=dg_1, (f_1,g_1)=1, (f^2,g^2)=d^2(f_1^2,g_1^2)$.而$(f_1,g_1)=1\iff (f_1^2,g_1^2)=1$(书上推论5.2.12,或由Bézout定理),从而得证.

    3.由1或引理显然.
\end{proof}
\paragraph{5.2.8}$f,g\in \F[x]$不全为0,且$uf+vg=(f,g)$,证明$(u,v)=1$.
\begin{proof}
    记$f=(f,g)f_1, g=(f,g)g_1$,其中$(f_1,g_1)=1$.从而有$(f,g)=uf+vg=(f,g)(uf_1+vg_1)$,因此$uf_1+vg_1=1$,这等价于$(u,v)=1$.
\end{proof}
\paragraph{5.2.9}设$f_1,\cdots,f_m,g_1,\cdots,g_n\in \F[x]$且$(f_i,g_j)=1 (\forall i\in [m], j\in [n])$,证明$(f_1\cdots f_m, g_1\cdots, g_n)=1$.
\begin{proof}
    首先证明$n=1$的情形,即$\forall i=1,\cdots,m, (f_i,g)=1$则有$(f_1\cdots f_m,g)=1$.对$m$归纳,$m=1$时已证,下设$<m$的情形已得证,而$(f_1\cdots f_{m-1},g)=(f_m,g)=1\iff (f_1\cdots f_m,g)=1$(书上推论5.2.12),从而得证.

    再对原命题考虑,记$f=f_1\cdots f_m$,由上知$(f,g_1)=\cdots=(f,g_n)=1$,从而又有$(f,g_1\cdots g_n)=1$.
\end{proof}
\paragraph{5.2.10}证明定理5.2.16
\begin{quotation}
    \textbf{定理5.2.16} 设$f_1,\cdots,f_k\in \F[x]$不全为0,则$(f_1,\cdots,f_k)$唯一存在,且
    $$(f_1,\cdots,f_k)=((f_1,\cdots,f_{k-1}),f_k)$$
    从而$\exists u_i\in \F[x], i\in [k]$使得
    $$(f_1,\cdots,f_k)=\sum_{i=1}^k u_i f_i$$
\end{quotation}
\begin{proof}
    对$k$归纳,$k=2$时已得证.下设$k\geq 3, <k$的情形已证.设$d_1=(f_1,\cdots,f_{k-1})$,由归纳假设知其唯一确定,且有$v_1f_1+\cdots+v_{k-1}f_{k-1}=d_1$.
    
    首先证明$d=(d_1,f_k)$为$f_1,\cdots,f_k$的最大公因式,从而证明存在性.显然$d\mid d_1\mid f_i (i=1,\cdots,k-1)$且$d\mid f_k$.又对于$f_1,\cdots,f_k$的任意公因式$g$,均有(由归纳假设)
    $$g\mid v_1f_1+\cdots+v_{k-1}f_{k-1}=d_1, g\mid f_k$$
    从而$g\mid (d_1,f_k)=d$,即$d$为最大公因式.

    再证明唯一性:若有多项式$d,d'$均为$f_1,\cdots,f_k$的最大公因式,则$d'\mid d, d\mid d'$,从而相同(差一个非零常数而首项系数均为1).

    最后,由归纳假设有$v_1f_1+\cdots+v_{k-1}f_{k-1}=d_1$,又有$ud_1+vf_k=d$,从而
    $$u(v_1f_1+\cdots+v_{k-1}f_{k-1})+vf_k=uv_1f_1+\cdots+uv_{k-1}f_{k-1}+vf_k=d$$
    综上得证.
\end{proof}
\paragraph{5.2.11}称多项式$m(x)$为多项式$f(x),g(x)$的最小公倍式,若$f\mid m, g\mid m$且$f,g$的任意公倍式是$m$的倍式.记$m=[f,g]$,证明若$f,g$首项系数为1,则$[f,g]=\frac{fg}{(f,g)}$.
\begin{proof}
    记$d=(f,g), m=fg/d, f=df_1, g=dg_1, (f_1,g_1)=1$.从而$m=f_1g=fg_1$,故$f\mid m, g\mid m$.
    
    再设$f,g$的任意公倍式$h=h_1f=h_2g$,有$h=dh_1f_1=dh_2g_1$,从而$h_1f_1=h_2g_1$.而$(f_1,g_1)=1$,因此$f_1\mid h_2, m=df_1g_1\mid dh_2g_1=h$.综上,$m$满足最小公倍式的所有条件,即$m=[f,g]$.
\end{proof}

\paragraph{思考题1}对于$f(x)=\sum_{i=0}^{m}a_ix^i, g(x)=\sum_{i=0}^{n}b_ix^i, h(x)=\sum_{i=0}^{m-n}c_ix^i$,若有$f(x)=g(x)h(x)$,显式表达出$c_i$.
\begin{proof}
    考虑$g(x)$的最低非零次数$r=\min\cbr{i|b_i\neq 0, i=0,1,\cdots,n}$,则$g(x)=\sum_{i=r}^{n}b_ix^i$.又由$a_k=\sum_{i+j=k}b_ic_j$有:
    $$a_{r+k}=\sum_{i+j=r+k}b_ic_j=b_rc_k+\sum_{i=0}^{k-1}b_{r+k-i}c_i, (k=0,\cdots,m-r)$$
    因此有
    $$c_k=\frac{1}{b_r}\br{a_{r+k}-\sum_{i=0}^{k-1}b_{r+k-i}c_i}$$
    其在$r=0$,即$b_0\neq 0$时化为
    $$c_k=\frac{1}{b_0}\br{a_{k}-\sum_{i=0}^{k-1}b_{k-i}c_i}$$
\end{proof}

\newpage
\subsection{因式分解定理}
\paragraph{5.3.1}$x^2+1$在$\Q$上不可约.
\begin{proof}
    由于$x^2+1$在$\R$上的唯一分解式为$(x-\i)(x+\i)$,故其不能被分解为$\Q[x]$中的一次多项式之积,故在$\Q$上不可约.
\end{proof}
\paragraph{5.3.2}判别下列多项式是否有重因式:
\begin{enumerate}
    \item $f(x)=x^4+x^3+2x^2+x+1$,
    \item $f(x)=x^6-3x^5+6x^3-3x^2-3x+2$.
\end{enumerate}
\begin{proof}
    \begin{enumerate}
        \item 辗转相除可得$(f,f')=1$从而无重因式.但辗转相除太过麻烦,有其他方法:
        \begin{itemize}
            \item 注意到$(f,g)=(f+qg,g), \forall q\in \F[x]$,故对第一小问有
        $$\begin{aligned}
            (f,f')&=(x^4+x^3+2x^2+x+1,4x^3+3x^2+4x+1)=\br{x^3+4x^2+3x+4,4x^3+3x^2+4x+1}\\
            &=\br{x^3+4x^2+3x+4,13x^2+8x+15}=\br{11x^2+6x+13,13x^2+8x+15}\\
            &=\br{11x^2+6x+13,10x-4}=\br{4 x + 5,10x-4}=1
        \end{aligned}$$
        但该方法对第二小问太麻烦.
        \item 由于该题为四次多项式,故可设
        $$x^4+x^3+2x^2+x+1=(x^2+ax+b)(x^2+cx+d)$$
        展开后比较系数可得
        $$a+c=1,\quad ac+b+d=2,\quad ad+bc=1,\quad bd=1$$
        尝试带入$b=d=\pm 1$发现$b=d=1, a=0,c=1$时方程成立,即
        $$f(x)=(x^2+1)(x^2+x+1)$$
        从而无重因式.
        \item 注意到方程系数$(1,1,2,1,1)$是对称的,因此可令$z=x+1/x$换元,即
        $$\begin{aligned}
            f(x)&=x^2\br{x^2+x+2+\frac{1}{x}+\frac{1}{x^2}}=x^2\br{\br{x+\frac{1}{x}}^2+\br{x+\frac{1}{x}}}\\
            &=x^2\br{x+\frac{1}{x}+1}\br{x+\frac{1}{x}}=(x^2+1)(x^2+x+1)\\
        \end{aligned}$$
        从而无重因式.
        \end{itemize}
        \item 辗转相除可得$(f,f')=x^3- x^2- x+1$从而有重因式.也可直接试根:注意到$f(x)$的有理根$x_0=r/s$总有$r\mid 2, s\mid 1$,故$x_0$仅可能为$\pm 1, \pm 2$,故带入验算发现$1,-1,2$均为根,相除得到
        $$\frac{f(x)}{(x-2)(x-1)(x+1)}=x^3-x^2-x+1=(x-1)^2 (x+1)$$
        从而$f(x)=(x-1)^3(x+1)^2(x-2)$,其有重根.
    \end{enumerate}
\end{proof}
\paragraph{5.3.3}求$A,B$使得$(x-1)^2\mid Ax^4+Bx^2+1$.
\begin{proof}[证明一]
    设$f(x)=Ax^4+Bx^2+1$,由题知$(x-1)\mid f'(x)=4Ax^3+2Bx$,从而$f(1)=f'(1)=0$,即$A+B+1=4A+2B=0$,解得$A=1, B=-2$.
\end{proof}
\begin{proof}[证明二]
    设$f(x)=Ax^4+Bx^2+1$,注意到$(x-1)\mid (f,f')=(Ax^4+Bx^2+1,4Ax^3+2Bx)=(Bx^2/2+1,4Ax^3+2Bx)$,而$(x-1)\mid \frac{B}{2}x^2+1$要求$B=-2$,以及$(x-1)\mid x(4Ax^2-4)$要求$A=1$.
\end{proof}
\paragraph{5.3.4}设$f(x)=x^5-3x^4+2x^3+2x^2-3x+1$,在$\Q[x]$中求一个没有重因式的多项式$g$,使其与$f$有完全相同的不可约多项式(不计重数).
\begin{proof}
    观察多项式系数可知其有理根仅可能有$\pm 1$,验算可知均为根,从而有
    $$\frac{f(x)}{(x-1)(x+1)}=x^3-3 x^2+3 x-1=(x-1)^3$$
    从而取$g(x)=(x-1)(x+1)=x^2-1$即可.
\end{proof}
\paragraph{5.3.5}证明多项式$f(x)=x^4+2x^3-15x^2+4x+20$有重根,并求其所有根.
\begin{proof}
    辗转相除可得$(f,f')=x-2$,从而知$(x-2)^2\mid f(x)$,
    $$\frac{f(x)}{(x-2)^2}=x^2+6 x+5=(x+1)(x+5)$$
    从而$2,-1,-5$为其所有根.
\end{proof}
\paragraph{5.3.6}证明: 不可约多项式$p$是多项式$f$的$k$重因式$\iff p\mid f, p\mid f',\cdots,p\mid f^{(k-1)}$但$p\nmid f^{(k)}$.
\begin{proof}[证明一]
    容易看出,该命题等价于: $p^k\mid f \iff p$整除$f, f',\cdots, f^{(k-1)}$.下对$k$归纳,$k=1$时显然,下设$k\geq 2, <k$时命题成立.

    $\implies:$显然$p^{k-1}\mid f$,故由归纳假设,$p$整除$f,f',\cdots,f^{(k-2)}$,下证$p\mid f^{(k-1)}$.由于有$f=p^kg$,即$f'=p^{k-1}(kp'g+pg')$,故$p^{(k-1)}\mid f'$.从而由归纳假设,$p\mid (f')^{(k-2)}=f^{(k-1)}$.
    
    $\impliedby:$由于$p$整除$f',(f')',\cdots,(f')^{(k-2)}$,故由归纳假设知$p^{k-1}\mid f'$,其等价于$p^k\mid f$.
\end{proof}
\begin{proof}[证明二]
    $k=1$时已证,$k>1$时: $p$是$f$的$k$重因式$\iff p$是$f'$的$k-1$重因式$\iff\cdots\iff p$是$f^{(k-1)}$的2重因式$\implies p$是$f^{(k-2)}$的1重因式$\iff p\nmid (f^{(k-1)},f^{(k)})$,故$p\nmid f^{(k)}$.

    另一方面,$p\nmid f^{(k)}, p\mid f^{(k-1)}$,故$p$不为$f^{(k-1)}$的重因式;而$p$整除$f^{(k-1)},f^{(k-2)}$,故$p$是$f^{(k-2)}$的重因式.综上,$p$是$f^{(k-1)}$的2重因式,其余同上,从而得证.
\end{proof}
\begin{remark}
    该结果只对$\charfield\F>k$或$\charfield\F=0$的数域$\F$上的多项式成立.
\end{remark}
\paragraph{5.3.7}举例否定``若$\alpha$是$f'$的$m$重根,则$\alpha$是$f$的$m+1$重根''.
\begin{proof}
    取$f(x)=x^{m+1}+1, f'(x)=(m+1)x^m, 0$为$f'$的$m$重根但不是$f$的$m+1$重根.
\end{proof}
\begin{remark}
    该命题若加上条件``$\alpha$是$f$的根''即正确.

    证明: 由题知$(x-\alpha)\mid f, (x-\alpha)^m\mid f', (x-\alpha)^{m+1}\nmid f'$.由5.3.6知$(x-\alpha)$整除$f', f'', \cdots, (f')^{(m-1)}=f^{(m)}$但$(x-\alpha)\nmid f^{(m+1)}$,加上题设$(x-\alpha)\mid f$再由5.3.6知$(x-\alpha)$是$f$的$m+1$重因式.
\end{remark}
\paragraph{5.3.8}证明: 若$(x-1)\mid f(x^n)$则$(x^n-1)\mid f(x^n)$.
\begin{proof}[证明一]
    显然$f(1)=0$,故$(x-1)\mid f(x), f(x)=(x-1)g(x)$, 从而$f(x^n)=(x^n-1)g(x^n), (x^n-1)\mid f(x^n)$.
\end{proof}
\begin{proof}[证明二]
    显然$f(1)=0$.考虑1的任意$n$次单位根$\omega_k=\e^{\frac{2k\pi\i}{n}}$,有$f(\omega_k^n)=f(1)=0$,故$(x-\omega_k)\mid f(x^n)$,从而
    $$\prod_{k=0}^{n-1}(x-\omega_k)=(x^n-1)\mid f(x^n).$$
\end{proof}
\paragraph{5.3.9}$p\in \F[x], \deg p>0$.若对于$\forall f\in \F[x]$均有$p\mid f$或$(p,f)=1$,则$p$在$\F$中不可约.
\begin{proof}
    若$p$可被分解为次数小于$\deg p$的多项式$q,r$之积,则必有其中一个多项式次数非零,设其为$q$.从而取$f=q, (p,f)\neq 1, p\nmid f$,矛盾.
\end{proof}
\paragraph{5.3.10}$p\in \F[x], \deg p>0$.若对于$\forall f,g\in \F[x], p\mid fg\implies p\mid f$或$p\mid g$,则$p$在$\F$中不可约.
\begin{proof}
    若$p$可被分解为次数小于$\deg p$的多项式$q,r$之积,则$p\mid qr=p$但$p\nmid q, p\nmid r$,矛盾.
\end{proof}

\paragraph{思考题2}$x^2-2$在$\Q$上不可约而在$\R$上可约.
\begin{proof}[证明一]
    在$\R$上有$x^2-2=(x-\sqrt{2})(x+\sqrt{2})$从而可约.而该多项式在$\Q$上若有根$a=p/q$,则$q\mid 1,p\mid (-2)$,即$a$仅可能为$\pm 1, \pm 2$,而这些均不为根,从而无根,即不可约.
\end{proof}
\begin{proof}[证明二]
    若在$\Q$上有唯一分解$x^2-2=(x-a)(x-b)$,即$a+b=0, ab=-2$,即$a^2=2$.对$\sqrt{2}$的无理性证明导出$x^2-2$在$\Q$上不可约.
\end{proof}
\begin{proof}[证明三]
    书上例5.3.1.
\end{proof}

\paragraph{思考题3}设$f=p_1^{\alpha_1}p_2^{\alpha_2}\cdots p_s^{\alpha_s}, g=p_1^{\beta_1}p_2^{\beta_2}\cdots p_s^{\beta_s}$,其中$p_i$均为不可约多项式.证明$(f,g)=p_1^{\gamma_1}p_2^{\gamma_2}\cdots p_s^{\gamma_s}$,其中$\gamma_i=\min(\alpha_i,\beta_i), i=1,\cdots,s$.

\begin{proof}
    设$d=p_1^{\gamma_1}p_2^{\gamma_2}\cdots p_s^{\gamma_s}$,显然$d\mid f, d\mid g$.若有$f,g$的公因式$d'=p_1^{\delta_1}p_2^{\delta_2}\cdots p_s^{\delta_s}$,则$\forall i=1,2,\cdots,s, \delta_i\leq \alpha_i$且$\delta_i\leq \beta_i$,故$\delta_i\leq \gamma_i$,从而$d'\mid d$,故$d=(f,g)$.
\end{proof}

\paragraph{思考题4}$f_1,\cdots,f_s\in \F[x]$之间两两互素, 记$f=f_1\cdots f_s, g_i=f/f_i$,证明$(g_1,g_2,\cdots,g_s)=1$.
\begin{proof}
    对$s$归纳,$s=2$时$(g_1,g_2)=(f_2,f_1)=1$从而成立.设$<s$时命题成立,考虑两两互素的多项式$f_1,f_2,\cdots,f_s$,如上定义$f,g_i$,则有
    $$d=\br{g_1,g_2,\cdots,g_s}=\br{\br{g_1,g_2,\cdots,g_{s-1}},g_s}$$
    而
    $$\br{g_1,g_2,\cdots,g_{s-1}}=\br{\frac{f_1\cdots f_s}{f_1},\frac{f_1\cdots f_s}{f_2},\cdots,\frac{f_1\cdots f_s}{f_{s-1}}}=f_s\br{\frac{f_1\cdots f_{s-1}}{f_1},\frac{f_1\cdots f_{s-1}}{f_2},\cdots,\frac{f_1\cdots f_{s-1}}{f_{s-1}}}$$
    由归纳假设知右端项为$f_s$,从而$d=(f_s,f_1\cdots f_{s-1})=1$,从而得证.
\end{proof}

\newpage
\subsection{复系数与实系数多项式的因式分解}
\paragraph{5.4.1}求多项式$x^5-1$在$\C$和$\R$上的因式分解.
\begin{proof}
    在$\C$上显然有
    $$x^5-1=(x-1)(x-\omega)(x-\omega^2)(x-\omega^3)(x-\omega^4)$$
    其中$\omega=\e^{\frac{2\pi\i}{5}}=\cos\frac{2\pi}{5}+\i \sin\frac{2\pi}{5}$.而由于复数根成对,故在$\R$上有
    $$\begin{aligned}
        x^5-1&=(x-1)\bbr{(x-\omega)(x-\omega^4)}\bbr{(x-\omega^2)(x-\omega^3)}\\
        &=(x-1)\br{x^2-2\cos\frac{2\pi}{5}x+1}\br{x^2-2\cos\frac{4\pi}{5}x+1}\\
        &=(x-1)\br{x^2+\frac{1-\sqrt{5}}{2}x+1}\br{x^2+\frac{1+\sqrt{5}}{2}x+1}
    \end{aligned}$$
\end{proof}
\paragraph{5.4.2}$f\in \R[x], \deg f=n$且$f$有$\ell$个实根(计重数),证明$n-\ell$是偶数.
\begin{proof}
    将$f$分解为不可约多项式的乘积,即
    $$f=a p_1^{\alpha_1}\cdots p_s^{\alpha_s}q_1^{\beta_1}\cdots q_t^{\beta_t}$$
    其中$p_i$均为一次多项式,$q_i$均为二次多项式,则
    $$n=\sum_{i=1}^s \alpha_i + \sum_{i=1}^t 2\beta_i,\qquad \ell=\sum_{i=1}^s \alpha_i,\qquad n-\ell=2\sum_{i=1}^t \beta_i$$
    从而$n-\ell$显然为偶数.
\end{proof}
\paragraph{5.4.3}求$x^4+1$在$\C$和$\R$上的标准分解.
\begin{proof}
    在$\C$上$x^4+1$有根$(-1)^{1/4}=\e^{\frac{\pi\i}{4}}, (-1)^{3/4}=\e^{\frac{3\pi\i}{4}},(-1)^{5/4}=\e^{\frac{5\pi\i}{4}},(-1)^{7/4}=\e^{\frac{7\pi\i}{4}}$,因此有
    $$\begin{aligned}
        x^4+1&=(x-\e^{\frac{\pi\i}{4}})(x-\e^{\frac{3\pi\i}{4}})(x-\e^{\frac{5\pi\i}{4}})(x-\e^{\frac{7\pi\i}{4}})\\
        &=\bbr{(x-\e^{\frac{\pi\i}{4}})(x-\e^{\frac{7\pi\i}{4}})}\bbr{(x-\e^{\frac{3\pi\i}{4}})(x-\e^{\frac{5\pi\i}{4}})}\\
        &=\br{x^2-2\cos\frac{\pi}{4}x+1}\br{x^2-2\cos\frac{3\pi}{4}x+1}\\
        &=\br{x^2-\sqrt{2}x+1}\br{x^2+\sqrt{2}x+1}
    \end{aligned}$$
\end{proof}
\paragraph{5.4.4}$f\in \R[x]$的首项系数$a_n>0$,若$f$无实根,则存在$g,h\in \R[x]$使得$f=g^2+h^2$.
\begin{proof}
    由于实系数多项式$f$无实根,故其不可约分解中均为二次不可约多项式,即在$\C$中有分解
    $$f(x)=q_1(x)\cdots q_m(x)=\prod_{i=1}^m \bbr{(x-\lambda_i)(x-\bar{\lambda_i})}=\br{\prod_{i=1}^m (x-\lambda_i)}\br{\prod_{i=1}^m (x-\bar{\lambda_i})}=p(x)q(x)$$
    其中$q_i(x)=(x-\lambda_i)(x-\bar{\lambda_i})$均为在$\R$上不可约的二次多项式,$\lambda_i\in\C$.将$p(x)$按系数的实部和虚部分为两个实系数多项式,即
    $$p(x)=g(x)+\i h(x), \qquad g,h\in \R[x]$$
    再对$p(x)$的系数取共轭,有
    $$g(x)-\i h(x)=\bar{p}(x)=\prod_{i=1}^m (x-\bar{\lambda_i})=q(x)$$
    从而$f=(g+\i h)(g-\i h)=g^2+h^2$.
\end{proof}
\paragraph{5.4.5}设$p,f\in \R[x]$且$p$在$\R$上不可约,证明:若$\exists\alpha\in\C, p(\alpha)=f(\alpha)=0$则$p\mid f$.
\begin{proof}
    显然$\deg p=1$或2.若$\deg p=1$则$\alpha\in\R, p(x)=a(x-\alpha)\mid f(x)$.若$\deg p=2$则$\alpha\notin\R, p(x)=a(x-\alpha)(x-\bar{\alpha})$,而$f(\alpha)=f(\bar{\alpha})=0$,从而$p(x)=a(x-\alpha)(x-\bar{\alpha})\mid f(x)$.
\end{proof}

\newpage
\subsection{有理系数多项式}
\paragraph{5.5.1}求下列多项式的有理根:
\begin{enumerate}
    \item $2x^4-x^3+2x-3$;
    \item $4x^4-7x^2-5x-1$;
    \item $x^4+6x^3+12x^2+11x+6$.
\end{enumerate}
\begin{proof}
    (1)$1$; (2)$-1/2$ (2重); (3)$-2, -3$.
\end{proof}
\paragraph{5.5.2}判别下列多项式在$\Q$上是否可约:
\begin{enumerate}
    \item $x^6+x^3+1$;
    \item $x^p+px+1, p$是奇素数;
    \item $x^4+4$;
    \item $x^4+4kx+1, k\in \Z$.
\end{enumerate}
\begin{proof}
    \begin{enumerate}
        \item 代换$x=t+1$,故原式$=(t+1)^6+(t+1)^3+1=t^6+6 t^5+15 t^4+21 t^3+18 t^2+9 t+3$.用Einstein判别法(取$p=3$)知其在$\Q$上不可约.
        \item 令$x=t-1$,则原式$=(t-1)^p+pt+1-p=t^p+\sum_{k=2}^{p-1}\binom{p}{k}(-1)^{k}t^k+2pt+p$,从而由Einstein判别法知其在$\Q$上不可约.
        \item $x^4+4=(x^2-2x+2)(x^2+2x+2)$,故可约.
        \item 若原式$f(x)$在$\Q$上可约,则也在$\Z$上可约.显然$f(x)$的有理根仅可能有$\pm 1$,但$f(\pm 1)\neq 0, \pm 1$均不是根,从而$f(x)$在$\Z$上没有一次(和三次)因式.若原式在$\Z$上有二次因式,即设
        $$x^4+4kx+1=(x^2+ax+b)(x^2+cx+d), \qquad a,b,c,d\in\Z$$
        比较系数可得$a+c=0, ac+b+d=0, ad+bc=4k, bd=1$,从而$b=d=\pm 1$,$ac=-a^2=\mp 2$,矛盾于$a\in\Z$,故原式在$\Z$上也没有二次因式,故在$\Z$和$\Q$上不可约.
    \end{enumerate}
\end{proof}
\paragraph{5.5.3}$p$为素数,证明$f(x)=x^p-px+(2p-1)$在$\Q$上不可约.
\begin{proof}
    令$x=t+1$,则
    $$f(t+1)=(t+1)^p-p(t+1)+(2p-1)=t^p+\sum_{k=2}^{p-1}\binom{p}{k}t^k+p$$
    从而由Einstein判别法知其在$\Q$上不可约.
\end{proof}
\paragraph{5.5.4}设$p_i (i=1,2,\dots,t)$为$t$个互异素数,证明$f(x)=x^n-p_1\cdots p_t$在$\Q$上不可约.
\begin{proof}
    取素数$p$为任一$p_i$,由Einstein判别法知$f(x)$在$\Q$上不可约.
\end{proof}
\paragraph{5.5.5}$f$是\textcolor{red}{首一}整系数多项式, 若$f(0)$和$f(1)$均为奇数,则$f$没有有理根.
\begin{proof}[证明一]
    设$f(x)=\sum_{k=0}^{n}a_kx^k$,若其有有理根$r/s$则$s\mid a_n=1, r\mid a_0=f(0)$,从而有理根仅可能为$c\in \Z, c\mid f(0)$.而$f(0)$为奇数,故$c$为奇数.由于$f(c)=0$,故
    $$-f(1)=f(c)-f(1)=\sum_{k=0}^{n}a_k(c^k-1)$$
    而对$\forall k\in\N, c^k-1$为偶数,故等式右端为偶数,但左端为奇数,矛盾,从而$f$无有理根.
\end{proof}
\begin{proof}[证明二]
    若$f$有有理根$r/s$,则$s\mid a_n=1, r\mid a_0=f(0)$,即$s=\pm 1, r$为奇数.而$(r-ms)\mid f(m)$,故$(r\pm 1)\mid f(1)$,但$r\pm 1$为偶数,$f(1)$为奇数,矛盾,故无有理根.
\end{proof}
\begin{remark}
    若无首一条件,可取$f(x)=2x-1, f(0)=1, f(1)=3$但有有理根$1/2$.
\end{remark}
\begin{remark}
    命题可作简单推广:$f$是首一整系数多项式, $p$是素数, 若$f(0)\not\equiv 0, f(1)\not\equiv 0 \pmod p$,则$f$没有有理根.
\end{remark}

% \subsection{复习题5}

% \section{第六章 相似标准形}
% \subsection{特征值与特征向量}
% \subsection{特征子空间与根子空间}
% \subsection{对角化}
% \subsection{$\lambda$-矩阵}
% \subsection{行列式因子、不变因子与初等因子}
% \subsection{Jordan 标准形}
% \subsection{复习题6}

% \section{第七章 双线性函数与二次型}
% \subsection{双线性函数}
% \subsection{标准形}
% \subsection{惯性定理}
% \subsection{正定性}
% \subsection{复习题7}

% \section{第八章 内积空间}
% \subsection{欧氏空间}
% \subsection{标准正交基}
% \subsection{欧氏空间的子空间}
% \subsection{正交变换}
% \subsection{对称变换}
% \subsection{复习题8}
\end{document}

\paragraph{5.2.1}
\paragraph{5.2.2}
\paragraph{5.2.3}
\paragraph{5.2.4}
\paragraph{5.2.5}
\paragraph{5.2.6}
\paragraph{5.2.7}
\paragraph{5.2.8}
\paragraph{5.2.9}
\paragraph{5.2.10}
\paragraph{5.2.11}
\paragraph{5.2.12}
\paragraph{5.2.13}
\paragraph{5.2.14}
\paragraph{5.2.15}