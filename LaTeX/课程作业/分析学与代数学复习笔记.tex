\documentclass{article}
% 用ctex显示中文并用fandol主题
\usepackage[fontset=fandol]{ctex}
\setmainfont{CMU Serif} % 能显示大量外文字体
\xeCJKsetup{CJKmath=true} % 数学模式中可以输入中文

% AMS全家桶,\DeclareMathOperator依赖之
\usepackage{amsmath,amssymb,amsthm,amsfonts,amscd}
\usepackage{pgfplots,tikz,tikz-cd} % 用来画交换图
\usepackage{bm,mathrsfs} % 粗体字母(含希腊字母)和\mathscr字体
\everymath{\displaystyle} % 全体公式为行间形式

% 纸张上下左右页边距
\usepackage[a4paper,left=1cm,right=1cm,top=1.5cm,bottom=1.5cm]{geometry}
% 生成书签和目录上的超链接
\usepackage[colorlinks=true,linkcolor=blue,filecolor=blue,urlcolor=blue,citecolor=cyan]{hyperref}
% 各种列表环境的行距
\usepackage{enumitem}
\setenumerate[1]{itemsep=0pt,partopsep=0pt,parsep=\parskip,topsep=0pt}
\setenumerate[2]{itemsep=0pt,partopsep=0pt,parsep=\parskip,topsep=0pt}
\setenumerate[3]{itemsep=0pt,partopsep=0pt,parsep=\parskip,topsep=0pt}
\setitemize[1]{itemsep=0pt,partopsep=0pt,parsep=\parskip,topsep=5pt}
\setdescription{itemsep=0pt,partopsep=0pt,parsep=\parskip,topsep=5pt}
\setlength\belowdisplayskip{2pt}
\setlength\abovedisplayskip{2pt}

% 左右配对符号
\newcommand{\br}[1]{\!\left(#1\right)} % 括号
\newcommand{\cbr}[1]{\left\{#1\right\}} % 大括号
\newcommand{\abr}[1]{\left<#1\right>} % 尖括号(内积)
\newcommand{\bbr}[1]{\left[#1\right]} % 中括号
\newcommand{\abbr}[1]{\left(#1\right]} % 左开右闭区间
\newcommand{\babr}[1]{\left[#1\right)} % 左闭右开区间
\newcommand{\abs}[1]{\left|#1\right|} % 绝对值
\newcommand{\norm}[1]{\left\|#1\right\|} % 范数
\newcommand{\floor}[1]{\left\lfloor#1\right\rfloor} % 下取整
\newcommand{\ceil}[1]{\left\lceil#1\right\rceil} % 上取整
% 常用数集简写
\newcommand{\R}{\mathbb{R}} % 实数域
\newcommand{\N}{\mathbb{N}} % 自然数集
\newcommand{\Z}{\mathbb{Z}} % 整数集
\newcommand{\C}{\mathbb{C}} % 复数域
\newcommand{\F}{\mathbb{F}} % 一般数域
\newcommand{\kfield}{\Bbbk} % 域
\newcommand{\K}{\mathbb{K}} % 域
\newcommand{\Q}{\mathbb{Q}} % 有理数域
\newcommand{\Pprime}{\mathbb{P}} % 全体素数,或概率
% 范畴记号
\newcommand{\Ccat}{\mathsf{C}}
\newcommand{\Grp}{\mathsf{Grp}} % 群范畴
\newcommand{\Ab}{\mathsf{Ab}} % 交换群范畴
\newcommand{\Ring}{\mathsf{Ring}} % (含幺)环范畴
\newcommand{\Set}{\mathsf{Set}} % 集合范畴
\newcommand{\Mod}{\mathsf{Mod}} % 模范畴
\newcommand{\Vect}{\mathsf{Vect}} % 向量空间范畴
\newcommand{\Alg}{\mathsf{Alg}} % 代数范畴
\newcommand{\Comm}{\mathsf{Comm}} % 交换
% 代数集合
\DeclareMathOperator{\Hom}{Hom} % 同态
\DeclareMathOperator{\End}{End} % 自同态
\DeclareMathOperator{\Iso}{Iso} % 同构
\DeclareMathOperator{\Aut}{Aut} % 自同构
\DeclareMathOperator{\Inn}{Inn} % 内自同构
% \DeclareMathOperator{\inv}{Inv}
\DeclareMathOperator{\GL}{GL} % 一般线性群
\DeclareMathOperator{\SL}{SL} % 特殊线性群
\DeclareMathOperator{\GF}{GF} % Galois域
% 正体符号
\renewcommand{\i}{\mathrm{i}} % 本产生无点i
\newcommand{\id}{\mathrm{id}} % 恒等映射
\newcommand{\e}{\mathrm{e}} % 自然常数e
\renewcommand{\d}{\mathrm{d}} % 微分符号,本产生重音符号
\newcommand{\D}{\partial} % 偏导符号
\newcommand{\diff}[2]{\frac{\d #1}{\d #2}}
\newcommand{\Diff}[2]{\frac{\D #1}{\D #2}}
% 运算符(分析)
\DeclareMathOperator{\Arg}{Arg} % 辐角
\DeclareMathOperator{\re}{Re} % 实部
\DeclareMathOperator{\im}{im} % 像,虚部
\DeclareMathOperator{\grad}{grad} % 梯度
\DeclareMathOperator{\lcm}{lcm} % 最小公倍数
\DeclareMathOperator{\sgn}{sgn} % 符号函数
\DeclareMathOperator{\conv}{conv} % 凸包
\DeclareMathOperator{\supp}{supp} % 支撑
\DeclareMathOperator{\Log}{Log} % 广义对数函数
\DeclareMathOperator{\card}{card} % 集合的势
\DeclareMathOperator{\Res}{Res} % 留数
% 运算符(代数,几何,数论)
\newcommand{\Span}{\mathrm{span}} % 张成空间
\DeclareMathOperator{\tr}{tr} % 迹
\DeclareMathOperator{\rank}{rank} % 秩
\DeclareMathOperator{\charfield}{char} % 域的特征
\DeclareMathOperator{\codim}{codim} % 余维度
\DeclareMathOperator{\coim}{coim} % 余维度
\DeclareMathOperator{\coker}{coker} % 余维度
\DeclareMathOperator{\Spec}{Spec} % 谱
\DeclareMathOperator{\diag}{diag} % 谱
\newcommand{\Obj}{\mathrm{Obj}} % 对象类
\newcommand{\Mor}{\mathrm{Mor}} % 态射类
\newcommand{\Cen}{C} % 群/环的中心 或记\mathrm{Cen}
\newcommand{\opcat}{^{\mathrm{op}}}
% 简写
\newcommand{\hyphen}{\textrm{-}}
\newcommand{\ds}{\displaystyle} % 行间公式形式
\newcommand{\ve}{\varepsilon} % 手写体ε
\newcommand{\rev}{^{-1}\!} % 逆
\newcommand{\T}{^{\mathsf{T}}} % 转置
\renewcommand{\H}{^{\mathsf{H}}} % 共轭转置
\newcommand{\adj}{^\lor} % 伴随
\newcommand{\dual}{^\vee} % 对偶
\DeclareMathOperator{\lhs}{LHS}
\DeclareMathOperator{\rhs}{RHS}
\newcommand{\hint}[1]{{\small (#1)}} % 提示
\newcommand{\why}{\textcolor{red}{(Why?)}}
\newcommand{\tbc}{\textcolor{red}{(To be continued...)}} % 未完待续

% 定理环境(随笔记形式更改)
\newtheorem{definition}{定义}
\newtheorem{remark}{注}
\newtheorem{example}{例}
\makeatletter
\@ifclassloaded{article}{
    \newtheorem{theorem}{定理}[section]
}{
    \newtheorem{theorem}{定理}[chapter]
}
\makeatother
\newtheorem{lemma}[theorem]{引理}
\newtheorem{proposition}[theorem]{命题}
\newtheorem{corollary}[theorem]{推论}
\newtheorem{property}[theorem]{性质}

\begin{document}
\tableofcontents

\section{分析学复习题}
\paragraph{1}有度量空间$(X_1,d_1),\cdots,(X_n,d_n)$,在$X=\prod_{i=1}^{n}X_i$上定义度量$\rho_1,\rho_2:\forall x=(x_1,\cdots,x_n),y=(y_1,\cdots,y_n)\in X,$
$$\rho_1(x,y)=\sqrt{\sum_{i=1}^{n}d_i(x_i,y_i)^2},\rho_2(x,y)=\sum_{i=1}^{n}d_i(x_i,y_i)$$
证明$\rho_1,\rho_2$诱导的度量拓扑相同.
\begin{proof}
    仅需证明度量等价,注意到
    $$\rho_1(x,y)=\frac{1}{\sqrt{n}}\sqrt{\br{\sum_{i=1}^{n}d_i(x_i,y_i)^2}\br{\sum_{i=1}^{n}1^2}}\geq \frac{1}{\sqrt{n}}\sum_{i=1}^{n}d_i(x_i,y_i)=\frac{\rho_2(x,y)}{\sqrt{n}}$$
    以及由$d_i(x_i,y_i)\leq \rho_2(x,y)$知$\rho_1(x,y)\leq\sqrt{n}\rho_2(x,y)$,故得证.
\end{proof}

\paragraph{2}度量空间$(X,\rho), x\in X, \varnothing\neq A\subset X$,证明(1)$f=\rho(\cdot,A)$一致连续.(2)$\overline{A}=\cbr{x\in X|\rho(x,A)=0}$.
\begin{proof}
    (1)由$\inf$性质知$\forall x_1,x_2\in X\forall a\in A, \rho(x_1,A)\leq \rho(x_1,a)\leq \rho(x_1,x_2)+\rho(x_2,a)$,从而$\rho(x_1,A)-\rho(x_2,a)\leq\rho(x_1,x_2)$,两端取$\inf_{a\in A}$即得$\rho(x_1,A)-\rho(x_2,A)\leq \rho(x_1,x_2)$.由$x_1,x_2$任意知$\abs{f(x_1)-f(x_2)}\leq \rho(x_1,x_2)$,故$f$ Lipschitz连续,从而一致连续.

    (2)$\forall x\in \overline{A}\forall\varepsilon>0\exists y\in A,\rho(x,y)<\varepsilon$,故取$\inf$即可.另一方面,可以取收敛列来写(取$\varepsilon=1/n$可取到$a_n\in A$趋于$a\in \overline{A}$),也可以注意到$A\subset \cbr{x\in X|\rho(x,A)=0}=f\rev(0)$,而后者是闭集,故$\overline{A}\subset \cbr{x\in X|\rho(x,A)=0}$.
\end{proof}

\paragraph{3}度量空间$(X,\rho)$完备$\iff$任意$X$中点列$\cbr{x_n}_{n=1}^\infty$,若有$\forall n\geq 1, \rho(x_n,x_{n+1})\leq 2^{-n}$则点列收敛.
\begin{proof}
    $\implies:$注意到$\rho(x_m,x_n)\leq 2^{1-m}(m\leq n)$故$\cbr{x_n}_{n=1}^\infty$是Cauchy列,由完备知收敛.

    $\impliedby:$任取$X$中Cauchy列$\cbr{y_n}_{n=1}^\infty$,对于$\forall k\in\N$取$\varepsilon_k=2^{-k}\exists N_k>N_{k-1}\forall m,n\geq N_k, \rho(y_m,y_n)<2^{-k}$,从而可取子列$\cbr{y_{N_i}}_{i=1}^\infty$,由题设知收敛,而Cauchy列有收敛子列即自身收敛,故得证.
\end{proof}

\paragraph{4}完备度量空间$(X,\rho)$上有$T:X\to X, \exists N\in \N^*\exists \alpha\in (0,1), \rho(T^N x,T^N y)\leq \alpha\rho(x,y)\forall x,y\in X$,其中$T^N=\underbrace{T\circ T\circ\cdots\circ T}_{N次}$,证明$T$有唯一不动点.
\begin{proof}
    由不动点定理知$T^N$有唯一不动点$x_0$,而$T^N T x_0=T^{N+1}x_0=T T^N x_0=T x_0$,故$T x_0$也为$T^N$不动点,$T x_0=x_0$,故$x_0$是$T$不动点.若$T$有其他不动点$x'$则$T^N$也有不动点$x'$,与唯一矛盾,从而$T$有唯一不动点.
\end{proof}

\paragraph{5}度量空间$(X,\rho)$内有开集$U$和非空紧集$A\subset U$,证明$\exists \delta>0\forall x\in A, B(x,\delta)\subset U$.
\begin{proof}
    由$U$开知$\forall x\in A\exists\delta_x>0, B(x,\delta_x)\subset U$,从而$\cbr{B\br{x,\frac{\delta_x}{2}}}_{x\in A}$是$A$的开覆盖,其中有有限子覆盖$\cbr{B\br{x_i,\frac{\delta_i}{2}}}_{i=1}^n$,取$\delta=\min_{1\leq i\leq n}\frac{\delta_i}{2}$,从而$\forall x\in A\forall y\in B(x,\delta)\exists x_i, \rho(y,x_i)\leq \rho(y,x)+\rho(x,x_i)<\delta+\frac{\delta_i}{2}\leq \delta_i$,故$y\in B(x_i,\delta_i)\subset U$.
\end{proof}

\paragraph{6}实线性赋范空间$X$上有线性泛函$f:X\to\R$,证明$f$连续$\iff N(f)=\cbr{x\in X|f(x)=0}$是闭集.
\begin{proof}
    $\implies:$由于$\cbr{0}$是闭集且$f$连续,故$N(f)=f\rev(\cbr{0})$是闭集.

    $\impliedby:$若$f$不连续,则由线性泛函连续的等价条件知,$\forall M>0\exists x\in X, \abs{f(x)}>M\norm{x}$,从而可取点列$x_n\in X-\cbr{0}, \abs{f(x_n)}>n\norm{x_n}$,取点列$y_n=\frac{x_n}{f(x_n)}-\frac{x_1}{f(x_1)}, f(y_n)\equiv 0$,从而$y_n\in N(f)$.但$\norm{y_n+\frac{x_1}{f(x_1)}}=\frac{\norm{x_n}}{\abs{f(x_n)}}<\frac{1}{n}\to 0(n\to\infty), y_n\to -\frac{x_1}{f(x_1)}\notin N(f)$.这与$N(f)$闭矛盾,因此$f$连续.
\end{proof}

\paragraph{7}域$\mathbb{K}$上线性赋范空间$(X,\norm{\cdot})$中有有限维真线性子空间$M\subsetneq X$.证明$\exists y\in X, \norm{y}=1$且$\forall x\in M, \norm{y-x}\geq 1$.
\begin{proof}
    任取$y_0\in X$,令$d=d(y_0,M)$, 由$\inf$性质可取$\forall n\geq 1\exists x_n\in M,d\leq \norm{y_0-x_n}\leq d+\frac{1}{n}, \norm{x_n}\leq \norm{x_n-y_0}+\norm{y_0}\leq \norm{y_0}+d+1$,故$\cbr{x_n}$是$M$中有界序列,由$M$有限维知$\overline{B_M}(0,\norm{y_0}+d+1)$是紧集,从而其中点列$\cbr{x_n}$有收敛子列$\cbr{x_{n_k}}$收敛于$x_0$,从而$d\leq \norm{y_0-x_0}=\lim_{k\to\infty}\norm{y_0-x_{n_k}}\leq d,\norm{y_0-x_0}=d$,故可取$y=\frac{y_0-x_0}{\norm{y_0-x_0}},\norm{y}=1$,从而$\forall x\in M, \norm{y-x}=\norm{\frac{y_0-x_0}{d}-x}=\frac{\norm{y_0-(x_0+dx)}}{d}\geq \frac{d}{d}=1$,从而得证.
\end{proof}

\paragraph{8}在$X=C^1([0,1],\R)$上定义范数$\norm{f}_{C^1}=\max\cbr{\norm{f}_\infty,\norm{f'}_\infty}$,证明$(X,\norm{\cdot}_{C^1})$构成实Banach空间.
\begin{proof}
    取$X$中的Cauchy列$\cbr{f_n}_{n=1}^\infty$,即$\forall \varepsilon>0\exists N\in\N\forall m,n\geq N, \norm{f_m-f_n}_{C^1}< \varepsilon$,而由$\norm{f}_\infty\leq \norm{f}_{C^1}, \norm{f'}_\infty\leq \norm{f}_{C^1}$知,$f_n,f_n'$在$(C[0,1],\norm{\cdot}_\infty)$中是Cauchy列,而$[0,1]$紧从而该空间完备, Cauchy列收敛,而函数列在范数$\norm{\cdot}_\infty$下收敛等价于一致收敛.记$f_n\to f, f_n'\to g$,则有
    $$f(x)=\lim_{n\to\infty}f_n(x)=\lim_{n\to\infty}\br{f_n'(0)+\int_0^x f_n'(t)\d t}=g(0)+\int_0^x \lim_{n\to\infty} f_n'(t) \d t=g(0)+\int_0^x g(t)\d t$$
    从而$f'=g,\norm{f_n-f}_{C^1}=\max\cbr{\norm{f_n-f}_\infty,\norm{f_n'-g}_\infty}\to 0$,$\cbr{f_n}_{n=1}^\infty$收敛,故$X$是Banach空间.
\end{proof}

\paragraph{9}域$\mathbb{K}$上线性赋范空间$(X,\norm{\cdot})$中有以$\theta$为内点的真凸子集$E$,其产生Minkowski泛函$P$.\\
证明(1)$E^\circ=\cbr{x\in X|P(x)<1}$.(2)$\overline{E^\circ}=\overline{E}$.
\begin{proof}
    (1)$\subset: \forall x\in E^\circ \exists \delta>0, B(x,\delta)\subset E$,故由$\norm{\br{1+\frac{\delta}{2\norm{x}}}x-x}=\frac{\delta}{2}<\delta$知$\br{1+\frac{\delta}{2\norm{x}}}x\in B(x,\delta)\subset E$,从而$P(x)\leq \br{1+\frac{\delta}{2\norm{x}}}\rev<1$. $\supset:$若$P(x)<1$则有$\lambda\in \babr{P(x),1}, x/\lambda\in E$.由$0$是$E$内点知$\exists \delta>0, B(0,\delta)\subset E$,故$\forall y\in B(0,\delta), \lambda\cdot(x/\lambda)+(1-\lambda)y=x+(1-\lambda)y\in E$,即$B(x,(1-\lambda)\delta)\subset E, x\in E^\circ$.

    (2)由$E^\circ \subset E$知$\overline{E^\circ}\subset \overline{E}$,仅需证$E\subset \overline{E^\circ}$.由$0$是内点知$\exists \delta>0, B(0,\delta)\subset E$,故$\forall x\in E\forall y\in B(0,\delta)\forall \lambda\in\babr{0,1}, \lambda x+(1-\lambda)y\in E$,故$B(\lambda x, (1-\lambda)\delta)\subset E, \lambda x\in E^\circ, x=\lim_{\lambda\to 1^-}\lambda x\in \overline{E^\circ}$.
\end{proof}

\paragraph{10}$\R^n$内Lebesgue可测子集$\Omega, m(\Omega)<\infty, 1\leq p_1<p_2<\infty$,证明$L^{p_2}(\Omega)\subset L^{p_1}(\Omega)$且
$$\norm{f}_{p_1}\leq [m(\Omega)]^{\frac{1}{p_1}-\frac{1}{p_2}}\norm{f}_{p_2}, \forall f\in L^{p_2}(\Omega).$$
\begin{proof}
    由H\"older不等式知
    $$\begin{aligned}
        \forall f\in L^{p_2}(\Omega), \norm{f}_{p_1}&=\norm{f\cdot 1}_{p_1}\leq \norm{f}_{p_2}\norm{1}_{r} \qquad 其中 \frac{1}{r}=\frac{1}{p_1}-\frac{1}{p_2}\\
        &=\norm{f}_{p_2} \br{\int_\Omega 1^r \d m}^{1/r}=\norm{f}_{p_2} m(\Omega)^{\frac{1}{p_1}-\frac{1}{p_2}}
    \end{aligned}$$
    从而不等式得证,且右端有限,故左端有限,即$f\in L^{p_1}(\Omega),L^{p_2}(\Omega)\subset L^{p_1}(\Omega)$.
\end{proof}

\paragraph{11}域$\mathbb{K}$上的Hilbert空间$H$中有可数规范正交集合$\cbr{e_n}_{n=1}^\infty$,证明Bessel不等式$\sum_{n=1}^{\infty}\abs{\abr{x,e_n}}^2\leq\norm{x}^2, \forall x\in H$.
\begin{proof}
    从$\cbr{e_n}_{n=1}^\infty$中取有限集$\cbr{e_i}_{i\in I}$,则有
    $$\begin{aligned}
        0&\leq \norm{x-\sum_{i\in I}\abr{x,e_i}e_i}^2=\abr{x-\sum_{i\in I}\abr{x,e_i}e_i,x-\sum_{i\in I}\abr{x,e_i}e_i}=\norm{x}^2-2\sum_{i\in I}\abs{\abr{x,e_i}}^2+\sum_{i,j\in I}\abr{x,e_i}\abr{e_j,x}\abr{e_i,e_j}\\
        &=\norm{x}^2-2\sum_{i\in I}\abs{\abr{x,e_i}}^2+\sum_{i\in I}\abs{\abr{x,e_i}}^2=\norm{x}^2-\sum_{i\in I}\abs{\abr{x,e_i}}^2
    \end{aligned}$$
    从而$\norm{x}^2\geq \sum_{i\in I}\abs{\abr{x,e_i}}^2$,因此$\sum_{n=1}^\infty \abs{\abr{x,e_n}}^2$收敛且仍有$\sum_{n=1}^{\infty}\abs{\abr{x,e_n}}^2\leq\norm{x}^2$.
\end{proof}

\paragraph{12}非零实线性赋范空间$X$中有点列$\cbr{x_n}_{n=1}^\infty, x_0\in X-\cbr{\theta}$.(1)证明$\exists f\in X^*, f(x_0)=\norm{x_0}, \norm{f}=1$.(2)若$\cbr{x_n}_{n=1}^\infty$在$X^*$中弱收敛,证明其弱极限唯一.
\begin{proof}
    (1)取$X$的线性子空间$X_0=\Span(x_0)$,定义线性泛函$f_0:X_0\to\R, \lambda x_0\mapsto \lambda\norm{x_0}$,则$f_0(x_0)=\norm{x_0}, \norm{f_0}=\sup_{\lambda\neq 0}\frac{\abs{\lambda\norm{x_0}}}{\norm{\lambda x_0}}=1$,从而由Hahn-Banach定理知$\exists f\in X^*, f|_{X_0}=f_0, f(x_0)=f_0(x_0)=\norm{x_0},\norm{f}=\norm{f_0}=1$.

    (2)若$x_n\overset{w}{\to}y_1$且$x_n\overset{w}{\to}y_2$,则$\forall f\in X^*,f(y_1)=\lim_{n\to \infty}f(x_n)=f(y_2), f(y_1-y_2)=0$.若$y_1-y_2\neq 0$,则$\exists f\in X^*, f(y_1-y_2)=\norm{y_1-y_2}=0$,矛盾,故$y_1=y_2$.
\end{proof}

\paragraph{13}Hilbert空间$H$中有非零闭线性子空间$Y, P:H\to Y$是正交投影算子.证明(1)$Y^{\perp\perp}=Y$.(2)$\norm{P}=1$.
\begin{proof}
    (1)一方面$y\in Y, y\perp Y^\perp$,故$y\in Y^{\perp\perp}, Y\subset Y^{\perp\perp}$.另一方面考虑$Y$上的正交分解$X=Y\oplus Y^\perp$,则$\forall x\in Y^{\perp\perp}, x=Px+Qx, Px\in Y\subset Y^{\perp\perp}, Qx\in Y^\perp$.而$Qx=x-Px\in Y^{\perp\perp}$,从而$Qx=0, x=Px\in Y, Y^{\perp\perp}\subset Y$.

    (2)由$\forall y\in Y-\cbr{\theta}, Py=y$知$\norm{P}\geq \frac{\norm{Py}}{\norm{y}}=1$,另一方面$\norm{P}=\sup_{x\neq 0}\frac{\norm{Px}}{\norm{x}}=\sup_{x\neq 0}\frac{\norm{Px}}{\sqrt{\norm{Px}^2+\norm{x-Px}^2}}\leq 1$,从而$\norm{P}=1$.
\end{proof}

\paragraph{14}$X,Y$分别是域$\mathbb{K}$上的Banach空间和线性赋范空间,$T\in \mathcal{L}(X,Y)$.若$\exists M>0\forall x\in X, \norm{Tx}\geq M\norm{x}$.证明$R(T)=\cbr{Tx|x\in X}$是$Y$的闭线性子空间.
\begin{proof}
    取$\forall x\in \ker T, 0=\norm{Tx}\geq M\norm{x}$知$x=0, T$是单射,故$T:X\to R(T)$是双射,其有逆映射$T\rev:R(T)\to X$,其是线性映射,且$\forall y\in R(T)\exists! x\in X, Tx=y, \norm{T\rev}=\sup_{y\in R(T)-\cbr{0}}\frac{\norm{T\rev y}}{\norm{y}}=\sup_{y\in R(T)-\cbr{0}}\frac{\norm{x}}{\norm{Tx}}\leq M$,故$T\rev$是连续线性映射,从而$R(T)=T(X)=(T\rev)\rev(X)$是闭集,显然$R(T)$是线性子空间,故得证.
\end{proof}

\paragraph{15}用闭图像定理证明Banach逆算子定理.
\begin{quotation}
    \textbf{闭图像定理}:对于Banach空间$X,Y$之间的线性映射$T\in\mathcal{L}(X,Y)$, $T$连续$\iff T$是闭线性算子,即$G(T)=\cbr{(x,Tx)\in X\times Y|x\in X}$是$X\times Y$中闭集.

    \textbf{Banach逆算子定理}: Banach空间$X,Y$之间的双射连续线性映射$T:X\to Y$的逆映射$T\rev:Y\to X$是连续线性映射.
\end{quotation}

\begin{proof}
    考虑双射连续线性映射$T:X\to Y$,其逆映射$T\rev$同样是线性映射,由闭图像定理知$T\rev$连续$\iff G(T\rev)=\cbr{(y,T\rev y)\in Y\times X|y\in Y}$是闭集.考虑$G(T\rev)$中收敛列$\cbr{(y_n,T\rev y_n)}_{n=1}^\infty, (y_n,T\rev y_n)\to (y_0,x_0)\in Y\times X$,即$y_n\to y_0, T\rev y_n\to x_0$,由$T$连续知$Tx_0=T\br{\lim_{n\to\infty}T\rev y_n}=\lim_{n\to\infty}TT\rev y_n=y_0$,故$(y_0,x_0)=(y_0,T\rev y_0)\in G(T\rev)$,即$G(T\rev)$闭,得证.
\end{proof}

\paragraph{真题2}线性赋范空间$X$中有紧集$A$与闭集$B$,证明$A+B$是闭集.

\paragraph{真题5}$\Omega$是$\R^n$中Lebesgue可测集.(1)任取$1\leq p<q<\infty$,证明$L^\infty(\Omega)\cap L^p(\Omega)\subset L^q(\Omega)$.\\
(2)若$\frac{1}{p}+\frac{1}{q}=\frac{1}{r}, p,q,r\in \babr{1,\infty}$,证明H\"older不等式$\norm{fg}_r\leq\norm{f}_p\norm{g}_q$.\\
(3)若$f\in \bigcap_{1\leq p<\infty}L^p(\Omega)$且$\exists C>0\forall p\in \babr{1,\infty}, \norm{f}\leq C$,证明$f\in L^\infty(\Omega)$.

\newpage
% \section*{代数学复习笔记}
\section{群}
\subsection{群作用与Sylow定理}
\paragraph{类数公式}
群$G$作用在集合$S$上有$\abs{S}=\abs{Z}+\sum_{a\in A} [G:G_a]$,其中$Z=\cbr{x\in S|\forall g\in G, g\cdot x=x}$是稳定点集合,$G_a$是$a$的稳定子,$A$是轨道非平凡元素.考虑群$G$共轭作用于自身,则有类公式$\abs{G}=\abs{C(G)}+\sum_{a\in A}[G:C_G(a)]$.

\paragraph{关于$p$-群的引理}
若$G$是$p$-群,则由$p|[G:G_a]|\abs{G}$知$\abs{Z}\equiv \abs{S}\bmod p$,对于共轭作用即$\abs{C(G)}\equiv 0\bmod p$.而$e\in C(G),\abs{C(G)}\geq 1$,故$\abs{C(G)}\geq p$,即$p$-群中心非平凡.

若有限$G$有$p$-子群$H$,则考虑$H$在左陪集$G/_l H=\cbr{gH|g\in G}$上的左平移作用,其稳定点
$$Z=\cbr{gH|\forall h\in H, g\rev hg\in H}=\cbr{gH|g\in N_G(H)}=N_G(H)/H$$
故$[N_G(H):H]=\abs{Z}\equiv \abs{G/_l H}=[G:H]\bmod p$.从而若$p|[G:H]$则$p|[N_G(H):H]\geq 1, N_G(H)\gneq H$.

\paragraph{Cauchy定理}
$G$是有限群,$\abs{G}$有素因子$p$,则$G$中总有$p$阶元.

\begin{proof}[证明(James McKay)]
    考虑$S=\cbr{(a_1,\cdots,a_p)|a_i\in G, a_1\cdots a_p=e}$,由$a_p$由前元素唯一决定,故$\abs{S}=\abs{G}^{p-1}\equiv 0\bmod p$.令$\Z/p$循环作用于$S$上,即$m\cdot(a_1,\cdots,a_p)=(a_{m+1},\cdots,a_p,a_1,\cdots,a_m)\in S$(容易验证$ab=e$则$ba=e$),显然该作用的稳定点$Z=\cbr{(a,\cdots,a)|a\in G}$,且有$\abs{Z}\equiv\abs{S}\equiv 0\bmod p$,而$e\in Z, \abs{Z}\geq 1$,故有$a\neq e, a^p=e$.
\end{proof}

\paragraph{Sylow第一定理}
$G$为有限群,则对$\abs{G}$的任意素因子$p$, $G$总含Sylow $p$-子群.$G$中的Sylow $p$-子群$P$即$P<G$为$p$-群且$p$与$[G:P]$互素,即$G$中的极大$p$-子群.换言之,$\abs{G}=p^rm, \abs{P}=p^r, \gcd(p,m)=1$.

定理的等价(由$p$-群性质)描述为:$G$为$p^nm$阶群($p$为素数且与$m$互素),则对$k\in [n]$总有$p^k$阶子群,且该子群是某个$p^{k+1}$阶子群的正规子群.

\begin{proof}
    首先由Cauchy定理,$G$中总含$p$阶子群.下对$k$归纳证明:若$H$是$G$的$p^k$阶子群($k<n$),则$0\equiv [G:H]\equiv [N_G(H):H]\bmod p$,后项非0故$N_G(H)\neq H$,且$N_G(H)/H$也含$p$阶子群,记为$H_1/H$,从而$H\rhd H_1, \abs{H_1}=\abs{H}\abs{H_1/H}=p^{k+1}$.
\end{proof}

\paragraph{Sylow第二定理}
$P$是有限群$G$中的Sylow $p$-子群,$H<G$是$p$-子群,则$H$在$P$的某个共轭中,即$H<gPg\rev$.特别的,$G$的Sylow $p$-子群间互相共轭.
\begin{proof}
    令$H$左乘作用于$G/_LP$上,作用不动点集的势$\abs{Z}\equiv [G:P]\not\equiv 0\bmod p$,故有作用不动点$aP, a\rev ha\in P(\forall h\in H), H<aPa\rev$.
\end{proof}

\paragraph{Sylow第三定理}
$G$为有限群,$\abs{G}=p^nm$,其中$p$为素数且与$m$互素,则$G$中的Sylow $p$-子群数量$N_p|m$且$N_p\equiv 1\bmod p$.
\begin{proof}
    $G$在$\mathcal{P}(G)$上的共轭作用中,任意Sylow $p$-子群$P$所在的轨道即$G$中所有Sylow $p$-子群构成的集合,而$P$的稳定子为$N_G(P)$.故由轨道-稳定子定理知$G$中Sylow $p$-子群的数量为$N_p=[G:N_G(P)]|[G:P]=m$.而$m=[G:P]\equiv [N_G(P):P]\bmod p$,因此$mN_p\equiv m\bmod p$,故由$m\perp p$得$N_p\equiv 1\bmod p$.
\end{proof}

\subsection{可解群}
$G$为可解群:存在有限长可解列$G=G_0\rhd G_1\rhd\cdots\rhd G_n=\cbr{e}, G_i/G_{i+1}$为非平凡交换群,其等价于存在循环列,即$G_i/G_{i+1}\cong\Z/p$.从而所有$p$-群均可解.
\begin{itemize}
    \item 四面体群$D_4=\abr{a,b|a^4=b^2=a^3bab=e}$可解:注意到Klein群$K_4\cong\Z/2\oplus\Z/2\cong\cbr{e,a^2,b,a^2b}\rhd D_4$,从而有可解列$D_4\rhd K\rhd \cbr{e}$.
    \item 对称群$S_3$可解:$S_3\rhd A_3=\cbr{(1),(123),(132)}\rhd \cbr{(1)}$.
    \item 对称群$S_4$可解:$S_4\rhd A_4\rhd \cbr{(1),(12)(34),(13)(24),(14)(23)}\cong K_4\rhd \cbr{e}$.\footnote{由于$S_n$中轮换$\sigma=(a_1,\cdots,a_k)$的共轭$\tau\sigma\tau\rev=(a_1\tau\rev,\cdots,a_k\tau\rev)$,从而$S_n$中置换共轭等价于置换的类type相同.而$S_4$中所有类为$[2,2]$的置换生成的群即为$K_4$.}
\end{itemize}

\section{环}
\subsection{素元与不可约元}
\begin{itemize}
    \item 素理想$P\lneq R: ab\in P\implies a\in P\lor b\in P$(对理想也成立),等价于$R/P$是整环.
    \item 极大理想$M\lneq R$:不存在含$M$的真理想,等价于$R/M$是域.
    \item 极大理想都是素理想, PID中极大理想$=$非零素理想.\footnote{$(a)$是非零素理想,若有$(a)\subset (b)$则有$a=bc$.$b\in (a)$则$(b)=(a)$;$c\in (a)$则$c=da, a=bda, bd=1\in (b)=R$.}
    \item $a|b\iff (a)\supset (b)$.
    \item $p$是素元即$p|ab\implies p|a\lor p|b$,等价于$(p)$为素理想.
    \item $m$为不可约元即$m=ab\implies a=1\lor b=1$,等价于$(m)$为主理想中极大理想.
    \item 不可约元都是素元, UFD中素元也是不可约元.
\end{itemize}
\subsection{UFD, PID和Euclid环}
\paragraph{UFD的等价条件}$R$是UFD$\iff R$中主理想满足升链条件,且不可约元均为素元.
\paragraph{PID是UFD}

\subsection{中国剩余定理(CRT)}

\section{模}
\subsection{图追踪法}

\subsection{有限生成交换群的结构}

\subsection{Jordan标准型的存在性}

\section{域}
\subsection{尺规作图}

\subsection{有限域的结构与构造}
\end{document}