
\documentclass[11pt,a4paper]{article}
\usepackage{ctex} %%%引入中文包
\usepackage{pifont}
\usepackage{pifont}
\usepackage{graphicx}
\usepackage{amsmath, amsthm}
\usepackage{amssymb}
\usepackage{amsfonts}
\usepackage{mathrsfs}
\usepackage{rotating}
\usepackage{fancyhdr}
\graphicspath{{./}{./figs/}}
\usepackage[
%showframe,
%paperwidth=235mm,paperheight=280mm,
top=15mm,
bottom=30mm,
left=25mm,right=25mm
]{geometry}
%\oddsidemargin=-10pt
%\evensidemargin=-20pt
\usepackage{CJK}

\renewcommand{\baselinestretch}{1.5}
\renewcommand{\arraystretch}{0.8}
\newcommand{\D}{\displaystyle}
\newcommand{\DF}[2]{\frac{\D#1}{\D#2}}
\def\la{$\lambda$}
\def \e{$\varepsilon$}
\def \de{$\delta$}
\def \tr{$\triangle$}
\def \li{$\lim_{x \to x_0}$}
\def \w{$\omega$}
\def \R{$\mathbb{R}$}
\def \p{$\partial$}
\def \De{$\Delta$}
\def\al{$\alpha$}
\def\d {{$\rm d}$}
\def\dx{$\d x$}
\def\dy{$\d y$}
\def\r{$\vec{r}$}
\def\pinf{${+\infty$}}
\def\k{$\kaishu$}
\def\Int{$\D{\int}$}
\def\q{\quad}
\def\A{$\mathscr{A}$}
\def\Lim{$\lim\limits$}
\newcommand{\intt}[2]{\D{\int}_{#1}^{#2}}
\def\theend{\hfill $\blacksquare$}
\newcommand{\sm}{\sum\limits}
\newcommand{\pian}[2]{\DF{\partial\,#1}{\partial\,#2}}



%\usepackage{lastpage}
\usepackage{fancyhdr}
\pagestyle{fancy}
\renewcommand{\headrulewidth}{0pt}

\begin{document}
	
	
	\fancyhf{}
	
	\cfoot{{\footnotesize《高等代数(下)》第4章单元测试试卷\quad 第\ \thepage \ 页\quad 共\ 4\ 页}}
	
	
	
	\vspace{0em}
	\begin{center}
		{\bf \Large 安徽大学2024-2025学年第二学期}
		
		\vspace{2mm}
		
		{\bf \Large {《高等代数(下)}》第4单元测试试卷}
		
		\vspace{5mm}
		
		{\large( 闭卷 \quad 时间 120分钟)}
		
		\vspace{0.5cm}
		
		姓名$\underline{~~~~~~~~~~~~~~~~~~~~}$
		学号$\underline{~~~~~~~~~~~~~~~~~~~~}$\\
		\vspace{0.3cm}
	\end{center}
	\setlength{\parindent}{0pt}
	{\bf 一、填空题}{(每小题5分, 共15分).}
	\renewcommand{\arraystretch}{1.2}
	\hfill
	\begin{tabular}{|c|c|}
		\hline
		\makebox[2.5em]{得分} & \makebox[3.0em]{} \\
		\hline
	\end{tabular}
	\renewcommand{\arraystretch}{1.0}
	\vspace{0.1cm}
	
	1.  设 $ V$ 是数域$\mathbb{F}$ 上的一维空间,写出$ V$ 上所有的线性变换 $\underline{~~~~~~~~~~~~~~~}$.
	
	2. 十维向量空间$  V  $上的线性变换$  \varphi $ 在一组基下的表示矩阵为  $\boldsymbol{A}  $,已知齐次线性方程组 $ A x=0 $ 的解空间维数为 $3 $,则  $\operatorname{dim} \operatorname{Im} \varphi=\underline{~~~~~~~~~~~~~~~}$.
	
	
	3.设 $ V$  是 $ n $维向量空间,则$V$上线性变换全体组成的向量空间的维数为 $\underline{~~~~~~~~~~~~~~~}$.
	\vspace{0.1cm}
	\renewcommand{\arraystretch}{0.8}
	
	{\bf 二、选择题}
	{(每小题5分, 共15分).}
	\renewcommand{\arraystretch}{1.2}
	\hfill
	\begin{tabular}{|c|c|}
		\hline
		\makebox[2.5em]{得分} & \makebox[3.0em]{} \\
		\hline
	\end{tabular}
	\renewcommand{\arraystretch}{1.0}
	
	
	\vspace{0.2cm}
	4. 设  $\varphi$  是  $n$  维向量空间  $V$  上的线性变换, 适合下列条件的  $\varphi$  不是同构的是(\quad\quad\quad ).\\
	(A) $\varphi$  是单映射; \ \ (B)  $\operatorname{dim} \operatorname{Im} \varphi=n$;\par
	(C) $\varphi$  是一一对应; \ \ (D)  $\varphi$  适合条件  $\varphi^{n}=0$.\\
	
	5. 设  $n$  维向量空间  $V$  有一组基, 使得这组基的每个基向量生成的子空间都是  $V$  上线性变换  $\varphi$  的不变子空间, 则  $\varphi$  在这组基下的表示矩阵(\quad\quad\quad ).\par
	(A) 必是可逆矩阵; \ \ (B)  必是上三角矩阵但不一定是对角矩阵;\ \ \par
	(C) 必是下三角矩阵但不一定是对角矩阵; \ \ (D)  必是对角矩阵.\\
	
	
	6. 下列条件不能保证  $n$  维向量空间  $V$  上的非零线性变换  $\varphi$  为可逆变换的是(\quad\quad\quad ).\par
	(A)  $\varphi$  在  $V$  的某组基下的表示矩阵的行列式不为零; \ \  (B)   $\varphi$  在  $V$  的某组基下的表示矩阵是一个对称矩阵;\ \ \par
	(C)  $\varphi$  将  $V$  的  $n$  个线性无关的向量变成  $n$  个线性无关的向量; \ \ (D) $\varphi$  没有非平凡不变子空间.\\
	
	\renewcommand{\arraystretch}{1.5}
	
	\vspace{0.2cm}
	
	%\vspace{0.1cm}
	

	
	
	{\bf 三、计算题}
	{\CJKfamily{kai}(每小题10分, 共40分).}
	\renewcommand{\arraystretch}{1.2}
	\hfill
	\begin{tabular}{|c|c|}
		\hline
		\makebox[2.5em]{得分} & \makebox[3.0em]{} \\
		\hline
	\end{tabular}
	\renewcommand{\arraystretch}{1.0}\\

	7.设线性空间$  V  $上的线性变换$  \varphi $ 在基$  \left\{e_{1}, e_{2}, e_{3}, e_{4}\right\}  $下的表示矩阵为$\boldsymbol{A}=\left(\begin{array}{cccc}1 & 0 & 2 & -1 \\0 & 1 & 4 & -2 \\2 & -1 & 0 & 1 \\2 & -1 & -1 & 2\end{array}\right)$,
	$ U=L\left(e_{1}+2 e_{2}, e_{3}+e_{4}, e_{1}+e_{2}\right) $,求$U$的维数并证明$U$ 是 $ \varphi $ 的不变子空间.
	


	\vspace*{8cm}
	
	
	
	
	
	
	
	
	8.设线性空间$ V $ 上的线性变换 $ \varphi$ 在基$\left\{e_{1}, e_{2}, e_{3}, e_{4}\right\} $下的表示矩阵为\\$\boldsymbol{A}=\left(\begin{array}{cccc}1 & 0 & 2 & 1 \\-1 & 2 & 1 & 3 \\1 & 2 & 5 & 5 \\2 & -2 & 1 & -2\end{array}\right)$\\ 求 $\varphi$ 的核空间与像空间(用基的线性组合来表示).
	
	
	
	
	\vspace*{9cm}
	
	9. 设$  V, W  $是数域$  P  $上的线性空间,$ \varphi: V \rightarrow W  $是线性映射且在$  V $ 的基  $\varepsilon_{1}, \varepsilon_{2}, \varepsilon_{3}, \varepsilon_{4}  $下的矩阵为$  A=\left\{\begin{array}{cccc}1 & 2 & 3 & 4 \\ 2 & 3 & 4 & 5 \\ 3 & 4 & 5 & 6\end{array}\right\}  $.求 $ P^{4} $ 的基  $\alpha_{1}, \alpha_{2}, \alpha_{3}, \alpha_{4} , P^{3} $ 的基 $ \beta_{1}, \beta_{2}, \beta_{3} $ ,使得 $ \varphi$  在这两个基下
	的矩阵形如
	  $\left(\begin{array}{cc}I_{r} & O \\ O & O\end{array}\right) $ .
	
	
	
	\vspace*{10cm}
	
	
	10.设$  V  $是由数域$  \mathbb{F}  $上的二阶矩阵全体组成的向量空间,定义$  V $ 上的线性变换 $ \varphi  $如下:\\$\varphi(\boldsymbol{A})=\left(\begin{array}{ll}1 & 1 \\1 & 1\end{array}\right) \boldsymbol{A}\left(\begin{array}{ll}2 & 0 \\0 & 1\end{array}\right)$,求 $ \varphi $ 的秩和零度.
	
	
	
	
	
	
	\vspace*{10cm}
	
	{\bf 四、证明题}
	{\CJKfamily{kai}(每小题10分, 共30分).}
	\renewcommand{\arraystretch}{1.2}
	\hfill
	\begin{tabular}{|c|c|}
		\hline
		\makebox[2.5em]{得分} & \makebox[3.0em]{} \\
		\hline
	\end{tabular}
	\renewcommand{\arraystretch}{1.0}\\
	11. 设  $\mathscr{A}$  是有限维线性空间  $V$  的线性变换, $W$  是  $V$  的子空间,  $\mathscr{A} W$  表示  $W$  中向量的像组成的子空间. 证明 $\operatorname{dim} \mathscr{A} W+\operatorname{dim}\left(\mathscr{A}^{-1}(0) \cap W\right)=\operatorname{dim} W$.
\vspace*{7cm}
	
	12. 设  $V$  是数域  $\mathbb{P}$  上  $n$  维线性空间.证明: $V$  上的与全体线性变换可以交换的线性变换是数乘变换.
	
	
	\vspace*{7cm}
	
	
	13. 设  $\mathscr{A}^{2}=\mathscr{A}, \mathscr{B}^{2}=\mathscr{B}$. 证明: \par
	(1) $\mathscr{A}$  与   $\mathscr{B}$  有相同值域的充要条件是  $\mathscr{A}  \mathscr{B}= \mathscr{B},  \mathscr{B} \mathscr{A}=\mathscr{A}$;\par
	(2) $\mathscr{A}$ 与   $\mathscr{B}$  有相同的核充要条件是  $\mathscr{A}  \mathscr{B}=\mathscr{A},  \mathscr{B} \mathscr{A}= \mathscr{B}$.
	
	
	
\end{document}

%%% Local Variables:
%%% mode: latex
%%% TeX-master: t
%%% End:

