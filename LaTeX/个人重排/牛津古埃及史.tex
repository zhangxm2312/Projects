\documentclass[UTF8]{book}
\usepackage{ctex,geometry}
% \usepackage{amssymb,amsmath,amsthm,amscd,latexsym,ctex}
% \usepackage{tikz,tikz-cd,pgfplots,geometry,enumitem,bm}
% \usepackage{mhchem}
\geometry{a4paper,left=2cm,right=2cm,top=2cm,bottom=2cm}
\title{牛津古埃及史\\ \Large{The Oxford History of Ancient Egypt}}
\date{\today}
\author{Edited by \textbf{Ian Shaw}}
\begin{document}
    \maketitle
    \newpage
    \begin{center}
        \vspace{10cm}
        \LARGE{THE OXFORD HISTORY OF ANCIENT EGYPT}

        \vspace{1cm}\Large{THE EDITOR}

        \vspace{0.6cm}\large{Ian Shaw is a Lecturer in Egyptian Archaeology at the University of Liverpool}
    \end{center}
    \newpage
    \section*{PREFACE}
    This book describes the emergence and development of the distinctive civilization of the ancient Egyptians, from their prehistoric origins to their incorporation into the Roman empire. In 1961 Alan Gardiner's Egypt of the Pharaohs presented a fresh and detailed view of Egyptian history, based on the textual and archaeological data then available. Gardiner's history was largely concerned with the activities of kings, governments, and high officials through the centuries, from the begin- ning of the pharaonic period until the arrival of the Ptolemies. The Oxford History of Ancient Egypt, however, is concerned not only with political change but also with social and economic developments, processes of religious and ideological change, and trends in material culture, whether in the form of architectural styles, techniques of mummification, or the fabrics of ceramics. This more wide-ranging historical picture draws on the new types of evidence that have become available as archaeologists have begun to survey and excavate types of sites that were previously neglected. 
    
    Each chapter describes and analyses a particular phase in ancient Egyptian history. The contributors outline the principal sequence of political events, traces of which have survived to varying degrees in the textual record. However, against this backdrop of the rise and fall of ruling dynasties, they also examine the cultural and social patterns, including stylistic developments in art and literature. This allows them to compare and contrast purely political phases with archaeological and anthropological evidence ranging from the changing styles of pottery to human mortality rates. Each contributor seeks to explore not only which aspects of culture change at different points in time, but also why some change more rapidly than others or remain surprisingly stable at times of political disruption. A major influence on all of the chapters, however, is the patchiness of the archaeological record, which means that some sites and periods can be viewed through a huge number of different types of sources, while others can be only tentatively reconstructed, because of a lack of certain kinds of evidence (through poor preservation, inadequate excavation, or a combination of both). Because each of the periods in Egyptian history is no more or less than the sum of its archaeological and textual parts, the individual chapters in this history are direct reflections of such abundance or inadequacy, and the differences in authors' style, emphasis, and content can largely be traced back to the nature of the evidence with which they are dealing.

    Although the sequence of chapters takes the form of a relatively straightforward historical progression from the Palaeolithic to the Roman period, the various sections incorporate critical approachs to each of the phases, sometimes questioning whether they deserve to be regarded as discrete chronological units, or whether there are broader trends in material culture that transcend (or even conflict with) the perceived political framework. It has been pointed out, for instance, that the decreasing size of royal pyramid complexes after the 4th Dynasty need not be evidence of a decline in royal power, as most historians have tended to assume, but might, on the contrary, indicate a more efficient use of resources in the late Old Kingdom and First Intermediate Period.
    The pace of change in such aspects of Egyptian culture as monu- mental architecture, funerary beliefs, and ethnicity was not necessarily tied to the rate of political change. Each of the authors of this history has set out to elucidate the underlying patterns of social and political change and to describe, with due regard to the dangers of archaeo- logical and textual distortion and bias, the changing face of Egyptian culture, from the biographical details of individuals to the social and economic factors that shaped the lives of the population as a whole.

    
    \raggedleft
    IAN SHAW
    
    \raggedright
    School of Archaeology, Classics and Oriental Studies,

    The University of Liverpool

    31 January 2000
    \section*{ACKNOWLEDGEMENTS}
    I am most grateful to Hilary O'Shea, Senior Editor for Ancient History at Oxford University Press, for her help in the early stages of this book in particular.
    
    Janine Bourriau would like to thank Manfred Bietak, Irmgard Hein, and David Aston for generously allowing her to draw on unpublished information on the current excavations at the site of Avaris (Tell el- Dab$^e$a).
    \tableofcontents
    \newpage
    \renewcommand\thesection{\Roman{section}}

    \section{Introduction: Chronologies and Cultural Change in Egypt}%\raggedright \emph{Ian Shaw}}
    \centering IAN SHAW
    \raggedright

    All history is clearly reliant on some form of chronological framework, and a great deal of time has been spent on the construction of such dating systems for ancient Egypt. Ever since the first Western-style history of Egypt was written by an Egyptian priest called Manetho in the third century BC, the 'pharaonic period', from c.3100 to 332 BC, has been divided into a number of periods known as 'dynasties', each con- sisting of a sequence of rulers, usually united by such factors as kin- ship or the location of their principal royal residence. This essentially political approach has served very well over the years as a way of divid- ing up Egyptian chronology into a series of convenient blocks, each with its own distinctive characteristics. It is, however, becoming increasingly difficult to reconcile this politically based chronology with the social and cultural changes revealed by excavations since the 1960s.
    \paragraph{Chronology}
    As Egyptian historical and archaeological data have expanded and durable, and convenient though it is—often strains to contain the many new chronological trends and currents that can be perceived outside the simple passing of the throne from one group of individuals to another. Some of the new work shows that at many points in time Egypt was far less culturally unified and centralized than was previously assumed, with cultural and political changes taking place at different speeds in the various regions. Other analyses show that short-term political events, which have often tended to be regarded as the para- mount factors in history, may often be less historically significant than the gradual processes of socio-economic change that can transform the cultural landscape more overwhelmingly in the long term. Just as the long 'pre-Dynastic' periods of Egyptian prehistory are commonly understood as sequences of cultural rather than political develop- ments, so the Dynastic Period (as well as the Ptolemaic and Roman periods) has begun to be understood not only in terms of the tradi- tional sequence of individual kings and ruling families but also in terms of such factors as the types of fabric being used for pottery, and the painted decoration applied to wooden coffins.
\end{document}
    Modern Egyptologists' chronologies of ancient Egypt combine three basic approaches. First, there are 'relative' dating methods, such as stratigraphic excavation, or the 'sequence dating' of artefacts, which was invented by Flinders Petrie in 1899. In the late twentieth century, as archaeologists have developed a more subtle understanding of the ways in which the materials and design of different types of Egyptian artefacts (particularly ceramics) changed over time, it has become possible to apply forms of seriation to many different types of object. Thus Harco Willems's seriation of Middle Kingdom coffins, for instance, has pro- vided a better understanding of cultural changes in the various provin- ces of iith-i3th-Dynasty Egypt, complementing the information already available about national political change during the same period. Secondly, there are so-called absolute chronologies, based on calen- drical and astronomical records obtained from ancient texts. Thirdly, there are 'radiometric' methods (the most commonly used examples of which are radiocarbon dating and thermoluminescence), by means of which particular types of artefacts or organic remains can be assigned dates based on the measurement of radioactive decay or accumulation.

    \paragraph{Radiocarbon Dating and Egyptian Chronology}
The relationship between the calendrical and radiometric chronological systems has been relatively ambivalent over the years. Since the late
1940s, when a series of Egyptian artefacts were used as a benchmark
in order to assess the reliability of the newly invented radiocarbon
dating technique, a consensus has emerged that the two systems are
broadly in line. The major problem, however, is that the traditional
INTRODUCTION 3
calendrical system of dating, whatever its failings, virtually always has
a smaller margin of error than radiocarbon dates, which are neces-
sarily quoted in terms of a broad band of dates (that is, one or two
standard deviations), never capable of pinpointing the construction of
a building or the making of an artefact to a specific year (or even a
specific decade). Certainly the advent of dendrochronological calibra-
tion curves—allowing the spans of radiocarbon years to be converted
into actual calendar years—represents a significant improvement in
terms of accuracy. However, the vagaries of the curve and the con-
tinued need to take into account associated error mean that dates must
still be quoted as a range of possibilities rather than one specific year.
The prehistory of Egypt, on the other hand, has benefited greatly
from the application of radiometric dating, since it was previously reli-
ant on relative dating methods (see Chapters 2 and 3). The radiometric
techniques have made it possible not only to place Petrie's 'sequence
dates' within a framework of absolute dates (however imprecise), but
also to push Egyptian chronology back into the earlier Neolithic and
Palaeolithic periods.
From Prehistory to History: Late Predynastic Artefacts and the
Palermo Stone
There are only a small number of artefacts from the late Predynastic
Period that can be used as historical sources, documenting the trans-
ition into full unified statehood. These are funerary stelae, votive
palettes, ceremonial maceheads, and small labels (of wood, ivory, or
bone) originally attached to items of elite funerary equipment. In the
case of the stelae, palettes, and maceheads, it was clearly the intention
that they should commemorate many different kinds of royal act,
whether the King's own death and burial or his act of devotion to one of
the gods or goddesses. Some of the smaller, earlier labels (particularly
those recently excavated from the late Predynastic 'royal tomb' U-j at
Abydos, see Chapter 4) are simply records of the nature or origins of
the grave goods to which they were attached, but some of the later
labels, from the Early Dynastic royal graves at Abydos, employ a simi-
lar repertoire of depictions of royal acts in order to assign the items in
question to a particular date in the reign of a specific king.
If the purpose of this mobiliary art of the late fourth and early third
millennia BC was to label, commemorate, and date, then their decora-
tion has to be seen as resulting from the desire to communicate the
'context' of the object in terms of event and ritual. Nick Millet has
4  IAN SHAW
particularly demonstrated this in his analysis of the Narrner mace-
head, which was part of a group of late Predynastic and early pharaonic
votive items (including the Narmer Palette and Scorpion Macehead)
excavated by Quibell and Green in the temple precinct at Hierakon-
polis. The analysis of the scenes and texts on these objects is com-
plicated by our modern need to be able to distinguish between event
and ritual. But the ancient Egyptians show little inclination to dis-
tinguish consistently between the two, and indeed it might be argued
that Egyptian ideology during the pharaonic period—particularly in so
far as it related to the kingship—was reliant on the maintenance of
some degree of confusion between real happenings and purely ritual
or magical acts.
With regard to the palettes and maceheads, the Canadian Egypt-
ologist Donald Redford suggests that there must have been a need to
commemorate the unique events of the unification at the end of the
third millennium BC, but that these events were 'commemorated'
rather than 'narrated'. This distinction is a crucial one: we cannot
expect to disentangle 'historical' events from scenes that are com-
memorative rather than descriptive—or, at least, if we do so we may
often be misled.
One of the most important historical sources for the Early Dynastic
Period (3000-2686 BC) and the Old Kingdom (2686-2160 BC) is the
Palermo Stone, part of a 5th-Dynasty basalt stele (£.2400 BC) inscribed
on both sides with royal annals stretching back to the mythical
prehistoric rulers. The main fragment has been known since 1866 and
is currently in the collection of the Palermo Archaeological Museum,
Sicily, although there are also further pieces in the Egyptian Museum,
Cairo, and the Petrie Museum, London. The slab must originally have
been about 2.1 m. long and 0.6 m. wide, but most of it is now missing,
and there is no surviving information about its provenance. This docu-
ment—along with the 'day-books', the annals and 'king-lists' inscribed
on temple walls, and the papyri held in temple and palace archives—
was doubtless the kind of document that Manetho consulted when he
was compiling his history or Aegyptiaca.
The text of the Palermo Stone enumerates the annals of the kings of
Lower Egypt, beginning with the many thousands of years that were
assumed to have been taken up by mythological rulers, until the time
of the god Horus, who is said to have given the throne to the human
king Menes. Human rulers are then listed up to the 5th Dynasty. The
text is divided into a series of horizontal registers divided by vertical
lines that curve in at the top, apparently in imitation of the hieroglyph
INTRODUCTION 5
for regnal year (renpet), thus indicating the memorable events of indi-
vidual years in each king's reign. The situation is slightly confused by
the fact that many Old Kingdom dates appear to refer to the number of
biennial cattle censuses (hesbet) rather than to the number of years that
the king had reigned; therefore the number of'years' in the Old King-
dom dates may well have to be doubled to find out the actual number
of regnal years.
The types of event that are recorded on the Palermo Stone are cult
ceremonies, taxation, sculpture, building, and warfare—that is, pre-
cisely the type of phenomena that are recorded on the protodynastic
ivory and ebony labels from Abydos, Saqqara, and various other early
historical sites. The introduction of the renpet sign on the labels, in the
reign of Djet, makes this comparison even closer. There are two differ-
ences, however: first, the labels include clerical information, while the
Palermo Stone does not, and, secondly, the Palermo Stone includes
records of the Nile inundation, whereas the labels do not. Both of these
types of information seem to have occupied the same physical part of
the document's format—that is, the bottom of the record. Redford
suggests that this shows that the Old Kingdom genut (the royal annals
that are assumed to have existed at this date, but have not survived
except in the form of the Palermo Stone) were concerned with
hydraulic/climatic change, which, with its crucial agricultural and eco-
nomic consequences, was potentially the most important aspect of
change as far as the reputation of each individual king was concerned.
This kind of hydraulic information would, however, have perhaps
been regarded as irrelevant to the function of the labels attached to
funerary equipment.
King-Lists, Royal Titles, and the Divine Kingship
Apart from the Palermo Stone, the basic sources used by Egyptologists
to construct the traditional chronology of political change in Egypt
are Manetho's history (which, unfortunately, has survived only in the
form of excerpts compiled by the later authors Josephus, Africanus,
Eusebius, and George Syncellus), the so-called king-lists, dated records
of astronomical observations, textual and artistic documents (such as
reliefs and stelae) bearing descriptions apparently referring to histori-
cal events, genealogical information, and synchronisms with non-
Egyptian sources, such as the Assyrian king-lists. For the 28th~3oth
Dynasties, the Demotic Chronicle (Papyrus 215 in the Bibliotheque
Nationale, Paris) serves as a unique early Ptolemaic source concerning
6  IAN SHAW
political events in this last phase of the Late Period, compensating to
some extent for the dearth of historical information provided by the
papyri and monuments of this date (as well as the fact that Manetho
gives only the names and reign lengths of the kings). Wilhelm Spiegel-
berg and Janet Johnson have shown that careful translation and
interpretion of the 'oracular statements' in this pseudo-prophetic
document can shed new light not only on the events of the period
(such as the suspected co-regency between Nectanebo I and his son
Tachos) but also on the ideological and political context of the fourth
century BC.
Like most other ancient peoples, the ancient Egyptians dated impor-
tant political and religious events not according to the number of years
that had elapsed since a single fixed point in history (such as the birth
of Christ in the modern Western calendar), but in terms of the years
since the accession of each current king (regnal years). Dates were,
therefore, recorded in the following typical format: 'day 2 of the first
month of the season per&t in the fifth year of Nebmaatra (Amenhotep
III)'. It is important to be aware of the fact that, for the Egyptians, the
reign of each new king represented a new beginning, not merely philo-
sophically but practically, given the fact that dates were expressed in
such terms. This means that there would probably have been a psycho-
logical tendency to regard each new reign as a fresh point of origin:
every king was, therefore, essentially reworking the same universal
myths of kingship within the events of his own time.
One important aspect of the Egyptian kingship throughout the
pharaonic period was the existence of a number of different names for
each individual ruler. By the Middle Kingdom, each king held five
names (the so-called fivefold titulary), each of which encapsulated a
particular aspect of the kingship: three of them stressed his role as a
god, while the other two emphasized the supposed division of Egypt
into two unified lands. The birth name (or nomen), such as Rameses or
Mentuhotep, introduced by the title 'son of Ra', was the only one to be
given to the pharaoh as soon as he was born. It was also usually the last
name given in inscriptions identifying the king by his whole sequence
of names and titles. The other four names—Horus, nebty ('he of the two
ladies'), (Horus of) Gold, and nesu-bit ('he of the sedge and the bee')—
were given to him at the time of his installation on the throne, and
their components may sometimes convey something of the ideology
or intentions of the king in question. As far as the rulers of Dynasty o
and the beginning of the Early Dynastic Period were concerned, we
know only their 'Horus names', typically written inside a serekh frame
INTRODUCTION 7
(a kind of diagram of the palace gateway), upon which a Horus-falcon
was perched. It was the late ist-Dynasty ruler Den (c.29oo BC) who was
the first to hold a nesu-bit name (Khasty), but it was not until the reign
of Sneferu, 2613-2589 BC, in the 4th Dynasty, that this name was first
framed by the familiar cartouche shape (an encircling loop that perhaps
signified the infinite extent of the royal domain).
The title nesu-bit has often been translated as 'King of Upper and
Lower Egypt', but it actually has a much more complex and significant
meaning. Nesu seems to be intended to refer to the unchanging divine
king (almost the kingship itself), while the word bit describes the
current ephemeral holder of the kingship: the one individual king in
power at a specific point in time. Each king was, therefore, a combina-
tion of the divine and the mortal, the nesu and the bit, in the same way
that the living king was linked with Horus, and the dead kings, the
royal ancestors, were associated with Horus' father Osiris. It was pri-
marily because of the Egyptians' sense of each of their kings as incar-
nations of Horus and Osiris that the tradition of the worship of divine
royal ancestors developed. This convention, whereby the current ruler
made obeisance to his predecessors, is the reason for the creation
of the so-called king-lists, which were lists of royal names mainly
recorded on the walls of tombs and temples (most notably the iQth-
Dynasty temples of Sety I and Rameses II at Abydos), but also in the
form of papyri, only one example of which survives (the so-called Turin
Canon), or remote desert rock carvings, as with the list at the Wadi
Hammamat siltstone quarries in the Eastern Desert. The continuity
and stability of the kingship were preserved by making offerings to all
those kings of the past who were regarded as legitimate rulers, just as
we see Sety I doing in his cult temple at Abydos. It is usually presumed
that king-lists were among the sources used by Manetho in compiling
his history.
The Turin Canon, a Ramessid papyrus dating to the thirteenth
century BC, is the most informative of the Egyptian king-lists. From the
Second Intermediate Period (1650-1550 BC), it stretched back with
reasonable accuracy to the reign of the ist-Dynasty ruler Menes
(^.3000 BC), and even beyond that into a mythical prehistoric time
when the gods ruled over Egypt. Each king's reign was recorded in
terms of years, months, and days. It also provides some support for
Manetho's system of dynasties by incorporating a break at the end of
the 5th Dynasty (see Chapter 5).
The king-lists were not concerned so much with history as with
ancestor worship: the past is presented as a combination of the general
8  IAN SHAW
and the individual, and the constancy and universality of the kingship
are celebrated through the listing of specific individual holders of the
royal titulary. In his commentary on Herodotus Book II, Alan Lloyd
writes, 'Since all historical study involves general and particular,
attempting to place particular phenomena against a background of
general principle or law, there is always a tension between the two, and
this tension is resolved in Egypt overwhelmingly in favour of the
latter.' The conflict between the general and the particular is undoubt-
edly an important factor in ancient Egyptian chronology and history.
The texts and artefacts that form the basis of Egyptian history usually
convey information that is either general (mythological or ritualistic)
or particular (historical), and the trick in constructing a historical
narrative is to distinguish as clearly as possible between these types of
information, taking into account the Egyptians' tendency to blur the
boundaries between the two.
The Swiss Egyptologist Erik Hornung describes Egyptian history as
a kind of'celebration' of both continuity and change. Just as the living
king could be regarded as synonymous with the falcon-god Horus, so
his individual subjects (from at least the First Intermediate Period
onwards) eventually came to identify themselves with the god Osiris
after their deaths. In other words, the Egyptians were used to the idea
of portraying human individuals as combinations of the general and
the particular. Their own sense of history therefore comprised both the
specific and the universal in equal measure.
The Role of Astronomy in Traditional Egyptian Chronology
The task of the modern historian of ancient Egypt is usually to attempt
to tie together all the strands of evidence in the form of individuals'
biographies on the walls of tombs, lists of kings on temple walls, strati-
graphic evidence of archaeological excavations, and a whole range of
other pieces of information. In the pharaonic, Ptolemaic, and Roman
periods, the 'traditional' absolute chronologies tend to rely on complex
webs of textual references, combining such elements as names, dates,
and genealogical information into an overall historical framework that
is more reliable in some periods than in others. The so-called inter-
mediate periods have proved to be particularly awkward phases, partly
because there was often more than one ruler or dynasty reigning
simultaneously in different parts of the country. The surviving records
of observations of the heliacal rising of the dog-star Sirius serve both
INTRODUCTION 9
as the linchpin of the reconstruction of the Egyptian calendar and its
essential link with the chronology as a whole.
The goddess Sopdet, known as Sothis in the Graeco-Roman period
(332 BC-AD 395), was the personification of the 'dog-star', which the
Greeks called Seirios (Sirius). She was usually represented as a woman
with a star poised on her head, although the earliest depiction, on an
ivory tablet of the ist-Dynasty king Djer (0.3000 BC) from Abydos,
appears to show her as a seated cow with a plant between her horns.
Since a depiction of a plant is used as the ideogram meaning 'year' in
the pharaonic writing system, the Egyptians may have already been
correlating the rising of the dog-star with the beginning of the solar
year, even in the early third millennium BC. Along with her husband
Sah (Orion) and her son Soped, Sopdet was part of a triad that paral-
leled the family of Osiris, Isis, and Horus. She was therefore described
in the Pyramid Texts as having united with Osiris to give birth to the
morning star.
In the Egyptian calendrical system, Sopdet was the most important
of the stars or constellations known as decans, and the 'Sothic rising'
coincided with the beginning of the solar year only once every 1,460
years (or, more accurately, 1,456 years). We know that this rare syn-
chronization of the heliacal rising of Sopdet with the beginning of
the Egyptian civil year (or 'wandering year', as it is sometimes
described, given that it gradually falls behind the solar year at a rate of
about a day every four years) took place in AD 139, during the reign of
the Roman emperor Antoninus Pius, because the event was com-
memorated by the issue of a special coin at Alexandria. There would
have been earlier heliacal risings in 1321-1317 BC and 2781-2777 BC,
and the period that elapsed between each such rising is known as a
Sothic cycle.
Two Egyptian textual records of Sothic risings (dating to the reigns
of Senusret III and Amenhotep I) form the basis of the conventional
chronology of Egypt, which, in turn, influences that of the whole
Mediterranean region. These two documents are a 12th-Dynasty letter
from the site of Lahun, written on day 16, month 4, of the second
season in year 7 of the reign of Senusret III, and an 18th-Dynasty
Theban medical papyrus (Papyrus Ebers), written on day 9, month 3,
of the third season of year 9 in the reign of Amenhotep I. By assigning
absolute dates to each of these documents (1872 BC for the Lahun
rising in year 7 of Senusret III, and 1541 BC for the Ebers rising in regnal
year 9 of Amenhotep I), Egyptologists have been able to extrapolate a
set of absolute dates for the whole of the pharaonic period, on the basis
10 IAN SHAW
of records of the lengths of reign of the other kings of the Middle and
New kingdoms.
It is not possible, however, to be totally confident of the absolute
dates cited above, since the precise dating is dependent on our knowl-
edge of the location (or locations) where the astronomical observations
were made. It used to be assumed—without any real evidence—that
such observations were made at Memphis or perhaps Thebes, but
Detlef Franke and Rolf Krauss have argued that they were all made at
Elephantine. William Ward, on the other hand, suggested that they are
all more likely to have been separate local observations, which would
have resulted in a time lag in terms of the various 'national' religious
festivals (that is, both the observations and the corresponding festivals
may actually have taken place at different times and in different parts
of the country). This continuing uncertainty means that our astro-
nomical linchpins are in reality somewhat floating, although it should
be noted that the differences between the 'high' and 'low' chronologies
(based largely on assumptions concerning different observation points)
are usually only a few decades at most.
Co-Regencies
One of the peculiarities of Egyptian chronology, provoking both con-
fusion and debate, is the concept of the 'co-regency', a modern term
applied to the periods during which two kings were simultaneously
ruling, usually consisting of an overlap of several years between the
end of one sole reign and the beginning of the next. This system seems
to have been used, from at least as early as the Middle Kingdom, in
order to ensure that the transfer of power took place with the mini-
mum of disruption and instability. It would also have enabled the
chosen successor to gain experience in the administration before his
predecessor died.
It seems, however, that the dating systems during co-regencies may
have differed from one period to another. Thus i2th-Dynasty co-
regents may have each used separate regnal dates, so that overlaps
occurred between the kings' reigns, producing examples of so-called
double dates, when both dating systems were used to date a single
monument (see Chapter 7). In the New Kingdom, there are no certain
instances of double dates, therefore a different system seems to have
been used. In the reigns of Thutmose III 1479-1425 BC and
Hatshepsut 1473-1458 BC, for instance, year dates appear to have been
counted with reference to Hatshepsut's accession, as if Hatshepsut
INTRODUCTION II
had become ruler at the same time as Thutmose III. It is a moot point
as to whether separate dates were used by two kings during the pos-
sible co-regencies of Thutmose III-Amenhotep II and Amenhotep
III—Amenhotep IV. The arguments for and against a co-regency
between the two latter kings have been carefully reviewed by Donald
Redford and later by William Murnane. However, there is still con-
siderable controversy over the question of which co-regencies actually
took place and how long they lasted. There are also some scholars
(including Gae Callender in Chapter 7 of this volume) who argue that
co-regencies may never have occurred at all.
'Dark Ages' and Other Chronological Problems
Some of the problems encountered in Egyptian chronology have
already been mentioned, such as the potential confusion of links
between astronomical observations and specific dates, the uncertainty
as to which co-regencies (if any) actually occurred, and the assumption
that the Egyptians of the pharaonic period and later continually dated
events according to an artificial 'wandering' civil year of 365 days,
which was rarely synchronized with the real solar year.
There are also, of course, a number of other Egyptian historical
problems, ranging from unreliability of sources (for example, Man-
etho's history, given that we neither know his sources nor have his
original text) and frequent uncertainty regarding lengths of kings'
reigns (for example, the Turin Canon says that Senusret II and III have
reigns of nineteen and thirty-nine years respectively, whereas their
highest recorded regnal years on documents that are actually con-
temporary with their reigns are only six and nineteen).
Egypt, like other cultures, has periods in history that are more or
less documented than others, and it is primarily this patchiness in the
survival of archaeological and textual records from different dates that
has led to the assumption that there were 'intermediate periods', when
the political and social stability of the pharaonic period appeared to
have been temporarily damaged. Thus, those periods of political and
cultural continuity described as the Old, Middle, and New kingdoms
were each thought to be followed by 'dark ages', when the country
became disunited and weakened by conflict (either civil war between
provinces or invasion by foreigners). This scenario was both denied
and bolstered by Manetho's history. First, Manetho created a mislead-
ing air of continuity in the succession of kings and dynasties through
his assumption that only one king could occupy the throne of Egypt at
12 IAN SHAW
any one time. Secondly, his descriptions of some of the dynasties
corresponding to the times of the intermediate periods suggest that
the kingship was changing hands with alarming rapidity.
The study of the Third Intermediate Period has become one of the
most controversial areas of Egyptian history, particularly during the
19905, when it has been subjected to intensive study by a number of
different scholars. Three areas of investigation have blossomed. First,
several aspects of the culture of the period (for example, ceramics and
funerary equipment) have been analysed in terms of changes in such
factors as style and materials. Secondly, anthropological, icono-
graphic, and linguistic studies have been undertaken with regard to
the 'Libyan' ethnic identity of many of the 2ist-24th-Dynasty rulers.
Thirdly, and most crucially from the point of view of the history of the
pharaonic period as a whole, it was argued by a small number of schol-
ars that the period of 400 years occupied by the Third Intermediate
Period (and numerous other, roughly contemporaneous, 'dark ages'
elsewhere in the Near East and the Mediterranean) may have been arti-
ficially inflated by historians. They suggested that the New Kingdom
might have ended not in the eleventh century but in the eighth century
BC, leaving a much smaller gap of about 150 years between the end of
the 2oth Dynasty and the beginning of the Late Period. Such a view,
however, has been widely dismissed, not only because Egyptologists,
Assyriologists, and Aegeanists have been able to refute many of the
individual textual and archaeological arguments for chronological
change, but also, more significantly, because the scientific dating
systems (that is, radiocarbon and dendrochronology) almost always
provide solid independent support for the conventional chronology.
Indeed, the irrelevance of such tinkering with the conventional chron-
ological framework, given the overwhelming and increasing signifi-
cance of scientific dates, has been memorably described by the classical
archaeologist Anthony Snodgrass as 'a bit like a detailed scheme for
re-organizing the East German economy, produced in 1989 or early
1990'.
On a more cultural, rather than chronological level, the significance
of the most basic historical divisions (that is, the distinctions between
the Predynastic, pharaonic, Ptolemaic, and Roman periods) have begun
to be questioned. On the one hand, the results of excavations during
the 19805 and 19905 in the cemeteries of Umm el-Qa e ab (at Abydos)
suggest that before the ist Dynasty there was also a Dynasty o stretch-
ing back for some unknown period into the fourth millennium BC.
This means that, at the very least, the last one or two centuries of the
INTRODUCTION 13
Tredynastic' were probably in many respects politically and socially
'Dynastic'. Conversely, the increasing realization that Naqada III pot-
tery types were still widely used in the Early Dynastic Period shows that
certain cultural aspects of the Predynastic Period continued on into the
pharaonic period (see Chapter 4).
Whereas there are definite political breaks between the pharaonic
and Ptolemaic periods, and between the Ptolemaic and Roman periods,
the gradually increasing archaeological data from the two latter periods
have begun to create a situation where the process of cultural change
may be seen to be less sudden than the purely political records suggest.
Thus it is apparent that there are aspects of the ideology and material
culture of the Ptolemaic Period that remain virtually unaltered by
political upheavals. Instead of the arrival of Alexander the Great and
his general Ptolemy representing a great watershed in Egyptian history,
it might well be argued that, although there were certainly a number of
significant political changes between the mid-first millennium BC and
the mid-first millennium AD, these took place amid comparatively
leisurely processes of social and economic change. Significant ele-
ments of the pharaonic civilization may have survived relatively intact
for several millennia, only undergoing a full combination of cultural
and political transformation at the beginning of the Islamic Period in
AD 641.
Historical Change and Material Culture
There has been an enormous growth in the study of Egyptian pottery
in the late twentieth century, both in terms of the quantity of sherds
being analysed (from a wide variety of types of site) and in terms of the
range of scientific techniques now being used to extract more infor-
mation from ceramics. Inevitably the improvement in our under-
standing of this prolific aspect of Egyptian material culture has had an
impact on the chronological framework. The excavation of part of the
city of Memphis (the site of Kom Rabi e a) in the 19805 provides a good
instance of the ways in which more sophisticated approaches to
pottery have enabled the overall process of cultural change to be better
understood.
Pottery vessels can be arranged in terms of relative date by such
traditional techniques as seriation of cemetery material and the analy-
sis of large quantities of stratified material at domestic or religious
sites, but they can also be given fairly precise absolute dates either by
the conventional method of association with inscribed or artistic
14 IAN SHAW
material (particularly in tombs) or by the use of such scientific tech-
niques as thermoluminescence dating. Some scholars have begun to
study the ways in which vessel and fabric types change over the course
of time. Thus, the form of pottery bread moulds, for instance, under-
went a dramatic change at the end of the Old Kingdom, but it is not yet
clear whether the source of this change lies in the social, economic, or
technological spheres of life, or whether it is merely the result of a
change in 'fashion'. Such analyses show that processes of change in
material culture took place for a whole variety of reasons, only some of
which were linked to the political changes that tend to dominate
conventional views of Egyptian history. This is not to deny the many
connections between political and cultural change, such as the correla-
tion between centralized production of pottery in the Old Kingdom
and resurgence of local pottery types during the more politically frag-
mented First Intermediate Period (and then the renewed homogeniza-
tion of pottery during the more unified i2th Dynasty).
In the study of certain phases of Egyptian history, such as the emer-
gence of the unified state at the beginning of the pharaonic period or
the decline and demise of the Old Kingdom, scholars have sometimes
examined numerous environmental and cultural factors in order to
explain sudden important political changes. One of the problems with
this selective attention to non-political historical trends, however, is the
fact tha t we still know so little about environmental and cultural
change during periods of stability and prosperity, such as the Old and
Middle kingdoms, that it is much more difficult to interpret these
factors at times of political crisis. The increased study of pottery vessels
and other common artefacts (as well as environmental factors such as
climate and agriculture) are beginning to create the basis for more
holistic versions of Egyptian history, in which political narratives are
viewed within the context of long-term processes of cultural change.
Egyptian 'History'
Art and texts throughout the pharaonic period continued to maintain
the Predynastic and Early Dynastic tension between recording and
commemorating, which might be characterized as the distinction
between, on the one hand, the utilitarian labels attached to grave
goods, and, on the other hand, such ceremonial votive items as palettes
and maceheads, described above. We know that the purpose of the
early funerary labels was to use history as a means of dating particular
things, and that the purpose of such mobiliary art as the palettes and
INTRODUCTION 15
maceheads—as well as of stelae and temple reliefs in the pharaonic
period—was not to record historical events but primarily to use them
as a means of commemorating universal acts undertaken by specific
rulers or by royal officials.
In the mortuary temple of Rameses III at Medinet Habu there is a
scene in which the Libyan chieftain Meshesher is brought into the
presence of the king. This is obviously intended to be a record of the
surrender of a particularly important foreign individual, whose per-
sonal humiliation encapsulates the defeat of his people, but to the left-
hand side we can also see the careful assembling and counting of a pile
of Libyans' hands—this alerts us to one of the ways in which the scene
differs from a more modern Western historical tableau. It is part of a
relief in a mortuary temple and as such it is fulfilling the king's piety to
the gods. Just as private individuals in the New Kingdom inscribed
'autobiographical' texts on the walls of their tomb chapels to remind
the gods of their piety and beneficence, so the reliefs in royal mortuary
temples were intended to symbolize a kind of accounting procedure, a
visual quantification of the success achieved by the king both for and
through the gods.
The Egyptian sense of history is one in which rituals and real events
are inseparable—the vocabulary of Egyptian art and text very often
makes no real distinction between the real and the ideal. Thus the
events of history and myth were all regarded as part of a process of
assessment, whereby the king demonstrated that he was preserving
Maat, or harmony, on behalf of the deities. Even when an Egyptian
monument appears to be simply commemorating a specific event in
history, it is often interpreting that event as an act that is simul-
taneously mythological, ritualistic, and economic.


\end{document}
THE OXFORD HISTORY OF ANCIENT
EGYPT
THE EDITOR
Ian Shaw is a Lecturer in Egyptian Archaeology at the University of
Liverpool
THE OXFORD HISTORY
OF ANCIENT EGYPT
Edited by
Ian Shaw
OXPORD
UNIVERSITY PRESS
OXFORD
UNIVERSITY PRESS
Great Clarendon Street, Oxford 0x2 6DP
Oxford University Press is a department of the University of Oxford.
It furthers the University's objective of excellence in research, scholarship,
and education by publishing worldwide in
Oxford New York
Auckland Bangkok Buenos Aires Cape Town Chennai
Dar es Salaam Delhi Hong Kong Istanbul Karachi Kolkata
Kuala Lumpur Madrid Melbourne Mexico City Mumbai Nairobi
Sao Paulo Shanghai Taipei Tokyo Toronto
Oxford is a registered trade mark of Oxford University Press
in the UK and in certain other countries
Published in the United States by
Oxford University Press Inc., New York
© Oxford University Press, 2000
The moral rights of the authors have been asserted
Database right Oxford University Press (maker)
First published in hardback 2000
First published in paperback 2002,
new edition 2003
All rights reserved. No part of this publication may be reproduced,
stored in a retrieval system, or transmitted, in any form or by any means,
without the prior permission in writing of Oxford University Press,
or as expressly permitted by law, or under terms agreed with the appropriate
reprographics rights organization. Enquiries concerning reproduction
outside the scope of the above should be sent to the Rights Department,
Oxford University Press, at the address above
You must not circulate this book in any other binding or cover
and you must impose this same condition on any acquirer
British Library Cataloguing in Publication Data
Data available
Library of Congress Cataloging in Publication Data
Data available
ISBN-I3: 978-0-19-280458-7
11
Typeset in Scala by Footnote Graphics Limited, Warminster, Wilts
Printed in Great Britain by
Clays Ltd, St Ives pic
PREFACE
This book describes the emergence and development of the distinctive
civilization of the ancient Egyptians, from their prehistoric origins to
their incorporation into the Roman empire. In 1961 Alan Gardiner's
Egypt of the Pharaohs presented a fresh and detailed view of Egyptian
history, based on the textual and archaeological data then available.
Gardiner's history was largely concerned with the activities of kings,
governments, and high officials through the centuries, from the begin-
ning of the pharaonic period until the arrival of the Ptolemies. The
Oxford History of Ancient Egypt, however, is concerned not only with
political change but also with social and economic developments,
processes of religious and ideological change, and trends in material
culture, whether in the form of architectural styles, techniques of
mummification, or the fabrics of ceramics. This more wide-ranging
historical picture draws on the new types of evidence that have become
available as archaeologists have begun to survey and excavate types of
sites that were previously neglected.
Each chapter describes and analyses a particular phase in ancient
Egyptian history. The contributors outline the principal sequence of
political events, traces of which have survived to varying degrees in the
textual record. However, against this backdrop of the rise and fall of
ruling dynasties, they also examine the cultural and social patterns,
including stylistic developments in art and literature. This allows them
to compare and contrast purely political phases with archaeological
and anthropological evidence ranging from the changing styles of
pottery to human mortality rates. Each contributor seeks to explore not
only which aspects of culture change at different points in time, but
also why some change more rapidly than others or remain surprisingly
stable at times of political disruption. A major influence on all of the
chapters, however, is the patchiness of the archaeological record,
which means that some sites and periods can be viewed through a
huge number of different types of sources, while others can be only
tentatively reconstructed, because of a lack of certain kinds of evidence
(through poor preservation, inadequate excavation, or a combination
VI PREFACE
of both). Because each of the periods in Egyptian history is no more or
less than the sum of its archaeological and textual parts, the individual
chapters in this history are direct reflections of such abundance or
inadequacy, and the differences in authors' style, emphasis, and
content can largely be traced back to the nature of the evidence with
which they are dealing.
Although the sequence of chapters takes the form of a relatively
straightforward historical progression from the Palaeolithic to the
Roman period, the various sections incorporate critical approachs to
each of the phases, sometimes questioning whether they deserve to be
regarded as discrete chronological units, or whether there are broader
trends in material culture that transcend (or even conflict with) the
perceived political framework. It has been pointed out, for instance,
that the decreasing size of royal pyramid complexes after the 4th
Dynasty need not be evidence of a decline in royal power, as most
historians have tended to assume, but might, on the contrary, indicate
a more efficient use of resources in the late Old Kingdom and First
Intermediate Period.
The pace of change in such aspects of Egyptian culture as monu-
mental architecture, funerary beliefs, and ethnicity was not necessarily
tied to the rate of political change. Each of the authors of this history
has set out to elucidate the underlying patterns of social and political
change and to describe, with due regard to the dangers of archaeo-
logical and textual distortion and bias, the changing face of Egyptian
culture, from the biographical details of individuals to the social and
economic factors that shaped the lives of the population as a whole.
IAN SHAW
School of Archaeology, Classics and Oriental Studies,
The University of Liverpool
31 January 2000
ACKNOWLEDGEMENTS
I am most grateful to Hilary O'Shea, Senior Editor for Ancient History
at Oxford University Press, for her help in the early stages of this book
in particular.
Janine Bourriau would like to thank Manfred Bietak, Irmgard Hein,
and David Aston for generously allowing her to draw on unpublished
information on the current excavations at the site of Avaris (Tell el-
Dab e a).
This page intentionally left blank
CONTENTS
List of Plates xi
List of Maps and Plans xiii
List of Contributors xv
1. Introduction: Chronologies and Cultural Change in Egypt i
IAN SHAW
2. Prehistory: From the Palaeolithic to the Badarian Culture
(c.700,000-4000 BC) 16
STAN HENDRICKX and PIERRE VERMEERSCH
3. The Naqada Period (0.4000-3200 BC) 41
BEATRIX MIDANT-REYNES
4. The Emergence of the Egyptian State (0.3200-2686 BC) 57
KATHRYN A. BARD
5. The Old Kingdom (0.2686-2160 BC) 83
JAROMIR MALEK
6. The First Intermediate Period (c.2160-2055 BC) 108
STEPHAN SEIDLMAYER
7. The Middle Kingdom Renaissance (c.2055-1650 BC) 137
GAE CALLENDER
8. The Second Intermediate Period (c.1650-1550 BC) 172
JANINE BOURRIAU
9. The 18th Dynasty before the Amarna Period
(c.1550-1352 BC) 207
BETSY M. BRYAN
10. The Amarna Period and the Later New Kingdom
(c.1352-1069 BC) 265
JACOBUS VAN DIJK
X CONTENTS
11. Egypt and the Outside World 308
IAN SHAW
12. The Third Intermediate Period (1069-664 BC) 324
JOHN TAYLOR
13. The Late Period (664-332 BC) 364
ALAN B. LLOYD
14. The Ptolemaic Period (332-30 BC) 388
ALAN B. LLOYD
15. The Roman Period (30 BC-AD 395) 414
DAVID PEACOCK
Epilogue 437
Further Reading 438
Glossary 472
Chronology 480
Index 491
LIST OF PLATES
i. King-list from temple of Sety I at Abydos
The Griffith Institute, Ashmolean Museum, Oxford
2. Body of a child from Taramsa
Belgian Middle Egypt Prehistoric Project of Leuven University/
P. M. Vermeersch, E. Paulissen, and P. Van Peer, Le Paléolithique
de la vallée du Nil égyptien: L'Egypte des millénaires obscurs
(Marseille, 1990)
3. Narmer Palette
© Jürgen Liepe Photo Archive
4. Tax inspection scenes in tomb of Ptahotep
Hirmer Fotoarchiv
5. (top) nth Dynasty stele of Wahankh Intef II
The Metropolitan Museum of Art, Rogers Fund, 1913 (13.182.3)
(bottom) stele of Djary
Copyright IRPA-KIK, Brussels
6. Tomb model from the tomb of Meketra at Thebes
Scala, Florence
7. Tell el-Yahudiya juglet
Ashmolean Museum, Oxford
8. Decorated tiles at the temple of Rameses III, Medinet Habu,
c.nSo BC
Jürgen Liepe Photo Archive
9. Detail of a Sherden soldier
Ian Shaw
10. Upper part of triumphal stele of Piy
Cairo Museum
11. Avenue of sphinxes of Nectanebo I, Karnak, 3oth Dynasty
Ian Shaw
12. Cartouche of Nectanebo I, temple of I sis at Philae, 3oth Dynasty
Ian Shaw
xii LIST OF PLATES
13. Inner coffin of Petosiris from his tomb at Tuna-el-Gebel
Jürgen Liepe Photo Archive
14. Two mummy portraits
David Peacock
LIST OF MAPS AND PLANS
MAPS
1. Egypt showing principal Palaeolithic, Neolithic, and
Badarian sites 21
2. Egypt showing principal sites of the Naqada I and II phases 42
3. The Nile Valley and Palestine, c.1650-1550 BC 189
4. Egypt and Nubia, c.1550-1352 BC 217
5. Egypt and the Levant, c.1550-1352 BC 224
6. Trade routes between Egypt and the Ancient Near East 312
7. North-east Africa and Punt 316
8. Principal sites and political divisions, c.1069 BC 335
9. Principal cities and dynastic centres, 0.730 BC 336
10. The Mediterranean region during the Ptolemaic Period,
332-30 BC 390
11. Forts in the Eastern Desert and routes from Red Sea ports
to the Nile, 30 BC-AD 395  427
PLANS
1. Tell el-Dab e a (site of Hyksos capital, Avaris) 174
After J. Dorner 1990
2. The stratigraphy and chronology of Tell el-Dab e a 181
Courtesy Manfred Bietak, from Avaris (British Museum Press)
3. El-Amarna 273
Barry J. Kemp, Ancient Egypt (Routledge)
4. A group of tombs in the New Kingdom necropolis at
Saqqara 280
Committee of the Egypt Exploration Society
5. Temples and tombs at the eastern Delta site of Tanis 326
After A. Lezine 1951
6. The tomb of Mentuemhat, showing underground
structures 385
xiv LIST OF MAPS AND PLANS
Asasif tomb no. 34; Leclant, Mentuemhat; Instirut Francais
d'Archaéologie Orientale, Cairo
7. Fleet dispositions at the Battle of Salamis 394
8. Alexandria and its three main harbours 399
9. Diagram showing the bureaucratic structure of
Roman Egypt 416
A. Bowman, Egypt After the Pharaohs (British Museum Press,
1986)
LIST OF CONTRIBUTORS
IAN SHAW University of Liverpool
STAN HENDRIKX Provinciale Hogeschool, Lirnburg (Hasselt)
PIERREVERMEERSCH Katholieke Universiteit, Leuven
BEATRIX MIDANT-REYNES Centre National de Recherches
Scientifiques, Paris
KATHRYN BARD Boston University, Massachusetts
JAROMIR MALEK Griffith Institute, Oxford
STEPHEN SEIDLMAYER Berlin-Brandenburgische Akademieder
Wissenschaften
GAECALLENDER Macquarie University, Australia
JANINE BOURRIAU McDonald Institute, Cambridge
BETSY BRYAN Johns Hopkins University, Baltimore
JACOBUS VAN DIJK Rijksuniversitat, Groningen
JOHN TAYLOR British Museum, London
ALAN LLOYD University of Wales, Swansea
DAVID PEACOCK University of Southampton
This page intentionally left blank
Introduction: Chronologies and
Cultural Change in Egypt
IAN SHAW
All history is clearly reliant on some form of chronological framework,
and a great deal of time has been spent on the construction of such
dating systems for ancient Egypt. Ever since the first Western-style
history of Egypt was written by an Egyptian priest called Manetho in
the third century BC, the 'pharaonic period', from c.^ioo to 332 BC, has
been divided into a number of periods known as 'dynasties', each con-
sisting of a sequence of rulers, usually united by such factors as kin-
ship or the location of their principal royal residence. This essentially
political approach has served very well over the years as a way of divid-
ing up Egyptian chronology into a series of convenient blocks, each
with its own distinctive characteristics. It is, however, becoming
increasingly difficult to reconcile this politically based chronology with
the social and cultural changes revealed by excavations since the
19605.
Chronology
As Egyptian historical and archaeological data have expanded and
diversified, it has become apparent that Manetho's system—simple,
durable, and convenient though it is—often strains to contain the many
new chronological trends and currents that can be perceived outside
the simple passing of the throne from one group of individuals to
another. Some of the new work shows that at many points in time Egypt
was far less culturally unified and centralized than was previously
I
2  IAN SHAW
assumed, with cultural and political changes taking place at different
speeds in the various regions. Other analyses show that short-term
political events, which have often tended to be regarded as the para-
mount factors in history, may often be less historically significant than
the gradual processes of socio-economic change that can transform
the cultural landscape more overwhelmingly in the long term. Just as
the long 'pre-Dynastic' periods of Egyptian prehistory are commonly
understood as sequences of cultural rather than political develop-
ments, so the Dynastic Period (as well as the Ptolemaic and Roman
periods) has begun to be understood not only in terms of the tradi-
tional sequence of individual kings and ruling families but also in
terms of such factors as the types of fabric being used for pottery, and
the painted decoration applied to wooden coffins.
Modern Egyptologists' chronologies of ancient Egypt combine three
basic approaches. First, there are 'relative' dating methods, such as
stratigraphic excavation, or the 'sequence dating' of artefacts, which was
invented by Flinders Petrie in 1899. In the late twentieth century, as
archaeologists have developed a more subtle understanding of the ways
in which the materials and design of different types of Egyptian artefacts
(particularly ceramics) changed over time, it has become possible to
apply forms of seriation to many different types of object. Thus Harco
Willems's seriation of Middle Kingdom coffins, for instance, has pro-
vided a better understanding of cultural changes in the various provin-
ces of iith-i3th-Dynasty Egypt, complementing the information already
available about national political change during the same period.
Secondly, there are so-called absolute chronologies, based on calen-
drical and astronomical records obtained from ancient texts. Thirdly,
there are 'radiometric' methods (the most commonly used examples
of which are radiocarbon dating and thermoluminescence), by means
of which particular types of artefacts or organic remains can be
assigned dates based on the measurement of radioactive decay or
accumulation.
Radiocarbon Dating and Egyptian Chronology
The relationship between the calendrical and radiometric chronologi-
cal systems has been relatively ambivalent over the years. Since the late
19405, when a series of Egyptian artefacts were used as a benchmark
in order to assess the reliability of the newly invented radiocarbon
dating technique, a consensus has emerged that the two systems are
broadly in line. The major problem, however, is that the traditional
INTRODUCTION 3
calendrical system of dating, whatever its failings, virtually always has
a smaller margin of error than radiocarbon dates, which are neces-
sarily quoted in terms of a broad band of dates (that is, one or two
standard deviations), never capable of pinpointing the construction of
a building or the making of an artefact to a specific year (or even a
specific decade). Certainly the advent of dendrochronological calibra-
tion curves—allowing the spans of radiocarbon years to be converted
into actual calendar years—represents a significant improvement in
terms of accuracy. However, the vagaries of the curve and the con-
tinued need to take into account associated error mean that dates must
still be quoted as a range of possibilities rather than one specific year.
The prehistory of Egypt, on the other hand, has benefited greatly
from the application of radiometric dating, since it was previously reli-
ant on relative dating methods (see Chapters 2 and 3). The radiometric
techniques have made it possible not only to place Petrie's 'sequence
dates' within a framework of absolute dates (however imprecise), but
also to push Egyptian chronology back into the earlier Neolithic and
Palaeolithic periods.
From Prehistory to History: Late Predynastic Artefacts and the
Palermo Stone
There are only a small number of artefacts from the late Predynastic
Period that can be used as historical sources, documenting the trans-
ition into full unified statehood. These are funerary stelae, votive
palettes, ceremonial maceheads, and small labels (of wood, ivory, or
bone) originally attached to items of elite funerary equipment. In the
case of the stelae, palettes, and maceheads, it was clearly the intention
that they should commemorate many different kinds of royal act,
whether the King's own death and burial or his act of devotion to one of
the gods or goddesses. Some of the smaller, earlier labels (particularly
those recently excavated from the late Predynastic 'royal tomb' U-j at
Abydos, see Chapter 4) are simply records of the nature or origins of
the grave goods to which they were attached, but some of the later
labels, from the Early Dynastic royal graves at Abydos, employ a simi-
lar repertoire of depictions of royal acts in order to assign the items in
question to a particular date in the reign of a specific king.
If the purpose of this mobiliary art of the late fourth and early third
millennia BC was to label, commemorate, and date, then their decora-
tion has to be seen as resulting from the desire to communicate the
'context' of the object in terms of event and ritual. Nick Millet has
4  IAN SHAW
particularly demonstrated this in his analysis of the Narrner mace-
head, which was part of a group of late Predynastic and early pharaonic
votive items (including the Narmer Palette and Scorpion Macehead)
excavated by Quibell and Green in the temple precinct at Hierakon-
polis. The analysis of the scenes and texts on these objects is com-
plicated by our modern need to be able to distinguish between event
and ritual. But the ancient Egyptians show little inclination to dis-
tinguish consistently between the two, and indeed it might be argued
that Egyptian ideology during the pharaonic period—particularly in so
far as it related to the kingship—was reliant on the maintenance of
some degree of confusion between real happenings and purely ritual
or magical acts.
With regard to the palettes and maceheads, the Canadian Egypt-
ologist Donald Redford suggests that there must have been a need to
commemorate the unique events of the unification at the end of the
third millennium BC, but that these events were 'commemorated'
rather than 'narrated'. This distinction is a crucial one: we cannot
expect to disentangle 'historical' events from scenes that are com-
memorative rather than descriptive—or, at least, if we do so we may
often be misled.
One of the most important historical sources for the Early Dynastic
Period (3000-2686 BC) and the Old Kingdom (2686-2160 BC) is the
Palermo Stone, part of a 5th-Dynasty basalt stele (£.2400 BC) inscribed
on both sides with royal annals stretching back to the mythical
prehistoric rulers. The main fragment has been known since 1866 and
is currently in the collection of the Palermo Archaeological Museum,
Sicily, although there are also further pieces in the Egyptian Museum,
Cairo, and the Petrie Museum, London. The slab must originally have
been about 2.1 m. long and 0.6 m. wide, but most of it is now missing,
and there is no surviving information about its provenance. This docu-
ment—along with the 'day-books', the annals and 'king-lists' inscribed
on temple walls, and the papyri held in temple and palace archives—
was doubtless the kind of document that Manetho consulted when he
was compiling his history or Aegyptiaca.
The text of the Palermo Stone enumerates the annals of the kings of
Lower Egypt, beginning with the many thousands of years that were
assumed to have been taken up by mythological rulers, until the time
of the god Horus, who is said to have given the throne to the human
king Menes. Human rulers are then listed up to the 5th Dynasty. The
text is divided into a series of horizontal registers divided by vertical
lines that curve in at the top, apparently in imitation of the hieroglyph
INTRODUCTION 5
for regnal year (renpet), thus indicating the memorable events of indi-
vidual years in each king's reign. The situation is slightly confused by
the fact that many Old Kingdom dates appear to refer to the number of
biennial cattle censuses (hesbet) rather than to the number of years that
the king had reigned; therefore the number of'years' in the Old King-
dom dates may well have to be doubled to find out the actual number
of regnal years.
The types of event that are recorded on the Palermo Stone are cult
ceremonies, taxation, sculpture, building, and warfare—that is, pre-
cisely the type of phenomena that are recorded on the protodynastic
ivory and ebony labels from Abydos, Saqqara, and various other early
historical sites. The introduction of the renpet sign on the labels, in the
reign of Djet, makes this comparison even closer. There are two differ-
ences, however: first, the labels include clerical information, while the
Palermo Stone does not, and, secondly, the Palermo Stone includes
records of the Nile inundation, whereas the labels do not. Both of these
types of information seem to have occupied the same physical part of
the document's format—that is, the bottom of the record. Redford
suggests that this shows that the Old Kingdom genut (the royal annals
that are assumed to have existed at this date, but have not survived
except in the form of the Palermo Stone) were concerned with
hydraulic/climatic change, which, with its crucial agricultural and eco-
nomic consequences, was potentially the most important aspect of
change as far as the reputation of each individual king was concerned.
This kind of hydraulic information would, however, have perhaps
been regarded as irrelevant to the function of the labels attached to
funerary equipment.
King-Lists, Royal Titles, and the Divine Kingship
Apart from the Palermo Stone, the basic sources used by Egyptologists
to construct the traditional chronology of political change in Egypt
are Manetho's history (which, unfortunately, has survived only in the
form of excerpts compiled by the later authors Josephus, Africanus,
Eusebius, and George Syncellus), the so-called king-lists, dated records
of astronomical observations, textual and artistic documents (such as
reliefs and stelae) bearing descriptions apparently referring to histori-
cal events, genealogical information, and synchronisms with non-
Egyptian sources, such as the Assyrian king-lists. For the 28th~3oth
Dynasties, the Demotic Chronicle (Papyrus 215 in the Bibliotheque
Nationale, Paris) serves as a unique early Ptolemaic source concerning
6  IAN SHAW
political events in this last phase of the Late Period, compensating to
some extent for the dearth of historical information provided by the
papyri and monuments of this date (as well as the fact that Manetho
gives only the names and reign lengths of the kings). Wilhelm Spiegel-
berg and Janet Johnson have shown that careful translation and
interpretion of the 'oracular statements' in this pseudo-prophetic
document can shed new light not only on the events of the period
(such as the suspected co-regency between Nectanebo I and his son
Tachos) but also on the ideological and political context of the fourth
century BC.
Like most other ancient peoples, the ancient Egyptians dated impor-
tant political and religious events not according to the number of years
that had elapsed since a single fixed point in history (such as the birth
of Christ in the modern Western calendar), but in terms of the years
since the accession of each current king (regnal years). Dates were,
therefore, recorded in the following typical format: 'day 2 of the first
month of the season per&t in the fifth year of Nebmaatra (Amenhotep
III)'. It is important to be aware of the fact that, for the Egyptians, the
reign of each new king represented a new beginning, not merely philo-
sophically but practically, given the fact that dates were expressed in
such terms. This means that there would probably have been a psycho-
logical tendency to regard each new reign as a fresh point of origin:
every king was, therefore, essentially reworking the same universal
myths of kingship within the events of his own time.
One important aspect of the Egyptian kingship throughout the
pharaonic period was the existence of a number of different names for
each individual ruler. By the Middle Kingdom, each king held five
names (the so-called fivefold titulary), each of which encapsulated a
particular aspect of the kingship: three of them stressed his role as a
god, while the other two emphasized the supposed division of Egypt
into two unified lands. The birth name (or nomen), such as Rameses or
Mentuhotep, introduced by the title 'son of Ra', was the only one to be
given to the pharaoh as soon as he was born. It was also usually the last
name given in inscriptions identifying the king by his whole sequence
of names and titles. The other four names—Horus, nebty ('he of the two
ladies'), (Horus of) Gold, and nesu-bit ('he of the sedge and the bee')—
were given to him at the time of his installation on the throne, and
their components may sometimes convey something of the ideology
or intentions of the king in question. As far as the rulers of Dynasty o
and the beginning of the Early Dynastic Period were concerned, we
know only their 'Horus names', typically written inside a serekh frame
INTRODUCTION 7
(a kind of diagram of the palace gateway), upon which a Horus-falcon
was perched. It was the late ist-Dynasty ruler Den (c.29oo BC) who was
the first to hold a nesu-bit name (Khasty), but it was not until the reign
of Sneferu, 2613-2589 BC, in the 4th Dynasty, that this name was first
framed by the familiar cartouche shape (an encircling loop that perhaps
signified the infinite extent of the royal domain).
The title nesu-bit has often been translated as 'King of Upper and
Lower Egypt', but it actually has a much more complex and significant
meaning. Nesu seems to be intended to refer to the unchanging divine
king (almost the kingship itself), while the word bit describes the
current ephemeral holder of the kingship: the one individual king in
power at a specific point in time. Each king was, therefore, a combina-
tion of the divine and the mortal, the nesu and the bit, in the same way
that the living king was linked with Horus, and the dead kings, the
royal ancestors, were associated with Horus' father Osiris. It was pri-
marily because of the Egyptians' sense of each of their kings as incar-
nations of Horus and Osiris that the tradition of the worship of divine
royal ancestors developed. This convention, whereby the current ruler
made obeisance to his predecessors, is the reason for the creation
of the so-called king-lists, which were lists of royal names mainly
recorded on the walls of tombs and temples (most notably the iQth-
Dynasty temples of Sety I and Rameses II at Abydos), but also in the
form of papyri, only one example of which survives (the so-called Turin
Canon), or remote desert rock carvings, as with the list at the Wadi
Hammamat siltstone quarries in the Eastern Desert. The continuity
and stability of the kingship were preserved by making offerings to all
those kings of the past who were regarded as legitimate rulers, just as
we see Sety I doing in his cult temple at Abydos. It is usually presumed
that king-lists were among the sources used by Manetho in compiling
his history.
The Turin Canon, a Ramessid papyrus dating to the thirteenth
century BC, is the most informative of the Egyptian king-lists. From the
Second Intermediate Period (1650-1550 BC), it stretched back with
reasonable accuracy to the reign of the ist-Dynasty ruler Menes
(^.3000 BC), and even beyond that into a mythical prehistoric time
when the gods ruled over Egypt. Each king's reign was recorded in
terms of years, months, and days. It also provides some support for
Manetho's system of dynasties by incorporating a break at the end of
the 5th Dynasty (see Chapter 5).
The king-lists were not concerned so much with history as with
ancestor worship: the past is presented as a combination of the general
8  IAN SHAW
and the individual, and the constancy and universality of the kingship
are celebrated through the listing of specific individual holders of the
royal titulary. In his commentary on Herodotus Book II, Alan Lloyd
writes, 'Since all historical study involves general and particular,
attempting to place particular phenomena against a background of
general principle or law, there is always a tension between the two, and
this tension is resolved in Egypt overwhelmingly in favour of the
latter.' The conflict between the general and the particular is undoubt-
edly an important factor in ancient Egyptian chronology and history.
The texts and artefacts that form the basis of Egyptian history usually
convey information that is either general (mythological or ritualistic)
or particular (historical), and the trick in constructing a historical
narrative is to distinguish as clearly as possible between these types of
information, taking into account the Egyptians' tendency to blur the
boundaries between the two.
The Swiss Egyptologist Erik Hornung describes Egyptian history as
a kind of'celebration' of both continuity and change. Just as the living
king could be regarded as synonymous with the falcon-god Horus, so
his individual subjects (from at least the First Intermediate Period
onwards) eventually came to identify themselves with the god Osiris
after their deaths. In other words, the Egyptians were used to the idea
of portraying human individuals as combinations of the general and
the particular. Their own sense of history therefore comprised both the
specific and the universal in equal measure.
The Role of Astronomy in Traditional Egyptian Chronology
The task of the modern historian of ancient Egypt is usually to attempt
to tie together all the strands of evidence in the form of individuals'
biographies on the walls of tombs, lists of kings on temple walls, strati-
graphic evidence of archaeological excavations, and a whole range of
other pieces of information. In the pharaonic, Ptolemaic, and Roman
periods, the 'traditional' absolute chronologies tend to rely on complex
webs of textual references, combining such elements as names, dates,
and genealogical information into an overall historical framework that
is more reliable in some periods than in others. The so-called inter-
mediate periods have proved to be particularly awkward phases, partly
because there was often more than one ruler or dynasty reigning
simultaneously in different parts of the country. The surviving records
of observations of the heliacal rising of the dog-star Sirius serve both
INTRODUCTION 9
as the linchpin of the reconstruction of the Egyptian calendar and its
essential link with the chronology as a whole.
The goddess Sopdet, known as Sothis in the Graeco-Roman period
(332 BC-AD 395), was the personification of the 'dog-star', which the
Greeks called Seirios (Sirius). She was usually represented as a woman
with a star poised on her head, although the earliest depiction, on an
ivory tablet of the ist-Dynasty king Djer (0.3000 BC) from Abydos,
appears to show her as a seated cow with a plant between her horns.
Since a depiction of a plant is used as the ideogram meaning 'year' in
the pharaonic writing system, the Egyptians may have already been
correlating the rising of the dog-star with the beginning of the solar
year, even in the early third millennium BC. Along with her husband
Sah (Orion) and her son Soped, Sopdet was part of a triad that paral-
leled the family of Osiris, Isis, and Horus. She was therefore described
in the Pyramid Texts as having united with Osiris to give birth to the
morning star.
In the Egyptian calendrical system, Sopdet was the most important
of the stars or constellations known as decans, and the 'Sothic rising'
coincided with the beginning of the solar year only once every 1,460
years (or, more accurately, 1,456 years). We know that this rare syn-
chronization of the heliacal rising of Sopdet with the beginning of
the Egyptian civil year (or 'wandering year', as it is sometimes
described, given that it gradually falls behind the solar year at a rate of
about a day every four years) took place in AD 139, during the reign of
the Roman emperor Antoninus Pius, because the event was com-
memorated by the issue of a special coin at Alexandria. There would
have been earlier heliacal risings in 1321-1317 BC and 2781-2777 BC,
and the period that elapsed between each such rising is known as a
Sothic cycle.
Two Egyptian textual records of Sothic risings (dating to the reigns
of Senusret III and Amenhotep I) form the basis of the conventional
chronology of Egypt, which, in turn, influences that of the whole
Mediterranean region. These two documents are a 12th-Dynasty letter
from the site of Lahun, written on day 16, month 4, of the second
season in year 7 of the reign of Senusret III, and an 18th-Dynasty
Theban medical papyrus (Papyrus Ebers), written on day 9, month 3,
of the third season of year 9 in the reign of Amenhotep I. By assigning
absolute dates to each of these documents (1872 BC for the Lahun
rising in year 7 of Senusret III, and 1541 BC for the Ebers rising in regnal
year 9 of Amenhotep I), Egyptologists have been able to extrapolate a
set of absolute dates for the whole of the pharaonic period, on the basis
10 IAN SHAW
of records of the lengths of reign of the other kings of the Middle and
New kingdoms.
It is not possible, however, to be totally confident of the absolute
dates cited above, since the precise dating is dependent on our knowl-
edge of the location (or locations) where the astronomical observations
were made. It used to be assumed—without any real evidence—that
such observations were made at Memphis or perhaps Thebes, but
Detlef Franke and Rolf Krauss have argued that they were all made at
Elephantine. William Ward, on the other hand, suggested that they are
all more likely to have been separate local observations, which would
have resulted in a time lag in terms of the various 'national' religious
festivals (that is, both the observations and the corresponding festivals
may actually have taken place at different times and in different parts
of the country). This continuing uncertainty means that our astro-
nomical linchpins are in reality somewhat floating, although it should
be noted that the differences between the 'high' and 'low' chronologies
(based largely on assumptions concerning different observation points)
are usually only a few decades at most.
Co-Regencies
One of the peculiarities of Egyptian chronology, provoking both con-
fusion and debate, is the concept of the 'co-regency', a modern term
applied to the periods during which two kings were simultaneously
ruling, usually consisting of an overlap of several years between the
end of one sole reign and the beginning of the next. This system seems
to have been used, from at least as early as the Middle Kingdom, in
order to ensure that the transfer of power took place with the mini-
mum of disruption and instability. It would also have enabled the
chosen successor to gain experience in the administration before his
predecessor died.
It seems, however, that the dating systems during co-regencies may
have differed from one period to another. Thus i2th-Dynasty co-
regents may have each used separate regnal dates, so that overlaps
occurred between the kings' reigns, producing examples of so-called
double dates, when both dating systems were used to date a single
monument (see Chapter 7). In the New Kingdom, there are no certain
instances of double dates, therefore a different system seems to have
been used. In the reigns of Thutmose III 1479-1425 BC and
Hatshepsut 1473-1458 BC, for instance, year dates appear to have been
counted with reference to Hatshepsut's accession, as if Hatshepsut
INTRODUCTION II
had become ruler at the same time as Thutmose III. It is a moot point
as to whether separate dates were used by two kings during the pos-
sible co-regencies of Thutmose III-Amenhotep II and Amenhotep
III—Amenhotep IV. The arguments for and against a co-regency
between the two latter kings have been carefully reviewed by Donald
Redford and later by William Murnane. However, there is still con-
siderable controversy over the question of which co-regencies actually
took place and how long they lasted. There are also some scholars
(including Gae Callender in Chapter 7 of this volume) who argue that
co-regencies may never have occurred at all.
'Dark Ages' and Other Chronological Problems
Some of the problems encountered in Egyptian chronology have
already been mentioned, such as the potential confusion of links
between astronomical observations and specific dates, the uncertainty
as to which co-regencies (if any) actually occurred, and the assumption
that the Egyptians of the pharaonic period and later continually dated
events according to an artificial 'wandering' civil year of 365 days,
which was rarely synchronized with the real solar year.
There are also, of course, a number of other Egyptian historical
problems, ranging from unreliability of sources (for example, Man-
etho's history, given that we neither know his sources nor have his
original text) and frequent uncertainty regarding lengths of kings'
reigns (for example, the Turin Canon says that Senusret II and III have
reigns of nineteen and thirty-nine years respectively, whereas their
highest recorded regnal years on documents that are actually con-
temporary with their reigns are only six and nineteen).
Egypt, like other cultures, has periods in history that are more or
less documented than others, and it is primarily this patchiness in the
survival of archaeological and textual records from different dates that
has led to the assumption that there were 'intermediate periods', when
the political and social stability of the pharaonic period appeared to
have been temporarily damaged. Thus, those periods of political and
cultural continuity described as the Old, Middle, and New kingdoms
were each thought to be followed by 'dark ages', when the country
became disunited and weakened by conflict (either civil war between
provinces or invasion by foreigners). This scenario was both denied
and bolstered by Manetho's history. First, Manetho created a mislead-
ing air of continuity in the succession of kings and dynasties through
his assumption that only one king could occupy the throne of Egypt at
12 IAN SHAW
any one time. Secondly, his descriptions of some of the dynasties
corresponding to the times of the intermediate periods suggest that
the kingship was changing hands with alarming rapidity.
The study of the Third Intermediate Period has become one of the
most controversial areas of Egyptian history, particularly during the
19905, when it has been subjected to intensive study by a number of
different scholars. Three areas of investigation have blossomed. First,
several aspects of the culture of the period (for example, ceramics and
funerary equipment) have been analysed in terms of changes in such
factors as style and materials. Secondly, anthropological, icono-
graphic, and linguistic studies have been undertaken with regard to
the 'Libyan' ethnic identity of many of the 2ist-24th-Dynasty rulers.
Thirdly, and most crucially from the point of view of the history of the
pharaonic period as a whole, it was argued by a small number of schol-
ars that the period of 400 years occupied by the Third Intermediate
Period (and numerous other, roughly contemporaneous, 'dark ages'
elsewhere in the Near East and the Mediterranean) may have been arti-
ficially inflated by historians. They suggested that the New Kingdom
might have ended not in the eleventh century but in the eighth century
BC, leaving a much smaller gap of about 150 years between the end of
the 2oth Dynasty and the beginning of the Late Period. Such a view,
however, has been widely dismissed, not only because Egyptologists,
Assyriologists, and Aegeanists have been able to refute many of the
individual textual and archaeological arguments for chronological
change, but also, more significantly, because the scientific dating
systems (that is, radiocarbon and dendrochronology) almost always
provide solid independent support for the conventional chronology.
Indeed, the irrelevance of such tinkering with the conventional chron-
ological framework, given the overwhelming and increasing signifi-
cance of scientific dates, has been memorably described by the classical
archaeologist Anthony Snodgrass as 'a bit like a detailed scheme for
re-organizing the East German economy, produced in 1989 or early
1990'.
On a more cultural, rather than chronological level, the significance
of the most basic historical divisions (that is, the distinctions between
the Predynastic, pharaonic, Ptolemaic, and Roman periods) have begun
to be questioned. On the one hand, the results of excavations during
the 19805 and 19905 in the cemeteries of Umm el-Qa e ab (at Abydos)
suggest that before the ist Dynasty there was also a Dynasty o stretch-
ing back for some unknown period into the fourth millennium BC.
This means that, at the very least, the last one or two centuries of the
INTRODUCTION 13
Tredynastic' were probably in many respects politically and socially
'Dynastic'. Conversely, the increasing realization that Naqada III pot-
tery types were still widely used in the Early Dynastic Period shows that
certain cultural aspects of the Predynastic Period continued on into the
pharaonic period (see Chapter 4).
Whereas there are definite political breaks between the pharaonic
and Ptolemaic periods, and between the Ptolemaic and Roman periods,
the gradually increasing archaeological data from the two latter periods
have begun to create a situation where the process of cultural change
may be seen to be less sudden than the purely political records suggest.
Thus it is apparent that there are aspects of the ideology and material
culture of the Ptolemaic Period that remain virtually unaltered by
political upheavals. Instead of the arrival of Alexander the Great and
his general Ptolemy representing a great watershed in Egyptian history,
it might well be argued that, although there were certainly a number of
significant political changes between the mid-first millennium BC and
the mid-first millennium AD, these took place amid comparatively
leisurely processes of social and economic change. Significant ele-
ments of the pharaonic civilization may have survived relatively intact
for several millennia, only undergoing a full combination of cultural
and political transformation at the beginning of the Islamic Period in
AD 641.
Historical Change and Material Culture
There has been an enormous growth in the study of Egyptian pottery
in the late twentieth century, both in terms of the quantity of sherds
being analysed (from a wide variety of types of site) and in terms of the
range of scientific techniques now being used to extract more infor-
mation from ceramics. Inevitably the improvement in our under-
standing of this prolific aspect of Egyptian material culture has had an
impact on the chronological framework. The excavation of part of the
city of Memphis (the site of Kom Rabi e a) in the 19805 provides a good
instance of the ways in which more sophisticated approaches to
pottery have enabled the overall process of cultural change to be better
understood.
Pottery vessels can be arranged in terms of relative date by such
traditional techniques as seriation of cemetery material and the analy-
sis of large quantities of stratified material at domestic or religious
sites, but they can also be given fairly precise absolute dates either by
the conventional method of association with inscribed or artistic
14 IAN SHAW
material (particularly in tombs) or by the use of such scientific tech-
niques as thermoluminescence dating. Some scholars have begun to
study the ways in which vessel and fabric types change over the course
of time. Thus, the form of pottery bread moulds, for instance, under-
went a dramatic change at the end of the Old Kingdom, but it is not yet
clear whether the source of this change lies in the social, economic, or
technological spheres of life, or whether it is merely the result of a
change in 'fashion'. Such analyses show that processes of change in
material culture took place for a whole variety of reasons, only some of
which were linked to the political changes that tend to dominate
conventional views of Egyptian history. This is not to deny the many
connections between political and cultural change, such as the correla-
tion between centralized production of pottery in the Old Kingdom
and resurgence of local pottery types during the more politically frag-
mented First Intermediate Period (and then the renewed homogeniza-
tion of pottery during the more unified i2th Dynasty).
In the study of certain phases of Egyptian history, such as the emer-
gence of the unified state at the beginning of the pharaonic period or
the decline and demise of the Old Kingdom, scholars have sometimes
examined numerous environmental and cultural factors in order to
explain sudden important political changes. One of the problems with
this selective attention to non-political historical trends, however, is the
fact tha t we still know so little about environmental and cultural
change during periods of stability and prosperity, such as the Old and
Middle kingdoms, that it is much more difficult to interpret these
factors at times of political crisis. The increased study of pottery vessels
and other common artefacts (as well as environmental factors such as
climate and agriculture) are beginning to create the basis for more
holistic versions of Egyptian history, in which political narratives are
viewed within the context of long-term processes of cultural change.
Egyptian 'History'
Art and texts throughout the pharaonic period continued to maintain
the Predynastic and Early Dynastic tension between recording and
commemorating, which might be characterized as the distinction
between, on the one hand, the utilitarian labels attached to grave
goods, and, on the other hand, such ceremonial votive items as palettes
and maceheads, described above. We know that the purpose of the
early funerary labels was to use history as a means of dating particular
things, and that the purpose of such mobiliary art as the palettes and
INTRODUCTION 15
maceheads—as well as of stelae and temple reliefs in the pharaonic
period—was not to record historical events but primarily to use them
as a means of commemorating universal acts undertaken by specific
rulers or by royal officials.
In the mortuary temple of Rameses III at Medinet Habu there is a
scene in which the Libyan chieftain Meshesher is brought into the
presence of the king. This is obviously intended to be a record of the
surrender of a particularly important foreign individual, whose per-
sonal humiliation encapsulates the defeat of his people, but to the left-
hand side we can also see the careful assembling and counting of a pile
of Libyans' hands—this alerts us to one of the ways in which the scene
differs from a more modern Western historical tableau. It is part of a
relief in a mortuary temple and as such it is fulfilling the king's piety to
the gods. Just as private individuals in the New Kingdom inscribed
'autobiographical' texts on the walls of their tomb chapels to remind
the gods of their piety and beneficence, so the reliefs in royal mortuary
temples were intended to symbolize a kind of accounting procedure, a
visual quantification of the success achieved by the king both for and
through the gods.
The Egyptian sense of history is one in which rituals and real events
are inseparable—the vocabulary of Egyptian art and text very often
makes no real distinction between the real and the ideal. Thus the
events of history and myth were all regarded as part of a process of
assessment, whereby the king demonstrated that he was preserving
Maat, or harmony, on behalf of the deities. Even when an Egyptian
monument appears to be simply commemorating a specific event in
history, it is often interpreting that event as an act that is simul-
taneously mythological, ritualistic, and economic.
Prehistory: From the Palaeolithic to
the Badarian Culture
(c.yoo,000-4000 BC)
STAN H E N D R I C K X and
PIERRE V E R M E E R S C H
It has become a truism that ancient Egypt was a gift of the Nile because
the river's flooding brought new life into the valley in the late summer
of every year. Egypt was, therefore, essentially a rich oasis amid the
very extensive expanse of the Sahara. This, however, has not always
been the case: the very earliest inhabitants of Egypt lived in a different
kind of environment. First, the climate was not always as arid as it is
now (modern Upper Egypt being one of the most arid regions in the
world), oscillating instead between the present hyperaridity and a dry
sahelian condition. Secondly, the river itself was not always a meander-
ing river in a wide floodplain, with its late summer high floods. During
some periods, the Nile was either reduced to a series of independent
ephemeral wadi basins or had a generally low discharge, choked by its
own huge floodplain deposits. It brought its rich alluvia into Egypt only
when its headwaters reached back to Ethiopia. Finally, although the
river clearly brought life to Egypt, it has also brought about the erosion
of older archaeological deposits—we should, therefore, not be sur-
prised to find that only very scarce remains from the earliest occupa-
tion have been preserved.
Because of its geographical position, Egypt certainly served as
an important conduit for early humans migrating from East Africa
towards the rest of the Old World. We know that early Homo erectus left
2
PREHISTORY 17
Africa and arrived in Israel as early as 1.8 million years ago. There is,
therefore, no reason to doubt that small bands of Homo erectus visited
and probably stayed in the Nile Valley. Unfortunately, only very sparse
evidence of this event is available and, worse still, it cannot be dated,
because circumstantial evidence is also very poor. In some Early and
Middle Pleistocene deposits, isolated choppers, chopping tools, and
flakes, similar to those associated with early hominids in East Africa,
have been recovered in gravel quarries at Abbassiya, as well as in
Theban gravel deposits. However, most of these published 'artefacts'
are probably not of human origin and all of them are from secondary
deposits.
The Lower Palaeolithic
Many Lower Palaeolithic artefacts, including numerous handaxes of
Acheulean type, have been found in and on local gravel deposits. No
human bones have been found in Egypt in association with this
Acheulean phase, but Homo erectus can probably be assumed to have
been the maker of these artefacts. Misunderstanding of the desert geo-
morphology has led many researchers to believe that the Acheulean
can be correlated with a Nile terrace chronology, but this is unfortu-
nately not the case. We can presume, however, that Homo erectus at
least passed by regularly and left his handaxes at numerous sites.
Pedimentation and fluviatile erosion led to the dispersal of most of the
handaxes and their related artefacts. It is, therefore, not exceptional to
find Acheulean handaxes on the present surface of the desert areas in
the Nile Valley. In the early twentieth century, the hills over which a
path leads from Deir el-Medina to the Valley of the Kings, overlooking
the western side of Luxor, were particularly popular for 'collecting'
handaxes; although these stray finds cannot be dated, they are probably
all that remain, after intensive erosion, of large Acheulean sites. At
some locations, such as Nag Ahmed el-Khalifa, near Abydos, it has
proved possible to observe that artefacts remained grouped together,
even when they were no longer in their original context. There, and in
other parts of the Qena region, such handaxe concentrations occur on
top of the first clay deposits that attest the connection of the river Nile
with its headwaters in Ethiopia. We presume that the age of those con-
centrations should be set at about 400,000-300,000 BP, but this is
only a guess. In order to document the Acheulean occupation prop-
erly, we would need more information about such factors as the
original spatial distribution and the associated faunal remains.
l8 STAN H E N D R I C K X and P I E R R E V E R M E E R S C H
Our knowledge of prehistoric Nubia is comparatively well docu-
mented as a result of the rescue excavations carried out in the 19605,
before most of the area was flooded by Lake Nasser. Acheulean handaxe
concentrations occurred mainly on 'inselbergs' (eroded hilltops), where
it was possible to extract a good raw material: ferruginous sandstone.
Since such sites remained exposed on the surface for many hundred
thousands of years, we should not expect any remains to have survived
apart from lithics. Even in the case of the lithics, we have only limited
information and no secure means of dating except by typological
approaches. According to these typologies, the sites can be assigned to
Early, Middle, and Late Acheulean respectively. It is remarkable that
cleavers, so characteristic for the rest of Africa, are lacking in the assem-
blages, suggesting that, in Acheulean times, Nubia probably constituted
a particular province, an original enclave, in the African interior.
In the Western Desert, several Final Acheulean sites are known,
especially at the oases of Kharga and Dakhla and at Bir Sahara and Bir
Tarfawi. These sites are located on the scarps surrounding the oases,
but the most important finds are associated with fossil springs in the
floor of the oasis depressions or in the playa deposits. All of these sites
are clearly related to wetter conditions, when life as hunter-gatherers
was possible. Most of the known sites are in a bad state of preservation,
but it has been suggested that ancient channels in the Western Desert,
discovered by radar from the space shuttle, are rich in well-preserved
Acheulean sites, none of which has yet been excavated.
The Middle Palaeolithic
The picture that emerges for the Egyptian Middle Palaeolithic is rather
complex. It originated in the Late Acheulean, when handaxes became
associated with bifacial foliates and a typical Nubian knapping method.
Such assemblages may date from before 250,000 BP. The fate of sites
with such assemblages is similar to that of the Acheulean: all over the
desert one can collect scattered artefacts which once belonged together
in a site that is now destroyed. Judging from the high number of such
artefacts, it is tempting to assume that the population density was high.
As in many areas of the Old World, the Egyptian Middle Palaeolithic
is characterized by the introduction of the Levallois method, a special
technique designed to produce flakes and blades of fixed dimensions
from a flint nodule. In addition to the classical Levallois approach, the
Nubian Levallois knapping method was introduced for the production
of pointed flakes. In the Egyptian Middle Palaeolithic, several artefactual
PREHISTORY 19
'entities' can be distinguished. The chronology is still unclear, but
research, especially in the Western Desert and in the Qena area, pro-
vides some clues.
The Nubian Middle Palaeolithic is characterized by the Nubian
Levallois technique and by bifacial foliates and pedunculates. It is
mainly known from Nubia, where several sites have been discovered.
Although it is certainly also present in Egypt, no well-preserved sites
have yet been found there. Lastly, important information has been dis-
closed in relation to the mid-Middle Palaeolithic. At Bir Tarfawi and
Bir Sahara in the Western Desert, numerous well-preserved sites from
the Saharan Mousterian were excavated. It is clear that sites in this area
were accessible only during wet phases, which should probably be
regarded as short spells punctuating a mainly dry climate.
During most periods of occupation, there were permanent lakes in
the Western Desert, or, in some intervals, seasonal playas, fed by local
rainfall of up to 500 mm. per annum. In some phases the lakes could
be more than 7 m. deep. The area was abandoned during the periods of
hyperaridity that separated the lacustrine events. Side-scrapers, points,
and denticulates are the best-represented tools. The lake and playa
environments were probably rich in floral resources that could easily
be exploited, but unfortunately there is no archaeological evidence
available. The fauna apparently exploited by people at this date consist
of hare, porcupine, and wild cat, at one end of the size spectrum, and
buffalo, rhinoceros, and giraffe, at the other end. Small gazelles, mainly
the dorcas species, dominate the assemblage. The presence of such
animals suggests that selective—perhaps seasonal—hunting of small
gazelles was combined with more opportunistic meat procurement
from bigger game.
The apparent differences in content among sites in different set-
tings may reflect variations in activities carried out at the sites. Sites
embedded in fossil hydromorphic soils, characterized by low artefact
densities, indicate limited use, probably comprising several brief
phases and these only during very dry years. Sites embedded in beach
sands were accessible for a greater part of the year, but probably not
during the season of highest water, presumably in summer. Sites asso-
ciated with dry lake bottoms reflect unusually arid episodes when the
lakes dried up and their beds were exposed.
Excavations in the Sodmein cave near Quseir in the Red Sea
mountains disclosed similar wet conditions during part of the mid-
Middle Palaeolithic, with the presence of crocodile, elephant, buffalo,
kudu, and other large mammals. The cave was apparently visited over
20 STAN H E N D R I C K X and PIERRE V E R M E E R S C H
a long period but always for a short time. Sometimes, large hearths
were utilized.
A comparable way of life may have existed in the Nile Valley, but no
sites from the floodplain have yet been disclosed. On the other hand,
the Nile Valley has furnished us with many sites that document the
extraction of raw material. Sites that are contemporaneous with the
Western Desert occupation occur at Nazlet Khater and Taramsa,
where mid-Middle Palaeolithic groups were in search of raw material,
mainly comprising chert cobbles from terrace deposits. These groups
differ in terms of the knapping methods they used: Egyptian group K
utilized the classical Levallois method, in addition to flake production
from single and double platform cores, while Egyptian group N fre-
quently used the Nubian Levallois method. Tools are always rare at
such quarrying sites, because the artefacts produced at such sites were
meant to be exported to the living sites, which were probably situated
on the Nile floodplain. Unfortunately, such floodplain sites have prob-
ably been covered by recent alluvia and remain unknown.
Late Middle Palaeolithic material, along with Halfan and Safahan
(Levallois Idfuan) artefacts, has been recovered from extraction sites,
such as Nazlet Safaha, near Qena, as well as from living sites near
Edfu. The Halfan industry, however, was mainly restricted to Nubia.
In comparison with the earlier mid-Middle Palaeolithic, the Nubian
Levallois technique was disappearing, and, in addition to flake and
blade production from single and double platform cores, only an
evolved classical Levallois was utilized for production of thin Levallois
flakes. At living sites, burins, notches, and denticulates were being
used. Meanwhile, the climate had again become arid to hyperarid and
continued to be such. The evolution of the climate changed the living
conditions completely, in that food resources were now almost entirely
restricted to the floodplain. This climatic development must have
obliged people living in the Sahara to leave the area, resulting in a con-
centration of human population in the Nile Valley.
During the last period of the Middle Palaeolithic (the Taramsan)
there was a clear tendency towards blade production from large cores,
where, instead of obtaining a few Levallois flakes from each individual
core, a virtually continuous process of blade production made it pos-
sible to create a large number of blades from each core. At Taramsa-i,
an impressive extraction and production site of this date near Qena, it
can be observed that there was increasing interest in blade production,
a system that was later to be generalized during the Upper Palaeolithic.
Similar assemblages have been identified in the Negev, where the
PREHISTORY 21
transition from Levallois flaking to blade production has been docu-
mented at the site of Boker Tachtit, around 45,000 BP. A burial of an
'anatomically modern' child at Taramsa-i is associated with the late
Middle Palaeolithic. This burial is probably the oldest grave that has so
far been identified in Africa.
Map of Egypt showing the principal Paleolithic, Neolithic, and Badarian sites
22 STAN HENDRICKX and PIERRE VERMEERSCH
The techniques employed at the extraction sites were simple but
well adapted to the natural chert occurrences. The chert cobbles were
removed from the terrace deposits by means of open-trench and pit
systems with a maximum depth of about 1.7 m. Only the uppermost
part of the cobble terrace was mined, and the pits and trenches are
characterized by a very irregular planimetry, with many tentacles and
bulges. They have vertical walls with only minor undercutting, and
their widths vary from about i m. to nearly 2 m. Since the chert cobble
deposit was not consolidated, only simple extraction tools were
required. Depressions in the trenches were often used as workshops
for the fabrication of Levallois products. Extraction was very extensive
and, in the region of Qena, affected areas covered many square kilo-
metres. The search for good-quality chert and the use of specialized
tool production demonstrate the complex organization of the inhabit-
ants of the Nile Valley at that time. It also indicates that Middle Palaeo-
lithic humans were not only capable of tridimensional reasoning but
also had developed a knowledge of geology and geomorphology.
If the 'out-of-Africa' theory of human origins is true (and it is
still contested by some good anthropologists), anatomically modern
Homo sapiens should have passed through the Nile valley on its way out
of East Africa to Asia. However, it remains unclear as to whether
archaeological data can confirm that there were similarities between
the Middle Palaeolithic in Egypt and in south-west Asia. Finally, it is
to be noted that the Aterian industry, which is so important for the
rest of North Africa, is present only in some oases in the Western
Desert.
The Upper Palaeolithic
Upper Palaeolithic sites are rare in Egypt. The oldest site of this date is
Nazlet Khater-4 in Middle Egypt, where chert was extracted not only by
trenches and mining pits (with a maximum depth of 2 m.) but also by
underground galleries starting from the trench walls or from the
bottom of a pit. In this manner, underground galleries covering an
area of more than 10 sq.m. were obtained. Hearths found in the fill of
the trenches where flaking activities took place suggest that mining
activities were spread over a long period extending from about 35,000
to 30,000 BP, which would make Nazlet Khater-4 one of the oldest
examples of underground mining activity in the world. The lithic
assemblage from Nazlet Khater-4 no longer showed any trace of the
Levallois technique. Production aimed at obtaining simple blades
PREHISTORY 23
from single platform cores. Among tools, some end-scrapers, burins,
and denticulates but also some bifacial foliates and bifacial axes occur.
As no other such sites have been disclosed in Egypt, it is difficult to
establish its importance for the evolution of Egyptian prehistory. Next
to the mine, and obviously in association with it, excavators revealed a
grave in which the deceased was buried lying on his back with a bifacial
axe next to his head.
The next oldest phase, after Nazlet Khater-4, was the Shuwikhatian
industry, which is attested at several sites in the neighbourhood of
Qena and Esna. The type site Shuwikhat-i has been dated to around
25,000 BP. The study of the environment and the animal remains
shows that the site, which was located within the floodplain at that
time, functioned as a hunting and fishing camp. It is possible that the
Shuwikhatian is contemporaneous with a short wetter spell, but this
climatic change was not important enough to bring about the repopu-
lating of the Western Desert, which remained devoid of human occu-
pation. The Shuwikhatian is characterized by robust blades obtained
from opposed platform cores. Most common tools are denticulated
blades, end-scrapers, and burins.
Within the framework of North Africa and south-west Asia, the
Upper Palaeolithic of Egypt seems to be rather insular, although it is
possible that there were some connections with the Dabban industry
of Cyrenaica and the Ahmarian of southern Israel and Jordan.
The Late Palaeolithic
In contrast to the Upper Palaeolithic Period, many Late Palaeolithic
sites have been found in Upper Egypt, dating between 21,000 and
12,000 BP. The climate remained hyperarid, as it had been during the
Upper Palaeolithic, but the river Nile had begun to contain less water
and more clays because of aridity in its headwaters and because of
important erosion activity due to the late glacial coldness affecting the
highlands of Ethiopia. These clays were deposited in the Nile Valley,
filling it in Upper Egypt with thick alluvia and resulting in a floodplain
that, in Nubia, was 25-30 m. higher than the modern one. No Late
Palaeolithic sites have been recorded in Lower and Middle Egypt,
apparently because this part of the Nile Valley was more deeply cut,
due to a very low water level in the Mediterranean Sea, a little more
than 100 m. below the present level. This resulted in regressive ero-
sion along the Nile, creating a surface that has been covered by more
recent alluvia, concealing the sites from archaeologists.
24 STAN H E N D R I C K X and PIERRE V E R M E E R S C H
There is great typological variety among Late Palaeolithic sites, and,
because of our limited knowledge of the Upper Palaeolithic, it is diffi-
cult to determine the origins of the Late Palaeolithic. Among the dif-
ferent groups, the Fakhurian (21,000-19,500 BP) and the Kubbaniyan
(19,000-17,000 BP) are the oldest. Although the Kubbaniyan was
defined at Wadi Kubbaniya, near Aswan, sites have also been found
near Esna and Edfu. At Wadi Kubbaniya, the sites occur in three differ-
ent physiographic settings, all of which are related to a temporary lake
barred yearly after the Nile flood inundation by a dune in the mouth of
the wadi. After the size of the dune became so significant that the
entire wadi was blocked, the lake was fed by the water table, thus creat-
ing an extremely favourable environment for hunter-gatherers. Some
of the sites are situated on a dunefield that was occasionally flooded by
the Nile; others are located on a flat silty plain of the wadi floor in front
of the dunes; and finally there are sites on hillocks of fossil dunes in
the flat area near the wadi mouth, which were surrounded by water
during the period of inundation.
Most sites at Wadi Kubbaniya are the result of repeated use by small
groups of people, perhaps several times a year, over a long period. The
floral remains clearly reflect seasonality. Many edible plants, such as
club-rush, camomile, and nut-grass tubers, must have been part of the
diet. The presence of nut-grass tubers is particularly remarkable, since
these would have had to have been thoroughly ground up in order to
remove the toxins and break up the fibres. This might well explain the
large number of grinding stones found at Wadi Kubbaniya. At Kub-
baniyan and other Late Palaeolithic sites, fish were caught seasonally
in large quantities, forming the major source of animal protein. One
annual fishing season is indicated by an overwhelming frequency of
catfish, indicating massive catches of spawning catfish, which appear
with the rising floods of July and August. A second fishing season is
characterized by the high frequency of surviving remains of yearling
and adult Tilapia and numerous catfish. This spectrum suggests that
fish were gathered in October or November in the shallow pools that
remained after the inundation. In addition to fishing, hunting for
hartebeest, wild cattle, and dorcas gazelle was an important aspect of
the subsistence pattern. Lithics mainly consisted of bladelets obtained
from opposed platform cores.
Four major tool classes are well represented in the Fakhurian.
Backed bladelets, some with Ouchtata retouch, are the most frequent,
followed by retouched pieces, perforators, notches, and denticulates.
End-scrapers are present but less frequent, while truncations and
PREHISTORY 25
burins are rare and generally poorly made. The tool inventory of the
Kubbaniyan is characterized by a predominance of backed bladelets,
often with a non-invasive nibbling retouch, representing up to 80 per
cent of all tools.
The kill-butchery camp site £71X12 near Esna belongs to the
Fakhurian or is closely related to it. This site, which consists of a dune
hollow in which a seasonal pond was fed by the rising groundwater
during the summer floods, attracted animals that were driven from the
floodplain by the rising water. This resulted in ideal hunting circum-
stances. There were three major prey animals: hartebeest, wild cattle,
and gazelle. This site most probably represents the basic manner of
subsistence during the late flood and the early post-flood period.
A distinctive feature of the Ballanan-Silsilian industry (16,000-
15,000 BP) is debitage from single and opposed platform cores. Tools
comprise backed bladelets and truncated bladelets. There was frequent
use of the microburin technique, an innovation also found in the
Negev and southern Israel and Jordan. While well-made burins are
quite common, Ouchtata-retouch and geometric microliths are rare,
while end-scrapers are never common.
Climatic changes by the end of the last Ice Age resulted in unusually
high Nile water discharges around 13,000-12,000 BP, creating excep-
tionally high floods. This 'Wild Nile' stage was caused by climatic con-
ditions in sub-Saharan Africa, but in Egypt itself there was no local
rainfall. One site that was out of reach of the catastrophic inundations
of the Wild Nile was Makhadma-4, an example of the Afian industry
(12,900-12,300 BP), located about 6 m. above the modern floodplain, a
little to the north of Qena. It was on the desert fringe, in a flat embay-
ment resulting from the joining of different wadi bottoms, and its rich
array of fish remains includes 68 per cent Tilapia and 30 per cent
Clarias, the rest consisting of Barbus, Synodontis, and Lates. The high
amount of Tilapia and the small size of both Tilapia and Clarias indi-
cate that fishing must have been practised rather late within the post-
flood season. The fish must have been caught in shallow basins
through which the fishers were able to wade. The small size of the fish
also suggests that sophisticated tackle, such as thrust baskets, nets,
and scoop baskets, were used. The fish that were caught in large quan-
tities were probably not all intended for immediate consumption, and
the fact that the site includes pits containing a large amount of char-
coal suggests that fish were being deliberately preserved by drying. The
expansion of the site demonstrates that the locality was repeatedly
used over a long period.
26 STAN H E N D R I C K X and P I E R R E V E R M E E R S C H
The Isnan industry has been attested on several sites between Wadi
Kubbaniya and the Dishna plain. The assemblage is characterized by
rough knapping techniques, resulting in thick and wide flakes, and the
tool inventory is largely dominated by end-scrapers on flakes. At the
site of Makhadma-2, fishing for Clarias seems to have been the eco-
nomic basis. The occupation dates to 12,300 BP and therefore coincides
with the Wild Nile floods.
The Qadan industry, between the second cataract and southern
Egypt, is a microlithic flake assemblage, but its interest lies primarily
in the fact that it is associated with three cemeteries. The most impor-
tant is the cemetery at Gebel Sahaba, where fifty-nine skeletons were
excavated. Each of them was in a semi-contracted position on the left
side of the body, with the head to the east, facing south. The graves are
simple pits, covered with slabs of sandstone, and the associated lithic
material can be attributed to the final phase of the Qadan, around
12,000 BP. Out of the fifty-nine individuals, twenty-four showed signs
of a violent death attested either by many chert points embedded in the
bones (and even inside the skull) or by the presence of severe cut
marks on the bones. The existence of multiple burials (including a
group of up to eight bodies in one grave) confirms the picture of
violence. Since women and children represent about 50 per cent of this
population, it is most probable that the Gebel Sahaba cemetery rep-
resents an exceptionally dramatic event. It has been suggested that this
may have been a consequence of the increasingly difficult conditions
of living caused by the Wild Nile and the subsequent cutting down of
the Nile into its former floodplain. A smaller cemetery, almost oppo-
site Gebel Sahaba on the other side of the Nile, where such 'projectiles'
were entirely absent from the bodies, shows that death was not always
caused by violence at this date.
The chronological position of the Sebilian industry is not clear,
despite the fact that it is the most widespread Late Palaeolithic indus-
try, occurring from the second cataract to the north of the Qena bend.
The Sebilian lithic technology is characterized by the manufacture of
large flakes and a preference for quartzitic sandstones or volcanic
rocks as raw material. This is completely incompatible with the lithic
tradition of the other Late Palaeolithic industries. The Sebilian might,
therefore, represent intrusive groups from the south, moving north-
wards along the Nile.
Before leaving the Late Palaeolithic it is necessary to mention that
there may already have been rock art in the Nile Valley at this remote
date. At Abka, near the second cataract, in Sudanese Nubia, a possible
PREHISTORY 27
instance of Late Palaeolithic rock art has been identified at 'site XXXI I'.
In Egypt proper, there are also a few rock-art sites that appear to be pre-
Neolithic in date. Among the most remarkable drawings are the fish
traps represented at el-H6sh, south of Edfu. The plan of these laby-
rinthine fish fences consists of a complicated layout of curvilinear
shapes leading to mushroom-shaped ends, which functioned as the
actual traps. This type of fishing in shallow waters would fit well with
the observations concerning massive fishing at Late Palaeolithic sites,
such as Makhadma-4.
After the Late Palaeolithic, there was a hiatus in the occupation of
the Nile Valley. No human presence has between attested in Egypt
between 11,000 and 8000 BP, apart from a group of very small Ark-
inian sites (around 9400 BP) in the region of the second cataract. It has
been suggested that the attested down-cutting of the Nile during this
period, with a reduced floodplain as a consequence, had a detrimental
effect on the environmental conditions. Although this environmental
change undoubtedly took place, it seems highly unlikely that the Nile
Valley was entirely deserted at this date. It is more likely that these sites
are simply covered by modern alluvial deposits, considering a narrow-
ing of the floodplain and the normal location of sites on the fringe of
the low desert.
Saharan Neolithic/Ceramic
The Western Desert was abandoned towards the end of the Middle
Palaeolithic, and people returned there only in about 9300 BC, as a
result of the Holocene wet phase. Because there was no human
presence immediately before the Early Neolithic, and because the area
was also unihabited after this period, the conditions of archaeological
preservation are very good. Since the annual rainfall in the early Holo-
cene was still only about 100-200 mm. (all of which probably fell
during a brief summer season), only desert-adapted animals such as
the hare and the gazelle could live there. Nevertheless, this meant an
enormous amelioration of living conditions in comparison with the
Upper and Late Palaeolithic. The amount of rainfall was not contin-
uous and arid intervals are most important for chronological differ-
entiation. The rainfall is a result of the northward shift of the monsoon
belt; therefore human occupation in the Western Desert started from
the south. The settlers came most probably from the Nile Valley, an
idea that is primarily based on the absence of other possibilities, but
28 STAN H E N D R I C K X and P I E R R E V E R M E E R S C H
seems to be confirmed by similarities with the lithic technology of sites
in the Nubian Nile Valley.
In Egypt, the earliest 'Neolithic' cultures emerged in the Western
Desert. It should, however, be made clear from the outset that agri-
culture has not yet been attested for the Saharan Neolithic. This
culture has been identified as Neolithic purely on the basis of the evi-
dence for cattle herding. The Saharan Neolithic is, therefore, com-
pletely different from the Neolithic culture that emerged at about the
same time in Israel, where the phrase 'Neolithic economy' is a syn-
onym for the process whereby agriculture was introduced and later
joined by animal domestication. Most probably, the Neolithization
process that occurred in Egypt was'completely independent from that
in Israel. Because of the absence of agriculture and the presence of
some ceramics, it has been suggested that the term 'Ceramic' should
be applied to this Saharan culture, as opposed to 'Neolithic'.
Two main periods can be distinguished: the Early Neolithic (8800-
6800 BC) and a more recent period consisting of Middle (6500-5100
BC) and Late Neolithic (5100-4700 BC). For the Early Neolithic, the
most complete information comes from sites near Nabta Playa and Bir
Kiseiba. Most sites are small, short-term camps of hunter-gatherers.
Larger sites are always located in the lower parts of playa basins.
Although these sites were apparently used for longer periods, they too
were seasonally abandoned, since the lower parts of the playa basins
were seasonally flooded. Sedentism was not yet known.
Lithics are characterized by numerous backed bladelets (often
pointed) and some rare geometries, as well as tools produced with the
microburin technique. Every faunal collection of any size includes a
few bones of cattle, which, according to the excavators, were domestic-
ated (although this interpretation is not generally accepted), since it
seems unlikely that cattle would have been able to survive without
human aid in an arid environment that otherwise supports only
desert-adapted animals. It is particularly significant that the fauna
includes no remains of hartebeest, an animal that often occurs in the
same ecological niche as wild cattle. It therefore seems most plausible
that pastoralists were keeping wild cattle in an environment where the
cattle would not have been able to survive by themselves. Before 7500
BC, it is possible that people and cattle came into the desert only during
and after the summer rains, which coincide with the period of inun-
dation of the Nile Valley, during which it would have been difficult to
find herding facilities. After 7500 BC, the digging of wells is attested at
Bir Kiseiba and other sites. Some of the wells have a shallow side basin
PREHISTORY 29
for watering animals. The paucity of cattle bones indicates that the
animals were not used for meat production but mainly for protein in
the form of milk and blood. In this manner, while humans helped
cattle to survive in the Western Desert, the animals permitted people
to live in this difficult environment. As well as keeping cattle, these
people were hunting local wild animals, predominantly hare and
gazelle.
It is presumed that the stone-grinding equipment found at nearly all
sites from the beginning of the Early Neolithic was used for processing
harvested wild plant foods, but the plants themselves have only been
recovered at site £-75-6 at Nabta Playa. Among them are wild grasses,
Ziziphus fruits, and wild sorghum.
All Early Neolithic sites, even the earliest, have yielded potsherds,
albeit in very small numbers. The vessels had very simple shapes, but
they were carefully made and fired, and all of them were decorated.
Usually the entire surface of the vessel was filled with lines and points,
often created by comb or cord impressions, and the general appear-
ance of the vessels was probably imitating basketry. Ostrich eggshells,
used as containers for water, were far more common than pottery
vessels. The relative dearth of potsherds suggests that pottery was not
being used regularly in daily life. It is not possible to determine the
exact function of the pottery, but it obviously must have had great
social significance and—because of the decoration—probably also
symbolic meanings. It seems beyond doubt that these ceramics were
an independent, African invention.
Site £-75-6 (around 7000 BC) is one of the most interesting Early
Neolithic localities at Nabta Playa. This drainage basin received enough
water to store large quantities of subsurface water, which could be
reached with wells during the dry season. The site consists of three or
four rows of huts, probably each representing different shore lines of
the lake, accompanied by bell-shaped storage pits and wells. It is not
possible to estimate the number of huts that were contemporaneously
in use. Despite its size, this was not a permanent settlement.
It was during the Middle and Late Neolithic periods (6600-5100
and 5100-4700 BC respectively) that the human occupation of the
Western Desert reached its peak. Sites of this date are very numerous,
and, although most of them are small, there are also some very large
ones. Structures are more common than before, including wells, slab-
lined houses, and evidence for wattle-and-daub constructions. The
large settlements, near the playa lakes, probably represent permanent
settlements, while the smaller ones are more likely to derive from task
30 STAN H E N D R I C K X and P I E R R E V E R M E E R S C H
forces of herdsmen who set out from the large sites to drive their
animals across the grassland after the summer rains. The presence of
shells proves that there was contact with both the Nile Valley and the
Red Sea, but it is likely that the people themselves remained in the
desert all year round. As in the Early Neolithic, domestic cattle were
kept as living sources of protein, but, despite the fact that sheep and
goat also appear for the first time during this period (about 5600 BC),
most meat was still obtained from wild animals. Again it is usually
assumed that a large variety of wild plants was consumed at this date.
In the Middle Neolithic there was a dramatic shift in lithic tech-
nology. Blade production was no longer so prevalent, and instead there
was a gradual introduction of bifacial flaking for foliates and concave-
based arrowheads. Geometries, except lunates, were rare. At Late Neo-
lithic sites, basin-type grinding stones are common. Ground and
polished stone celts, palettes, and ornaments are also present in
assemblages of this date: together with side-blow flakes, they are
considered characteristic of the period. Ceramics before 5100 BC fall
within the 'Saharo-Sudanese' or 'Khartoum' tradition, similar to the
Early Neolithic ceramics, although the decoration tends to consist of
more complicated patterns. Somewhat before 4900 BC, this type of
pottery disappeared somewhat abruptly and was replaced by burn-
ished and smoothed (occasionally black-topped) pottery at Nabta Playa
and Bir Kiseiba. The reason for this sudden transition is by no means
obvious, but its occurrence in the Western Desert is of great impor-
tance for our understanding of the origin of the Predynastic cultures in
the Nile Valley.
At Nabta Playa, a remarkable megalithic complex has been dis-
covered adjacent to an exceptionally large Late Neolithic site. It con-
sists of three parts: an alignment of 10 large (2 x 3 m.) stones, a circle of
small upright slabs (almost 4 m. in diameter), and two slab-covered
tumuli, one of which had an underlying chamber containing the
remains of a long-horned bull. Small alignments of megaliths have
also been observed elsewhere in the Nabta Basin. Although their func-
tion is not obvious, these megalithic constructions clearly represent
public 'architecture' and therefore refer to increasing social com-
plexity.
In the Dakhla Oasis, several archaeological units have been dis-
tinguished, and the main phases are known as Masara, Bashendi,
and Sheikh Muftah. The Masara phase is contemporaneous with
(and similar to) the Early Neolithic of Nabta Playa and Bir Kiseiba.
The Middle and Late Neolithic Bashendi and Sheikh Muftah cultures
PREHISTORY 31
continued into dynastic times. These two Neolithic cultures are
characterized by contrasting types of settlement, with Sheikh Muftah
sites situated in close correlation with lake sediments and Bashendi
sites being located just outside the oasis proper. It has been suggested
that two different types of occupation may be represented. Thus the
Sheikh Muftah sites might represent full-time oasis-dwellers, while
the Bashendi sites might have belonged to periodic visitors, probably
nomadic pastoralists. Starting in about 5400 BC, people relied heavily
on their flocks and herds of domesticated animals (imported from the
Levant and mainly consisting of goats), while still undertaking some
hunting.
The lithic technology of the Bashendi culture is similar to that of the
Middle and Late Neolithic, with the addition of a variety of arrowheads,
often bifacially retouched. From a little before 4900 BC, burnished and
smoothed pottery, somewhat similar to fragments of vessels found at
Nabta Playa and Bir Kiseiba, was produced at Bashendi sites, while
black-topped pottery occurs occasionally at sites in the Dakhla Oasis.
In the south-east corner of Dakhla, various stone-built structures are
present; it remains unclear how typical this oasis was for the whole of
the Western Desert, but it obviously contains the strongest cultural
parallels with the Nile Valley.
After 4900 BC and especially from 4400 BC onwards, the desert
became less and less inhabitable because of the onset of the arid
climate that continues up to the present day. However, a few select
areas were still occupied in historic times.
The Nile Valley Epipalaeolithic
From 7000 BC onwards, human groups are again present in the Nile
Valley, but the number of Epipalaeolithic sites is very limited, and they
have only been discovered in exceptional circumstances. Thus, only
two cultures—the Elkabian and the Qarunian—can be distinguished.
During the Epipalaeolithic, there was a continuation of the Palaeolithic
style of subsistence, based on hunting, fishing, and gathering.
At Elkab, a few small Epipalaeolithic sites, dating to about 7000-
6700 BC, have been found in an exceptionally good state of preserva-
tion because they are located within the far more recent Dynastic-
Period enclosure wall. The sites were located on the beach of a
silting-up Nile branch, the occupations having taken place after the
floodplain inundations. The Epipalaeolithic fishing practices were
more highly developed than those of the Late Palaeolithic. Indeed,
32 STAN H E N D R I C K X and P I E R R E V E R M E E R S C H
fishing took place not only in the receding high waters but also in the
main channel of the Nile, which suggests that by this date the people
must have been using boats with a reasonable degree of stability.
Because of the more humid climate, hunting for aurochs, dorcas
gazelle, and barbary sheep was possible in the wadi area. The Elkabian
industry is microlithic, including a large number of microburins. It is
readily comparable with the Early Neolithic of the Western Desert. The
presence of numerous grinding stones cannot be used as evidence for
plant processing, because red pigment was still visible on a number of
them. The presence of an Elkabian occupation in the Tree Shelter site
at Wadi Sodmein, near Quseir in the Eastern Desert, suggests that the
Elkabians should be viewed as nomadic hunters, following east-west
routes with wintertime fishing and hunting in the Nile Valley and
exploitation of the desert during the wet summer.
The Qarunian is a renaming of the Faiyum B culture (attributed by
Caton-Thompson to the Mesolithic). Qarunian sites, originally located
on high ground overlooking the Proto-Moeris Lake (which dates to
about 7050 BC), have been identified in the area north and west of the
present Faiyum lake. The Holocene history of the lake is characterized
by a number of fluctuations, which are of the utmost importance for
the understanding of the history of occupation around the lake. There
were three transgressions (that is, submergences of land caused by
rises in sea level) preceding the Neolithic. In the Qarunian phase, fish-
ing conditions were exceptionally good in the shallow waters of the
lake and it comes as no surprise that fish provided the basis of sub-
sistence. In addition, hunting and food gathering were practised. The
Qarunian industry is also microlithic and fits in with the general tech-
nological context of the Elkabian and the Early Neolithic of the
Western Desert. A single burial is known for the Qarunian. The body
of a woman aged about 40 was buried in a slightly contracted position,
on the left side, head to the east, facing south. Her physical character-
istics are far more modern than the Late Palaeolithic Mechtoids.
The presence of microlithic industries in the neighbourhood of
Helwan has been known since the nineteenth century, showing simi-
larities with the Pre-Pottery Neolithic from the Levant, but the real
significance of these industries cannot be determined because of the
poor information available. Also in the Eastern Desert, in the Red
Sea mountains, there are Neolithic settlements. According to the evi-
dence from Sodmein Cave near Quseir, these settlers would have
introduced domesticated sheep/goat during the first half of the sixth
millennium BC.
PREHISTORY 33
The Nile Valley Neolithic
In the Nile Valley, no other traces have been found of the people that
dwelled in the Eastern and Western Desert, except for the Elkabian and
Qarunian cultures. There is no indication of any shift towards agri-
culture, which was already well established in the Levant from about
8500 BC onwards. The Egyptian population seems to have continued
their traditional way of life, based on fishing, hunting, and gathering.
Unfortunately, we have no information on human population in the
Nile Valley for the period between 7000 and 5400 BC.
The Tarifian culture is known from a small site at el-Tarif, in the
Theban necropolis, and from another one in the neighbourhood of
Armant. It is a ceramic phase of a local Epipalaeolithic culture, which,
however, remains unknown. It shows no connection with the later
Naqada culture, and its relation with the Badarian culture is also
unclear, although apparently the lithic industries show no close links.
The Tarifian is characterized by a flake industry, with, on the one hand,
a small microlithic component referring to the Epipalaeolithic and on
the other hand some bifacial pieces announcing Neolithic technology.
Pottery, mainly organic tempered, is restricted to a number of small
fragments. Traces of agriculture or animal breeding are lacking. No
remains of structures have been found and the settlement at el-Tarif is
presumed to have been similar to Final Palaeolithic camps.
The Faiyumian culture, which is identical to Caton-Thompson's
Faiyum A culture, starts in about 5450 BC and disappears around 4400
BC. Technological and typological differences between the Qarunian
and the Faiyumian are so significant that there can be no question of
the Faiyumian having developed out of the Qarunian. The Faiyumian
lithic technology is clearly related to that of the Late Neolithic in the
Western Desert. People were living along the ancient beach of lake
Faiyum, and the most important remains found so far are groups of
storage pits for grain, often lined with matting. For the first time in
Egypt, agriculture, most probably introduced from the Levant, is clearly
the basis of subsistence. Six-row barley and emmer wheat were grown
and probably also flax. Because the storage pits are in groups, it is sup-
posed that agriculture was practised on a community basis. One stor-
age area consists of 109 silos, with diameters between 30 and 150 cm.,
and a depth between 30 and 90 cm., which obviously represents a
major storage capacity. Besides agriculture, animal husbandry was
certainly important, with evidence of the presence of sheep/goat,
cattle, and pigs. Fishing also remained basic to the economy.
34 STAN HENDRICKX and PIERRE VERMEERSCH
Faiyumian pottery is coarsely made and fashioned into simple
shapes. A limited number of pieces were red coated and burnished,
but no decorated pottery has been found. The lithic industry is a flake
industry with a minor bifacial component. Links with distant places,
presumably indirect, have been inferred from seashells of both Med-
iterranean and Red Sea species, as well as cosmetic palettes of Nubian
diorite and beads of green feldspar, but no copper has been found.
The large settlement of Merimda Beni Salama is situated on a low
terrace at the edge of the western Nile Delta. The settlement debris has
an average depth of 2.5 m. and consists of five levels, corresponding to
three main cultural stages. These span a long period between 5000
and 4100 BC. Level I, labelled Urschicht, is clearly different from the
more recent stages, and is characterized by ceramics without temper,
both polished and unpolished; decoration consisting of a herringbone
pattern is typical of this ceramic phase (but neverthless rare). Level I
lithics are characterized by a flake technology and the presence of
numerous end-scrapers as well as bifacial retouched tools. The settle-
ment remains of this level are restricted to hearths and possible
remains of flimsy shelters. The economy was probably a mixture of
agriculture, animal husbandry (sheep, cattle, and pig) related to the
Levant, but also fishing and hunting. Radiocarbon dates suggest a
chronological position at about 4800 BC, although this estimate is con-
sidered by the excavator of the site to be too recent. Ceramics with
herringbone pattern decorations have also been found in recent exca-
vations at the Sodmein Cave, near Quseir.
There was probably a break in occupation between levels I and II at
Merimda. Level II, known as the Mittleren Merimdekultur and con-
sidered by the excavator to be related to Saharo-Sudanese cultures, is
marked by a denser occupation of the site, with simple oval dwellings
of wood and wickerwork, well-developed hearths, storage jars sunk in
the clay floors, and large clay-coated baskets in accessory pits serving
as granaries. Contracted burials were also located among the dwell-
ings. Ceramics are radically different from the previous period because
they are straw tempered, but the shapes were still simple. Nearly half
of the pottery was polished, and none of it appears to have been decor-
ated. The lithic industry is predominantly bifacial. Concave-base
arrowheads appear for the first time at Merimda. A large number of
artefacts in bone, ivory, and shell have been found, and three-barbed
harpoons are typical. Agriculture continues as the basic economic
activity, but, judging from the number of bones, cattle become more
important, while fishing and hunting are both still well attested. No
P R E H I S T O R Y 35
radiocarbon dates are available, but a date between 5500 and 4500 BC
has been suggested by the excavator.
Levels III-V are called Jungeren Merimdekultur, and correspond to
the phase identified as 'classic' Merimda culture by the site's first exca-
vator in the early twentieth century. The settlement at this date con-
sisted of a large village of mud dwellings, huts, and work spaces.
Well-made oval houses were laid out densely along narrow streets. The
buildings are between 1.5 and 3 m. wide, with floors dug into the
ground to a depth of about 40 cm., and walls made of straw-tempered
mud and mud clods; they were roofed with light materials such as
branches and reeds. Within the houses, hearths, grinding stones,
sunken water jars, and holes once containing pottery were discovered,
indicating a variety of domestic activities carried out indoors. Gran-
aries were associated with individual dwellings, demonstrating that
the family units had probably become more or less economically
independent. In general, it can be concluded that settlement organiza-
tion at Merimda certainly represents a 'formal' organization of village
life. Contracted burials in shallow oval pits are located among the
houses. Remarkably, hardly any grave goods were included. Both the
absence of grave goods and the location of burials within the settle-
ment are aspects of funerary protocol that appear to contrast sharply
with Upper Egyptian burial customs. However, it seems likely—given
the limited number of graves (less than 200), the restricted presence of
adult males, and the occurrence of stratigraphic confusions—that only
children and adolescents were buried within the settlement, which is
also well known for Upper Egypt, while the adults were buried in areas
that were only later occupied by houses. It is however to be supposed
that the majority of the cemeteries remain at present undiscovered.
The ceramic evolution shows a tendency towards closed shapes.
Polishing is used for decorative effects, and during this period polished
pottery becomes dark red/black, with half of the repertoire comprising
large rough vessels. The bifacial chert technology is improved, com-
pared to the previous phase of occupation at Merimda. Implements
made from bone, ivory and shell remain frequent. Most remarkable,
however, are a small number of figurines, one of which is a roughly
cylindrical head of a human figure, covered with small holes that evi-
dently served for the application of hair and a beard. The shape of the
holes seems to indicate that feathers were used for the imitation of hair
and beard. The head must originally have been fixed to a wooden body,
which makes it the oldest human representation yet known from
Egypt. According to the excavator, this most recent period at Merimda
36 STAN HENDRICKX and P I E R R E  VERMEERSCH
would be equivalent to the Faiyumian. However, this is only partially
confirmed by radiocarbon dating, according to which the Jungeren
Merimdekultur is to be assigned to the period between 4600 and
4100 BC, and would therefore be contemporaneous only with the
second half of the Faiyumian.
Still in Lower Egypt, several sites in the neighbourhood of Wadi
Hof-Helwan consist of separated settlements and cemeteries. They
represent a Neolithic culture that has been called the el-Omari culture,
after its discoverer, Amin el-Omari. It dates to about 4600-4350 BC
and is therefore contemporaneous with the Jungeren Merimdekultur. In
the settlements, mainly pits have been found, used for storage or the
dumping of refuse. Associated constructions could not be described
exactly, but were certainly very light. Cemeteries developed in settle-
ment areas that were no longer in use. All graves are pit burials, with
contracted bodies, ideally orientated to the south, lying on their left side.
The el-Omari pottery always has an organic temper; the shapes are
very simple and many vessels are polished, often with a red coating.
The lithic industry shows the same improvement of the bifacial tech-
nique as at Merimda II-V. Agriculture and animal husbandry (goat/
sheep, cattle, pigs) are the base of subsistence, but fishing was par-
ticularly important at el-Omari. Desert hunting, on the contrary, was
hardly practised at all.
The presence of domesticated goats from about 5900 BC, in both the
Western and Eastern deserts, is astonishing when compared to the age
of their presence in the Nile Valley, where they did not appear until
some five centuries later.
The Badarian Culture
The Badarian culture, which is the earliest attestation of agriculture in
Upper Egypt, was first identified in the region el-Badari, near Sohag. A
large number of mainly small sites near the villages of Qau el-Kebir,
Hammamiya, Mostagedda, and Matmar yielded a total of about 600
graves and forty poorly documented settlements.
The chronological position of the Badarian culture is still the subject
of some debate. Its relative chronological position in relation to the
more recent Naqada culture was established some time ago through
excavation at the stratified site of North Spur Hammamiya, and, accord-
ing to a number of thermoluminescence dates, the culture might
already have existed by about 5000 BC. However, it can only be defi-
nitely confirmed to have spanned the period around 4400-4000 BC.
PREHISTORY 37
The existence of a still earlier culture, called the Tasian, has been
claimed. This culture would have been characterized by the presence
of round-based caliciform beakers with incised designs filled with
white pigment, which are also known from contexts of similar date in
Neolithic Sudan. However, the existence of the Tasian as a chrono-
logically or culturally separated unit has never been demonstrated
beyond doubt. Although most scholars consider the Tasian to be
simply part of the Badarian culture, it has also been argued that the
Tasian represents the continuation of a Lower Egyptian tradition,
which would be the immediate predecessor of the Naqada I culture.
This, however, seems rather implausible, first because similarities
with the Lower Egyptian Neolithic cultures are not convincing, and,
secondly, because of the Tasian's obvious ceramic links with the
Sudan. If the Tasian must be considered as a separate cultural entity,
then it might represent a nomadic culture with a Sudanese back-
ground, which interacted with the Badarian culture.
Despite the existence of some excavated settlement sites, the Badar-
ian culture is mainly known from cemeteries in the low desert. All
graves are simple pit burials, often incorporating a mat on which the
body was placed. Bodies are normally in a loosely contracted position,
on the left side, head to the south, looking west. Graves of very young
children are lacking. There is sufficient evidence to show that these
were buried within the settlement, or rather within parts of the settle-
ments that were no longer used. Analysis of Badarian grave goods
demonstrates an unequal distribution of wealth. In addition, the
wealthier graves tend to be separated in one part of the cemetery. This
clearly indicates social stratification, which still seems limited at this
point in Egyptian prehistory, but which became increasingly impor-
tant throughout the subsequent Naqada Period.
The pottery that accompanies the dead in their graves is the most
characteristic element of the Badarian culture. All pottery is made by
hand, from Nile silts, which, except for the very fine wares, always has
a very fine organic temper. This very characteristic temper is always
finer than that used for the so-called rough ware during the Naqada
Period. For their best products, the Badarian potters spared no efforts
in refining the clay and obtaining very thin walls, which have never
been equalled in any subsequent period of the Egyptian past. Pottery
shapes are simple, mainly comprising cups and bowls with direct rims
and rounded base. A significant proportion of the vessels are black
topped, but they generally have a more brownish surface than the
Naqada I black-topped pottery. Red slip, with which the Naqada I
38 STAN HENDRICKX and PIERRE VERMEERSCH
black-topped pottery is covered, is far more exceptional for the Badar-
ian. The most characteristic element of the Badarian pottery is the
'rippled surface' that is present on the finest pottery, meaning that the
surface has been combed with an instrument and afterwards polished,
resulting in a very decorative effect. Carinated vessels are also con-
sidered highly characteristic of the culture, but decorated pottery is
rare: occasionally, incised, white-filled, geometrical motifs have been
applied, perhaps imitating basketry.
The lithic industry is mainly known from settlement sites, although
the finest examples have been found in graves. It is principally a flake
and blade industry, to which a limited number of remarkable bifacial
worked tools are added. Predominant tools are end-scrapers, perfor-
ators, and retouched pieces. Bifacial tools consist mainly of axes,
bifacial sickles, and concave-base arrowheads. It should also be noted
that the characteristic side-blow flakes were also present in the
Western Desert.
Other products of the Badarian culture include such personal items
as hairpins, combs, bracelets, and beads in bone and ivory. The reper-
toire of greywacke cosmetic palettes was at this date limited to long
rectangular or oval shapes, but they would later become very charac-
teristic aspects of the Naqada culture, when they were produced in a
great variety of shapes. A few clay and ivory female statuettes have
been found, varying immensely in style from fairly realistic examples
to others that are highly stylized. It should also be noted that ham-
mered copper was present in limited quantities.
For a long time it was thought that the Badarian culture remained
restricted to the Badari region. However, characteristic Badarian finds
have also been made much further to the south, at Mahgar Dendera,
Armant, Elkab, and Hierakonpolis, and also to the east, in the Wadi
Hammamat.
Originally, the Badarian culture was considered a chronologically
separate unit, out of which the Naqada culture developed. However,
the situation is certainly far more complex. For instance, the Naqada I
period seems to be poorly represented in the Badari region; therefore,
it has been suggested that the Badarian was largely contemporary with
the Naqada I culture in the area to the south of the Badari region. How-
ever, since a limited number of Badarian or Badarian-related artefacts
have also been discovered south of Badari, it might instead be argued
that the Badarian culture was present between at least the Badari
region and Hierakonpolis. Unfortunately most of these finds are very
limited in number, and a comparison with the lithic industry or the
P R E H I S T O R Y 39
settlement ceramics from the Badari area is in most cases impossible
or has not yet been published. The Badarian culture may, therefore,
have been characterized by regional differences, the unit in the Badari
region itself being the only one that has so far been properly investi-
gated or attested. On the other hand, a more or less 'uniform' Badarian
culture may have been represented over the whole area between Badari
and Hierakonpolis, but, since the development of the Naqada culture
took place more to the south, it seems quite possible that the Badarian
survived for a longer time in the Badari region itself.
The origins of the Badarian are equally problematic, having been
sought in all directions. For a long time the Badarian was considered to
have emerged from the south, because it was thought that the Badar-
ians had 'poor knowledge' of chert, which would show that they came
from the non-calcareous part of Egypt to the south; on the other hand,
the origins of agriculture and animal husbandry were assumed to lie in
the Near East. The theory that the Badarian originated in the south is,
however, no longer accepted. The selection of chert is perfectly logical
for the Badarian lithic technology, which seems to show links with the
Late Neolithic from the Western Desert. Rippled pottery, one of the
most characteristic features of the Badarian, probably developed from
burnished and smudged pottery, which is present both in late Sahara
Neolithic sites and from Merimda in the north down to the Khartoum
Neolithic sites in the south. Rippled pottery may thus have been a local
development of a Saharan tradition.
It seems obvious that the Badarian culture did not appear from a
single source, although the Western Desert was probably the pre-
dominant one. On the other hand, the provenance of domesticated
plants remains controversial: an origin in the Levant, via the Lower
Egyptian Faiyum and Merimda cultures, might be possible.
Evidence from Badarian settlements shows that the economy of the
culture was primarily based on agriculture and husbandry. Among the
contents of storage facilities, wheat, barley, lentils, and tubers have
been found. A number of circular constructions at Hammamiya,
previously identified as houses, most probably represent small animal
enclosures. In some of them, 20-30 cm. thick layers of sheep or goat
droppings have been found. Furthermore, fishing was certainly very
important, and may have been the principal economic activity during
certain periods of the year. Hunting, on the other hand, was apparently
only of marginal importance.
Settlement sites in the Badari region show a pattern of small villages
or hamlets, which seem to have moved horizontally after fairly short
40 STAN H E N D R I C K X and P I E R R E V E R M E E R S C H
periods of occupation. Storage pits and vessels are the most obvious
features in these sites, which is, of course, partially due to their pref-
erential preservation. The constructions are all very light and seem in
most cases to have been temporary. Indeed, it is quite possible that the
settlements on low desert spurs that are attested in the Badari region
are only marginal outliers or seasonal encampments. On that basis,
the larger, permanent settlements would have been closer to the flood-
plain and would have long ago either been washed away by the Nile or
covered with alluvium, thus remaining unknown.
The temporary character of Badarian settlements is confirmed at
Mahgar Dendera, about 150 km. to the south of Badari. The site was
seasonally used from the end of the low-water season onwards, at the
moment when the harvest was finished and when areas of land suit-
able for herding had to be looked for along the Nile, within the alluvial
plain. Besides herding, the second economic activity at Mahgar Dend-
era was fishing, which was practised in the main channel of the Nile,
while it was at its lowest level. At Mahgar Dendera, the alluvial plain is
very small; therefore, the site is both close to the Nile and out of reach
of the inundation, allowing people to stay at the same place when the
inundation started and even at its highest point. During this period,
when the living conditions reached an annual low, a part of the flock,
mainly young males, seems to have been butchered. People had left
Mahgar Dendera before the alluvial plain became fordable, because at
that time they had to start working the fields, which cannot have been
situated at Mahgar Dendera because of the limited floodplain.
Only limited information is available concerning the foreign con-
tacts of the Badarian culture. Relations with the Red Sea are attested
through the presence of Red Sea shells in graves, while copper ore may
have come from the Eastern Desert or the Sinai. The latter may also
have already been the source of turquoise but recently the identifi-
cation of turquoise from Badarian contexts was shown to be errone-
ous. If there were contacts between the Badari region and the Sinai,
they most probably passed through the Eastern Desert rather than
Lower Egypt, where there appear to be no indications of the Badarian
culture. This possibility of Badari-Sinai links through the Eastern
Desert may eventually be confirmed by reported finds from the Wadi
Hammamat, which unfortunately are still not fully published.
3
The Naqada Period
(^.4000-3200 BC)
BEATRIX MIDANT-REYNES
The second major phase of the Predynastic Period—the Naqada cul-
ture—derives its name from the site of Naqada, in Upper Egypt, where
in 1892 Flinders Petrie uncovered a vast cemetery of more than 3,000
graves. Petrie, struck at once by the unusual nature of these burials,
compared with those previously known in Egypt, mistakenly ascribed
them to a group of foreign invaders. This group was supposed to have
continued in existence until the end of the Old Kingdom, and it was
even suggested that they might have been responsible for its decline.
Archaeologists in Egypt had grown used to monumental funerary
architecture, but the humble Naqada burials consisted of little more
than the body of the deceased in foetal position, wrapped in an animal
skin, sometimes covered by a mat, and most often deposited in a simple
pit hollowed out of the sand. None of the offerings accompanying the
deceased corresponded to the usual hallmarks of pharaonic civiliza-
tion, as it was recognized in Petrie's day. The pottery vessels of black-
topped polished red ware, zoomorphic schist palettes, combs and
spoons of bone or ivory, and flint knives and other artefacts together
constituted a peculiar type of assemblage. Jacques de Morgan was the
first to suggest that these might be the remains of a prehistoric popu-
lation. Petrie then set about testing de Morgan's assumption scien-
tifically; eventually, after excavating thousands of other graves from
comparable sites, he was able to establish the first chronology of Pre-
dynastic Egypt. Petrie must, therefore, undoubtedly be regarded as the
father of Egyptian prehistory.
42 BEATRIX MIDANT-REYNES
Chronology and Geography
Having established that the graves were Predynastic, the next task was
to organize the considerable quantity of material uncovered, and to
place the newly defined Predynastic culture within a chronological
framework. Using the pottery from 900 graves in the cemeteries of
Hiw and Abadiya, Petrie devised a method of seriation that formed the
basis for a system of'sequence dates', in which the new categories of
Map of Egypt showing principal
sites of the Naqada I and II
phases
THE NAQADA P E R I O D 43
pottery were defined according to the form and decoration of the
vessels. Petrie reached the intuitive hypothesis that wavy-handled pots
evolved gradually from globular vessels with clearly moulded func-
tional handles towards cylindrical forms on which the handles were
merely decorative. The 'sequence-dates' chronology was initially organ-
ized around this concept of evolution in wavy-handled design.
A table of fifty sequence dates resulted, numbered from 30 onwards,
in order to leave space for earlier cultures that had not yet been dis-
covered. This turned out to be a wise precaution, given that Brunton's
excavations at Badari would later result in the identification of the
Badarian Period, the first stage of the Upper Egyptian Predynastic (see
Chapter 2). The lengths of the individual phases represented by each of
these sequence dates were uncertain, and the only link with any abso-
lute date was that between SD 79-80 and the accession of King Menes
at the beginning of the ist Dynasty, which was assumed to have taken
place in c.3ooo BC.
The sequence dates were grouped into three periods. First, there
was the Amratian (or Naqada I), from the type site of el-Amra, contain-
ing styles SD 30-38; this phase corresponds to the maximum develop-
ment of the black-topped red ware and of vessels with painted white
decorative motifs on a polished red body. Secondly, there was the
Gerzean (or Naqada II), from el-Gerza, containing styles SD 39-60
and characterized by the appearance of pottery with wavy handles,
coarse utilitarian ware, and decorations comprising brown paint on a
cream background. Finally, there was Naqada III, representing the
final phase of SD 61-80, which was marked by the appearance of a so-
called late style, whose forms were already evoking Dynastic pottery.
According to Petrie, it was during the Naqada III phase that an Asiatic
'New Race' arrived in Egypt, bringing with it the seeds of pharaonic
civilization.
Scholars have frequently praised Petrie's relative dating system,
and, although various analyses have corrected it and improved its pre-
cision, the three basic phases of the late Predynastic have never been
fundamentally questioned, and today they still constitute the loom
upon which Egyptian prehistory is woven.
The reliability of the ceramics corpus is fundamental to the validity
of the system. In 1942 Walter Federn, a Viennese exile to the USA,
exposed certain flaws in Petrie's corpus. In order to classify the vessels
from de Morgan's collection in the Brooklyn Museum, he was obliged
to revise Petrie's groups, removing two of them from the sequence. It
was Federn who introduced a factor that had been ignored by Petrie:
44 BEATRIX MIDANT-REYNES
the fabric of the vessels. It also became apparent that a system based on
material from Upper Egyptian cemeteries was not necessarily transfer-
able either to the necropolises of the north or to those of Nubia.
In spite of its recognized shortcomings, Petrie's work nevertheless
formed the sole means of organizing the Predynastic into cultural
phases until the system devised by Werner Kaiser in the 19605, which
even then could not actually replace it. Kaiser seriated the pottery of 170
tombs from cemeteries 1400-1500 of Armant using the publication of
the site made by Robert Mond and Oliver Myers in the 19305. His work
showed that there was also a 'horizontal' chronology in the cemetery.
The black-topped red ware abounded in the southern part of the
cemetery, while the 'late' forms were concentrated towards the northern
end. A very detailed analysis of the classification, still based on Petrie's
corpus, allowed him to correct and fine-tune the sequence-dating
system. Petrie's three major periods were thus confirmed, but refined
by the addition of eleven subdivisions (or Stufen) from la to Illb. In 1989
Stan Hendrickx's doctoral thesis allowed Kaiser's system to be applied
to all of the Naqadan sites in Egypt. This resulted in slight modifications,
particularly to the transitional phases between Naqada I and II.
The other important progress in Predynastic chronology has
involved advances in absolute dating. Both Petrie's sequence dates and
Kaiser's Stufen constitute relative dating systems; they have a terminus
ante quern of £.3000 BC (the presumed date for the unification of Egypt),
but they cannot in themselves provide any absolute date for the begin-
nings and ends of each of the Naqada phases and subdivisions. The
necessary links to an absolute chronology were made possible in the
second half of the twentieth century by the development of methods of
dating based on the analysis of physical and chemical phenomena. As
far as the Egyptian Predynastic is concerned, thermoluminescence
(TL) and radiocarbon (€-14) dating are the most important of these
scientific methods.
Libby tested the accuracy of the radiocarbon dating system on
material from the Faiyum region, and since then the testing of sam-
ples for dating has been sufficiently systematic to enable the construc-
tion of a fairly precise chronological framework, in which Petrie's
three great phases have come to take their place. The first Naqada
phase (Amratian) lies between 4000 and 3500 BC, followed by the
second phase (Gerzean), from 3500 to 3200 BC, and the final Pre-
dynastic phase runs from 3200 to 3000 BC.
The geographical locations of Naqada I sites all lie within Upper
Egypt, from Matmar in the north to Kubaniya and Khor Bahan in the
THE NAQADA P E R I O D 45
south. This situation changes, however, in the Naqada II culture,
which is particularly characterized by a process of expansion: emerg-
ing from its southern nucleus, it diffuses northwards as far as the
eastern edge of the Delta, and also southwards, where it comes into
direct contact with the Nubian 'A Group'.
Naqada I (Amratian)
Petrie and Quibell uncovered several thousand Predynastic graves
between them (15,000 for the whole Predynastic Period). As a result,
our knowledge of the period was—for over a century—based almost
entirely on funerary remains.
In broad terms, the Amratian is not different from the earlier Badar-
ian culture. The burial rituals and the types of funerary offerings are so
similar that one wonders if the latter does not constitute an older,
regional version of the former.
In general, the Amratian dead were buried in simple oval pits in a
contracted position, lying on the left side with the head pointing south,
looking towards the west. A mat was placed on the ground below the
deceased, and sometimes the head rested on a pillow of straw or
leather. Another mat or the skin of an animal, usually goat or gazelle,
covered or enclosed the deceased and most of the time covered the
offerings as well. The surviving remains of clothing suggest that the
usual apparel worn by the dead was a sort of fabric loincloth or a hide
loincloth trimmed with fabric. Although simple burials of single indi-
viduals were in the majority, multiple burials were also fairly frequent,
most notably involving a woman (possibly the mother) and a newborn
infant. Compared with the previous period, larger burial places
appeared, provided with coffins of wood or earth, and more lavishly
equipped. Although plundered, the Amratian tombs of Hierakonpolis
are remarkable for their rectangular form and unusual size (the largest
being 2.50 m. x 1.80 m.). In two instances, the inclusion of magnifi-
cent disc-shaped porphyry maceheads probably indicates the burials of
powerful individuals. The Amratian culture particularly differs from
the Badarian in terms of the diversity of types of grave goods and con-
sequent signs of hierarchy, and Hierakonpolis was clearly already an
important site from the point of view of such diversification.
The differences between the Badarian and Amratian cultures can be
seen above all in changes in material culture. The black-topped red
ware gradually became less common, and this trend would eventually
lead to its total disappearance at the end of the Predynastic. The
46 BEATRIX MIDANT-REYNES
rippling effect on the surface of the pottery became rarer, as did black-
polished pottery. At the same time, however, red-polished pottery con-
tinued to flourish in a variety of forms, often incorporating different
styles of surface decoration. The best-decorated examples feature
sculpture in the round and white painted designs comprising geo-
metrical, animal, and vegetal motifs. These constitute the beginnings
of an iconography that would eventually lie at the core of pharaonic
civilization.
The fauna represented on the vessels are essentially riverine, such
as hippopotami, crocodiles, lizards, and flamingos, but there were
also scorpions, gazelles, giraffes, ichneumons, and bovids. The bovids
are rendered schematically, thus making their precise identification
difficult. Sometimes a boat might also be depicted, prefiguring the
leitmotif of the Naqada II phase. Human figures, although at this date
unobtrusive, were nevertheless present in the Amratian version of the
universe. Such figures, however, were represented schematically, each
with a small round head on a triangular torso terminating in thin hips
and standing on stick legs, often without feet. The arms were rep-
resented only when the figure was engaged in some activity.
The depictions involving human figures can be divided into two
types: the first—and most frequent—is the hunt, and the second is the
victorious warrior. A good example of the hunt is shown on a Naqada I
vessel in the Pushkin Museum of Fine Arts, Moscow. This scene com-
prises a person holding a bow in his left hand, while in his right he
controls four greyhounds on a leash. This is the very image of the
hunter, with the king wearing the tail of an animal at his belt, that can
still be seen several centuries later on the so-called Hunter's Palette
or on the Gebel el-Arak knife handle (the former now in the British
Museum and the latter in the Louvre), and indeed continued to be a
powerful image until the end of the pharaonic period.
The theme of the victorious warrior occurs on the elongated body
of a Naqada I vessel in the collection of the Petrie Museum, University
College London. The depiction comprises two human figures among
plant motifs; the larger figure, with stalks or plumes fastened in his
hair, lifts his arms above his head, while his virility is unequivocally
marked by a penis or penis sheath. Interlaced ribbons descending
from between his legs may represent decorated cloth. A white line
emerges from the larger figure's chest and wraps around the neck
of the second figure, a much smaller person with long hair. A swelling
on the back of the smaller figure could represent bound arms. Despite
a clear pelvic protuberance, the sexuality of the smaller figure remains
THE NAQADA P E R I O D 47
ambiguous; if it were feminine, this would justify the small size. A
similar scene decorates an identical vessel in the Brussels Museum, as
well as one of the same material excavated in the 19905 by German
archaeologists at Abydos. The prevalence of the bound figure, and
the absence or obstruction of the arms of small persons, strongly
suggest the imagery of the conqueror and the vanquished. This early
theme of domination appears to be the prototype of traditional scenes
of victory in the pharaonic phase. It is interesting to note that, as
early as the Naqada I phase, the dual theme of hunting and war—
always understood to be victorious—is established, implying the
existence of a group of hunter-warriors already invested with an aura of
power.
The graves and the funerary offerings indicate not so much increas-
ing hierarchization as a tendency towards social diversity in the Naqada
I culture. The offerings in this period appeared initially to be intended
simply to mark the identity of the deceased. It is not until the Naqada II
phase (and even more so Naqada III) that larger accumulations of
funerary artefacts are clearly in evidence.
The funerary statuettes are particularly significant. Both men and
women are represented standing, more rarely seated, with emphasis
on the primary sexual characteristics. Only a few of the thousands of
excavated tombs contained such statuettes, and usually they occurred
only singly, groups of two or three in one tomb being comparatively
rare. The maximum number found in a single burial was a set of six-
teen figurines. Based on an analysis of the other offerings, the tombs
that contained multiple statuettes were not particularly rich in other
respects, and such small sculpted figures were sometimes the sole
funerary offering. Could these be the tombs of sculptors? Whatever
their significance, the presence of these objects indicates greater exclu-
sivity than wealth as determined by sheer quantity of grave goods. The
use of copper and flint knives as funerary offerings raises the same
kind of question during the Naqada II phase.
The more or less schematically rendered heads of bearded men
seem to constitute another new category of human representation in
Naqada I, which was to be further developed in Naqada II. Found on
small throwsticks of carved ivory or on the tips of hippopotamus or
elephant tusks, the one repeated feature of these representations is the
presence of a triangular beard, often balanced by a sort of 'phrygian'
cap pierced by a suspension hole. Unlike women, men were no longer
being solely identified by their primary sexual characteristics, but by a
secondary sexual characteristic and the social status that this conferred
48 BEATRIX MIDANT-REYNES
on them. The beard was evidently a symbol of power, and, in the form
of the ceremonial 'false beard', it later became strictly reserved for the
chins of kings and gods.
Another symbol of power that characterizes the Naqada I phase is
the disc-shaped macehead, usually carved from a hard stone, but
sometimes also occurring in softer materials such as limestone, terra-
cotta, or even unfired pottery, and occasionally provided with a haft. It
was during this period that techniques of working both hard and soft
stones (including greywacke, granite, porphyry, diorite, breccia, lime-
stone, and Egyptian alabaster) began to be developed, and this crafts-
manship would eventually ensure that the Egyptian culture became
the 'civilization of stone' par excellence. Greywacke cosmetic palettes
constituted the item of choice for funerary equipment during the
Amratian. These palettes exploded into a diversity of forms, from a
simple oval shape, sometimes incised with figures of animals, to
complete zoomorphs, including fish, tortoises, hippopotami, gazelles,
elephants, and birds (although the range of beasts depicted on the
painted vessels was nevertheless much greater).
The production of bone and ivory objects, including punches,
needles, awls, combs, and spoons, extended—and improved upon—
the repertoire of the preceding Badarian culture. Not many worked
stone tools have been found in Naqada I graves, but the rarity of such
finds was equalled by their quality. These delicate and long bifacially
flaked blades, some as much as 40 cm. long, were regularly serrated.
Their most unusual feature was that they had all been polished before
retouching. This process was also used on beautiful daggers with
bifurcated blades, which look ahead to the Old Kingdom forked instru-
ments known as pesesh-kefused in the Opening of the Mouth funerary
ceremony.
Glazed steatite, already known in the Badarian period, continued in
use. The first attempts at crafting Egyptian faience appear to date from
the Naqada I phase, whereby a nucleus of crushed quartz was shaped
into the desired form and coated with a natron-based glaze coloured by
metallic oxides.
Metal work shows few differences from the Badarian period, apart
from an extension of the repertoire, including such artefacts as pins,
harpoons, beads, looped pins for attachment and bracelets, often
executed in hammer-worked native copper. The tips of bifurcated
spears from a tomb in el-Mahasna, which imitate worked-stone speci-
mens, evoke comparison with the techniques of metal production
employed by their northern neighbours at Maadi (see below).
THE NAQADA P E R I O D 49
The picture derived from the analysis of the tombs and their con-
tents is of a structured and diversified society, with a tendency towards
hierarchical organization, in which the major traits of pharaonic civili-
zation can already be seen in embryonic form.
Compared with the significant remains of the world of the dead, the
surviving traces of Naqada I settlement are poor, not only because too
few sites of this type have been preserved but also because of the
nature of Predynastic land-use practices. Since the buildings making
up the settlements were essentially constructed from a mixture of mud
and organic materials (such as wood, reed, and palm), they have not
survived well, and the work invested by the archaeologist would have to
be considerable to yield even a minimum of data. Among the vestiges
of subdivided huts made from beaten earth (which are not even defi-
nitely known to be dwellings) are hearths and post-holes. The zones of
habitation are indicated by deposits of organic material dozens of
centimetres thick. The sole surviving built structure has been exca-
vated at Hierakonpolis, where the American team uncovered a burnt
man-made structure consisting of an oven and a rectangular house
partially enclosed by a wall, measuring 4.00 x 3.50 m. Although it is
possible that such houses may have been present at all Nile Valley
settlements of this date, it should be borne in mind that Hierakonpolis
may well have been unusual—it had been an important site from an
early date, and from this time onwards it was the centre of an elite
group, judging from its large-scale graves.
One of the results of the lack of excavated settlements is an impre-
cise knowledge of the Naqada I economy. The domesticated animal
species represented among the grave goods include goats, sheep,
bovids, and pigs, which have survived either in the form of food offer-
ings or as small statuettes modelled in clay. As far as wild fauna were
concerned, gazelles and fish appear to have been plentiful. Barley and
wheat were cultivated, as were peas and tares, the fruits from the
jujube, and a possible ancestor of the watermelon.
Naqada II (Gerzean)
During the second phase of the Naqada culture, fundamental changes
took place. These developments, however, took place not at the mar-
gins of the culture but in its Amratian heartland; in essence, they
can be regarded as an evolution rather than a sudden break. The
Naqada II phase was characterized primarily by expansion, as the
50 BEATRIX MIDANT-REYNES
Gerzean culture extended from its source at Naqada northwards
towards the Delta (Minshat Abu Omar) and southwards as far as
Nubia.
There was a distinct acceleration of the funerary trend first seen in
the Amratian, whereby a few individuals were buried in larger, more
elaborate tombs containing richer and more abundant offerings.
Cemetery T at Naqada and Tomb 100 (the so-called Painted Tomb) at
Hierakonpolis are good examples of this overall trend.
Gerzean cemeteries comprise a wide range of grave types, ranging
from small oval or round pits, poorly provided with offerings, to burials
in pottery vessels and the construction of rectangular pits subdivided
by mud-brick partitions, with specific compartments for offerings.
There were coffins of wood and air-dried pottery, as well as the first
indications of the wrapping of the body in strips of linen. Early
'mummification' of this type is attested in a double tomb at Adaima, an
Upper Egyptian site near Hierakonpolis, excavated since 1990 by the
French Archaeological Institute at Cairo. The Naqada II burials gen-
erally remained simple, but multiple burials, containing up to five
individuals, became more common. Funerary rituals appear to have
become more complex, sometimes involving dismemberment of the
body, a practice that was not attested in the preceding period. In Tomb
T5 at Naqada, a series of long bones and five crania were arranged
along the tomb walls, and at Adaima there are some examples of skulls
detached from their torsos. The possibility of human sacrifice was
noted by Petrie at Naqada, and two cases of throat slitting followed by
decapitation have been identified at Adaima. Although sparse and
scattered, this possible evidence for self-sacrifice could be an early
prelude to the mass human sacrifices around the Early Dynastic royal
tombs at Abydos, which represented a turning point in the emergence
of the Egyptian kingship of the Dynastic Period.
Two new types of pottery make their appearance: first, 'rough'
pottery, which has been found in tombs dating to this period but was
later found in domestic contexts, and, secondly, 'marl ware', which
was fashioned partly in a calcareous clay derived from the desert wadis
rather than the Nile Valley. The marl pottery, sometimes decorated
with ochre-brown paintings on a cream ground, replaces the white-
painted red ware of the Naqada I phase. There are two types of motifs:
geometrical (consisting of triangles, chevrons, spirals, check patterns,
and wavy lines) and representational. The repertoire is limited to about
ten elements, combined according to a system of symbolic representa-
tion that is still not properly understood.
THE NAQADA P E R I O D 51
The predominant motif in the representational art of the period
is the boat; its omnipresence reflects the importance of the river, not
only as the provider of fish and wild fowl but also as the principal
channel of communication that was to be indispensable to the
northward and southward expansion of the Naqada culture. It was by
boat that raw materials were obtained, such as ivory, gold, ebony,
incense, and skins of wild cats from the south, and copper, oils, stone,
and seashells from the north and east, mostly destined for an elite
whose social position was becoming increasingly distinct from the rest
of the population. In these depictions, the boat represents both a mode
of travel and a status symbol. It is clear, however, that from this date
onwards the Nile, flowing from the north to the south, had also been
transformed into a mythical river on which the first gods sailed. The
links between the human and cosmic orders were already being estab-
lished.
During the Naqada II phase, there was considerable development in
techniques of stone working: various limestones, alabasters, marbles,
serpentine, basalt, breccia, gneiss, diorite, and gabbro were being
discovered and exploited all along the Nile Valley as well as in the
desert, particularly at Wadi Hammamat. The increasing skills in the
carving of stone vessels prepared the way for the great achievements of
pharaonic stone architecture. The ripple-flaked knives of this period
are among the most accomplished examples of the working of flint
anywhere in the world.
Cosmetic palettes became fewer in number, evolving towards simple
rectangular and rhomboidal shapes, but at the same time they began to
be decorated with reliefs, starting a line of development towards the
narrative-style decorated palettes of the Naqada III phase. The disc-
shaped macehead of the Amratian period was replaced by the pear-
shaped type, two examples of which had already appeared at an earlier
date in the Neolithic settlement of Merimda Beni Salama. By the
Naqada II phase the macehead had become mysteriously charged as
a symbol of power, and in the pharaonic period it was the weapon
characteristically held by the victorious king.
Copper working intensified, no longer being limited to small objects
but gradually beginning to produce artefacts that were substitutes for
stone objects, such as axes, blades, bracelets, and rings. Alongside
developments in copper production there was also a growth in the use
of gold and silver, and the evidence at sites such as Adaima suggests
that the increased attraction of metal might well account for much of
the tomb robbery during the Predynastic Period.
52 BEATRIX M I D A N T - R E Y N E S
The picture of Naqada II society that is thus revealed is a blueprint
for the development of a class of artisans who were specialized in the
service of the elite. This fact has two implications: first, there had to
be an economy that was capable of supporting groups of non-self-
sufficient craftsmen, at least during a part of the year, and, secondly,
there must have been urban centres that brought together the clients,
the workshops, the apprentice craftsmen, and the facilities for com-
mercial exchange.
This process of cultural development was always tied closely to the
Nile. As Michael Hoffman showed in his interpretation of the Pre-
dynastic remains at Hierakonpolis, settlement clustered near the river,
which supported the cultivated land, where simple artificial irrigation
techniques could take advantage of the annual flood. The entire Nile
Valley was covered by a string of villages, which are often known
simply because of their surviving cemeteries. We have evidence for
different species of barley and wheat, flax, various fruits (such as water-
melon and dates) and vegetables. As in the preceding period, cattle,
goats, sheep, and pigs comprise the domestic livestock. Among
the domesticates, the dog enjoyed special status, judging from its
burials within the settlement of Adaima. Fish also played an important
role in diet, but the hunting of large riverine and desert mammals
(such as hippopotami, gazelle, and lions) gradually became more
socially restricted until it became the prerogative of the dominant elite
groups.
Three large centres arose in Upper Egypt: Naqada, the 'gold town' at
the mouth of Wadi Hammamat; Hierakonpolis, further to the south;
and Abydos, where the necropolis of the first pharaohs was to be
located. Two large residential zones were uncovered at Naqada by
Petrie and Quibell in 1895: the 'South Town' (in the central part of the
site) and the 'North Town'. The South Town incorporates a large
rectangular mud-brick structure measuring 50 x 30 m., which may
possibly be the remains of a temple or a royal residence. To the south
of this large structure, a group of rectangular houses and an enclosure
wall can be made out. These two elements, the rectangular house and
the enclosure wall, are typical of the emerging towns of Naqada II.
There may be a shortage of primary archaeological evidence of settle-
ments at this date, yet two artefacts from funerary contexts help to
compensate for this defficiency. The first is a terracotta model house
from a Gerzean grave at el-Amra. An Amratian tomb from Abadiya
has provided a second model (Oxford, Ashmolean) representing a
crenellated wall, behind which two people are standing; the Amratian
THE NAQADA P E R I O D 53
date of the second model suggests the early date at which dwellings of
this type began to be used.
Northern Cultures (including the Maadian Complex)
The Maadian cultural complex of about a dozen sites has only recently
been brought to light. These sites include the excavated cemetery and
settlement complex at Maadi itself, a suburb of modern Cairo. The
Maadian culture appeared during the second part of Naqada I and
continued until Naqada Ilc/d, when it was eclipsed by the spread of the
Naqada II culture, exemplified by the cemeteries of el-Gerza, Haraga,
Abusir el-Melek, and Minshat Abu Omar.
The earliest Neolithic sites have been discovered in this part of the
Nile Valley, in the Faiyum region and at Merimda Beni Salama and el-
Omari (see Chapter 2), and it is these sites that represent the tradition
from which the Maadian material culture emerged. Maadian culture
differs in all its characteristics from sites of similar date in Upper
Egypt. In a reversal of the situation at the sites of the Naqada culture,
cemeteries were much less prominent in the archaeological record,
and most of our knowledge derives instead from settlements.
At Maadi, the Predynastic remains cover nearly 18 ha., including the
cemetery. In the first half of the twentieth century an area of about
40,000 sq.m. was excavated. The depth of archaeological deposits is
almost 2 m., including heaps of refuse preserved in situ, the stratig-
raphy of which is complex. The excavated structures show that there
were three types of settlement remains, one of which is unique in an
Egyptian context, strongly reminiscent of the settlements at Beersheba
in southern Palestine. It involves houses excavated from the living
rock in the form of large ovals measuring 3 x 5 m. in area and up to 3 m.
in depth, each of which was entered via an excavated passageway; the
walls of one of these subterranean houses were faced with stone and
dried Nile-silt mud bricks, but this is the only known instance of the
use of mud brick at Maadi. The presence of hearths, half-buried jars,
and domestic debris suggests that these were genuine permanent
habitations. The other types of domestic structures at Maadi are already
well attested elsewhere in Egypt: first, a form of oval hut accompanied
by external hearths and half-buried storage jars, and, secondly, a
rectangular style of house in which narrow foundation trenches are all
that remain of walls that were presumably made from plant material.
In general, Maadian pottery is globular with a broad, flat base, a
more or less narrow neck, and flared rims, partially fashioned from
54 BEATRIX MIDANT-REYNES
alluvial clay. They are rarely decorated, except sometimes with incised
marks applied after firing. It is interesting to note that the oldest strata
at the late Predynastic sites of Buto (Tell el-Fara e in), Tell el-Iswid, and
Tell Ibrahim A wad include sherds decorated with impressions that
are reminiscent of Saharo-Sudanese pottery. Links with Upper Egypt,
dating back to the period before the Maadian culture, are indicated by
the presence of sherds from imported vessels of black-topped red
ware, which mingle with their pale imitations made locally at Maadi.
Conversely, the commercial links with Early Bronze Age Palestine
account for the presence of distinctive footed ceramics, with neck,
mouth, and handles decorated en mamelons, made from a calcareous
clay fabric, which contained imported products (oils, wines, resins).
Maadian culture was thus a kind of cultural crossroads, subject to the
influences of the Western Desert (perhaps an extremely early asso-
ciation), the Near East, and the emerging kinglets of Naqada to the
south.
Palestinian influence is also clearly discernible in the worked flint of
the Maadian culture. In contrast to the local flint industry essentially
employing pressure-flake technology, the Maadian assemblages also
include large circular scrapers knapped from large nodules with
smooth surfaces, which are well known throughout the Near East.
Beautiful edged blades with rectilinear ribbing, known as 'Canaanite
blades', also appear at Maadian sites; these were to develop into the
pharaonic-period 'razors' (actually double scrapers) that were ele-
ments of royal funerary equipment until the end of the Old Kingdom,
sometimes polished and sometimes reproduced in copper and even
gold. The bifacial pieces, few in number, include projectile points, dag-
gers, and sickle blades. The latter were products of the local tradition
(Faiyum bifacial sickles) and were gradually replaced by a Near Eastern
style of sickle mounted on a blade.
The comparative rareness of greywacke cosmetic palettes imported
from Upper Egypt is presumably an indication of their limited avail-
ability and therefore the luxury nature of the object. The more numer-
ous limestone palettes, on the other hand, show signs of wear that
indicate their regular daily use. Hard stone maceheads are of the disc-
shaped type characteristic of the Amratian and early Gerzean cultures.
Apart from several combs imported from Upper Egypt, objects in
polished bone and ivory include the traditional repertoire of needles,
harpoons, punches, and awls. Catfish darts, consisting of the first spine
of the pectoral and dorsal fins, are found in great number, particularly
in jars that were probably stockpiled for export.
THE NAQADA P E R I O D 55
There are many indications of Maadi's involvement in intercultural
contacts and commerce. In this regard, the role of copper is particu-
larly significant. Metallic objects seem to have been particularly com-
mon at Maadi. Not only are there simple pieces such as needles or
harpoons, but also rods, spatulas, and axes. These forms of artefacts
were made from stone in the Faiyum and Merimda cultures, but at
Maadi they were made from metal. This situation is paralleled in
Palestine during the same period, where polished stone axes totally
disappear and are replaced with metal versions, albeit using different
techniques from those at Maadi. This substitution of metal for stone
cannot be mere coincidence, but the result of a process of techno-
logical progress that is an indication (and a direct result) of the genuine
symbiosis between the two regions. Large quantities of copper ore
have also been found at Maadi, which under analysis reveal a probable
provenance in the region of Timna or Fenan, both of which are copper-
mining sites in Wadi Arabah, at the south-eastern corner of the Sinai
peninsula. However, rather than the ore being processed at Maadi
itself, it was perhaps imported primarily for processing into cosmetics,
and the initial processing must have been undertaken near the mines
themselves.
Despite the involvement of the Maadian people in a network of
contacts with the Near East, their culture was above all pastoral-
agricultural and sedentary. There are few traces of wild fauna to
counterbalance the enormous quantity of domesticated animals (pigs,
oxen, goats, sheep) that, apart from the dog, comprised the basic meat
diet of the community. The donkey doubtless served as transport for
the merchandise. Kilos of grain found in jars and in storage pits
include wheat and barley (Triticum monococcum, Triticum dicoccum,
Triticum aestivum, Triticum spelta, Hordeum vulgare) as well as pulses
such as lentils and peas.
Compared with the good evidence of agricultural activity at Maadi,
the interment of the deceased was relatively unobtrusive, indicating a
community that had perhaps undergone little social change since the
Neolithic and was evidently lacking in stratification or hierarchy. A
total of 600 Maadian tombs has been recovered, as opposed to more
than 15,000 Predynastic graves in the south. Geographical and geo-
logical factors contribute to this imbalance: the northern cemeteries,
located in areas prone to heavy flooding, might well have been buried
in thick layers of Nile silt. This, however, does not explain everything,
because there is also a contrast in the quality and quantity of funerary
equipment in the north, compared with the Upper Egyptian situation.
56 BEATRIX MIDANT-REYNES
The Lower Egyptian graves are characterized by extreme simplicity,
comprising basic oval pits with the deceased placed in a foetal position,
shrouded in a mat or in fabric, and accompanied by only one or two
pottery vessels or sometimes even nothing at all.
However, as we review the development of the Northern Cultures
(consisting of three phases roughly corresponding to the cemeteries
of Maadi, Wadi Digla, and Heliopolis), certain tombs appear better
equipped than others, without ever displaying conspicuous luxury like
that found in Upper Egypt. Nevertheless, a gradual tendency towards
social stratification can be discerned, and it is possible that the mixing
of the graves of dogs and gazelles with those of humans is part of this
process of social change. The final phase of the Maadian culture,
represented by the earliest stratigraphic layers at Buto, is equivalent to
the middle of the Naqada II phase (levels Ilc-d).
At the exceptional site of Buto, there are seven successive archaeo-
logical strata in which the transition between the Maadian phases and
the overlapping protodynastic can be observed. During this transition,
there is a perceptible increase in Naqada pottery styles, while the
Maadian pottery progressively disappears. Thus the end of the
Maadian culture was not an abrupt phenomenon, as the site of Maadi
would suggest, but was instead a process of cultural assimilation. It is
probable that, with its fluvial and maritime location, Buto was well
placed for important trade, and perhaps also incorporated a palace for
local rulers. While the archaeological data from Buto are less startling
than the remains at Naqada, there was a comparable process of cul-
tural development here which led, in the same way, to increased social
complexity, eventually producing a society characterized by its own
beliefs, rites, myths, and ideologies. This was the necessary precon-
dition for the next great step forward in the history of Egypt, which
took place in the Naqada III and Early Dynastic periods.
4
The Emergence of the Egyptian State
(0.3200-2686 BC)
KATHRYN A. BARD
The Naqada III phase, £.3200-3000 BC, is the last phase of the Pre-
dynastic Period, according to Kaiser's revision of Petrie's sequence
dates. It was during this period that Egypt was first unified into a large
territorial state, and the political consolidation that laid the founda-
tions for the Early Dynastic state of the ist and 2nd Dynasties must also
have occurred then. In the latter part of this phase there is evidence of
kings preceding those of the ist Dynasty, in what is now called
'Dynasty o'. They were buried at Abydos near the royal cemetery of the
ist Dynasty. On the Palermo Stone, a late 5th-Dynasty king-list (see
Chapter i), the presence of names and figures of seated kings in com-
partments in the broken top part of the list suggests that the Egyptians
believed that there had been rulers preceding those of the ist Dynasty.
There is considerable debate, however, regarding such factors as the
precise nature of the process of unification, the date when it took place,
and the question of the origins of Dynasty o.
State Formation and Unification
From the Naqada II phase onwards, highly differentiated burials are
found in cemeteries in Upper Egypt (but not in Lower Egypt). Elite
burials in these cemeteries contained large quantities of grave goods,
sometimes made from exotic materials such as gold and lapis lazuli.
These burials are symbolic of an increasingly hierarchical society,
probably representing the earliest processes of competition and
the aggrandizement of local polities in Upper Egypt, as economic
58 KATHRYN A. BARD
interaction and long-distance trade developed. Control of the distribu-
tion of exotic raw materials and the production of prestigious craft
goods would have reinforced the power of chiefs in Predynastic
centres, and such goods were important status symbols. Despite a lack
of archaeological evidence, it seems likely that the larger Predynastic
towns in Upper Egypt were becoming centres of craft production.
Some of these centres also became walled settlements, like the South
Town at Naqada, documented by Petrie.
The core area of the Naqada culture was in Upper Egypt, but, in the
Naqada II phase, sites of the Naqada culture began to be established
in northern Egypt for the first time. Petrie excavated a Naqada II
cemetery at el-Gerza in the Faiyum region, from which he derived the
term Gerzean (Naqada II) for his middle Predynastic phase. Some-
what later, Naqada culture burials are found much further north, at the
Delta site of Minshat Abu Omar. Such evidence suggests the gradual
northward movement of peoples from Upper Egypt in Naqada II
times.
Since the major Upper Egyptian sites were located near the Eastern
Desert, from which gold and various kinds of stone used for beads,
carved vessels, and craft goods were obtained, they were much richer
in natural resources than Lower Egyptian sites: the ancient name of
Naqada is Nubt ('[city] of gold') and it is no coincidence that the largest
Predynastic cemetery is located there. As cereal agriculture was prac-
tised with increasing success on the floodplain of Upper Egypt, sur-
pluses accrued and could be exchanged for craft goods, the production
of which was becoming increasingly specialized. Possibly the first
southerners to go north were traders, and as the economic interaction
increased they may have been followed by colonists. There is no
archaeological evidence to demonstrate the northward movement of
people (as opposed to artefacts), but if such migration occurred it
seems more likely to have been a peaceful expansion rather than a
military invasion, at least in the early stages.
A motivating factor for the expansion of the Naqada culture into
northern Egypt might have been the desire to gain direct control over
the lucrative trade with other regions in the eastern Mediterranean,
which had developed earlier in the fourth millennium BC. But the
development of the technology to construct large boats was also the
key to control and communication on the Nile and large-scale
exchange. Timber (cedar) for the construction of such boats did not
grow in Egypt, but came from the area of the Levant now occupied by
Lebanon.
THE E M E R G E N C E OF THE EGYPTIAN STATE 59
As the discussion of the Maadian culture in Chapter 3 indicates,
Lower Egypt was not a cultural vacuum in the fourth millennium BC,
and eventually Naqada expansion would probably have met with some
resistance. The archaeological evidence in the north, however, demon-
strates only the eventual replacement of the Maadian culture. At Maadi
itself, occupation came to an end in the Naqada Ilc/d phase, while
stratigraphic evidence at sites in the northern Delta, such as Buto, Tell
Ibrahim Awad, Tell el-Rub c a, and Tell el-Farkha, demonstrates that
there were earlier strata containing only Maadian and local wares, but
above these were strata comprising only ceramics of the Naqada III
culture and the later forms of the ist Dynasty. At Tell el-Farkha a trans-
itional layer of aeolian sand between such strata suggests the abandon-
ment of the settlement by the local population for unknown reasons
(intimidation?) and the later reoccupation of the site in Dynasty o by
people of the Naqada culture, which by then had spread throughout
Egypt.
By the end of the Naqada II phase (^.3200 BC) or early Naqada III,
the indigenous material culture of Lower Egypt had disappeared and
was replaced by artefacts (especially pottery wares) deriving from
Upper Egypt and the Naqada culture. This archaeological evidence has
sometimes been interpreted as an indication of the political unifica-
tion of Egypt by this time, but the material evidence does not neces-
sarily imply (unified) political organization and a number of alternative
socio-economic factors might be proposed to explain this change.
Given that the evidence from the elite burials in three major Pre-
dynastic centres in Upper Egypt (Naqada, Abydos, Hierakonpolis) sug-
gests separate (and possibly competing) centres or polities during the
Naqada II phase, the first unification of Upper Egyptian polities prob-
ably took place in early Naqada III times, either as a result of a series of
alliances or through warfare (or perhaps through a combination of
both), followed by the political unification of the north and south and
the emergence of Dynasty o towards the end of Naqada III.
Naqada III burials in the largest Predynastic cemetery at Naqada
and the elite Cemetery T are impoverished compared to the earlier
Naqada II burials at this site. More than 6 km. south of these cemet-
eries, two large niched mud-brick tombs and a cemetery with Early
Dynastic graves were excavated at the end of the nineteenth century by
Jacques de Morgan. The location of this cemetery and the sudden
appearance of a new style of 'royal' burial at the end of Naqada III,
together with the more impoverished (earlier) burials in the cemet-
eries far to the north, all suggest a break with the polity centred at
60 KATHRYN A. BARD
South Town (located only 150 m. north-east of the large Predynastic
cemetery), probably coinciding with the absorption of the Naqada
polity into a larger one.
In contrast, in the Umm el-Qa e ab region of Abydos the graves in one
area (Cemeteries U and B and the 'royal cemetery') evolved from fairly
undifferentiated burials in early Naqada times, to an elite cemetery in
late Naqada II, and finally to the burial place of the kings
of Dynasty o and the ist Dynasty. One Naqada III tomb, U-j, dating
to £.3150 BC, consisted of twelve rooms covering an overall area of
66.4 sq. m. Although robbed, it contained many artefacts in bone and
ivory, a great deal of Egyptian pottery, and about 400 imported jars
from Palestine that may possibly have contained wine. The 150 small
labels found in this tomb are inscribed with what appear to be the
earliest known hieroglyphs. According to the excavator, Giinter Dreyer,
traces of a wooden shrine in the burial chamber and an ivory model
sceptre demonstrate that this was the tomb of a ruler, possibly King
Scorpion, whose estates may be listed on a number of labels. This
ruler, however, probably reigned in the thirty-first century BC.
Excavations at 'Locality 6' in Hierakonpolis, 2.5 km. up the Great
Wadi, revealed several large tombs, each measuring up to 22.75 sc l- m -
in floor area and containing Naqada III ware. Tomb n, although
looted, still contained beads in carnelian, garnet, turquoise, faience,
gold and silver, fragments of artefacts in lapis lazuli and ivory, obsid-
ian, and crystal blades, and a wooden bed with carved bulls' feet. Such
a rich burial suggests that elite individuals of considerable means were
being buried at Hierakonpolis, but that they were still not of the same
class as the rulers at Abydos.
Whereas Naqada was politically insignificant in the Early Dynastic
Period, Abydos was the most important centre for the cult of the dead
king, and Hierakonpolis remained an important cult centre associated
with the god Horus, symbolic of the living king. In a late Predynastic
power struggle in Upper Egypt, it is possible that the Naqada polity was
vanquished, whereas rulers whose power base was originally at Abydos
went on to control the entire country, perhaps in alliance with less
powerful elite groups (the so-called Followers of Horus) at Hierakon-
polis, who were none the less in a strategic position because of valued
raw materials coming from the south.
The final unification of Upper and Lower Egypt may have been
achieved through one or more military conquests in the north, but
there is not much evidence for this apart from scenes with sym-
bolically military content carved on a number of ceremonial palettes
THE E M E R G E N C E OF THE EGYPTIAN STATE 6l
dated stylistically to the late Predynastic (Naqada Ill/Dynasty o), such
as the fragmented Tjehenu (Libyan), Battlefield, and Bull palettes. The
interpretation of such scenes is problematic, because these artefacts
are without known provenances and the fragmented scenes are sym-
bolic of conflict without specifying real historical events.
Fortunately, three important artefacts with carved scenes relevant to
this period were excavated at Hierakonpolis: the Macehead of King
Scorpion, and the Palette and Macehead of King Narmer. All three of
these ceremonial objects were found in or near the area described as
the 'main deposit' by }. E. Quibell and F. W. Green when they exca-
vated the temple of Horus at Hierakonpolis. They were possibly royal
donations to the temple and suggest that Hierakonpolis was still an
important centre at the end of the Naqada III phase. While the unifi-
cation of Upper and Lower Egypt is too specific an interpretation for
the scenes on the Narmer Palette, the scenes illustrate dead enemies
and vanquished peoples and/or settlements. Scenes and signs on the
Narmer Macehead represent war captives and booty, and conquered
peoples are also represented on the Scorpion Macehead. Such scenes
suggest that warfare played a role at some point in the forging of the
early state in Egypt. Even if there is no evidence of destruction layers of
Naqada III date at settlement sites in the Delta, warfare could still have
implemented the consolidation of this early state and its expansion
into Lower Nubia and southern Palestine, which occurred in the early
ist Dynasty.
From Petrie onwards, it was regularly suggested that, despite the
evidence of Predynastic cultures, Egyptian civilization of the ist Dyn-
asty appeared suddenly and must therefore have been introduced by
an invading foreign 'race'. Since the 19705, however, excavations at
Abydos and Hierakonpolis have clearly demonstrated the indigenous,
Upper Egyptian roots of early civilization in Egypt. While there is
certainly evidence of foreign contact in the fourth millennium BC, this
was not in the form of a military invasion.
Ceramics from excavated strata at sites in northern Egypt and
southern Palestine now make it possible to coordinate specific cultural
periods in the two regions, and demonstrate continuing contact as the
Maadi culture in the north was replaced by the Naqada culture. While
the Naqada lib phase corresponds to the Early Bronze Age (EBA) la
phase in Palestine, Naqada II c-d and Naqada Ill/Dynasty o were evi-
dently contemporaneous with the EBA Ib culture. Contact between
northern Egypt and Palestine at this time was overland, as evidence in
the northern Sinai demonstrates. Between Qantar and Raphia, about
62 KATHRYN A. BARD
250 early settlements have been located by the North Sinai Expedition
of Ben Gurion University, with 80 per cent of the ceramics of Egyptian
wares dating to Naqada II-III and Dynasty o. The settlement pattern
consists of a few larger core sites interspersed with seasonal encamp-
ments and way stations.
Israeli archaeologists suggest that this evidence represents a com-
mercial network established and controlled by the Egyptians as early as
ERA la, and that this network was a major factor in the rise of the
urban settlements found later in Palestine in EBA II. Naomi Porat's
technological study of ceramics from EBA sites in southern Palestine
clearly demonstrates that in EBA Ib strata many of the pottery vessels
used for food preparation were probably manufactured by Egyptian
potters using Egyptian technology but local Palestinian clays. In EBA
Ib strata there are also many storage jars made from Nile silt and marl
wares, which must have been imported from Egypt. Not only did the
Egyptians establish camps and way stations in the northern Sinai, but
the ceramic evidence also suggests that they established a highly
organized network of settlements in southern Palestine where an
Egyptian population was in residence.
The importance of the Delta for Egyptian contact with south-west
Asia is also suggested by enigmatic evidence from Buto. In strata of the
Lower Egyptian Predynastic culture at this site, two unexpected types
of ceramic were found by Thomas von der Way in the late 19805: clay
'nails' and a so-called Grubenkopfnagel (a tapering cone with a concave
burnished end) that resemble artefacts used in the Mesopotamian
Uruk culture to decorate temple facades. Von der Way suggests that
contact with the Uruk culture network may have taken place via north-
ern Syria, as the earliest Predynastic stratum at Buto was found to
contain sherds decorated with whitish stripes characteristic of the
Syrian e Amuq F ware. The clay nails and the Grubenkopjhagel are not
associated with any (mud-brick) architecture in the Predynastic levels,
which might be expected if von der Way's interpretation were correct,
but the ongoing excavations at Buto may yet provide more data on con-
nections between the Delta and south-west Asia in the fourth millen-
nium BC.
Both imported and Egyptian-made cylinder seals, an artefact type
unquestionably invented in Mesopotamia, are found in a few elite
graves of the Naqada II and III phases. Beads and small artefacts in
lapis lazuli, which could only have come from Afghanistan, are first
found in Upper Egyptian Predynastic graves. Mesopotamian motifs
also appear in Upper Egypt (and Lower Nubia), including the motif of
THE E M E R G E N C E OF THE EGYPTIAN STATE 63
the heros dompteur (a victorious human figure between two lions/
beasts), painted on the wall of Tomb 100 at Hierakonpolis, which dates
to Naqada II. Other typically Mesopotamian motifs, such as the niched
palace facade and high-pro wed boats, are also found on Naqada II and
III artefacts and also in the rock art. The styles of these motifs are more
characteristic of the glyptic art of Susa in south-west Iran than of the
Uruk culture, and the fact that such artefacts are not found in Lower
Egypt has raised the possibility of some southern route of contact
between Susa and Upper Egypt, the nature of which is unknown at
present.
In Lower Nubia there are numerous burials of the A-Group culture
(which was roughly contemporaneous with the Naqada culture), and
these contain many Naqada craft goods. The A-Group wares are very
distinct from the Naqada ones, and Egyptian products were probably
obtained through trade and exchange. It has been suggested by Bruce
Williams that the elite A-Group Cemetery L at Qustul in Lower Nubia
represents Nubian rulers who conquered and unified Egypt, founding
the early pharaonic state, but most scholars do not agree with this
hypothesis. The model that may best explain the archaeological evi-
dence is one of accelerated contact between the cultures of Upper
Egypt and Lower Nubia in later Predynastic times. Luxury raw
materials, such as ivory, ebony, incense, and exotic animal skins, all
greatly desired in Egypt in Dynastic times, largely came from further
south in Africa, passing through Nubia. Some A-Group chiefs must,
therefore, have benefited economically from the trade in raw materials,
as is clearly evident from the rich burials excavated at Qustul and
Sayala, but the kind of socio-political complexity attested in Upper
Egypt at that date is unlikely to have occurred in Nubia. The floodplain
of the Nile is much narrower in Lower Nubia than in Upper Egypt, and
Lower Nubia simply did not have the agricultural potential to support
greater concentrations of population and full-time specialists such as
craftsmen and government administrators. The fact that the material
culture of the Naqada culture was later found in northern Egypt with
no Nubian elements would also seem to argue against any Nubian
origin for the unified Egyptian state.
The Early ist-Dynasty State
By ^.3000 BC, the Early Dynastic state had emerged in Egypt, con-
trolling much of the Nile Valley from the Delta to the first cataract at
Aswan, a distance of over 1000 km. along the Nile. While the presence
64 KATHRYN A. BARD
of the Naqada culture is clearly evident in the Delta in later Naqada II
and Naqada III times, the extension of Egyptian political control south-
wards during the ist Dynasty is demonstrated by the remains of a
fortress on the highest point of the shore on Elephantine Island, a
region that had been occupied by A-Group peoples in Predynastic
times. With the ist Dynasty, the focus of development shifted from
south to north, and the early Egyptian state was a centrally controlled
polity ruled by a (god-)king from the Memphis region.
What is truly unique about the early state in Egypt is the integration
of rule over an extensive geographic region, in contrast to contempor-
aneous polities in Nubia, Mesopotamia, and Syria-Palestine. Although
there is certainly evidence of foreign contact in the fourth millennium
BC, the Early Dynastic state that emerged in Egypt was unique and
indigenous in character. It is likely that a common language, or dia-
lects of that language, facilitated political unification, but nothing is
really known about the spoken language, while early writing preserves
specialized information that is of a very cursory nature at this point in
cultural development.
One result of the expansion of Naqada culture throughout northern
Egypt would have been a greatly elaborated (state) administration, and
by the beginning of the ist Dynasty this was managed in part by early
writing, used on sealings and tags affixed to state goods. Archaeo-
logical evidence for state control consists of the names of ist-Dynasty
kings (serekhs) on pots, sealings, labels (originally attached to con-
tainers), and other artefacts found at major Early Dynastic sites in
Egypt. Such evidence also suggests that a state taxation system was
already in place in the early dynasties.
At Memphis the earliest archaeological strata that have so far been
excavated date to the First Intermediate Period, and strata from the
Early Dynastic city may be buried under much alluvium. Further west,
drill cores taken by David Jeffreys have revealed both Old Kingdom
and Early Dynastic pottery. Graves and tombs, however, are found in
this region from the ist Dynasty onwards; therefore it is likely that the
city was founded around then. Tombs of high officials have been
found at nearby North Saqqara, and officials of all levels were buried at
other sites in the Memphite region. Such funerary evidence suggests
that the Memphis region was the administrative centre of the state and
also indicates that the early Egyptian state was highly stratified in its
social organization.
In the south, Abydos remained the most important cult centre, and
it has been suggested that in the ist Dynasty the smaller Predynastic
THE E M E R G E N C E OF THE EGYPTIAN STATE 65
settlements, which have left more ephemeral archaeological evidence,
were replaced by one town constructed in mud brick at Abydos. The
kings of the ist Dynasty were buried at Abydos, another indication of
the Upper Egyptian origins of this state. From the very beginning of
the Dynastic Period the institution of kingship was a strong and
powerful one and would remain so throughout the major historical
periods. Nowhere else in the ancient Near East at this early date was
kingship so important and central to control of the early state.
Other towns must have developed or been founded as adminis-
trative centres of the state throughout Egypt, but the spatial organiza-
tion of communities was not like that in contemporaneous southern
Mesopotamia, where huge cities were organized around large cult
centres. On the other hand, neither was early Egypt a 'civilization
without cities', as was once suggested. Egyptian towns and cities may
have been more loosely organized spatially than Mesopotamian ones,
and the royal residence is known to have shifted in location. Owing to a
number of factors, towns and cities in ancient Egypt have not been well
preserved, or are deeply buried under alluvium or modern settlements
and thus cannot be excavated. Nevertheless, some archaeological evi-
dence for the earliest towns has survived. At Hierakonpolis, an elab-
orately niched mud-brick facade within the town (Kom el-Ahmar) has
been interpreted as the gateway to a 'palace', possibly an administra-
tive centre of the early state. At Buto, in the Delta, a rectangular mud-
brick building dating to the early ist Dynasty, which was constructed
above earlier mud-brick buildings of Naqada II and III and Dynasty o,
may be the remains of a temple within the town.
Most ancient Egyptians in the Early Dynastic Period (and all later
periods), however, were farmers living in small villages. Cereal agri-
culture was the economic base of the ancient Egyptian state. Through-
out the fourth millennium BC, villages became increasingly dependent
on the cultivation of emmer wheat and barley, which was incredibly
successful in the environment of the Nile floodplain in Egypt.
By the Early Dynastic Period, simple basin irrigation may have been
practised, thus extending the amount of land under cultivation and
producing increased yields. Unlike practically any other irrigation sys-
tem in the world, salinization did not occur in Egypt, because the
annual Nile flood flushed out the salts. Given that rainfall by this time
was negligible, the annual flooding provided the necessary moisture at
the right time of year—July and August—so that the wheat could be
sown in September after the flooding receded. The species of wheat
that were introduced into Egypt matured during the winter months
66 KATHRYN A. BARD
and could be harvested before spring, when the return of high tem-
peratures and drought might otherwise have killed the crops. Huge
agricultural surpluses were possible in this environment, and when
such surpluses were controlled by the state they could support the
flowering of Egyptian civilization that is seen in the ist Dynasty.
The Royal Cemetery at Abydos
The nature of early Egyptian civilization was expressed primarily
through monumental architecture, especially the royal tombs and
funerary enclosures at Abydos, and the large tombs of high officials at
North Saqqara. Formal art styles, which are characteristically Egyptian,
also emerged in the Naqada Ill/Dynasty o and Early Dynastic periods.
What is characteristically Egyptian in the monumental architecture
and commemorative art (such as the Narmer Palette) is reflective of
full-time craftsmen and artisans supported by the crown. Artefacts
redolent of the highest quality of craftsmanship are found in royal and
elite tombs of the period. Examples include a steatite disk inlaid with
an Egyptian alabaster carved scene of two hounds hunting gazelles
(from Tomb 3035 at Saqqara), and bracelets with beads of gold, tur-
quoise, amethyst, and lapis lazuli (from King Djer's tomb at Abydos).
A similarly high standard of craftsmanship may be observed in the
ebony and ivory artefacts and the copper tools and vessels found in the
elite tombs, all of which were reflective of court sponsorship. The
presence of copper artefacts in the tombs was probably the result of
royal expeditions to copper mines in the Eastern Desert and/or
increased trade with copper-mining regions in the Negev/Sinai, as
well as the expansion of copper working in Egypt.
Although it was previously thought that the ist-Dynasty rulers were
buried at North Saqqara, where Bryan Emery excavated the large mud-
brick superstructures with elaborately niched facades, it is now thought
by most scholars that these tombs belonged to ist- and 2nd-Dynasty
high officials while the royal cemetery in the Umm el-Qa c ab area at
Abydos is the burial place of their kings. Only at Abydos is there a
small number of large tombs that correspond to the kings (and one
queen) of this dynasty, and only at Abydos are there the remains of the
funerary enclosures for all but one of the rulers of this dynasty, as has
been demonstrated by David O'Connor's excavations during the 19805
and 19905.
What is clearly evident in the Abydos royal cemetery is the ideology
of kingship, as symbolized in the mortuary cult. The development of
THE E M E R G E N C E OF THE EGYPTIAN STATE 67
monumental architecture symbolized a political order on a new scale,
with a state religion headed by a god-king to legitimize the new politi-
cal order. Through ideology and its symbolic material form in tombs,
widely held beliefs concerning death came to reflect the hierarchical
social organization of the living and the state controlled by the king—a
politically motivated transformation of the belief system with direct
consequences in the socio-economic system. The king was accorded
the most elaborate burial, which was symbolic of his role as mediator
between the powers of the netherworld and his deceased subjects, and
a belief in an earthly and cosmic order would have provided a certain
amount of social cohesion for the Early Dynastic state.
Seven tomb complexes of the ist Dynasty were first excavated by
Emile Amelineau in the 18905 and then re-excavated more carefully by
Petrie. These belong to the following kings: Djer, Djet, Den, Anedjib,
Semerkhet, and Qa e a, as well as Queen Merneith, who may have been
the mother of Den and perhaps also regent during the earlier part of
his reign. Not only had these tombs been plundered, but there is
evidence that they had been intentionally burned. In the Middle King-
dom the tombs were excavated and rebuilt for the cult of Osiris, and
Djer's tomb was converted into a cenotaph for the god. Given such a
history, it is remarkable that the work of Petrie in 1899-1901 and the
excavations undertaken by the German Archaeological Institute since
the 19705 have enabled the appearance of the early tombs to be
reconstructed. Although only subterranean chambers of mud brick
remain, the tombs would originally have been roofed and may have
been covered by a mound of sand before which stone stelae carved with
the royal name (several of which have survived) would probably have
been placed. Rows of subsidiary graves surrounded each royal tomb.
In the area to the north-east of the royal cemetery, called Cemetery
B, is the tomb complex of Aha, now conventionally listed as the first
king of this dynasty. Also in Cemetery B are tombs that have been
identified by Werner Kaiser as those of the last three kings of Dynasty o:
Iri-Hor, Ka, and Narmer. These tombs consist of double chambers,
whereas the tomb complex of Aha is made up of several separate
chambers built in three stages with a number of subsidiary burials to
the north-east. Although looted, a new dimension in burial can clearly
be seen in Aha's tomb complex: traces of large wooden shrines are
found in three chambers and thirty-three subsidiary burials contained
the remains of young males, 20-25 years old, who had probably been
killed when the king was buried. Near one of these subsidiary graves
were the remains of the burials of at least seven young lions.
68 KATHRYN A. BARD
All of the other ist-Dynasty royal burials at Abydos have subsidiary
burials in wooden coffins, and this is the only period in ancient Egypt
when humans were sacrificed for royal burials. Nancy Lovell, who has
examined the skeletons from some of these subsidiary burials, sug-
gests that their teeth show evidence of death by strangulation. Perhaps
officials, priests, retainers, and women from the royal household were
all sacrificed to serve their king in the afterlife. Crude stelae carved
with the names of the deceased accompany many of these burials,
which are found with grave goods, such as pots, stone vessels, copper
tools, and ivory artefacts. Dwarfs (who may perhaps have been
employed to amuse the king) and dogs that may have been hounds or
pets have also been found in these graves. The tomb of Djer has the
most subsidiary burials (338), and it is in general the later royal burials
that have fewer ones. For unknown reasons, the practice seems to have
been discontinued after the ist Dynasty, and in later times small
servant statues and then shabtis (funerary figurines) may have become
more acceptable substitutes.
All of the ist-Dynasty tombs at Abydos contained wooden shrines
where the actual burial was located. The tomb complex of Djer is the
largest, covering an area of 70 x 40 m. (including the subsidiary
burials in rows). The royal burial was located in the centre of a mud-
brick-lined chamber, measuring 18 x 17 m. (306 sq. m. in floor area)
and 2.6 m. deep, with short walls perpendicular to three sides of the
burial chamber, forming separate storage chambers. Although this
tomb was later converted into a shrine for the god Osiris, Petrie still
found a linen-wrapped arm with bracelets that apparently derived
from the original burial; the arm itself no longer survives, but the
jewellery is in the Egyptian Museum, Cairo.
By the reign of Den, in the middle of the ist Dynasty, a major inno-
vation can be seen in the design of the royal tombs: the addition of a
staircase. This made it possible for the entire tomb, including the
roofing, to be built during the king's lifetime, and would have facili-
tated the construction work in a very deep pit. In the middle of the
staircase was a wooden door, and beyond this, at the entrance to the
burial chamber, was a portcullis to block grave robbers. The tomb and
136 subsidiary burials cover about 53 x 40 m., and the burial chamber
itself is 15 x 9 m. in area and 6 m. deep. The tomb's design and
decoration are the most elaborate at Abydos: the floor of the burial
chamber was paved in slabs of red and black granite from Aswan,
which is the earliest known use of this very hard stone on a large scale.
A small room to the south-west, with its own small staircase, may have
THE E M E R G E N C E OF THE EGYPTIAN STATE 69
been an early serdab (a chamber where statues of the deceased were
placed). Excavations by the German Archaeological Institute in the
debris from earlier excavations indicate that grave goods would have
included many pots with seal impressions, stone vessels, inscribed
labels, and other carved artefacts in ivory and ebony, as well as inlays
from boxes or furniture. To the south of the tomb chamber the unusu-
ally long subsidiary chambers contained many jars, probably originally
containing wine.
In a later royal burial belonging to Semerkhet, Petrie found the
entrance ramp (not a staircase as in Den's tomb) saturated up to 'three
feet' deep with aromatic oil. Almost 5,000 years after the burial, the
scent was still so strong that it permeated the entire tomb. In the tomb
belonging to the last king of the ist Dynasty, Qa e a, thirty inscribed
labels describing the delivery of oil were found during re-excavation by
the German Institute. Most likely these oils were imported from
Syria-Palestine, and may have been made from berries or resins of
trees found there. The presence of such huge quantities of oil in
Semerkhet's tomb (perhaps in the course of his funeral ceremony)
certainly suggests very large-scale foreign trade controlled by the crown
and indicates the importance of such luxury goods for royal burials.
The royal tombs at Abydos are located in the low desert (Umm el-
Qa e ab). To the north-east of them, closer to the edge of the cultivation,
are the funerary enclosures, called 'fortresses' by earlier excavators,
where the cults of each king may have been perpetuated by priests and
other personnel after the burial in the royal tomb, as was the custom in
later royal mortuary complexes. The best-preserved funerary enclosure,
now known as the Shunet el-Zebib, belonged to Khasekhemwy of the
2nd Dynasty. Its niched inner walls are still preserved up to a height of
lo-n m., enclosing an area of about 124 x 56 m. In 1988 O'Connor
discovered a large mound of sand and gravel covered with mud brick,
approximately square in plan, within the enclosure. This mound was
located more or less in the same area as the Step Pyramid of King
Djoser's funerary complex at Saqqara in the 3rd Dynasty (which began
as a low 'mastaba' structure and only in its fourth stage was expanded
to a stepped structure). Both Khasekhemwy's and Djoser's complexes
were surrounded by huge niched enclosure walls, with only one
entrance on the south-east.
Djoser's complex was constructed 40-50 years after Khasekhemwy's,
and the mound at the Shunet el-Zebib may possibly be evidence for a
'proto-pyramid' structure or mound. It is not known if mounds were
constructed in the earlier ist-Dynasty funerary enclosures at Abydos,
70 KATHRYN A. BARD
but this seems likely. Thus, at Abydos the evolution of the royal mor-
tuary cult and its monumental form can clearly be seen. By the 3rd
Dynasty, the royal funerary cult came to reflect a new order of royal
power, deploying vast resources and labour for the construction of the
earliest monument in the world built entirely in stone.
In the early 19905, twelve 'boat burials' were discovered by
O'Connor to the south-east of Djer's funerary enclosure and just
outside the north-east outer wall of Khasekhemwy's. These burials
consist of pits that contained wooden hulls of boats 18-21 m. long but
only about 50 cm. high. Mud bricks had been placed within the hulls
and built up around the outside, forming structures up to 27.4 m. in
length. The pottery associated with the boats is all Early Dynastic in
date, but it is not known at present if the boats date to the ist or 2nd
Dynasty. They all seem to have been created at the same time, and
possibly more boat burials will be discovered when excavations are
extended in this area.
Smaller boat burials have been found associated with the Early
Dynastic tombs of high officials at Saqqara and Helwan. The most
famous Old Kingdom examples are the two undisturbed boats asso-
ciated with Khufu's pyramid at Giza. The purpose of these boat burials
is unknown: possibly the boats were actually used in a funerary cere-
mony or they may have been symbolically buried for the journey in the
afterlife. The examples at Abydos are the earliest evidence of an asso-
ciation between boats and the royal mortuary cult.
The Abydos evidence demonstrates the huge expenditure of the
state on the mortuary complexes—both tombs and funerary enclos-
ures—of the ist-Dynasty kings. These rulers had control over vast
assets, including craft products from royal workshops, exotic goods,
and raw materials imported in huge quantities from abroad, and
probably also conscripted labour (as well as individuals who could be
sacrificed for burial with the king). The paramount role of the king is
certainly expressed in these monuments, and the symbols of the royal
mortuary cult which evolved at Abydos were to become further elab-
orated in the pyramid complexes of the Old and Middle kingdoms.
The Tombs of High Officials at North Saqqara and Elsewhere
At North Saqqara there are some very impressive tombs of high
officials of the ist Dynasty, although none is on the scale of the com-
bined monuments (tomb and funerary enclosure) which the ist-
Dynasty kings built at Abydos. Some of the North Saqqara tombs are
THE E M E R G E N C E OF THE EGYPTIAN STATE 71
very substantial, and what is truly impressive are the elaborately
niched mud-brick superstructures, which are missing in the royal
burials at Abydos. The North Saqqara tombs were much better pre-
served than the Abydos royal tombs; when they were excavated, some
of the niched facades still retained evidence of painted geometric
designs and the burial chambers still had wooden floors. A number of
the North Saqqara tombs were also accompanied by rows of subsidiary
burials, but there are fewer of these than in the royal cemetery at
Abydos.
It is possible that the North Saqqara tombs combined in one struc-
ture the two monumental symbols of status at Abydos: a subterranean
tomb and an above-ground niched enclosure. For example, Tomb
3357, which dates to the reign of Aha at the beginning of the ist
Dynasty, is an elaborately niched superstructure surrounded by two
mud-brick walls, measuring 48.2 x 22 m. in area. The substructure is
divided by mud-brick walls into five large chambers, roofed with
timbers, while the superstructure contains twenty-seven additional
chambers for more grave goods. To the north of this is a so-called
model estate with small-scale rooms, three granary-like structures, a
mud-brick boat grave, and traces of a garden. The hundreds of pottery
vessels found in this tomb are inscribed with the king's name and
information about their contents. Although the owner of the tomb is
unknown, he must have been one of the most important officials of the
kingdom, as indicated not only by the size of the superstructure and its
contents but also by the additional structures and the boat burial.
In the course of time, the design of these Saqqara tombs became
even more elaborate, with a more complex arrangement of chambers,
both subterranean and within the superstructure or the enclosure
walls. As at Abydos, staircases down into the tomb were introduced at
North Saqqara. Two tombs constructed later in the ist Dynasty were
designed with low, rectangular stepped superstructures of mud brick,
which were later surrounded by niched walls. Emery thought that
Djoser's Step Pyramid evolved from these two stepped structures, but
it is more likely that the elements of the first pyramid complex derive
from the funerary enclosures and royal tombs at Abydos.
Although large tombs with niched facades have been recorded at
other sites (Tarkhan, Giza, and Naqada), the largest number—and
those that are largest in size—are concentrated at North Saqqara. What
is found at North Saqqara in the ist Dynasty, then, is evidence of an
official class of a large state. These tombs would also have been the
most important monuments of the state in the north and thus were
72 KATHRYN A. BARD
symbolic of the centralized state ruled very effectively by the king and
his administrators. That huge quantities of craft goods were going out
of circulation in the economy and into tombs is indicative of the wealth
of this early state, which was shared by a number of officials.
Clearly, the mortuary cult was also of great importance to non-
royalty, and the elements of royal burials were emulated in more
modest form in the exclusive cemetery at North Saqqara. Apart from
the subsidiary burials (of retainers or servants?), there is no evidence
from the ist Dynasty at North Saqqara of smaller burials of middle and
lower officials; they were buried elsewhere—for instance, in the
cemetery near the village of Abusir. The North Saqqara cemetery is on
a prominent limestone ridge overlooking the valley, and the presence
of large, elaborately niched superstructures would have been very
impressive symbols of status seen by the other classes of officials at
Memphis.
Smaller tombs and simple pit graves dating to the ist Dynasty are
found throughout Egypt, which is evidence not only of social stratifica-
tion but also of the importance of the mortuary cult for all classes. The
simplest burials of this period are pits excavated in the low desert, such
as those in the 'Fort Cemetery' at Hierakonpolis. These burials are
without coffins and grave goods consist mostly of a few pots. Higher
status burials were larger and supplied with a greater variety and
quantity of grave goods. Sometimes such burials were lined with wood
or mud brick and provided with roofs, as in the case of the graves that
Petrie excavated at Tarkhan. A more elaborate grave of this type was
found at Minshat Abu Omar in the Delta, where the burial pit was
partitioned by mud-brick walls into two or three rooms and contained
up to 125 items of funerary equipment; the largest of these graves
measures 4.9 x 3.25 m. Tombs with mud-brick superstructures, such
as those that George Reisner excavated in Cemetery 1500 at Nag el-
Deir, are found in both Upper and Lower Egypt. Superstructures of
this type, which were sometimes niched, covered a simple burial pit or
more elaborate substructures with one to five rooms. In such tombs,
the contracted body was found in a wooden or ceramic coffin and a
great variety of grave goods accompanied the burial.
Given that most of the archaeological evidence for the ist Dynasty is
mortuary, inferences about socio-political and economic organization
are mostly drawn from these data. As tells in the Delta continue to be
excavated, however, more early settlement data from this period will
become available. From the present evidence, a pattern can be dis-
cerned that points to the establishment of many new settlements and
THE E M E R G E N C E OF THE EGYPTIAN STATE 73
their associated cemeteries on both banks of the Nile in the Memphis
region, as the socio-economic centre shifted to the north by the ist
Dynasty. New sites also emerged in the eastern Delta, undoubtedly
connected to increasing trade and other ventures abroad.
Expansion of the Early State into Southern Palestine and Nubia
In Dynasty o and the early ist Dynasty there is evidence of Egyptian
expansion into Lower Nubia and a continued Egyptian presence in the
northern Sinai and southern Palestine. The Egyptian presence in
southern Palestine did not last to the end of the Early Dynastic Period,
but with Egyptian penetration into Nubia the indigenous A-Group
culture came to an end later in the ist Dynasty.
The source of A-Group wealth was the trade in exotic raw materials
coming from southern regions through Nubia to Upper Egypt. With
the unification of Egypt into a large territorial state, the Crown most
likely wanted to control this trade more directly, which resulted in
Egyptian military incursions in Lower Nubia. A late Predynastic
scene carved on a rock at Gebel Sheikh Suliman near Wadi Haifa
suggests some kind of military victory by the Egyptians, and a Nubian
campaign may possibly be depicted on an ebony label from Abydos.
With the display of force by the Egyptians, A-Group peoples may
simply have left Lower Nubia and gone elsewhere (to the south or
desert regions), and there is no evidence of indigenous peoples living
in Lower Nubia until the C-Group culture, beginning in the late Old
Kingdom. How Egypt controlled Lower Nubia in the Early Dynastic
Period is unknown. Evidence of an Egyptian installation has been
found at Buhen North, with strata which possibly date as early as the
2nd Dynasty. More secure dating at Buhen, however, is provided by
seals of kings of the 4th and 5th Dynasties, and it is uncertain if there
were permanent Egyptian forts or administrative/trading centres in
Nubia in the Early Dynastic Period.
Fortified cities found in the north and south of Palestine have been
dated to the EBA II period, which corresponds to the ist Dynasty, a
connection that depends on evidence excavated by Petrie in two royal
tombs at Abydos (those of Den and Semerkhet). Petrie found sherds
of an imported ware bearing painted designs, which he interpreted
as 'Aegean'. This pottery has been called 'Abydos Ware', and is now
known to derive from the EBA II culture of southern Palestine.
In stratum III at the site of Ain Besor in southern Palestine, ninety
74 KATHRYN A. BARD
fragments of seal impressions of Egyptian kings have been found asso-
ciated with a small mud-brick building and ceramics that are mainly
Egyptian, including many fragments of bread moulds. The seal
impressions are made from local clay and evidently belonged to royal
officials of the ist Dynasty. Four kings' names are attested (Djer, Den,
Anedjib, and probably Semerkhet), and the ceramics and seal impres-
sions suggest state-organized trade directed by Egyptian officials
residing at this settlement for most of the ist Dynasty. Alan Schulman,
who identified the seal impressions, thinks that the site operated as an
Egyptian border-control checkpoint, which would have been an early
prototype for those described in two papyri dating to the Ramessid
Period. Such evidence in southern Palestine is missing during the 2nd
Dynasty, however, and active overland contact may have been broken
off by then, as the sea trade with the Lebanon intensified. As raw
materials from this region (wood, oils, and resins from coniferous
trees) were imported in increasing quantities, which could perhaps
only have been conveyed by sea, the land route to Palestine may have
been gradually bypassed. It is probably significant that the earliest
inscriptional evidence of an Egyptian king at the Lebanese site of
Byblos belongs to the reign of Khasekhemwy, the last ruler of the 2nd
Dynasty.
The Invention and Use of Writing
Depending on when the early state in Egypt emerged, the earliest
known use of writing (in Tomb U-j at Abydos) may predate political
unification of the north and south. Certainly by Dynasty o, writing was
used by scribes and artisans of the Egyptian state. Although some
scholars believe that the Egyptian writing system was invented in the
late fourth millennium BC, with stimulus from Mesopotamia, where
the earliest writing is found, the two writing systems are so different
that it seems more likely that they are both the result of independent
invention.
The earliest codification of signs probably occurred in Naqada III/
Dynasty o. Like Egyptian writing in the Dynastic Period, these early
hieroglyphs consist of elements of ideographic and phonetic signs.
Specific decipherments of many of the Early Dynastic inscriptions,
however, remain uncertain. The use of writing by the early state in
Egypt has a royal context, and was an innovation of great importance to
this state. Just as a royal style of art developed as a court-centred insti-
tution following the unification, so did writing. The early state used
THE E M E R G E N C E OF THE EGYPTIAN STATE 75
writing in two contexts: for economic and administrative purposes and
in royal art.
The economic function of writing must have developed as more
resources of the state came under royal control. Hieroglyphs appear on
royal seal impressions, labels, and potmarks to identify goods and
materials marshalled for and by the state, as well as on seals of officials
of the state. Titles of owners of these goods and places of origin are also
sometimes recorded.
Beginning in Dynasty o, royal serekhs are first seen. The serekh is the
earliest format of the king's name in hieroglyphs, comprising phonetic
signs, placed inside a 'palace-facade' design that was surmounted by
the image of a falcon. Serekhs are found inscribed or painted on jars
and labels and impressed on jar sealings. Such containers were prob-
ably storage jars, for agricultural products collected by the state
(perhaps as taxation), and some of these goods were traded or exported
abroad through the northern Sinai to southern Palestine.
From this economic use of writing it can be inferred that there was
already a functioning administrative system by Dynasty o. Early in the
ist Dynasty, a more complex message of identification developed, and
a combination of hieroglyphs and graphic art is found on labels. In the
absence of texts composed of signs ordered in a format by grammar,
which are not known until later, the information conveyed on labels,
especially those arranged in registers, is probably to be read as a text (a
year name) containing historical information. Donald Redford has
suggested that the context of this information on royal labels is an
annals system. The addition of the year sign by the middle of the ist
Dynasty represents a more specific system for recording regnal years
than on earlier labels.
The second use of early writing was on royal commemorative art,
such as the Narmer Palette. Hieroglyphs identify specific persons and
possibly places in representational scenes that are symbolic of the
king's legitimacy to rule. In such scenes, the king is depicted in roles,
both real and symbolic, based on a new ideology: the institution of
Egyptian kingship. Numerical signs, such as those on the Narmer
Macehead, represent captured booty and prisoners, and are probably
greatly exaggerated, as is so often the case in later Egyptian historical
texts.
The iconography of power is clearly seen within the context of such
royal art and includes the use of several important conventions. The
king and his officials are shown in the special dress of their offices,
while their conquered enemies wear next to nothing. A hierarchy of
76 KATHRYN A. BARD
social classes is also evident, from the large-sized king, who is followed
by his smaller sandal-bearer, to his even smaller officials, to the small-
est figures of conquered enemies, farmers, and servants. The king is
frequently depicted trampling on his enemies, in visual puns. The
early Egyptian signs do not replicate the information conveyed in the
scenes, but serve as name labels for places and persons.
Part of the problem of understanding how writing developed in
Early Dynastic Egypt is connected both with the types of artefacts on
which early writing appears and with their archaeological contexts.
Most examples of early writing are associated with the funerary cult
and are not records of economic activities from settlements. Thus the
early labels inscribed with hieroglyphs have been found in royal and
elite tombs. From the royal cemetery at Abydos are stelae with the
kings' names in serekhs and smaller inscribed stelae associated with
the subsidiary burials. The one funerary stele with a longer text, from
the late ist-Dynasty tomb of Merka at Saqqara, is simply a list of his
titles. The early state probably kept economic records of some sort to
facilitate its economic and administrative control, but there is only
indirect evidence of this in the form of inscribed labels.
Early Dynastic Cult Centres
Some of the inscribed labels from the ist Dynasty bear scenes with
structures that are temples or shrines, such as the walled compound
for the goddess Neith in the top register of a wooden label from Aha's
tomb at Abydos. Early writing also appears on some of the small votive
artefacts that were probably offerings or donations to cult centres.
Early Dynastic carved stone vessels were sometimes inscribed, and
signs on some of these suggest that they may have come from cult
centres. A number of such stone vessels may have been usurped from
cult centre(s) of gods and buried in Djoser's Step Pyramid at Saqqara.
Such evidence points to the existence of cult temples outside the royal
mortuary cult in the Early Dynastic Period, but there is very little
archaeological evidence of such architecture.
Perhaps the most impressive examples of early temple art are the
three colossal limestone figures of a fertility god (Min?) that Petrie
excavated at Koptos. One restored figure in the Ashmolean Museum is
over 4 m. high. Stylistically, the colossi seem to date either to Dynasty
o or the early ist Dynasty. Buried in a deep deposit beneath the floor of
the later temple of Isis and Min were figurines (possibly votive items)
that are now thought to date to the Old Kingdom, but there are also
THE E M E R G E N C E OF THE EGYPTIAN STATE 77
potsherds that are clearly from late Predynastic (Naqada) wares. Such
evidence strongly suggests the existence of a temple or shrine at this
location since Predynastic times. Given the huge size of the colossi,
they were probably placed in a temple courtyard, although no remains
of any early structures were found. The quarrying, transport, carving,
and erection of such large pieces of stone imply large-scale (com-
munity) organization for renovating and furnishing a cult centre.
Given that such expenditure of energy is much more evident in the
royal mortuary cult of the ist Dynasty, the association of the Koptos
colossi with a cult centre is remarkable.
During the 19805 and 19905, German Archaeological Institute
excavations on Elephantine Island at the first cataract revealed the
remains of a shrine dating to the Early Dynastic period, a fortress built
during the ist Dynasty, and a large fortified wall encompassing the
town in the 2nd Dynasty. What cult was practised at this early shrine
cannot be identified, but it was located beneath an 18th-Dynasty stone
temple of the goddess Satet. The early shrine is very simple, consisting
only of some mud-brick structures less than 8 m. wide nestled into a
natural niche formed by granite boulders. Hundreds of small votive
artefacts, mainly comprising human and animal faience figurines,
were excavated beneath the i8th-Dynasty temple. Many of these date to
the Old Kingdom, but some are Early Dynastic, including a fragment
of a small statue of a seated king with a sign that has been identified as
Djer's name. Such a concentration of so many votive figurines made
over the course of six dynasties (c.8oo years) suggests a craft workshop
associated with this temple where worshippers and/or petitioners
could obtain such artefacts to leave during their visits.
Similar figurines have been also found in deposits at Abydos,
beneath an Old Kingdom structure that has been identified either as a
temple of the god Khenti-amentiu or a ka-chapel of the 6th-Dynasty
ruler Pepy II. Probably some of these figurines derive from an Early
Dynastic temple. At Hierakonpolis, more animal figurines in faience,
fired clay, and stone, which belong stylistically to the late Predynastic
and Early Dynastic, have been found in Quibell and Green's 'Main
Deposit/ beneath a later temple. The same archaeological context
(near the Main Deposit) produced the Scorpion Macehead, the Narmer
Palette, and the Narmer Macehead, as well as another ceremonial
palette (the Two Dog Palette) which appears to be stylistically earlier
than that of Narmer, a number of small ivories inscribed with the
names of Narmer and Den, two statues of King Khasekhemwy of the
2nd Dynasty, and inscribed stone vessels made during his reign.
78 KATHRYN A. BARD
Structural evidence for an early temple is found in the same area,
where a low oval revetment of sandstone blocks, about 42 x 48 m.,
encased a mound of sterile sand that had been brought to the site from
the desert. This structure was made sometime between the late Pre-
dynastic period and the 3rd Dynasty; it was located within a walled
enclosure, which O'Connor has suggested was a temple compound
similar in design to Khasekhemwy's funerary enclosure and mound at
Abydos.
If O'Connor is correct, the main Early Dynastic cult temples at
Abydos, Hierakonpolis, and Elephantine have not yet been located and
excavated, but what evidence there is points to the existence of cult
temple compounds within towns. Such temples served a different
function from those associated with the funerary complexes, which
were located outside the towns. The architectural evidence of Early
Dynastic Egyptian cults (of unknown deities) is much less impressive
than the contemporaneous remains of temples in southern Meso-
potamia. Nevertheless, town cult centres in Early Dynastic Egypt may
have served to integrate society in towns and nomes in a shared belief
system that was perhaps of more immediate significance to the lives
of the local peoples than the mortuary cults in royal and elite cemet-
eries.
The 2nd-Dynasty State
There is much less evidence for the kings of the 2nd Dynasty than
those of the ist Dynasty until the last two reigns (Peribsen and
Khasekhemwy). Given what is known about the early Old Kingdom in
the 3rd Dynasty, the 2nd Dynasty must have been a time when the
economic and political foundations were put in place for the strongly
centralized state, which developed with truly vast resources. Such a
major transition, however, cannot be demonstrated from the archaeo-
logical evidence for the 2nd Dynasty.
In 1991-2 the tomb of the last king of the ist Dynasty, Qa c a, was
re-excavated at Abydos by the German Archaeological Institute, and
seal impressions of Hetepsekhemwy, the first king of the 2nd Dynasty,
were found in it. The German archaeologists have interpreted this
find as evidence that Hetepsekhemwy completed the tomb of his
predecessor and that there was no break in the dynastic succession.
Where the early kings of the 2nd Dynasty were buried is uncertain,
however, as there is no evidence of their tombs at Abydos. The
only 2nd-Dynasty monuments at Abydos are two tombs and two
THE E M E R G E N C E OF THE EGYPTIAN STATE 79
funerary enclosures that belonged to Peribsen and Khasekhemwy.
There is also a large niched enclosure known as the Tort' at Hiera-
konpolis, by the entrance to the Great Wadi, which has been dated
to the reign of Khasekhemwy by an inscribed stone jamb. The exis-
tence of this sole structure at Hierakonpolis cannot be explained, and
it is unclear whether it was a second royal funerary enclosure for
Khasekhemwy.
At Saqqara two enormous series of underground galleries, each
over 100 m. long, have been found south of Djoser's Step Pyramid
complex. Associated with these galleries are seal impressions of the
first three kings of the 2nd Dynasty (Hetepsekhemwy, Raneb, and
Nynetjer), whose names are also listed on the shoulder of a granite
statue of a 2nd-Dynasty priest called Hetepdief (found at nearby
Mitrahina and now in the Egyptian Museum, Cairo). The super-
structures of these Saqqara tombs are entirely gone, but it is possible
that two of the kings of this dynasty were buried there. Two sets of
underground galleries have also been found beneath the north court of
the Step Pyramid complex, and may have been created for royal burials
of the 2nd Dynasty. When Djoser's monument was constructed in the
3rd Dynasty, the superstructures of the two earlier tombs would have
had to be removed. Such a reconstruction of events is not impossible,
given that huge quantities of stone vessels from the ist and 2nd
Dynasties, presumably usurped from earlier mortuary complexes and/
or cult centres, were found beneath Djoser's complex.
The tomb of Peribsen (perhaps also known as Horus-Sekhemib) in
the royal cemetery at Abydos is fairly small (16.1 x 12.8 m.). The central
burial chamber is made of mud brick, unlike the ist-Dynasty royal
burial chambers, which were lined with wood. When the name
Peribsen is written in a serekh, it is surmounted not by the usual Horus
falcon (as the Sekhemib name is) but by the Seth animal, a hound- or
jackal-like creature with a wide, straight tail. This dramatic change in
the format of the royal name has been interpreted as representing
some kind of rebellion, which was squashed or reconciled by the last
king of the dynasty, Khasekhemwy, whose name appears in serekhs
surmounted by both the Horus falcon and the Seth animal. Such a
conflict may be symbolized in Egyptian mythology, as in the case of the
literary tale The Contendings of Horus and Seth. Whether mythologies,
which are known from much later texts, and symbols in the serekhs of
two kings of the late 2nd Dynasty represent actual historical reality is
uncertain. An epithet of Khasekhemwy's from seal impressions, 'the
Two Lords are at peace in him', however, lends support to the theory
8o KATHRYN A. BARD
that he resolved some internal conflict, if Two Lords' can be taken to
refer to Horus and Seth (and their followers).
The last tomb constructed in the royal cemetery at Abydos is that of
Khasekhemwy, who was known as 'Khasekhem' earlier in his reign. It
is much larger than Peribsen's, and its design is different, comprising
one long gallery, 68 m. long and 39.4 m. at its widest point, divided
into fifty-eight rooms with a central burial chamber made of quarried
limestone. The constructed burial chamber, measuring about 8.6 x
3 m. and preserved to a height of 1.8 m., is the earliest known large-
scale construction in stone. Although most of the contents were
removed by Amelineau, they were well recorded, and Petrie discusses
them in his 1901 publication. The funerary equipment includes huge
quantities of copper tools and vessels, stone vessels (some with gold
covers), flint tools, and pottery vessels filled with grain and fruit. Petrie
also describes small glazed artefacts, carnelian beads, model tools,
basketwork, and a great quantity of sealings. Given the large number
of storerooms in this tomb, it could certainly have held more grave
goods than all the ist-Dynasty tombs in this cemetery.
High officials of the state continued to be buried at North Saqqara in
the 2nd Dynasty. Near the pyramid of the 5th Dynasty ruler Unas,
Quibell excavated five large subterranean gallery tombs, carved into
the limestone bedrock, and he suggested that they represented a kind
of house for the afterlife, with men's and women's quarters, a 'master
bedroom' for the burial, and even bathrooms with latrines. The largest
of the five, Tomb 2302, consisted of twenty-seven rooms beneath a
mud-brick superstructure, covering an area of 58.0 x 32.6 m. The
superstructures of these 2nd-Dynasty tombs were no longer elabor-
ately niched on all four sides as in the ist Dynasty, but were designed
with only two niches on the east side, perhaps indicating places where
offerings could be left by priests or family members after the burial (a
design feature that would later be found in private tombs throughout
the Old Kingdom).
The plans of the 2nd-Dynasty elite tombs clearly evolved from the ist
Dynasty high officials' tombs at North Saqqara. Because the Saqqara
plateau was made up of good quality limestone, these 2nd-Dynasty
tombs were designed with rooms for funerary goods that were exca-
vated deep in the bedrock, where the storage rooms may have been
better protected from grave robbing than when they had been located
in the superstructure. The later 2nd-Dynasty tombs at Saqqara, which
probably belonged to middle-level officials, are similar in design to
standard Old Kingdom mastaba-tombs, consisting of a vertical shaft
THE E M E R G E N C E OF THE EGYPTIAN STATE 8l
excavated in the bedrock leading to a walled-off burial chamber. Above
the shaft and chamber was a small mud-brick superstructure with two
niches on the eastern side.
At Helwan, on the east bank of the Nile, excavations have revealed
over 10,000 graves dating from Naqada III to the ist and 2nd Dynas-
ties, and probably the early Old Kingdom. These tombs were somewhat
modest in size and belonged to middle-level officials. A distinctive
feature of a number of the 2nd-Dynasty tombs at Helwan was the
presence of a stele set in the tomb's ceiling, which was carved with a
seated representation of the tomb owner, as well as his name, titles,
and the so-called offering formula.
Short wooden coffins for contracted burials, which had been found
only in elite tombs in the ist Dynasty, became much more common in
2nd-Dynasty graves such as those at Helwan. At Saqqara, Emery and
Quibell found 2nd-Dynasty corpses wrapped in linen bandages soaked
in resin, early evidence of some attempt to preserve the actual body
before mummification techniques had been worked out. Such
measures were necessitated by burial in a coffin, as opposed to Pre-
dynastic burials, in which the body was naturally dehydrated by the
warm sand in a pit in the desert. The increased use of wood and resin
in middle-status burials of the 2nd Dynasty probably also points to
greatly increased contact and trade with the Lebanese region at this
time.
Conclusions
The architecture, art, and associated beliefs of the early Old Kingdom
clearly evolved from forms of the Early Dynastic period. What is seen
in the Step Pyramid complex of Djoser is a transformation of the Early
Dynastic tombs into the first monument in the world made entirely of
stone—on a truly huge scale. While this monument is also symbolic of
the enormous control exercised by the Crown, such power must have
been developing incrementally throughout the ist and 2nd Dynasties,
following the unification of the large territorial state in Naqada III and
Dynasty o.
The Early Dynastic Period was a time of consolidation of the enor-
mous gains of unification, which could easily have failed, when a state
bureaucracy was successfully organized and expanded to bring the
entire country under royal control. This was done through taxation, to
support the Crown and its projects on a grand scale, including expedi-
tions for goods and materials to the Sinai, Palestine, the Lebanon,
82 KATHRYN A. BARD
Lower Nubia, and the Eastern Desert. Conscription must presumably
also have been practised in order to build the large royal mortuary
monuments and to supply soldiers for military expeditions. The use of
early writing no doubt facilitated such state organization.
There were obvious rewards for those who were bureaucrats of the
state, as the early cemeteries on both sides of the river in the Memphis
region clearly attest. Belief in the benefits of a mortuary cult, where
huge quantities of goods were constantly going out of circulation in the
economy, was a cohesive factor that helped to integrate this society in
both the north and south. In the early dynasties, when the Crown
began to exert enormous control over land, resources, and labour, the
ideology of the god-king legitimized such control and became increas-
ingly powerful as a unifying belief system.
The flowering of early civilization in Egypt was the result of major
transformations both in socio-political and economic organization and
ideology. That such transformations were successful in the Early Dyn-
astic Period is truly remarkable, given that contemporaneous polities
elsewhere in the Near East were much smaller in territory and popu-
lation. That this state was successful for a very long time—a total of
about 800 years until the end of the Old Kingdom—is in part due to
the enormous potential of cereal agriculture on the Nile floodplain, but
it is also a result of Egyptian organizational skills and the strongly
developed institution of kingship.
5
The Old Kingdom
(£.2686-2160 BC)
JAROMIR MALEK
The term 'Old Kingdom' was imposed on Egyptian chronology by
nineteenth-century historians and its connotations can be misleading.
It reflects an approach to the periodicity of history about which we may
now entertain serious reservations. The ancient Egyptians never used
it and would have found the difference between the Early Dynastic
Period (3000-2686 BC) and the Old Kingdom (2686-2160 BC) rather
difficult to grasp. The last king of the Early Dynastic Period and the
first two rulers of the Old Kingdom were, it seems, all related to Queen
Nimaathap, who was described as mother of the king's children under
Khasekhemwy and as 'mother of the king of Upper and Lower Egypt'
under Djoser 2667-2648 BC. For the Egyptians even more important
was the fact that the place of the royal residence did not change, but
remained at White Wall (Ineb-hedj), on the west bank of the Nile south
of modern Cairo.
However, the Egyptians were aware of, and acknowledged, the revo-
lutionary contribution made by King Djoser's builders to royal funer-
ary architecture. Large state-organized building projects exerted an
immediate and profound effect on Egyptian economy and society. For
us, this is the main justification of a division between the Early Dyn-
astic Period and the Old Kingdom, although it is signalled by progress
in architecture rather than personal royal changes.
84 J A R O M I R MALEK
Chronological Considerations and the Main Characteristics of
the Period
Thanks to the information provided by a Ramessid king-list written on
a papyrus in the Museo Egizio in Turin, the so-called Turin Canon,
there are remarkably few weak links in the order and dating of Old
Kingdom rulers. Among the chronologically significant kings, only the
reigns of Menkaura (2532-2503 BC, but perhaps less) and Neferirkara
(2475-2455 BC, but this is almost certainly too long) present more
serious difficulties. We have no safe dates based on contemporary
astronomical observation, and calculations made for other periods
may change the relative position of the Old Kingdom in the chrono-
logical scheme of ancient Egyptian history. The degree of reliability
with which we credit ancient sources and our understanding of the
Egyptian dating system are also important. On the whole, however, it
seems that 2686 BC as the beginning of the reign of Nebka (the first
ruler in Manetho's 3rd Dynasty, although his position in the dynasty
has recently been challenged) is secure within a margin of error of
about twenty-five years.
The end of the period, about five and a half centuries later, is more
obscure, but the ancient Egyptians and modern historians are in broad
agreement on its characteristics. For the Egyptians, the transfer of the
royal residence away from Memphis was represented by a sharp divi-
sion in their king-lists. As this approximately coincided with profound
political, economic, and cultural changes in Egyptian society, it is
convenient to follow their example. All the same, the lack of accurate
chronological indicators is daunting, and the degree of uncertainty is
such that much of the often lively polemic is, in the present state of our
knowledge, purely academic.
Although the division of Egyptian kings into dynasties (royal ruling
houses), introduced by the Ptolemaic historian Manetho in the third
century BC is generally followed, its weaknesses have rarely been
exposed more convincingly than in the case of the Old Kingdom. We
can establish contemporary reasons for nearly all dynastic breaks, but
more often than not it would be difficult to defend them as sound
historical criteria or discontinuity in the line of kings. Nevertheless, in
the absence of a radical alternative, Manetho's system provides a con-
venient chronological scheme that avoids the more fluid absolute dates
(in years BC).
During the Old Kingdom Egypt experienced a long and uninter-
rupted period of economic prosperity and political stability, in con-
THE OLD K I N G D O M 85
tinuation of the Early Dynastic Period. It rapidly grew into a centrally
organized state ruled by a king believed to be endowed with qualified
supernatural powers. It was administered by a literate elite selected at
least partly on merit. Egypt enjoyed almost complete self-sufficency
and safety within its natural borders; no external rivals threatened its
dominance of the north-eastern corner of Africa and the immediately
adjacent areas of Western Asia. Advances in religious ideas were
reflected in breathtaking achievements in arts and architecture.
Large-Scale Building Projects as Catalysts of Change
King Djoser, known from his monuments as Netjerikhet (his Horus
and nebty names), is one of the most famous rulers in Egyptian history.
On the Turin Canon, his name is preceded by a rubric in red ink. As
late as the reign of Ptolemy V Epiphanes (205-180 BC), nearly 2,500
years later, the Famine Stele on the island of Sehel, in the first-cataract
region, still bore testimony to his image as a paragon of a wise and
pious ruler (djoser means 'holy', 'sacred'). Although the stele was a
tendentious and spuriously historic text put out by the priests of the
local god Khnum, its importance lies in the late awareness of Djoser
that it conveys rather than in the historicity of the events it records.
The annals preserved on the Palermo Stone record the construction
of a stone building called Men-netjeret either in the reign of Khasek-
hemwy, the last ruler of the 2nd Dynasty, or Djoser's predecessor,
Nebka (2686-2667 BC). We learn nothing more about the building
although there is a good chance that this is the structure known as Gisr
el-Mudir at North Saqqara, to the south-west of Djoser's pyramid.
However, it hardly got beyond the initial stages and so the credit for the
first successfully completed large stone building in the world, the Step
Pyramid, goes to Djoser.
The superstructure of Djoser's tomb is the result of six variants of
the plan adopted in turn as the full potential of the new building
material was being realized. Before Nebka and Djoser, stone had been
used only in a limited way for elements of brick-built tombs. The final
structure is a pyramid of six steps, with a ground plan of 140 x 118 m.
and a height of 60 m. It stands within an enclosure measuring some
545 x 277 m., the walls of which probably imitated the facade of the
royal palace. The king's body was laid to rest in a chamber constructed
beneath the pyramid, below ground level. While for us this new
architectural form ushered in a new historical period, it also contains
a clear link with the past. In its initial design it was a mastaba of a
86 J A R O M I R MALEK
rectangular ground plan, a typical royal tomb of the Early Dynastic
period.
A remarkable feature of the enclosure is a large open court and a
complex of shrines and other buildings, the replicas in stone of struc-
tures that would have been built in perishable materials for sed-
festivals (royal jubilees) in the king's lifetime. Here Djoser hoped to
continue to celebrate—during his afterlife—such periodic occasions
in which his energy and powers, and so his ability to rule effectively,
would be renewed. In the southern part of the enclosure, there is a
building (the so-called South Tomb) that imitates the underground
parts of the pyramid. Its function is unclear, but it may be compared to
the satellite pyramids in later pyramid complexes.
Tradition had it that Imhotep (Greek form: Imouthes) was the archi-
tect of Djoser's pyramid and inventor of building in stone. Later he was
deified and regarded as a son of the god Ptah and the patron of scribes
and physicians, equated with the Greek god Asklepios. His historicity
has been confirmed by the discovery of the base of a statue of Djoser
that also bears Imhotep's name. Imhotep's tomb was probably located
at Saqqara, perhaps at the edge of the desert plateau to the east of the
pyramid of his royal master, but it has not yet been located and so
offers one of the most exciting prospects for future fieldwork.
The fact that Imhotep was a high priest of Heliopolis is a pointer to
the early importance of the sun-god Ra (or Ra-Atum). The royal resi-
dence and Egypt's administrative centre were situated in the area
where the god Ptah was the chief local deity, but it is likely that Heli-
opolis (Egyptian lunu, Biblical On), to the north-east of the Old King-
dom capital and on the east bank of the Nile (now a Cairo suburb), was
recognized as the country's religious capital early in the Old Kingdom.
Djoser was the first ruler to dedicate a small shrine there.
The striving for monumental grandeur appropriate to a royal burial
can be detected early in Djoser's reign; it reflected the prevailing view
at the time concerning the position of the king in Egyptian society.
This view may have been further strengthened when it found an ideal
means of expression in funerary architecture. In the course of the next
two centuries the approach was explored to its limits, and this, in its
turn, became a powerful catalyst in the development of Egyptian
society. The step pyramid was now adopted as the norm for a royal
tomb, but none of those planned by Djoser's successors was com-
pleted. The pyramid intended for Sekhemkhet (2648-2640 BC) was
begun to the south-west of that of Djoser and its design was even more
ambitious. A graffito on the enclosure wall mentions Imhotep, who
THE OLD K I N G D O M 87
may still have been active. The ownership of the pyramid was deduced
from the presence of Sekhemkhet's name on clay impressions of seal-
ings in its underground rooms. Although the pyramid's burial chamber
contained a sealed sarcophagus carved from Egyptian alabaster, this
was found to be empty, and it is clear that the superstructure was
abandoned when it reached a height of about 7 m.
A similarly unfinished structure at Zawiyet el-Aryan, to the north of
Saqqara, is assigned with some probability, though without certainty,
to Khaba (2640-2637 BC). The short duration of the reigns of these two
kings (only six years each) was almost certainly to blame for their
failure to complete the pyramids. Little can be said with any confidence
about the family relationships between the kings of the 3rd Dynasty,
but the first two, Nebka and Djoser, may have been brothers.
The 4th Dynasty (2613-2494 BC)
In the reign of King Sneferu (Horus Nebmaat, 2613-2589 BC) the
external form of the royal tomb changed to that of a true pyramid. This
might be regarded as a straightforward architectural development if it
were not for other profound changes that occurred at the same time.
New elements were added to the overall plan, and together they now
formed a pyramid complex. A new orientation was applied to its plan
(the main axis of the complex was now from east to west, while pre-
viously the north-south direction predominated). The pyramid temple
that served as the focus of the funerary cult was built against the
eastern face of the pyramid (that of Djoser is to the north). It was linked
by a causeway to a valley temple, close to the edge of the cultivated area
further to the east, which provided a monumental entrance to the
whole complex. A small satellite pyramid was placed near the southern
face of the pyramid proper. These architectural innovations must have
resulted directly from changes in the doctrine concerning the king's
afterlife. It seems that the earlier astronomically oriented star concepts
were gradually being modified by the incorporation of ideas centred
around the sun-god Ra. Although textual evidence is lacking, already at
this early stage beliefs concerning Osiris were probably also beginning
to influence Egyptian concepts of the afterlife.
Sneferu, probably as the result of planning that went wrong rather
than by choice, had two pyramids constructed at Dahshur, to the
south of Saqqara. The first is the southern Rhomboidal (or Bent)
Pyramid, where the angle of the sloping sides was altered some two-
thirds up its height after structural flaws had been discovered during
88 J A R O M I R MALEK
its construction. The other is the northern Red Pyramid (named from
the colour of the limestone blocks used in the core of the structure), in
which Sneferu was buried. He may also have completed a third struc-
ture at Meidum, still further south, but the ownership of this pyramid
remains in doubt. Visitors who came to see it in the i8th Dynasty, some
1,200 years later, made it quite clear in their graffiti that they thought it
belonged to Sneferu. It is possible that it was originally conceived as a
step pyramid for Sneferu's predecessor Huni (more correctly known
as Nysuteh, and perhaps also to be equated with Horus Qahedjet,
2637-2613 BC), but such a substantial contribution to the pyramid of
one's predecessor would be unique in Egyptian history. Sneferu's later
reputation as a benign ruler may owe much to the etymology of his
name, in that snefer can be translated as 'to make beautiful'.
The sheer volume of material involved in Sneferu's building activi-
ties was greater than that of any other ruler in the Old Kingdom. The
Turin Canon puts the length of his reign at twenty-four years, although
stonemasons' graffiti found on the blocks inside his northern (and
later) pyramid at Dahshur may suggest a longer reign. The problem
could easily be solved if it could be shown that the eponymous occa-
sions of a census that were used for dating purposes (the year was of
the nth census or it was the year after the nth census), and that are
known to have been regularly biennial during the Early Dynastic
Period, now became more frequent (less regular) occasions. The con-
temporary dating system probably required annals or similar records
to which one could refer in order to calculate dates accurately.
Manetho began a new dynasty, his 4th, with Sneferu. It seems that
once again architectural changes provided the criterion for a dynastic
division. The perfection of pyramid design and construction reached
its peak under Sneferu's son and successor, Khufu (Herodotus'
Cheops, Horus Medjedu, 2589-2566 BC), whose full name was
Khnum-khufu, meaning 'the god Khnum protects me'. Khnum was
the local god of Elephantine, near the first Nile cataract, but the reason
for the king's name is not known. Information about the reign and the
king himself is remarkably meagre. He must have been a middle-aged
man when he ascended the throne, but this did not affect the planning
of his grandiose funerary monument. The Great Pyramid at Giza, with
a ground plan of 230 sq. m. and a height of 146.5 m., is the largest in
Egypt. Unusually, the burial chamber is situated in the core of the
pyramid, and not below or on ground level. The plan was, it seems,
changed in the course of the construction, but hardly more than once,
and the design of the superstructure was probably foreseen at the
THE OLD K I N G D O M 89
outset. The usually quoted figure of some 2,300,000 building blocks
averaging about 2.5 tons that were required may be approximate, but
probably not far off the mark. The valley and pyramid temples and
the causeway were originally decorated in low raised relief with scenes
that conveyed the ideas of the Egyptian kingship and recorded in
anticipation certain events that the king hoped to enjoy in afterlife,
such as sed-festivals. The reliefs are, unfortunately, almost completely
lost.
A dismantled boat, some 43.4 m. long and built mainly of cedar-
wood, discovered in a pit near the southern face of the pyramid, has
been successfully excavated and restored. Another such boat still lies
in another pit nearby, but is not as well preserved. It seems likely that
these craft were intended to be used by the deceased king in his jour-
neys across the sky in the company of gods. Two more large boat-
shaped pits were cut in the rock against the eastern face of the
pyramid, and a fifth is situated near the upper end of the causeway.
Three pyramids that contained the burials of Khufu's queens are
lined up to the east of the pyramid. A cache with objects belonging to
Khufu's mother Hetepheres was also discovered to the east of the
pyramid. It was undisturbed and contained some remarkable examples
of furniture, but the body of Hetepheres was not present. A settlement
of priests and craftsmen connected with the king's funerary cult prob-
ably grew up near the valley temples of most pyramids. Khufu's valley
temple is located under the houses of the densely populated modern
village of Nazlet el-Simman, below the desert plateau, but conditions
are too difficult for a full excavation.
The man ultimately responsible for the successful completion of the
project before the end of Khufu's twenty-three-year reign was his vizier
Hemiunu, who was buried in a huge mastaba-tomb in the cemetery to
the west of the pyramid of his royal master. Hemiunu's father, Prince
Nefermaat, was King Sneferu's vizier and may have organized the
building of Sneferu's pyramids. The two family lines, of the kings and
their viziers, ran parallel here for at least two generations. The pyra-
mid's date and its function as a tomb are in no doubt, despite the fact
that the king's body and all funerary equipment fell victim to tomb-
robbers and disappeared without a trace. However, its enormous size,
the astonishing mathematical properties of its design, and the perfec-
tion and accuracy of its construction still invite unscientific explana-
tions. It may have been the scale of the pyramid that contributed to
Khufu's later reputation as a heartless despot, hinted at in Egyptian
literature and reported by Herodotus.
90 J A R O M I R MALEK
The long reigns of Huni, Sneferu, and Khufu and the large number
of royal offspring complicated royal succession. One of them, Khufu's
son Hardjedef, is known from several Egyptian sources. His tomb has
been located at Giza, to the east of the pyramid of his father. Hardjedef
achieved fame as a wise man and supposedly author of a literary work
known as The Instructions of Hardjedef, which continued to be read,
transmitted down on papyri, throughout the rest of Egyptian history.
Kawa, the eldest son of Khufu by his chief queen, Mertiotes, pre-
deceased his father, and so the Egyptian throne passed on to another of
Khufu's sons, probably by a minor queen.
The pyramid of Khufu's immediate successor, Djedefra (Horus
Kheper, 2566-2558 BC), was started at Abu Rawash, to the north-west
of Giza. Another pyramid, at Zawiyet el-Aryan, south of Giza, belongs
to a king whose name, although attested several times in masons'
graffiti, remains uncertain (readings such as Nebka, Baka, Khnumka,
Wehemka, and others have been suggested). Even his place in the 4th
Dynasty is disputed. Djedefra was the first to use the epithet 'son of the
god Ra' and incorporate the name Ra into his own. Both pyramids
were abandoned in the early stages of their construction (although, it
seems, both were used for the intended burial).
King Khafra (Chephren of Herodotus, Horus Weserib, 2558-2532
BC), whose name may alternatively have been pronounced Rakhaef,
was another son of Khufu, and his own son Menkaura (Mycerinus of
Herodotus, Horus Kakhet, 2532-2503 BC) built their pyramids at Giza.
Their plans, measurements, and the choice of building material dif-
fered from those of Khufu and show further development of ideas
associated with such monuments. The ground plan (side 214.5 m -) an d
the height (143.5 m -) of Khafra's pyramid make it the second largest in
Egypt, and a judicious choice of location, on somewhat higher ground
than the pyramid of Khufu, gives the impression that it is its equal.
Khafra's pyramid complex contains a feature not repeated else-
where, a huge guardian statue to the north of the valley temple, close to
the causeway ascending to the pyramid temple and the pyramid. It is a
human-headed lion couchant now known as the Great Sphinx (a
Greek term that may derive from the Egyptian phrase shesep-ankh:
living image'). Its size, some 72 m. long and 20 m. tall, makes it the
largest statue in the ancient world. The Sphinx was not worshipped in
its own right until early in the i8th Dynasty, when it came to be
regarded as the image of a local form of the god Horus (Horemakhet,
Greek Harmachis, Horus on the Horizon). In front of it, though
apparently unconnected with it, was a building constructed according
THE OLD KINGDOM 91
to an unusual plan, with an open court, and this is interpreted as an
early sun-temple. The designation 'son of Ra' now became a standard
part of the royal titulary and both Khafra and Menkaura followed
Djedefra's example in incorporating the name of the sun-god into their
own.
The pyramid of Menkaura shows extensive use of granite, a more
prestigious building material than limestone, but it was built on a
smaller scale (side 105 m. and 65.5 m. in height), suggesting that the
striving for sheer size had passed its peak. It is a precursor of the
smaller and less painstakingly constructed pyramids of the 5th and 6th
Dynasties. The Giza pyramids display a clear relationship in the layout
of the site, but this is more likely due to the techniques used in the
initial surveying than to an overall plan conceived at the outset. A
theory according to which the positions of the pyramids at Giza reflect
the stars of Orion in the sky is unlikely to be correct.
The pyramid complex of Menkaura was apparently hastily com-
pleted by his son and successor, Shepseskaf (Horus Shepseskhet,
2503-2498 BC). He was the only ruler of the Old Kingdom who aban-
doned the pyramidal form, instead constructing a huge sarcophagus-
shaped mastaba at South Saqqara, the base of which measured 100 by
72 m. The monument is known as Mastabat el-Fara e un. Khentkawes,
probably a queen of Menkaura, had a similar tomb at Giza, but a small
pyramid complex was also constructed for her at Abusir. The sig-
nificance of Shepseskaf s move away from a pyramid towards a
sarcophagus-shaped tomb escapes us, and it is tempting to regard it as
a sign of religious uncertainty, if not crisis. The Turin Canon inserts a
reign of two years after Shepseskaf, but the name of the king is lost
(perhaps he is Manetho's Thamphthis) and it has not yet been possible
to confirm it from contemporary monuments. It seems, therefore, that
all of the 4th-Dynasty kings were Sneferu's descendants. The idea of
the son burying his father and succeeding him was ubiquitous in
Egypt, but this was not an absolute precondition for royal succession
and did not automatically confer such a right.
The precise location of White Wall (Ineb-hedj), the capital of Egypt
traditionally founded by King Menes at the beginning of Egyptian
history, has not yet been established. It may have been near the modern
village of Abusir, in the Nile Valley approximately to the north-east of
the pyramid of Djoser. The reasons for the choice of Zawiyet el-Aryan,
Meidum, Dahshur, Saqqara, Giza, and Abu Rawash for the siting of
the pyramids of the 3rd and 4th Dynasties are far from clear. The
location of the royal palaces and the availability of a suitable building
92 J A R O M I R MALEK
site near the pyramid of the King's predecessor may have played a part
in the decision.
Kingship and the Afterlife
For a modern mind, especially one that no longer knows profound
religious experience and deep faith, it is not easy to understand the
reasons for such huge and seemingly wasteful projects as the building
of pyramids. This lack of understanding is reflected in the large
number of esoteric theories about their purpose and origin. The pro-
fusion of these views is helped by an almost complete reticence on the
subject by Egyptian texts.
In ancient Egypt, the king enjoyed a special position as a mediator
between the gods and people, an interface between divine and human,
who was responsible to both. His Horus name identified him with the
hawk-god (of whom he was a manifestation), and his nebty ('two
ladies') name related him to the two tutelary goddesses of Egypt,
Nekhbet and Wadjet. He shared the designation netjer with the gods,
but it was usually qualified as netjer nefer, junior god (although this
could also be understood as perfect god). From the reign of Khafra
onwards, one of his names was introduced by the title 'son of Ra'. The
king had been chosen and approved by the gods and after his death he
retired into their company. Contact with the gods, achieved through
ritual, was his prerogative, although for practical purposes the more
mundane elements were delegated to priests. For the people of Egypt,
their king was a guarantor of the continued orderly running of their
world: the regular change of the seasons, the return of the annual
inundation of the Nile, and the predictable movements of the heavenly
bodies, but also safety from the threatening forces of nature as well as
enemies outside Egypt's borders. The king's efficacy in fulfilling these
responsibilities was therefore of paramount importance for the well-
being of every Egyptian. Internal dissent was minimal, and support for
the system was genuine and widespread. Coercive state mechanisms,
such as police, were conspicuous by their absence; people were tied to
the land and control over every individual was exercised by local com-
munities who were closed to newcomers.
The king's role did not end with his death: for his contemporaries
who were buried in the vicinity of his pyramid and for those involved in
his funerary cult their relationship with the king continued for ever. It
was, therefore, in everybody's interests to safeguard the king's position
and status after his death as much as in his lifetime. At this period of
THE OLD K I N G D O M 93
Egyptian history, monumentality was an important way of expressing
such a concept. Given the degree of economic prosperity enjoyed by
the country, the availability of labour-force resources, and the high
standard of management, there is no need to doubt that the Egyptians
were perfectly capable of successfully completing pyramid projects. To
look for extraneous motives and forces behind them is futile and
unnecessary.
The tombs of the members of the royal family, priests, and officials
of the 3rd Dynasty were separated from the exclusive areas with the
royal pyramids. Almost all of these tombs continued to be built in mud
brick, although early examples of private mastaba-tombs in stone may
exist at Saqqara. However, in the 4th Dynasty such tombs, now stone-
built, surrounded the pyramids, as if the tombs themselves were part
of the complexes (and this, indeed, is how they may have been per-
ceived). Because many of them were gifts from the king and made by
royal craftsmen and artists, the volume of royal building activities was
even larger than suggested by the pyramids alone. Extensive fields of
mastaba-tombs built according to an overall plan, separated by streets
intersecting at right angles, are unique to the 4th Dynasty and are
especially known from around the pyramid at Meidum, Sneferu's
northern pyramid at Dahshur, and Khufu's pyramid at Giza. We must
not forget that most of the evidence used in our reconstruction of the
history of the Old Kingdom derives from funerary contexts and so
carries a possibility of being biased; Old Kingdom settlements have
rarely been preserved or excavated (the towns at Elephantine and Ayn
Asil being unusual survivals). The state of technology can be deduced
from the projects in which it was applied, but detailed information is
lacking. So, for example, only post-Old Kingdom sources make it quite
clear that the pyramid-builders did not use wheeled vehicles (although
the wheel was known).
The Old Kingdom Economy and Administration
The enormous volume of construction work carried out during the two
centuries when the kings of Manetho's 3rd and 4th Dynasties held
sway had a profound effect on the country's economy and society. It
would be wrong to underestimate the considerable effort and expertise
required in the construction of large brick-built mastaba-tombs of the
Early Dynastic Period, but pyramid construction in stone elevated such
enterprises onto a completely different plane. The number of profes-
sional builders required must have been large, especially if one takes
94 J A R O M I R MALEK
into account all those involved in the quarrying and transport of stone
blocks, the construction of approach ramps needed by the builders,
and all the logistics, such as provision of food, water, and other neces-
sities, the maintenance of tools and many other related tasks.
The Egyptian economy was not based on slave labour. Even if one
allows for much of the work to have been carried out at the time when
the annual inundation made it impossible to work in the fields, a large
section of the labour force required for pyramid building had to be
diverted from agricultural tasks and food production. This must have
exerted considerable pressure on the existing resources and provided
powerful stimuli for efforts to increase agricultural production, to
improve the administration of the country, to develop an efficient way
of collecting taxes, and to look for additional sources of revenue and
manpower abroad.
Demands on Egyptian agricultural production changed dramatic-
ally with the inauguration of pyramid building because of the need to
support those who had been removed from food production. The
consumption and expectations of those who joined the managerial
elite increased in line with their new status. However, agricultural
techniques remained the same. The state's main contribution was
organizational, including such acts as the prevention of local famines
by bringing in surplus resources from elsewhere, the lessening of the
effects of major calamities (such as low inundations), the elimination
of damaging local conflicts by providing arbitration, and the improve-
ment of security. Irrigation works were the responsibility of local
administrators, and the attempts to increase agricultural production
focused on expanding cultivated land for which the state was able to
provide labour forces and other resources.
This went hand in hand with the need for a better administrative
organization of the country and a more efficient way of collecting taxes.
The existing major centres of population, often royal estates, now
became capitals of administrative districts (nomes), with the strategic-
ally placed capital of the country, at the vertex of the Delta, providing
the equilibrium between Upper Egypt (ta shemau) in the south, and
Lower Egypt (ta mehu) in the north. Old Kingdom cities are, however,
overlaid by later settlements and, especially in the Delta, they often lie
below the present water-table. These early settlements are therefore
archaeologically practically unknown; even the capital of Egypt has not
yet been excavated, and towns such as Elephantine, or Ayn Asil in the
Dakhla Oasis, are exceptional. The earlier semi-autonomous village
communities now lost their independence and privately owned land
THE OLD K I N G D O M 95
practically disappeared, all replaced by royal estates. The earlier rudi-
mentary census was transformed into an all-embracing fiscal system.
Egypt during much of the Old Kingdom was a centrally planned and
administered state, headed by a king who was the theoretical owner of
all its resources and whose powers were practically absolute. He was
able to commandeer people, to impose compulsory labour, to extract
taxes, and to lay claim to any resources of the land at will, although in
practical terms this was tempered by a number of restrictions. During
the 3rd and 4th Dynasties, many of the top officials of state were
members of the royal family, in direct continuation of the system of
government of the Early Dynastic Period. Their authority derived from
their close links with the king. The highest office was that of a vizier
(the word conventionally used to translate the Egyptian term tjaty),
who was responsible for overseeing the running of all state depart-
ments, excluding the religious affairs. It was under the kings of the 4th
Dynasty that a whole series of royal princes held the vizierate with
spectacular success.
Titles of various officials represent a major source of information on
Egyptian administration. Explicit, detailed texts, such as that of the
early 4th-Dynasty official Metjen, were exceptional. The intensity of
state control over every individual now increased dramatically and the
number of officials at all levels of administration grew in a corres-
ponding fashion. A consequence of this was that a bureaucratic career
was open to competent literate newcomers not related to the royal
family. These officials were remunerated for their services in several
different ways, but the most significant was an ex qfficio lease of state
(royal) land, usually estates settled with their cultivators. Such estates
produced practically all that their personnel needed—internal trade at
this economic level was limited to opportunist bartering—and the ex
offido remuneration was their surplus produce. This land reverted, at
least in theory, to the king after the official's term of office expired and
so could be assigned as remuneration of another official. In an eco-
nomic system that did not know money it was a very effective way of
paying salaries of officials, but it also represented a significant erosion
of the king's resources.
Royal Funerary Cults
The effect of pyramid building did not stop with the completion of the
structure itself. Each pyramid complex was the focus of the cult of
a deceased king that was meant to continue indefinitely. Its aim was
96 JAROMIR MALEK
to provide for the king's needs and, less directly, those of his
dependants—that is, members of his family and his officials buried in
the tombs nearby. The primary benefactor was the king himself, who,
in his lifetime, endowed his pyramid establishment with land or made
arrangements for contributions from the state treasury. The cult
arrangements involved presentations of offerings, although it is likely
that only a small part of the produce available to these establishments
ended on their altars and offering tables (and even this was probably
not wasted but recycled, being either consumed by the temple person-
nel or redistributed more widely). Most of it was used to support
priests and officials involved in the funerary cult, and craftsmen living
in the pyramid town, or else it was redirected to support funerary cults
in non-royal tombs. This was a distinctive ancient Egyptian way of
redistributing the national produce, and its benefits trickled down
through all the strata of Egyptian society. However, land donations
made to pyramid establishments were protected for ever by royal
decrees that made them permanent and inalienable, and the result was
a gradual reduction of the king's economic power.
Arrangements for the royal funerary cult were made even in the
provinces. Sneferu's cult may have focused on a number of small step
pyramids, each with a ground plan of 0.20 sq. m., at least seven of
which are known (at Elephantine, Edfu, el-Kula, Ombos, Abydos, el-
Seila, and Zawiyet el-Mayitin). Only one of them, at el-Seila, can be
dated with precision to the reign of Sneferu by a stele and a statue.
Large building projects also provided stimuli for expeditions that
were sent abroad to secure mineral and other resources not available
in Egypt itself. These were state organized: no other form of long-
distance trade was known before the 6th Dynasty. The names of Djoser,
Sekhemkhet, Sneferu, and Khufu are attested in rock inscriptions at
the turquoise and copper mines of Wadi Maghara in the Sinai penin-
sula. Djoser may have been preceded there by Nebka, if this is the same
king as Horus Sanakht. The Palermo Stone contains a record of forty
ships that brought wood from an unnamed region abroad in the reign
of Sneferu. The names of both Khufu and Djedefra were inscribed in
the gneiss quarries deep in the Nubian Western Desert, 65 km. to the
north-west of Abu Simbel. Greywacke and siltstone for the making of
statues came from Wadi Hammamat, between Koptos (modern Qift)
and the Red Sea. Commerce or diplomacy probably explain the
presence of Egyptian objects at Byblos, north of Beirut, in the reigns of
Khufu, Khafra, and Menkaura, and also at Tell Mardikh (Ebla) in Syria,
in the time of Khafra.
THE OLD K I N G D O M 97
No serious threat to Egypt from abroad existed during the }rd and 4th
Dynasties. Military campaigns in foreign countries, especially in Nubia
and Libya, must be perceived as exploitation of the neighbouring areas
in search of ready resources. It was one of the main duties of the
Egyptian king to subjugate Egypt's external enemies, and the kingship
doctrine and realpolitik here conveniently coincided. Most evidence
comes from the reign of Sneferu, but this probably was not unique, only
better documented. Such crude forms of external policy seem to have
been particularly common during the 4th Dynasty when the country's
economy was probably stretched to its limits. Nubia was the destination
of a large expedition sent by Sneferu in search of such resources as
human captives and herds of cattle, as well as raw materials, including
wood. The Palermo Stone records a booty of 7,000 captives and
200,000 head of cattle. These campaigns destroyed local settlements
and depopulated Lower Nubia (between the ist and 2nd Nile cataracts),
apparently resulting in the disappearance of the local culture known as
the A Group (see Chapter 4). During the 4th Dynasty, a southern settle-
ment was established at Buhen, in the second-cataract area.
Monumental building provided unprecedented opportunities for
artists, especially those making statues and carving reliefs. The experi-
ence in small-scale working in stone acquired during the preceding
periods was turned to large-scale sculpture, with brilliant results. Royal
pyramid complexes were provided with statues, mostly of the king,
sometimes accompanied by deities. Although for us their aesthetic
qualities are so striking, these works of art were, in the first instance,
functional. Thus, the earliest preserved large royal statue, that of
Djoser, was found in his pyramid temple at Saqqara. It was placed in
his serdab ('statue-room', from the Arabic word for cellar), at the
northern side of the pyramid, and was intended as a secondary mani-
festation of the king's ka (spirit), after the body itself. A similar motive
must be ascribed to tomb statues made for private individuals.
The number of royal statues set up in temples increased when the
developed pyramid complex appeared during the 4th Dynasty. The
gneiss statue of Khafra, protected by a hawk (perched on the back of his
throne as a manifestation of the god Horus, with whom the king was
identified), is a masterpiece that was often imitated in later periods,
but never equalled. Statues of gods were also presented to the temples
of local deities, but hardly any of these have survived.
The temples and causeways associated with pyramids were decor-
ated in superb raised relief, and the same was true of the chapels of
many tombs from the mid-4th Dynasty. These reliefs were not mere
98 J A R O M I R  MALEK
decoration but expressed concepts such as kingship in royal monu-
ments, or fulfilment of needs in the afterlife in non-royal tombs, and
their inclusion in temples and tombs guaranteed their perpetuity. The
wooden niche stelae from the tomb of Djoser's official Hesyra at
Saqqara (now in the Egyptian Museum, Cairo) display a high standard
of relief decoration at a remarkably early period. These reliefs were
created by the same artists who decorated royal monuments and, like
the tombs and their statues, were royal gifts.
The hieroglyphic script now became a fully developed system
employed for monumental purposes. Its cursive counterpart, called
hieratic by Egyptologists, was used for writing on papyrus, but finds of
such documents dating from before the 5th Dynasty remain extremely
scarce.
Sun-Temples and the Ascendancy of the God Ra
Until quite recently, the rise of Manetho's 5th Dynasty used to be
described in terms of a literary text set out in Papyrus Westcar. This is an
incompletely preserved collection of stories, probably compiled during
the Middle Kingdom and written down a little later. The Arabian Nights
setting is the court of King Khufu, where royal princes entertain their
fretful father by stories. Prince Hardjedef s narrative foretells the birth
of triplets, the future kings Userkaf, Sahura, and Neferirkara, to
Radjedet, the wife of a priest of the god Ra at Sakhbu (in the Delta) as the
result of her union with the sun-god. To Khufu's sorrow, these children
are destined to replace his own descendants on the throne of Egypt. The
beginning of Manetho's new Dynasty, the 5th, appears to be linked to a
major change in Egyptian religion and, as Papyrus Westcar shows, the
division may reflect ancient Egyptian tradition.
The first king of the new Dynasty was Userkaf (Horus Irmaet,
2494-2487 BC), whose name is of the same pattern as that of the last
(or perhaps penultimate) king of the 4th Dynasty, Shepseskaf. It has
been suggested that Userkaf was a grandson of Djedefra, but, although
there were undoubtedly some family links between him and the rulers
of the 4th Dynasty, their precise nature is uncertain. We know nothing
about the history of Userkaf s reign and there is no contemporary
evidence to support the version of events given in Papyrus Westcar.
The main surviving architectural achievement of Userkaf s reign
was the building of a temple specifically dedicated to the sun-god Ra.
This was the beginning of a trend; six of the first seven kings of
Manetho's 5th Dynasty (Userkaf, Sahura, Neferirkara, Raneferef,
THE OLD K I N G D O M 99
Nyuserra, and Menkauhor) built such temples in the next eighty years.
The names of these temples are known from the titularies of their
priests, but only two have so far been located and excavated, those of
Userkaf and Nyuserra. The sun-temple built by Userkaf is at Abusir,
north of Saqqara (although it seems that current excavations confirm
the view that the division between Saqqara and Abusir has been
created by modern archaeologists and was not felt to exist in antiquity).
Userkaf s pyramid is at North Saqqara, close to the north-eastern
corner of Djoser's enclosure. A substantial re-evaluation of rigid monu-
mentality had taken place by this time, judging from the pyramid's
small size (side 73.5 m. and height 49 m.), the less painstaking method
of construction, and the evident willingness to improvise (the main
pyramid temple is, unusually, set against the southern face of the
pyramid, perhaps in order not to interfere with an already existing
structure). Userkaf, whose reign lasted for only seven years, may have
come to the throne as an old man.
The building of sun-temples was the outcome of a gradual rise in
importance of the sun-god. Ra now became Egypt's closest equivalent
to a state god. Each king built a new sun-temple and their proximity to
the pyramid complexes, as well as their similarity to the royal funerary
monuments in plan, suggest that they were built for the afterlife rather
than the present. A sun-temple consisted of a valley temple linked by a
causeway to the upper temple. The main feature of the upper temple
was a massive pedestal with an obelisk, a symbol of the sun-god. An
altar was placed in a court open to the sun. There were no wall reliefs in
Userkaf s construction, the earliest of the sun-temples, but in Nyu-
serra's they were extensive. On the one hand, they emphasized the
sun-god's role as the ultimate giver of life and the moving force in
nature, and, on the other, they established the king's place in the
eternal cycle of events by showing his periodic celebration of the sed-
festivals. A large mud-brick replica of a barque of the sun-god was built
nearby. The temples were, therefore, personal monuments to each
king's continued relationship with the sun-god in the afterlife. Like
pyramid complexes, sun-temples were endowed with land, received
donations in kind on festival days, and had their own personnel.
The 5th Dynasty
The explanation of the origins of the 5th Dynasty given in Papyrus
Westcar can be confronted by evidence contemporary with the reigns
of Sahura and Neferirkara. Queen Khentkawes is identified by a unique
100 J A R O M I R MALEK
title in her mastaba-tomb at Giza: 'mother of the two kings of Upper
and Lower Egypt'. The same title is known from her pyramid (recently
discovered by Czech archaeologists), which is situated next to Neferir-
kara's pyramid at Abusir. If the Giza Khentkawes and the Abusir
Khentkawes are the same person, the two sons referred to in her title
were Sahura (Horus Nebkhau, 2487-2475 BC) and Neferirkara (Kakai,
Horus Userkhau, 2475-2455 BC), and Papyrus Westcar is partly
correct. The pyramids of these two kings are at Abusir, as are the
pyramids of all the kings who built sun-temples (and probably also that
of Shepseskara, 2455-2448 BC). The causeway linking the valley and
pyramid temples of Sahura's pyramid complex was decorated with
very accomplished reliefs which anticipated the better-known reliefs of
King Unas (2375-2345 BC). These Abusir kings form a closely knit
group and their monuments display many similarities.
The pyramid temple of Neferirkara has yielded the most important
group of administrative papyri known from the Old Kingdom. These
documents throw light on the day-to-day running of the pyramid
establishment and include detailed records of produce delivered to it,
lists of priests on duty, inventories of temple equipment, and letters.
The pyramid complex, however, was left unfinished and its valley
temple and causeway were later incorporated by Nyuserra into his own
pyramid complex.
King Shepseskara (Horus Sekhemkhau, 2455-2448 BC) is the most
ephemeral of the Abusir group, and no textual or archaeological evi-
dence for his sun-temple has yet been found. This is probably due to
the brevity of his reign. That of King Raneferef (Isi, Horus Neferkhau,
2448-2445 BC) was even shorter. Although his pyramid did not
progress beyond its lowermost courses, the pyramid temple has
recently produced papyri comparable to those found in the temple
of Neferirkara.
The sun-temple of King Nyuserra (Iny, Horus Setibtawy, 2445-
2421 BC) is at Abu Gurab, north of Abusir. The last king who built a
sun-temple was Menkauhor (Ikauhor, Horus Menkhau, 2421-2414
BC). His pyramid has not yet been located, but the tombs of its priests
and other indications suggest that it may be concealed by the sand
somewhere at southern Abusir or North Saqqara.
The most striking development in Egyptian administration during
this period was the withdrawal of members of the royal family from the
highest offices. Another noteworthy feature was the skilful way in
which sun-temples were incorporated into the economic system. Some
of the appointments to the priesthood in sun-temples were purely
THE OLD K I N G D O M IOI
nominal and made in order to entitle their holder to benefits derived
from such offices; these may have included temple land leased ex
qfficio. The same was true of appointments to the personnel of pyramid
establishments. There was no glaring contradiction between the
demands of the world of the gods and the dead, and the needs of the
living. One could well visualize a system where most of the national
product would, in theory, be earmarked for the needs of the deceased
kings, their sun-temples, and shrines of the local gods, but would, in
fact, be used to support most of the Egyptian population.
Religious beliefs of the ancient Egyptians were locally diverse and
socially stratified. Practically every area of Egypt had its local god,
which for its inhabitants was the most important deity, and the eleva-
tion of Ra to the level of state god had little effect on this. If anything,
the annals show that the kings now began to pay even greater attention
to local deities in all parts of the country by making donations, often of
land, to their shrines, or exempting them from taxes and forced labour.
Expeditions continued to be dispatched to the traditional places out-
side Egypt, especially to bring turquoise and copper from Wadi Mag-
hara (Sahura, Nyuserra, Menkauhor) and Wadi Kharit (Sahura) in the
Sinai, and gneiss from the quarries north-west of Abu Simbel (Sahura).
During the reign of Sahura, there is a reference to an expedition to
procure exotic goods (malachite, myrrh, and electrum, an alloy of gold
and silver) from Punt, an African country somewhere between the
upper reaches of the Nile and the Somali coast. Contacts with Byblos
were maintained (Sahura, Nyuserra, Neferirkara). The discovery of
objects bearing the names of several 5th-Dynasty kings at the site of
Dorak, near the Sea of Marmara, remains ambiguous.
During the 5th Dynasty there was an increase in the number of
priests and officials who were able to secure tombs by their own effort.
Some of these mastabas are among the largest and best decorated in
the Old Kingdom, as in the case of the tombs of Ty (Saqqara) and
Ptahshepses (Abusir), both probably of the reign of Nyuserra. Many of
them are located in provincial cemeteries rather than in the vicinity of
the royal pyramids. Such loosening of the dependence on royal favour
was, inevitably, accompanied by a corresponding variety in the forms
and quality of artistic quality of statues and reliefs. 'Autobiographical'
texts that appeared in these tombs provide new insights into con-
temporary society. Most of them consisted of conventional phrases
and less usual topics often concerned with the tomb-owner's relation-
ship to the king. These trends were to continue throughout the rest of
the Old Kingdom.
102 J A R O M I R MALEK
The Kings of the Pyramid Texts
The portents of change were in the wind after the death of Menkauhor,
but the nuances of the process escape us. A degree of standardization
and rationalization pervaded royal building activities. Menkauhor's
successors did not build sun-temples, although the position of the sun-
god Ra remained unaffected. The long reign of King Djedkara (Isesi,
Horus Djedkhau, 2414-2375 BC) links the Abusir group of kings with
those who followed. Some of his officials were buried in the Abusir
necropolis, and so attest to continuity rather than a break, but the
king's pyramid is at southern Saqqara. Its modest measurements (side
78.5 sq. m., height 52.5 m.) were, with the exception of his immediate
successor Unas, adopted by all the remaining major rulers of the
Old Kingdom (Teti, Pepy I, Merenra, and Pepy II). The Maxims of
Ptahhotep, a major literary work of the Old Kingdom, which sum-
marizes the rules of conduct of a successful official, is ascribed to the
vizier of Djedkara.
The reign of King Unas (Horus Wadj-tawy, 2375-2345 BC) was also a
long one. His pyramid is at the south-western corner of Djoser's
enclosure, but it is even smaller than that of his predecessor. Its long
causeway, stretching for nearly 700 m., was originally decorated with
remarkable scenes, now very fragmentary, which surpass the stereo-
typed means of expression of Egyptian kingship, or at least convey it in
a novel way. They include records of events in Unas's reign, such as
transport of columns from the granite quarries at Aswan to the king's
pyramid complex. But the main innovation of Unas's pyramid, and
one that was to be characteristic of the remaining pyramids of the Old
Kingdom (including some of the queens), was the first appearance of
the Pyramid Texts inscribed on the walls of its burial chamber and
other parts of its interior. The Pyramid Texts represent the earliest
large religious composition known from ancient Egypt; some of their
elements were created well before the reign of Unas and map out the
development of Egyptian religious thought from Predynastic times.
The deceased King Unas is identified with the god Osiris and referred
to as Osiris Unas. The Osirian religious doctrine is by far the most
important in the Pyramid Texts, but there are also ideas associated
with the sun-god, as well as the remains of star-oriented concepts and
some others, probably even older. However, the complexity of the
Pyramid Texts makes interpretation of individual spells difficult, and
understanding of their mutual relationship is especially hard. The
reason for their inclusion inside the pyramid was to provide the
THE OLD K I N G D O M 103
deceased king with texts that were regarded as essential for his survival
and well-being in afterlife. Their mere presence was probably deemed
sufficient to make them effective. While the distribution of the Pyra-
mid Texts within the structure is not accidental, it is unlikely that they
are connected with such a transient event as a funeral.
The belief that after death the deceased entered the kingdom of the
god Osiris now became widespread. Osiris, originally a local deity in
the Eastern Delta, was a chthonic (linked to the earth) local god asso-
ciated with agriculture and annually recurring events in nature. He
was probably an ideal choice for the universal god of the dead, given
that the myths concerning his resurrection mirrored the revitalization
of Egyptian soil after the annual flood receded (which used to happen
until the building of a dam at Aswan at the beginning of this century
and the High Dam in the 19605). The early stages of the development
of the cult of Osiris are far from clear. He was an appropriate counter-
part for the sun-god Ra and his rise to prominence may have been
caused by corresponding considerations. Our written records are,
however, inadequate to establish exactly when this happened. In their
tombs, deceased persons are described as imakhu ('honoured') by
Osiris: in other words, their needs in afterlife were satisfied because of
their association with him. The concept of imakhu (which can also be
translated as 'being provided for') was an expression of a remarkable
moral dictum that ran through all levels of Egyptian society and that
corrected the extreme cases of social inequality: it was the duty of a
more influential and richer person to take care of the poor and socially
disadvantaged in the same way as the head of a family was responsible
for all of its members.
The 6th Dynasty
According to Manetho, the reign of Unas concluded the 5th Dynasty,
and the next king, Teti (Horus Seheteptawy, 2345-2323 BC), ushered in
the 6th Dynasty. We have no definite information on the personal
relationship between Teti and his predecessors, but his chief wife Iput
was probably Unas's daughter. Teti's vizier Kagemni began his career
under Djedkara and Unas. However, the Turin Canon also inserts a
break at this point followed by a total for the kings between Menes (the
first king of the ist Dynasty) and Unas (the figure is now lost). This
gives us some food for thought, because the criterion for such divi-
sions in the Turin Canon invariably was the change of location of the
capital and royal residence.
104 J A R O M I R MALEK
The original capital at White Wall, founded at the beginning of the
ist Dynasty, was probably gradually replaced in importance by the
more populated suburbs further to the south, approximately to the east
of Teti's pyramid. Djed-isut, the name of this part of the city, derived
from the name of Teti's pyramid and its pyramid town. The royal
palaces of Djedkara and Pepy I (and possibly also that of Unas) may,
however, have already been transferred further south, away from the
squalor, noise, and smell of a crowded city, to places in the valley east
of the present South Saqqara and separated from Djed-isut by a lake.
This would, at least, explain the choice of South Saqqara as the site for
the pyramids of Djedkara and Pepy I.
In a development that paralleled that near the pyramid of Teti, the
adjacent settlement took its name Mennefer (Greek Memphis) from
the name of Pepy I's pyramid and its pyramid town. Later, in the
second millennium, this became physically linked with the settle-
ments around the temple of the god Ptah further to the east, and the
city in its entirety began to be known as Mennefer. So, to some extent,
the site of the royal residence may have changed at the end of the 5th or
early in the 6th Dynasty and this may explain the division in the Turin
Canon, later reflected in Manetho's account (Pepy I's father Teti was
included in the new line of rulers). But here we are entering a realm of
speculation and only future fieldwork in the Memphite region will
show how much of it is justified.
Teti may have been followed by King Userkara (2323-2321 BC),
although his existence can be disputed. Some confusion may be due to
the fact that Pepy I (Horus Merytawy, 2321-2287 BC), the son of Teti
and Queen Iput, was called Nefersahor in the first part of his reign.
This was his 'prenomen' or 'throne name', received at his coronation,
preceded by the title nesu-bit ('he of the sedge and bee') and enclosed in
an oval cartouche. Later he changed it to Meryra. The 'nomen' or 'birth
name', Pepy (the number that conventionally follows is ours, and was
never used by the ancient Egyptians), predated his accession to the
throne; it was introduced by the title sa Ra ('son of the god Ra') and was
also written in a cartouche.
Egypt's internal situation now began to change. The king's position
remained theoretically unaffected, but there can be no doubt that diffi-
culties appeared. This impression can be only partly explained by the
increase in the volume and quality of information that allows us a
deeper insight into Egyptian society, beyond the monolithically monu-
mental and largely formal facade of the earlier periods. The king's
person was no longer untouchable: a biographical text of Weni, a high
THE OLD K I N G D O M 105
court official, mentions an unsuccessful plot against Pepy I inspired by
one of his queens late in his reign. Her name is not given, but marriage
politics were known: in his declining years, the king married two
sisters, both called Ankhnes-meryra ('King Meryra [Pepy I] lives for
her'). Their father Khui was an influential official at Abydos. These
were dramatic events, but the growth of power and influence of local
administrators (especially in Upper Egypt, further away from the
capital) and the corresponding weakening of the royal authority may
have had less dramatic, but potentially much more serious, conse-
quences. A new office of'overseer of Upper Egypt' was created late in
the 5th Dynasty.
The kings of the 6th Dynasty built extensively, constructing shrines
of local gods all over Egypt, but these fell victim to later rebuilding or
have not yet been excavated. Upper Egyptian temples, such as those of
Khenti-amentiu at Abydos, Min at Koptos, Hathor at Dendera, Horus
at Hierakonpolis, and Satet at Elephantine, were especially favoured.
Donations made to these temples and exemptions from taxes and
compulsory service granted to them multiplied.
The pyramid temples of the late 5th and 6th Dynasties include
scenes that appear so convincing that one might be tempted to take
them at face value. So, for example, a scene showing the submission of
Libyan chiefs during the reign of Pepy II is a close copy of such a
representation in the temples of Sahura, Nyuserra, and Pepy I (and,
some 1,500 years later, it was repeated in the temple of King Taharqo at
Kawa, in Sudan). These scenes were standard expressions of the
achievements of the ideal king and as such bore little resemblance to
reality. Their inclusion in the temples guaranteed their continuity. The
same explanation may be given to the scenes of ships returning from
an expedition to Asia and a raid on the nomads of Palestine, depicted
in the causeway of Unas. However, other sources show that similar
events did take place. The already mentioned Weni describes repeated
large-scale military actions against the Aamu of the Syro-Palestinian
region. In spite of the way they were presented, they were preventative
or punitive raids rather than defensive campaigns.
The exploitation of mineral resources in the deserts outside Egypt
continued. Turquoise and copper continued to be mined at Wadi
Maghara in the Sinai (Djedkara, Pepy I and II), Egyptian alabaster at
Hatnub (Teti, Merenra, Pepy I and II), and greywacke and siltstone in
the Wadi Hammamat (Pepy I, Merenra) in the Eastern Desert, gneiss
in the quarries north-west of Abu Simbel (Djedkara). Expeditions were
sent to Punt by Djedkara, and commercial and diplomatic contacts
I06 J A R O M I R  MALEK
were maintained with Byblos (by Djedkara, Unas, Teti, Pepy I and II,
and Merenra) and also with Ebla (Pepy I).
Nubia became particularly important during the later 6th Dynasty
and attempts were made to improve navigation in the first-cataract
region in the time of Merenra. The area now began to receive an influx
of new settlers (the so-called Nubian C Group) from further south,
between the 3rd and 4th cataracts, with the centre at Kerma. There
were occasional clashes with these people as Egypt tried to prevent a
potential threat to its economic interests and its security. Caravan
expeditions across the Nubian territory (the lands of Wawat, Irtjet,
Satju, and lam) were organized by administrators of the southernmost
Egyptian nome at Elephantine, such as Harkhuf, Pepynakht Heqaib,
and Sabni. African luxury goods that reached Egypt this way included
incense, hard wood (ebony), animal skins, and ivory, but also dancing
dwarfs and exotic animals. The employment of Nubians, especially in
border police units and as mercenaries in military expeditions, dates
from this period onwards.
The Western Desert was criss-crossed by caravan routes. One of
them left the Nile in the area of Abydos for the Kharga Oasis and then
proceeded southwards along the track now known as Darb el-Arbain
(Arabic: 'forty-day route') to the Selima Oasis. Another departed from
Kharga westwards, to the Dakhla Oasis, where an important settle-
ment thrived at Ayn Asil, near modern Balat, especially during the
reign of Pepy II.
The Decline of the Old Kingdom
Pepy I was succeeded by two of his sons, first by Merenra (fully
Merenra-nemtyemsaf, Horus Ankh-khau, 2287-2278 BC), and then by
Pepy II (Horus Netjerkhau, 2278-2184 BC). Both of them came to the
throne very young and both built their pyramids at South Saqqara.
Pepy IFs reign of some ninety-four years (he inherited the throne at
the age of 6) was the longest in ancient Egypt, but its second half
probably was rather ineffective, as the forces that had been insidiously
eroding the theoretical foundations of the Egyptian state became
apparent. The ensuing crisis was inevitable, because its seeds were
contained in the system itself. It was, in the first instance, ideological,
because the king whose economic power had been greatly weakened
could no longer perform the role assigned to him by the doctrine of
Egyptian kingship. The consequences of this for the whole of Egyptian
society were serious; the ex officio system of remuneration no longer
THE OLD K I N G D O M 107
functioned satisfactorily and the fiscal system was probably on the
verge of collapse.
Some offices became, in effect, hereditary and were kept in the same
family for several generations. In Middle and Upper Egypt, rock-cut
tombs at sites such as Sedment, Dishasha, Kom el-Ahmar Sawaris,
Sheikh Said, Meir, Deir el-Gebrawi, Akhmim (el-Hawawish), el-
Hagarsa, el-Qasr wa C l-Saiyad, Elkab, and Aswan (Qubbet el-Hawa)
testify to the aspirations of the local administrators, now would-be
semi-independent local rulers. We know less about the corresponding
cemeteries in the Delta, although sites such as Heliopolis and Mendes
prove that they existed. The proximity of the capital may have made any
moves towards increased autonomy more difficult, but the main
reason for the lack of evidence is local geography and geology. Old
Kingdom levels are close to or below the current water table and this
makes excavations very difficult. We know much more about the local
administrators of Dakhla Oasis who lived in the settlement of Ayn Asil
and were buried in large mastaba-tombs in the local cemetery (Qilat el-
Dabba).
Centralized government all but ceased to exist, and the advantages
of a unified state were lost. The situation was further aggravated by
climatic factors, especially a series of low Niles and a decline in pre-
cipitation that affected areas adjacent to the Nile Valley and produced
pressure on Egypt's border areas by nomadic inhabitants. The fact that
many potential royal successors were waiting in the wings after Pepy
II's exceptionally long reign probably contributed to the chaotic situa-
tion that followed.
Pepy II was succeeded by Merenra II (Nemtyemsaf), Queen Nitiqret
(2184-2181 BC), and some seventeen or more ephemeral kings who
represent Manetho's yth and 8th Dynasties. His dynastic separations
are, again, hard to explain except as accidental divisions in the lists.
Most of these rulers are little more than names for us, but several of
them are known from the protective decrees issued for the temple of
Min at Koptos. Qakara Iby is the only one whose small pyramid (side
31.5 sq. m.) has been found at South Saqqara. So it was mainly the
Memphite residence and the theoretical claim to the whole of Egypt
that linked these kinglets with the giant kings of the earlier Old King-
dom. The Turin Canon's grand total of 955 years that separated Menes,
at the beginning of the ist Dynasty, from the last of these ephemeral
rulers, concludes the line of Memphite kings and the period described
by us as the Old Kingdom.
6
The First Intermediate Period
(c.2160-2055 BC)
STEPHAN SEIDLMAYER
Egyptologists traditionally distinguish between the major periods of
pharaonic history on the basis of the political state of the country.
'Kingdoms'—defined as times of political unity and strong, central-
ized government—alternate with 'intermediate periods', which are in
contrast characterized by the rivalries of local rulers in their claims for
power. In the case of the First Intermediate Period, the long line of
kings who had ruled the country from Memphis ended with the last
pharaohs of the 8th Dynasty. After the 8th Dynasty power was held by
a succession of rulers originating from Herakleopolis Magna, which
was located in northern Middle Egypt, near the entrance to the Faiy-
um. These kings appear as both the 9th and loth Dynasties in Man-
etho's history, having been mistakenly subdivided in the course of the
transmission of the original king-list (see Chapter i for a discussion of
Manetho's Aegyptiaca).
The shift of the royal residence from Memphis to Herakleopolis was
evidently regarded by the ancient Egyptians as a major break. This is
suggested by the fact that the compilers of the 19th-Dynasty Turin
Canon inserted a grand total for the earlier part of Egyptian history
after the list of 8th-Dynasty rulers. In addition, the king-list in the
temple of Seti I at Abydos gives no royal names for the period between
the 8th Dynasty and the beginning of the Middle Kingdom.
In fact, the Herakleopolitans never wielded control over southern
Upper Egypt. Here, in the course of prolonged struggles between local
magnates, a family of Theban nomarchs established itself as the
THE FIRST I N T E R M E D I A T E P E R I O D 109
leading force, assumed the titles of royalty, and duly appeared in the
annals of pharaonic kingship as the nth Dynasty. From this moment
onwards, two competing states confronted each other within the terri-
tory of Egypt, until, terminating an era of intermittent war, the Theban
king Nebhepetra Mentuhotep II managed to defeat his Herakleo-
politan opponent and reunite the country under Theban control, thus
inaugurating the Middle Kingdom. This chapter therefore deals with
the period between the end of the 8th Dynasty and the reign of
Nebhepetra Mentuhotep II.
Chronological Problems
We are comparatively well informed concerning the second part of
the First Intermediate Period—the phase of competition between
Herakleopolitans and Thebans, which lasted for some 90 to no years.
However, the earlier part of the period—the phase of Herakleopolitan
rule before the advent of the nth Dynasty—is rather less clear. There is
a dearth of information of immediate chronological value because of
the loss of most of the names of the Herakleopolitans and of all infor-
mation concerning the lengths of their reigns in the Turin Canon, and
because of the unsatisfactory state of archaeological research in north-
ern Middle Egypt and the Delta, the heartlands of the Herakleopolitan
kingdom. Because of the scarcity of data directly relating to the
Herakleopolitans, it was even at one stage proposed that there must
have been no period during which Herakleopolitans were (at least
nominally) the sole rulers, and that they must have been entirely coeval
with the nth Dynasty. This is impossible, however, since we know of
prominent individuals and important political events that can be
placed only in the period between the 8th and nth Dynasties.
Detailed studies of the succession of holders of important adminis-
trative and priestly posts in several towns of Upper Egypt, as well as
studies of developments in the archaeological material, strongly sug-
gest that this interval between the 8th and nth Dynasties occupied a
fairly long span of time, probably amounting to some three or four
generations. In addition, the figure that Manetho reports as the length
of his loth Dynasty can be interpreted to support an estimate of nearly
two centuries for the whole of the First Intermediate Period, an assess-
ment that would also be in perfect agreement with the prosopo-
graphical and archaeological evidence.
110 STEPHAN S E I D L M A Y E R
The Nature of the First Intermediate Period
The First Intermediate Period, however, was not just a time of disorder
in terms of the succession to the throne of Egypt; it was also a period of
crisis and of new developments, both of which deeply affected the
whole of Egyptian society and culture. This point can be appreciated as
soon as we turn to the evidence of the monuments. The Old Kingdom
mortuary complexes of the kings and the highest officials in the
cemeteries of the capital, Memphis, play a prominent part in shaping
our ideas of the Egyptian state. This series of spectacular buildings
came to a halt after the reign of Pepy II, and they were revived only by
Mentuhotep II with his mortuary temple at Deir el-Bahri in western
Thebes.
To match this state of affairs, the upper chronological limit of the
First Intermediate Period is sometimes raised to include the three
decades during which the last kings of the Memphite line after Pepy II
still held power. While taking liberties with the scheme whereby
Egyptian history is divided into dynasties, this approach is not wholly
unjustified. In fact, large-scale building may be understood as
good evidence not only for the nature of the core institutions of the
state but also for the fact that they were still actually functioning.
The glaring gap in the monumental record during the First Inter-
mediate Period therefore suggests that the social system had become
fragmented, both in its political organization and in its cultural
patterns.
It is equally apparent, however, that the First Intermediate Period
archaeological and epigraphic data indicate the existence of a thriving
culture among the poorer levels of society, as well as vigorous social
development in the provincial towns of Upper Egypt. Rather than
being an outright collapse of Egyptian society and culture as a whole,
the First Intermediate Period was characterized by an important,
though temporary, shift in its centres of activity and dynamism.
To understand both the crisis of the pharaonic state and the pro-
cesses that ultimately led to the re-establishment of a unified political
organization on a new basis, it is crucial to investigate the ways in
which political institutions were rooted in society. Much of Egyptian
history tends to concentrate on the royal residence, the kings, and
'court culture', but in writing the history of the First Intermediate
Period it is necessary to focus instead on provincial towns and on
the people themselves, who make up the most basic elements of
society.
THE FIRST I N T E R M E D I A T E P E R I O D III
The Capital and the Provinces
The pharaonic state originally emerged as a centralized system. From
the earliest times, its key institutions—the king and his court—were
firmly installed in the capital. The social elite was also concentrated
there, as well as the administrative expertise and the control of the
traditions of high culture. In addition, the installations of state
religion, and the cult of the king and his divine ancestors, were located
in the immediate vicinity of the capital. The administration of the
country was controlled by royal emissaries, who were put in charge of
extensive sections of the Nile Valley. Although these administrators
were dealing with the provinces, they still retained their attachment to
the royal residence and continued to regard themselves as members of
the elite society of the capital. Until well into the 5th Dynasty, nothing
of the cultural achievements that attest to the grandeur of the Old
Kingdom was to be seen outside the Memphite region. There was a
vast chasm of social and cultural inequality between the country and
its rulers.
However, a profound change in the system began to appear in the
5th Dynasty and was completely in place by the end of the 6th. From
this period onwards, provincial administrators were appointed for
single nomes and took up permanent residence in their districts. As in
other branches of the administration, members of the same family
frequently succeeded each other in office. Although this political move
was probably intended to enhance the efficiency of the provincial
administration, it was to have far-reaching and unforeseen conse-
quences. In the first place, it meant a change in the socio-economic
patterns that lay at the heart of the system. Originally, economic
resources were concentrated at the royal residence and redistributed to
the beneficiaries by the central administration. Now, however, the
nobles residing in the provinces were able to gain direct access to the
products of the country. The opposition between the centre and the
provinces began to act as a differentiating factor within the formerly
homogeneous elite group of officials.
The provincial aristocracy was eager to ensure that its way of life was
on a par with the style of the royal court. This is evident in the decor-
ated monumental tombs that began to appear in the cemeteries of the
regional centres throughout the country. Iconographic patterns, text-
ual models, religious and ritualistic knowledge flowed from the stock
of court culture to the periphery. In addition, the king himself pro-
vided specialist craftsmen, ritualists trained at the residence, and
112 STEPHAN S E I D L M A Y E R
costly goods to maintain and strengthen the bonds of loyalty between
the provincial aristocrats and the court. These tombs, however, are
only the tip of the iceberg; in fact, the various provincial elites and their
staff acted as separate centres within the political organization, sus-
taining specialist professionals and keeping a growing amount of local
production for use within the provinces themselves (rather than allow-
ing it to be exploited by the royal court), thus leading to a change in the
social and economic patterns of the provinces. Rural Egypt became
economically richer and culturally more complex.
The Provincial Milieu
The transformation of the culture and economy of the provinces
affected the whole of society. This process can be followed in profound
changes in the archaeological record, which were rooted in the 6th
Dynasty, and which reached their climax in the earlier half of the First
Intermediate Period. Again we have to turn to the cemeteries for
essential data—partly because of the unfortunate lack of excavated
settlements of this period, but mainly because of the inherent signifi-
cance of the remains of funerary culture.
If we compare the situation in the earlier Old Kingdom with that in
the late Old Kingdom and the First Intermediate Period, a change in
the quantity of graves becomes immediately obvious. For the later
period, many more cemeteries are known, and, whenever a particular
region has been explored systematically, there is a marked increase in
the number of tombs. To explain this phenomenon, two factors must
be taken into account. First, the increase in tombs clearly attests to
demographic growth during the Old Kingdom, and probably the most
influential factors for change were rooted in the local settings them-
selves, where population growth was probably accompanied and
accentuated by the development of more intensive and more efficient
uses of the available agrarian resources. Secondly, during the late Old
Kingdom and the First Intermediate Period, ordinary tombs became
considerably larger and burials began to be provided with much better
grave goods. Not only have such tombs been more easily identified and
dated (because of their greater size and more varied contents), but they
have also attracted more excavators. In fact, provincial cemeteries of
the first part of the Old Kingdom had a reputation, among earlier
archaeologists, for not repaying the labour of excavation.
Like the appearance of decorated monumental tombs in Upper
Egypt, the increase in numbers of graves in the provincial cemeteries
THE FIRST I N T E R M E D I A T E P E R I O D 113
therefore reflects, to an important degree, a change in the social pat-
tern of consumption. This phenomenon seems particularly obvious
in the funerary record, but it was not restricted to this sphere. In fact,
the most valuable objects that become most abundant and widely rep-
resented in the graves of the early First Intermediate Period—
cosmetic stone vessels, ornaments and amulets of gemstones, and
even gold—were everyday items of daily life, rather than being
specially made for funerary use. It seems clear, therefore, that the
provinces enjoyed favourable economic conditions during the late Old
Kingdom and the First Intermediate Period.
The distribution of the cemeteries can also provide some indica-
tions of settlement patterns. The landscape was dotted with villages,
while the sites of the nome capitals are marked not only by the groups
of rock tombs or monumental mastaba-tombs belonging to the provin-
cial aristocracy, but also by very extensive cemeteries of ordinary
townspeople. The tombs of the urban population do not differ, in prin-
ciple, from those of the villagers; however, they are often larger and
better equipped. An urbanized structure, therefore, dominated the
provincial settlement pattern not only politically and socially but also
demographically and economically.
Changes in Styles and Shapes as Signs of Cultural and Social
Development
The period that followed the close of the Old Kingdom brought about
fundamental changes in material culture. In fact, during the First
Intermediate Period, almost all artefacts took on a different appear-
ance. We can review a few of the most significant aspects of this pro-
cess.
From an archaeologist's point of view, pottery is clearly the most
important type of artefact. Since the Early Dynastic Period and
throughout the Old Kingdom, the repertoire of containers had been
dominated, morphologically, by ovoid shapes: the point of maximum
extension almost always lay slightly above the middle of the vessel.
During the First Intermediate Period, this style was quickly aban-
doned. Now, baglike or even droplike sagging shapes were made. It is
not difficult to identify the driving force behind this process. Clearly,
the aim was to adapt vessel shapes in order to take advantage of the
capabilities of the potter's wheel. In the case of ovoid containers, a
large part of the outer surface had to be scraped into shape manually
after throwing. In the case of bag-shaped vessels, the amount of work
114 STEPHAN SEIDLMAYER
necessary could be reduced considerably. It seems significant, how-
ever, that this process took place only some 200 years after the first
introduction of the potter's wheel to Egyptian workshops during the
5th Dynasty. It was apparently only with the emergence of the First
Intermediate Period that people were prepared to dispose of tradi-
tional models and give preference to more efficient modes of pro-
duction.
Furthermore, a whole range of new object classes became popular
in provincial burials during the First Intermediate Period. In the Old
Kingdom, the grave goods of poorer burials had been chosen entirely
from among the types of objects used in daily life, but in the First
Intermediate Period objects began to be made exclusively for funerary
use. Crudely made wooden figures of offering bearers, boats, even
whole workshop scenes, are good instances of this trend. Another
example is the appearance and increasing use of coloured masks made
from gypsum and linen (cartonnage) to cover the heads of the mum-
mified bodies. It also became increasingly common to use simple slab
stelae as a means of marking the offering place in the superstructure of
small mastaba-tombs or in the chapels of simple rock tombs.
The appearance of these objects indicates that both the demand and
the means available in the provincial towns were sufficient to support
an area of craftsmanship specializing in 'non-functional' products.
Even more important, however, is the fact that the prototypes of these
types of object have their origins in Old Kingdom elite culture. The
model funerary figures of people engaged in fundamental tasks can be
traced back directly to the repertoire of scenes from daily life depicted
in Old Kingdom mastaba-tomb decoration. It appears that by the First
Intermediate Period those factors that had previously inhibited cul-
tural communication between different social strata now ceased to
operate.
Passing on the traditions of elite culture to a wider circle of users
went hand in hand with a marked loss of artistic quality. Not infre-
quently even iconographic patterns were misunderstood and formu-
laic inscriptions misconstrued. While the provincial art of the First
Intermediate Period often exhibits an astonishing degree of originality
and creativity (as will become plain later in this chapter), there is no
way of denying that many pieces are simply ugly and incompetently
made. This aspect, in particular, has caught the attention of historians
and was taken as a sign of overall cultural decline during the First
Intermediate Period. However obvious the latter interpretation may
seem, to assume that this was simply a period of cultural decay would
THE FIRST I N T E R M E D I A T E P E R I O D 115
be to overlook two important processes: first, the assimilation of cul-
tural models developed in Old Kingdom court culture on a nationwide
basis, and, secondly, the emergence of mass consumption.
Religious Ideas
Some of the changes in material culture are indicative of develop-
ments in religious beliefs and ritual practices, as in the case of the
introduction of mummy masks. However, the most important body of
evidence for belief systems in provincial society during the First
Intermediate Period and the Middle Kingdom is the vast corpus of
Coffin Texts, which were magical and liturgical spells inscribed prin-
cipally onto the sides of wooden coffins. While it is obvious that the
bulk of the evidence for these texts dates to the Middle Kingdom, there
are a few instances that show that they had already emerged during the
First Intermediate Period. The textual origins of the Coffin Texts are
still a matter of much debate, in terms of both date and geographical
origin. Clearly, the corpus of royal Pyramid Texts of the Old Kingdom,
which were sometimes inscribed onto the coffins along with the Coffin
Texts, provided important models, but the Coffin Texts themselves
included crucial new material and fresh concepts.
There are only a few surviving examples of Coffin Texts from the
First Intermediate Period, and ownership of inscribed coffins always
remained restricted to the uppermost level of provincial society. Some-
times, however, it seems possible to connect ideas explicitly featured in
the Coffin Texts with aspects of the archaeological record. Only then
does the great antiquity and popularity of some of these concepts
become apparent. This observation lends support to the notion that it
was the provincial setting of the First Intermediate Period that played a
significant role in the origins of the Coffin Texts and contributed to its
conceptual content.
One series of Coffin Text spells was designed to 'assemble a man's
family in the realm of the dead'. The range of persons envisaged is
extensive; the texts mention not only close relatives but also servants,
followers, and friends. The same desire makes itself felt in the develop-
ment of tomb types as early as the 6th Dynasty. Egyptian tombs were
originally built to house only one burial, but by the late Old Kingdom
extensive multi-chambered mastaba-tombs were sometimes construc-
ted, providing space for a whole family or even an extended family in
the sense defined above. The architecture of the tombs provides evi-
dence for ranking within these groups, in that some shafts are deeper
Il6 STEPHAN SEIDLMAYER
and some chambers larger than others, thus providing for more sump-
tuous burials. In fact, wherever the burials themselves are preserved,
both aspects of this new situation—the size of the family groups
involved and the inequality between persons within these groups—are
particularly prominent, since chambers were often used for successive
multiple burials on a regular basis.
The burial customs of the First Intermediate Period therefore
emphasize the crucial importance of interpersonal relations on a
primary level of social organization. This strain of religious thought
closely reflects the role that extended families played as the basic units
of social organization. The funerary spells in question emphasize the
authority wielded by the head of the family over its members, but also
stress the fact that he was able to shelter them from outside demands.
Thus the family, as a unit of solidarity and collective responsibility, was
acting as an interface between the higher levels of social and political
organization. Thanks to this role, the extended family also appears as a
recognized institution in juridical texts of the 6th to 8th Dynasties.
Regional Style and Identity
One of the most intriguing aspects of First Intermediate Period
archaeology is the stylistic variation between different regions. While
the differences between the pottery styles of northern and southern
Egypt are certainly clear-cut, matters are less straightforward with
regard to differences between pottery in different regions within
Upper Egypt or regional variations in terms of other types of artefacts.
In fact, some types of objects appear to have been more affected by
regional variation than others, and it seems that Egyptian material
culture in general was not broken up into a series of unconnected local
variants.
There is one aspect of regional variation, however, that seems to be
of particular significance. Throughout the Old Kingdom, the architec-
ture of mastaba-tombs in Upper Egypt followed uniform patterns and a
continuous course of development. But during the 6th Dynasty and
the First Intermediate Period, distinctive local traditions of tomb con-
struction arose. Examples of such local architectural styles include the
Theban saff-tombs (discussed below) and the mastaba-tombs with
niched facades and long, sloping access corridors leading to subter-
ranean chambers, which have been found at Dendera.
These local types are so different from the principal architectural
styles of earlier periods that the change can hardly be explained simply
THE FIRST INTERMEDIATE PERIOD 117
in terms of the development of local workshop traditions. Instead, it
seems likely that these architectural innovations were deliberately
introduced by the local elites in order to express their own regional
identities.
Society and Government
Even this short survey of the archaeological material provides ample
evidence that far-reaching change occurred in the provinces during the
late Old Kingdom and the First Intermediate Period. In the present
state of research, the meanings of many of the archaeological phenom-
ena discussed (and the mechanisms that produced them) are still
poorly understood. Even our present knowledge, however, strongly
suggests that internal forces of change and powerful external influ-
ences (particularly the impact of Old Kingdom provincial politics) con-
spired to produce greater cultural, economic, and social complexity
throughout the country.
These developments inevitably affected the political system: ten-
sions between the centre and the provinces now gained greater impor-
tance, and the provincial nobility in particular—occupying a crucial
position between the court and the local groups—won new options for
independent action, and, at the same time, had to mediate between
competing interests. This situation raises the question of the ways in
which the organization and ideology of government were adapted to
the social and cultural conditions in the country at large. During the
Old Kingdom, provincial districts were usually (though not always)
run by a two-tiered administration. 'Overseers of priests' of the local
cults were important because of the role played by their temples as
nodes in the network of economic administration, but the leading
office was that of 'great overlord of the nome' (often translated as
'nomarch').
It is important to realize, however, that the end of the Old Kingdom
was not brought about by the increasing power of the great families of
nomarchs. In fact, new lines of local magnates appeared during the
First Intermediate Period. It is, therefore, likely that the Old Kingdom
aristocracy—despite the degree to which they contributed, as a social
group, to the process of change in the political structure of the
country—nevertheless remained dependent on their links with the
Crown. By tracing these new developments we can gain insights into
the relationships between social conditions and political developments
during the First Intermediate Period.
Il8 STEPHAN S E I D L M A Y E R
The Case of Ankhtifi: Crisis, Care, and Power
Ankhtifi, a nomarch of the 3rd and 2nd Upper Egyptian nomes during
the earlier part of the Herakleopolitan period, embodies the new type
of local ruler that emerged during the First Intermediate Period. His
autobiographical text, inscribed on the pillars of his rock tomb near
el-Mo e alla (some 30 km. south of Thebes), is one of the most spec-
tacular examples of its genre to survive from ancient Egypt. It provides
the ideal guide to the great issues of the time, and compellingly evokes
the political atmosphere of southern Upper Egypt during the First
Intermediate Period.
As 'great overlord of the nomes of Edfu and Hierakonpolis' and
'overseer of priests', Ankhtifi simultaneously held key positions in
both the religious and secular wings of the Old Kingdom provincial
administration. In fact, this combination of offices was typical for the
largely independent local rulers during the First Intermediate Period.
The two key events in Ankhtifi's political career were his intervention
in order to pacify and reorganize the nome of Edfu, and his military
expedition against the Theban nome, where his opponents, a coalition
of the Theban and Koptite nomes, actually refused to give battle. All
this was essentially small-scale politics, and, reading between the lines,
he was probably not even particularly successful. It is notable, for
instance, that there are no known successors to Ankhtifi in his role as
semi-independent ruler of the southernmost nomes. Nevertheless, his
inscription proclaims his glory without a trace of false modesty:
His Excellency, the overseer of priests, overseer of desert-countries, overseer of
mercenaries, great overlord of the nomes of Edfu and Hierakonpolis, Ankhtifi, the
brave, he says: 'I was the beginning and the end (i.e. the climax) of mankind, since
nobody like myself existed before nor will he exist; nobody like myself was ever born
nor will he be born. I surpassed the feats of the ancestors, and coming generations
will not be able to equal me in any of my feats within this million of years.
I gave bread to the hungry and clothing to the naked; I anointed those who had no
cosmetic oil; I gave sandals to the barefooted; I gave a wife to him who had no wife. I
took care of the towns of Hefat [i.e. el-Mo e alla] and Hor-mer in every [situation of
crisis, when] the sky was clouded and the earth [was parched (?) and when everybody
died] of hunger on this sandbank of Apophis. The south came with its people and
the north with its children; they brought finest oil in exchange for the barley which
was given to them. My barley went upstream until it reached Lower Nubia and
downstream until it reached the Abydene nome. All of Upper Egypt was dying of
hunger and people were eating their children, but I did not allow anybody to die
of hunger in this nome ... I cared for the house of Elephantine and for the town of
lat-negen in these years after Hefat and Hor-mer had been satisfied I was like a
(sheltering) mountain for Hefat and like a cool shadow for Hor-mer.' Ankhtifi said:
THE FIRST I N T E R M E D I A T E P E R I O D 119
'The whole country has become like locusts going upstream and downstream (in
search of food); but never did I allow anybody in need to go from this nome to
another one. I am the hero without equal.'
Economic crisis is one of the great issues in the texts of the time.
Local magnates were accustomed to boasting that they managed to
feed their own towns while the rest of the country was starving. These
accounts have tended to make a considerable impression on modern
readers, with the result that famines and economic crisis are often
regarded as an essential hallmark of the period. It has even been
argued that the dire consequences of repeated failures of the Nile
flood, caused by climatic change, were responsible for the end of the
Old Kingdom. There can be no doubt that these texts indeed relate to
fact. This becomes obvious when references to famine occur in less
grandiose contexts. An employee of a Koptite 'overseer of priests', for
instance, relates: 'I stood in the doorway of his excellency the overseer
of priests Djefy handing out grain to (the inhabitants of) this entire
town to support it in the painful years of famine.'
It remains to be carefully considered, however, to what extent this
situation was really specific to the First Intermediate Period. In fact,
independent evidence confirming climatic change during the First
Intermediate Period is lacking. Instead, the available data seem to sug-
gest that the 'Neolithic Wet Phase' had already ended during the Old
Kingdom, bringing drier climatic conditions in the adjacent desert
areas in particular, as well as encouraging a general process of adap-
tation to lower levels of annual Nile flooding. These environmental
changes showed no signs of affecting the development of pharaonic
civilization at that date, thus calling into question any supposed con-
nections with the First Intermediate Period. Recent archaeological
observations from Elephantine even seem to indicate that Egypt was
experiencing flood levels slightly above average during the First Inter-
mediate Period.
Considering the long-term regularities and variations of the flood of
the Nile, it seems clear that the spectre of famine due to Nile failure in
individual years must have haunted the Egyptians to greater or lesser
degrees throughout all periods of Egyptian history. To understand the
prominence of this issue in the texts of the First Intermediate Period, it
is therefore necessary to place it in a wider literary context.
The introductory phrase that forms the basis for Ankhtifi's account
is a very traditional one. It is actually one of the stock phrases of the
autobiographical texts of Old Kingdom officials, asserting their moral
integrity. During the First Intermediate Period, the principle of caring
120 STEPHAN SEIDLMAYER
for the weak was greatly elaborated. At this time the great men were
prepared to step in whenever and wherever need might arise in society,
through economic problems, political crises, or individual misfor-
tunes. The provincial rulers were not merely sheltering and support-
ing a few people (as a father might shelter and support the members of
his family) but taking responsibility for the whole of society, whether
the population of their home town or that of the nome or nomes they
ruled. The message is clear: people would be helpless without their
rulers. Left on their own, they would simply not be able to face the
hazards of life. It goes without saying that this beneficent role of the
ruler was indissociable from his right to obeisance and his authority—
thus Ankhtifi points out, 'on whomsoever I laid my hand—no harm
could approach him, because my reasoning was so expert and my
plans were so excellent. But every ignorant person, every wretch who
opposed me—I retaliated against him for his deeds.'
In the First Intermediate Period, crises had evidently become
socially significant as contexts in which personal power and social
dependence could be legitimized, and this observation probably helps
a good deal in explaining why the issue of famine and sustenance was
so important to local magnates at that time.
Competition and Armed Conflict
During the Old Kingdom, the local administrators were obliged to
organize the military service of the people under their jurisdiction and
to lead such troops on aggressive and peaceful missions into the
regions adjoining the Nile Valley. As early as the 6th Dynasty, foreign
mercenaries—particularly Nubians—were already being recruited
into the Egyptian army. During the First Intermediate Period, the use
of local troops and the military experience of the local governors there-
fore emerged as decisive forces in their struggle for ascendancy. Thus
Ankhtfi declares:
I was one who found the solution when it was lacking, thanks to my vigorous plans;
one with commanding words and untroubled mind on the day when the nomes
allied together (to wage war). I am the hero without equal; one who spoke freely
while people were silent on the day when fear was spread and Upper Egypt did not
dare to speak... As long as this army of Hefat is calm, the whole land is calm; but if
one steps on (its) tail like (that of) a crocodile, then the north and south of this whole
land are trembling (with fear). ... I sailed downstream with my strong and trust-
worthy troops and moored on the west bank of the Theban nome ... and my trust-
worthy troops searched for battle throughout the west of the Theban nome, but
nobody dared to come out through fear of them. Then I sailed downstream again
THE FIRST I N T E R M E D I A T E P E R I O D 121
and moored on the east bank of the Theban nome . . . and his [probably Ankhtifi's
opponent's] walls were besieged since he had locked the gates through fear of these
strong and trustworthy troops. They became a search party looking for battle
throughout the west and the east of the Theban nome, but nobody dared to come out
through fear of them.
It was not really new for an official to claim authority over more than
one nome. At the end of the 5th Dynasty, for example, the kings had
established the office of 'overseer of Upper Egypt' to supervise the
administrators of the individual Upper Egyptian nomes. During the
First Intermediate Period, there are also documented instances of
officials who were responsible for a larger territory, such as Abihu,
who governed the nomes of Abydos, Diospolis Parva, and Dendera in
the early Herakleopolitan period. Thus there was nothing unusual
about Ankhtifi's double nomarchy or even his claim to military
supremacy as far south as Elephantine.
The narrative of Ankhtifi's wars, however, makes it plain that, by
this time, the king was not being mentioned even nominally as an
authority who could control the distribution of power between local
rulers. It is important to realize that this situation implies a radical
change in mentality. In the closed political system of the Old Kingdom,
the king had been the sole source of legitimate authority. All actions of
the officials relied on his command, and he judged and rewarded their
merits. When the power of kingship faded, however, a more open
situation emerged. Now, local rulers could act in accordance with their
own aims. They had to rely on their own power bases; they had to
defend their positions in competition with others; and they also gained
a new awareness of their own achievements, which is such a prom-
inent feature of Ankhtifi's inscriptions.
Gods, Politics, and the Rhetoric of Power
In the inscriptions on the walls of Ankhtifi's tomb, the king is men-
tioned only once in a short label appended to one of the wall paintings:
'May Horus grant a (good) Nile flood to his son Neferkara.' It is sig-
nificant that in this instance an appeal is being made to the king in his
sacred role as a mediator between human society and the forces of
nature. His political role, however, has evidently been taken over by
other authorities:
The god Horus fetched me to the nome of Edfu for life, prosperity and health to re-
establish it... In fact, Horus wished to re-establish it, and therefore he fetched me to
re-establish it. I found the domain of (the administrator) Khuu like a swampy estate
122 STEPHAN SEIDLMAYER
neglected by its keeper, in a condition of civil strife and under the lead of a wretch.
Now I caused a man to embrace (even) those who had killed his father or brother in
order to re-establish the nome of Edfu.
In Ankhtifi's texts, it is not the king but Horus, the god of Edfu, who
appears as the supreme authority guiding political action. This concept
is not unique in First Intermediate Period inscriptions. Even the
reunification of Egypt under Mentuhotep II (2055-2004 BC) was
described in similar terms as a result of intervention by Montu, the
great god of the Theban nome: 'A good beginning came about when
Montu gave both lands to King Nebhepetra (Mentuhotep II)' (on the
Abydos stele of an overseer of the treasury, Meru in the time of
Mentuhotep II).
This ideology rested on solid foundations, given that local rulers
usually acted as 'overseers of priests', which secured them a privileged
role in the cult of the gods. Ankhtifi himself is depicted in a scene in
his tomb supervising one of the great festivals of his local god Hemen,
and the earliest mention of the temple of Amun of Karnak derives
from a stele of a Theban overseer of priests who claims to have taken
care of it in years of famine.
From earliest times, provincial temples were both administrative
centres and foci of the personal loyalty of the local population, and it
seems likely that the priesthoods attached to these temples formed the
core group of an early provincial elite. In a way, provincial cults may be
understood as symbolic representations of collective identity. There-
fore, during the First Intermediate Period, god and town often appear
side by side in phrases referring to social embeddedness. People say, 'I
was one beloved by his town and praised by his god', and curses
directed against transgressors threaten that 'his local god shall despise
him and his townspeople (or sometimes "his family group") shall
despise him'. By integrating their personal authority with that wielded
by the local cults, provincial magnates therefore managed to link their
power with one of the moral foundations of local society.
The intriguing subject matter of Ankhtifi's inscription should, how-
ever, not be allowed to eclipse its merits as literature. It is a composition
of unusual brilliance, abounding in original and striking expressions.
Similar qualities may be found in the painted decoration of his tomb
and indeed generally in the art of Upper Egypt during the First Inter-
mediate Period. The Upper Egyptian painters of this time no longer
conformed to Old Kingdom court conventions. Their style is angular,
even bizarre at times, and boldly expressive. Having freed themselves
from outdated models, they created a whole range of new scenes: files
THE FIRST I N T E R M E D I A T E P E R I O D 123
of soldiers and hunters, mercenaries engaged in battle, and religious
festivals. In addition, they introduced new pictures of everyday occupa-
tions, such as spinning and weaving, and updated age-old scenes to tie
in with the latest cultural and technological developments. Far from
being a period of cultural decline, these turbulent years witnessed an
upsurge of outstanding creativity, adapting and developing the exist-
ing media of literary and pictorial expression to correspond to a new
range of social experiences.
This process of change also indicates that the elite of the First Inter-
mediate Period felt the need to communicate new social develop-
ments; when government could no longer rely on the simple imposition
of power, its foundations had to be made explicit. Ankhtifi's text may,
therefore, be read as a speech concerning the necessity of government
and the benefits of strong rule. It is also notable how closely these
ideals—to which Ankhtifi so persuasively appeals—link up with the
underpinnings of local social organization and provincial traditions.
The 'Theban Ascendancy' and the Necropolis of el-Tarif
During the Old Kingdom, Thebes, the capital of the 4th nome of
Upper Egypt, had been a third-rate provincial town. However, during
the early Herakleopolitan period, a series of overseers of priests in
charge of the local affairs is known from funerary stelae recovered
from the extensive cemetery of el-Tarif, on the west bank immediately
opposite Karnak temple. This series of officials was succeeded by a
nomarch Intef, who combined (as Ankhtifi had done) the post of'great
overlord of the Theban nome' with that of 'overseer of priests'. In
addition, however, he claimed the titles of'the king's confidant at the
narrow gateway of the south [i.e. Elephantine]' and 'great overlord of
Upper Egypt'. Since an inscription referring to this Intef was found in
the cemetery of Dendera (the capital of the 6th nome of Upper Egypt),
it seems fair to assume that his authority was recognized far beyond
the confines of his native province.
This nomarch Intef is in all probability identical with a certain
'Intef the Great, born of Iku', who was named in contemporary
inscriptions and to whom even the early Middle Kingdom ruler
Senusret I (1956-1911 BC) dedicated a statue in the temple of Karnak.
Furthermore, this man is described as 'count Intef, the ancestor of the
Theban nth Dynasty, in the king-list inscribed on the walls of Thut-
mose Ill's 'chapel of royal ancestors' in Karnak. Only his immediate
successor, Mentuhotep I, however, was designated as a king in later
124 STEPHAN S E I D L M A Y E R
tradition, although the Horus name assigned to him, namely Tepy-a
(literally 'the ancestor'), clearly betrays this as posthumous fiction.
Contemporary epigraphic sources are lacking both for Mentuhotep I
and for his son, Sehertawy Intef I (2125-2112 BC), but the latter's tomb
is still the most prominent landmark in the necropolis of el-Tarif,
serving as the sole surviving monument to the power and the grandeur
of the earliest Theban kings.
In the necropolis of el-Tarif, a special type of rock tomb developed
during the First Intermediate Period, apparently as an adaptation to
the local topography. For the smaller tombs of private individuals, a
broad court was sunk into the strata of gravel and marl of the low
desert terrace. In the rear face of this court, a portico with a row of
heavy, square pillars formed the facade of the tomb, and this row of
pillars gave rise to the modern designation of the architectural type as a
saj^tomb (sajfbeing Arabic for 'row'). A short, narrow corridor in the
centre of the facade led to the tomb chapel, which also contained the
burial shaft leading down to the tomb.
King Intef I chose to build for himself a saff-tomb of gigantic dimen-
sions. The court of the Saff Dawaba, as it is called today, was sunk
into the ground as a huge rectangle, 300 m. long and 54 m. wide;
400,000 cu. m. of gravel and soft rock were excavated from it and piled
up as two long, low heaps along the sides of the court. The front part of
the court (where some form of entrance chapel would have been built)
is unfortunately lost, but the rear part of the tomb, with its broad facade
comprising a double row of rock-cut pillars and three chapels (one for
the king himself and two probably for his wives), is still relatively well
preserved. Since the surfaces of the walls have entirely flaked off, it is
not known whether they were originally painted. Nevertheless, the Saff
Dawaba appears to have been an impressive piece of architecture that
reveals some of the fundamentals of the newly constituted kingship.
Above all, there is not the slightest attempt to emulate Old Kingdom
royal funerary architecture. Rather, the Theban kings created an
explicitly Theban type of royal tomb from the stock of local tradition.
Furthermore, unlike many rulers of the Old Kingdom, they did not
strive for exclusivity of location. The royal tombs continued to be situ-
ated in the main cemetery of Thebes, directly opposite the town and
its temples across the river. Here, the burial place of the king was
surrounded not just by the tombs of a narrow circle of courtiers but
by the cemetery of the local population. In addition, smaller tomb
chapels along the sides of the court of the royal tomb provided space
for the burials of some of his followers. The message conveyed by this
THE FIRST I N T E R M E D I A T E PERIOD 125
architecture, therefore, was focused not only on the exalted position
of the king, but also on the fact that these rulers were rooted in the
Theban setting and in local society.
The immediate successors of Intef I (Wahankh Intef II and Nakht-
Nebtepnefer Intef III) continued to build for themselves very similar
so^-tombs in the necropolis of el-Tarif, parallel to the Saff Dawaba.
When Mentuhotep II moved to the new site of Deir el-Bahri, it was
perhaps only because the suitable building ground for monumental
architecture at el-Tarif had been used up by his time.
King Wahankh Intef II (2112-2063 BC)
While Mentuhotep I and Intef I, the first two kings of the nth Dynasty,
reigned for only fifteen years, the fifty-year reign of Intef I's brother
and successor, Wahankh Intef II, stands out as the most decisive phase
in the development of the new monarchy. A large quantity of archaeo-
logical, epigraphic, and artistic evidence has survived from his reign,
thus facilitating crucial insights into the nature of Theban kingship.
Intef II claimed the traditional dual kingship (the nesu-bit) as well as
the title 'son of Ra', which referred to the dogma of divine descent. He
did not, however, assume the complete royal protocol with its five
'Great Names', the so-called royal fivefold titulary (see Chapter i for a
discussion of the five royal names). In fact, he added only the 'Horus
name' Wahankh ('enduring of life') to his 'birth name', Intef, and had
no 'throne name' (which would traditionally have incorporated the
name of the sun-god Ra). Unfortunately, only a few representations
of the king have been preserved, so it remains impossible to decide
whether he used the whole array of royal crowns and other insignia,
although the present balance of evidence suggests that this is unlikely.
The early Theban kings were evidently well aware of the limited
character of their rule.
True to his social origins among the provincial magnates, Intef II
created a biographical stele that stood in the entrance chapel to his saff-
tomb in el-Tarif. This monument, which bears a depiction of the king
accompanied by his favourite dogs, sums up in retrospect the accom-
plishments of his reign; and the statements made in the text are amply
confirmed by the inscriptions of his followers.
As mentioned above, there is good reason to believe that the last
non-royal Theban nomarch already held sway over a large part of
southern Upper Egypt. Intef II, however, launched the decisive north-
ward push. He captured the nome of Abydos, which, since the days of
126 STEPHAN S E I D L M A Y E R
the Old Kingdom, had been the most important administrative centre
in Upper Egypt, and he carried his attack even further into the territory
of the loth nome of Upper Egypt. This constituted a policy of open
hostility against the Herakleopolitan kings, and for several decades
war was to be waged intermittently in the stretch of land between
Abydos and Asyut.
The King's Men
We know some of the men who served under Intef II. The Theban
military officer Djary, for example, who fought the Herakleopolitan
army in the Abydene nome and who pushed northwards into the zoth
nome; Hetepy from Elkab, who managed the administration of the
three southernmost nomes for the king; and Intef s treasurer Tjetjy,
whose magnificent stele is now in the collection of the British Museum.
Although the biographical inscriptions of these men were primarily
intended to praise their owners' achievements, there cannot be the
slightest doubt concerning the man to whom ultimate authority was
due:
So says Hetepy: I was one beloved of my lord and praised by the lord of this land; and
his majesty truly made this servant [i.e. Hetepy] happy. In fact, his majesty said:
'There is no one who [...] of (my) good command, but Hetepy!', and this servant did
it exceedingly well, and his majesty praised this servant on account of it. And his
nobles said: 'May this face praise thee!'.
It is no doubt extremely significant that there were no longer
'nomarchs' in the territory controlled by the Theban rulers, and none
of the officials who carried out important missions for these kings was
given the chance to establish himself as a local ruler mediating
between the interests of his dominion and the king's demands. The
newly founded state was organized not as a loosely knit network of
semi-independent magnates, as the Old Kingdom had become towards
its end, but as a powerful system relying on strong bonds of personal
loyalty and on tight control.
Monuments and Art
Apart from his military feats, Intef II emphasizes in his biographical
inscription that he had built many temples to the gods, and, in fact, the
earliest surviving fragment of royal construction at Karnak temple is a
column of Wahankh Intef II. At Elephantine, excavations in the temple
THE FIRST I N T E R M E D I A T E PERIOD 127
of the godess Satet have revealed an unbroken series of building stages
reaching back to the Early Dynastic Period. While the rulers of the Old
Kingdom dedicated only a few votive offerings to Satet on Elephantine,
Intef II was the first king to erect chapels for both Satet and Khnum,
and to commemorate his activity in inscriptions on their door frames.
Each of his successors during the nth Dynasty followed this example.
The sequence of events that has been revealed so clearly in the
excavations at Elephantine was also true for many other temple sites.
In fact, apart from a few specific exceptions, royal building activity in
the provincial temples of Upper Egypt is attested only from the nth
Dynasty onwards. Intef II, therefore, may be said to have inaugurated a
new policy of royal presence and activity in the sanctuaries throughout
the country—a policy that was to be continued on an even larger scale
by Senusret I and many later kings.
The private and royal monuments from the time of Intef II also
include splendid examples of Theban nth-Dynasty art. Some of the
lesser monuments, such as the stelae of Djary, still exhibit the bold
artistic style of the First Intermediate Period in Upper Egypt, but at the
same time the royal workshops were beginning to produce beautifully
balanced works characterized by thick, rounded modelling, and often
deriving a special aesthetic effect from the contrast between large,
plain surfaces and areas filled with finely carved detail such as elab-
orately pleated kilts or intricately patterned hairstyles. In these works,
there is a clearly visible desire for the creation of a fitting medium to
convey the aspirations of the new Dynasty.
By concentrating on the developments in southern Upper Egypt, it
is possible to trace the emergence of a new political structure that was
to lead, in unbroken sequence, to the formation of the state of the
Middle Kingdom. This process, which was to have an enormous effect
on Egypt's future, should probably be regarded as the most important
phenomenon in the history of the First Intermediate Period. We
should not forget, however, that the Theban kingdom occupied only a
small, remote, and relatively unimportant part of Egypt as a whole. The
periods of war and conflict that make for such startling reading in
biographical narrative were therefore no doubt only localized and
short-lived episodes. At most places, for most of the time and for most
of the people, the First Intermediate Period probably would have been
a rather less thrilling experience.
Most of the country, during the First Intermediate Period, was in
the hands of the Herakleopolitan successors to the ancient Memphite
monarchy. To reach a balanced assessment of the period it is, therefore,
128 STEPHAN S E I D L M A Y E R
crucial to concentrate on the situation in the Herakleopolitan realm
just as much as that of the south.
The Herakleopolitan Kingdom
We know very little about the eighteen or nineteen kings who made up
Manetho's Herakleopolitan Dynasty, occupying the throne of Egypt
for a period of perhaps 185 years. Even their names remain largely
unknown, and, with only one or two exceptions, it is impossible to
assign the few named kings to their correct places within the dynastic
sequence. In addition, none of the lengths of their reigns is known.
According to Manetho, the Herakleopolitan Dynasty was founded by a
king called Khety, and this piece of information is confirmed by
contemporary epigraphic evidence that refers to the northern king-
dom as the 'house of Khety'. We remain totally ignorant, however,
either of the social origins of Khety or the circumstances of his rise to
the throne.
Contemporary sources unequivocally corroborate Manetho's asser-
tion of a connection between this dynasty and the town of Herakleopo-
lis Magna. Most probably the kings actually resided at Herakleopolis,
although the fact that Merykara (0.2025 BC)> me ^nal or penultimate
Herakleopolitan ruler, was buried in a pyramid complex in the ancient
royal necropolis of Saqqara is a clear indication that the Herakleo-
politan kings felt themselves to be within the tradition of the Mem-
phite kingship. The fact that the throne name of Neferkara Pepy
II—the last great ruler of the Old Kingdom—was assumed by at least
one of the Herakleopolitan kings (like several of the rulers of the 8th
Dynasty) obviously points in the same direction.
None of the Herakleopolitan kings left any monuments, or at least
none has yet been found, although this may partly be due to the fact
that the archaeological exploration of the site of Herakleopolis Magna
itself (modern Ihnasya el-Medina) has only been under way since
1966. The fact that none of the Herakleopolitan pyramids has been
hitherto securely identified in the necropolis of Saqqara may be
regarded as evidence that these were rather inconspicuous buildings,
perhaps similar to the small pyramid of the 8th Dynasty King Qakara
Iby. Clearly the Herakleopolitans did not succeed in establishing a
powerful centralized system along the lines of the state of the Old
Kingdom, even in the heartlands of their own dominion.
Most of the contemporary references to the Herakleopolitan Dynasty
derive from the monuments of private individuals, mainly comprising
THE FIRST I N T E R M E D I A T E P E R I O D 129
biographical inscriptions, from southern Middle Egypt and Upper
Egypt, and they tend to centre on the Herakleopolitan-Theban war, an
issue to which we will return later. The Herakleopolitan era also
formed the historical setting for two of the most important literary and
philosophical texts to have survived from ancient Egypt, the Teachings
for King Merykara and the Tale of the Eloquent Peasant. Nowadays, there
is widespread consensus that these 'wisdom texts' were actually com-
posed during the Middle Kingdom, although the precise circumstan-
ces of their origins and the vicissitudes of their textual transmission
remain the subject of controversy. The utmost caution is therefore
advisable in any attempt to use them as historical sources. The
Teachings for King Merykara, for instance, incorporate a background
narrative in which the addressee's royal father is engaged in warding
off Asiatic infiltration into the eastern Delta. Viewing the overall
situation, such a scenario does not sound unlikely, but there is not yet
any independent evidence that Asiatic immigration was a problem
during the First Intermediate Period (although it is certainly attested
for the later Middle Kingdom).
The Herakleopolitan Era in Social and Cultural History
Considering the lack of data concerning the Dynastic history of the
Herakleopolitan rulers, it seems all the more important to investigate
whether the Herakleopolitan kingdom can be regarded as a distinctive
social and cultural entity. Turning to the archaeological evidence, we
should focus attention on the core areas of the Herakleopolitan king-
dom: the Memphite and Faiyum regions. From an archaeological
point of view, southern Middle Egypt was effectively an Upper Egypt-
ian region.
In the north, we face a twofold problem. The available sources of
evidence do not form a rich and coherent historical framework like the
data from Upper Egypt; it is, therefore, exceedingly difficult to estab-
lish a sound archaeological sequence. Furthermore, there are no key
groups of material that can be firmly dated in dynastic terms. It, there-
fore, often remains doubtful as to which monuments are to be
assigned to the Herakleopolitan period proper, and which, in fact,
derive only from the time after the reunification of the country and the
early Middle Kingdom.
In many respects, the development of the archaeological material in
the north follows the same course as in Upper Egypt. For instance,
wooden models of servants and workshops, cartonnage masks, and
130 STEPHAN S E I D L M A Y E R
extensive family tombs all appear in both areas, and burial customs are
in general largely the same. For some classes of artefacts, such as stone
vessels and button seal amulets, the north and the south evidently
drew on the same models. Judging from the archaeological material,
the communities making up Herakleopolitan society seem to have
undergone similar patterns of social and cultural development to the
rest of the country.
Important differences, however, must not be overlooked. The
development of the shapes of pottery vessels, for example, follows an
entirely different course in the north. Here, the age-old ovoid pattern
was not abandoned as it was in the south. Rather, a series of very special
types of slender ovoid jars, often with pointed bases and quite peculiar
cylindrical or funnel-shaped necks, emerged. The morphological pat-
terns developed in the north during the First Intermediate Period
evidently adhered much more closely to Old Kingdom tradition.
Even in the Herakleopolitan kingdom, however, the elite culture in
the style of the Old Kingdom aristocracy did not survive. The social
profile of the occupants of the ancient court cemeteries in the Mem-
phite region therefore changed fundamentally. To earlier Egyptolo-
gists, who used to rely for their standards of judgement entirely on
comparison with Old Kingdom court culture, this seemed to indicate
dramatic events. Set against a broader background, however, it is clear
that we are simply witnessing the change from very extraordinary
conditions to a phase of comparative normality, when the Memphite
necropoleis became similar to the cemeteries of provincial towns.
Certainly, when Memphis lost its dominant status at the end of the Old
Kingdom, this must undoubtedly have entailed severe changes in the
living conditions of its inhabitants. But the archaeological record from
the Memphite cemeteries cannot be construed as evidence for a social
revolution or a civil war after the demise of the Old Kingdom.
At several important sites—Saqqara, Heliopolis, and Herakleopolis
Magna—small mastaba-tombs incorporating decorated offering
chapels and false-door stelae are attested, thus allowing the style of
Herakleopolitan art to be assessed. Old Kingdom tradition looms
large. Ritual scenes and scenes of daily life, the arrangement of the
decoration, and the style of carving closely follow Old Kingdom
patterns—but everything is in miniature. Here, in the Memphite
region and its surroundings, where the monuments of Egypt's glorious
past were available for ready inspection and where its workshop
traditions had been entrenched for centuries, the legacy of the Old
Kingdom was not to be forgotten.
THE FIRST I N T E R M E D I A T E P E R I O D 131
The full range of situations in which these traditions were exercised
during the First Intermediate Period probably escapes us because of
the state of archaeological research at the end of the twentieth century.
Immediately after the reunification of the country, however, the nth
Dynasty King Nebhepetra Mentuhotep II was able to draw on the
expertise of Memphite artists and stonemasons for the construction
and the embellishment of his funerary temple at Deir el-Bahri. It is
in his reign that we witness the sudden reappearance of a level of
expertise that had not been attested since the pyramids of the Old
Kingdom.
The Internal Organization of the Herakleopolitan Kingdom
During the early Herakleopolitan Period, southern Upper Egypt slip-
ped out of royal control, but what happened to those parts of the
country that remained under Herakleopolitan rule until its end? The
relevant sources include the prosopographical records and bio-
graphical inscriptions from southern Middle Egypt. Among these,
pride of place goes to the tombs of the overseers of priests at Asyut.
During the latter part of the Herakleopolitan period, Asyut emerged as
the most important military stronghold in Upper Egypt, remaining
faithful to the Herakleopolitan kings in their struggle against the
Theban rebels. The biographical inscriptions of three successive hold-
ers of office provide crucial information both on the course of political
events and on current views on the ideology of rule.
Additional information can be derived from a group of graffiti that
were inscribed on the walls of the travertine quarry at Hatnub by
emissaries of a nomarch Neheri of the nome of el-Ashmunein, whose
rock tomb is known at el-Bersha. A date for these texts immediately
after the end of the Herakleopolitan Period seems most likely (although
some would contest this). Certainly, however, their intellectual outlook
is firmly rooted in Herakleopolitan tradition.
The topics addressed in these texts from Asyut and Hatnub are
similar, in many respects, to those encountered in the texts from
further south. Again, the local rulers' claims to have cared for their
towns in critical situations feature prominently. The biographical
inscription of the earliest overseer of priests at Asyut even provides a
detailed description of the measures he took to improve the irrigation
system in order to ensure sufficient harvests in bad years. Further-
more, the military prowess of the nomarchs is emphasized. Both their
successes in struggles with a foreign enemy (the Theban ruler) and the
132 STEPHAN SEIDLMAYER
establishment of public security within their own nomes are stressed.
Lastly, the local magnates' care for the temples of their towns is not
forgotten: both construction work in the temples and the making of
provisions for the needs of the associated cults are mentioned.
In stark contrast to the text of Ankhtifi, however, the maintenance of
close connections with the king plays an important part in the texts of
the Asyut magnates. They themselves claim descent from venerable
aristocratic stock, and close personal ties seem to have linked them
with the Herakleopolitan house of rulers. One of them, for example,
mentions that, in his childhood, he received swimming lessons
together with the royal children. In addition, the intervention of the
Herakleopolitan army in Upper Egypt is mentioned. Herakleopolitan
rule, therefore, was something very real to the local rulers of southern
Middle Egypt.
Our sources for the internal structure of the Herakleopolitan king-
dom remain extremely sketchy. Nevertheless, the available material
seems to suggest that the Herakleopolitan monarchs may have relied
on a class of provincial aristocrats who stayed loyal to the Crown,
particularly in those cases where there were strong personal bonds
(perhaps through kinship, marriage, or friendship). These aristocrats,
however, would, at the same time, have regarded their own towns as
crucially important to them, perhaps even making them the principal
focuses of their loyalty. In this respect, the Herakleopolitan kingdom
again seems to have inherited one of the characteristics of the Old
Kingdom: it may also, therefore, have shared one of its structural
weaknesses.
Kom Dara
In this context, an important, though rather enigmatic monument
may be significant. In the cemetery of Dara, some 27 km. downstream
from Asyut in Middle Egypt, a truly gigantic mud-brick mastaba-tomb,
known as Kom Dara, occupies a commanding position. This building
has not yet been properly investigated. In its present condition, an area
of 138 x 144 m. (that is, 19,872 sq. m.) is delimited by massive outer
walls that originally rose to a height of about 20 m. The remains of the
mortuary chapel that must surely once have formed part of the com-
plex have not yet been found. The interior, however, was reached by a
sloping corridor entering the building in the middle of its north side,
and leading down to a single subterranean burial chamber constructed
from large limestone slabs.
THE FIRST INTERMEDIATE P E R I O D 133
The enormous size of this tomb, along with its square layout and the
location of its burial chamber, are immediately reminiscent of a
pyramid. Closer analysis of its construction, however, reveals beyond
any doubt that the building was never planned as a pyramid. In fact,
access to the burial chamber from the north is a fairly common feature
in private tomb architecture of the late Old Kingdom, while the square
layout of the superstructure is paralleled by lesser tombs in Dara
cemetery itself. Kom Dara, therefore, may be understood as a monu-
mental tomb that derived from a local prototype, very much in the way
that the royal saff-tombs at Thebes developed from the simpler types of
saff-tombs built for the funerary cults of the ordinary people.
On the basis of pottery, Kom Dara can be dated to the earlier half of
the First Intermediate Period. Its owner remains unknown to us, and
there is not yet any definite evidence to support the frequently repeated
identification with an otherwise unattested King Khuy, whose name
appears on a relief fragment found reused in another building at the
site. The tomb itself, however, attests unequivocally to its owner's
aspirations to a political role that far surpassed that of a mere nomarch,
regardless of whether he actually dared to assume the titles of royalty.
There are no historical records that can tell us what was actually
happening at this site, but the whole context makes it plain that the
owner of the Kom Dara tomb did not in fact succeed in establishing an
independent centre of power, as the Thebans did at a slightly later date.
It is tempting, however, to speculate a little further. In the wide, fertile
plains of Middle Egypt, every ambitious local dynast was bound to find
himself immediately surrounded by a score of powerful competitors.
The geographical situation itself, therefore, may have helped to stabil-
ize the balance of power between a number of Middle Egyptian local
rulers, which, in turn, could have been material in maintaining royal
overlordship. In addition, it does not seem too far-fetched to assume
that here, in one of the agriculturally most productive areas of the
country, the Crown saw important interests at stake and, accordingly,
felt rather less inclined to tolerate the political adventures of provincial
rulers than in the remote stretches of the 'head of the south' (that is,
the Theban region).
The Final War
Matters probably reached a head when Wahankh Intef II attacked the
Thinite nome and pushed northwards, finding his advance eventually
checked by the Asyut nomarchs. A record of at least one counter-attack
134 STEPHAN SEIDLMAYER
by the Herakleopolitans has survived in the form of a very fragmentary
inscription in the tomb of Ity-yeb (the second in a sequence of over-
seers of priests at Asyut), who reports successful military operations
against the 'southern nomes'. In addition, the narrative recounted in
the Teachings for Merykara claims that King Merykara's father had
recaptured Abydos. Whether these facts are to be connected with the
'rebellion of Thinis', recorded on a stele from Mentuhotep IFs four-
teenth regnal year, remains a matter of speculation.
It is clear, however, that this Herakleopolitan military success had
no lasting effect on the outcome, since the tomb of Ity-yeb's son, Khety
II, from the time of King Merykara, contains a report concerning
further conflict with the Theban aggressors. No record has survived of
the sequence of events in this final phase of the war, but there can
hardly be any doubt that Asyut was taken by force. In any event, the
ruling family of Asyut did not survive the Theban victory.
Information on Mentuhotep IFs advances further northwards is
lacking, but it seems unlikely that he had to fight every step of the way.
Instead, it is probable that the network of Herakleopolitan rule over
Middle Egypt collapsed after Asyut had been defeated, and local rulers
might then have been eager to side with the winning party before it
was too late, thus hoping to save themselves and their towns from 'the
terror which was spread by the [Theban] king's house'.
We do not know the fate of the last Herakleopolitan king nor the
details of the capture of the town of Herakleopolis, but recent excava-
tions in the cemetery of Ihnasya el-Medina show that its funerary
monuments were literally hacked to pieces at some point in the early
Middle Kingdom. It seems tempting to construe this archaeological
observation as evidence for the eventual sacking and pillaging of
Egypt's northern capital.
The First Intermediate Period in Retrospect
Modern Egyptologists still largely present a negative image of the First
Intermediate Period. It is characterized as a period of chaos, decline,
misery, and social and political dissolution: a 'dark age' separating two
epochs of glory and power. This picture, however, is based only partly
on an evaluation of sources contemporary with the period. It largely
reproduces—sometimes with surprising naivety—the literary theme
developed in a group of Middle Kingdom literary texts. The so-called
Admonitions of an Egyptian Sage and the Prophecy ofNeferti form the
core of this genre, but several other 'pessimistic' texts, such as the
THE FIRST I N T E R M E D I A T E P E R I O D 135
Complaints ofKhakheperraseneb and the Dialogue between a Man Tired of
Life and his 'ba', might also be added to this list. In this class of texts, a
state of disorder is lamented and contrasted with the way in which
things ought to be. Social order is turned upside down; the rich are
poor, and the poor are rich; political unrest and insecurity prevails
throughout the country; the administrative documents are torn to
pieces; there are numerous different rulers in power at the same time;
the country is invaded by foreigners; the moral basis of social life is
destroyed; people neglect and hate each other; and the sacred scrip-
tures are profaned. This state of general disturbance is not confined to
the social world: it attains truly cosmic dimensions in that the river is
sometimes said to be no longer flowing as it ought to do, and even the
sun is found not to have retained its former brightness.
It should be noted that these texts do not actually claim to be set in the
First Intermediate Period; nor do they mention any historical partic-
ulars. In the Prophecy of Neferti, the advent of Amenemhat I (1985-
1956 BC) is foretold as bringing relief from a state of chaos which must
be situated, chronologically, in the late nth Dynasty and not in the First
Intermediate Period. Careful scrutiny is, therefore, required if we are to
determine whether these texts bear any relation to the history of the
First Intermediate Period, and, even if they do, we need to investigate
precisely how they relate to the actual historical events.
Texts deriving from the First Intermediate Period itself are entirely
lacking in that very note of despair that is the hallmark of Middle King-
dom 'pessimistic' literature. They do talk about crisis, but crisis bril-
liantly overcome: vigour, self-confidence, and pride in one's own
achievement characterize the mood of the time. Certainly there are a
number of striking thematic similarities between First Intermediate
Period biographies and the Middle Kingdom pessimistic texts (such
as Nile failure, famine, social unrest, war, and a crisis affecting the
foundations of the state), but these similarities prove, in the first place,
literary connections between the two.
Another aspect of the textual evidence seems to be even more
important. In First Intermediate Period inscriptions, tales of crisis
served to legitimize the power of local rulers. In the same way, the
greatly elaborated picture of a period of utter chaos in the later pessim-
istic literature provides the black background against which the tight
politics of law and order implemented by Middle Kingdom kings can
be justified and even made to appear beneficent. The foundations of
the ruling ideology of Middle Kingdom monarchy, therefore, rest firmly
on what we know of First Intermediate Period political thought.
136 STEPHAN SEIDLMAYER
These comparisons between Middle Kingdom 'pessimistic' litera-
ture and First Intermediate Period contemporary texts reveal just how
deeply the experience of the First Intermediate Period affected the
Middle Kingdom Egyptians' collective consciousness and their views
on social and political relations. On the other hand, it would be
extremely misguided to attempt to use Middle Kingdom literary texts
as authentic sources for First Intermediate Period history. The view of
the First Intermediate Period presented in this chapter has been
entirely based on contemporary sources; this attempt to evaluate the
surviving documentation in all its aspects makes it much more dif-
ficult to subscribe to the traditional negative view of the period. In con-
trast, one can only be struck by the dynamism and creativity of the
period.
When Senusret I donated a statue of the 'count' Intef, the ancestor
of the nth Dynasty, to Karnak temple, he was acknowledging the
origins of Middle Kingdom kingship in the struggles that local rulers
fought for power and ascendancy during the First Intermediate Period.
Apart from its political importance, the impact that the First Inter-
mediate Period had on Egypt's cultural history cannot be denied. A
whole range of new morphological types was developed in nearly every
sphere of material culture, including such singularly successful new
inventions as the scarab-shaped seal.
Above all, however, popular culture was given the opportunity to
flourish at a time when the overpowering influence of court culture
had faded, and when there was a great weakening of central govern-
ment, which had previously (in the Old Kingdom) imposed heavy
demands on provincial communities. In the First Intermediate Period,
the local populations throughout the country enjoyed conspicuous, if
modest, wealth. They also acquired various new means of cultural
expression and communication, and were able to arrange their lives
within the small-scale horizon of their immediate concerns.
7
The Middle Kingdom Renaissance
(^2055-1650 BC)
GAE CALLENDER
Unlike the First and Second Intermediate periods, the Middle King-
dom (2055-1650 BC) formed a political unity, the core of which com-
prised two political phases: the nth Dynasty ruling from the Upper
Egyptian city of Thebes, and the i2th Dynasty centred in the region of
Lisht in the Faiyum. Earlier historians considered that the nth and
12th Dynasties marked the full extent of the Middle Kingdom, but
more recent scholarship shows clearly that at least the first half of the
so-called i3th Dynasty (which apparently bears no resemblance to a
proper political dynasty) belongs unequivocally to the Middle King-
dom. There was no shift of the location of the capital or royal residence,
little diminution in the activities of the government, and no decline in
the arts of the time—indeed, some of the finest works of Middle
Kingdom art and literature date from the i}th Dynasty. There was,
however, a decline in large-scale monumental building, a significant
indication that the ijth Dynasty was neither as strong nor as inspired
by the grandiose ideas that marked the reigns of the later 12th-Dynasty
rulers. Doubtless, this phenomenon was due to the brevity of reigns
for the majority of 13th-Dynasty kings, although the reasons for such
changes in the political picture are as yet unknown.
The simplest way to gain some sense of the general flavour of
Middle Kingdom history is to study the succession of kings and their
achievements, since they set the tone for the political and cultural
directions of the period. However, in pursuing this course, we are
forced to confront one of the biggest problems in our understanding of
138 GAE CALLENDER
Middle Kingdom history: the issue of the 'co-regencies' of the i2th-
Dynasty kings. Simply put, the question is: did some of these rulers
share the throne with their successors? Crucial elements in the debate
are so-called double-dated stelae, texts incorporating the names of two
successive kings, together with different dates for each of these rulers.
These stelae have left scholars divided on the issue of whether the
records represent a sharing of power by two pharaohs or merely the
years during which the owners of the stelae held office under each of
the two kings.
The standard chronology for the i2th Dynasty has been remodelled
over the years in the light of intensive studies of dated monumental
records. Some of this new work has yielded much shorter reigns than
those suggested by the fragmentary Turin Canon and the epitomes of
Manetho. The most controversial reigns are those of Senusret II and
III, and there are wide discrepancies between the proposed chron-
ologies of different scholars. Discoveries of certain 'hieratic control
marks' carved on the masonry of the monuments of Senusret III have
added further confusion to these chronologies, so that the dating
problems of the i2th Dynasty are still in a state of flux. Josef Wegner,
for instance, has provided very strong evidence for a reign of thirty-
nine years for Senusret III that—together with the discovery at Lisht of
references to a 'year 30' of Senusret III and evidence for his celebration
of a serf-festival (royal jubilee)—would argue for a much longer reign
by this king than most modern chronologies suggest. There are also
grounds for suspecting that the reign of Senusret II is more likely to
have lasted for nineteen years (as suggested by the papyri discovered at
the town of Lahun) rather than the shorter length of the revised chron-
ologies, but there is some difficulty in accommodating these expanded
reigns within the absolute dates proposed by some scholars. The evi-
dence for longer 12th-Dynasty reigns would fit in well with the co-
regency theory, which is based on the double-dated monuments, but
convincing arguments have also been put forward by a number of
scholars seeking to refute individual co-regencies such as those of
Amenemhat I-Senusret I, Senusret I-Amenemhat II, and Senusret
III—Amenemhat III.
Since there are no true 'absolute dates' yet established in Egyptian
history (apart from the radiocarbon-based chronologies) until the late
New Kingdom at the earliest, and since argument still persists regard-
ing the high, middle, and low dating schemes, there is room for
revision in the chronologies for all pharaonic periods. It is possible that
new archaeological material emerging from Tell el-Dab c a (see Chapter
THE M I D D L E KINGDOM R E N A I S S A N C E 139
8) will help to solve some of the problems in Middle Kingdom chron-
ology, but in the meantime the account given in this chapter leaves
co-regencies out of the equation, pending further proof.
The nth Dynasty
The first nth-Dynasty ruler to gain control of the whole of Egypt was
Nebhepetra Mentuhotep II (2055-2004 BC), who probably succeeded
Nakhtnebtepnefer Intef III (2063-2055 BC) on the Theban throne.
Mentuhotep's tremendous achievement in reuniting Egypt was recog-
nized by the ancient Egyptians themselves, and as late as the 2oth
Dynasty there were numerous private tombs containing inscriptions
celebrating his role as founder of the Middle Kingdom. The increase in
historical records and buildings, the evident prosperity of the land
during the latter years of his reign, and the resurgence and develop-
ment of all forms of art are particular indicators of his success in
restoring peace. It is sobering to reflect that, after such a promising
start, the nth Dynasty was to collapse only nineteen years after his
death.
Nebhepetra Mentuhotep II
Among the many rock carvings of various dates on the cliffs at Wadi
Shatt el-Rigal, 8 km. north of Gebel el-Silsila, there is a relief incor-
porating a colossal figure of the nth-Dynasty ruler Nebhepetra
Mentuhotep II dwarfing three other figures: his mother, his likely pre-
decessor Intef III, and Khety, the chancellor who served both kings.
This has long been taken as proof that Mentuhotep II was the son of
Intef III. Further such proof seems to be provided by a relief on a
masonry block from the site of Tod that portrays Mentuhotep II
towering over a line of three kings named Intef, lined up behind him,
thus again suggesting family connections with the Intefs as well as a
lengthy royal ancestry. This insistence on lineage', however, seems to
beg the question of Mentuhotep's actual origins, and it would not be
surprising to discover either that Mentuhotep had not been a royal son
or that these monuments were a deliberate attempt to counterbalance
claims made by the Herakleopolitan rulers as members of the 'House
of Khety' (see Chapter 6).
Mentuhotep II appears to have reigned quietly over his Theban
kingdom for fourteen years before the last phase in the civil war
between Herakleopolis and Thebes erupted. We know virtually noth-
ing of this conflict, but a graphic image of its savagery may well have
140 GAE CALLENDER
survived in the form of the so-called tomb of the warriors at Deir
el-Bahri, not far from Mentuhotep II's mortuary complex. The
unmummified linen-wrapped bodies of sixty soldiers, clearly killed in
battle and subsequently placed together in a rock-cut common tomb,
were preserved by dehydration. Despite the absence of any embalm-
ing, these corpses are the best preserved of all Middle Kingdom bodies.
Because they were buried as a group and within sight of the royal
cemetery, it has been surmised that they died in some particularly
heroic conflict, perhaps connected with the war against Herakleopolis.
The Herakleopolitan ruler Merykara died before Mentuhotep
reached Herakleopolis, and with his death Herakleopolitan resistance
must have collapsed, for Merykara's successor governed the northern
kingdom for only a few months. Mentuhotep's victory over the last
Herakleopolitan ruler provided him with the opportunity to reunite
Egypt, but we have only indirect knowledge about how long this took
and how severe such struggles were. This process may well have taken
many years, for there are scattered references to other fighting through-
out this stage of Mentuhotep's reign. One of the clues to the insecurity
felt at this time is the inclusion of weapons among the grave goods of
ordinary men; another is the depiction of administrators carrying
weapons instead of official regalia on funerary stelae. However, as
peace and material prosperity advanced within the country, such items
seem to have diminished in frequency.
Part of Mentuhotep's reconquest included forays into Nubia, which
had returned to native rule during the last stages of the Old Kingdom.
There was at least one line of native rulers controlling parts of Nubia at
the time when Mentuhotep II's armies descended upon them. An
inscription on a masonry block from Deir el-Ballas, thought to belong
to his reign, mentions campaigns in Wawat (Lower Nubia), and we
also know that a garrison was established by Mentuhotep in the fort-
ress at Elephantine, from which troops could more rapidly be deployed
southwards.
In addition to the emphasis on his lineage, part of Mentuhotep's
strategy to enhance his reputation with his contemporaries and suc-
cessors was a programme of self-deification. He is described as 'the
son of Hathor' on two fragments from Gebelein, while at Dendera and
Aswan he usurped the headgear of Amun and Min, and elsewhere
wears the red crown surmounted by two feathers. At Konosso, near
Philae, he took on the guise of ithyphallic Min. Both this iconography
and his second Horus name, Netjeryhedjet ('the divine one of the
white crown'), emphasize his self-deification. Evidence from his Deir
THE M I D D L E K I N G D O M R E N A I S S A N C E 141
el-Bahri temple indicates that he intended to be worshipped as a god in
his House of Millions of Years, thus pre-dating by hundreds of years
ideas that became a central religious preoccupation of the New King-
dom. It is evident that he was reasserting the cult of the ruler.
Mentuhotep's self-promotion was accompanied by a change of
name as well as this process of self-deification. His Horus name was
altered several times during his reign, each change evidently marking
a political watershed. Sematawy ('the one who unites the two lands')
was the last alteration, the earliest dated occurrence for this being
year 39. However, prior to year 39 the king had celebrated his sed-
festival, and perhaps this was the occasion when he took that name.
The government of the kingdom
Mentuhotep ruled from Thebes, which, until then, had not been a par-
ticularly prominent town in Upper Egypt. It was a good location from
which to exercise control over the remaining nomarchs (regional
governors), and most of Mentuhotep's officials were local men. The
scope of their duties was wide: the vizier, Khety, conducted campaigns
in Nubia for the king, while the chancellor, Meru, controlled the
Eastern Desert and the oases. The latter office was much more signifi-
cant than it had been in the Old Kingdom. In addition to the existing
post of 'governor of Upper Egypt', an equally powerful new post,
'governor of Lower Egypt', was created. This strengthening of the
central government increased the king's control over his officials while
simultaneously curtailing the power of the nomarchs, who had enjoyed
complete independence in the First Intermediate Period.
The numbers of nomarchs were probably reduced by Mentuhotep—
the governors of Asyut, for instance, certainly fell from power because
of their support for the Herakleopolitan cause. The nomarchs of Beni
Hasan and Hermopolis, however, retained control as before, perhaps
as their reward for assisting the armies of the Theban nomarchs. The
governors of Nag el-Deir, Akhmim, and Deir el-Gebrawi also remained
in office. The nomarchs' conduct, however, was monitored by officials
from the royal court, who moved around the land at regular intervals.
Another indication of a return to a strong and united Egyptian
government is found in the journeys being taken beyond Egyptian
borders. One of the famous expedition leaders of this time was Khety
(the official depicted on the Shatt el-Rigal relief described above), who
patrolled the Sinai area and also carried out assignments in Aswan.
Henenu, the 'overseer of horn, hoof, feather and scale', was the king's
steward; amid his numerous jobs, he travelled as far as Lebanon for
142 GAE CALLENDER
cedar for his master. Such journeys suggest that Egypt was beginning
to restore its influence in the outside world.
The building projects ofMentuhotep II
In addition to the numerous military campaigns launched by Mentu-
hotep in his fifty-one-year reign, he was also responsible for numerous
building projects, although most of these have been destroyed. New
temples and chapels were erected, the majority of these being located
in Upper Egypt at Dendera, Gebelein, Abydos, Tod, Armant, Elkab,
Karnak, and Aswan. A combined Dutch and Russian team has dis-
covered a Middle Kingdom temple near Qantir, in the eastern Delta.
Its architecture reflects that of Mentuhotep's mortuary complex at
Deir el-Bahri, but firm dates have not been published.
Throughout the Middle Kingdom, the royal cemeteries were contin-
ually evolving, not only architecturally, but structurally and spatially.
This constant change seems to reflect the search for a spiritual
solution to the question of what constituted the most effective type of
tomb, and this is very evident in Mentuhotep's mortuary monument,
at Deir el-Bahri in western Thebes. This was by far the most impres-
sive of his surviving buildings, although little remains of it today. The
temple design was unique, for neither of his nth-Dynasty successors
(Sankhkara Mentuhotep III and Nebtawyra Mentuhotep IV) com-
pleted their tombs, while the 12th-Dynasty kings chose monuments
inspired by Old Kingdom models. The saff-tomb (see Chapter 6) had
been the tomb design used by previous Theban rulers in the el-Tarif
region of western Thebes, but Mentuhotep's monument altered this
tradition. Even though some of its architects seem to have been pre-
viously involved in the construction of so^tombs, his complex reveals
a vision previously absent from both the Theban and Herakleopolitan
models; therefore it is rightly recognized as the most important build-
ing of the phase between the end of the Old Kingdom and the
beginning of the i2th Dynasty.
This inspiring symbol of the reunification of Egypt epitomized a
new beginning. It was, for example, the first royal structure overtly to
stress Osirian beliefs—a reflection of the religious levelling' between
the funerary cults of kings and commoners that had taken place in the
First Intermediate Period. Significant innovations in this temple were
the use of terraces, and the verandah-like walkways (or ambulatories)
that were added onto the central edifice. The design incorporated
groves of sycamore and tamarisk trees, which were planted in front of
the temple, each in a pit cut 10 m. down into the rock and filled with
THE M I D D L E K I N G D O M R E N A I S S A N C E 143
soil. A long, unroofed causeway ran up from this tree-lined court to the
upper terrace, upon which the central edifice was built. This main con-
struction may have taken the form of a square mastaba-tomb (perhaps
surmounted by a hill); behind it lay a hypostyle hall and the intimate
cult centre.
The tombs of the king's wives, Queens Neferu and Tern, were
included in the complex, the latter being buried in a dromos tomb at
the rear of his temple, the former in a separate rock-cut tomb on the
northern temenos wall in the forecourt. Several chapels and tombs for
six other women, four of whom are named as 'royal wife', were found
behind the central edifice, within the western walkway. Their original
burials belong to the earliest phase of Mentuhotep's temple. When
excavated, several of these tombs still contained their original burials,
as well as the earliest evidence for the use of models depicting both the
coffins and the bodies of the deceased—the precursors of the shabti
figures that became more popular at a later date. These women buried
in the western walkway seem to have been of lower status than Neferu
and Tern, and all of them were young: the eldest, Ashaiyet, was 22, and
the youngest, Mayt (whose badly destroyed chapel contains no indica-
tions of the title of'wife'), was only a 5-year-old child. The significance
of these less-important wives is uncertain; they may have been the
daughters of nobles whom the king wished to keep in check, but most
of them are named as priestesses of Hathor; therefore it has also been
suggested that their tombs may have formed part of a Hathoric cult for
the king within his mortuary monument. Another enigma is that the
burials appear to be contemporaneous. Did these young girls die
together in some disaster?
The six women's chapel tombs evidently belong to the same period
in the development of the Deir el-Bahri monument as the tomb known
as the Bab el-Hosan, which lies beneath the temple forecourt. This
royal tomb is thought by Dieter Arnold to have been an earlier and
incomplete burial for the king. It was in this structure that a black-
skinned statue in festival robes was found. The unusual skin colour is
another of the many references to Osiris, symbolizing the fertility and
regenerative powers of Mentuhotep II.
Although the temple was decorated throughout, not enough of its
art has survived to be able confidently to reconstruct the overall system
of design and decoration, although there are some distinct themes.
The king's supernatural and Osirian aspects are emphasized, but there
are also scenes from court life. The regional nature of the artwork is
evident in many of the surviving fragments of painted decoration, and
144
GAE C A L L E N D E R
such characteristic touches as thick lips, large eyes, and exaggeratedly
thin and awkward bodies are all apparent. However, there is also some
masterly carving (especially that from the chapels of the young wives),
which is more typical of the Memphite school. This mixture of tech-
niques reflects the political situation indicated by some of the crafts-
men's biographies, which also show that they came from various
regions of Egypt, bringing with them their local traditions. In time, the
Memphite style prevailed, but it was several generations before it
replaced the regional artistic genres throughout Egypt.
Although we cannot point to any monuments of Mentuhotep II in
the Temple of Amun at Karnak, there is a reference to the god in
Mentuhotep's temple, and the location of the latter in the curve of the
cliffs at Deir el-Bahri is itself significant, being directly aligned with
Karnak on the opposite bank. This position may have been intended to
allow the king's funerary cult to benefit from the annual visit of the god
Amun to Deir el-Bahri in a rite known as the Beautiful Festival of the
Valley. Certainly, the cult of Amun began to grow at Thebes from this
time onwards.
Mentuhotep III and IV
Queen Tern was the mother of Sankhkara Mentuhotep III (0.2004-
1992 BC), who was an energetic builder. In 1997 a Hungarian team led
by Gyoro Voros not only discovered a hitherto unknown Coptic sanc-
tuary below the peak of Thoth Hill, on the west bank at Thebes, but
also found an early Middle Kingdom tomb that surely belonged to
Mentuhotep III. Its architecture may have been the inspiration for the
bab-tombs of the early i8th Dynasty.
The reign of Mentuhotep III was characterized by a certain amount
of architectural innovation, including a triple sanctuary at the site of
Medinet Habu, which foreshadowed the i8th-Dynasty temples to
'family' triads. In addition, the remains of the brick temple that he
constructed on the 'hill of Thoth', the highest peak overlooking the
Valley of the Kings, not only contained another triple shrine but also
incorporated the earliest surviving examples of temple pylons. Not far
from the temple lie the remains of the sed-festival palace of
Mentuhotep III.
The art preserved from his brief reign is no less innovative, with the
relief sculpture arguably reaching its peak for the Middle Kingdom at
this point. The carving of the stone is extremely fine, the raised relief
conveying tremendous spatial depth with a differentiation of no more
than a few millimetres of thickness within the stone. The subtlety of
THE M I D D L E K I N G D O M R E N A I S S A N C E 145
the portraiture and details within clothing on his reliefs from Tod are
far superior to the sculptures of Mentuhotep II.
Mentuhotep III was also the first Middle Kingdom ruler to send an
expedition to the East African land of Punt to obtain incense, although
such expeditions to the Red Sea and Punt became more frequent in
the i2th Dynasty. Mentuhotep's expedition, led by an official called
Henenu, was sent via the Wadi Hammamat, thus apparently neces-
sitating the construction of ships on the shores of the Red Sea using
timbers that had been transported across with them. He also attempted
to protect the north-eastern border through the construction of fortifi-
cations in the eastern Delta.
When Mentuhotep III died, in about 1992 BC, there seem to have
been 'seven empty years', corresponding to the reign of Nebtawyra
Mentuhotep IV (who perhaps usurped the throne, since he is missing
from the king-lists). His mother was a commoner with no royal titles
other than 'king's mother', so he may not even have been a member of
the royal family.
Little is known of Mentuhotep IV's reign, except for his quarrying
expeditions. Inscriptions from the Hatnub travertine quarry suggest
that some of the nomarchs in Middle Egypt might have been trouble-
some at about this time. The most important event attested from his
reign was the sending of a quarrying expedition into the Wadi Ham-
mamat. Amenemhat, the vizier who led the expedition, ordered the
cutting of an inscription at the quarry to record two remarkable omens
that the party were said to have witnessed. The first was a gazelle who
gave birth to her calf on the stone that had been chosen for the lid of the
king's sarcophagus, and the second was a ferocious rainstorm that,
when it died down, disclosed a well, 10 cubits square, full of water to
the brim. In such barren terrain, this would certainly have been a
spectacular, even miraculous, discovery. It seems almost certain that
the man who became the first king of the i2th Dynasty was this same
Amenemhat. Like most of the nth-Dynasty high officials, he would
have held various powerful posts; it may have been either the weakness
of the king or the lack of a viable male heir that caused the throne to
pass to the vizier.
The 2th Dty me izm uynasiy
The much greater sophistication of the i2th Dynasty, compared with
the nth, is perhaps the factor that has persuaded so many scholars that
the Middle Kingdom only properly begins with the i2th Dynasty.
146 GAE CALLENDER
Amenemhat I
Sehetepibra Amenemhat I (Manetho's 'Ammenemes', £.1985-1956
BC) was the son of a man called Senusret and a woman called Nefret,
who came from outside the royal family, and the names Amenemhat,
Senusret, and Nefret were later to become popular with the I2th-
Dynasty kings and their wives. If Amenemhat the vizier really was the
same person as Amenemhat I, then his reporting of the two miracles
would appear to signal that he was the one for whom miracles were
performed. His contemporaries must have understood that this man
had been favoured by the gods.
The Prophecy of Neferty, a text which may have been composed at
about the time of the beginning of the reign of Amenemhat I, starts
with a list of problems in the land, then 'predicts' the emergence of a
strong king:
Then a king will come from the South,
Ameny, the justified, by name,
Son of a woman of Ta-Seti, child of Upper Egypt.
He will take the white crown,
He will wear the red crown;
He will join the Two Mighty Ones [the two crowns]
Asiatics will fall to his sword,
Libyans will fall to his flame,
Rebels to his wrath, traitors to his might,
As the serpent on his brow subdues the rebels for him.
One will build the Walls-of-the-Ruler,
To bar Asiatics from entering Egypt...
Since this early i2th-Dynasty 'prophecy' (the date of which is very
questionable) clearly refers to King Amenemhat, we once again have a
statement of divine intervention, calling attention to the king's super-
natural status. There are a number of other texts that refer to the chaos
before the arrival of new kings; however, the references to the Asiatics
in The Prophecy of Neferty are new, as is the reference to the Walls-of-
the-Ruler, a structure built by Amenemhat across the eastern approach
to Egypt. It was during his reign that the first definitely attested Middle
Kingdom military campaigns against the Near East were undertaken.
One of Amenemhat's most significant moves was to transfer
Egypt's capital from Thebes to the new town of Amenemhat-itj-tawy
('Amenemhat the seizer of the two lands'), sometimes known simply
as Itjtawy, a still-undiscovered site in the Faiyum region, probably near
the Lisht necropolis. The name of the city implies a rather violent
THE M I D D L E K I N G D O M R E N A I S S A N C E 147
beginning to the reign, but the precise date of the transfer to Itjtawy is
not known. Most scholars argue that it occurred at the beginning of
Amenemhat's reign, although Dorothea Arnold advocates a date much
later in his reign (around the twentieth year). While a case can be made
for Amenemhat spending some years at Thebes, the fact that the
building preparations on the platform near Deir el-Bahri identified as a
possible tomb for Amenemhat I probably only took about three to five
years suggests that the move may not have been as late as the twentieth
year of his reign. The negligible number of Theban monuments con-
structed by Amenemhat I, and the suspicious absence of official
burials after the time of Meketra (a high official buried in the vicinity of
the above-mentioned platform), may suggest that the move took place
in the earlier years of his reign. On the other hand, inscriptions on the
foundation blocks of Amenemhat's mortuary temple at Lisht show
firstly that Amenemhat I had already celebrated his royal jubilee, and,
secondly, that year i of an unnamed king (thought to be Amenemhat's
successor Senusret I) had already elapsed, thus suggesting an
extremely late date for the Lisht pyramid complex. For these reasons,
the date of the move to the Faiyum is still a source of considerable
debate.
The site of Itjtawy may have been chosen because it was closer to the
source of Asiatic incursions than Thebes had been, but it was also
politically wise for Amenemhat to found a new capital, thus signalling
a new beginning. It also meant that the officials who served him at
Itjtawy would have been entirely dependent on the king rather than
having their own power bases. This new beginning was celebrated in
the king's second choice of Horus name, Wehemmesu ('the renais-
sance', or, more literally, 'the repeating of births', perhaps an allusion
to the first of the 'miracles'). This was no empty phrase: the i2th
Dynasty looked back to the Old Kingdom for its models (for example,
the pyramidal form of the king's tomb and the use of Old Kingdom
styles of artistic decoration) and also promoted the cult of the ruler.
There was a steady but inexorable return to a more centralized govern-
ment, together with an increase in the bureaucracy. There was also an
exponential growth in the mineral wealth of the king, emphasized by
the jewellery caches found in several 12th-Dynasty royal burials. These
changes resulted in rising living standards for middle-class Egyptians,
whose level of wealth was proportional to their official posts.
Amenemhat's earliest use of the feudal armies was against Asiatics
in the Delta; the scale of these operations is unknown. He then
strengthened the region with the construction of the so-called
148 GAE CALLENDER
Walls-of-the-Ruler, which play a dramatic role in the Story of Sinuhe
and are also mentioned in the Prophecy ofNeferty. No fortress of this
date has yet been discovered at the north-eastern frontier of Egypt, but
the remains of a large canal may date from this period. Other fort-
resses are known to have been constructed elsewhere in Amenemhat's
reign, including one named Rawaty at Mendes, and the outposts of
Semna and Quban in Nubia, the purpose of which was mainly to
protect and service the gold mines in Wadi Allaqi.
Although the king and his conscript army pushed southwards as far
as Elephantine quite early in his reign, they do not appear to have been
active any further south before year 29. By this time, the policy towards
Nubia had been transformed from the loose network of sporadic
trading and quarrying ventures that characterized the the Old King-
dom to a new strategy of conquest and colonization, principally with
the aim of obtaining raw materials, especially gold. An inscription at
the lower Nubian site of Korosko, midway between the first and second
Nile cataracts, states that the people of Wawat (Lower Nubia) were
defeated in the twenty-ninth year of Amenemhat's reign. Only one
military foray against the Libyans is recorded; this is said to have taken
place in year 30, with the army under the command of the king's son
Senusret. By the time the Libyan campaign had ended, Amenemhat
was dead.
Senusret I
According to Fragment 34 of Manetho's history, a conspiracy took place
at the end of Amenemhat's reign. The Teaching of Amenemhat I also
hints at a dispute over the succession, and it was while Senusret
was campaigning in Libya that he was told of his father's death.
Amenemhat was almost surely murdered, and a text from Senusret I's
times presents the account supposedly spoken by his father from
beyond the grave:
It was after supper, when night had fallen, and I had spent an hour of happiness. I
was asleep upon my bed, having become weary, and my heart had begun to follow
sleep. When weapons of my counsel were wielded, I had become like a snake of the
necropolis. As I came to, I awoke to fighting, and I found that it was an attack of the
bodyguard. If I had quickly taken weapons in my hand, I would have made the
wretches retreat with a charge! But there is none mighty in the night, none who can
fight alone; no success will come without a helper. Look, my injury happened while
I was without you, when the entourage had not yet heard that I would hand over to
you when I had not yet sat with you, that I might make counsels for you; for I did not
plan it; I did not foresee it, and my heart had not taken thought of the negligence of
servants.
THE M I D D L E K I N G D O M RENAISSANCE 149
The manuscript from which this brief extract derives is thought to
be an early 12th-Dynasty composition, possibly created on behalf of
Senusret I to support his claim to the throne. The piece would very well
serve as a 'justification' for any punitive measures Senusret might
have taken after he gained the throne.
The king-lists give Kheperkara Senusret I (£.1956-1911 BC) a reign of
forty-five years, and this situation is backed up by a text at Amada, in
Nubia giving a date of year 44 for him. It has been accepted for some
time that Senusret I's reign consisted of thirty-five years of sole reign
and ten years of a co-regency shared with his father, but this assump-
tion was questioned by Claude Obsomer in 1995. If his claim is
correct, then we can at last make sense of the ending of The Teaching of
Amenemhat I, in which the king requests that Senusret succeed him.
This poetic request is only explicable if there had been no co-regency to
ensure the smooth transmission of the throne.
Senusret sent one expedition to Nubia, in the tenth year of his reign.
Eight years later, he dispatched another army as far south as the second
cataract. His general, Mentuhotep, went even further south, but it was
the site of Buhen that became Egypt's new southern border. Here
Senusret set up a victory stele, and constructed a fort, thus transform-
ing Lower Nubia into a province of Egypt. While Kush (Upper Nubia)
was mainly exploited for its gold, the Egyptians were also procuring
amethyst, turquoise, copper, and gneiss for jewellery and sculpture. In
the north, trading caravans passed between Egypt and Syria, exchang-
ing cedar and ivory for Egyptian goods. These more prolific expedi-
tions into Nubia and Asia show the extent to which foreign policy had
changed between the nth and the i2th Dynasties.
The king's numerous monuments were distributed from lower
Nubia in the south to Heliopolis and Tanis in the north, and it was in
order to obtain the raw materials for building, decorating, and equip-
ping these constructions that officials were sent to exploit the stone
quarries of Wadi Hammamat, Sinai, Hatnub, and Wadi el-Hudi. Just
one of these expeditions extracted sufficient rock to make sixty sphinxes
and 150 statues. The Egyptian Museum in Cairo includes a large col-
lection of statues of Senusret retrieved from his mortuary temple, but
many of his other monuments and statues were remodelled, copied,
and replaced by later kings, so that few of the originals have survived.
At Thebes, he is considered to have founded the temple of Ipet sut
(Karnak) and erected an Egyptian alabaster bark shrine celebrating his
serf-festival in the thirty-first year of his reign. The relief work of his
time was particularly fine, judging from such surviving fragments as a
150 GAE CALLENDER
damaged relief figure of the king from Koptos (now in the Petrie
Museum, University College London), but his statues lack vivacity and
movement, and the portraits are impersonal. Nevertheless, the effect
of this flurry of artwork had important results: because of Senusret's
long reign, the 'royal style' reached the regions with sufficient force to
cast its shadow throughout Egypt, and regional styles steadily retreated
before it.
Senusret was the first to introduce a construction programme
whereby monuments were set up in each of the main cult sites
throughout the land. This move, which was an extension of the policy
of later Old Kingdom pharaohs, had the effect of undermining the
power bases of local temples and priests. Today there are only a few
surviving remnants of the major sculptures and thematic works from
these regions, thus reducing our impression of the impact of Senus-
ret's programme. Among his more important measures, Senusret
remodelled the temple of Khenti-amentiu-Osiris at Abydos. Following
this royal impetus, the king's officials also erected numerous mem-
orial stelae and small shrines (or 'cenotaphs') at Abydos, thus inaugu-
rating a practice that was to become standard for devout men of means
in both Middle and New Kingdoms. Because of the attention Senusret
paid to the cult of Osiris, there was a great flowering of Osirian beliefs
and practices in Egypt, as well as a more significant levelling between
the king's belief in the afterlife and the beliefs of his subjects. John
Wilson has described this as the 'democratization of the afterlife'.
The 'Hekanakhtepapers'
By a stroke of good fortune, a collection of Middle Kingdom letters
provide us with many details of agricultural life at about this time. The
letters were written by an old farmer named Hekanakhte to his family,
while he was absent on business for a considerable period of time.
Although this material was, until recently, thought to date to the reign
of Mentuhotep III, the fact that the papyri were found in association
with pottery of the early i2th Dynasty suggests that they were actually
written in the early years of Senusret I.
Hekanakhte's personality emerges from these letters, which are full
of sharp commands to his several sons to do his bidding, to stop
whingeing about the slender rations he has allowed them, and to be
kind to his new wife. The letters provide a most intimate view of family
dynamics in the i2th Dynasty, as well as indicating some of the ways in
which richer farmers juggled their commitments and crops. They sug-
gest that there was famine in Egypt in Hekanahkte's later years, a
THE M I D D L E K I N G D O M RENAISSANCE 151
phenomenon also implied by the inscriptions in the roughly con-
temporary tomb of the nomarch Amenemhat at Beni Hasan (tomb
BH2).
The Hekanakhte papers include a rare letter from a woman to her
mother—a find that raises the question of the extent to which ancient
Egyptian women were able to read and write. Unfortunately, however,
this does not constitute definite proof, given that the woman in
question may have dictated the letter to a male scribe (as indeed many
illiterate male correspondents would have done) and the style of the
handwriting can provide no clues. References elsewhere to two Middle
Kingdom female scribes suggest that a few women may nevertheless
have been literate at this date.
Royal annals and the reign of Amenemhat II
Further information for the historical events of the i2th Dynasty
comes from a set of official records (known as genut or 'day-books') that
have been partly preserved in the temple at Tod. The king's building
dedications also contain elements of these annals; P. Berlin 3029, for
instance, describes the process by which the king founded a new
building. These are some of the most useful surviving texts in terms of
understanding the day-to-day world of the Egyptian palace. In addi-
tion, in 1974, the Egyptian Antiquities Organization discovered one of
the most important genut inscriptions at Mit Rahina (ancient Mem-
phis). Although the inscription mentions Senusret I, it clearly belongs
to the reign of his son, Nubkaura Amenemhat II (£.1911-1877 BC).
These annals contain very detailed descriptions of donations made to
various temples, lists of statues and buildings, reports of both military
and trading expeditions, and royal activities such as hunting. It is
undoubtedly the most important text of Amenemhat II, but it also
refers to other 12th-Dynasty kings; most importantly, it reveals that the
superficial 'peace' that was said to exist between Asia and Egypt at this
time was only selective, with a number of treaties existing between
Egypt and various individual Levantine cities. Herodotus' references to
Asiatic wars and to the attitude of contempt held by 'Sesostris' towards
the Asiatics (Histories 2.106) are thus perhaps closer to historical
reality than modern readers have tended to believe.
Wall paintings in the tomb of the nomarch Khnumhotep at Beni
Hasan (BH 3) depict the visit of a Bedouin chieftain named Abisha,
while numerous Egyptian statuettes and scarabs have been found at
Near Eastern sites, reaffirming such Asiatic links. There had long been
steady commerce with the Syrian port of Byblos, where the native
152 GAE C A L L E N D E R
rulers wrote short inscriptions in hieroglyphs, held the Egyptian titles
of count and hereditary prince, referred to Egyptian gods, and acquired
Egyptian royal and private statuary. In addition, the above-mentioned
Mit Rahina annals of Amenemhat II identify the north Syrian city of
Tunip as an Egyptian trading partner. Other Asiatic contacts appear to
have been more warlike. The annals refer to a small group of Egyptians
entering bedouin territory (probably a region of Sinai) in order to 'hack
up the land', and two more operations were directed against unknown
walled towns. The victims are described as Aamu (Asiatics), and 1,554
of them are said to have been captured as prisoners. These large
numbers of foreign captives may well explain the extensive lists of
Asiatic slaves working in houses in Thebes in later times. There were
also campaigns in the south at this time; thus the 'biography' in the
tomb of Amenemhat at Beni Hasan mentions that he went on an
expedition to Kush (Upper Nubia) and that the East African kingdom
of Punt was visited by the king's official, Khentykhetaywer, in the
twenty-eighth year of Amenemhat II.
Unlike so many of the 12th-Dynasty rulers, Amenemhat II does not
appear to have had a prolific building record, although this impression
may be partly a result of later plundering. His pyramid complex, the
so-called White Pyramid at Dahshur (poorly preserved and not yet
thoroughly examined), was unique in being set on a platform. His
daughters were buried in the forecourt and a queen called Keminebu
was also buried within the complex. It was long thought that Kemi-
nebu was Amenemhat's wife, but it is now recognized, on the basis of
her name and the style of her inscriptions, that she was actually a i3th-
Dynasty queen.
Senusret II and the inauguration of the Faiyum irrigation system
The reign of Amenemhat II's successor, Khakheperra Senusret II
(1877-1870 BC), was a time of peace and prosperity, when trade with
the Near East was particularly prolific. There are no records of military
campaigns during his reign; instead, his greatest achievement appears
to have been the inauguration of the Faiyum irrigation scheme. A dyke
was built and canals were dug to connect the Faiyum with the water-
way that is now known as the Bahr Yusef. These canals siphoned off
some of the waters that normally would have flowed into Lake Moeris,
resulting in a gradual evaporation of waters around the edges of the
lake, the canals extending the amount of new land; the reclaimed land
was then farmed. This was a far-sighted scheme, and would have been
unique for its time, if it were not for the fact that land was reclaimed
THE M I D D L E K I N G D O M R E N A I S S A N C E 153
with a similar system of dams and drainage canals in the Copaic Basin
of Boeotia, in central Greece, in the Middle Helladic Period (c.iqoo-
1600 BC).
We do not in fact know how many of these irrigation works are to be
ascribed specifically to the reign of Senusret II, but his connection
with the overall revival of the Faiyum is probably indicated by the fact
that he erected religious monuments at the edge of the region. The
unique statue shrine of Qasr el-Sagha in the desert at the north-eastern
corner of the Faiyum has been dated to around his reign by associated
pottery. Like other buildings of his reign, however, this one was left
undecorated and incomplete, thus contributing to the impression that
he enjoyed only a short reign. The use of various sites in the Faiyum
for royal pyramid complexes from this time onwards perhaps indicates
the importance of the irrigation scheme, since it is usually assumed
that the royal palaces of each ruler would have been built close to their
funerary monuments.
A small group of statues of Senusret II are known, at least two of
which were usurped by Rameses II (1279-1213 BC). The wide, muscu-
lar shoulders are reminiscent of the statues of Senusret I, although the
influence of Old Kingdom royal statuary is also apparent. The facial
appearance of Senusret II is more vigorous and plastic, with none of
the blandness that typified the statuary of his i2th-Dynasty predeces-
sors: his broad cheekbones and small mouth are very distinctive, and
probably indicative of actual portraiture, foreshadowing the startling
studies of Senusret III (1870-1831 BC). The customary imitation of a
royal trend by the wealthy members of society subsequently occurred,
including many vivid examples of individuality among the private
statuary of the late i2th Dynasty. The reign of Senusret II perhaps
deserves to be regarded as the first major phase of human portraiture
in the history of Egyptian art.
Even better known than the king's statuary are a pair of highly
polished black granite statues of Queen(?) Nefret now in the Egyptian
Museum, Cairo. Larger than life, they depict a royal woman whose
position at court is still uncertain. Although Nefret did not hold the
title of'royal wife', she possessed other titles held by queens. Was she
perhaps the first wife of Senusret II, who may have died prior to her
husband's fairly late accession, or was she his sister? As with many
Egyptian queens, the records concerning Nefret are ambiguous and
incomplete. In 1995 the remains of his chief wife, Khnumetnefer-
hedjetweret, were discovered in the pyramid of her son (Senusret III)
at Dahshur, together with a few items of jewellery.
154 GAE CALLENDER
Senusret II built his funerary complex at Lahun, the pyramid being
a massive mud-brick structure with a rocky core; large limestone
cross walls provided support for the brick sectors, which were then
cased in limestone. Trees were planted at the southern end of the
complex, and the entrance to the pyramid was also on the south. The
layout of corridors and rooms within the pyramid is unique, and may
reflect beliefs concerned with Osiris and the afterlife. It is suspected
that another tomb, very well made and situated on the northern side of
the complex (Tomb 621), could be a cenotaph like those in Old King-
dom royal complexes. The female members of the king's family may
be represented by eight solid mastaba-tombs and a satellite pyramid;
all of them were aligned with the northern side of the king's tomb, but
they were apparently symbolic structures rather than actual burial
places. In a shaft tomb at the southern end of the king's pyramid
enclosure, the jewellery and other possessions of Princess Satha-
thoriunet were found by Petrie and Brunton in 1914; the workmanship
of these pieces is among the best in the entire repertoire of Egyptian
jewellery.
Nubian conquest under Senusret III
Although the Turin Canon gives Khakaura Senusret III (0.1870-
1831 BC) a reign of over thirty years, the latest regnal year recorded by
dated sources is 19. On the other hand, discoveries during the 19905
may support the longer date (see chronological discussion at the
beginning of the chapter). There is no real evidence for a co-regency
with Senusret II, but if one could be proven it would help resolve some
difficulties caused by the unusually long reign.
Senusret is perhaps the most Visible' monarch of the Middle King-
dom: his exploits gathered renown over time and substantially con-
tributed to the character of 'Sesostris' (a kind of composite heroic
Middle Kingdom ruler) described by Manetho and Herodotus. The
king campaigned in Nubia in regnal years 6, 8,10, and 16, and these
wars appear to have been very brutal: Nubian men were killed, their
women and children enslaved, their fields burnt, and their wells
poisoned. Soon after this, the Egyptians were again mining and
trading with the inhabitants, but conditions had changed. In the eighth
and sixteenth regnal years, stelae were set up in the fortresses of
Semna and Uronarti, at what appears to have been the southern
border, with their inscriptions reminding everyone of Senusret's con-
quest and punishments. This frontier region was sealed off by
reinforcing the huge fortresses, and guards were placed on round-the-
THE MIDDLE KINGDOM RENAISSANCE 155
clock duty, waiting for any movement. The year 8 stele at Semna states
that no Nubians are allowed to take their herds or boats to the north of
the specified border.
These forts emphasize the unsettled nature of the Egyptian control
of Nubia. The so-called Semna dispatches—a set of military letters and
accounts sent from Semna to Thebes in the i3th Dynasty—reveal just
how rigorously the Egyptians policed the native people. They also show
how closely these forts kept in contact with each other. Although the
major forts were of comparable size, they fulfilled several different
functions. Some, such as Mirgissa, were more involved with trade than
others (bread and beer were exchanged for the native goods), while
some (such as Askut) appear to have been used as supply depots for
campaigns into Upper Nubia. Reports were sent backwards and
forwards from the forts to the vizier, and in this way the king kept in
touch with the limits of his domain. Senusret's final Nubian cam-
paign, in year 19, was of lengthy duration and it was ultimately not
particularly successful: the king had to retreat when the water level in
the river dropped alarmingly, making journeys dangerous.
He undertook at least one campaign into Palestine, apparently
similar to the expedition sent by Amenemhat II against the Aamu
(Asiatics). There appear to have been large numbers of Asiatics in
Egypt by this date; some of them were prisoners taken earlier, but the
biblical account of Joseph's brothers selling him as a slave to an Egypt-
ian master (Gen. 37: 28-36) may suggest another way in which some
of these immigrants arrived. Egyptian intolerance toward the
'easterners' was already apparent in the reign of Senusret I, who
described himself as 'the throat-slitter of Asia', and this general per-
ception is reinforced by the so-called execration texts. These were lists
of enemies inscribed on pottery objects and figurines, many of which
name individual Asiatics and the people of Asia in general. The inten-
tion of the texts seems to have been to ensure the magical destruction
of Egypt's enemies by burying or smashing the pots or figurines in
question.
Senusret also took a different direction in his political reforms.
Although he has often been credited with the dismantling of the
system of nomarchs, there is no real evidence to support this assertion
(see section on political change below). Nevertheless, his attempts to
pull Egypt back into a more centralized form of government resulted
in significant political and social readjustment (especially for the
middle classes), and his reign is quite rightly regarded as a crucial
watershed in Middle Kingdom history.
156 GAE CALLENDER
The tomb of Senusret III, a 60-metre-high mud-brick pyramid,
cased with limestone blocks, was located at Dahshur, like that of
Amenemhat II. Mastaba-tombs were built for his immediate family
within the temenos wall, but their real burials lay below ground in
galleries, one level for queens, the other for princesses. Dieter Arnold
has shown that this complex takes some of its ideas from the 3rd-
Dynasty step-pyramid complex of Djoser at Saqqara. The burial cham-
ber has a vaulted ceiling and is built of granite plastered over with
white gypsum. Neither the king's chamber nor the sarcophagus appears
to have been used. However, at the southern end of Abydos a second
funerary complex was constructed for Senusret, consisting of a sub-
terranean tomb and a mortuary temple, where a cult for the king lasted
for over two centuries. Some scholars suspect that the Abydos complex
may have been his actual burial place, but no remains were found
there either.
Amenemhat III: the cultural climax of the Middle Kingdom
Senusret's only known son was Nimaatra Amenemhat III (£.1831-
1786 BC). It was arguably around the time of his long and peaceful
reign that the Middle Kingdom reached its cultural peak. Consolida-
tion of what had gone before appears to have been the hallmark of
Amenemhat's style of government. He strengthened the Semna border
and enlarged some of the fortresses. Other building works included
numerous shrines and temples, and a huge structure at Biahmu (in
the north-western Faiyum), featuring two colossal quartzite seated
statues of the king facing onto the lake, which was later described by
Herodotus (2. 149). He also constructed a large temple to Sobek at
another Faiyum site, Kiman Faras (Crocodilopolis), and expanded the
Ptah temple at Memphis. The surviving statues of Amenemhat III are
striking, distinguished both by their originality and their workman-
ship, as in the case of a small head of the king now in the collection of
the Fitzwilliam Museum, Cambridge, which is one of the most elegant
and subtle of his many portraits. His so-called Hyksos sphinxes and
parts of shrines were found reused in the Third Intermediate Period
temples at Tanis, as were the twin black granite statues of the king in
the guise of the Nile god, bringing offerings offish, lotus flowers, and
geese—a design later imitated by such New Kingdom rulers as
Amenhotep III (1390-1352 BC).
Numerous inscriptions record Amenemhat Ill's mining activities.
In the Sinai region alone, where the king's officials worked the tur-
quoise and copper mines on a quasi-permanent basis, fifty-nine graf-
THE M I D D L E KINGDOM RENAISSANCE 157
fiti have been identified. The quarries at Wadi Hammamat, Tura,
Aswan, and various Nubian sites were also worked. All this building
and industrial activity symbolizes the prosperity that Egypt enjoyed
during the reign, but it may also have exhausted the economy and,
combined with a series of low Nile floods, late in his reign, resulted in
political and economic decline. Ironically, the large intake of Asiatics,
which seems to have occurred partly in order to subsidize the extensive
building work, may have encouraged the so-called Hyksos to settle in
the Delta, thus leading eventually to the collapse of native Egyptian
rule.
Before the construction of the modern dams at Aswan and the
creation of Lake Nasser, the annual Nile inundation was critical for
Egypt's food supply. Amenemhat's records of inundation levels at
Kumma and Semna in Nubia are numerous, revealing extremely high
levels for the Nile during part of his reign, the highest being in year 30,
when it reached 5.1 m. But these levels subsequently tapered away
sharply, so that in year 40 the level was only 0.5 m. Such fluctuations
must have had a destabilizing effect on the economy. Since the Faiyum
is the only oasis in Egypt dependent upon the Nile, the Faiyum irriga-
tion scheme would have needed to draw on some of the annual flood
water, thus perhaps explaining the king's apparent intense interest in
the flood levels. Alternatively, the high Niles may have been closely
watched in order to allow possible flood damage in the north to be
averted. Amenemhat III certainly maintained the Faiyum scheme,
and later peoples worshipped him as Lamares, the god of the Faiyum,
but, as with Senusret II, it is not clear just how much hydraulic work
was undertaken in his particular reign. His deification may have taken
place as early as the reign of his successor, Queen Sobekneferu, for she
had the most to gain from elevating the man who was probably her
own father.
Amenemhat built his first pyramid at Dahshur, but, as in the case of
the 4th-Dynasty Bent Pyramid of Sneferu, cracks appear to have devel-
oped in the structure during building. The finished pyramid had a
mud-brick core and was originally encased in limestone (now robbed);
its stone pyramidion is in the Egyptian Museum, Cairo. The remains
of Queen Aat and another royal woman were found in two recently
discovered corridors inside the south-western section of the pyramid.
Their crypts were provided with separate entrances outside the pyra-
mid, a feature that would have enabled access after the main entrance
to the pyramid had been sealed. Queen Aat's sarcophagus is identical
to that of the king.
158 GAE CALLENDER
The two queens' burial chambers at Dahshur each included a sep-
arate 'ka chamber' where the canopic chests were placed. This was a
type of funerary room that had once been the privilege of kings, thus
presumably representing a rather specialized aspect of the so-called
democratization of the afterlife (see section on religion below); it is
likely that these chambers expressed new beliefs concerning the after-
life of royal women. Their corridors were linked with the king's, and
they would have shared the tomb with him if it had not been for the
structural faults that developed.
Amenemhat's final resting place, however, was at Hawara in the
south-eastern Faiyum. His most famous monument was the mortuary
temple attached to this pyramid, which may have been reminiscent of
Djoser's sed-festival court, attached to his pyramid at Saqqara. The
Hawara temple became known as the Labyrinth because of its maze of
rooms and corridors. Although described by six classical writers, includ-
ing Herodotus (2. 148-9), Strabo (17. I. 3, 37, 42) and Pliny (Natural
History 36. 13), no details of its plan were coherent even when Petrie
made his survey in 1888; therefore efforts at reconstructing its original
appearance have been unsuccessful. Amenemhat's burial chamber at
Hawara was originally intended to be shared with Princess Neferuptah,
who was probably his sister, but she was later transferred to a small,
separate pyramid (now almost totally destroyed by stone-robbers and
water damage) a few kilometres away. The prominence of Neferuptah
both during his reign and after her death, together with the mortuary
privileges provided for her and for the two queens at Dahshur, suggests
the increased status of royal women in the late i2th Dynasty.
Amenemhat IV and Sobekneferu
Given the long reign of Amenemhat III, there is a possibility that
Maakherura Amenemhat IV (1786-1777 BC) might have been his
grandson, but it is also possible that this last male ruler of Dynasty XII
was an aged son whose life was nearing its end when he came to the
throne, for he ruled for only nine years. He is likely to have been
married to Queen Sobekkara Sobekneferu (1777-1773 BC), whom
Manetho says was his sister. Few of his monuments have been pre-
served and little is known of events during his reign, which may have
been primarily occupied in completing several temples begun by his
predecessor, such as the limestone sanctuary of the harvest goddess,
Renenutet, at Medinet Maadi in the south-western Faiyum. There
were also continued expeditions to the turquoise mines in Sinai and
further trade with the Levant.
THE M I D D L E KINGDOM RENAISSANCE 159
There are only a handful of records relating to the last ruler of the
i2th Dynasty, Queen Sobekneferu, but some of them offer very inter-
esting clues relating to her reign. She is listed in the Turin Canon;
there is a Nile graffito at the Nubian fortress of Kumma giving the
height of inundation at 1.83 m. in the third year of her reign; and there
is a fine cylinder seal bearing her name and titulary, which is now in
the British Museum. Usually, the queen uses feminine titles, but
several masculine ones were also used. Three headless statues of the
queen were found in the Faiyum, and a few other items contain her
name. She contributed to Amenemhat Ill's 'Labyrinth', and also built
at Herakleopolis Magna.
There is an interesting, but damaged statue of the queen of unknown
origin; the costume on this figure is unique in its combination of
elements from male and female dress, echoing her occasional use of
male titles in her records. This ambiguity might have been a deliberate
attempt to mollify the critics of a female ruler. An intriguing statuette
of Sobekneferu in the Metropolitan Museum, New York shows the
queen wearing a sed-festival cloak and a most unusual crown, which
may have resulted from the attempt to combine unfamiliar icono-
graphic elements of male and female rulers. The queen's reign lasted
for less than four years, and her tomb—like Amenemhat IV's—has
not yet been identified.
The i3th Dynasty
The rulers of the i3th Dynasty continued to use Itjtawy as their capital,
and carried on the policies of the 12th-Dynasty rulers; but the new
dynasty was made up of different lineages, and the question of how the
king might have been chosen is unresolved. Stephen Quirke has sug-
gested a 'circulating succession' among leading families, which would
help to account for the great brevity of most of the reigns. Never-
theless, the bureaucracy continued to function in the same way as it
had done throughout the i2th Dynasty. The Egyptians still controlled
the area around the second Nile cataract, Nile floods were measured,
trade still flourished, and royal monuments continued to be built
(although these were far less impressive than those of the great i2th-
Dynasty kings). The visual arts, on the other hand, show no alterations
in finesse or style from the best of the i2th-Dynasty pieces. This con-
tinuity—broken at times—lasted until the reign of Neferhotep I.
Although many i3th-Dynasty names have been recorded in the
Turin Canon, we know little about the individual rulers. Wegaf Khuta-
160 GAE CALLENDER
wyra was the first, followed by Khutawy-Sekhemra Sobekhotep II.
After the reign of the third king, Sankhtawy-Sekhemra lykhernefert-
Neferhotep, Nile records were not kept for some time, and it is possible
that this might have been a period of political unrest: it is perhaps
significant that there are also few records during this time at the Sinai
turquoise mines. Nevertheless, trading contacts continued, and the
ruler of Byblos still described himself as the 'servant of Egypt'.
Sealings from the Nubian forts show that affairs to the south ran as
before. It is to this period that King Awibra Hor belongs; his burial—a
mere shaft tomb—was discovered by Jacques de Morgan in the
mortuary complex of Amenemhat III at Dahshur. In spite of the
cultural continuity remarked above, nothing so clearly expresses the
reduced circumstances of the rulers at this time as the impoverished
nature of the tomb of Awibra Hor.
After this brief, unsettled period, a series of less ephemeral kings
emerged, including Sekhemra-Khutawy Sobekhotep II, to whose reign
is dated a most interesting papyrus revealing details of Theban court
life for a period of twelve days. Stephen Quirke's analysis of this docu-
ment (Papyrus Bulaq 18) has revealed a great deal about the hierarchi-
cal structure of the i3th-Dynasty palace and its modus operandi. Some
four reigns later, in about 1744 BC, Sekhemra-Sewadjtawy Sobekhotep
III became king, and for a time it seemed that there would be a revival
in the fortunes of Egyptian rulers. A relief carved in the cliff above Nag
Hammadi, in Middle Egypt, provides very specific information about
members of the king's family. His highest regnal date was year 5,
although the Turin Canon gives him only three years and two months;
despite this brevity, he left dated inscriptions on monuments from the
Delta site of Bubastis down to Elephantine in the south.
Sobekhotep Ill's successor, Khasekhemra Neferhotep I (0.1740-
1729 BC), was also evidently of non-royal stock, but he too left many
monumental records, suggesting that his reign was a vigorous one. He
was acknowledged as overlord by Inten, the ruler of Byblos, and his
inscriptions have been found as far south as the island of Konosso, just
south of the first cataract in Nubia. Despite these tokens of strength,
however, he was not in control of the entire Egyptian kingdom, judg-
ing from the evidence of local rulers governing independently at Xois
and Avaris in the Delta.
The throne passed to the two brothers of Neferhotep I, Sahathor and
Sobekhotep IV, followed by the brief reign of the son of Sobekhotep
IV. This mini-dynasty ended with Sobekhotep V in about 1723 BC.
Nevertheless, enough evidence has survived of the reign of Sobekho-
THE MIDDLE KINGDOM RENAISSANCE
tep IV to suggest that he had all the hallmarks of a strong king, and
continued to hold some control over Nubia, where two of the king's
statues were found south of the third cataract (other statues of this king
survived reused at Tanis). It was, however, during the reign of
Sobekhotep IV that the first signs of revolt emerged in Nubia, which
was eventually to slip out of Egyptian control, to be ruled instead by a
line of Nubian kings based at Kerma (see Chapter 8). By that time,
Middle Kingdom Egypt had broken up into those spheres of interest
that formed the basis for government in the Second Intermediate
Period.
Processes of Political Change in the Middle Kingdom
Government in the Middle Kingdom was loosely based on the struc-
ture created under the Old Kingdom, but there were significant varia-
tions. The bureaucracy and the Crown were supported by taxation,
although little direct information concerning the latter has survived
from Middle Kingdom sources. The fiscal system was essentially
based on assessment of yields from lands and waterways, and paid in
kind. Temples and other pious foundations were often tax exempt in
part, if not in full (see below). In addition, there was a system of
enforced labour, whereby men and women of the middle and lower
classes were enlisted to undertake specific physical tasks, including
military service. This corvee system was organized through town
officials, but there was central control under the office of the 'organiza-
tion of labour'. Although it was possible to escape the burden of the
work legitimately by paying another person to do it, those who avoided
the corvee altogether were punished very severely, and their families,
or anyone aiding their evasion, were also punished. Records from the
fortress at Askut in Lower Nubia show that this was one place to which
evaders of the corvee could be sent; no doubt other defaulters were sent
to the quarry regions.
The corvee practice continued into the zyth Dynasty, and only the
people of Nubia appear to have been exempt from both taxation and
corvee impositions. For its part, the government kept the peace at
home and patrolled the borders north of the second cataract and west
of the Walls-of-the-Ruler. By means of raids into Palestine, and cam-
paigns in Nubia, the Middle Kingdom rulers were able to extend
Egypt's influence and prosperity. Trade was the king's monopoly,
supervised by state officials, and in Nubia the rewards were extremely
substantial.
161
162 GAE CALLENDER
Many of the titles held by Middle Kingdom officials were the same
as those used in the Old Kingdom, but there were also a number of
additional posts. One of the noticeable characteristics of the Middle
Kingdom was a refining of official titles into more specific offices and
duties, which must have been part of a general growth in the bureau-
cracy, although the range of activities within each office would have
become more restricted. An exception to this narrowing of duties was
that of 'royal sealbearer', who was given wide supervisory duties,
especially under Mentuhotep II. The vizier, whose responsibilities are
enumerated in a New Kingdom funerary text from the tomb of
Rekhmira (The Duties of the Vizier), was still the chief minister under
the king, albeit less prominently in the records after the nth Dynasty.
The practice of having two viziers is uncertain for the Middle King-
dom, although under Senusret I there do seem to have been two
(Antefoker and Mentuhotep) who were serving at the same time.
The scant source material of the later Middle Kingdom suggests that
there were other political changes between the Old and Middle King-
doms: central government in the Middle Kingdom was much more
pervasive in the regional areas (whereas there is little evidence for this
in the Old Kingdom). There was also more control over individuals and
the obligation that each was deemed to owe the government. This
more intense intrusion into private life may be partially attributed to
the Middle Kingdom custom of delegating so much local control to the
mayors of the towns, but there was a marked change, too, in bringing
the provinces into line with the styles and practices of the capital.
Artwork is the most visible indicator of this phenomenon.
It was, however, the office of nomarch that experienced the widest
fluctuation of all during the Middle Kingdom. Thanks to their distance
from Memphis, the earlier nomarchs had always enjoyed a certain
amount of independence in the Old Kingdom. This independence
was strengthened by the collapse of the Memphite government, and a
major goal of the Middle Kingdom rulers was to minimize it. Different
kings chose different strategies to effect their policy.
Under Mentuhotep II the nomarchs were retained in many of the
areas for which we have records (although much of this kind of evi-
dence has not survived), but it appears that those nomarchs considered
unhelpful to the Thebans would have automatically lost their positions.
Throughout the nth Dynasty the nomarchs played their traditional
roles, but they were now supervised by the king's officials. Many of
those who retained power still had delusions of grandeur: Count
Nehry of the Hermopolite nome, for instance, dated his inscriptions to
THE M I D D L E K I N G D O M R E N A I S S A N C E 163
his own 'reign' during the time of Mentuhotep IV, and his statements
at the Hatnub quarry strongly suggest challenges to the king.
The basic plan adopted by Amenemhat I was to make the individual
town the focus of administration. Each town was controlled by a mayor,
and only the chief official in the most important towns inherited the
position of nomarch. With this concentration on the city as the basic
unit of government, the political impact of the larger region of the
nome now declined. Amenemhat I's nomarchs held the titles of'great
overlord, mayor and overseer of priests', and were mainly located in
the central and border regions of Egypt. The key factor in royal control
over these men seems to have been the fact that, in the first two i2th-
Dynasty reigns at least, they were all personally appointed by the king
(although by the time of Amenemhat II the office had once more
become hereditary).
Such nomarchs made the most of their positions, some of them
adapting titles for their own staff that imitated those at the royal court:
one can find here and there a 'treasurer', a 'chancellor', and even an
army captain in household retinues. Despite these pretensions, the
great overlords were not allowed to forget their benefactor, the king,
who had organized them in quasi-feudal fashion: they owed him direct
allegiance and, in return for royal favours, they were obliged to protect
the borders of Egypt, to undertake expeditions for the king, and prob-
ably to act as deputies for official receptions of foreigners, such as the
arrival of the Bedouin traders in the Oryx nome, depicted in the tomb
of Khnumhotep at Beni Hasan (BH 3), in the reign of Amenemhat II.
The main title of the nomarch, 'great overlord', disappeared about
the time of Senusret III, and the conventional view is that this hap-
pened as a result of force majeure by the king. The real reason, however,
is likely to have been somewhat different: by the time of Senusret III,
only the nomarchs of el-Bersha and Elephantine are still definitely
recorded as holders of the office of 'great overlord' (other areas were
controlled by mayors, but records of many towns are missing, so we
cannot be entirely certain). Detlef Franke has demonstrated that the
practice under Senusret II was for the king to educate the sons of
nomarchs in the capital, and then to give them appointments either at
the capital or in other areas. With the family scions dissipated in this
way, the office of nomarch would eventually have been eclipsed by that
of the town mayors, who would inevitably not have enjoyed the same
material wealth and power as the provincial governors. This would
explain why the era of richly decorated provincial tombs came to an
end. Senusret III is thus unlikely to have been the instrument of the
164 GAE CALLENDER
nomarchs' demise, for the record shows that, although the office
eventually expired under Senusret III, it had been in decline since at
least the time of Amenemhat II.
Nevertheless, Senusret III did install other officials (based at the
royal court) as governors of very large sections of the country, and in
this way made a sharp break with practices of the past. Two bureaus
(waret) were created, one each for the northern and southern areas of
Egypt, operated by a hierarchy of officials. Other departments, such as
the 'treasury', the 'bureau of the people's giving', and the 'organization
of labour', were also inaugurated. The military sector was organized
under a chief general, and there was a new 'bureau of the vizier'. In
addition to these state departments, there was a separate administra-
tion for the palace. As a result of this new hierarchy, there were also
fresh titles, and a corresponding increase in the size of the middle-
class bureaucracy, which was reflected in greater numbers of funerary
stelae for this period, a visible marker of the greater affluence of the
middle class.
Outside the governmental boundaries were the estates of the temple
and its dependencies. As the contracts for the mayor Djefahapy of
Asyut reveal, this was an equally bureaucratic world. Djefahapy's ten
contracts—which have survived because they were inscribed on an
inner wall of his tomb—were drawn up to ensure that his mortuary
cult would be maintained after his death. Apart from the legal implica-
tions, the contracts also reveal some of the other conditions applying to
the temple, such as the fact that each person in the district was
required to give the temple a hekat (nearly 5 litres) of grain from every
field on his property on the occasion of the first harvest each year. The
contracts are very specific, indicating that the temples were self-
supporting, and that they too had to pay taxes to the Crown, unless they
received an exemption decree from the king. Senusret I's policy of
building provincial temples throughout the land effectively reduced
the local temples' power bases.
The Royal Court
Very few explicit statements concerning the role of the pharaoh have
survived from the Old Kingdom, but there are some Middle Kingdom
texts that shed light on the nature of kingship, such as The Teachings/or
Merykara, The Teaching of Amenemhat I, and the Hymns to Senusret III.
Some private records, too, can provide insights, as in the case of a long
THE M I D D L E K I N G D O M R E N A I S S A N C E 165
poem on the stele of Sehetepibra from Abydos (Egyptian Museum,
Cairo), which describes the importance of the king to his people.
The concluding episodes of The Tale ofSinuhe (describing the return
of an Egyptian courtier from exile) supply details of 12th-Dynasty court
life, but it is the i3th-Dynasty Papyrus Bulaq 18 that provides the most
revealing evidence concerning the social hierarchy of the royal family
and the quantities of daily rations given out, thus indicating the rela-
tive importance of these and other palace dependants. This papyrus
also indicates the fluidity of the movements of different people, with
their sojourns away from the palace proper. With regard to the palace
complex itself, the papyrus indicates that there were three inner divi-
sions within its precincts. In descending order of importance, these
were: the kap, or nursery, which was the domain of the royal family,
their personal servants and select children being educated at the king's
expense; the wahy, or audience area of the columned hall, a place
where banquets were held; and the khenty, or outer palace, where the
business of the court was conducted. These three groups of buildings
were set within a less august area known as the shena, where pro-
visions were handed out to the palace dependants. The vizier and
senior officials occupied the khenty, while serving staff were restricted
to the shena. The interior overseer of the kap appears to have been the
only official who could operate in both the inner and outer parts of the
palace. Without the information in Papyrus Bulaq 18, our knowledge
of Middle Kingdom palace organization would barely extend further
than the architectural plans of a i2th-Dynasty palace at Tell Basta and
an early i3th Dynasty palace at Tell el-Dab e a in the Delta.
Urban Life: The Pyramid Town at Lahun
The lives of ordinary people are accessible to us via the town of Hetep-
Senusret beside the pyramid complex of Senusret II at the site of
Lahun. Mistakenly named 'Kahun' by Flinders Petrie, who excavated
there in 1888-9, it was closely associated with the funerary cult of
Senusret II. Laid out in a single architectural plan like the much
smaller New Kingdom walled villages at Amarna and Deir el-Medina
(see Chapters 9 and 10), Hetep-Senusret was founded to accom-
modate the king's workers and their families. It is likely, however, that
it included among its inhabitants many who were not connected with
the funerary cult. It has been estimated, on the basis of the capacities
of grain silos throughout the town, that a population as high as 5,000
could have been supported. The modern site, however, is barely
166 GAE CALLENDER
distinguishable from the surrounding desert, since the mud brick has
been almost entirely removed, leaving only the foundations and lower
courses of the buildings.
The material from Lahun is particularly precious because it derives
from the living world rather than the necropolis (although Middle
Kingdom settlements have been excavated more recently at Abydos,
Memphis, and Elephantine, allowing the Lahun material to begin to be
viewed in a much broader geographical and social context). Unfortu-
nately, much of the material left behind at Lahun, when it was first
abandoned in the ijth Dynasty, was thrown into huge rubbish pits by
the post-Middle Kingdom occupants of the site. Thus a great deal of
the precious context of the material was destroyed long before the site
was excavated. Nevertheless, some houses were left comparatively
undisturbed, and these have the potential to provide glimpses into the
lives of the kinds of individuals who tend not to feature in the surviving
textual and funerary material. Thanks to Percy Newberry's collection
of seed types during Petrie's expedition, it has even proved possible to
recreate the vegetation of the area (despite a certain amount of con-
tamination by Graeco-Roman botanical material). There were flowers
such as poppies, lupins, mignonette, jasmine, heliotrope, and irises
(as well as weeds), and vegetables, including peas, beans, radishes, and
cucumbers.
The material from Lahun also includes such intriguing finds as a
'firestick' for lighting fires (probably the only surviving Egyptian
example), the earliest known mud-brick mould (identical to those used
by Egyptians today), a set of doctor's instruments, and many other
tools used by farmers and professional craftworkers. There was also a
rich variety of pottery, and a large number of papyri (some still unpub-
lished), the contents of which shed light on many areas of religion and
daily life. Among the most interesting of the texts from Lahun is the
so-called Gynaecological Papyrus, which, as its name suggests, com-
prises the oldest surviving collection of remedies for women's ail-
ments.
Foreign Trade
Commercial contacts between Middle Kingdom Egypt and the Aegean
are indicated by a few sherds of Minoan pottery in the 12th-Dynasty
phase of the Lahun settlement, as well as a pyxis lid and fragments of
local Egyptian pottery that were clearly imitating Minoan types.
Because these sherds were found in refuse deposits, however, it is
THE M I D D L E K I N G D O M R E N A I S S A N C E 167
difficult to be sure of their dates or their original stratigraphic contexts.
Curiously, they appear to have been common vessels used by the work-
men (rather than luxury imports), perhaps even indicating the presence
of foreign workers from Crete among the town's population. In the
12th Dynasty, there are also a few deposits of sherds from Minoan
'Kamares vessels' at such sites as Lahun, el-Haraga, and Abydos, and
in a i2th-Dynasty grave as far south as Elephantine. Numerous items
from this time also reveal the presence of a Mediterranean network of
artistic and iconographic exchange: Egyptian motifs can be found
on items as far-flung as the dedicatory clay scarab beetles offered in
the peak sanctuaries in parts of Crete. Egyptian stone vases also
made their way to Crete, where their styles were imitated by Minoan
craftsmen. Although such local imitations of Egyptian styles and icon-
ography are often from undated contexts, they are nevertheless
important in that they suggest frequent contact leading to exchange of
ideas as well as materials and products.
At Lahun and Lisht, there is also early evidence for the distinctive
Tell el-Yahudiya ware (see Chapter 8), comprising jugs that perhaps
once contained Near Eastern oil. The Egyptian kings actively promoted
imports of timber, oil, wine, silver, and perhaps ivory from Syria-
Palestine. Both Cypriot and Minoan pottery are also attested from
other occasional finds in Egypt. Egyptian goods, such as scarabs,
statues, vases, jewellery, and even several sphinxes have been found in
sites as far apart as Byblos, Ras Shamra, and Crete. Via Syria, further
contacts were made with Cyprus and Babylon, but very little of this
material comes from properly dated contexts.
Increasing contact with the Near East is suggested by the fact that
Asiatic weights actually outnumber Egyptian ones at Lahun. In addi-
tion, one of the richest finds of the Middle Kingdom is a collection of
Asiatic (or perhaps Minoan) gold and silver material discovered in four
bronze caskets underneath the temple of Montu at Tod. Conversely,
Pierre Montet discovered a hoard of 1,000 Egyptian items buried in a
jar at the Syrian city of Byblos, including jewellery bearing a strong
likeness to the 'treasure' from the tombs of 12th-Dynasty princesses in
the Lahun necropolis. Neferhotep and other Egyptian rulers were
acknowledged as overlords by the local rulers of Byblos, who not only
copied Egyptian insignia and titles but also imitated Egyptian hiero-
glyphic inscriptions.
There were also strong contacts with the areas to the south of Egypt.
Apart from their activities in Nubia, many of the Middle Kingdom
rulers, particularly Mentuhotep III and Senusret I, maintained trading
l68 GAE CALLENDER
links with the African region of Punt (probably located somewhere in
the vicinity of modern Eritrea). The i2th-Dynasty port of Sa c waw has
been discovered at the eastern end of the Wadi Gawasis, on the Red
Sea coast (a short distance to the north of modern Quseir), and several
inscribed stelae, found both along the wadi and at the port itself,
provide records of 12th-Dynasty journeys to Punt.
Religion and Funerary Practices
The most important developments in Middle Kingdom religion con-
cerned the cult of Osiris, who had by then become the Great God of all
necropolises. One of the reasons for the cult's growth was the patron-
age lavished on it by the Middle Kingdom rulers, especially at Abydos
in the i2th Dynasty. This climaxed in the reign of Senusret III, whose
'cenotaph' at Abydos was the first royal monument to be erected there
in the Middle Kingdom. A decree from the time of the i3th-Dynasty
ruler Wegaf (usurped by Neferhotep I) forbids tombs to be built on the
processional way at Abydos. Sobekhotep III also erected stelae for
several members of his family there, and Neferhotep I went to Abydos
to take part in the mysteries of Osiris in the second year of his reign,
erecting a stele to commemorate this event. Given the potency of Osiris
and Abydos in terms of legitimizing royal power, the 13th-Dynasty
rulers' interest in Abydos may have been due to their mainly non-royal
background, but the same cannot be said for the i2th-Dynasty rulers.
The growing influence of Osiris must have derived to some extent
from active promotion of the site of Abydos and the so-called mysteries
of Osiris. Some details of this set of rites are mentioned on a i2th-
Dynasty stele (now in the Berlin Museum) that was set up at Abydos by
Ikhernofret, the organizer of the annual festival during the reign of
Senusret III.
The growth of the Osirian cult was accompanied by a cultural phe-
nomenon sometimes described as the 'democratization of the after-
life': the extension of once-royal funerary privileges to ordinary people.
Large numbers of stelae at Abydos in particular show that it was
becoming increasingly common for private individuals to take part
in the rites of Osiris, thus receiving blessings that had once been
restricted to kings. As a result of this development, the funerary beliefs
and rites of the entire population began to change. One of the earliest
such changes was the practice of decorating non-royal coffins with the
Coffin Texts, a combination of extracts from the royal Pyramid Texts
with new funerary compositions, which emerged during the First
THE M I D D L E K I N G D O M RENAISSANCE 169
Intermediate Period (see Chapter 6). In the mid-i2th Dynasty, how-
ever, the use of these texts suddenly ceased, primarily as a result of
further funerary changes, such as the introduction of the mummiform
coffin, which, because of its more irregular shape, was less suited to
long inscriptions of religious text.
Another religious development of the Middle Kingdom was the idea
that all people (not just the king) had a ba, or spiritual force. The most
evocative evidence for this is the literary text, the Dialogue between a
Man Tired of Life and his 'Ba', which must be the world's earliest debate
on the issue of suicide—a powerful philosophical treatise. There was
also a noticeable emphasis on 'personal piety' (that is, direct personal
access to deities rather than via the king or priests, a religious concept
that further increased in popularity during the New Kingdom). Stelae
from the Middle Kingdom stress the piety of their deceased owners,
and out of this grew the concept of the 'negative confession' (ritual lists
of misdemeanours that the deceased claimed not to have committed).
Stelae themselves became popular memorials, especially those decor-
ated with wed/at-eyes, the paramount symbol of protection, but other
insignia (the shen-ring and the winged sun-disc, for instance)—like
those found on royal stelae—also appeared during this period.
Royal mortuary complexes of the nth and i2th Dynasties also
underwent considerable changes in design as the kings sought for the
most appropriate architectural form to reflect their religious beliefs.
Engineers and architects reached great heights of mastery, and the
masons exceeded the considerable skills of their Old Kingdom
counterparts. These skills were put to use not only in the service of the
royal complexes, but also in the creation of larger and more skilfully
constructed temples. In this period we find the complex internal
engineering of the royal pyramids and structural experimentation in
architecture, such as the terraced ambulatories of Mentuhotep II at
Deir el-Bahri, the pylons and triple shrine of Mentuhotep III on the
'hill of Thoth' at Thebes, and the galleries of Senusret II in his pyramid
at Lahun. Relief carving, previously found only in Old Kingdom mor-
tuary complexes, now decorated the walls of Middle Kingdom temples
for the gods, as well as kings. It was also in this period that the vast
temple complex at Karnak was inaugurated, and the once command-
ing temples of the Faiyum were built.
Experimentation is also to be found in the regional tombs of the
nomarchs from the nth Dynasty onwards: they indicate the world-view
of these officials, with their interests in hunting, fishing, and wrestling
matches, and their fascination with the exotic world of the Asiatics.
170 GAE CALLENDER
The large, lavishly decorated rock-cut tombs usually featured pillared
facades, and the tombs themselves were elevated above the burials of
the members of their 'courts', scattered across the slopes below. The
nomarchs' coffins—especially those from Deir el-Bersha—carry the
finest artwork of all those that have survived. In a number of instances
they are decorated with the earliest copies of the Book of Two Ways,
which was a set of instructions for safely reaching the afterlife. As the
nomarch's office diminished in importance, however, the character of
the provincial necropolis changed: the size and number of the smaller
tombs increased and there was less overt 'ranking' among the posi-
tions of graves. In the capital, on the other hand, things were rather
different: the officials' tombs were located in the royal necropolises
rather than their local family cemeteries, the mastaba-tombs became
the preferred style of private tomb, and the provision of a memorial at
Abydos became imperative for all.
By the Middle Kingdom, mummification had become much more
widespread, but it was not effective. Although evisceration was more
common, bodies were badly mummified and the residual flesh has
seldom survived, despite the fact that their exterior wrappings were
often lavish. The mummies were given cartonnage masks, which were
often beautifully painted, and the bodies were placed on their sides in
rectangular coffins oriented with regard both to the main compass
points and to the texts written on their tomb walls.
A further significant change in funerary practices was the intro-
duction of the shabti, a word that is sometimes also spelled ushabti or
shawabti, and that may mean 'stick', 'answerer', or perhaps both.
Shabtis were statuettes made from a variety of materials (wax, clay,
pottery, faience, wood, or stone) which were meant to act as magical
substitutes when work had to be done by the tomb-owner for Osiris.
The earliest examples, dating to the time of Mentuhotep II, often took
the form of small naked figures with no funerary formulas written on
them, while others were mummiform in shape. These figurines were
evidently three-dimensional reminders of Coffin Text spell 472, which
appeared inside a few Middle Kingdom coffins. By the late i2th
Dynasty, however, the text had begun to be written on the shabti itself.
It is thought that the role of the shabti might be linked either with the
corvee system by which each individual was obliged to work for the
king, or with the work that ordinary people had to carry out in the
maintenance of their local waterways. Like human workers, the later
shabtis carried hoes and seed bags with which to undertake their tasks.
THE M I D D L E K I N G D O M R E N A I S S A N C E 171
The Cultural Achievements of the Middle Kingdom
The Middle Kingdom was a time when art, architecture, and religion
reached new heights but, above all, it was an age of confidence in
writing, no doubt encouraged by the growth of the 'middle class' and
the scribal sector of society, which was in turn due in no small measure
to the expansion of the bureaucracy under Senusret III. Many different
literary forms flourished, and the ancient Egyptians themselves appear
to have regarded it as the 'classical' era of literature. Such narratives as
the Story of Sinuhe (the popularity of which is indicated by the many
copies that have survived), The Shipwrecked Sailor, and the fantastical
episodes in Papyrus Westcar were all composed in the Middle King-
dom, while religious and philosophical works (such as the Hymn to
Hapy, the Satire of the Trades, and the Dialogue between a Man Tired
of Life and his 'Ba*) were also very popular. Furthermore, a wide variety
of official documents have survived, including reports, letters and
accounts, which not only help to round out the overall picture of the
period but also indicate that literacy was more widespread than it had
been during the Old Kingdom.
Under the direction of the Middle Kingdom rulers, Egypt had its
eyes opened to the wider world of Nubia, Asia, and the Aegean,
benefiting from the exchange of materials, products, and ideas. The
Middle Kingdom was an age of tremendous invention, great vision,
and colossal projects, yet there was also careful and elegant attention to
detail in the creation of the smallest items of everyday use and decora-
tion. This more human scale is present in the pervading sense that
individual human beings had become more significant in cosmic
terms, whether in terms of their obligations to the state (through
taxation and the corvee work), their provisions for burial, or their
increased presence within the literature of the times. Neither Sinuhe
nor the 'shipwrecked sailor' could ever have been central characters
in any Old Kingdom tale, but these individuals sit comfortably in the
literature of the Middle Kingdom, which was an age of greater
humanity.
8
The Second Intermediate Period
(^.1650-1550 BC)
J A N I N E BOURRIAU
The Second Intermediate Period is defined by the division of Egypt—
the fragmentation of the Two Lands. 'Why do I contemplate my
strength while there is one Great Man in Avaris and another in Kush,
sitting united with an Asiatic and a Nubian while each man possesses
his slice of Egypt'. This was the complaint of the Theban King Kamose
(1555-1550 BC) at the end of the iyth Dynasty.
The beginning of the Second Intermediate Period is marked by the
abandonment of the Residence at Lisht, 32 km. south of Memphis, and
the establishment of the royal court and seat of government at Thebes,
the Southern City. The end of the period came with the conquest of the
capital of the Hyksos kings at Avaris in the eastern Delta by Ahmose,
King of Thebes. The reunification of Egypt which Ahmose achieved
was not to be broken again for over 400 years. The time between these
two events was approximately 150 years. The final pharaoh at Lisht was
probably Merneferra Ay (£.1695-1685 BC) because he is the last ruler of
the i3th Dynasty (following the sequence given in the Turin Canon
king-list) who has inscribed monuments in both Upper and Lower
Egypt. The conquest of Avaris can be dated much more closely,
between years 18 and 22 of Ahmose, 1532-1528 BC on the chronology
used here.
In the course of a mere six generations (each calculated as twenty-
five years), profound cultural and political changes took place, but the
disunity of Egypt meant that they happened in different ways and at
different rates in the various regions. Rather than presenting the
THE SECOND I N T E R M E D I A T E PERIOD 173
history of the period as a single narrative, therefore, it seems better to
describe it from the vantage point of each of the principal regions of
Egypt, from north to south. The regions can only be defined by our
sources and, given the gaps in the evidence, it is likely that the country
was even more fragmented than we currently think. It is only after the
beginning of the war between the Hyksos and Theban kings, eventu-
ally involving the whole of Egypt, that a single historical narrative
seems appropriate.
The written sources present peculiar problems, due to abundance
rather than scarcity, but the difficulty of integrating what they tell us
with the archaeological evidence remains profound. They may be
divided into six categories: the king-lists, of which the most detailed is
the hieratic papyrus known as the Turin Canon (compiled from pre-
existing lists at Memphis during the reign of Rameses II); Manetho's
Aegyptiaca, a history written in the third century BC but surviving only
in fragments excerpted by later chroniclers; contemporary and non-
contemporary royal inscriptions written as 'propaganda', but for that
reason, creating a vivid and dramatic mise-en-scene; contemporary
private inscriptions, particularly 'funerary biographies'; the records of
administration, both public and private; and, finally, literary and scien-
tific texts such as Papyrus Sallier I and the Rhind Mathematical Papy-
rus. Such texts are always valuable, but ambiguities can be introduced
because the most significant ones, the royal inscriptions, have often
been removed from their original contexts. Most of the Theban royal
stelae were found broken and reused in later buildings, while at Avaris
none of the inscribed stone elements in the monumental mud-brick
buildings of the Hyksos kings has been found in the stratum to which
it originally belonged.
Archaeological sources have their own pitfalls, the most funda-
mental being gaps in the record either through poor survival or patchy
excavation. No sites of the period have been excavated in the central
or western Nile Delta, or in Middle Egypt between Maiyana and Deir
Rifa. The mud-brick fortresses of the second-cataract region in Lower
Nubia tell the history of relations between Egypt and Kush, but, after
only partial excavation in the 19608 UNESCO campaign, they were
lost in the waters of Lake Nasser. What remains is a large but sporadic
patchwork of information. The adoption of a regional approach to
the evidence serves to emphasize a recurring theme in Egypt's history:
the rivalry between Upper and Lower Egypt, which was at its most
extreme in the battle between Thebes and Avaris at the end of the
period.
174
J A N I N E BOURRIAU
The Territory of Avaris
The question that lies at the heart of the Second Intermediate Period is
the nature of the Hyksos. Most histories depend upon written sources,
and, with few exceptions (the Rhind Papyrus is one), these emanate
from the Egyptian side. There is no Hyksos counterpart to the Kamose
texts. What we have instead is evidence from the systematic excavation
of their capital, Avaris (Tell el-Dab c a). We now know what their palaces,
temples, houses, and graves looked like, and we can observe how their
culture evolved through time, but the Hyksos were not a single or
simple phenomenon.
Aamu was the contemporary term used to distinguish the people of
Avaris from Egyptians. It was used long before the Second Intermedi-
ate Period and was still in use long after (Rameses II, for instance, uses
it of his opponents at Kadesh) in order to denote, in a general sense,
the inhabitants of Syria-Palestine. Egyptologists conventionally trans-
late aamu as 'Asiatics' (that is, inhabitants of Western Asia). The term
'Hyksos', on the other hand, derives, via Greek, from the Egyptian
epithet hekau khasut, 'rulers of foreign (lit. mountainous) countries'
and was applied only to the rulers of the Asiatics. In itself it held no
pejorative meaning except to denote a lower status than that of the
Egyptian king, and was used both by the Egyptians and by the Hyksos
kings of themselves.
Plan of the historical landscape of Tell el-Dab e a (site of the Hyksos capital, Avaris).
Excavated areas shown black
THE SECOND I N T E R M E D I A T E P E R I O D 175
When their etymology can be established, all private and royal per-
sonal names of Asiatics in Egypt at this time derive from West Semitic
languages. Earlier suggestions that some were Human or even Hittite
have not been confirmed. References to Asiatics are numerous in the
Middle Kingdom: they worked in a variety of occupations, sometimes
adopting Egyptian names while retaining the designation 'Asiatic'
(aamu). These immigrants were thought to be economic migrants, but
an inscription of the 12th-Dynasty ruler Amenemhat II records, in
unmistakable language, a campaign by sea to the Lebanese coast that
resulted in a list of booty including 1,554 Asiatics. Such campaigns fit
the archaeological evidence from Tell el-Habua, which shows that the
eastern border of Egypt was as heavily fortified as the southern one.
Tell el-Habua is a large site situated to the east of Tell el-Dab e a, dating
from the Middle Kingdom onwards. Mohammed Maksoud, the exca-
vator, has found traces of a major installation, probably a fort, judging
from the thickness of the walls, underneath settlement strata of the
Second Intermediate Period. By analogy with the Nubian second-
cataract forts, patrols would doubtless have gone out to the surround-
ing desert recording in dispatches sent to the capital the movement of
all people wishing to 'cross down into Egypt'.
There is evidence from Tell el-Dab c a that a community of Asiatics,
albeit very Egyptianized, existed there as early as the early i3th Dyn-
asty. So far, however, this is the only convincing archaeological evi-
dence for a population of Asiatics within Egypt (but living differently
from the Egyptians) during the Middle Kingdom. There are also ref-
erences in contemporary texts to 'camps of Asiatic workmen'.
It is likely that the oldest settlement at Tell el-Dab c a, which dates to
the First Intermediate Period, was deliberately built as a component in
a defensive system constructed to protect the eastern boundary. During
the late i2th and early i3th Dynasties the site expanded enormously,
including the emergence of a settlement populated by Asiatics. The
non-Egyptian character of the community is evident from the layout of
the houses (apparently following a Syrian model) and from the fact
that tombs were integrated with the living areas rather than being
placed in a cemetery outside the settlement. Not only are there differ-
ences in material culture, defined by pottery and weapon types, but the
nature of the burials indicates a mixture of Egyptian and Palestinian
traits. From a robber's pit cut into a tomb chapel come fragments of an
over-life-size limestone statue of a seated man holding a throwstick;
the artistic style and the clothes are non-Egyptian, but the size indi-
cates a person of the greatest importance. Ironically, the best parallel to
176 J A N I N E BOURRIAU
this statue is a tiny wooden figure from a Middle Kingdom tomb at
Beni Hasan, depicting an Asiatic woman and her baby.
In the next stratum (d/i), Middle Bronze Age culture becomes more
pronounced, and tombs include burials of donkeys, sometimes in
pairs. Other finds include an impression of a cylinder seal in North
Syrian style, fragments of Minoan Kamares ware pottery, and a gold
pectoral of two opposed hunting dogs, also thought to be Minoan.
Such objects, together with the 'ordinary' testament of Middle Bronze
Age imported pottery and Egyptian imitations, confirm the mixed
character of the settlement. The origins of these Asiatics—if they had a
single origin—are not easy to determine. The Asiatic culture was
certainly heavily adulterated by the underlying Egyptian one, the bulk
of the pottery was Egyptian (although dropping from 80 per cent to 60
per cent by stratum d/i) and the administration, judging by the titles of
officials on scarabs, was carried out on the Egyptian model. Parallels
for the foreign traits have been found at southern Palestinian sites
such as Tell el-Ajjul, at the Syrian site of Ebla, and at Byblos (in modern
Lebanon). In a study of the non-Egyptian pottery from Tell el-Dab c a,
Patrick McGovern has postulated that most of it originated from the
cities of Southern Palestine. Since the wealth of the late Middle
Kingdom town at Tell el-Dab e a centred around the seagoing trade
along the Levantine coast, the caravan route across northern Sinai to
Palestine (and perhaps also expeditions to the turquoise mines), the
idiosyncratic culture of its inhabitants should not surprise us.
The culture of the people of Tell el-Dab e a is not static but rapidly
develops new traits and discards old ones. This makes the characteriza-
tion of each stratum in terms of its architecture, burial customs, pottery,
metal and other artefacts relatively clear, but leaves unanswered the
question of why and how this cultural mixing and rapid development
took place. One hypothesis is that the basic population of Egyptians
received from time to time a new influx of settlers, first from the region
of Lebanon and Syria, and subsequently from Palestine and Cyprus.
The elite among them married local women—a suggestion supported
by preliminary study of the human remains, although bone preserva-
tion is poor.
Tell el-Dab c a has provided hundreds of artefacts that can be recog-
nized as belonging to the well-known period of the Middle Bronze Age II
A-C of Syria-Palestine. This material is found in nine strata (H-D/2),
the upper and lower ends of which have been linked by the Austrian
excavator Manfred Bietak to the reigns of two Egyptian kings, respec-
tively Amenemhat IV (1786-1777 BC) and Ahmose (1550-1525 BC). He
THE SECOND I N T E R M E D I A T E P E R I O D 177
divides the resulting period by nine, allotting roughly thirty years to
each stratum and thus obtaining a framework of absolute dates for his
relative sequence. However, when these dates have been imported to
sites in Syria-Palestine where objects similar to those from Tell el-
Dab'a have been found, there have sometimes been clashes with the
existing chronology. The resulting fierce debates, when resolved, will
eventually demand radical revisions not only in the dating of strata at
Tell el-Dab c a but in the methods used for dating the Middle Bronze
Age over the whole east Mediterranean region.
The initial expansion of Tell el-Dab e a was checked temporarily by an
epidemic. In several parts of the site, Bietak has found large com-
munal graves in which many bodies were placed, without any discern-
ible ceremony. Thereafter, from stratum F onwards, the patterns of
both settlements and cemeteries suggest a less egalitarian society than
before. Large houses with smaller ones fitted in around them, more
elaborate buildings in the centre than on the edge of the settlement,
servants buried in front of the tombs of their masters: all suggest the
social dominance of a wealthy elite group.
At this point in the city's history, its identification with the textually
attested Hyksos capital of Avaris becomes clear. Two limestone door
jambs were found naming the 'good god, lord of the two lands, son of
Ra of his body, Nehesy'. Inscribed fragments from Tell el-Habua,
Tanis, and Tell el-Muqdam provide further titles and epithets of this
man, 'beloved of Seth, lord of Avaris, eldest king's son'. The last epi-
thet is a title that implies high military rank but does not mean that the
holder was literally 'son of the king'. The reference to the god Seth
shows that his cult was already established, and that he was the local
god of Avaris, just as Amun was the patron deity of Thebes. Seth's cult
may have evolved from a blending of a pre-existing cult at Heliopolis
with a cult of the North Syrian weather-god Baal Zephon, which was
introduced by the Asiatics.
Nehesy is listed in the Turin Canon in the group generally identified
as the i4th Dynasty, the capital of which—according to Manetho—was
Xois in the western Delta. Nehesy was a high official who for a short time
(no regnal years are known) assumed a royal status at Avaris. Probably
Nehesy was an Egyptian, or perhaps a Nubian (the literal meaning of
Nehesy); nothing in his inscriptions suggests otherwise. The king whom
he originally served was probably still reigning from the city of Itjtawy,
near Lisht, which was not abandoned until after 1685 BC, although
Sobekhotep IV (£.1725 BC) was the last really powerful king of the i}th
Dynasty. After Sobekhotep's reign, it is likely that the unity of Egypt began
178 J A N I N E BOURRIAU
to break up, and an obvious candidate for elevation into an independent
kingdom was the region around the rich and powerful city of Avaris.
How far did the authority of King Nehesy extend? If we judge from
the sites where his name occurs, his territory appears to have encom-
passed the eastern Delta from Tell el-Muqdam to Tell el-Habua, but the
universal practice of usurpation and quarrying of earlier monuments
complicates the picture. Given that the only examples that were certainly
found at the sites where they once stood are those from Tell el-Habua
and Tell el-Dab e a, his kingdom may actually have been much smaller.
One of the Second Intermediate Period burials at Tell el-Dab e a seems
to confirm that the structure of Egyptian bureaucracy still existed at
Avaris. The tomb owner is identified by a scarab on his finger as the
Deputy Treasurer, Aamu ('the Asiatic'). His burial was an extremely
wealthy one, but it was characterized by several non-Egyptian traits: the
body lay in a contracted position (not extended, as is normal for Egyptian
burials), the weapons and pottery were of Syro-Palestinian type, and in
front of the tomb five or six donkeys had been buried. Such a high official
would normally be interred close to his king, having expected to spend
his life close to the royal residence, the seat of government, which, for
him, was Avaris.
If the Danish Egyptologist K. S. B. Ryholt's reconstruction of the
Turin Canon is accepted, in the columns allotted to the group of kings
which include Nehesy, there are 32 names, 17 lost names and two gaps,
one covering the five predecessors of Nehesy and one of unknown
length indicated by the scribe as present in the earlier manuscript
from which the Turin Canon was copied.
For all except five of the named kings, the length of the reign is
either missing or given as less than one year. Apart from Nehesy, only
three of them appear elsewhere: Kings Nebsenra and Sekheperenra on
a jar and a scarab respectively, and King Merdjedefra, who is shown on
a contemporary stele, accompanied by 'the seal-bearer of the King, the
treasurer, Renisoneb'. The findspot is unknown, but the eastern Delta,
and more precisely Saft el-Hinna, about 30 km. north of Tell el-
Yahudiya, has been suggested. The king is shown offering to Soped,
Lord of the East, a god whose sphere was the desert routes to the Red
Sea and the turquoise mines of Sinai. His cult centre in the 22nd
Dynasty was Saft el-Hinna. The stela of Merdjedefra has significance
beyond confirming the existence of a minor king, because it confirms
that the names of the i4th-Dynasty kings are not fictitious, although
they are unlikely to represent a single line of kings ruling one after the
other from the same place.
THE SECOND I N T E R M E D I A T E P E R I O D 179
The inscription of Nehesy is the first contemporary evidence of the
fragmentation of the Egyptian kingdom. According to Bietak, Nehesy
fits into the relative chronology of Tell el-Dab c a at stratum F (or b/3),
corresponding to the late ijth Dynasty. Thereafter, no single ruler was
able to control the whole of Egypt, until the conquest of Avaris. Over
105 royal names are preserved from the period, and most of these
occur in the Turin Canon. The implication of this is that records were
kept at Memphis of the names of all these kings, however short their
reigns, and however localized their rule. Ryholt's painstaking recon-
struction of the damaged papyrus uses fibre matches as well as textual
analysis and as a result we have a much more coherent record. The
royal names now fall into four groups which correspond to Dynasties
14—17 of Manetho. Dynasties 14 and 15 were based in the eastern Delta
with their capital at Avaris (although the i5th Dynasty also controlled
part of Egypt south of Memphis, see below) and Dynasties 16 and 17
were centred at Thebes in Upper Egypt. The fragmentary nature of the
papyrus allows for more than one interpretation even if Ryholt's physical
reconstruction of the papyrus is accepted. One of his most debatable
and far-reaching ideas is to assign the earlier group of Theban kings to
Manetho's Dynasty 16. Africanus, the most accurate of the excerpters,
described Dynasty 16 as 'Shepherd (Hyksos) Kings', while Eusebius
records them as Theban. Ryholt's interpretation is followed here.
There are a few kings whose names occur on monuments but who
cannot be identified in the Turin Canon (perhaps having been listed on
a portion that is now missing). One such ruler is Sekerher, who bore a
full Egyptian titulary (three out of his five names are preserved), but
described himself as a heka khasut ('ruler of foreign countries'); his
inscription is preserved on a door jamb found reused in an early i8th-
Dynasty building at Tell el-Dab e a. Bietak identifies him with Salitis,
whose name is preserved in Josephus' version of Manetho's history as
the conqueror of Memphis.
There is, however, also a large group of about fifteen royal names
that occur only on scarabs. These personal names are sometimes
Egyptian, sometimes West Semitic, and are preceded by epithets such
as 'the good god', 'the son of Ra', and 'the ruler of foreign countries'.
The first two epithets were held by Egyptian kings for many hundreds
of years, referring in the most general terms to the status of king. The
term nesu ('king'), which is used in Egyptian sources such as the Turin
Canon, is never employed to describe these rulers. Stylistically, the
scarabs belong to a set of types that were made and used in both Egypt
and Palestine. Their archaeological contexts show that they belong to
180 J A N I N E BOURRIAU
the period following the i3th Dynasty, and their style links them with
scarabs bearing the names of the kings of the i4th and i5th Dynasties.
It is possible that we have here further examples of officials with a
purely local authority abrogating to themselves royal epithets on their
seals at a time and place where normally rigid protocols were no longer
enforceable.
Without confirmation from other sources, it seems unsafe to use
the distribution of scarabs as an indicator of the extent of the authority
of such 'kings' or to use changes in scarab design and shape to place
them into a chronological sequence. The finds from Tell el-Dab e a do
not, to date, help us to place any of them, except indirectly. It is likely,
given the model of Middle Bronze Age IIB Palestine and a literal inter-
pretation of the names adopted by Sekerher, that he was the overlord to
whom minor kings paid tribute. If so, this would explain the use of the
title 'ruler of foreign countries' both on the scarabs of otherwise
unknown men and on the inscriptions of the rulers of Avaris.
Bietak associates the final Hyksos phase at Tell el-Dab e a (strata
b/i-a/2; E/2-D/2; VI-V) with Manetho's i5th Dynasty, and a frag-
ment of the Turin Canon preserves '6 rulers of foreign countries
ruling for 108 years'. Only the name of the last one, Khamudi, can be
read. Sekerher, Apepi, and Yanassi, the son of Khyan, are recorded
from Tell el-Dab e a, and the first and last can be identified with
Manetho's Salitis and lannan. All evidence, written and archaeo-
logical, suggests that the authority of these rulers was very much
greater than that of their predecessors. The father-to-son succession of
two of them, and the exceptionally long reign of Apepi (at least forty
years), shows that a real dynasty, in the sense of, for example, the
Egyptian i2th Dynasty, was now ruling from Avaris.
At its largest extent, the city itself covered an area of almost 4 sq. km.,
which would have made it twice as large as it had been in the i}th
Dynasty and three times larger than Hazor, the largest Middle Bronze
Age IIA-C site in Palestine. In the latest Hyksos stratum, D/2, a citadel
was built on previously unsettled ground on the western edge of the city
commanding the river, and a watchtower some 200 m. to the south-
east guarding the land approach. Around these, an enormous enclosure
wall was built, with walls 6.2 m. wide, later enlarged to 8.5 m., and
buttressed at intervals. The fortifications were built over extensive
gardens, which had originally been part of a large palace complex.
The zenith of the Hyksos period is the reign of Aauserra Apepi
(c.i555 BC), despite the fact that two Theban kings led campaigns
against him. There are signs of a conscious revival of Egyptian scribal
THE SECOND I N T E R M E D I A T E PERIOD 181
B.C.
1410
1440
1470
1500
1530
1560
1590
1620
1650
1680
1710
1740
1770
1800
1830
1860
1890
1920
1950
1980
2000
2050
EGYPT
RELATIVE
CHRONO-
LOGY
CORE OF TOWN
(Middle Kingdom)
EZ. RUSHDI
NEW CENTRE
MB-population
FI
EASTERN
TOWN
A I -IV
NORTHEASTERN
TOWN
A V
(ADAM 1959)
HIATUS
DENUl
HYKSOS
PERIOD
OCCUPATION
XIII th DYN.
OCCUPATION
TEMPLE
RENEWED
XII th DYN.
TOWN
TEMPLE
XI th DYN.
OCCUPATION
HERAKLED-
POLITAN
FOUNDATION
DED
a/2
b/1
b/2
b/3
EPIDEMIC
c
HIATUS
d/1
d/2a
d/2b
HIATUS
e/1
e/2-3
D/2
D/3
E/1
E/2
E/3
F
*>
G/1-3
G/4
H
D/2
D/3
E/1
E/2
CITADEL
EZ. HELMI
H I -IV
IV / 1-3
C/1-3
V
VI
VII
GENERAL STRATIGRAPHY
UNOCCUPIED
EXPANSION OF SETTLEMENT
The stratigraphy and chronology of Tell el-Dab c a
traditions, indispensable for creating and controlling the complex
bureaucracy needed to govern in the Egyptian mode. On the palette of a
scribe called Atu, Auserra is described as 'a scribe of Ra, taught by
Thoth himself. . . with numerous [successful] deeds on the day when
he reads faithfully all the difficult [passages] of the writings, as flows the
Nile'. It was in the thirty-third year of his reign that the Rhind Mathe-
matical Papyrus was copied, a task that could have been undertaken
only by a scribe trained to the highest level of his craft and with access to
a specialized archive of mathematical texts such as could hardly have
existed outside the Temple of Ptah in Memphis. A stele from Memphis
of post-New Kingdom date records the genealogy of a line of priests
d / 2
182 J A N I N E BOURRIAU
going back to the nth Dynasty. It also preserves the names of reigning
kings, and records Apepi and Sharek for the period before Ahmose.
Fragments of a shrine were found at Tell el-Dab e a commemorating
Apepi and his sister Tany and dedicated by two Asiatics whose scribes
adapted their West Semitic names to the Egyptian hieroglyphic script.
A plate inscribed in fine hieroglyphs for Apepi's daughter Herit was
also found in the tomb of the 18th-Dynasty ruler Amenhotep I
(1525-1504 BC).
As a cultural phenomenon, the Hyksos have been described as
'peculiarly Egyptian'. The mixture of Egyptian and Syro-Palestinian
cultural traits—as expressed in objects from strata D/3 and D/2 (= reign
of Apepi) at Tell el-Dab c a—can be recognized over a wide area of the
Delta, from west to east: Tell Fauziya and Tell Geziret el-Faras, west of
the Tanitic Nile branch and including Farasha, Tell el-Yahudiya, Tell el-
Maskhuta, and Tell el-Habua to the east of it. These sites are all very
much smaller than Tell el-Dab e a, and the principal period of occupation
coincides in each case with the latest Hyksos strata, but two of them, Tell
el-Maskhuta and Tell el-Yahudiya, had come to an end before the period
represented by the last Hyksos stratum (D/2) at Tell el-Dab e a. Tell el-
Maskhuta and its satellite sites are located in the Wadi Tumilat, which
led to one of the main routes across northern Sinai to Palestine. It was
a small settlement, perhaps only seasonally occupied. The wealth of
Avaris derived from trade not only with Palestine and the Levant, but
also, in its latest phase, especially with Cyprus. The Kamose stelae list
the commodities imported by the Hyksos ('chariots and horses, ships,
timber, gold, lapis lazuli, silver, turquoise, bronze, axes without number,
oil, incense, fat and honey'), but there is very little surviving evidence
regarding the goods that the Hyksos kings were providing in exchange.
The ruler of Avaris claimed to be King of Upper and Lower Egypt,
although from the stelae of Kamose we know that Hermopolis marked
his theoretical southern boundary and Cusae, a little further south, the
specific border point. This region includes both Memphis and Itjtawy,
the capital of the i2th- and i3th-Dynasty kings. How was the authority
of the king of Avaris exercised in this region and can we recognize the
distinctive culture of the eastern Delta there?
Memphis: The Mansion of Ptah
Josephus claims to be quoting directly from Manetho in his descrip-
tion of the conquest and occupation of Egypt by the Hyksos:
THE SECOND I N T E R M E D I A T E PERIOD 183
By main force they easily seized it without striking a blow; and having overpowered
the rulers of the land, they then burned our cities ruthlessly, razed to the ground the
temples of the gods . . . Finally, they appointed as king one of their number whose
name was Salitis. He had his seat at Memphis, levying tribute from Upper and
Lower Egypt, and always leaving garrisons behind in the most advantageous posi-
tions.
This picture of Hyksos rule is confirmed by the very fact that the
Theban ruler Karnose rejected his status as vassal. The strict control of
the border at Cusae, the imposition of taxes on all Nile traffic, and the
existence of garrisons of Asiatics led by Egyptian commanders are all
mentioned in the Kamose texts. The Hyksos kings seem to be follow-
ing the model set up by the 12th-Dynasty kings in their rule over Nubia,
for which the bureaucratic and military institutions were probably still
in place. The key role of Memphis is also clear from Kamose's account.
Avaris was the Hyksos king's home city, the centre of his power, but
Egypt, even the northern part of it, could not be ruled from the eastern
Delta. Ruling Egypt meant controlling the Nile and every ruler of Egypt
has done this from the apex of the Delta, the region of Memphis and
modern Cairo.
Indisputable evidence of destruction and looting by the Hyksos is
rare. Four colossal sphinxes of the i2th-Dynasty ruler Amenemhat III
and two statues of the i3th-Dynasty ruler Smenkhera were found at
Tanis, inscribed with the names of Aqenenra Apepi (another name of
Aauserra Apepi). Their original dedication inscriptions to Ptah indi-
cate that they were first set up at Memphis. It is usually assumed that
they were looted by Apepi, taken to Avaris, and then removed in
Ramessid times to Tanis, but all that we can be sure of is that Apepi
claimed them by writing his name on them, and they may never have
left Memphis at all until Ramessid times. Nevertheless, at least one
royal monument of a i}th-Dynasty king was violated: the pyramidion
from the top of King Merneferra Ay's pyramid, which was probably
built at Saqqara, was found at Faqus, close to Tell el-Dab e a.
There is, to date, nothing to show that the Hyksos kings commis-
sioned funerary monuments in the Memphite tradition in the Western
Desert overlooking the city. However, we need to remember the
wholesale demolition at Tell el-Dab c a by the victorious Ahmose and the
greed of later kings for building stone before we accept too readily an
argument ex silentio. For example, two blocks, one of limestone and
one of granite, carrying the names of Khyan (0.1600 BC) and Aauserra
Apepi, have been found in the temple of Hathor at Gebelein. Since
there is no unequivocal evidence that the Hyksos ever controlled this
184 JANINE BOURRIAU
part of Egypt, let alone built monuments so far south, the blocks are
more likely to originate from Memphis and to have been brought to
Gebelein during the New Kingdom.
During the 19805, as part of a survey of the vast ruin field of
Memphis by the Egypt Exploration Society, a small area of the town
was excavated, revealing strata of the Second Intermediate Period. The
culture of this community, revealed by pottery, domestic architecture,
mud sealings with scarab impressions, metalwork, and beads, is
entirely Egyptian (especially when compared with that of Tell el-Dab c a)
and shows an unbroken cultural development from the i3th Dynasty.
Similarities in Egyptian ceramics allow strata at Memphis to be related
to those at Tell el-Dab e a, and this reveals a major break at both sites
after the last Hyksos stratum, Tell el-Dab c a D/2. There follows at Mem-
phis a sequence of sandy deposits in which no permanent structures
were built and in which the ceramics contain increasing quantities of
Upper Egyptian types dating to the very beginning of the i8th Dynasty.
The subsequent phase shows buildings aligned quite differently and
ceramics of pronounced early i8th-Dynasty style. These sandy deposits
are thought to coincide with the period of the Hyksos-Theban wars.
What is missing at Memphis is the presence of Middle Bronze Age
traits such as those that are visible at Tell el-Dab e a from the late i2th
Dynasty onwards. Imports and Egyptian copies of Palestinian pottery
are present at both sites, but at Memphis they represent less than 2 per
cent and at Tell el-Dab e a, 20-40 per cent, of the repertoire. There is no
cultural break at Memphis from the earliest strata excavated, which are
mid-i3th Dynasty, until the end of the Second Intermediate Period.
Can this pattern be observed at any of the other major centres of the
region?
At Saqqara, the necropolis closest to Memphis, the focus for activity
in the late Middle Kingdom was the mortuary temple of King Teti
(2345-2323 BC).There are private tombs and evidence of the continu-
ous celebration of the cult of the king until the first half of the 1301
Dynasty. As far as the late i3th Dynasty and Second Intermediate
Period are concerned, there is so far only one isolated intact burial
comprising a man placed in a rectangular coffin. The man's name,
Abdu, suggests that he was an Asiatic, and he was furnished with a
dagger inscribed with the name of Nahman, a follower of King Apepi.
Since the dagger is the only part of the find that has so far been pub-
lished, it is not known whether the burial compares with those of simi-
lar date from Tell el-Dab e a, but the rectangular coffin suggests that it
does not. Nor do we know if the dagger is contemporary with the burial
THE SECOND I N T E R M E D I A T E P E R I O D 185
or an heirloom. Apart from this ambiguous find, there is clear evi-
dence in the same area of an extensive cemetery of rich surface graves
belonging to the reigns of the early i8th-Dynasty rulers Ahmose and
Amenhotep I.
At Dahshur, site of the mortuary complexes of two great kings of the
12th Dynasty, Senusret III and Amenemhat III, ritual activity must
have continued at least into the early i3th Dynasty, because King
Awibra Hor was buried there at that time. At some later date large
mud-brick grain silos were built within the mortuary complex of
Amenemhat III. When the silos had fallen into disrepair, they were
used as convenient rubbish pits for the pottery discarded from a small
nearby settlement. Similar pottery occurs at Memphis in strata below
the sand deposits and at Tell el-Dab c a in strata G/4 onwards. Its
character is emphatically Middle Kingdom and Egyptian. It appears
that buildings were erected on the sacred space at Dahshur some time
after the early i3th Dynasty; these structures were associated with a
settlement that continued to be occupied, although it is not yet clear
how long this occupation lasted, except in relative terms. Thereafter
there is no evidence of activity until Ramessid times. The 'silo' pottery
at Dahshur is also present at Lahun, in the settlement which grew up
close to the mortuary complex of Senusret II. Thereafter, at Lahun,
there is a gap until pottery of the mid-18th Dynasty appears.
At Lisht, the necropolis closest to Itjtawy (the royal residence of the
i2th- and i3th-Dynasty kings), the picture is more complex. A large
private cemetery grew up around the pyramid of Amenemhat I, which
eventually intruded into the royal funerary complex itself. Among
these latest graves were a few fairly rich burials containing types of
Tell el-Yahudiya' pottery vessels that occur both at Tell el-Yahudiya
itself and at Tell el-Dab c a in graves of strata D/3 and D/2 (that is, the
strata dating to the end of the Hyksos period). These latest burials at
Lisht are wholly Egyptian in character. A settlement of workers con-
nected with the necropolis grew up during the i3th Dynasty in the
same area, and some burial shafts were dug within house complexes
both during and after their occupation. This un-Egyptian style of burial
is paralleled at Tell el-Dab e a, but there is no further evidence to suggest
that the inhabitants were not Egyptians. In the surface debris from the
excavation of houses and graves were found two scarabs with the name
of the 16th-Dynasty ruler Swadjenra Nebererau I (0.1615-1595 BC). His
dates, tentative though they are, fall within the range of those assigned
by Bietak to 0/3. There is no evidence of the i8th Dynasty at Lisht until
the reign of Thutmose III.
l86 J A N I N E BO URRIAU
Even this evidence both of the use of the necropolis at Lisht and of
continuity of Middle Kingdom culture there until far into the Second
Intermediate Period does not answer the question of when the king
and his court moved from Itjtawy to Thebes. The last i3th-Dynasty
king known to have had monuments in the area is Merneferra Ay
(^.1695-1685 BC). There is also the testimonial of an official called
Horemkhauef, a chief inspector of priests who was sent to collect the
temple statues of Horus of Nekhen (the local deity of Elkab) and of the
goddess Isis. His funerary stele, found in the courtyard of his tomb at
Elkab, describes a visit to Itjtawy in the course of this mission:
Horus, avenger of his father, gave me a commission to the Residence, to fetch
(thence) Horus of Nekhen together with his mother, Isis ... He appointed me as
commander of a ship and crew because he knew me to be a competent official of his
temple, vigilant concerning his assignments. Then I fared downstream with good
dispatch and I drew forth Horus of Nekhen in (my) hands together with his mother,
this goddess, from the good office of Itjtawy in the presence of the king himself.
The divine images collected by Horemkhauef were presumably newly
made or restored statuettes that had perhaps been used in a festival
connected with the kingship. Significantly, therefore, the Residence
appears at this time to have been the only place where craftsmen,
scribes, and lector priests were able to make such images. This
explains Horemkhauef s need to undertake a long journey and his
pride in his success. Unfortunately for us, the king who sent him is
never named. The making of such statues was one of the most signifi-
cant acts of the Egyptian ruler, enabling him to validate his own divine
status. References to the kings' creation of such images occur in all
surviving royal annals going back to the beginning of the Old King-
dom. This tradition of sacred craftsmanship, of which the king was
guardian, was evidently broken when the Residence was abandoned
and ties with Memphis were cut.
One result of the loss of this artistic tradition was a break in what has
been described as the 'hieroglyphic tradition'. The writing of the
formulas used in funerary inscriptions changed because they were
being produced under the influence of scribes trained in the cursive
hieratic script (used in administrative documents), whereas previously
the inscriptions had been created by the scribes who were specifically
trained in the carving of hieroglyphic inscriptions on stone monu-
ments. This change in the writing of the funerary formula can be used
as a means of dating inscriptions to the period before or after the end
of the Middle Kingdom. The writing on Horemkhauef s stele is of the
THE SECOND I N T E R M E D I A T E P E R I O D 187
post-Middle Kingdom type, which perhaps suggests that the political
fragmentation may actually have taken place during his lifetime. A
chronology has been deduced from genealogies of officials from Elkab
recorded in inscriptions, and on this basis it is suggested that
Horemkhauef s tomb was prepared between 1650 and 1630. If his visit
to the Residence took place at the beginning of a twenty-year tenure of
high office, it may date to between 1670 and 1650, at least fifteen years
after the end of the reign of Merneferra Ay in 1685.
Three small cemeteries at the mouth of the Faiyum Oasis (Maiyana,
Abusir el-Melek, and Gurob) date to the period of the wars between the
Hyksos and the Thebans, which is otherwise represented only at
Memphis. These Faiyum burials are Egyptian in character, with the
bodies laid extended in rectangular coffins. At Gurob, two burials con-
tain Kerma-ware pottery, indicating that they may belong to Kerma
Nubians serving in the Theban army (see below). One intact burial at
Abusir contained a scarab of the Hyksos ruler Khyan, which provides a
terminus post quern for the burial.
The pottery at Maiyana (a small cemetery of men, women, and
children, situated close to Sedment el-Gebel) includes cylindrical
combed Tell el-Yahudiya juglets, like those in stratum D/2 at Tell el-
Dab c a, as well as imported Cypriote base-ring I juglets, like those in the
earliest i8th-Dynasty strata both at Tell el-Dab c a and at Memphis. There
are no weapons, apart from a throwstick, but the use of sheepskins and
the decoration of the dead with feathers and flowers is not typically
Egyptian. This small cemetery seems to record the short-lived exist-
ence of a foreign community, but one that was distinct from that
flourishing at Avaris.
A small group of graves in the large New Kingdom cemeteries at el-
Haraga and el-Riqqa provide parallels to the Maiyana-Gurob-Abusir
el-Melek-Memphis pottery corpus and confirm that there is a short-
lived but distinct archaeological phase marking the transition between
the final phase of the Second Intermediate Period and the beginning
of the i8th Dynasty in this region. Roughly 130 years before this period
of transition, the king moved his Residence from Itjtawy to Thebes.
Even before this defining event took place, the sacred spaces at the
mortuary complexes of the 12th-Dynasty kings began to be encroached
upon, as the cults of the royal ancestors ceased to be celebrated. At
Lisht, however, the cemetery (and possibly its settlement) continued in
use until the end of the Second Intermediate Period. If the life of the
necropolis paralleled that of the Residence, then it too continued in
some form.
l88 JANINE BOURRIAU
Cusae: The Boundary between the Egyptian and the Asiatic
Nile
The Theban ruler Kamose was advised by his councillors: 'The middle
country is with us as far as Cusae', and the texts from Kamose's reign
remain our best written source for the history of Middle Egypt in the
Second Intermediate Period. An inscription of Queen Hatshepsut
(1473-1458 BC) in the Speos Artemidos, 100 km. north of Cusae (el-
Qusiya), records intensive restoration and reconsecration of temples
in the area: 'I have raised up what was dismembered from the first
time when the Asiatics were in Avaris of the North Land (with) roving
hordes in the midst of them overthrowing what had been made... The
temple of the Lady of Cusae . . . was fallen into dissolution, the earth
had swallowed up its noble sanctuary, and children danced upon its
roof This piece of royal propaganda was designed to show Hatshep-
sut in the traditional kingly role of restorer of order after chaos. Her
scribe was writing more than eighty years after the Hyksos-Theban
wars and it is as likely that the 'roving hordes' were the armies of
Thebes as it is those of Avaris. It is interesting that, so long after the
event, the rulers of Egypt were still boasting of the expulsion of the
Hyksos.
Cusae lies about 40 km. south of Hermopolis (el-Ashmunein),
which was the centre of the administration of the area during the
Middle Kingdom. When Horemkhauef visited the Residence at Lisht,
possibly between 1670 and 1650 BC, the river was still open, but shortly
thereafter Cusae marked the boundary at which any traveller from the
south had to pay tax to the ruler of Avaris if he wished to proceed.
Judging from Kamose's account of his arrest of a messenger with a
letter from King Apepi to the king of Kush, the Hyksos appear to have
controlled the route from 'Sako' (probably modern el-Qes) via the
Western Desert oases to the Nubian site of Tumas, midway between
the first and second Nile cataracts. This route gave the king of Avaris
access to allies—the fierce kings of Kush—and to gold. At least three of
the cataract forts (Buhen, Mirgissa, and Uronarti) were still function-
ing, although there is some debate as to whether they were subject to
the rule of Egypt or of Kush; nevertheless the organization still existed
to control the oasis route (from the southern end) and to send expedi-
tions to the gold mines. Despite the boundary at Cusae, regular contact
and exchange of goods continued between Lower Egypt and Nubia, via
the oasis route. This is clear from finds of pottery and mud sealings
both at the cataract forts and at the Kushite capital, Kerma. Moreover,
THE SECOND I N T E R M E D I A T E P E R I O D 189
Map of the Nile Valley and
Palestine in the Second
Intermediate Period
at Buhen at least, that contact seems to have continued without a break
from the i3th Dynasty until the beginning of the Hyksos i5th Dynasty
(see below).
We can enlarge our picture of Middle Egypt by looking at a group of
cemeteries excavated about 50 km. south of Cusae, at Deir Rifa,
Mostagedda, and Qau. Cemetery S at Deir Rifa contains the burials of
a group of Nubians known as 'pan-grave' people (because of their
190 J A N I N E BOURRIAU
distinctive shallow oval graves), who were semi-nomadic cattle-
breeders living on the edge of the desert. Their cemeteries and settle-
ments appear in Egypt during the i3th Dynasty, and they have been
identified with the Medjay of the Kamose texts, who were sent to scout
the land in advance of Kamose's fleet. Their distinctive handmade
pottery is ubiquitous in Middle Kingdom settlements and is found as
far north as Memphis. At Deir Rifa, their graves contained Tell el-
Yahudiya ware of types comparable with those from level E/i at Tell el-
Dab c a, which are datable to the middle of the i5th Dynasty. The
associated Egyptian pottery belongs to the Middle Kingdom style of the
Memphis region and suggests that the cemetery goes back to the
beginning of the i}th Dynasty.
Mostagedda, almost opposite Deir Rifa on the right bank of the Nile,
also contained the burials of pan-grave people, and these can be placed
into a chronological sequence according to the degree to which they
follow Egyptian or Nubian burial customs (whereas the Deir Rifa
cemetery is too poorly published to allow this to be done). Two phases
before the beginning of the i8th Dynasty are present at Mostagedda
and both contain Egyptian pottery remarkably different from that at
Deir Rifa. These two phases, as well as earlier ones, have also been
found in the large Egyptian cemetery at Qau, 15 km. to the south of
Mostagedda and Deir Rifa. The pottery is characterized by elaborate
incised decoration, the use of sandy marl clays, high-shouldered
narrow-necked storage jars, and carinated jars. This ceramic corpus
very clearly belongs to an Upper Egyptian tradition and provides the
prototypes for vessels that appear at Memphis and Tell el-Dab e a in fully
developed form in the early 18th-Dynasty strata.
The cemeteries of Deir Rifa and Mostagedda, on opposite sides of
the river, belonged to the same Nubian cultural group, but the differ-
ences in funerary equipment show that Deir Rifa was in contact with
the Memphis region, while Mostagedda was linked with Upper Egypt.
The Nubian artefacts in both are similar enough to suggest that the
difference between them is not one of time, but of wealth, status
(Mostagedda being generally richer), and, above all, regional associa-
tions. Their location suggests that the region of Cusae did indeed, as
the texts state, mark the border between Upper and Lower Egypt, and
that the boundary existed at least by the beginning of the i3th Dynasty.
It is possible to speculate that we have here the burial grounds of two
groups of Medjay mercenaries patrolling the border region: perhaps
one group based at Deir Rifa guarded the west bank for the Hyksos
while the other looked after the east bank for the Theban kings.
THE SECOND I N T E R M E D I A T E P E R I O D 191
Thebes, the Southern City: The Emergence of the i6th and iyth
Dynasties
On the basis of Ryholt's reconstruction of the Turin Canon, we can
now identify 15 names of kings (Dynasty 16 of Manetho) as the pre-
decessors of the kings of the iyth Dynasty Five of them occur in con-
temporary sources and these indicate that the centre of their power
was in Upper Egypt. We cannot be certain that they all ruled from
Thebes, and some may have been local rulers in important towns such
as Abydos, Elkab, and Edfu. King Wepwawetemsaf, not listed in the
Turin Canon, who left his modest stele at Abydos, may have been one
of these local kings; the stele shows him offering to Wepwawet, the
local deity after whom he was named. The style of its writing, design,
and royal regalia place it in a line of development between the i3th-and
lyth-Dynasty royal stelae.
King lykhernefert Neferhotep, who definitely ruled from Thebes,
left behind a much more impressive stele, on which he describes
himself as a victorious king, beloved of his army, one who nourishes
his town, who defeats rebels, who reconciles rebellious foreign lands.
Neferhotep is shown protected by the gods Amun and Montu and by
a goddess personifying the city of Thebes itself. She appears armed
with a scimitar, bow, and arrows. The language of the formal eulogy is
familiar from earlier hymns composed for kings but also for
nomarchs, great warlords who, in the First Intermediate Period, ruled
like local kings. The stele was set up, like those of Kamose, to celebrate
a precise event, which may have been the raising of a blockade of
Thebes. We do not know if Neferhotep fought the Hyksos, their
Egyptian vassals, or rival local rulers, but the Canadian Egyptologist
Donald Redford has noted a destruction layer after the ijth-Dynasty
level in part of the town underlying East Karnak. Neferhotep's name is
known also from contemporary monuments at Elkab and Gebelein. In
such uncertain times, the king's role as army commander becomes
more and more prominent and so enshrined in the royal litanies. The
ideology as well as some of the phraseology survives into the i8th
Dynasty.
Kings may fail but the officials who served them had their own
monuments, and from the genealogies recorded there a relative chron-
ology has been built up. Son often followed father into royal service,
and kings took wives from the great official families, so that a network
of interdependence gradually bound the king to the home towns of his
officials, at Elkab and Edfu as well as at Thebes. Genealogical evidence
192 J A N I N E BOURRIAU
suggests that only three generations separated the abandonment of
Itjtawy from the reign of King Nebererau I, sixth king of the iyth
Dynasty, and that the transition from the 13th- to the 16th-Dynasty
group of kings went officially unremarked by the officials who served
them.
We know a great deal more about the nine kings assigned (after
Ryholt) to the i7th Dynasty, but so far only two are known to have been
related to each other: the brothers Nubkheperra Intef VI and
Sekhemra Intef VII. It is possible but not certain that their father was
Sobekemsaf I. Their names do not occur in the Turin Canon, the
relevant section having been cut away in antiquity, but they occur on
other king-lists from Thebes; royal stelae have survived from reuse in
later building, and excavations have produced rich objects from their
burials. The bodies of Seqenenra Taa (c.i$6o BC) and his wife Ahhotep,
and possibly his mother Queen Tetisheri, were found in the Deir el-
Bahri cache of royal mummies, and most curious of all, we have a
tomb-robbers' description of the burial of King Sobekemsaf II and his
wife, still intact over 600 years later in the 2Oth Dynasty. Kings' names
also occur in private tombs, and objects. These Theban kings are
thought to have ruled at the same time as the Hyksos i5th Dynasty, but
there is no fixed point for dating the beginning of the lyth Dynasty,
only the end being marked by the death of Kamose at an unknown
point in or after his third regnal year. The fortunes of the kings seem to
have fluctuated: Nubkheperra Intef is mentioned on over twenty con-
temporary monuments, whereas Intef VII is known only from his
coffin, now in the Louvre.
The continuing military ethos of the time is illustrated by the popu-
larity of military titles such as 'commander of the crew of the ruler' and
'commander of the town regiment'. They show a defensive grouping
of military resources around the king and confirm the importance of
local militias based on towns. Instability remained characteristic of
Upper Egypt for the rest of the Second Intermediate Period.
Rahotep, first king of the iyth Dynasty, boasts of restorations in
temples at Abydos and Koptos, while an inscription of Sobekemsaf II
shows that he sent a quarrying expedition of 130 men to the Wadi
Hammamat. These quarries, however, were well within Theban terri-
tory, and the numbers of quarry-workers involved do not compare with
the thousands of men sent to the wadi in the i2th Dynasty. Neverthe-
less, confidence was growing and both the territory and the activities of
the king were expanding. Sobekemsaf s expedition has a distinctly ad
hoc air: only one man holds the appropriate title of'overseer of works',
THE SECOND I N T E R M E D I A T E P E R I O D 193
while the rest have honorific titles or offices connected with provision-
ing. The scribe does not observe the strict hierarchy of status in his
listing, and uses a mixture of hieroglyphic and hieratic signs. It
appears that traditional skills and protocols were having to be relearnt
after a decisive breakdown. At the Gebel Zeit galena mines, over-
looking the Red Sea, two modest stelae were found recording expedi-
tions in the reigns of Nubkheperra Intef VII and Swaserenra Bebiankh
of Dynasty 16, the latter previously hardly known beyond his listing in
the Turin Canon. Large numbers of pan-grave sherds were also found
there, suggesting another purpose for which the Theban kings may
have used Nubian mercenaries.
Thebes was cut off from contact with Lower Egypt and denied access
to the centres of scribal learning at Memphis. Such centres, with their
archives, were not destroyed and may even have flourished under the
Hyksos, but the Thebans would have been unable to consult them,
thus perhaps necessitating the creation of a new compilation of texts
needed for the all-important funerary rituals. One of the first collec-
tions of spells that we know as the Book of the Dead dates to the i6th
Dynasty and comes from a coffin of Queen Mentuhotep, wife of King
Djehuty. The funerary culture of Thebes also evolved in other ways, in
response to an impoverishment of resources. Large rectangular coffins
made of cedarwood were replaced with roughly shaped anthropoid
coffins of sycamore painted in a feather pattern, but in so crude and
idiosyncratic a style that no one is exactly like any other. This feature
betrays a lack of training in the erstwhile rigid conventions of funerary
art, which were perhaps also less in demand. However, a few coffins
demonstrate that in some Theban workshops the tradition of Middle
Kingdom coffin making survived well into the i8th Dynasty.
The location of five of the royal tombs of the iyth Dynasty, those of
Nubkheperra Intef VI, Sekhemra Intef, Sobekemsaf II, Seqenenra
Taa, and Kamose, is described in the Abbott Papyrus, which contains
the record of a judicial enquiry into tomb robbery by the mayor of
Thebes in the 2oth Dynasty. In 1923 Herbert Winlock set out to
relocate the tombs using the itinerary of the inspectors given in the
papyrus. He was also inspired by the fact that many objects from royal
burials of the same date had appeared for sale from illicit excavations
in the 18208 and 1859-60. The robbers of the 2oth Dynasty described
how they found the burial of Sobekemsaf II:
He was equipped with a sword and there was a ... set of amulets and ornaments of
gold at his throat; his crown and diadems of gold were on his head and the . . .
mummy of the king was overlaid with gold throughout. His coffins were wrought
194 J A N I N E BOURRIAU
with gold and silver within and without and inlaid with every splendid costly stone
... we stole the furniture which we found with them, consisting of vases of gold,
silver and bronze.
These kings and their officials spent their increasing wealth at the
end of the dynasty on the objects in their tombs rather than on the
tomb structures themselves. Decorated tombs are rare; instead, earlier
tombs were often taken over and reused. To understand where the
wealth was coming from we need to look to the south, to Elephantine,
to the forts guarding the second Nile cataract, and finally to Kerma,
capital city of the King of Kush, over 800 km. south of Thebes.
Elephantine and the Cataract Forts
Elephantine, an island opposite the modern town of Aswan, is an
interesting vantage point from which to study the Second Inter-
mediate Period. As a provincial town, it provides a counterbalance to
the Theban sources, and there is an unbroken series of private and
royal dedications dating from the late i2th to the i6th Dynasty. The
stratified town site and cemeteries of the same period are being exca-
vated by the German Institute of Archaeology.
The fortunes of Elephantine are inextricably linked to those of Nubia.
During most of the Middle Kingdom it did not mark the southern
boundary at all; that was fixed by Senusret III at Semna, 400 km. to the
south. However, during the nadir of the power of the Theban kings, it
is possible that Elephantine was ruled independently and even that
Nubians raided the city from time to time. Booty from a raid against
Elephantine or the forts is the favoured explanation for the fact that a
royal tomb at Kerma in the late Second Intermediate Period contained
statues of a nomarch of Asyut and his wife who lived in the reign of
Senusret I (1956-1911 BC).
The value of Lower Nubia lay in its quarries, principally of diorite,
granite, and amethyst; its access to gold and copper mines; and its
strategic location in terms of the control of the desert and river routes.
A 6th-Dynasty local official of Elephantine, Heqaib, was deified after
his death, and a series of votive stelae and statues has been found in his
shrine. The I3th-i6th Dynasties are particularly well represented and,
as at Memphis, the continuity is broken only with the advent of the
18th Dynasty. The genealogies recorded in the inscriptions show that
the same families served both the late i3th-Dynasty kings and those of
the 16th Dynasty. The status of the mayor of Elephantine evidently
changed from one of great local significance to a military one within
THE SECOND I N T E R M E D I A T E P E R I O D 195
the retinue of the King of Thebes. One such man was Neferhotep, who
was responsible to the king for the whole region from Thebes to
Elephantine. After his time (the i6th Dynasty, judging from the
orthography of his stele), dedications in the Heqaib shrine cease and it
may be no coincidence that this is the period when the Prince of Kush
was at his most powerful and even the cataract forts were falling under
his control.
The fortunes of one of the forts, Buhen, may be pieced together
from evidence not yet fully published. After the late i2th Dynasty,
soldiers were buried with their families in Cemetery K at Buhen; these
burials are characterized by pottery from the Memphite region, con-
firmation that the fort's supplies were still coming from the workshops
of the Residence. Cemetery K shows continuous occupation until well
into the Second Intermediate Period, and there are at least two groups
of intact, multiple burials that contain Tell el-Yahudiya ware juglets,
including one type that does not appear at Tell el-Dab c a until stratum
E/i (probably the early i5th Dynasty). One of the bodies has a large gold
nugget around its neck, suggesting that settlers remained at Buhen
primarily because of its proximity to the gold-mining region. By this
time the boundary between Upper and Lower Egypt was in place, so
that supplies from Lower Egypt could have reached Buhen only via the
oasis route, which we know was in use during the reign of Apepi. Who,
one wonders, was organizing this trade at the northern end? We may
speculate that officials were still working at Itjtawy under the Hyksos
king and we know that the Lisht cemetery was still in use. Avaris itself
was the centre for the manufacture and distribution of Tell el-
Yahudiya juglets, the contents of which have not yet been identified
but were clearly much prized.
The fortress settlers must have felt themselves increasingly isolated
and vulnerable, despite their links with Lower Egypt, and so had to
accommodate themselves to the local military power, which was
neither the Hyksos nor the Kings of Thebes, but the King of Kush. A
family covering five generations left inscriptions at Buhen and these
show that the last two generations served the King of Kush and even
carried out local campaigns on his behalf; this period is marked
archaeologically by the presence of pottery imported from Upper
Egypt, from the Theban area, instead of pottery from Lower Egypt. The
river was open between Thebes and the forts, but only, as the Kamose
texts imply, if taxes were paid to the master of the southern Nile, the
King of Kush. Buhen was eventually sacked (there are traces of a great
fire) but more probably by the army of Kamose than by the Nubians.
196 J A N I N E BOURRIAU
Other forts, Mirgissa and Askut, show a similar history of continued
occupation by Egyptians, but alongside Nubians until the end of the
Second Intermediate Period. Eventually control of the cataract region
by the King of Kerma became intolerable to the Theban rulers, making
it essential that they should retake the forts before they could proceed
in safety against the Hyksos. In the third year of Kamose's reign, we
have the earliest evidence that the region was again under Theban
control. The building of a wall is recorded at Buhen, probably a
renewal of the fortifications after the successful campaign mentioned
in the letter from the Hyksos ruler Apepi to the King of Kush.
The Kingdom of Kush
The King of Kush is the name given in Egyptian sources to the king
whose capital lay at Kerma. Archaeologists use Kerma as an adjective
to describe the culture of the Kushites and to distinguish it from other
contemporary Nubian cultures, such as C group and pan grave. Kerma
is situated south of the third cataract, at the termination of the western
oasis routes, and is being excavated by Charles Bonnet of the Uni-
versity of Geneva.
The Kerma people kept no written records, but we know that their
culture, found throughout Nubia, goes back to the early Old Kingdom.
The king was at his most powerful during the Classic Kerma phase,
which corresponds roughly to the Second Intermediate Period.
Kamose may have succeeded in retaking Buhen, but only much later in
the 18th Dynasty, after at least three more long campaigns, was Kerma
itself conquered. The destruction that followed was so thorough that it
is difficult now to reconstruct the city as it stood during the reigns of
the last independent rulers. We do know that the great tumulus tombs
in which the kings were buried contained slaughtered servants and
great stocks of provisions, many imported from Upper Egypt, which
may have been the taxes paid by those wishing to pass south beyond
Elephantine. Until at least the middle of the ijth Dynasty the king was
trading with both Upper and Lower Egypt, a trade probably adminis-
tered through the cataract forts.
The Kerma Nubians were cattle-breeders and warriors, particularly
famous as bowmen. The bows and arrows in their graves and the
massive fortifications at Buhen designed to defend against archers,
confirm this reputation. At the centre of the city was an enormous
round hut set within a stockade which functioned in royal ceremonies.
There were also large sacred sites and administrative buildings. An
THE SECOND I N T E R M E D I A T E P E R I O D 197
extensive building and rebuilding programme during the Classic
Kerma phase testifies to the immense resources in materials and
manpower at the king's disposal.
The presence of Kerma Nubians in the armies of Kamose and
Ahmose is beyond dispute, but it is unclear whether they were there
voluntarily or forcibly recruited during Kamose's campaign. It seems
likely that the Kerma Nubians were a federation of tribes, not all of
whom necessarily accepted the authority of the King of Kerma, and,
with it, the policy of enmity towards the Theban kings. In any case,
whatever the king's policy, trade flourished between Kerma and Thebes
during the late Second Intermediate Period. People travelled as well as
goods: Egyptian craftsmen to Kerma, perhaps, and certainly Kerma
Nubians to Egypt. The burials of a handful of individuals have been
found scattered between Thebes and Abydos. One rich burial, found
intact at Thebes, is of the time of Kamose and belonged to a woman and
her young child. It is entirely Egyptian in style and the woman wears a
royal gift, 'the gold of honour', a necklace of many tiny gold ring beads.
Beside her coffin was a carrying pole from which hung nets containing
six pottery beakers, made in a style so diagnostic of the Kerma culture
that it is called 'Kerma ware'. Gold drew the Thebans and the Kerma
Nubians together, first as allies but finally and inevitably as enemies.
Avaris and Thebes at War
The scene was set for war. The Theban kings had mastered their own
region; Kamose had retaken Buhen, so the route to the gold mines lay
open to him; the Kerma Nubians had been driven south; and a battle
fleet had been made ready. As Kamose phrases it, 'I will close with him
that I may slit open his belly; for my desire is to rescue Egypt and to
drive out the Asiatics.'
Most of our written sources for the war come from the Theban side
and predictably they show the Thebans as both the stronger and the
more belligerent of the protagonists. The war must have lasted for at
least thirty years, since we know that Seqenenra Taa, the father of
Ahmose, fought the Hyksos but that Avaris was not taken until
between regnal years 18 and 22 of Ahmose. After the sack of the city,
whether immediately or not, Ahmose took his army to Palestine in a
campaign culminating in a three-year siege of Sharuhen, near Gaza. It
is usually assumed that Sharuhen was the Hyksos king's last strong-
hold, but the sources are silent on this point. The war was not con-
tinuously fought: campaigns were short and armies, by modern
198 J A N I N E BOURRIAU
standards, small. Ahmose, son of Ibana, an important military official
who was buried in a rock tomb at Elkab, describes slaying two men and
capturing another in battles around Avaris that were important enough
for him to receive rewards of gold from the king.
The first known engagement occurred during the reign of
Seqenenra Taa (now thought to be the same king as Senakhtenra Taa).
A papyrus written in the reign of the 19th-Dynasty ruler Merenptah
(1213-1203 BC), about 350 years later, preserves fragments of a story of
a quarrel between Seqenenra and Apepi. It begins with a complaint by
Apepi that the roaring of the hippopotami at Thebes was keeping him
from sleep. Seqenenra is described as the 'Prince of the Southern City'
while Apepi is King (nesu), to whom the whole of Egypt pays tribute.
The story breaks off as Seqenenra summons his councillors, but the
narrative structure, so close to that of the Kamose texts, looks as if this
is the prologue to a battle.
We have further evidence of military activity in Seqenenra's reign
from Deir el-Ballas, the site of a settlement constructed on virgin
ground at the desert edge, 40 km. north of Thebes. The interpretation
of the remains, first excavated by George Reisner in 1900 and more
recently examined by Peter Lacovara in 1980-6, is not straightforward,
but the date of the site's first phase, the reigns of Seqenenra Taa,
Kamose, and Ahmose, is not in doubt. During the reign of Seqenenra
himself, a palace with an enormous enclosure wall was built. Like all
the surviving buildings at Balias, it was made of mud brick, with lime-
stone door frames and columns. It consisted of a series of courts and a
long entrance corridor around an elevated central area where, we pre-
sume, the private royal apartments stood. The walls were painted in a
rough style, with scenes depicting men and weapons, and decorated
with faience tiles. In an enclosure to the west were large animal pens.
Beyond the enclosure wall were widely scattered groups of large pri-
vate houses; an artificially laid-out group of smaller houses for work-
men; an open area for food preparation; and a textile workshop. At the
southernmost extremity, on a hill dominating the river and surround-
ing desert, was a platform supporting a building, now destroyed,
reached by a monumental stairway. It seems most likely that this was a
military observation post.
Among the pottery from Ballas was a large quantity of Kerma ware,
especially types used for cooking and food storage. There can be no
doubt that Kerma Nubians were living there alongside Egyptians in
considerable numbers. It seems hard to avoid the conclusion that the
purpose of this settlement, deliberately built in a remote place, was
THE SECOND I N T E R M E D I A T E PERIOD 199
military, perhaps intended for the mustering of an army containing a
large contingent of Kerma Nubians.
The examination of the mummy of Seqenenra shows that he died by
violence. His forehead bears a horizontal axe cut; his cheek bone is
shattered and the back of his neck carries the mark of a dagger thrust.
It has been argued that the shape of the forehead wound is consistent
only with the use of an axe of Middle Bronze Age type, similar to those
found at Tell el-Dab e a. Egyptian axes, such as those depicted on the
walls of the palace at Ballas, are of a different form. This is the most
telling evidence so far that a major battle against the Hyksos took place
in Seqenenra's reign—one in which the king himself was brutally
slaughtered. The angle of the dagger thrust suggests that the king was
already prone when it was inflicted.
Kamose succeeded Seqenenra Taa. It is often stated that he was the
king's son, elder brother of Ahmose, but we do not know who his
parents were and his coffin carried no uraeus, emblem of royalty. Only
the third year of Kamose's reign is attested, on the stelae from Karnak
and the inscription from Buhen. Both expeditions, to Buhen and to
Avaris, took place in or before the third regnal year, the former pre-
ceding the latter. Kamose was a warrior, 'Kamose the Brave' being one
of his most frequent epithets, but he probably died shortly after year 3.
However, his funerary cult, associated with that of Seqenenra Taa,
survived till Ramessid times and at least one of his Karnak stelae was
still standing over 200 years after his death.
We can use the texts of the two 'Kamose stelae' and the near-
contemporary copy found on a writing tablet in a Theban tomb to
reconstruct his expedition to Avaris. Setting aside the hyperbole, this
campaign was far from definitive, perhaps no more than a raid, given
that the final destruction of Avaris did not take place until over twenty
years later and that Kamose's opponent was Aauserra Apepi, the most
powerful and long-lived of the Hyksos kings.
Kamose first moved north from Thebes with his army and battle
fleet, sending Nubian scouts ahead to reconnoitre the position of
enemy garrisons. The sack of Nefrusi, north of Cusae, is graphically
described: 'as lions are with their prey, so were my army with their
servants, their cattle, their milk, fat and honey, in dividing up their
possessions with joyous hearts.' As he continued north, he intercepted
at Sako (el-Qes) a messenger sent from Apepi to the King of Kush, and
this led him to send soldiers to Bahariya Oasis to cut communications
and to 'prevent there being any enemies in my rear'. There follows a
gap in the account until Kamose reaches Avaris, where he deploys his
200 J A N I N E BOURRIAU
fleet on the waterways around the city to form a blockade while patrol-
ling the banks to prevent counter-attacks. He describes the palace
women peering out at the Egyptians from the citadel like 'young
lizards from within their holes'. Then follows the traditional boastful
speech to Apepi, 'Behold, I am drinking of the wine of your vine-
yards. ... I am hacking up your place of residence, cutting down your
trees', and a list of the plunder he was carrying away. Despite the
bombast, it is clear that Avaris was not attacked and Apepi refused to
engage him. The Kamose texts end with the king's happy return: 'every
face was bright, the land was in affluence, the river bank was excited
and Thebes was in festival.'
It is difficult from our vantage point to judge how much damage was
inflicted on the Hyksos by Kamose's campaign. All his achievements,
however, had to be repeated by his successor, and the admiral, Ahmose,
son of Ibana, makes no mention of Kamose, although his father and he
served successively in the battle fleets of Seqenenra Taa and Ahmose.
There was no immediate follow-up by the Thebans and it was at least
eleven years before an army under Ahmose began to fight its way
north again. The reason for the lull was that both Kamose and his
opponent Aauserra Apepi had died. They were succeeded by Ahmose
and Khamudi respectively. Ahmose was a young boy at his succession,
and the kingdom was held together by the queen mother, Ahhotep.
Unique epithets are given to her: 'one who cares for Egypt; she has
looked after her soldiers . . . she has brought back her fugitives, and
collected her deserters; she has pacified Upper Egypt, and expelled her
rebels.'
The final phase of the war was in the eleventh regnal year of an
unknown king, sometimes identified as Ahmose, sometimes as
Khamudi. The evidence consists of fragmentary notes on the verso of
the Rhind Mathematical Papyrus. The recto was copied in year 33 of
Aauserra Apepi, thus in a region where events were dated by the regnal
years of Hyksos kings; the specialist subject matter and high quality of
the papyrus suggest Memphis as the place of origin. On the verso are
some notes: 'Regnal year n, second month ofshemu—Heliopolis was
entered; first month ofakhet, day 23—this southern prince broke into
Tjaru. Day 25—it was heard tell that Tjaru had been entered.' Tjaru is
probably to be identified as the fortress site of Tell el-Habua, and—in
this author's view—the 'southern prince' is to be identified with
Ahmose, while year u belongs to Khamudi, whose name, without
regnal years, is given in the Turin Canon.
The strategy of Ahmose seems to have been to bypass Memphis to
THE SECOND I N T E R M E D I A T E P E R I O D 2OI
take Heliopolis and then, three months later, in mid-October (after the
water level of the inundation had begun to fall, and men in chariots
could move again in the valley), to attack Tell el-Habua, which had the
effect of cutting off the Hyksos from a retreat across northern Sinai to
Palestine. The assault on Avaris followed.
We have three contemporary sources for the campaign: the bio-
graphy of Ahmose, son of Ibana; the physical evidence from Tell el-
Dab'a; and fragments of narrative relief from Ahmose's temple at
Abydos. Ahmose, son of Ibana, naturally focuses on his own role, so
his perspective is a narrow one, but it is totally free of the grandilo-
quent posturing of the Kamose texts. Ahmose's Abydos reliefs (dis-
covered in 1993) give us fascinating glimpses of the protagonists: the
horses and chariots of the Egyptians; the royal battle fleet; soldiers
hacking at crops; a Hyksos captive, shown with shaved head, stubble
beard, and rope around his neck; a Hyksos warrior with his arm
upraised, and wearing a long-sleeved fringed garment; and the chaos
of falling and struggling bodies. The relief may include episodes from
the king's later campaigns in Syria and Palestine but the central
narrative involves a battle fleet and this can only refer to the siege of
Avaris.
Ahmose, son of Ibana, describes a series of engagements at Avaris,
and, since we do not know how long the campaign lasted from siege to
sack, his account may contain events spread over several years. The
straightforward narrative style does suggest strongly that events are
being reported in chronological order. Assuming this, we can recon-
struct the campaign as follows: Ahmose, son of Ibana, is a member of
the troop of soldiers on the ship 'Northern' (perhaps the king's ship),
leading the battle fleet. They arrive at Avaris and, after a battle, the king
begins the siege. While this continued, the army fought to pacify the
surrounding area. Ahmose, son of Ibana, was appointed to a new ship,
appropriately named 'Rising in Memphis', and fought on the water of
Avaris, killing an enemy. He fought two more engagements, one
'again in this place'—presumably Avaris—and another south of the
city. Only after these skirmishes does he laconically report: 'Avaris was
despoiled and I brought spoil from there: one man, three women . . .
his majesty gave them to me as slaves.'
Because Josephus considered the Hyksos to have been the founders
of Jerusalem, his version of Manetho includes a detailed account of
events after they were driven out of Egypt by Ahmose. Of the siege of
Avaris he says: 'They [the Hyksos] enclosed [Avaris] with a high strong
wall in order to safeguard all their possessions and spoils. The Egyptian
202 J A N I N E BOURRIAU
king attempted by siege to force them to surrender, blockading the
fortress with an army of 480,000 men. Finally, giving up the siege in
despair, he concluded a treaty by which they should all depart from
Egypt.'
Evidence from Avaris itself tends to confirm this picture of mass
exodus rather than slaughter after Ahmose's victory. A clear cultural
break is visible between the latest Hyksos stratum and that of the
earliest i8th Dynasty all over the site, largely because of the appearance
of a new ceramic repertoire. The same phenomenon appears also at
Memphis (see above). After the break there is no evidence of any con-
tinued occupation by people with a mixed Egyptian/Middle Bronze
Age culture and in some parts of the site occupation ceased altogether.
On the other hand, the cult of Seth, retaining the attributes of a Syrian
storm-god, continued and even expanded during the New Kingdom.
The latest Hyksos stratum, as we have seen, saw the greatest expansion
of the city and the building of immense defensive fortifications. These
may have been carried out early in the reign of Khamudi, but they were
not enough. Some explanation for defeat may be found in a clue that
suggests that the ideal of a warrior elite among the Hyksos did not
correspond to reality by the time of the Thebans' final assault. Battle
axes and daggers from stratum D/3 were of unalloyed copper, whereas
weapons from earlier strata were made of tin bronze, which produced
a weapon with a far superior cutting edge. It has been suggested that
an interruption in the supply of tin can be ruled out and the explana-
tion lies rather in a change in the function of weapons from practical
use to one of status and display. In contrast, weapons of the same
period from Upper Egypt were made of tin bronze and this would have
given the Thebans a clear advantage in hand-to-hand fighting.
It is generally thought that the Hyksos introduced the horse and
chariot into Egypt, since there is no firm evidence of either during the
Middle Kingdom yet they are present from the beginning of the i8th
Dynasty. There is no evidence so far from Tell el-Dab c a of chariots, and
the evidence for the presence of the bones of horses is equivocal. At
Tell el-Habua, however, a complete skeleton, found in a late Second
Intermediate Period context, has been positively identified as a horse.
The Kamose texts mention the enemy's horses and the chariot teams
of Avaris as part of Kamose's loot and this may account for their intro-
duction into Upper Egypt. Both horses, and horses hitched to chariots,
appear on the Ahmose reliefs at Abydos; moreover the chariots are not
simple prototypes but exactly comparable with those shown in the
mortuary temple of Tuthmose II.
THE SECOND I N T E R M E D I A T E P E R I O D 203
Despite the defeat of the Hyksos, the boast of Queen Hatshepsut, 'I
have banished the abomination of the gods, and the earth has removed
their footprints', has been disproved by the painstaking work of Bietak
and his team at Tell el-Dab e a.
The Reunification of the Two Lands under Ahmose
The sack of Avaris was only the first step in a series of campaigns
needed to secure the unity of Egypt. The sequence of events is not
universally agreed, but following the account of Ahmose, son of Ibana,
after the Avaris campaign came a campaign to southern Palestine
during which Sharuhen was taken. We do not know whether the
intention was to destroy the remnants of the Hyksos or to exploit the
vacuum they left to push on into Palestine and even as far as Lebanon.
There are later references to the importation of cedars of Lebanon and
the bullocks of 'Fenekhu'—a term thought to refer to Phoenicia.
Ahmose, son of Ibana, continues, 'Now when his majesty had slain the
nomads of Asia, he sailed south to Khent-hen-nefer (below the second
cataract) to destroy the Nubian bowmen/ We have confirmation that
King Ahmose restored (if that were necessary) Egyptian control of
Buhen, because a door jamb shows him and his mother offering to
Min and Horus (of Buhen) and names a commander of Buhen called
Turo.
After Ahmose returned from Nubia, he had to deal with two upris-
ings. The first was a minor mutiny in which a non-Egyptian (possibly a
Nubian ) called Aata brought a small force into Upper Egypt from the
north. This may have been no more than a raid for booty, since Aata
did not seek to engage the king's army. He was found and defeated and
he and his men were captured alive, two young warriors being given as
a reward to Ahmose, son of Ibana. Assuming that Aata was a Nubian,
and given that Kerma Nubians served in the army at Avaris and
Memphis and disposed of enough wealth to have substantial burials, it
is plausible that a group of such Nubians might have attempted to
exploit the king's absence in Nubia to go on a plundering raid into
Upper Egypt.
The second uprising was of a different character. It was led by an
Egyptian, Teti-an, who 'gathered the malcontents to himself; his
majesty slew him; his troop were wiped out'. The seriousness of this
rebellion is shown by the severity of its punishment. We can only
speculate that the malcontents were those who had, up till then, served
Ahmose's rival, the king of Avaris. The last five years of Ahmose's
204 J A N I N E BO URRIAU
reign were devoted to a massive building programme at the great cult
centres (Memphis, Karnak, Heliopolis, and, above all, Abydos), and at
the northern and southern boundaries of Egypt, Avaris, and Buhen.
The earliest 18th-Dynasty stratum at Tell el-Dab e a has produced
discoveries, extraordinary even in the context of this unique site. In the
immediate aftermath of the sack, the fortifications and palace of the
last Hyksos king were systematically destroyed. Ahmose replaced them
with similar fortifications and palatial buildings that were equally
short lived and can now be reconstructed only from their foundations
and from fragments of wall paintings found in dumps created as the
buildings were levelled. The wall paintings are, in style, technique, and
motif, Minoan, but there is as yet no consensus among Aegean
scholars as to whether they were painted by Minoan artists or by
Egyptians imitating them. Hundreds of fragments have been found,
but in very poor condition, and it will take years of conservation and
study before they can be fully assessed. Nevertheless, their presence in
contexts over 100 years earlier than the first representations of Cretans
in Theban tombs, and earlier than the surviving frescos at Knossos,
whose subject matter they share, has revolutionized ideas of the
relations between Egypt and Crete.
One of the buildings they came from was a royal palace and the only
comparable building of the time is the North Palace at Deir el-Ballas.
The few surviving wall paintings from there are utterly different,
painted in a simple style similar to that of contemporary tomb paint-
ings. The Tell el-Dab e a frescos seem to owe little to the traditions of
Egyptian wall decoration, which go back to the beginning of the Old
Kingdom. By analogy with the Knossos frescos, they seem to have been
executed to serve a ritual purpose, and are full of symbolic references
to the Cretan ruler cult. Bull-leapers and acrobats, associated with
motifs of the bull's head and maze pattern (labyrinth), belong totally to
the Aegean world. The varying scales of the frescos, their subject
matter and background colour, all indicate that the decorative scheme
was extremely complex and spread over not one but a series of build-
ings. Other frescos, less complex and more clearly imitations of the
Minoan style, have been found at Tell Kabri in Palestine. One of their
most puzzling features at Tell el-Dab e a is that they appear in a vacuum.
There is a small amount of Cretan Kamares ware pottery but it occurs
in early i3th-Dynasty strata and there is no continuity in buildings or
artefacts between it and the strata of the frescos. Strangest of all, there
are no Cretan artefacts associated with the frescos themselves or in the
strata from which they originally came.
THE SECOND I N T E R M E D I A T E P E R I O D 205
The discovery of the frescos has revived old ideas, dismissed until
now, that Ahmose was an ally of the Cretan kings and may have taken a
Cretan princess as a wife. Evidence cited has been a Minoan-style
griffin on an axe of Ahmose, and the fact that Ahhotep, the king's
mother, had a title 'mistress of the Hau-nebuf that was originally
thought to refer to the islands of Greece, although it has recently been
argued that this interpretation is implausible. Nevertheless, the
frescos prove that Minoans were present at Tell el-Dab e a, whether as
artists themselves or as supervisors guiding Egyptian artists.
The questions posed by the frescos inevitably lead to another
problem, the date of the eruption of the Thera volcano, since the best
preserved frescos found so far are those from the Cycladic island of
Thera sealed beneath layers of lava. The eruption is a key event for
relating the chronological sequences of the Aegean and eastern
Mediterranean to each other and to an absolute chronology. Much
effort has been expended in attempts to identify the event in Egyptian
sources so that it can be dated by regnal years. The Rhind Papyrus's
references to storms, and a stele of Ahmose describing a destructive
upheaval, have been produced in the argument, but the most telling
evidence so far comes from Tell el-Dab e a. Pumice, identified by
analysis as deriving from the Thera volcano, has been found in
settlement strata datable to the period from the reign of Amenhotep I
until the beginning of that of Tuthmose III. However, the pumice
occurs in a workshop where it was being used as raw material, and the
context provides only a terminus ante quern, since the pumice could
have been collected, from the seashore, for example, at an earlier date,
and, in any case, could have lain there for some considerable time. Not
all of the pumice derived from Thera: the source of at least one of the
samples has been identified as an eruption in Turkey that took place
over 100,000 years ago. It is remarkable that no pumice has been
found so far in earlier strata at Tell el-Dab c a and no ash (that is, 'fall-
out' from the eruption) has been found at all. Using a combination
of evidence, including data from records of ice cores and tree rings,
where exceptional atmospheric conditions can sometimes be linked to
historical events, it has been suggested that the Thera eruption occur-
red in 1628 BC. The evidence from Tell el-Dabca could be interpreted as
support for the traditional date, ^.1530 BC (within the reign of Ahmose),
but much more work needs to be done to clarify the interpretation of
the scientific data, and the question must be left open for the present.
Little of Ahmose's reign was left after his reconquest of Egypt. Many
building projects were left unfinished, but the benefits of unification
206 J A N I N E BOURRIAU
were clear to see. The fine objects from royal burials and lists of
donations to the gods of Thebes testify to growing wealth and artistic
skill. The fragments of relief from Abydos left to us after the depre-
dations of Ramessid masons show that a style that we can easily
recognize as i8th Dynasty had already evolved by the end of his reign.
9
The 18th Dynasty before the Amarna
Period (0.1550-1352 BC)
BETSY M. B R Y A N
Archaeological discoveries in the 19805 and 19905, combined with the
re-examination of older inscriptional evidence, suggest that the reuni-
fication of Egypt took place only in the last decade of the twenty-five-
year reign of Ahmose (1550-1525 BC), first king of the i8th Dynasty.
Thus the dynasty may be said to have begun officially around 1530 BC,
but it was already well under way during Ahmose's reign. Indeed, the
nature of the Egyptian state at the beginning of the dynasty was surely
mainly a continuation of forms and traditions that had never been
entirely disrupted by the internal squabbles of the Second Inter-
mediate Period. It must have been in part the commanding faith in
those traditions that enabled Ahmose's predecessors in the iyth
Dynasty to consolidate a power base among other powerful Upper
Egyptian families. As Ahmose and his successors later moved to
assure their family's dynastic line, they created or modified aspects of
the kingship that, together with external pressures from the north-east
and south, profoundly affected the rest of the i8th Dynasty.
Ahmose and the Beginning of the New Kingdom
The inscriptions in the tomb of Ahmose, son of Ibana, at Elkab des-
cribe the defeat of the Hyksos by his namesake King Ahmose, as well
as the latter's siege of the stronghold of Sharuhen in southern
Palestine, and his campaigns in Kush, the capital of which was the city
of Kerma near the third Nile cataract. The completion of this Nubian
208 BETSY M. BRYAN
campaign was left to Amenhotep I (1525-1504 BC), and a series of
monuments on the island of Sai commemorated the victories of both
rulers; it is possible that all of these were erected by Amenhotep I, but
the fact that Ahmose was active in the region is not disputed.
Early 18th-Dynasty levels at Avaris (Tell el-Dab e a) record the name of
Ahmose, and the several kings who succeeded him. During this time,
several monumental buildings decorated with Minoan frescos were in
use at the site (see Chapter 8). Certainly this fact suggests that there
was increased contact with the Aegean, even if only through itinerant
artists commissioned to undertake or oversee the work. Since weapons
found in the small coffin of Queen Ahhotep I (mother of Ahmose), in
her tomb in western Thebes, illustrate Aegean or east Mediterranean
motifs and craft techniques applied to Egyptian objects, the exotic
foreign elements prized in the Delta appear to have been valued in
Thebes as well, at least in an adapted form. Actual Aegean objects
contemporary with the early i8th Dynasty are far more difficult to
document in Egypt, although Egyptian small-trade items occur in fair
numbers in Crete, and to a lesser degree on the Greek mainland.
However, it remains unclear (if not doubtful) whether there was direct
diplomatic exchange between Egypt and Crete in the early i8th Dyn-
asty. Ahmose and his immediate successors may instead have con-
tinued to participate in an east Mediterranean exchange system, just
as the Hyksos had. Whatever the case, the creativity in forging an
Aegeanizing style, as seen on the objects of Ahmose's time, as well
as the Minoan-style paintings at Tell el-Dab e a, did not survive the
early part of the i8th Dynasty. Ultimately, as was frequently the case
in periods of strong kingship, traditional Egyptian iconography
dominated. The few elements that persisted (the 'flying-gallop' motif,
for example) were quickly adapted to more familiar iconographic con-
texts.
Ahmose's most immediate construction project appears to have
been within the capital of Avaris, which he had wrested from the
Hyksos. Manfred Bietak's excavations have identified an early i8th-
Dynasty palace platform abutting a Hyksos fortification wall. Seals
naming the rulers of the i8th Dynasty between Ahmose and Amen-
hotep II have been found in later strata, but Bietak considers that
Ahmose was the builder of the original palace complex decorated with
Minoan frescos. He may have had other building projects in the Delta
region, but Avaris was certainly planned to be a major centre—quite
likely commercial—for the new government to utilize. It is clear from
excavations during the 19805 and 19905 that Memphis was also
THE 18TH  DYNASTY BEFORE THE AMARNA PERIOD 209
redeveloped in the early i8th Dynasty: as the river moved eastwards,
land was reclaimed and used for new settlement. Ceramic sequences
and royal scarabs indicate that, already in the reign of Ahmose, Mem-
phis was being resettled following a hiatus that may correspond to the
wars between Thebes and Avaris, described in Chapter 8.
The temple monuments from the last years of Ahmose's reign con-
stitute the foundations of a traditional pharaonic building programme,
honouring gods whose temples had flourished in the Middle King-
dom—Ptah, Amun, Montu, and Osiris. Ahmose certainly venerated
the traditional deities of Egypt's cult centres. Ahmose's affiliations
with the moon-god lah (represented in the 'Ah' element of his name)
are best attested in the inscriptions on the jewellery of Ahhotep I and
Kamose (1555-1550 BC), which describe Ahmose as 'son of the moon-
god, lah'. This god's major cult centre is unknown, despite the ubiqui-
tous presence of the 'Ah' element in the royal family names. Perhaps,
at the very time that he effected the reunification, Ahmose began to
have his name written with the lunar crescent of lah pointing its ends
downward. All monuments showing this form of the name Ahmose
must, therefore, date after years 17 or 18 of his reign. Being the first
king in more than 100 years to be able to erect monuments for the
gods of both southern and northern Egypt, Ahmose opened limestone
quarries at Maasara with a view to building at Memphis, the old and
venerated northern centre, and also at Thebes, the home of Amun and
Montu. Although his constructions at Memphis have not been found,
some from Thebes, and elsewhere, are still extant.
Ahmose undoubtedly made significant contributions to the cult of
Amun at Karnak. If he had lived longer, he would perhaps have begun
the rebuilding in stone of far more buildings there, but his surviving
monuments nevertheless comprise a doorway and several stelae, as
well as perhaps a boat shrine, probably located near the entrance ways
to the temple. His desire to be recognized as Amun's pious dedicant
would, therefore, have been apparent not only to those whose priestly
offices or elite status gained them access to the god's home, but also to
the lesser inhabitants of Thebes who were able to visit the front court-
yards only at festival times.
Several limestone stelae recording major episodes connected with
Karnak temple are known from Ahmose's reign—probably all from
the last seven or so years of the reign. On two stelae discovered in the
foundations of the Third Pylon at Karnak, the king presents himself as
a propitiator and benefactor to the temple. On one of these, the so-
called Tempest Stele, the king claims to have rebuilt the tombs and
210 BETSY M. BRYAN
pyramids in the Theban region destroyed by a storm inflicted on
Upper Egypt by the power of Amun, whose statue appears to have been
left in extreme want. Ahmose describes the fact that the land was
covered with water and that he had brought costly goods to support the
restoration of the region. The other stele from the Third Pylon (known
as the Donation Stele) records the purchase by King Ahmose of the
'second priesthood of Amun' on behalf of his wife, the god's wife of
Amun, Ahmose-Nefertari. The cost of this office was paid to the temple
by the king, thus making him its benefactor again, and also securing
the tie between the god and the royal family.
A third stele of Ahmose, from the Eighth Pylon court at Karnak,
dates to year 18 of his reign; it extols the universal power of the royal
family, and details the cult equipment that Ahmose had fashioned and
dedicated to Karnak temple: gold and silver libation vessels, gold and
silver drinking cups for the god's statue, gold offering tables, necklaces
and fillets for the divine statues, musical instruments, and a new
wooden boat for the temple statue's processions. The objects donated
by the king to Karnak are the most essential cult furniture, and their
dedication may indicate that the temple was utterly without precious
metal objects at this point. It is impossible to say whether this would
have been due to the action of a great storm, as the king asserts in the
Tempest Stele, but temple cult objects, along with royal burial objects,
might also have been important financial resources for the Thebans
during the arduous years of the iyth Dynasty.
It is important to note the great dearth of precious metal objects
known from Upper Egypt in the Second Intermediate Period. Only
with the funerary equipment of Ahmose's mother, Ahhotep, and the
mummy of Kamose is there evidence again of extravagant gold royal
funerary objects, such as were known in the Middle Kingdom. Despite
the claims of tomb-robbers several hundred years after the Second
Intermediate Period, that they had robbed the gold-laden body of King
Sobekemsaf II of the iyth Dynasty, only comparatively modest coffins
and funerary objects have been recovered for the period preceding
Ahmose. Could the king's Karnak inscriptions have been an official
explanation for the impoverishment of the Theban region and, more
importantly, Ahmose's role in restoring the riches of the Karnak temple
and its god? This is not to suggest either that there was no tempest in
Ahmose's reign or that there was no purchase of the 'second priest-
hood' for Ahmose-Nefertari, but rather that these particular events
might have been recounted on the stelae simply in order to suit
historico-religious purposes.
THE 18TH DYNASTY BEFORE THE AMARNA PERIOD 211
Royal and Elite Tombs in the Late iyth and Early i8th Dynasties
Ahmose also built monuments at a number of other sites traditionally
favoured by kings, including Abydos, the major site of Osiris's cult.
These remains, currently being excavated and analysed by Stephen
Harvey during the 19905, are known to have included pyramid monu-
ments as well as temples. Abydos had long been a site that honoured
Osiris and the royal ancestors who had merged with Osiris at their
deaths. Pyramids were used to mark the Theban tombs of the iyth
Dynasty kings, and their brick remains may still have been visible in
the Theban region of Dra Abu el-Naga as recently as the nineteenth
century. Although the body of Ahmose was found in the royal mummy
cache at Deir el-Bahri (see below), the location of his tomb remains
unknown.
Ahmose's mother, Ahhotep, was almost certainly buried in the
Theban cemetery, as were kings and queens from earlier in the dynasty.
Excavation in the region during the 19905 has focused on what may be
one of these royal tombs, and, although no certain evidence yet exists,
Daniel Polz's work at Dra Abu el-Naga has shown the continuity of this
north Theban cemetery from the iyth to the early i8th Dynasty. He has
also demonstrated the existence of elite tomb clusters (each compris-
ing smaller graves scattered around a large tomb), in which single free-
standing cult structures may have been shared by several adjacent
graves. These clusters of graves are located on the desert floor beneath
the Dra Abu el-Naga hills, just south of the entrance to the Valley of the
Kings. The royal tombs, some of which were perhaps reused Middle
Kingdom chapels, are cut into the hills themselves, overlooking the
lesser graves.
So far, the archaeological evidence suggests that funerary wealth
was indeed curtailed in the iyth Dynasty, and that decorated tombs
were almost unknown in Thebes at this time. Still, the practice of
clustering the graves of the elite and the slightly less wealthy beneath
royal burial places, despite recalling the old practice of burying follow-
ers near the king, may also reflect some new organizational pattern
(although without further study it is impossible to conclude more). It is
interesting to point out in this regard, however, that in the Saqqara
region a non-royal cemetery of the time of Ahmose and Amenhotep I
consisted of surface graves, described as rich. Since the burial places of
the highest officials of these two reigns (viziers, high priests, treasur-
ers) are largely unknown, identifying the patterns of cemetery develop-
ment could ultimately help to locate missing tombs. Such work has
212 BETSY M. BRYAN
already been undertaken by Geoffrey Martin and Martin Raven in
central Saqqara south of Unas' causeway, and by Alain Zivie in North
Saqqara.
The bodies of some rulers and the coffins and funerary equipment
of others were moved from their original locations in antiquity (and
perhaps also in more recent times). Priests of the late New Kingdom
and early Third Intermediate Period reburied some royal mummies
in a tomb near Deir el-Bahri, where the mummies of Ahmose and
Seqenenra Taa (£.1560 BC) were found, both placed in non-royal coffins
of slightly later date. The large outer coffin of Ahmose's mother,
Ahhotep, made probably at the time of her death (perhaps as late as the
reign of Amenhotep I), was also found in the cache, although her inner
coffin (presuming both belonged to a single queen named Ahhotep)
was found earlier in what may have been her tomb. It contained
objects naming both Ahmose and Kamose. The area of Dra Abu el-
Naga continued for centuries to be associated with the royal family of
Ahmose and with Ahhotep and Ahmose-Nefertari particularly, and
later Ramessid tombs, chapels, and stelae in the region venerated their
memory.
The cemetery area itself changed dramatically, however, after the
early i8th Dynasty. Once royal tombs were no longer being con-
structed at Dra Abu el-Naga, it retained its status as the most elite
portion of the Theban necropolis only for another thirty years or so, up
to the reign of Hatshepsut (1473-1458 BC). With the establishment of
the Valley of the Kings as the royal burial ground, a few elite burials
began to be placed in Sheikh Abd el-Qurna, the line of hills to the
south of Deir el-Bahri. The clusters of valley shaft tombs, largely
without chapel structures, followed the movement of the cemetery
southward, and through the reign of Hatshepsut, and into that of
Thutmose III (1479-1425 BC), shafts were dug into Deir el-Bahri and
the Asasif to make family tombs of one or more chambers similar to
those at Dra Abu el-Naga. With the sudden increase of wealth held by
the elite in the later reign of Thutmose III, this practice seems to have
largely disappeared. Tomb-builders were kept busy excavating and
decorating rock-cut tombs at Sheikh Abd el-Qurna for the growing
royal administration.
Amenhotep I and the Nature of the i8th Dynasty
Like his father, Amenhotep I may not yet have been an adult at
his accession, particularly since another elder brother had been a
THE 18TH  DYNASTY BEFORE THE AMARNA PERIOD 213
designated heir only about five years earlier. There may have been a
brief co-regency with Ahmose to ensure the peaceful transition and
continuity of the recently established Dynasty, and his mother,
Ahmose-Nefertari, certainly figured prominently in his reign. In a
general way, Amenhotep Fs reign was a continuation of his father's;
buildings that may have been conceived by Ahmose were constructed,
and military expeditions in the south, completing earlier campaigns,
were carried out. Despite this apparent lack of personal imprimatur,
Amenhotep I was successful as a ruler in his own right. This is per-
haps best borne out by the fact that, soon after his death, both he and
his mother were deified and worshipped at Thebes, especially at Deir
el-Medina, the royal tomb-workers' village.
Deir el-Medina, situated in western Thebes to the south of the hill of
Sheikh Abd el-Qurna, was built early in the i8th Dynasty to house the
craftsmen who would build and decorate the royal tombs. Thutmose I
is the earliest attested royal name from contemporary monuments,
but Amenhotep I and his mother, Ahmose-Nefertari, were patron-
deities of the village throughout the New Kingdom and quite likely
from the founding of the settlement. Not only were there cult centres
for the two in the town, but most houses of the Ramessid era contained
in their front rooms a scene honouring the king and queen. The factors
linking Amenhotep I and his mother with the necropolis region, with
deified rulers, and with rejuvenation generally was visually transmitted
by representations of the pair with black or blue skin—both colours of
resurrection. The third month of peret was devoted to (and named
after) Amenhotep I, and within Deir el-Medina several rituals that
dramatized his death, burial, and return took place during that period.
However, Amenhotep I was a major god of the region and as such had
festivals throughout the year. It is probable that the king and his
mother became important deified rulers because of their connection to
the beginning of the New Kingdom and their activity in building on the
west bank of the river.
Amenhotep's military successes and consequent financial gains
from Nubia began to improve the overall economy of Egypt, and his
temple monuments made a significant impact as symbols of royal
power. Military action against Nubians south of the second cataract
took place around year 8, judging from inscriptions dating to years 8
and 9. Although it is not possible to ascertain with certainty, this may
be the campaign described in the tombs of Ahmose, son of Ibana, and
Ahmose Pennekhbet at Elkab. It is important to point out, however, that
both of these men's autobiographies derive from tombs constructed
214 BETSY M. BRYAN
long after the events retold in their narratives—as much as sixty or
seventy years after.
According to Ahmose, son of Ibana, he himself carried the king to
Kush, where 'his Majesty killed that Nubian bowman in the midst of
his army' and then pursued the people and cattle (presumably inland).
Ahmose was later rewarded with gold after bringing the king back to
the Nile Valley in two days, from an area designated as the Upper Well.
An extremely eroded stele left at Aniba and bearing a date in year 8
records that the Bowmen (iuntyu) and the Eastern Desert dwellers
(mentyu) delivered gold and large quantities of products to the king.
This stele may commemorate the fact that the successful expedition to
Kush was followed up by an official visitation to a secure part of Lower
Nubia by the royal family.
By the end of Amenhotep Fs reign, the main characteristics of the
18th Dynasty had already been established: its clear devotion to the cult
of Amun of Karnak, its successful military conquests in Nubia aimed
at extending Egypt southwards for material rewards, its closed nuclear
royal family (which avoided political or economic claims on the king-
ship), and a developing administrative organization presumably
drawn from powerful families and collateral relations, primarily asso-
ciated, at this point, with the regions of Elkab, Edfu, and Thebes. How-
ever, only a small number of the tombs of the high officials of the first
two reigns have so far been located.
The Monuments of Amenhotep I
It has been pointed out that Amenhotep I enjoyed at least a dozen years
of peaceful rule during which he was able to revive traditional activities
associated with monument building: the opening of the Sinai tur-
quoise mines (and consequent expansion of the Middle Kingdom
Hathor temple at the Serabit el-Khadim mines), the quarrying of
Egyptian alabaster at Bosra (in the name of Ahmose-Nefertari) and at
Hatnub, and the opening of work at the sandstone quarries of Gebel el-
Silsila, providing most of the stone necessary to rebuild Karnak temple.
Amenhotep I built at several of the sites where his father had been
active: at Abydos, for example, he erected a chapel that commemorated
Ahmose himself. Following successes in Upper Nubia, Amenhotep
dedicated monuments on Sai Island, including a statue similar to that
of his father and perhaps some type of building, judging from the sur-
vival of blocks inscribed in his name and that of his mother, Ahmose-
Nefertari.
THE 18TH DYNASTY BEFORE THE AMARNA PERIOD 215
Amenhotep I's interest in Delta sites and at Memphis remains
unverified, but Karnak figured prominently in his designs. A large
limestone gateway at Karnak, now reconstructed, was decorated with
jubilee festival decoration. According to its inscription, this was a
'great gate of 20 cubits' and a 'double facade of the temple'. It may once
have been the main south entrance that was later replaced by the
Seventh Pylon. To the east the king built a stone enclosure around the
Middle Kingdom court, with chapels on the interior of the wall. These
chapels contained scenes depicting the king, the god's wife, Ahmose-
Nefertari, and other temple personnel performing the ritual for Amun,
and dedications on behalf of the nth-Dynasty rulers. Thutmose III
dismantled all these chapels and rebuilt them in sandstone some forty
or fifty years later, but blocks and lintels with offering texts were found
in several locations within Karnak. A jubilee peripteral chapel for
Amenhotep I probably stood along the southern alleyway and was of a
type similar to that of Senusret I (1956-1911 BC) from the i2th Dynasty.
Indeed, the style of Amenhotep I's relief carving on the limestone
monuments at Karnak so consciously emulated that of Senusret I's
artisans that some blocks have been difficult to assign to the proper
ruler.
Clearly Karnak's function as a site for venerating the kingship was
central to Amenhotep I's construction plans. Whether that emulation
included celebrating a royal jubilee prior to thirty years of reign (the
ideal time a king waited before his first serf-festival), or whether he
erected the monuments in anticipation of ruling three full decades, is
unknown. Several of Amenhotep I's buildings, none the less, mention
the jubilee, such that it is certain the king intended to claim the
honour, just as did the great Middle Kingdom rulers.
Limestone jambs unearthed from the foundations of the Third
Pylon at Karnak provide a list of religious festivals and their dates of
celebration. Anthony Spalinger's study of these blocks has indicated
that in his festal calendar, as in most things at Karnak, Amenhotep I
was heavily influenced by i2th-Dynasty calendars. Amenhotep I also
had a bark shrine built for the god Amun and erected (most likely) in
the west front court of the temple.
Across the river from Karnak, Amenhotep I left funerary monu-
ments in the bay of Deir el-Bahri and to the north and east along the
edge of the cultivation. Built from mud brick, the Deir el-Bahri monu-
ment has been reconstructed with a pyramid, but only a few bricks
naming Amenhotep I and Ahmose-Nefertari were found there in situ.
No tomb has been certainly identified for either.
2l6 BETSY M. BRYAN
The building sites of Amenhotep I and his successors may relate to
the question of where and how astronomical observations for calen-
drical purposes were carried out (see Chapter i). Some discussions
have argued that Elephantine may have housed an observatory for
Sothic sightings, and recently a graffito from the Hierakonpolis region
has suggested that some sightings took place in desert locations.
Renewed interest in the cult sites between Aswan and Thebes during
the 18th Dynasty does indicate a similar concern with the natural phe-
nomena associated with these cults, such as the rise of the dog-star
Sirius (Sopdet/Sothis), the beginning of the rise of the Nile, and
attendant lunar cycles. The existence of a festival calendar recorded on
papyrus for the reign of Amenhotep I (Papyrus Ebers verso), raises the
possibility that the king wished to rework earlier calendars.
The Significance of the Royal Women of the Early
i8th Dynasty
A number of princesses, some of whom were also royal wives, are
known from the royal cache of mummies at Deir el-Bahri. They were
offspring of rulers from the end of the iyth or the beginning of the i8th
Dynasty, and their names are often known also from late New King-
dom private tomb chapels that venerated the royal family of the early
18th Dynasty. The titles held by these women, and the absence of hus-
bands other than kings, show the limitations that were placed on
females born of the king. The success of the dynastic line in the early
18th Dynasty was certainly attributable, in part, to a decision to limit
access to the royal family. In economic terms, this would have meant
that holdings gained in the wars were not divided with families whose
sons married a princess. The kings were therefore free to enrich mili-
tary followers as they wished, and thereby build new constituencies.
Followers such as Ahmose, son of Ibana, and Ahmose Pennekhbet are
two examples of these new members of the elite, but legal documents
later in the New Kingdom inform us of other men whose fortunes
derived from grants by Ahmose.
In political and religious terms, the closed royal family apparently
reached back into the Middle Kingdom (and the Old Kingdom before
it), when princesses were frequently married to kings or associated
throughout life with their reigning fathers. In order to assure the
exclusivity of the line, however, the family of Seqenenra and Ahhotep
apparently established the additional prohibition that royal daughters
were to marry no one other than a king. This was not the case in the
THE 18TH  DYNASTY BEFORE THE AMARNA PERIOD 217
Old and Middle kingdoms, at least not always, since we know examples
of high officials marrying kings' daughters, but, once the custom was
established at the end of the iyth Dynasty, it persisted through the i8th
Dynasty. Only with the reign of Rameses II do we again have definite
evidence of princesses marrying anyone other than kings.
Map of Egypt and Nubia between the reigns of Ahmose and
Amenhotep III (£.1550-1352 BC)
2l8 BETSY M. BRYAN
There were no enfeebling effects on the kinship line as a result of
this practice, because it did not mean that the kings themselves were
only able to marry princesses. Indeed, throughout the i8th Dynasty,
kings were most commonly born to their fathers by non-royal second-
ary queens, such as Tetisheri. If our understanding of the document-
ation is correct, then Tetisheri bore both the mother and father of King
Ahmose. His mother, Ahhotep, bore him by her brother (full or half),
most probably Seqenenra, possibly Kamose. Ahhotep had several
daughters as well, but Seqenenra also had daughters by at least two
and possibly three other women. Ahmose married his sister, Ahmose-
Nefertari, by whom he fathered at least two sons, Ahmose-ankh and
Amenhotep. He may, however, have fathered children by other
women as well. At least two princesses, Satkamose and (Ahmose-)
Merytamun, had the titles of king's daughter, king's sister, great royal
wife, and god's wife. The first was described on later stelae as a sister of
Amenhotep I, while the second is often identified as Ahmose-
Nefertari's daughter, who also married her brother, Amenhotep I,
although no document actually states this explicitly.
Despite the restrictions on marriage for kings' daughters, several
princesses who emerged as major queens (Ahhotep, Ahmose-Nefertari,
Hatshepsut) were extremely active in the reigns of their husbands and
heirs. Ahmose's mother, Queen Ahhotep, whose large outer coffin was
found in the Deir el-Bahri royal cache, was, according to her titles on
that coffin, a king's daughter, king's sister, great royal wife, and king's
mother. On Ahmose's year 18 stele from Karnak, he honoured Ahhotep
with titles that implied her de facto governance of the land. Although we
are ignorant of Ahmose's age at accession, he may have been only a boy
for some period of his reign. It is highly significant that the queen
mother was honoured later by her son for pacifying Upper Egypt and
expelling rebels. Ahhotep apparently carried on the fight without suc-
cessful challenge from within the region—although the implication is
that the family's hold on the kingship was tested during this period.
Claude Vandersleyen has suggested that the battles that Ahmose fought
against Aata and Teti-an were against Upper Egyptian enemies, the
latter perhaps representing a family line with whom the zyth Dynasty
Theban rulers Nubkheperra Intef VI and Kamose had also fought (and
this would accord well with Ahhotep's honouring Sobekemsaf, the
widow of Nubkheperra Intef VI, at Edfu). In any case, Ahhotep
apparently commanded the respect of local troops and grandees to pre-
serve a fledgling dynastic line, and she continued to function as king's
mother well into the reign of Amenhotep I.
THE 18TH  DYNASTY BEFORE THE AMARNA PERIOD 219
Perhaps not long after year 18 of Ahmose's reign, Ahhotep ceded
pride of place to Princess Ahmose-Nefertari, who may have been her
daughter. Ahmose's Donation Stele at Karnak (mentioned above) is
the first known monument on which Ahmose-Nefertari figures; she is
described on this stele as king's daughter, king's sister, king's great
wife, god's wife of Amun, and, like Ahhotep, mistress of Upper and
Lower Egypt. Ahmose and Ahmose-Nefertari are depicted with their
son, Prince Ahmose-ankh. Only a few years after this inscription was
made, in year 22, Ahmose-Nefertari claimed the title of king's mother,
although it is not known whether the designation referred to Ahmose-
ankh or Amenhotep. In any case, the queen survived her husband
Ahmose and even her son Amenhotep I, and still held the position of
god's wife of Amun in the reign of Thutmose I (1504-1492 BC).
Ahmose-Nefertari used the god's wife title more frequently even
than that of great royal wife. She also operated independently of both
her husband and her son in monument building and cult roles. When
she died, a stele of a non-royal contemporary recorded simply that 'the
god's wife... had flown to heaven'. The emphasis on her role as priest-
ess was perhaps due to the independent economic and religious power
ceded to the office of god's wife by Ahmose. The Donation Stele
records Ahmose's creation of a trust relating to the 'second priesthood
of Amun', whose benefices were then granted to the god's wife in
perpetuity, to be passed on, without interference, to whom she wished.
The institution of the divine adoratrice, an office separate from the
god's wife but also held by Ahmose-Nefertari, was also mentioned on
the Donation Stele. The economic holdings of the priestess institution
apparently continued to grow, such that some 100 years after Ahmose's
death, and following reorganization of the descent of the offices, the
produce of the 'house of the adoratrice' were a significant focus of
account papyri.
Ahmose-Nefertari functioned as great royal wife and particularly
god's wife of Amun throughout her son's reign. No certain wife is
known for Amenhotep I of his own generation, although it is often
presumed that the 'king's daughter, god's wife, great royal wife, united
to the white crown, lady of the two lands' (Ahmose-)Merytamun,
whose coffin was found in a tomb at Deir el-Bahri, was his sister and
consort. It should be noted, however, that the only connection between
the two is the fact that her coffin (like those of Ahhotep and Ahmose-
Nefertari) dates stylistically to Amenhotep I's reign. There are no
monuments of this date that refer to (Ahmose-)Merytamun, apart
from a possible reference to her on a monument in Nubia. On his year
220 BETSY M. BRYAN
8 stele, the figure of Amenhotep I was followed by king's mother
Ahmose-Nefertari and a second god's wife, king's daughter, sister, and
king's wife (not 'great') whose name was later restored as Ahmose-
Nefertari, before Horus of Miam (Aniba). This may instead have been
Merytamun, who had been elevated to queen, but then predeceased
Ahmose-Nefertari. Monuments that represent the presence of female
royal family members at border regions are attested several times in
the 18th Dynasty, perhaps following an older tradition. There are rep-
resentations of this type at Sinai, the Aswan rock outcrops, and Nubia
from the first to the fourth cataracts, in the Middle and New kingdoms.
Perhaps they are meant to link the queens and princesses to Hathor,
goddess of foreign lands, whose role as daughter of the sun-god was to
be protective of her father.
Another female family member in the early i8th Dynasty was
Amenhotep I's daughter, king's sister, and god's wife, Satamun, who
is known both from her coffin in the royal mummy cache and from
two statues at central and southern Karnak. Attested from the reign of
Ahmose onwards, she never became queen, but appears to have been
honoured by Amenhotep I, along with Ahmose-Nefertari, for her
priestly role as Amun's wife. Even in the Ramessid Period, Satamun
and Merytamun were both venerated as members of the family of
Ahmose-Nefertari and were included in scenes depicting the deified
royal family. Precise chronology of the early i8th Dynasty and specific
genealogy of the family appears to have been as obscure to the late New
Kingdom Thebans as it is to us today, so we cannot rely on these votive
references to provide secure parentage.
It is interesting to note that, notwithstanding the kings' apparent
ability to marry as many women as they wished, no offspring of
Amenhotep I have been identified with certainty, despite his twenty-
year reign. A king's son Ramose kn wn from a statue now in Liverpool
may have been from the Ahmosid tamily, but his specific parentage is
not given. None the less, perhaps owing to the stability provided by
Amenhotep's rule, the succession passed without event to Thutmose I,
who is not known to have been a member of the Ahmosid family.
Thutmose I and his Family
The first succession of the i8th Dynasty that did not descend from
father to son did not result in a lengthy reign. In 1987 Luc Gabolde
published a study of the chronology of the reigns of Thutmose I and II,
estimating eleven years for the former and three for the latter. The
THE 18TH DYNASTY BEFORE THE AMARNA PERIOD 221
short duration of Thutmose I's rule was in inverse proportion to its
impact on the character of later i8th-Dynasty kingship. Thutmose's
interest in the military and economic exploitation of Nubia may have
built upon the efforts of Amenhotep I, but his expedition to Syria
opened new horizons that led later to Egypt's important role in the
trade and diplomacy of the Late Bronze Age Near East. The effect of
Thutmose's efforts on cultural material generally is most visible today
in Thebes and Nubia, but the importance of Memphis, and regions
further north, is also evident.
Thutmose I's father is unknown, but his mother was named
Seniseneb, a rather common name of the Second Intermediate Period
and early i8th Dynasty. The families of both Ineni and Hapuseneb
(high priest of Amun under Hatshepsut) contained female members
with this name. Seniseneb appeared behind Thutmose I and in front
of Ahmose-Nefertari on the Wadi Haifa copy of the coronation stele
of Thutmose's first regnal year. Seniseneb's parentage is equally
unknown, but she had no title during her son's reign other than 'king's
mother'. Thutmose's principal wife was Ahmose, who had the titles
'king's sister, great royal wife'. Claude Vandersleyen has assumed that
she was Thutmose's own sister, primarily because she lacked the title
'king's daughter'. The king would then have been attempting to
recreate the situation of the two preceding reigns, with brother and
sister rulers. Her name may suggest, however, that Ahmose was a
member of Amenhotep I's family, perhaps by Prince Ahmose-ankh,
and that it was her important connection to the Ahmosid family
that facilitated Thutmose's accession to the throne. At present
Ahmose's origins and the succession of Thutmose cannot be better
explicated.
It was by Ahmose that Thutmose I fathered the future Queen
Hatshepsut and probably also a princess called Nefrubity, to judge
from the latter's appearance with them in scenes from the temple of
Hatshepsut at Deir el-Bahri. The 'god's wife of Amun', Ahmose-
Nefertari, died in the reign of Thutmose I and was replaced by
Hatshepsut. By a non-royal wife, Mutnefret, the king fathered the
future King Thutmose II (1492-1479 BC); the female parentage of his
two other sons, Amenmose and Wadjmose, is uncertain, but the latter
was honoured along with Thutmose I on a statue of Mutnefret dedi-
cated by Thutmose II in the chapel on the south side of the Rames-
seum. Indeed, it has been suggested that this chapel was a family
funerary temple; it would have been, therefore, more specifically a
family temple for Thutmose I's heirs by Mutnefret.
222 BETSY M. BRYAN
The Monuments of Thutmose I
Thutmose I and his viceroy Turi left monuments and inscriptions at a
number of sites in Upper and Lower Nubia. Several brick installations
may date from his reign in the region of Kenisa (at the fourth cataract)
and at Napata. Blocks from buildings (or fragments of blocks) have
survived at Sai Island, held at least since Ahmose's reign, and traces
remain at Semna, Buhen, Aniba, Quban, and Qasr Ibrim. The prob-
ability is that, apart from stelae, the monuments were small in scale,
comprising stone elements within brick structures. Thutmose III and
Hatshepsut may well have reconstructed brick buildings of this type in
sandstone, particularly at Semna and Buhen. Within the traditional
borders of Egypt, Thutmose I left indications of building at Elephant-
ine, Edfu (probably), Armant, Thebes, Ombos (near the late zyth- to
early i8th-Dynasty palace centre at Deir el-Ballas), Abydos, el-Hiba,
Memphis, and Giza. Votive objects dedicated in his name have been
found in Sinai at the temple of Serabit el-Khadim.
The materials from Thebes, Abydos, and Giza are of particular inter-
est. Giza became a major pilgrimage site during the New Kingdom, as
the location of the tombs of Khufu and Khafra, and as the cult place for
the god identified with the Great Sphinx, Horemakhet ('Horus in the
horizon'). It is no coincidence that the monuments at Giza, like those
at Abydos and Karnak, emphasized the veneration of rulers. Like
Ahmose and Amenhotep I before him, as well as the next four mon-
archs, Thutmose I chose to embellish cult places that promoted the
connections between king and god and between king and king. How-
ever, he seems to have associated himself with distant royal precursors
rather than immediate ones.
At Abydos, Thutmose I left a stele recording his contributions to the
temple of Osiris. Instead of honouring his royal predecessors directly,
he donated cult objects and statues. According to the stele, priests then
proclaimed him as the offspring of Osiris, whose intended role was to
restore the divine sanctuaries with the vast wealth given to him by the
earth deities Geb and Tatjenen. Thutmose I did not choose to honour
the two previous kings, perhaps because their monuments stressed
the Ahmosid family line of which he was not a part; instead he wished
to claim his kingship from the great gods themselves. As a royal
ideology, divine descent was common to the 18th-Dynasty kings, but it
may well have received its first emphasis in the reign of Thutmose I. It
was subsequently consistently exploited in royal inscriptions from
Hatshepsut (1473-1458 BC) to Amenhotep III (1390-1352 BC).
THE 18TH DYNASTY BEFORE THE AMARNA PERIOD 223
At Karnak, Thutmose I left an indelible mark. He enlarged and com-
pleted an ambulatory worked on by Amenhotep I around the Middle
Kingdom court, and he extended its walls westwards to join two new
pylon gates (the Fourth and Fifth) which he built as the entrance to the
temple. He then finished the court space between the two gateways.
He also completed the decoration of Amenhotep I's alabaster chapel at
Karnak, which appears to be his only claim to direct connection with
his predecessor. In northern Karnak, he replaced a monument of
Ahmose with his 'treasury', but appears to have preserved a block from
the earlier structure and built it into his own.
The Policy of Thutmose I in Nubia and Syria-Palestine
Thutmose I's campaign to Nubia was very likely the true death knell to
Kush and its capital at Kerma. The tombs of three of his officials—Turi
(king's viceroy of the south), Ahmose, son of Ibana, and Ahmose
Pennekhbet—all contained descriptions of this campaign, which
probably took place during the second and third years of his reign. The
longest description of the major battle, however, was inscribed on the
rock outcrop of Tombos, at the third cataract, a stone's throw from the
entrance to Kerma. The king's inscription described the campaign's
successes in the third and fourth cataract regions, in vividly violent
terms: "The Nubian bowmen fall by the sword and are thrown aside on
their lands; their stench floods their valleys . . . The pieces cut from
them are too much for the birds carrying off the prey to another place.'
Thutmose's armies (like those of Amenhotep I before him) then
struck out eastwards away from the Nile Valley and into the desert
behind Kerma, eventually reaching the fourth cataract area around
Kurgus and Kenisa. Since the river makes a great bend between the
third and fourth cataracts, a west-east overland route connected the
two cataracts. Thutmose I then left an inscription at Kenisa. According
to Ahmose, son of Ibana, on his consequent return from Kerma to
Thebes, 'his Majesty sailed northward, all countries in his grasp, with
that defeated Nubian bowman [probably the ruler of Kush] being
hanged head down at the [front] of the [boat] of his Majesty, and landed
at Karnak'.
Following this success, Thutmose I led his army to Syria for a first
campaign in that region. No doubt well aware of the Mitanni overlords
in the vicinity, the king steered clear of direct confrontation with
them, and, following several local successes, departed southwards to
Niy, where he may have hunted elephants. The descriptions of this
224
BETSY M. BRYAN
expedition derive only from the tombs of Ahmose Pennekhbet and
Ahmose, son of Ibana, both built and decorated in the reign of Thut-
mose III (and later). They characterize Syria as the Mitanni aggressor
with accompanying epithets otherwise unknown until late in the
fourth decade of Thutmose Ill's reign. No document contemporary
with the reign of Thutmose I mentions this campaign.
Map of Egypt and the Levant, showing the limits of incursions into the Near East
between the reigns of Ahmose and Amenhotep III (0.1^0-1^2 EC)
THE 18TH DYNASTY BEFORE THE AMARNA PERIOD 225
Egyptian engagement with Mitanni was extremely limited in the
early i8th Dynasty. Skirmishes with Mitanni vassals first occurred
during Thutmose I's reign, but the conquest of north-eastern regions
did not occur until at least thirty-six years later, when Thutmose III
began his Syrian expedition. Perhaps Thutmose I, on his brief expedi-
tion to Syria, encountered enemies and military technology beyond the
capability of Egypt's armies, which almost certainly had fewer chariots
than Mitanni at the time. Newly found relief fragments of the time of
Ahmose at Abydos, however, show that chariots were already being
depicted at the very beginning of the i8th Dynasty. Had Thutmose I
made substantial territorial or material gains, it is difficult to believe
that Mitanni would not have been mentioned more frequently on the
preserved monuments of Thutmose I, Thutmose II, or Hatshepsut. It
is instead far more likely that Thutmose I simply found the Mitanni
vassals to be superior military powers and that he departed after
leaving an inscription and perhaps conducting an elephant hunt in the
region of Niy, which lay to the south of the Mitanni-dominated cities.
A brief reference to Thutmose I's Syro-Palestinian expedition has
been preserved in a fragmentary inscription at Deir el-Bahri, associated
with the description of Hatshepsut's Punt expedition. This text, which
essentially celebrates the fame of Thutmose I, mentioning elephants
and horses, as well as the region of Niy, suggests that, in the time of
Hatshepsut, Thutmose I was vaunted primarily for bringing back the
exotica of the land of Niy, rather than for having conquered Mitanni.
The Tomb of Thutmose I and Royal 'Ancestor Worship'
Thutmose I's original burial location remains a subject of debate. His
name occurs on sarcophagi from two tombs in the Valley of the Kings
(KV 20 and KV 38), but there is no agreement on which of the locations
is earlier or whether either was originally excavated for Thutmose. The
body of the king may be among those from the royal cache, but this too
is uncertain. Two coffins of Thutmose I, usurped for Pinudjem I (one
of the chief-priests of Anum at Thebes in the 2ist Dynasty), contained
an unidentified mummy, which may possibly be the body of the king
himself. One of his high officials, Ineni, describes his overseeing of
the work on Thutmose's tomb: 'I oversaw the excavation of the cliff
tomb of his Majesty, in privacy; none saw, none heard.' His vague
description of the tomb as a heret, usually taken to mean 'cliff tomb,
may indicate a location in the Valley of the Kings, but the question
remains unsettled.
226 BETSY M. BRYAN
There is no known funerary temple for Thutmose I; bricks bearing
his name—and some bearing both his and Hatshepsut's—are attested
from several locations near Deir el-Bahri's 'valley temple'. A chapel
honouring Thutmose I was included by Hatshepsut in her temple, but
this does not necessarily mean that he had no funerary cult before her
reign. Rather, she venerated her ancestral line within her funerary
temple, because such temples were both 'family' shrines and temples
honouring the union between the god Amun and the king. This
'ancestor worship' was already evident in the monuments of Ahmose
and Amenhotep I at Abydos, while non-royal tomb chapels of con-
temporary and mid-i8th-Dynasty date frequently included niches or
scenes venerating living and deceased family members.
The Brief Reign of Thutmose II
The highest preserved year date for the reign of Thutmose II is his
first, and scholarship in the 19805 and 19905 suggests that his reign
lasted for no more than three years. Hatshepsut, the half-sister of
Thutmose, served as his great royal wife and was also god's wife of
Amun. Like Ahmose-Nefertari, from whom she inherited her reli-
gious role, Hatshepsut was frequently featured in the reliefs decorat-
ing the Theban monuments of her husband, most commonly in the
guise of god's wife. Thutmose II's brief tenure has left few records of
external activities, but the Egyptian army continued to quell uprisings
in Nubia and brought about the final demise of the kingdom of Kush at
Kerma.
The nearly ephemeral nature of Thutmose II's rule is underlined by
the paucity of his monuments generally, and their absence in the north
of Egypt. Thutmose II left no identifiable tomb (not unusual in the
early i8th Dynasty) or any completed funerary temple. There are indi-
cations that the temple of Hatshepsut at Deir el-Bahri was originally
begun in the reign of Thutmose II, perhaps even then under the
queen's direction. However, it may have been intended as his (and her)
funerary cult location. A small temple near Medinet Habu was erected
for him by Thutmose III, perhaps carrying out a plan already con-
templated by Thutmose II.
Thutmose II's only major monuments are from Karnak: a pylon-
shaped limestone gateway was erected at the front of the Fourth
Pylon's forecourt. Both the gate and another limestone structure of
unknown type were later dismantled and the blocks placed in the
Third Pylon foundations. The gateway has been reconstructed in the
THE 18TH  DYNASTY BEFORE THE AMARNA PERIOD 227
Karnak 'Open Air Museum'. The structure with raised relief scenes
contained a preponderance of scenes of the king, some showing him
with Hatshepsut, and some depicting Hatshepsut alone. This building
was completed in the first years of Thutmose III (during the Hatshep-
sut regency); following her accession, the queen's agents actually
replaced the small boy-king's name in a few places with her own
cartouches. On one face of a four-sided pillar fragment Thutmose II is
shown receiving crowns, while two other sides bear reliefs of Nefrura
(his daughter) and Hatshepsut receiving life from the god. This monu-
ment may have been created after Thutmose II had died, but it is
undeniable that Hatshepsut was already an important influence on the
monarchy before her brother's death.
Other constructions in the name of Thutmose II are known from
Napata, where Thutmose I may already have left building remains. At
Semna and Kumma, as well as at Elephantine, there are surviving
blocks from buildings of Thutmose II. In addition, recent exavations at
Elephantine have revealed a statue that was dedicated by another ruler
(presumably Hatshepsut) in the name of her 'brother'; Vandersleyen
notes that there is-also an identical uninscribed royal torso in the
Elephantine Museum.
The only known military expedition of Thutmose II's reign is
recorded on a rock-cut stele at Sehel, south of Aswan. It is dated to the
first year of his reign and describes a local uprising in Kush that was
punished with the death of all involved, except for one son of the ruler
of Kush, who was brought back as a hostage, evidently resulting in the
restoration of peace. Clearly this was a minor rebellion, but the family
of the local Kerma king was still active, so the action was brutal and
swift. This effectively ended Egypt's major problems with Kush.
Inhabitants of the region were pursued through the desert from near
an Egyptian fortress on the river.
Ahmose Pennekhbet notes in his funerary inscriptions that numer-
ous Shasu were brought away as prisoners for Thutmose II during an
otherwise unattested campaign. Since the ethnic term Shasu could
refer to peoples of either Palestine or Nubia, this brief entry probably
referred to the year i Nubian expedition. It is important to note again,
however, that these autobiographies were carved on the wall several
decades after the events they describe. The effects of creating a single
narrative may have made any single entry somewhat less than com-
plete.
Thutmose II's mother, Mutnefret, was alive in his reign, to judge
from the statue dedicated for her in the Wadjmose chapel at Thebes
228 BETSY M. BRYAN
mentioned above. Although the king's age at accession (and death) is
unknown, it is quite possible that he was younger than his sister and
wife Hatshepsut. She was the offspring of Thutmose I and Ahmose,
the queen officially recognized in the previous reign. A stele of
Thutmose II's reign shows the king followed by Ahmose and Hatshep-
sut. Apparently the latter was already 'god's wife of Amun' in the.reign
of Thutmose I, following Ahmose-Nefertari's death. Thutmose II was
not so young that he could not father a child, however, since Nefrura is
portrayed at Karnak with him and Hatshepsut.
The Regency of Hatshepsut
The fifty-four-year reign of Thutmose III began in his early childhood
with Hatshepsut, his aunt and stepmother, acting as regent. According
to Ineni, whose funerary 'autobiography' ended just before Hatshepsut
became ruler: 'his [Thutmose II's] son was set in his place as king of the
Two Lands upon the throne of him who engendered him. His sister, the
god's wife Hatshepsut, executed the affairs of the Two Lands according
to her counsels. Egypt worked for her, head bowed, the excellent seed of
the god, who came forth from him ...'. Ahmose Pennekhbet's inscrip-
tion similarly refers to Hatshepsut's regency in unabashed terms, not
only describing her as god's wife but also calling her Maatkara, which
was her chosen throne name (prenomen).
It has been argued that Hatshepsut saw herself as Thutmose I's heir
even before her father died, thus implying that the dating of Thutmose
Ill's rule may have applied to her own reign as much as to the child
king's. It is also possible that she capitalized on the role of'god's wife
of Amun', its economic holdings, and its connection to the family of
Ahmose-Nefertari (possibly Hatshepsut's own genealogical link,
through her mother, Ahmose) in order to support her regency in a
manner similar to her female predecessors, Ahhotep and Ahmose-
Nefertari. She also appears to have been preparing Nefrura for the
same type of role.
However, once Hatshepsut had given herself a throne name and
begun to transform herself publicly into a king, she can have had only
one certain earlier model to follow: Sobekkara Sobekneferu (1777-1773
BC), the woman who ruled at the end of the i2th Dynasty (see Chapter
7). Hatshepsut did not attempt to legitimize her reign by claiming to
have ruled with or for her husband Thutmose II. Instead she empha-
sized her blood line, and in the period before she had taken a throne
name the royal steward Senenmut left an inscription at Aswan
THE 18TH  DYNASTY BEFORE THE AMARNA PERIOD 229
(commemorating the quarrying of her first obelisks), naming her as:
'king's daughter, king's sister, god's wife, great royal wife Hatshepsut'.
At Deir el-Bahri, scenes and texts of Hatshepsut claim that Thutmose I
had proclaimed her as heir before his death, and that Ahmose had
been chosen by Amun to bear the new divine ruler. Hatshepsut
had the same pure genealogy as Ahmose-Nefertari, Ahhotep, and
Sobekneferu. The latter was never a queen: she was a king's daughter,
whose embodiment of the pure family line was apparently sufficient to
maintain her rule as pharaoh. Hatshepsut must have felt she embod-
ied the same aspects, and for nearly twenty years she was correct.
Her only known offspring (by Thutmose II) was Nefrura, who was
frequently described as 'king's daughter' and 'god's wife', and also,
more than once, 'mistress of the two lands' and 'lady of Upper and
Lower Egypt'. The debate continues as to whether she was wife to
Thutmose III during the co-regency period, but she did appear as god's
wife with him as late as the twenty-second or twenty-third year of his
reign. At some time Thutmose III replaced her name with that of
Sitiah, whom he married after his sole rule began. If Nefrura was ever
'king's great wife' to Thutmose III, the king must have ended the
formal relationship soon after Hatshepsut's disappearance in the
twentieth or twenty-first year of his reign. Children born to Nefrura are
not explicitly identified, although the prince Amenemhat has been
suggested as her son on purely circumstantial grounds.
Hatshepsut's Ambitious Building Projects
As ruler, Hatshepsut inaugurated building projects that far out-
stripped those of her predecessors. The list of sites touched by
Thutmose I and II was expanded in Upper Egypt, to include places that
the Ahmosid rulers had favoured: Kom Ombo, Nekhen (Hierakon-
polis), and Elkab in particular, but also Armant and Elephantine. Both
Hatshepsut and Thutmose III left numerous remains in Nubia: at
Qasr Ibrim, at Sai (a seated statue of the queen recalling those of
Ahmose and Amenhotep I), Semna, Faras, Quban, and especially
Buhen, where the queen built for Horus of Buhen a peripteral temple
of a type common in the mid-i8th Dynasty. The scenes on the walls of
the temple originally included figures of both Hatshepsut and
Thutmose III, but he later replaced her name with his own and that of
his father and grandfather. The Buhen temple (now entirely moved to
the Khartoum Museum) contains scenes of Hatshepsut's coronation
and veneration of her father.
230 BETSY M. BRYAN
Memphis may have received attention from Hatshepsut as ruler. An
alabaster jar fragment from the region of the Ptah temple has been
identified, but, more significantly, the colossal Egyptian alabaster
sphinx that sits within the south wall of the Ramessid temple precinct
may have formed part of an earlier approach to the temple and was
very likely accompanied by a second sphinx. The Hatnub quarries,
probable source of stone for the sphinx, were located in Middle Egypt,
not very far from another of her monuments, the rock-cut shrine at
Beni Hasan that is now called the Speos Artemidos. Apart from the
evidence of quarrying at Hatnub, there is no record of i8th-Dynasty
kings building in Middle Egypt before Hatshepsut, and her lengthy
inscription at Speos Artemidos documented that she was the first to
restore temples in the area since the destructive days of the wars with
the Hyksos. During those wars, Middle Egypt was a strategic region,
owing to the roads stretching through the Western Desert to oases,
and thence south to Nubia.
Hatshepsut claimed in her inscription to have rebuilt temples at
Hebenu (the capital of the Oryx nome), at Hermopolis, and at Cusae,
and to have acted for the lioness-goddess Pakhet sacred to the region
around the Speos itself. This work must have been carried out under
the supervision of Djehuty, overseer of the treasury and also nomarch
in Herwer in Middle Egypt, as well as overseer of priests of Thoth in
Hermopolis. The inscriptions in his tomb at Dra Abu el-Naga mention
the numerous works he supervised on behalf of Hatshepsut, and
invoke a number of regional deities, including Hathor of Cusae. The
gods of those cult centres (Horus, Thoth, and Hathor, respectively)
therefore received—like the other deities of Nubia and Egypt—a new
share of the economic resources of Egypt.
However, no site received more attention from Hatshepsut than
Thebes. The temple of Karnak grew once more under her supervision,
with the construction work being directed by a number of officials,
including Hapuseneb (her high priest of Amun), Djehuty (the over-
seer of the treasury, mentioned above), Puyemra (the second priest of
Amun), and, of course, Senenmut (the royal steward, also mentioned
above). With the country evidently at peace during most of the twenty
years of her reign, Hatshepsut was able to exploit the wealth of Egypt's
natural resources, as well as those of Nubia. Gold flowed in from the
eastern deserts and the south; the precious stone quarries were in
operation, Gebel el-Silsila began to be worked in earnest for sandstone,
cedar was imported from the Levant, and ebony came from Africa (by
way of Punt, perhaps). In the inscriptions of the queen and her
THE 18TH  DYNASTY BEFORE THE AMARNA PERIOD 231
officials, the monuments and the materials used to make them were
specifically detailed at some length. Clearly Hatshepsut was pleased
with the amount and variety of luxury goods that she could acquire and
donate in Amun's honour; so much so that she had a scene carved at
Deir el-Bahri to show the quantity of exotic goods brought from Punt.
Likewise, Djehuty detailed the bounties from Punt that Hatshepsut
donated to Amun, and he also described the electrum from the mines
in the Eastern Desert, with which he was entrusted to embellish
Karnak. Djehuty, Hapuseneb, and Puyemra all described participating
in the making of the ebony shrine donated at Mut's temple of Isheru at
Karnak. Work in that temple was conducted for Hatshepsut by Senen-
mut, whose name occurs on a gate excavated there, but Hapuseneb
also left a statue in the precinct.
At Karnak Hatshepsut left, most significantly in terms of her per-
sonal imprimatur, the Eighth Pylon, a new southern gateway to the
temple precinct. Lying along the north-south processional way that
connected Karnak central to the Mut precinct, the new sandstone
pylon was the first stone-built one on that route. Ironically, evidence of
Hatshepsut's building effort is today invisible, since the face of the
pylon was erased and redecorated in the first years of Amenhotep II
(1427-1400 BC), son of Thutmose III. Nevertheless, Hatshepsut's
desire to create a new main entrance was part of a grander plan,
designed to ensure that her involvement with the temple would not be
forgotten easily. By connecting Karnak to Mut's temple, the queen was
perhaps deliberately shifting attention away from Thutmose II's gate-
way before the Fourth Pylon. She likewise built a temple in the north-
south alley dedicated to Amun-Ra-Kamutef, a creator form of the god.
Taken together, her constructions at Luxor temple, to the south, which
housed the yearly royal renewal festival, the Mut temple, where Amun's
consort resided, and the Kamutef shrine formed a set of buildings in
which Hatshepsut could describe and celebrate her birth from Amun,
gain the favour of the deities for her rule, and expand the claim to
divinity for the kingship itself
Elsewhere in Karnak central Hatshepsut had a palace built for her
ritual activities, and she constructed a series of rooms around the
central bark shrine where she had depicted her purification and
acceptance by the gods. Precisely where she had her great quartzite
bark shrine set up remains an issue of debate, but it is now being
reconstructed in the Open Air Museum at Karnak. This shrine bears
depictions of the processions associated with the Opet Festival (in
which Amun of Karnak visited Luxor temple) and the Beautiful Feast
232 BETSY M. BRYAN
of the Valley. During the latter festival, Arnun left Karnak to travel
westwards to Deir el-Bahri and the temples of other rulers. This
festival became the most prized one on the Theban west bank during
the New Kingdom.
Hatshepsut had a tomb excavated in the Valley of the Kings for
herself as ruler. Tomb KV 20 appears to be the earliest tomb in the
valley, and Hatshepsut had it enlarged to accommodate both her own
sarcophagus and a second that had been initially carved for herself
but then recarved for her father Thutmose I. Both Hatsheput and
Thutmose I may have initially been laid to rest there, but Thutmose III
later removed Thutmose I's body to KV 38, which he had built for a
similar purpose. The confusion of multiple tombs and sarcophagi for
Hatshepsut is not entirely at an end, but research by Luc Gabolde and
others has contributed to a better understanding of early work in the
Valley of the Kings. The queen also built a temple to Amun at Medinet
Habu at the southern end of Thebes. Completed by Thutmose III, this
chapel housed an important cult of the god on the west, becoming part
of the regular festival processional cycle which included Deir el-Bahri
and Karnak, and later also involved Osiris.
The Temple at Deir el-Bahri: A Statement of Hatshepsut's
Reign
The temple at Deir el-Bahri remains Hatshepsut's most enduring
monument. Built of limestone and designed in a series of terraces set
against the cliff wall in a bay formed naturally by river and wind action,
the temple called 'Holy of Holies' (djeser djeseru) was Hatshepsut's
most complete statement in material form about her reign. The design
of the temple followed a form known since the First Intermediate
Period, and particularly inspired by the nth-Dynasty temple of
Mentuhotep II (2055-2004 BC) just to the south. Terrace temples,
however, had continued to be built in the Second Intermediate Period
and, more recently, in the early i8th Dynasty (most particularly by
Ahmose at Abydos). Hatshepsut borrowed forms developed by many
of her royal ancestors; for example, colossal Osirid statues set in front
of square pillars on her colonnades resemble closely statues of
Senusret I. Hatshepsut's inspiration may instead have been her father,
Thutmose I, however, since his Osirid colossi at Karnak, although of
sandstone, were similar to those at Deir el-Bahri.
By the time of its completion, the temple contained scenes and
inscriptions that carefully characterize a number of projects and events
THE 18TH  DYNASTY BEFORE THE AMARNA PERIOD 233
in the life and rule of Hatshepsut. The most accessible areas, the lower
and middle colonnades, showed, for example, a Nubian campaign, the
transport of obelisks for Karnak temple, an expedition to Punt to bring
back incense trees and African trade products, and the divine birth of
the ruler. Officials associated with the work were mentioned by name
in the inscriptions, including the treasurer Nehesy and Senenmut.
The funerary inscriptions of Djehuty and Senenmut suggest that they
were both active in the building and embellishment of the 'Holy of
Holies' temple at Deir el-Bahri.
On the south end of the middle terrace, a chapel was constructed for
Hathor, goddess of the western cemetery, and it was fronted by a pil-
lared court, whose capitals were fashioned as emblems of the cow-
faced deity. Scenes of the king feeding the sacred cow flank the
entrance to the chapel itself. On the upper terrace there was a central
door into a peristyle court behind which was the main temple sanctu-
ary. Scenes of the Beautiful Feast of the Valley procession decorated
the north side of the court, while the Opet Festival appeared on the
south. Another enclosed court to the north contained niche shrines to
the gods, including Amun, and a large Egyptian alabaster open-air
altar for the sun-god Ra-Horakhty. This sun-temple feature was a sig-
nificant addition to the complex, recalling an old form seen as early as
the 3rd Dynasty Step Pyramid at Saqqara. Its meaning for the royal cult
was further underscored in rooms on the south of the central court,
where the ruler's desire to accompany the sun-god on his daily route
through the heavens and the netherworld was expressed in scenes and
texts. Hymns describing the deities who governed each hour of the day
and night gave Hatshepsut power over time itself so that she could
merge with the sun for eternity. On this terrace, too, were chapels for
Hatshepsut herself and for her father, Thutmose I. An inscription
accompanied a scene of that king declaring his daughter's future
reign.
A set of phrases designed to communicate with the few who could
read and who would ever see these private areas of the temple allude
obliquely to the unusual nature of Hatshepsut's rule. Her high officials
are twice warned: 'he who shall do her homage shall live, he who shall
speak evil in blasphemy of her Majesty shall die.' It is likely that this
was the official court position of the time and that the inscription
merely monumentalized a statement well known to elite circles of the
time. Hatshepsut was very generous to those who supported her, judg-
ing from the sudden increase in large decorated private tombs at
Thebes and Saqqara, as well as the increasing number of private
234 BETSY M. BRYAN
statues dedicated in temples such as Karnak. The ruler appears to have
forged a symbiotic relationship with her nobles, so that she became as
important to them as they were to her. During this period, for the first
time in Theban private tombs, the enthroned ruler appears arrayed
like the sun-god himself, acting as an eternal intermediary for the
tomb-owner. The Theban tombs of the royal steward Amenhotep
(TT 73) and the royal butler Djehuty (TT no) show Hatshepsut in this
manner, and several tombs dating to the sole rule of Thutmose III con-
tinued the practice. Such loyalist representations recall the inscribed
stelae of the Middle Kingdom elite that described how the i2th-
Dynasty kings acted for the good of Egypt.
Foreign Relations in the Reign of Hatshepsut
Hatshepsut's co-regency with Thutmose III was not a period of pro-
tracted warfare. There were several Nubian military expeditions that
appear to have dealt with local uprisings, but nothing indicates that
overall administration of the south by the 'viceroy and overseer of
southern countries' was interrupted. The viceroy Seni gave way to
Amennakht during Hatshepsut's reign, and the latter ceded to Nehy
under Thutmose Ill's sole rule. At least one other viceroy was in
service at the end of Hatshepsut's tenure, but his name is uncertain.
Each of these men not only governed Nubia but also supervised con-
struction projects. They oversaw the delivery of Nubian products as
'tribute' to the ruler, but no doubt saw little direct military action.
Hatshepsut's trade mission to Punt was promoted in Egypt as a
major diplomatic coup. The African products that were brought back,
along with gold and incense (including the incense trees themselves),
stimulated interest in exotic luxury goods. Soon the Nubian tribute-
bearers were pictured in private tomb paintings bringing the same
items: ivory tusks, panther skins, live elephants, and, of course, gold. It
is not entirely clear how the mission to Punt opened more extensive
trade to areas of Africa south of Egypt's control, but it was only after
this time that consistent reports of Nubian tribute from the conquered
regions were recorded, including lists of the exotic materials obtained.
The possibility exists that Egypt's connection to the Aegean, as
attested by the Minoan paintings at Tell el-Dab e a (Avaris), underwent a
change during Hatshepsut's reign. Although Avaris continued to be
occupied until the reign of Amenhotep II, there is no certain indi-
cation that Egypt was in contact with Crete following the first part of
the 18th Dynasty. Trade may have been maintained through Cyprus
THE 18TH DYNASTY BEFORE THE AMARNA PERIOD 235
and the Levant, however, since imported pottery occurs in some
quantities. In the reign of Hatshepsut, when delegations of Keftiu
(Minoans, judging from the Egyptian representations) appear along-
side other foreign emissaries in mural paintings from Theban private
tomb chapels, Egypt may have forged its own trade connection with
Minoan Crete or Mycenaean Greece. The consistency of the contact,
however, is dubious. Similar paintings in the reigns following
Hatshepsut show less familiarity with the dress and trade objects from
Crete, and scholars have concluded that the trade contact may have
been through Syria-Palestine rather than directly.
Thutmose Ill's Sole Rule
The kingship reverted to Thutmose III alone sometime in the twentieth
or twenty-first year of Hatshepsut's reign. He then wasted little time in
establishing a reputation both for himself and for Egypt that was to be
remembered a millennium later, if somewhat imperfectly. Thutmose
III must have carefully assessed his situation as a now mature but
unproven ruler and, no doubt with counsel from associates and fellow
military colleagues, identified the potential for glory and wealth lying
to the north-east. The rewards of conquering Nubia could not belong
to Thutmose III, and Hatshepsut had reaped what there was from
establishing contact with Punt. The new locale for quick gains was the
Levant, where Egypt might gain control of the trade routes that had
until then been dominated by Syrian, Cypriot, Palestinian, and Aegean
rulers and traders. At the end of some seventeen years of military
campaigns, Thutmose III had successfully established Egyptian domi-
nance over Palestine and had made strong inroads into southern Syria.
His own reputation was assured, and the proceeds were extravagantly
expended on behalf of the temples of Amun and other gods, as well as
on those men who followed the king on his quests.
The king did not dishonour the name and monuments of Hatshep-
sut until the last years of his reign, but instead attempted to fill the
landscape of the Nile Valley with reminders of his own reign. It is
interesting to note that the artistic style and portraiture of Thutmose
III are extremely difficult to differentiate from those of Hatshepsut in
her later monuments. Only in his body type did Thutmose choose to be
shown somewhat differently, for his images routinely show him with
broader shoulders and a heavier upper torso than Hatshepsut in both
relief and statuary, and this more virile body type was the one used
later by Amenhotep II. The face of Thutmose III continued the
236 BETSY M. BRYAN
'Thutmoside' profile seen already with Thutmose I, comprising a long
nose with slight hump and downturned end, broad at the base. The
mouth was wide, with a protruding upper lip due to the overbite that
ran in the family.
Thutmose III used his thirty-two years of sole rule to make his name
prominent throughout Egypt and Nubia. He was active at Gebel Barkal
at the farthest southern point in Nubia, at Sai, Pnubs at the third
cataract, Semna, Kumma, Uronarti, Buhen, Quban, Amada, Faras,
and Ellesiya, as well as several other locations where blocks are known
in his name. His monuments further north are well attested at
Elephantine, where he built a temple to the goddess Satet of the first-
cataract region, at Kom Ombo, Edfu, Elkab, Tod, Armant, Thebes,
Akhmim, Hermopolis, and Heliopolis. A statue of the overseer of
works, Minmose, active in the later reign of Thutmose III, listed cult
sites at which he worked. He named, in addition to the places men-
tioned already, Medamud, Asyut, Atfih, and a number of localities in
the Delta, including Buto, Busiris, and Chemmis. Although no build-
ings of Thutmose III have yet been identified in the Delta, Minmose's
inscription suggests that he and earlier 18th-Dynasty kings may well
have been active there.
Karnak continued to be a favoured site. Thutmose III somewhat
ruthlessly restructured the central areas of the temple, removing
Amenhotep I's cult chapels of limestone and replacing them in
sandstone. Soon after beginning his period of sole rule, he inaugu-
rated the construction of his major building in Karnak: '[Thutmose III
is] Effective of Monuments' (akh menu). The overall theme of the relief
scenes in the building concerns the renewal of Thutmose Ill's
kingship, primarily through the serf-festival, which he first celebrated
in the thirtieth year of his reign. The veneration of kingship generally
fitted well with this purpose for the building and connected it with the
chapels around the central court. Later in his reign, Thutmose III had
the entire central area redecorated with scenes and particularly inscrip-
tions detailing his campaigns in Asia. These Annals, inscribed in the
forty-second year of his reign, have become the primary historical
record of the king's conquests, containing, as they do, specific episodes
of the warfare and lists of booty taken. The enrichment of the Amun
temple was enormous as described in the Annals: the buildings alone
were numerous. The Sixth and Seventh Pylons were added by the king,
the latter covered with scenes and inscriptions naming the places over
which he claimed mastery. A temple to the god Ptah was built on the
north side of the precinct, and a granite bark shrine was made for the
THE 18TH DYNASTY BEFORE THE AMARNA PERIOD 2 37
centre of the temple, as well as an Egyptian alabaster one later joined to
a shrine of Thutmose IV (1400-1390 BC) and set near the Fourth
Pylon. Transformations to the works of Hatshepsut also took place in
the reign of Thutmose III and were completed by his son Amenhotep
II, but even without these the activity was unceasing. The king's high
priests of Amun included the energetic Menkheperraseneb, owner of
Theban tomb 86, his nephew of the same name (TT 112), and
Amenemhat (TT 97). Amenemhat was probably Thutmose Ill's last
high priest of Amun and largely in service under Amenhotep II, after
Menkheperraseneb handed over the office to his nephew for a brief
period.
The high priests were responsible not only for Karnak, but for works
on Amun's behalf on the west bank as well. Thutmose III was
extremely active at Medinet Habu, where he completed the small
temple to Amun and also built a memorial temple for his father just to
the north. Late in his reign, he converted an elevated shrine at Deir el-
Bahri into his own chapel called 'Sacred Horizon' (djeser akhet). The
tomb of Thutmose III in the Valley of the Kings (KV 34) was hewn high
in a cliff, descending deep into the rock face. The walls of the burial
chamber are covered with black- and red-painted hieratic renditions of
the netherworld texts: the Litany ofRa, which calls upon the names of
the sun-god to aid the king in his afterlife journeys, and the Book of
what is in the Netherworld (Amduat), which provided the king with a
map of the underworld and spells to help him achieve eternal justifi-
cation.
Thutmose III in the Levant
Almost immediately after his sole rule began, Thutmose III began an
expedition to the Levant, where he sought to wrest control of a number
of city states and towns who recognized a Mitannian overlord from
north-east Syria. Having apparently taken as an excuse the need to deal
with local squabbles in Sharuhen and its vicinity, the king went to
Gaza from the Egyptian border fortress at Tjaru. Gaza had been under
Egyptian rule at least since Ahmose's time, and we presume that
Sharuhen's loyalty had been expected since the same reign. The
Annals record that in this first campaign of his twenty-third regnal year
Thutmose III left Gaza and planned his attack on Megiddo from the
city of Yehem, a major city-state then occupied by the ruler of Kadesh.
It was also protected by a group of chiefs representing regions of
the Levant as far as Nahrin (Mitanni and Mitanni-dominated Syria).
238 BETSY M. BRYAN
Thutmose's inscription indicated that these chiefs should have been
loyal to Egypt, and this must be seen as the true threat. Access to
Lebanese cedar, copper and tin sources, and other prized products
may have been jeopardized by Mitanni overlordship in northern
Palestine and the coastal strip.
Once in the field, Thutmose III discovered the actual rewards of
war. The spoils were evidently so great that he continued to campaign
intermittently, until the forty-second year of his reign, in the regions of
northern Palestine, the Lebanon, and parts of Syria. The spoils taken
from the battle of Megiddo, together with the peace offerings that
ended the seven-month siege of the town, were considerable and
included 894 chariots, including two covered with gold, 200 suits of
armour and two of bronze belonging to the chiefs of Megiddo and
Kadesh, as well as over 2,000 horses, and 25,000 animals. Following
the siege of Megiddo, Thutmose III replaced the defeated local chiefs
and continued northward in the direction of the Litani River. The
luxury objects taken from the several towns he defeated were meticu-
lously described in the Annals, and the different classes of captives
taken were also enumerated. The campaigns of years 24-32 detailed
the king's focus on the Levantine littoral, with its forests and harbours,
as well as areas of west Syria. The Egyptian proceeds included a range
of materials from precious metals (gold, silver, copper, and lead) to
wood, to oils, and even foodstuffs and cereal harvests. The king sent
the children of the city rulers back to Egypt to be Egyptianized. Accord-
ing to the Annals for year 30, 'whoever died from among these chiefs,
his Majesty caused that his son stand in his place'.
If we are correct in assuming that the toponym Nahrin does not
feature in Egyptian inscriptions before Thutmose Ill's eighth cam-
paign (in year 33 of his reign) simply because they were regarded as too
powerful to be mentioned on Egyptian royal monuments, then the
king's conquest of the Syrian vassals was a truly significant achieve-
ment. The hitherto poorly attested state of Nahrin suddenly appears in
the later years of Thutmose Ill's reign in every type of hieroglyphic
inscription: in addition to the Annals of Thutmose III, the king's
apparent crossing of the Euphrates appears in the Gebel Barkal Stele
erected at the fourth cataract in Nubia, on a Karnak obelisk, on the
Poetical Stele from Karnak, and on the Armant Stele. References to
Nahrin also occur among the numerous toponym lists from the reign.
The amount of booty taken during the Syrian campaigns was impres-
sive, both for the ruler and for his soldiers. With the exception of the
aftermath of the eighth campaign, in year 33, throughout the Annals
THE 18TH DYNASTY BEFORE THE AMARNA PERIOD 239
revenue from Nahrin was listed as booty, either the plunder of the
army or what the king captured. Apparently Nahrin did not at this time
offer yearly deliveries (inu), as the Annals clearly indicate by contrast-
ing its one-time delivery after the year-33 campaign with that of other
areas designated as 'from this year'. This might be interpreted to mean
that the defeated Mitanni vassals alone were the source of Egypt's
revenues, not the Mitanni king in his capital, Washshukanni. Although
the listed objects and people taken from Nahrin are sizeable, the yearly
deliveries from Retenu and Djahy included far more items of precious
materials. Clearly Thutmose III was still in the process of warfare with
Mitanni.
The participation in the conquest of Syria, including Nahrin, by a
newly formed Egyptian military elite is commemorated in at least
eleven Theban tombs from the reign of Thutmose III and early in that
of Amenhotep II, in addition to numerous private statue and stele
inscriptions (tombs TT 42, 74, 84, 85, 86, 88, 92, 100, 131, 155 and
200). In these tomb chapels, the emphasis was upon the captives of
military expeditions and upon the wars or soldiers themselves, as
much as it was upon luxury items acquired from foreign deliveries.
The military aspect of Egyptian-Mitanni encounters was to be short
lived, however. Instead, the prestige of things Syrian began to soar.
Tombs decorated after the first decade of Amenhotep II's rule cele-
brated the revenues as foreign impost, particularly of an exotic nature,
the elements of conquest being formalized within celebratory proces-
sions. For example, in the tomb of Kenamun (TT 93), decorated late in
the reign of Amenhotep II, there is no text describing the Syrian wars,
no accounting of booty as in Suemniwet's chapel (TT 92), or presenta-
tion of the foreign chiefs' children, as in Amenemheb's (TT 85).
Instead, one wall shows the New Year's presents for the king. Among
them are numerous weapons and coats of armour, as well as two
chariots. The label for the chariot in the higher register boasts of the
wood being brought from the foreign country of Nahrin, while a chariot
below it is designated for use in warfare against the southerners and
the northerners. A pile of Syrian-style helmets is beneath the upper
chariot, while a heap of ivory is beneath the lower one—clearly an
allusion to former warfare in the two regions (Asia and Nubia
respectively).
Also among the New Year's gifts in Kenamun's tomb is a group of
glass vessels imitating marble. This type of glass was particularly
characteristic of north-east Syria and northern Iraq. Indeed, the large-
scale introduction of core-formed glass into Egypt may well have been
240 BETSY M. BRYAN
a direct result of the Mitanni wars. Quite possibly first developed in
Mitanni centres, such as Tell Brak and Tell Rimah, glass vessels
quickly became among the prized objects copied (and frankly
improved upon) in Egypt. Silver and gold vessels (often described in
the booty lists as 'flat bottomed') associated with the Mediterranean
littoral (referred to as the 'workmanship of Djahy') also came as
revenue from Nahrin (in year 33), and, as with glass, Egyptian-style
copies of these Syrian vessels rapidly became fashionable. The famous
flat-bottomed silver vessel inscribed for the soldier Djehuty under
Thutmose III is just such a bowl a gold bowl of Djehuty, also at
the Louvre, may be a modern copy of the silver one, and there are
numerous representations of them from temple and tomb walls in
Thebes.
Along with Syrian-style luxury items came the gods of the region,
and it is in the reign of Amenhotep II that the cults of the Asiatic
deities Reshef and Astarte were heavily promoted in Egypt. It is
significant that the fashion for Mitanni-style items far outlasted the
fashion for military decoration. A special type of gold lion award that
was issued to soldiers in the Syrian campaigns is not found after the
early reign of Amenhotep II, but Syrian-style metal and glass vessels
continued to be status symbols throughout the i8th Dynasty and were
copied in a variety of forms within Egypt. Likewise, the scenes of
presentation of Mitanni war captives and booty gave way after the early
reign of Amenhotep II to the preferred scene of foreign representa-
tives offering their prized luxury objects in obeisance to the pharaoh.
In the iconographical transformation of Mitanni from arch-enemy
to a compliant source for prestige luxury goods, we can track Egypt's
path towards an alliance with Nahrin. It is not certain that the three
wives of Thutmose III buried in the Wadi Qubbanet el-Qirud (in
western Thebes) were Syrian, but their names were certainly Asiatic
and their wealth in gold was profound. This perhaps reflects the
changing Egyptian view towards the east—the same king who cam-
paigned to conquer Retenu and Nahrin for twenty years then married
women from the region and showered them with riches. Despite the
battles of Amenhotep II yet to be fought in Syria, Egypt's interest in
peace was imminent at the close of Thutmose Ill's reign.
Thutmose Ill's wives included one woman called Sitiah, daughter
of a royal nurse. She had the titles of 'great royal wife' and—in one
surviving text—'god's wife'. If she in fact replaced Nefrura in the priest-
ess's position, it was only until Thutmose Ill's daughter Merytamun
was old enough to take up the role. Sitiah is not definitely known to
THE 18TH DYNASTY BEFORE THE AMARNA PERIOD 241
have had any children, while the mother of Amenhotep II, Merytra,
appears to have produced several children. Merytra (daughter of Huy,
a divine adoratrice of Amun and Atum, and chief of choristers for Ra)
apparently gave birth to Amenhotep, Princess Mery(t)amun, prince
Menkheperra, Princesses Isis and another Mery(t)amun, and a small
Princess Nebetiunet. Merytra as queen appeared in the temple of
Medinet Habu and in the tomb of Thutmose III. A third wife, Nebetta,
and a Princess Nefertiry are depicted in the royal tomb.
Amenhotep II
It is not known whether any members of Hatshepsut's branch of the
family (descended from Queen Ahmose) were still alive at the end of
Thutmose Ill's reign. The ageing king, however, did take his son
Amenhotep as co-regent in the fifty-first year of his reign, and then
shared the monarchy with him for a little more than two years. The so-
called dishonouring of Hatshepsut, which had begun around year 46
or 47, may have paved the way for the joint rule, for Amenhotep II
himself completed the desecration of the female king's monuments.
In order to eliminate the claims of Hatshepsut, and her family line, her
monuments were systematically adjusted: some were obscured by new
work; some were mutilated to remove any evidence of her name; and
many were altered such that the names of Thutmose III or Thutmose
II replaced those of Hatshepsut. Since Thutmose sought to destroy the
memory of the queen twenty-five years after her disappearance, it is
unlikely that this was carried out as pure vengeance against his step-
mother, particularly since the king had retained a number of Hatshep-
sut's officials, who completed their career and built tombs with the
name of Thutmose III prominently inscribed in them. Perhaps the
death of men who served both rulers, such as Puyemra, second priest
of Amun, and Intef, the mayor of Thinis (the region of Abydos) and
governor of the oases, also vitiated objections to the execration of
Hatshepsut.
Amenhotep II's reign was a pivotal one in the early New Kingdom,
although today it is often dwarfed by the shadow of his two pre-
decessors and his successors in the late i8th Dynasty. During a reign
of nearly thirty years (with a highest known regnal year of twenty-six)
the king had military successes in the Levant, brought peace to Egypt
together with its economic rewards, and faithfully expanded the
monuments to the gods. In his own time Amenhotep II commanded
recognition most particularly for his athleticism, and his monuments
242 BETSY M. BRYAN
often allude to this capability. As a young man, the king lived in the
Memphite region and trained horses in his father's stables (if we are to
believe the inscription he left on a stele at the Sphinx temple at Giza).
His greatest athletic achievement was accomplished when he shot
arrows through copper targets while driving a chariot with the reins
tied around his waist. The fame of this deed was monumentalized not
only in the stele inscription from Giza but in carved relief scenes
in Thebes. It was also miniaturized on scarabs that have been found
in the Levant. Sara Morris, a classical art historian, suggests that
Amenhotep II's target shooting success formed the basis hundreds of
years later for the episode in the Iliad when Achilles is said to have shot
arrows through a series of targets set up in a trench.
The majority of Amenhotep II's reign was peaceful, providing a
lengthy period of stability. Several administrative papyri from his reign
document flourishing agricultural and industrial organizations in
several areas of Egypt. A well-developed bureaucracy was at work, and
Amenhotep II appears to have made good use of the services of admin-
istrators. He encouraged men who had served his father to stay on, and
he installed close friends of his own in key roles. Several Middle King-
dom literary compositions were recopied at this time, suggesting a
growing interest in cultural refinement rather than military valour.
Although royal art remained as idealized and highly formal as it had
been in the reign of Thutmose III, painting style in non-royal contexts
began to betray an artistic individualism that was later to be accentuated.
Amenhotep II's Building Programme
Amenhotep II left buildings or additions to standing monuments at
nearly all the major sites where his father had worked. In the first three
years of his reign, constructions in the names of the two kings were
erected, most notably at Amada in Lower Nubia, where a temple cele-
brating both equally was built to honour Amun and Ra-Horakhty, and
at Karnak, where both kings participated in eliminating the vestiges of
Hatshepsut's monuments by masking them with their own. In the
court between the Fourth and Fifth pylons the columns added and the
masonry placed around the queen's obelisks carried sometimes the
name of one ruler and sometimes the name of the other. It remains
impossible to say whether the alterations were effected simultaneously
(during a co-regency) or consecutively.
He left monuments at Pnubs on Argo Island, at Sai, Uronarti,
Kumma, Buhen, Qasr Ibrim, Amada, Sehel, Elephantine, Gebel Tingar
THE 18TH DYNASTY BEFORE THE AMARNA PERI OD 243
(a chapel near the quartzite quarry on the west bank at Aswan), Gebel
el-Silsila, Elkab, Tod (a bark chapel of the co-regency), Arrnant,
Karnak, Thebes (including his tomb, KV35 in the Valley of the Kings
and a now-destroyed funerary temple), Medamud, Dendera, Giza, and
Heliopolis. A temple construction of limestone was the object of the
reopening of the Tura quarries in year 4 of the reign, but the location of
that temple is uncertain; it was not the king's funerary temple at
Thebes, since that structure was built of sandstone and brick.
The sites where Arnenhotep II's construction efforts left the deepest
impressions were Giza and Karnak, despite the fact that the king's
work at Giza was not particularly ambitious. None the less, he built a
temple to the god Horemakhet, the sun-god identified with the Great
Sphinx. It has been noted that, since the time of Thutmose I's reign,
the area around the Sphinx was frequented by princes and pilgrims
who visited the great pyramid complexes of Khufu and Khafra. The
Sphinx and its amphitheatre became the site of a cult of royal ances-
tors, including Arnenhotep II himself and his son, Thutmose IV, who
set up the Sphinx Stele between the paws of the great lion statue. The
cult of Horemakhet and the royal veneration continued into Roman
times, such that pilgrims left votive offerings in the enclosure wall of
the amphitheatre or in the chapels if possible. Amenhotep II's dedi-
cation of a small temple to Horemakhet (also described as Hauron on
the king's foundation deposit from the site) was thus an important
development in the history of the Sphinx as a focus of worship. His
own sons left stelae in his temple, some bearing depictions that indi-
cate that a statue of Amenhotep II once stood against the breast of the
Sphinx. Mark Lehner has reconstructed the appearance of the Sphinx
with this 18th-Dynasty statue in place.
When Amenhotep II had finished his programme of erasures on
the monuments of Hatshepsut at Karnak, he was able to concentrate
on preparations for the royal jubilee at this temple. Just as Thutmose
III had constructed the festival temple known as 'Effective of Monu-
ments' in the precinct of Amun at Karnak, so Amenhotep II created a
building for his sed-festival. His pavilion, as reconstructed by Charles
Van Siclen, was a court of relief-carved square pillars with decorated
walls on the sides. Dated to the later part of his reign both by its artistic
style and its inscriptions, it fronted the temple's south entrance at the
Eighth Pylon, effectively creating a new main gateway to the complex,
just as Hatshepsut had done before him. In front of this serf-festival
court were the estates of Amun, or gardens that produced vegetables
and other sweet plants. The pillars carried the unusual dedication of'a
244 BETSY M. BRYAN
first occasion of repeating [or "and repetition of] the sed festival' which
may imply that he had already celebrated a jubilee before building this
court. These formulas are, however, difficult to interpret and may
simply be wishes expressed for the king's coming jubilees. Following
an old tradition, Amenhotep II's relief decoration in the festival
pavilion contained elaborate royal regalia for the king, that particularly
emphasized solar connections—for example, multiple sun discs on
top of crowns, and tiny falcons set above the sun discs, creating
identity with the falcon-headed Ra-Horakhty.
The small temple of Thutmose III at Deir el-Bahri had used simi-
larly extravagant solar symbolism and was also a monument dating to
the period after the king's jubilee preparations had been made. Amen-
hotep II's festival building included scenes of his mother, Merytra,
who served as his queen and, more importantly, 'god's wife of Amun'.
The building was dismantled at the end of the i8th Dynasty, to accom-
modate alterations of the quadrant by Horemheb (1323-1295 BC), and
it was later rebuilt in a different architectural form by Sety I (1294-
1279 BC) at the beginning of the i9th Dynasty.
Amenhotep II also built a temple to Amun in northern Karnak, a
precinct later dedicated to Montu of Thebes. However, the blocks of
this building now form part of the foundations of a temple constructed
under Amenhotep III and later adapted in the Ptolemaic Period. Its
original function remains unknown. Other gateways and blocks from
North Karnak, however, indicate that the king was interested in
developing this sector, perhaps because of its position in terms of
extending the north-south axis of the central part of Karnak. Stone
door elements from a palace of the king were found north of the
temple proper, perhaps indicating the location of a ceremonial resi-
dence for Amenhotep II. The king's interest in Montu's temple at
Medamud some 8 km. to the north is perhaps also notable, since later
there was certainly a processional way between northern Karnak and
Medamud.
Amenhotep II in the Levant
Amenhotep II carried out two campaigns in Syria, the first probably in
year 7, the latter in year 9. These are described on stelae left at Amada,
Memphis and Karnak. The first campaign concentrated on the defeat
of unaligned chiefs and rebellions among recently acquired vassals.
Among the latter, the region of Takhsy, mentioned in the Theban
tomb of Amenemheb (TT 85), was a primary, and successful, target.
THE 18TH DYNASTY BEFORE THE AMARNA PERIOD 2 45
The seven defeated chiefs of that region were taken back to Thebes,
head-down on the royal barge, where six were hung upon the temple
wall. One was carried all the way to Napata, in the Sudan, where his
body was hung, no doubt as an example to the local population.
According to the stelae, the plunder claimed from Amenhotep's first
campaign comprised a staggering 6,800 deben of gold and 500,000
deben of copper (1,643 an< ^ 120,833 pounds respectively), along with
550 mariannu captives, 210 horses, and 300 chariots. The second cam-
paign in year 9 was largely carried out in Palestine.
Apart from the standard toponyms in 'name rings', none of the
monumental texts of Amenhotep II contains a hostile reference to
Mitanni or Nahrin (despite the fact that the inscriptions narrated his
Syrian campaigns)—and this is probably intentional. Instead of Thut-
mose Ill's designation, 'that foe of Nahrin', Amenhotep II several
times uses the archaic Egyptian generic term setjetyu ('Asiatics'). The
language of the stelae, composed after the conflicts had ended, in year
9 or later, reflects the fact that peace with Mitanni was at hand. Indeed,
the Memphis stele contains an addition at the end, reporting that the
chiefs of Nahrin, Hatti, and Sangar (Babylon) arrived before the king
bearing gifts and requesting offering gifts (hetepu) in exchange, as well
as asking for the breath of life. This was certainly the first official
announcement of the creation of a Mitanni peace, although good
relations with Babylon and others already existed in the reign of Thut-
mose III.
The importance of Amenhotep II's new alliance with Nahrin was
underlined by its exposition in a column inscription from the Thutmo-
sid wadjyt, or columned hall, between the Fourth and Fifth Pylons at
Karnak. This location was significant, because the hall was venerated
as the place where Thutmose III received a divine oracle proclaiming
his future kingship. In addition, the association of the hall with the
Thutmoside line going back to Thutmose I, the first king to venture to
Syria, made it a logical place to boast of the Mitanni relationship. The
inscription singles out Syria, stating: 'The chiefs (weru) of Mitanni
(My-tri) come to him, their deliveries upon their backs, to request
offering gifts (hetepu) from his majesty in quest of the breath of life.' By
the close of Amenhotep II's reign the portrayal of Mitanni, so recently
the vile enemy of the king, was brought into line with that of Egypt's
other close allies. In monuments within the Nile Valley, these brother
kings of Babylon, Hatti, and Nahrin were always portrayed as sup-
pliants who requested life from the Egyptian king. The hard-won peace
with Syria is betrayed, however, by Amenhotep II's enthusiasm for it.
246 BETSY M. BRYAN
Clearly Amenhotep II considered this alliance to be a boon at home as
well as abroad.
Royal Wives in the Mid-i8th Dynasty
A number of princes can be documented for the reign of Amenhotep
II: Amenhotep, Thutmose, Khaemwaset(P), Amenemopet, Ahmose,
Webensenu, and Nedjem, as well as the unnamed Princes A and B
known from stelae left at Giza. Perhaps another, named Aakheperura,
was born late in Amenhotep's reign, or in Thutmose IV's. In striking
contrast to earlier reigns, princesses are difficult to document. The
plurality of young royal males is in contrast to the earlier part of the
dynasty when adult princes appeared to be scarce, perhaps because
they died on military campaigns, or from childhood illnesses. The
scarcity of princes, perhaps due in part to the dynastic preference for
princess sisters as queens, may have inspired rulers to take minor
queens in addition to their great royal wives. These 'royal wives', such
as Nebetta and the three Levantine queens of Thutmose III, all men-
tioned above, were probably distinct from court females of unknown
rank with whom the kings had sexual liaisons. The latter women, such
as Mutnofret, I sis, Tiaa, and Mutemwiya, produced sons who became
king and promoted their mothers as queens. It is not known, however,
which women (apart from Tiaa, mother of Thutmose IV) were the
mothers of Amenhotep II's numerous offspring.
It was not only his able procreative powers that separated Amen-
hotep II from his predecessors. Unlike those before him, this king had
no publicly acknowledged wife other than his mother, Merytra, who
served as 'great royal wife' for much of Amenhotep's reign. The
absence of wives might be considered a conscious rejection of the
dynastic role played by princesses as queens and 'god's wives of
Amun' from the establishment of the dynasty through to the reign of
Hatshepsut. Perhaps Thutmose III and Amenhotep II now realized
that queens like Hatshepsut, who represented the dynastic family,
could be dangerous if they were too wealthy and powerful. In addition
the queen-turned-king's usurpation of the throne may have given
Thutmose III and Amenhotep II a particular incentive to produce
sons. This conclusion further motivated kings to choose as great royal
wives women from outside the main royal line, as did Thutmose III in
choosing Sitiah and Merytra.
THE 18TH DYNASTY BEFORE THE AMARNA PERIO 24 7
The Legitimization of Thutmose IV
The succession of Thutmose IV appears to have had no recognition at
all by Amenhotep II, either by co-regency or announced intent. On a
statue dedicated in the reign of Amenhotep II by Prince Thutmose
(later Thutmose IV) in the Temple of Mut at Karnak, the tutor accom-
panying the prince, named Hekareshu, was designated simply as
nurse of the royal children; however, after Thutmose's accession,
Hekareshu was retrospectively termed 'god's father' and 'nurse of the
king's eldest son'. Although Merytra may have appeared on Thutmose
Ill's late monuments, Thutmose IV's mother, Tiaa, cannot be cer-
tainly attested on a monument of Amenhotep II's other than as a later
addition by Thutmose himself. There is no evidence before her son's
reign that Tiaa's position influenced the succession.
Royal nurses (male and female), together with tutors from the ranks
of retired courtiers, nurtured and educated royal children during the
18th Dynasty. The burgeoning documentation for princes at this time
is thus probably no accident at all. Competition among the swelling
ranks of capable young princes, particularly with the cessation of
regular military campaigns in Asia after the first decade of Amenhotep
II's reign, is not difficult to imagine. And competition can erupt
unexpectedly into struggle among ambitious youths. The story of
Thutmose IV's elevation to the kingship related by the Giza Sphinx
Stele inscription has been interpreted in the past to suggest that he was
not the legitimate heir, but it need tell us no more than that royal
ideology often drew upon divine legitimization in the New Kingdom.
The sheer romance of the 'Sphinx Stele' is perhaps a good enough
reason to quote part of it here:
Now the statue of the very great Khepri [the Great Sphinx] rested in this place, great
of fame, sacred of respect, the shade of Ra resting on him. Memphis and every city
on its two sides came to him, their arms in adoration to his face, bearing great
offerings for his ka. One of these days it happened that prince Thutmose came
travelling at the time of midday. He rested in the shadow of this great god. [Sleep
and] dream [took possession of him] at the moment the sun was at zenith. Then he
found the majesty of this noble god speaking from his own mouth like a father
speaks to his son, and saying: 'Look at me, observe me, my son Thutmose. I am your
father Horemakhet-Khepri-Ra-Arum. I shall give to you the kingship [upon the land
before the living] [Behold, my condition is like one in illness], all [my limbs being
ruined]. The sand of the desert, upon which I used to be, (now) confronts me; and it
is in order to cause that you do what is in my heart that I have waited.'
The request addressed to Thutmose to excavate the Sphinx from
the sand was answered, and the king's retaining wall around the
248 BETSY M. BRYAN
amphitheatre, as well as a set of stelae set up around the arena, docu-
ment his work in the region. Possibly his construction efforts were
intended to distract attention from problems with the succession. The
suggestion of a struggle for the throne can be seen in several monu-
ments dedicated by Thutmose's brothers at their father Amenhotep
IFs Giza Sphinx temple. They were found broken and mutilated, and
their defacement suggests some sort of damnatio memoriae, but there
is presently no way to demonstrate what provoked it. Prince Weben-
senu is the most likely son of Amenhotep to have been the owner of
defaced Giza stelae A and B. Webensenu's canopic jars and shabtis
were found in Amenhotep IFs tomb (KV 35 in the Valley of the Kings),
but it is difficult to know when they were placed there. We may sup-
pose that this prince was of some importance, but more than this is not
possible. The defaced Giza stelae should thus not be ignored as evi-
dence of a struggle, but we cannot confirm or deny that Thutmose IV
was the usurper.
The Monuments of Thutmose IV
Thutmose IV's reign of at least eight years was brief but active. It is a
commonplace observation that Egyptian rulers built numbers of
monuments in direct proportion to the amount of peace and affluence
they enjoyed. As king, Thutmose IV had the wealth and peace, but
time apparently was cut short. He began construction at most of
Egypt's major temple sites and at four sites in Nubia. The original sizes
of the monuments and of their remains vary greatly, but in general he
added to pre-existing temples. The distribution of Thutmose IV's
monuments, within the context of the mid-i8th Dynasty, is unremark-
able. He honoured the established cult centres and was hardly an
iconoclast. On the other hand, at several locations he left certain
harbingers of things to come. Indeed we may suggest that he delib-
erately followed in the footsteps of his grandfather and father, building
additions to their temples, and in similar fashion suggested new sites
and monuments to his son.
Monuments of the reign have been found at the following places: in
the Delta at Alexandria, Seriakus, and Heliopolis (?); in the Memphite
region at Giza, Abusir, Saqqara, and the city of Memphis itself; in the
Faiyum at Crocodilopolis; in Middle Egypt at Hermopolis and Amarna;
and in Upper Egypt at Abydos (where he left a chapel of brick with
limestone revetments), Dendera, Medamud, Karnak, Luxor, western
Thebes (where he built a mortuary temple and a tomb, KV43, in me
THE 18TH DYNASTY BEFORE THE AMARNA PERIO D 24 9
Valley of the Kings), Armant, Tod, Elkab, Edfu, Elephantine, and
Konosso. In Nubia he left blocks at Faras (?) and Buhen. He decorated
the peristyle court at Amada, began a building at Tabo (later completed
by Amenhotep III), and left a foundation deposit at Gebel Barkal. In
addition, some decoration was carried out in the Hathor temple at the
Serabit el-Khadim turquoise mines in Sinai.
The king's interest in the sun-gods may be documented throughout
his building campaigns and in his inscriptions as well. At Giza, he
devoted himself not to a display of equestrianism and archery, but to
the god Horemakhet and the Heliopolitan cult. He made no reference
to Amun-Ra on the Sphinx Stele, allowing the northern deity
(Horemakhet-Khepri-Ra-Atum) to dominate both as sun-god and as
royal legitimator. Given that Amun, even on Amenhotep IFs Sphinx
Stele, was the primeval creator and the god who determined the king-
ship, Thutmose's omission of Amun from his stele must surely have
been deliberate, perhaps reflecting both the increasing importance of
the Heliopolitan gods and the political influence of the north itself as
the administrative centre of Egypt.
At Karnak, the king shifted the main axis back to east-west, thus
reducing the importance of Amenhotep ITs north-south entrance-
way. Placing a porch and door before the Fourth Pylon, Thutmose IV
probably first left the original court untouched and changed only the
monumental doorway itself He erected a porch for the Fourth Pylon
doorway with columns made of wood (ebony and meru according to an
inscription), probably gilded with electrum. This porch would have
been a protected space used during court rituals, and two contempor-
ary representations of it have been preserved.
A few years later he created a new appearance for the Fourth Pylon
limestone court erected by Thutmose II. Over the earlier limestone
walls, Thutmose IV built a sandstone peristyle court elaborately decor-
ated with reliefs showing treasures donated by the king to the god
Amun. This was to have commemorated the celebration of a first
jubilee planned without waiting for thirty years to elapse, as was
certainly the case with Amenhotep II too. The style of Thutmose's
sculpture from Karnak changed in the last years of rule, becoming
more elaborate and expressive.
The king also erected a single obelisk at the eastern end of the
precinct at Karnak. It had been produced for Thutmose III but lay in
the stone workshop for thirty-five years until Thutmose IV ordered it to
be set up. It became a focus of the solar cult place designed by
Thutmose III, and it was placed directly on the temple axis.
250 BETSY M. BRYAN
Thutmose IV in Syria-Palestine and Nubia
With regard to foreign policy in the east, Thutmose IV's contacts with
Mitanni are best considered in the context of the pre-existing peace
with that power. This situation would have restricted military activity
to campaigns against either upstart Egyptian vassals or Mitanni king-
lets asserting pressure on the Egyptian city states. Thutmose IV took a
daughter of the Mitanni ruler Artatama as wife, in order to seal a
diplomatic relationship with the king.
The best-known inscription noting military activity for Thutmose IV
is the laconic dedication text on a statue at Karnak that consists of a
single line: 'from the plunder of his Majesty from [. . .]na, defeated,
from his first campaign of victory'. The toponym referred to on his
Karnak dedication (and a statue base from Luxor temple) is likely to
have been in Syria, given the several references in the Amarna Letters
to the king in that region. The two most likely cities to restore on the
Karnak dedication would have been Sidon (Zi-du-na), where Thut-
mose IV was known to have travelled and where Egypt clearly lacked
support in the Amarna period; or Qatna, near Tunip in Nukhashshe
(an amorphous area to the east of the Orontes). Whether the toponym
was Qatna or Sidon or some other city, the northern Levant remains
the likely area for the main campaign. This is all the more evident since
the Mitannian king Artatama would have been impressed by a show of
strength at his doorstep, particularly if negotiations for a diplomatic
renewal were in progress.
A scene in the tomb of the standard-bearer Nebamun (TT 90)
records the man's promotion in year 6 and shows the Chiefs of Nahrin
before the king in his kiosk. Captives also appear in this scene and are
rare enough after the reign of Amenhotep II that they should be taken
seriously. However, as captives taken in a campaign against both Mit-
anni vassals and rebellious Egyptian city states, these foreigners make
the statement of Egypt's obvious superiority over Mitanni. Such an
assertion of dominance would have been appropriate at the moment of
Egypt's treaty renewal with Washshukanni. It may be that, rather than
help us to document a war against the Mitanni ruler, this scene
informs us of the date for Thutmose IV's diplomatic marriage with the
Syrian princess.
In the southern regions of Palestine, Thutmose can only be said to
have taken punitive action against Gezer; actual warfare cannot be
proven, but some of the population of this town were transported to
Thebes. It is presently impossible to prove that the Levantine holdings
THE 18TH DYNASTY BEFORE THE AMARNA PERIOD 25 1
of Egypt at the end of Thutmose's reign were not similar to those of
Amenhotep II. And it is similarly impossible to demonstrate that
Artatama I could have been dealing from a position of strength when
he decided to form a brotherhood with Thutrnose IV. Thutrnose never
fought the Mitanni ruler directly, but his power in the far northern
provinces was intact. Thus Artatama may have been renewing a dip-
lomatic relationship established under Amenhotep II, or he may have
been reaching an accord to achieve stability for the region as a whole
(particularly as the threat of a united Assyria and Babylon may already
have been looming). The Egyptians were hardly disgraced in this
peace—they appear to have given up nothing.
Turning to the areas south of Egypt, there is no clear attestation of
Thutmose IV's military activity in Nubia proper. The Konosso Stele,
carved on the rock south of Aswan, details a journey by Thutmose IV
over the gold-mine routes east of Edfu; it is very likely that the Nubians
were interfering with gold transports, attacking from hiding places in
the high desert where the mines themselves were located. Since the
expedition terminated at Konosso, it is possible that the king used the
Wadi el-Hudi to return, having taken an elliptical route eastwards
through the Wadi Mia, then south, then westwards back to the Nile
Valley. There is, however, little in the text to imply any major warfare
against these Nubians. Rather, this was a desert police action that
merited attention because of a threat to transportation through the
desert.
Kingship and Royal Women in the Reign of Thutmose IV
Thutmose IV may have begun a course that Amenhotep III completed,
particularly in deliberately identifying himself with the sun-god. At
Giza, on one stele he was shown wearing the gold sfoe/nw-collar and
armlets strongly associated with the solar deity's favour. These jewels
are often shown on representations of the king in funerary contexts,
but on this stele (as well as on an ivory armlet from Amarna, and on the
king's chariot) Thutmose IV is shown wearing them as a living ruler.
Thutmose IV left a statue of himself as falcon king at Karnak (now in
the Cairo Museum), and on a relief from his sandstone court at Karnak
a statue of the king as falcon was pictured among other royal statuary.
In these images the divine and solar aspects of the kingship are
supreme.
The trend of elevating the royal associations with Egypt's major gods
(as seen in Thutmose Ill's veneration of his own and earlier kingships
252 BETSY M. BRYAN
in his jubilee temple within the precinct of Amun) became even more
prominent during Thutmose IV's reign. While never abandoning the
notion that the dynastic line was best strengthened by marriage of the
king to a king's daughter (for both political and economic reasons),
Thutmose IV, like Amenhotep II, increasingly emphasized divine
associations of royal females. He placed his mother in the role of'god's
wife of Amun', as if she were the goddess Mut herself. This was her
primary role, although Tiaa also held the titles of 'king's mother' and
'great royal wife' during most of Thutmose IV's reign. Monuments
with her name are known from Giza, the Faiyum, Luxor, Karnak, and
the Valley of the Kings. This intentional association with the mother-
goddess Mut was supplemented by iconographic and inscriptional
connections between Tiaa and the goddesses Isis and Hathor. The
king appears to have apportioned the ceremonial roles of priestess and
queen among Tiaa and two other great royal wives. Tiaa appears in the
Karnak jubilee court of her son, where she holds a mace while witness-
ing the monument's foundation ceremony. In Amenhotep II's jubilee
pavilion Merytra (name later changed to Tiaa) was shown likewise
holding a mace and a sistrum in her other hand. The imagery here
probably signifies these queens' status as 'god's wives of Amun'. The
mace became a standard iconographic element of the 'god's wives'
later on.
A non-royal wife Nefertiry, attested in Giza and Luxor temple, was
'great royal wife' alongside Tiaa during the earlier years of rule, and
Thutmose capitalized on this mother-son-wife triad (as did
Amenhotep III later) to portray roles—for example, at Luxor temple—
where he, as both god and king, accompanied his mother and wife
goddesses enacting the roles of mother, wife, and sister-goddesses.
Later, after Nefertiry had apparently either died or been set aside, he
followed the trend of his family and married a sister, whose name may
be read as laret. It is possible that he may have had to wait for laret to
reach a marriageable age. Amenhotep Ill's mother, Mutemwiya, was
never acknowledged by Thutmose IV, either as major or minor queen,
but a statue of Amenhotep's court counsellor, the treasurer Sobekho-
tep (buried in TT 63), shows the Prince Amenhotep in a favoured
position before his father's death. The tomb of Amenhotep's royal
nurse, Hekarnehhe (TT 64) also shows the young heir, but, since the
tomb was completed in Thutmose IV's reign, Mutemwiya does not
appear. Several other princes are mentioned in texts in Hekarnehhe's
tomb, as well as in a rock graffito at Konosso, but it is not clear whether
these are sons of Amenhotep II or Thutmose IV.
THE 18TH DYNASTY BEFORE THE AMARNA PERIOD 2 53
Amenhotep III
The thirty-eight-year reign of Amenhotep III was primarily a period of
peace and affluence. The construction of royal monuments during the
reign was on a scale with few parallels, and the retinue of the king left
tombs, statues, and shrines that rivalled those of many former rulers.
Sadly, as in most periods, it is impossible to compare the fortunes of
the rich with those of the poor. Whether the peasant's life was eco-
nomically improved due to the overall wealth in Egypt is unknown.
The official documentation might suggest that the population as a
whole enjoyed prosperity at some point, since Amenhotep III and his
granary official Khaemhet boasted of the 'bumper' crop of grain
harvested in the king's crucial jubilee year 30. The king was remem-
bered even 1,000 years later as a fertility god, associated with agri-
cultural bounty. Still, this type of evidence is hardly unbiased, so we
must admit our ignorance.
It is probable that Amenhotep III was a child at his accession. A
statue of the treasurer Sobekhotep holding a prince Amenhotep-mer-
khepesh probably shows the king shortly before his father's death, and
a wall painting in the tomb of the royal nurse Hekarnehhe (TT 64)
describes the tomb-owner as the royal nurse of Prince Amenhotep,
portraying the prince as a youth rather than a small naked child. The
age of the king at accession could have been anywhere between 2 and
12, with a later age perhaps to be preferred given that Amenhotep's
mother, Mutemwiya, was barely more visible than Tiaa and Merytra,
the preceding two kings' mothers. A regency by Mutemwiya appears
unlikely, and, if the king was indeed a small child at accession, his rule
was conducted for him quite unobtrusively. An alternative possibility
might be that members of Queen Tiye's family assisted the king in his
early rule. A scarab dated in year 2 of Amenhotep's reign established
the early date of his marriage to Tiye, and the identification on another
scarab of the queen's parents, Yuya and Tuya, underscores their prom-
inence. There is, at present, no documentary evidence that Tiye's family
acted as a power behind the throne. This presumption has become so
strong, however, that other non-royal 'king-makers', such as Ay
(whose name in Egyptian resembles that of Yuya), have been thought
to be from the same Akhmim family. The discovery of colossal statuary
of the late i8th Dynasty at Akhmim, along with some of Amenhotep
III, appears to give support to this idea, in so far as that geographic
region benefited during the reigns of Amenhotep III and Tutankha-
mun/Ay.
254 BETSY M. BRYAN
The Divinity of Amenhotep III
Recent discussions of the reign of Amenhotep III have suggested that
he was deified during his lifetime, not only in Nubia, where he built a
cult temple for himself, but also in Egypt proper. Raymond Johnson
has argued that Amenhotep Ill's insistent identification with the sun-
god in his monumental iconography and inscriptions should be
understood as his deification, and he further contends that Amen-
hotep IV/Akhenaten (1352-1336 BC) transformed his deified father
into the disembodied solar disc Aten, thereby worshipping the living
Amenhotep III as the sole god of the world. The view that Amenhotep
IV worshipped his father as the Aten (albeit after his death) was earlier
espoused by Donald Redford. It must be observed that, at the same
time, such a transmogrification would have deprived the father of both
his physical existence and his name, and it would also have forced
Amenhotep III to participate in the ruination of the god celebrated in
his own name, Amun. Although the interpretation of Amenhotep III
as his son's god carries within it the unmistakable influence of modern
Freudian psychology, Egyptian notions of the king's relationship to the
gods might support the basis of this idea.
While there is at present no text or iconography within Egypt proper
that identifies Amenhotep III as a cult deity during his lifetime, all
kings (whom Jaromir Malek describes in Chapter 5 as netjeru neferu,
'junior gods') were considered to be major gods at their decease and
were frequently invoked as intercessors by their successors and by
private persons as well. Moreover, it is arguable that Amenhotep III
intended to be identified with the sun-god from the time of his first
jubilee in years 30-31, since scenes representing that festival show
him taking the specific role of Ra riding in his solar boat. The degree to
which Amenhotep III was associatt d with the sun-god on monuments
might well have encouraged the view that, having merged with the
sun, as the king was expected to do after death, he was present in
Akhenaten's deity, the solar disc Aten. To claim that this was
Akhenaten's intention remains a psychologically informed specula-
tion.
It is also noteworthy that Amenhotep III named his own palace
complex 'the gleaming Aten' and used stamp seals for commodities
that may be read 'Nebmaatra [his prenomen] is the gleaming Aten'. Of
course, sealings are economic documents and could as such refer to
the palace complex itself; they might, therefore, have been intended to
be read as 'the gleaming Aten of Nebmaatra'. What is certain is that the
THE 18TH DYNASTY BEFORE THE AMARNA PERIOD 2 55
association of the Aten with Amenhotep III was well established in his
own documentation prior to the reign of Amenhotep IV/Akhenaten.
It is impossible at this point to prove or disprove Johnson's argu-
ment. There are no stelae or statues that were, with certainty, dedicated
to Amenhotep III as a major deity within Egypt in his lifetime—much
less as the Aten. The deification of Rameses II, some 100 years later,
was accompanied by significant numbers of monuments, both royal
and private, that identified the god Rameses in a number of cult loca-
tions within Egypt proper. These monuments date from the reign of
Rameses himself and do not refer to the king as 'beloved of X-deity' (as
the numerous monuments of Amenhotep III do). They name Rameses
himself as the god and show him being offered to, usually as a statue.
Nothing of this type exists for Amenhotep III in Egypt, and the exam-
ples that most closely parallel monuments offered to gods cannot be
safely assigned to the king's lifetime. One stele from Amarna shows
Amenhotep and Tiye receiving food offerings under the bathing rays
of the Aten. While this might be seen to contradict Johnson's thesis
that Amenhotep III was the Aten, it is perhaps significant that it derives
from the late years of Akhenaten's reign. It therefore raises the
question as to whether the king and queen were still alive, or whether
the stele, from a private house owner's shrine, venerated the deceased
royal couple to invite their intercession. Such votive stelae offered to
deceased kings were common in houses at Deir el-Medina both earlier
and later than the Amarna Period.
A major obstacle is our inability to ascertain whether Amenhotep III
and his son Amenhotep IV/Akhenaten ruled as co-regents for an
appreciable length of time. Were this proposition (supported by John-
son's thesis) to be demonstrated, then objects venerating Amenhotep
III and made in Akhenaten's reign could be seen as worship of him as
a living deity, but not necessarily as the Aten. Co-regency was rare
enough in ancient Egypt that scholars remain uncertain as to whether
it had consistent hallmarks (see Chapters i, 7, and 10). After years of
debate, we are no closer to a resolution of the debate about co-regency
or about the deification of Amenhotep III as the Aten. It might not
be unfair to suggest, however, that Amenhotep III would have been
pleased that, 3,350 years after his death, it is difficult to ascertain
whether he ruled as a living god or merely strived to give that impres-
sion.
256 BETSY M. BRYAN
The Building Programme of Amenhotep III
It may be fair to describe the numerous constructions of Thutmose III
as a building programme, in that he developed and expanded cults at
a number of sites, including Amada (for Amun and Ra-Horakhty),
Karnak (the East Temple for the sun-god and his own festival build-
ing), and Hermopolis. More importantly, however, at Karnak his impact
was thematic and left the dramatic impression of the warrior pharaoh
whose victories simultaneously honoured the king himself and the
god Amun. The geographic regions that he conquered appear there in
eternal captivity to the god, and the king proudly claimed Amun's
favour when he built his festival temple known as 'Effective of Monu-
ments', a cult place that overshadowed those of his royal predecessors
at Karnak. Thutmose Ill's divinity as he designed it for eternity
described him as the 'best among equals', referring to the earlier kings
of Egypt. This divinity gained him entrance to the council of supreme
deities such that he shared the solar boat with Ra and was introduced
before Amun.
Amenhotep Ill's building programme gave him space to design an
eternal divinity for himself that reached beyond Thutmose Ill's vision.
He consistently identified himself with the national deities, not his
deceased royal predecessors, and he represented himself as the sub-
stitute for major gods in a few instances. In addition, his buildings
document an unparalleled emphasis on solar theology, such that the
cults of Nekhbet, Amun, Thoth, and Horus-khenty-khety, for example,
were heavily solarized during Amenhotep Ill's reign. Trends apparent
in 18th-Dynasty funerary literature reveal that the sun's cyclicity and its
potential for fertility or famine were manifest in the world and in the
ruler, but monuments and objects made in Amenhotep Ill's time may
have disseminated these notions more widely. It is impossible to ascer-
tain whether the intellectuals of the age influenced the royal iconog-
raphy or were requested to formulate it.
Amenhotep built temples or shrines in Nubia at Quban, Wadi es-
Sebua, Sedeinga, Soleb, and Tabo Island. There are building elements
or stelae in his name at Amada, Aniba, Buhen, Mirgissa, and Gebel
Barkal (perhaps reused in the latter). There are statues or scarabs in his
name at a variety of sites, including Gebel Barkal and Kawa, and most
of the statues originated at other sites, particularly at Soleb. In Egypt
proper the king built a shrine at Elephantine (now destroyed) and com-
pleted a chapel at Elkab, probably partially erected by his father. Some
20 km. south of Thebes Amenhotep III built a temple at Sumenu, site
1. (right) In this 'king-list', on
a wall in the temple of Sety I
at Abydos, ^.1300 BC, Sety
and the young prince
Rameses (the future Rameses
II) bring offerings to the list
of names of kings written out
in a continuous sequence
from the ist to the i9th
Dynasty. Certain kings'
names (and sometimes whole
dynasties) were omitted from
the list when the priests at
Abydos regarded them as ille-
gitimate.
2. (below) This body of a child,
excavated in 1994 at Taramsa-
i, near the temple of Hathor,
at Dendera, is the earliest
Egyptian so far identified, dat-
ing to c.55,ooo BP.
3. One of the most famous Dynasty o artefacts, the Narmer Palette, was excavated
from the so-called Main Deposit at Hierakonpolis, £.3000 BC. This side shows King
Narmer wearing the White Crown and smiting a captive foreigner held by the hawk-
god Horus.
4. The late 5th Dynasty mastaba-tomb of Ptahhotep at Saqqara includes a depiction of
the inspection of cattle, perhaps for taxation purposes.
5- (top) This rectangular nth Dynasty slab-stele was probably originally erected in the
tomb of Wahankh Intef II at western Thebes. The scene shows the king offering beer
and milk to Ra and Hathor, and the accompanying text contains a long hymn that the
king addresses to both deities. The relief work on this stele represents a superb exam-
ple of the emerging 'court' style of the early nth Dynasty.
(bottom) Slab stele of Djary (a military commander in Intef II's service) from his saff-
tomb in the necropolis of el-Tarif. It is an excellent example of the bold—even bizarre
—style of provincial art at this date. This is evident not only in the representation of
Djary and his wife seated on a bench to receive funerary offerings, but also in the
peculiar shapes and unusual arrangements of many of the hieroglyphs, betraying the
considerable distance that separates this piece from Old Kingdom artistic conventions.
6. (top) Many Middle Kingdom tombs contained funerary models depicting scenes
from daily life. This model, from the tomb of Meketra at Thebes, portrays the cattle
census, a regular event, whereby the necessary amount of royal tax on livestock would
have been calculated.
7. (below) An example of a Tell el-Yahudiya ware juglet of a type which occurs in level
E/i-b/i at Tell el-Dab c a during the 1501 Dynasty.
8. (top) the palace of the mortuary temple of Rameses III at Medinet Habu, c.nSo BC,
contained these faience tiles decorated with detailed figures of foreigners.
9. (bottom left) By the Ramessid period, the Egyptian army had begun to incorporate
many mercenaries from the eastern Mediterranean whose distinctive physical features
and weapons were faithfully reproduced by the artists of the time, as in this detail of a
Sherden soldier in the depiction of the Battle of Qadesh on the external walls of the
temple of Rameses II at Abydos.
10. (bottom right) The upper part of the triumphal stele of the Kushite ruler Piy from
Gebel Barkal shows the provincial rulers of Egypt standing and prostrating themselves
before Piy (whose figure was later erased).
11-13. (top left) An avenue of 3Oth Dynasty sphinxes of Nectanebo I stretching north-
wards from the pylon of the Luxor temple towards the temple of Amun at Karnak.
3oth Dynasty activity in this temple-complex asserted continuity with the long tradi-
tion of architectural work on the site and also with the many great kings in whose
names these structures were erected.
(top right) Detail of a cartouche of Nectanebo I in the oldest standing section of the
temple of Isis at Philae. The prenomen (first cartouche) reading Hpf-kj-r is identical
to that of Senwosret I, one of the greatest pharoahs of the i2th Dynasty, doubtless
intentionally. The earliest cult buildings at Philae were 26th Dynasty, but Nectanebo
I's architectural work greatly enhanced the cult site, and was clearly intended to pro-
mote the shrine as a major centre of Isiac worship.
(bottom) The inner coffin of Petosiris from the burial chamber of his tomb at Tuna el-
Gebel, probably dating to the second Persian Period. Almost 2m. long and made of
highly valuable pine inlaid with well-formed coloured glass hieroglyphs, this piece is
unequivocally Egyptian in character, unlike some of the decoration in the tomb
chapel. The text consists of a version of Ch. 42 of the Book of the Dead. The language,
like that of the Book of the Dead as a whole, is the long obsolete classical Egyptian
which would only have been intelligible to the learned at this period.
14- (left) Mummy portrait of a young lady from Hawara, wood, encaustic, and paint.
Roman period, 2nd century AD.
(rigto)Mummy of a young boy with inserted portrait in encaustic on wood, Roman
Period, early 2nd century AD.
THE 18TH DYNASTY BEFORE THE AMARNA PERIOD  257
of a cult to the crocodile Sobek. Although the temple itself remains
elusive, numerous objects from it and the cemetery associated with its
town, have come to light since the 19605.
It is in Thebes that Amenhotep's penchant for the colossal is most
visible today. The Colossi of Memnon were the towering quartzite
images of Amenhotep that protected the king's first pylon at his
funerary temple (the single largest royal temple known from ancient
Egypt). More fragments of colossal sculpture have been found within
his mortuary temple than in any other known sacred precinct. Build-
ings on the east bank of the Nile at Thebes included a series of con-
structions at Karnak, as well as Luxor temple, which was entirely
rebuilt.
Amenhotep's tomb, KV 22, was excavated in a western valley wadi,
away from earlier royal tomb locations. Excavations during the 19905
by a Japanese team have carefully mapped this remarkably large and
beautifully finished tomb. The body of Amenhotep III himself (or a
mummy so labelled) was found in the tomb of Amenhotep II (KV 35).
On the west bank of Thebes, south of the king's enormous funerary
temple, was located his enormous palace of'the gleaming Aten', now
termed Malkata after the Arabic designation for the Queen's Valley
nearby. Still further south, at Kom el-Samak, the king built a jubilee
pavilion of painted mud brick. A Japanese expedition excavated and
carefully recorded this building, which is now destroyed. Next to the
Malkata complex is the great harbour that Amenhotep created for use
during his constructions and habitation at the palace. In the early
19705 the Birket Habu harbour was the subject of an investigation by
David O'Connor and Barry Kemp, who also studied the Malkata palace.
A Japanese expedition worked at the palace in the 19805.
Amenhotep was particularly active in Middle Egypt, although little
remains of his temple works at Hebenu and Hermopolis. To the north,
blocks of brown quartzite with relief decoration remain from the
king's great temple in Memphis, 'Nebmaatra United with Ptah'. Colos-
sal quartzite statues of Ptah, reworked by Rameses II, now stand in the
foyer of the Egyptian Museum, Cairo, but probably derived from the
Memphite temple of Amenhotep III. In the 19905 the Egypt Explora-
tion Society with W. Raymond Johnson have investigated limestone
blocks of a small temple of Amenhotep reused by Rameses II. The
king's interest in Memphis is further attested by his association with
the first known Apis-bull burial in the Serapeum through the agency
of his son Thutmose, the high priest of Ptah. Building elements at
Bubastis, Athribis, Letopolis, and Heliopolis attest to the king's interest
258 BETSY M. BRYAN
in the eastern Delta. At Athribis a temple was constructed under the
supervision of the king's confidant Amenhotep, son of Hapu.
The work of Amenhotep III at Karnak, Luxor, and his funerary
temple reveals his interest in stressing the royal identification with the
sun-god. After completing the monuments of his father, Thutmose IV,
he changed the face of the Karnak temple. At some undetermined
point in his reign, Amenhotep Ill's workers dismantled the peristyle
court in front of the Fourth Pylon and the shrines associated with it,
using them as fill for a new pylon, the Third, on the east-west axis.
This created a new entrance way to the temple, and two rows of
columns with open papyrus capitals were erected down the centre of
the newly formed forecourt. He also began the construction of the
Tenth Pylon at the south end of Karnak, changing its orientation
slightly from that of the Seventh and Eighth in order that it led to a new
entrance for the precinct of the goddess Mut, for whom he may also
have built or begun a temple. Balancing the south-temple complex was
a new building to the north of central Karnak, which was a shrine to
the goddess Maat, the daughter of the sun-god. Both Mut and Maat
could represent the solar eye of Ra, his agent in the world. David
O'Connor has noted that the north-south opposition corresponds to
heavenly and terrestrial settings, a fact that accords well with the divine
roles of Maat and Mut respectively. The rituals and offerings that
Amenhotep III provided may have been designed to demonstrate
architecturally and inscriptionally his ability, like the sun-god, to create
stability in the cosmos. Deeply carved reliefs from a granary within
Karnak show the king in elaborate regalia, crowned with multiple solar
discs, and bejewelled on his kilt apron and body with solar imagery. In
addition, the king's face is childlike, and his body type is thicker and
shorter waisted than on most of the temple reliefs. This is a rejuven-
ated Amenhotep III, who also exhibits the jubilee iconography with
elaborated divine, and particularly solar, elements.
The construction of Luxor temple by Amenhotep III may have been
carried out in several stages. He replaced an earlier Thutmosid build-
ing with a sandstone temple that celebrated the renewal of the divine
kingship during the Opet feast, added into it a birth room wherein he
was born of the union of Amun-Ra and his real mother, Mutemwiya,
and completed the temple with a new cult place for Amun of Ipet resy,
or Luxor.
The royal penchant for ritual drama was further monumentalized in
Amenhotep Ill's funerary temple. The temple contained large num-
bers of life-sized and colossal statuary in the form of both well-known
THE 18TH DYNASTY BEFORE THE AMARNA PERIOD 259
and obscure deities, frequently with human bodies topped by animal
heads. These statues represented both the gods of the jubilee and a
three-dimensional astronomical calendar to guarantee a propitious
festival year. A litany to satisfy Sekhmet, the solar eye of Ra, began the
rituals in Thebes, and it was followed in the king's temple in the
Sudan, at Soleb, with the ritual propitiation of the deified Nebmaatra,
the lunar eye of Ra. After this sequence, the jubilee began in earnest.
Queen Tiye
Tiye was the most influential woman of the king's reign, and she sur-
vived her husband by at least a few years. She was so important to him
that she not only appears with him on temple walls at Soleb and west
Thebes, accompanying him at the jubilee festivities, but she was
deified in her own temple at Sedeinga in Upper Nubia and became
part of the royal solar programme. As the solar eye of Ra in the Sudan,
she would have joined the deity Nebmaatra to return to Egypt and
restore order ('Maat') to the world. The role she did not play was that of
god's wife of Amun, and it is this fact that accounts for her scarcity on
the monuments from Karnak and Luxor. She is known only from a
small shrine at Karnak later usurped for Tutankhamun—not at all at
Luxor.
After her husband's death, the king of Mitanni, Tushratta, wrote to
Tiye asking her to remind her son Amenhotep IV/Akhenaten of the
close relationship between him and Amenhotep III. Perhaps upon her
own death she was first entombed at Amarna, then moved to either (or
both) KV 22 or 55. Tiye gave birth to Satamun, Henuttaneb, Nebetiah,
and I sis, all of whom appear on statues and smaller objects associated
with the royal couple. Satamun was the most elevated of Tiye's
daughters, and chairs made for her were found in the tomb of Yuya
and Tuya (KV 46). She bore the title of'great royal wife' simultaneously
with Tiye, while the other daughters were called 'king's wife' or 'king's
consort'. The economic and, particularly under Amenhotep III, reli-
gious significance of the king's marriage to his own daughters has
been discussed a number of times already in this chapter and dates
back to the beginning of the dynasty. In pairing his wife and
daughter (s) with himself on monuments, Amenhotep encouraged the
image of the sun-god accompanied by the mother goddess (Nekhbet,
Nut, Isis) and the daughters of Ra (Hathor, Maat, Tefnut). More
practically, the king enlarged his own holdings, not by giving his
daughters to non-royal men to marry, but by himself marrying into
260 BETSY M. BRYAN
wealth. He asked for and received a Babylonian princess as wife, and
he married two Mitannian princesses (one of the latter, Taduhepa,
having reached Egypt only just in time to become a widow and then
marry Amenhotep IV).
Male offspring of Amenhotep III and Tiye certainly included
Amenhotep IV. The mother of a king's son and sera-priest Thutmose,
who may have been older than Amenhotep, is unknown. Whether the
king had offspring by his foreign wives is unknown, but there are a
number of court women, princes, and princesses known by name
from funerary objects unearthed near Malkata. Some of these may
have been royal family members, others minor wives.
The body of a royal woman was found in the cache of mummies in
the tomb of Amenhotep II (KV 35). She has been identified as Queen
Tiye on the basis of hair samples matched to strands of the queen's
hair carefully boxed in Tutankhamun's tomb. The certainty of this
identification is in question, and confusion persists, given that objects
in the name of Tiye were found both in KV 22 and in the enigmatic
KV 55. The Japanese expedition at KV 22 has found elements of a
coffin that could belong to a queen, but whether that would be Tiye or
Satamun, the daughter whom Amenhotep III took as great royal wife
during his reign, is unknown.
International Relations in the Reign of Amenhotep III
A Nubian campaign took place in year 5 of Amenhotep Ill's reign and
was commemorated on the Island of Sai, as well as at Konosso and
along the road south of Aswan. The viceroy of Kush may have super-
vised the military action, but whether this was Merymose or the earlier
office-holder Amenhotep is unknown. Merymose left his own inscrip-
tion at Semna, describing an action against Ibhet (probably Lower
Nubia). The year 5 campaign was in Kush, perhaps even to the south of
the fifth cataract. The building of the fortress of Khaemmaat at Soleb,
where the king also constructed a temple, may have been intended to
prevent further disruptions from Upper Nubia. The earlier Upper
Nubian capital at Kerma was almost directly across the river from
Soleb, so the site may have been chosen to underscore Kushite sub-
jection to Egypt.
International relations with the rest of the ancient world were con-
ducted through diplomatic missions. The amount of Egyptian material
on the Greek mainland increased dramatically in the reign of Amen-
hotep III, and the names of Aegean cities, including Mycenae,
THE 18TH DYNASTY BEFORE THE AMARNA PERIO D2 61
Phaistos, and Knossos, appear for the first time in hieroglyphic writing
on statue bases from the king's funerary temple. Letters between
Amenhotep III and several of his peers in Babylon, Mitanni, and
Arzawa are preserved in cuneiform writing on clay tablets. These
letters, many found in the archive of Akhenaten's capital of Amarna,
demonstrate the powerful position enjoyed by Amenhotep III as he
negotiated to marry the daughters of other rulers. A strong connection
between Amenhotep III andthe Mitanni king Tushratta is apparent in
the letters, while the Babylonian king Burnaburiash, who came to
power late in Amenhotep's rule, appears more suspicious of Egyptian
strength. The mid-i4th century BC certainly represents one of the high
points of Egypt's influence on the ancient world, and it was the
culmination of activities by nearly all the rulers of the i8th Dynasty.
Administration in the i8th Dynasty
The overall administrative structures in use during the i8th Dynasty
are characterized both by clear trends and by some inconclusive situa-
tions. Too few of the officials of Ahmose and Amenhotep I have been
securely identified to indicate the families and regions represented in
the early 18th-Dynasty royal retinue. By the middle of the dynasty,
however, the kings' closest associates were buried either in Thebes or
at Saqqara, with more of our documentation deriving from the
southern city. From the reign of Hatshepsut onwards, the elite officials
for whom we may expect to find a decorated tomb chapel and burial
shaft at Thebes or Saqqara included the vizier, the treasurer (literally
the overseer of the seal), overseers of gold and silver houses, royal
stewards, overseers of the granary (of Egypt or Amun), the king's son
and overseer of southern countries, royal heralds or butlers (often
involved in diplomacy), royal nurses (male and female), regional
mayors (sometimes buried in their home districts), the high priest of
Amun (Thebes), the high priest of Ptah (Saqqara), the second, third,
and fourth priests of Amun, and overseers of the army, as well as
various levels of royal scribes.
The 18th-Dynasty pharaohs' need to garner support from powerful
elite families has been mentioned with respect to scenes of the
enthroned ruler in private tombs of the reign of Hatshepsut and
Thutmose III, and powerful families held the positions of vizier and
high priest of Amun during the reigns of Hatshepsut and Thutmose
III. Important members of Thutmose Ill's retinue, including the
vizier User (TT 61 and TT131), his steward and the counter of grain for
262 BETSY M. BRYAN
Amun, Amenemhat (TT 82), and the overseer of the granary of Amun,
Minnakht (TT 87), had burial chambers with similar versions of the
Litany ofRa and the Amduat. Erik Hornung's recent study of User's
texts has underscored the royal prerogatives assumed by elite individ-
uals in the time of Hatshepsut and Thutmose III. One of the two
tombs of Senenmut (TT 71 and TT 373) was designed to emulate a royal
burial, including an astronomical ceiling such as those later used in
the Valley of the Kings. Privileged access to the king arose in other
ways as well (for example, through burials granted in the Valley of the
Kings). This was true for the reigns of Thutmose III and Amenhotep
II.
In contrast to the elite families well known in the time of his aunt
and father, many of Amenhotep II's close associates had earlier served
in the military both under Thutmose III and under Amenhotep him-
self. Such close relations as army service can foster were perhaps made
all the stronger by their origins in youth, when the king and his court
associates learned to hunt and drive chariots. Usersatet, the Viceroy
of southern countries', may well have been one of these childhood
friends who then served as a royal herald abroad under Thutmose III.
The inscription on a stele which he left at the fortress of Semna in the
second-cataract region contains within it the text of a remarkable letter
sent by Amenhotep II to his old friend posted abroad: Tou sit... a
chariot-soldier who fights for his Majesty . . . the [possessor of a
wojman from Babylon, and a servant from Byblos, of a young maiden
from Alalakh and an old lady from Arapkha.' Another man who had
served Thutmose III, Amenemheb (TT 85), must have died rather
early in Amenhotep II's reign. In an inscription from his tomb,
Amenemheb described the appointment of Amenhotep as king and
then related how the king spoke to him: 'I knew your character when I
was (still) in the nest, when you were in the retinue of my father. May
you watch over the elite troops of the king.'
A courtier who perhaps best typifies the whole of Amenhotep II's
rule was a friend from the military campaigns and childhood play. The
great steward Kenamun fought together with Amenhotep in Retenu.
When recognized for his service, Kenamun was appointed as steward
of Peru-nefer, the seat of a large naval dockyard and ship-building
centre. A royal residence was also active there in the mid-i8th Dynasty.
Later in his life Kenamun's sinecure included the profitable steward-
ship of the king's own household. Kenamun appears to have been
active for almost the whole of Amenhotep II's reign. His tomb (TT 93)
shows elegant stylistic elements known only from tombs painted late
THE 18TH DYNASTY BEFORE THE AMARNA PERIOD 263
in this three-decade period, but there is no hint that Kenamun survived
into Thutmose IV's rule. The decidedly non-military character of
Kenamun's chosen tomb-painting themes, coupled with images of the
prosperous elite lifestyle, are in harmony with the tone set by tomb
paintings contemporary with both Thutmose IV and Amenhotep III.
Two other men were greatly advanced in the time of Amenhotep II,
probably because of early court acquaintance. The vizier Amenemopet
and his brother the mayor of Thebes, Sennefer, became extremely
affluent owing to the king's attentions. These two men were so influ-
ential in the Theban region that they were both afforded burial in the
Valley of the Kings, and Sennefer's wife Sentnay, a royal nurse, was
interred there as well. Both men also had large tomb chapels at Sheikh
Abd-el-Qurna on the Theban west bank (TT 29 in the case of
Amenemopet); indeed Sennefer had two tombs (TT 96 upper and
lower) in order to accommodate several different female contempor-
aries, probably including both wives and sisters. The elder daughter of
Sennefer, Muttuy, shown on statuary and in the lower part of tomb
TT 96, appears to have married a man called Kenamun who succeeded
Sennefer as mayor of Thebes. This couple, Muttuy and Kenamun,
were contemporaries of Amenhotep III and were interred in tomb
TTi62.
Thutmose IV's approach to the administration was to allow the
military offices to shrink, replacing them with bureaucrats, often
selected from long-established elite families. However, every king had
his favourites, and Thutmose IV's was the steward Tjenuna (TT 76).
Tjenuna's fragmentary tomb biography suggests he had a personal
relationship with Thutmose IV that resembled that of a son to a father:
he called himself 'true foster child of the king, beloved of him'.
Although there is not sufficient documentation to support the notion
that Tjenuna was as powerful as either Senenmut or Kenamun,
Thutmose IV may well have trusted his chief steward (who was also
steward for Amun) as much as any other single individual. An official
called Horemheb must also have been a powerful and close ally, to
judge both from the size of his burial (TT 78) and from the fact that
it contained a depiction linking him with one of Thutmose IV's
daughters, Amenemopet.
The civil officials often represented traditional families of influence.
Hepu was vizier in the south during Thutmose IV's reign, and a
Ptahhotep administered the north. That the two viziers existed simul-
taneously is confirmed by the Munich papyrus dated to Thutmose's
reign in which both men called 'vizier' appear as judges. Hepu's tomb
264 BETSY M. BRYAN
(TT 66) is situated in the prestigious cemetery of Sheikh Abd el-
Qurna, a placement that conforms to that of viziers under Thutmose
III and Amenhotep II. Although it is the most deeply placed tomb of
the reign, it is rather small and comparatively unimpressive when
viewed beside others of the period (for example, TT 76 and TT 63).
Clearly the royal administration prospered during Thutmose IV's
rule, court and bureaucratic connections supplanting military ones
almost entirely. The rank of 'general' or 'military officer' is practically
unknown in the period, while that of'royal scribe' abounds, such that
even the viceroy of Nubia, Amenhotep, came from a 'paper-pusher's'
background. The office of'scribe of recruits' was never so well attested,
but the fact that the holders were often clearly court associates suggests
the position required not the hardened military man but the loyal civil
official. With the exception of the Konosso 'police action' (see above, in
the section headed 'Thutmose IV in Syria-Palestine and Nubia'), even
the employment to which the levied 'recruits' were put in this period
and later remains a mystery. It would not surprise us to find that they
were as common in quarry expeditions and building enterprises as in
military manoeuvres.
The court of Amenhotep III is unusual in being known to us nearly
as much from monuments outside Thebes as from those within it. The
king's treasurers, Sobekmose and his son Sobekhotep (Panehsy), do
not have Theban tombs, but the former was buried in Rizeikat. Tombs
of the reign, including one of a vizier, Aper-el, have been discovered at
North Saqqara by Alain Zivie, and numerous stelae found in the i9th
century at that same site name people from the reign. The king's best-
known associates, however, did reside in or leave tombs in Thebes. His
viziers Ramose (TT 55) and Amenhotep both built extravagant chapels
of carved limestone in Thebes, but the latter's is destroyed. This
family, though associated heavily by titles with the Memphite region,
may, as William Murnane notes, have in fact been Theban. The chief
of the king's granary, Khaemhet, likewise left a relief carved tomb at
Thebes (TT 47), as did Queen Tiye's steward, Kheruef (TT 192). The
most beloved courtier of all was Amenhotep, son of Hapu, to whom
the king granted the privilege of his own funerary temple, overlooking
the funerary temple of Amenhotep III himself. Amenhotep, son of
Hapu, a military scribe from a Delta family, oversaw the completion of
many of Amenhotep Ill's most challenging monuments; the king's
recognition of his service led to his eventual deification in the first
millennium BC.
10
The Amarna Period and the Later
New Kingdom (c.i352-io69 EC)
JACOBUS VAN D I J K
When Amenhotep III died, he left behind a country that was wealthier
and more powerful than it had ever been before. The treaty with Mit-
anni concluded by his father had brought peace and stability, which
resulted in a culture of extraordinary luxury. A large percentage of the
income generated by Egypt's own resources and by foreign trade went
into building projects of an unprecedented scale; inscriptions enumer-
ate the enormous quantities of gold, silver, bronze, and precious
stones used in the construction and decoration of the temples. Egypt's
wealth was symbolized by the sheer size of the monuments—every-
thing had to be bigger than before, from temples and palaces to
scarabs, from the colossal statues of the king to the shabti figures of his
elite.
Peace had also changed the Egyptians' attitude towards their foreign
neighbours, who were no longer primarily seen as the hostile forces of
chaos surrounding Egypt, the ordered world created at the beginning
of time. Amenhotep's court had become a diplomatic centre of inter-
national importance, and friendly contact with Egypt's neighbours had
led to an atmosphere of openness towards foreign cultures. During the
earlier part of the dynasty, immigrants had introduced their native
gods into Egypt and some of these deities had become associated with
the Egyptian king, especially in his warlike aspect, but now foreign
peoples were themselves seen as part of god's creation, protected and
sustained by the benevolent rule of the sun-god Ra and his earthly
representative, the pharaoh.
266 JACOBUS VAN D I J K
New Kingdom Religion
The sun-god and the king lay at the heart of Egyptian theological think-
ing and cultic practice as they had developed over the previous centuries.
The daily course of the sun-god, who was also the primeval creator-
god, guaranteed the continued existence of his creation. In the temple,
the sun-god's daily journey through the heavens was symbolically
enacted by means of rituals and hymns, the principal aim of which was
to maintain the created order of the universe. The king played a crucial
role in this daily ritual; he was the main officiant, the sun priest, who
had an intimate knowledge of all aspects of the sun-god's daily course.
Every sunrise was a repetition of the 'first occasion', the creation of the
world in the beginning. Ra himself went through a daily cycle of death
and rebirth; at sunset he entered the netherworld, where he was
regenerated and from which he was reborn in the morning as Ra-
Horakhty. Light could not exist without darkness; without death there
could be no regeneration and no life. Together with the sun-god the
dead were also reborn; they joined Ra on his daily journey and went
through the same eternal cycle of death and rebirth. Osiris, the god of
the dead and the underworld, with whom the deceased were tradition-
ally identified, was increasingly seen as an aspect of Ra, and the same
held true for all other gods, for, if the sun-god was the primeval creator,
then all the other gods had emerged from him and were therefore
aspects of him. In this sense a tendency towards a form of mono-
theism is inherent in the religion of New Kingdom Egypt.
Towards the end of the reign of Amenhotep III the cult of many
gods as well as that of his own deified self were increasingly solarized,
but at the same time the king appears to have tried to counterbalance
this development by commissioning an enormous number of statues
of a multitude of deities and by developing the cult of their earthly
manifestations as sacred animals. However, in hymns from the very
end of the reign, the sun-god is clearly set apart from the other gods—
he is the supreme god who is alone, far away in the sky, whereas
the other deities are part of his creation, alongside men and animals.
Amenhotep Ill's successor was soon to find a radically different solu-
tion to the problem of unity and plurality.
Although the seat of government during most of the New Kingdom
was the northern capital, Memphis, the 18th-Dynasty kings had origin-
ated from Thebes, and this city remained the most important religious
centre of the country. Its local god, Amun ('the hidden one'), had
become associated with the sun-god Ra and as Amun-Ra King of the
THE AMARNA PERIOD AND LATER NEW KINGDOM 267
Gods was worshipped in every major temple in Egypt, including Mem-
phis. The king was the bodily son of Amun born from the union of the
god with the queen mother in a sacred marriage that was ritually re-
enacted during the annual Opet Festival in Amun's temple at Luxor.
During the great processions that formed part of this important
festival, the king was publicly acclaimed as the earthly embodiment of
Amun; thus the king and the god were intimately linked by a powerful
amalgam of religious and political ties. All of this had made Amun-Ra
the most important god of the country, whose temple received a sub-
stantial part of Egypt's wealth and whose priesthood had acquired con-
siderable political and economic power. This, too, was soon to change
under Amenhotep's successor.
Amenhotep IV and Karnak
There can be little doubt that Amenhotep IV was officially crowned by
Amun of Thebes, for he is described as 'the one whom Amun has
chosen (to appear in glory for millions of years)' on some scarabs from
the beginning of his reign, but this token reference to Amun cannot
conceal the fact that the new king was clearly determined right from
his accession to go his own way. When exactly this accession took place
is still the subject of controversy; clearly Amenhotep was not originally
meant to succeed his father, for a crown prince Thutmose is known
from earlier in Amenhotep Ill's reign. Amenhotep IV is mentioned as
'real king's son' on one of the many mud jar sealings found in his
father's palace at Malkata, most of which are associated with the three
serf-festivals (jubilees) celebrated by Amenhotep III during the last
seven years of his reign. Opinions are divided over the issue of a pos-
sible co-regency between Amenhotep III and IV; some scholars have
opted for such a period of joint rule lasting for some twelve years,
others have at best admitted the possibility of a short overlap of one or
two years, whereas the majority of scholars reject it entirely.
Amenhotep IV began his reign with a major building programme at
Karnak, the very centre of the cult of Amun. The exact location of these
temples is unknown, but some, perhaps all of them, were situated to
the east of the Amun precinct and orientated towards the east—that is,
to the place of sunrise. The temples that he started to build here and
elsewhere were dedicated not to Amun, however, but to a new form of
the sun-god whose official name was 'The living one, Ra-Horus of the
horizon who rejoices in the horizon in his identity of light which is in
the sun-disc', a long formula that was soon enclosed in two cartouches
268 JACOBUS VAN DIJK
just like the names of a king, and that was often preceded in royal
inscriptions by the words 'my father lives'. The name of the god could
be shortened to 'the living sun-disc' or simply 'the sun-disc' (or, to use
the Egyptian word, the Aten). The word itself was not new; it had pre-
viously been used to refer to the visible celestial body of the sun.
During the reign of Amenhotep III this aspect of the sun-god had
become increasingly important, especially in the later years of his
reign. During the king's sed-festivals, his deified self had been identi-
fied with the sun-disc and in several inscriptions, most clearly in one
on the back pillar of a recently discovered statue, the king calls himself
'the dazzling Aten'. Originally this 'new' form of the sun-god was
depicted in the traditional manner, as a man with a falcon's head
surmounted by a sun-disc, but early in the reign of Amenhotep IV this
iconography was abandoned in favour of a radically new way of depict-
ing a god—as a disc with rays ending in hands that touch the king and
his family, extending symbols of life and power towards them and
receiving their offerings. Although the Aten clearly takes precedence
over the other gods, he does not yet replace them entirely.
One of the Karnak temples is devoted to a sed-festival, a remarkable
fact because kings did not normally celebrate their first jubilee until
their thirtieth regnal year. Unfortunately there is no indication of the
exact date of this festival of Amenhotep IV, but it must have taken
place within the first five years of the reign, possibly around years 2 or
3; if so, it might well have come at the regular interval of three years
after the last sed- festival of Amenhotep III, which had been celebrated
not long before the latter's death. This would provide an additional
argument against the assumption of a co-regency between Amenhotep
III and IV. The Aten, who is present in every single episode of the
jubilee rituals depicted on the walls of the new temple, is now evidently
identical with the deceased solarized Amenhotep III, and the sed-
festival celebrated by his son is as much a festival for the Aten as for the
new king, even though the latter is of necessity the chief actor in the
rituals. The Aten is the 'divine father' who rules Egypt as the celestial
co-regent of his earthly incarnation, his son. That the Karnak jubilee
was not considered to be Amenhotep IV's own official first serf-festival
is proved by a later inscription in which a courtier at Amarna includes
a wish to see the king 'in his first jubilee' in his funerary prayers,
clearly indicating that such a festival had not yet taken place.
Another extraordinary feature of the Karnak buildings of Amenho-
tep IV is the unprecedented prominence of the king's wife, Nefertiti, in
their decoration, and hence in the rituals that took place in them. One
THE AMARNA PERIOD AND LATER NEW KINGDOM 269
structure is devoted entirely to her alone, her royal husband being
absent from the reliefs. Nefertiti is given a new name, Neferneferua-
ten, and she, often accompanied by her eldest daughter, Meritaten,
performs many rituals that had until then been reserved for the king,
including those of 'presenting Maat' (maintaining the order of the
universe) and 'smiting the enemy' (subduing the powers of chaos). At
this early stage of the reign it is not so much that she is acting as an
official co-regent of her husband, but rather that the royal couple
together now represent the mythical twins that in the traditional
religion were called Shu and Tefnut, the first pair of divinities to issue
from the androgynous creator-god Arum. The original triad consisting
of Arum, the primeval father, his son Shu and his daughter Tefnut is
now replaced by a triad consisting of the Aten as the father and the
living king and queen as his children. The unique iconography of both
royals as displayed in statuary and reliefs reflects this new interpreta-
tion of their divine status.
Akhenaten and Amarna
Early in the fifth year of his reign, Amenhotep IV decided to sever all
links with the traditional religious capital of Egypt and its god Amun,
and to build an entirely new city on virgin soil that would be devoted
solely to the cult of the Aten and his children. At the same time he
changed his name to Akhenaten, meaning 'he who acts effectively on
behalf of the Aten' or perhaps 'creative manifestation of the Aten'. The
new city, nowadays known as Amarna, was called Akhetaten, 'Horizon
of the Aten'—that is, the place where the Aten manifests himself and
where he acts through his son, the king, who is 'the perfect child of the
living Aten'. Whether there were political as well as religious motifs for
this drastic decision remains unknown, although the king appears to
hint at opposition to his religious reforms in the decree inscribed on a
series of'boundary stelae' defining the territory of Akhetaten. Opposi-
tion there must have been, especially among the dispossessed priestly
establishment of the great temples of Amun at Thebes and probably
elsewhere as well. Even before the move to Akhetaten some of the
revenues of the established cults had been diverted to the cult of the
Aten, and the situation must have deteriorated even further when the
king abandoned the city of Amun for his new capital.
Before we examine this city, its inhabitants, and the new Atenist
religion as it was practised there, we must briefly summarize the main
political events of the reign of Akhenaten. We do not know when
270 JACOBUS VAN D I J K
exactly he took up residence in Akhetaten, but presumably it was
within a year or two of its foundation; the oaths sworn on that occasion
by the king regarding the boundaries of the city's territory were
renewed in regnal year 8. As soon as the decision to move had been
made, all building activities at Thebes ceased, although the king's
original name was removed from the inscriptions and replaced by the
new one.
Once Akhenaten was firmly settled in his new residence, a further
radicalization of his religious reforms took place. In year 9, the official
name formula of the Aten was changed to 'the living one, Ra, ruler of
the horizon who rejoices in the horizon in his identity of Ra the father
who has returned as the sun-disc'. Although this new formula removed
the name of the god Horus (which smacked too much of traditional
concepts), it clearly put even more emphasis on the father-son
relationship between the Aten and the king. Probably at the same time
as this name change took place, the traditional gods were banned
completely and a campaign was begun to remove their names and
effigies (particularly those of Amun) from the monuments, a Hercu-
lean task that can only have been carried out with the support of the
army. The traditional state temples were closed down and the cults of
their gods came to a standstill. Perhaps most important of all, the
religious festivals with their processions and public holidays were no
longer celebrated either.
The role of the military during the Amarna Period has long been
underestimated, partly because Akhenaten was thought to have been
a pacifist. More recently, however, it has been recognized not only
that the king's programme of political and religious reform could
never have succeeded without active military support, but also that
Akhenaten sent his army abroad to quash a rebellion in Nubia in year
12. It has even been suggested that he may have been involved in a con-
frontation with the Hittites, who during Akhenaten's reign defeated
the Human empire of Mitanni, Egypt's ally, thus destroying the care-
fully maintained balance of power that had existed for several decades,
although the diplomatic archive from Akhetaten (the 'Amarna letters')
shows that Egyptian military activity in northern Syria usually took the
form of limited police actions, the main goal of which was to prevent
the volatile vassal states in the area from switching sides. It was also
in year 12 that a great ceremony took place, during which the king
received the tribute from 'all foreign countries gathered together as
one', an event that may well be connected with the Nubian campaign
of the same year.
THE AMARNA PERIOD AND LATER NEW KINGDOM 271
Royal Women in the Amarna Period
At about the same time as these political events, an important change
took place within the royal family. Nefertiti had so far produced six
daughters, but no son, and, although she never lost her prime position
as 'great royal wife', a second wife of Akhenaten had appeared on the
scene at Akhetaten. It has often been speculated that she was a
Mitannian princess, but her name Kiya is a perfectly normal Egyptian
one and there is nothing to suggest that she was of foreign extraction.
She was given the newly created title 'greatly beloved wife of the king',
which sets her apart from other ladies in the royal harem, while at the
same time distinguishing her clearly from Nefertiti. In or shortly
before regnal year 12 she suddenly disappears from the monuments;
her name was erased from the inscriptions and replaced by those of
Akhenaten's daughters, most frequently that of Meritaten, and her
representations were likewise altered. Since even the funerary equip-
ment prepared for her, including a magnificent anthropoid coffin, was
adapted for a different royal person, it is most likely that Kiya at some
point fell from grace, perhaps because she had become too much of a
rival to Nefertiti after she had borne Akhenaten not only a further
daughter, but perhaps also a male heir. There is no hard evidence to
support this theory, but a single inscription from about this time
mentions 'the King's bodily son, his beloved, Tutankhaten' (the future
king Tutankhamun (1336-1327 BC)), who was almost certainly a son of
Akhenaten, but not of Nefertiti.
The latter's influence increased even further during the later part of
the reign, when she became the official co-regent of her husband as
Neferneferuaten with the throne name Ankh(et)kheperura; her role as
queen consort was taken over by her eldest daughter, Meritaten. What
prompted Akhenaten to appoint a co-regent, a step taken only in excep-
tional circumstances, is unknown. Perhaps opposition to his regime
elsewhere in the country (that is, in Thebes) was threatening to get out
of control, making it necessary to have someone who could act as king
and perhaps even take up residence outside Amarna; at any rate, a
Theban graffito dated to her regnal year 3 reveals that Neferneferuaten
owned a 'Mansion of Ankhkheperura in Thebes' that employed a
scribe of divine offerings of Amun, a clear indication that an attempt at
reconciliation with the old cults was undertaken. Most of this text con-
sists of the scribe's prayer to Amun, with a poignant appeal to the god
to come back and dispel the darkness that had descended upon his
followers.
272 JACOBUS VAN D I J K
Whether or not Nefertiti survived Akhenaten, who died in his year
17, is uncertain. An ephemeral king Smenkhkara with virtually the
same throne name as Nefertiti/Neferneferuaten appears in some
inscriptions from the end of the Amarna Period; in one or two rare
representations he is accompanied by his queen Meritaten. The
identity of this Smenkhkara is uncertain. Many scholars continue to
see him as Nefertiti's male successor, perhaps a younger brother or
even another son of Akhenaten, but there is a strong possibility that
'he' was actually none other than Nefertiti herself who, like Hatshep-
sut before her, had assumed a male persona and ruled alone for a brief
period after the death of Akhenaten, with Meritaten in the ceremonial
role of'great royal wife'. Akhenaten's successor probably did not sur-
vive him for very long, and, when he/she died, the very young Tutan-
khaten, the only remaining male member of the royal family, mounted
the throne. Early in his reign he and his queen, his half-sister
Ankhesenpaaten, abandoned Amarna and restored the traditional
cults. With him, one of the most incisive periods in Egyptian history
came to an end.
The Art and Architecture of the Amarna Period
The earliest representations of Amenhotep IV show him in a tradi-
tional style closely resembling the one used to portray both Thutmose
IV and Amenhotep III, but not long after his accession Amenhotep IV
had himself depicted with a thin, drawn-out face with pointed chin and
thick lips, an elongated neck, almost feminine breasts, a round pro-
truding belly, wide hips, fat thighs, and thin, spindly legs. At first the
new style was still fairly restrained, but on most of the Theban monu-
ments and during the early years at Amarna the king's features were
depicted in such an exaggerated way as to make him look like a carica-
ture; later in the reign a more balanced style developed. It was not only
Akhenaten, Nefertiti, and their daughters who were depicted in this
style, but all other human beings as well, albeit in a less exaggerated
form. This is not surprising, since representations of private individ-
uals had always followed the artistic model of the king of their time,
and Akhenaten in particular put much emphasis on the fact that he
was the 'mother who gives birth to everything' who had 'created his
subjects with his ka'. He was the creator-god upon earth who fash-
ioned mankind after his own image.
There can be little doubt that the extraordinary manner in which
Akhenaten portrayed himself, his family (and, to a lesser extent, all
THE AMARNA PERIOD AND LATER NEW KINGDOM 273
other human beings) on his monuments somehow reflects the king's
actual physical appearance, albeit in an exaggerated style that has been
termed 'expressionist' or even 'surrealist'. Inscriptions tell us that it
was the king himself who instructed his artists in the new style. Not
Plan of the site of Amarna, showing the main city and the outlying
temples, shrines, and settlements
274 JACOBUS VAN D I J K
only the human figure is affected by it, but also the way they interact.
Scenes of the royal family display an intimacy such as had never before
been shown in Egyptian art even among private individuals, let alone
among royalty. They kiss and embrace under the beneficent rays of the
Aten, whose love pervades all of his creation. Another characteristic
feature of the Amarna style is its extraordinary sense of movement and
speed, a general looseness' and freedom of expression that was to
have a lasting influence on Egyptian art for centuries after the Amarna
Period had come to an end.
In a different way, speed is also the determining factor of a new
building technique. Again, the earliest structures of Amenhotep IV
employed the traditional large sandstone blocks commonly used for
temple walls, but these were soon replaced in both Thebes and
Amarna by very much smaller blocks, the so-called talatat, typically
measuring about 60 x 25 cm. and therefore small enough for a single
man to lift and carry. This made it much easier to erect a large building
in a relatively short space of time. The new method was abandoned
again after the Amarna Period, perhaps because it had by then become
apparent that the reliefs carved on walls constructed of such small
blocks, needing as they did a great deal of plaster finishing to close the
gaps between individual stones, did not withstand the test of time as
well as traditionally built walls. Certainly Akhenaten's successors soon
found out that it also took far less time and effort to demolish buildings
constructed of talatat.
The 'looseness' of the Amarna art style is perhaps also matched by
the city plan of Akhetaten, at least as far as the living quarters are
concerned. Despite the fact that it was a newly planned city, it was not
built on a rigid orthogonal grid like the Middle Kingdom town of
Kahun, which had reflected the highly structured, bureaucratic society
of its time. The layout of Amarna is far more like a cluster of small
villages centred around loosely grouped houses both large and small,
each with its own subsidiary buildings such as grain silos, animal
pens, sheds, and workshops. The variety in size of these compounds
matches the differences in wealth and status between their owners.
Many of them have their own well, a unique feature of this city, which
made its inhabitants independent of the Nile for their daily water sup-
plies. In general Amarna looks more like a city that developed naturally
over a period of time, rather than as a result of careful planning.
Needless to say, however, the temples and palaces are a different
matter. Both were intimately linked with Akhenaten's religious ideas
and for this reason they must have been designed and planned by the
THE AMARNA PERIOD AND LATER NEW KINGDOM 275
king himself in close cooperation with architects and artists who
worked under his personal 'instruction', as inscriptions never tire
of telling us. We cannot describe these buildings in detail here, but a
few significant features must be mentioned. First of all, Akhenaten
and his family lived some distance away from the main city in what is
now known as the North Riverside Palace. A long spacious avenue, the
'royal road', ran via the North Palace (the queen's residence) in a
straight line of about 3.5 km. to the Central City with its two palaces
(one used among other things for ceremonial state occasions like the
reception of foreign envoys, the other serving as the king's working
palace with a 'window of appearances', through which he rewarded
loyal officials) and two major Aten temples. Of these, the Great Temple
to the Aten was the Amarna equivalent of the great temple enclosure of
Amun-Ra at Thebes; it contained several separate buildings, including
a structure with a benben-stone, the sacred sun symbol, the archetype
of which stood in the temple of Ra at Heliopolis. This is one of the
indications of the influence of Heliopolitan theology on Akhenaten's
thinking, another being that the king had planned a cemetery for the
sacred Mnevis bull of Ra-Atum of Heliopolis at Amarna. The other
Aten temple was very much smaller and lay immediately to the south
of the king's working palace. It appears to have been dedicated to the
king as well as to the Aten and may have been the equivalent of the
traditional so-called temples of a million years, and, like the temples of
that name on the Theban west bank, may have served as a mortuary
chapel for Akhenaten that was orientated towards the entrance of the
wadi in which the royal tomb was located.
The most conspicuous difference between, on the one hand, the
Aten temples both at Amarna and earlier on at Karnak, and, on the
other hand, the traditional temples is that the former are open to the
skies. A typical temple of the traditional type began with a pylon and an
open peristyle court followed by a succession of further courts and
rooms, which gradually became smaller and darker as the worshipper
penetrated further into the building. In the innermost sanctuary the
cult image of the god was kept in a shrine that for most of the time was
in total darkness. Akhenaten's god was there for all to see, however,
and no man-made cult image was, therefore, needed. The only statues
to be found in Atenist temples are representations of Akhenaten and
other members of the royal family. In the architecture of these temples
a deliberate effort has been made to create as little shadow and dark-
ness as possible; even the lintels above the doorways were open in the
middle. These 'broken' lintels were an architectural innovation that
276 JACOBUS VAN DIJK
continued to be used for certain temple doorways until Graeco- Roman
times. The king worshipped his god in open courtyards studded with a
large number of small altars on which offerings to the Aten were
made. Why there are so many altars remains a mystery; perhaps the
most likely explanation is that they are altars for the dead who are
being fed in the temples as part of the daily cult.
Light was the most essential aspect of the Aten, who was a god of the
light that emerged from the sun's disc and kept every living being alive
in continuous creation. He was the creator-god who ruled the world as
the celestial king. And, just as the Aten was king of the world, so
Akhenaten was the god of his subjects. His daily 'procession', when he
drove in his chariot along the royal road from the North Riverside
Palace to the Central City, replaced the traditional divine processions
during which the inhabitants of a town could come into contact with
the deities whose statues were normally hidden from view in the
temple. Akhenaten was, as his name indicates, the 'creative manifesta-
tion of the Aten', through whom the Aten does his beneficial work. It
was the king who 'made' mankind and especially his elite, whom he
had chosen himself. In their inscriptions these officials denied their
true background, even though some of them must surely have come
from influential families; they all presented themselves as having been
poor, wretched orphans, owing their whole existence to the king who
had 'created them with his ka'. The king's work was likened to the
annual inundation of the Nile, which sustained mankind and all other
living beings. Personal piety was now identical with total loyalty
towards Akhenaten personally. In their private houses the Amarna
elite kept small shrines with altars and stelae representing the holy
royal family, which replaced the old household shrines for local deities.
Tombs and Funerary Beliefs at Amarna
Even in the tombs of the elite at Akhetaten, the king totally dominated
the wall decoration. Representations of Akhenaten and his wife and
children (as well as depictions of the various temples of Akhetaten)
were ubiquitous, and hymns and offering formulas were dedicated
as often to the king as to the Aten. It is notable that these offering
formulas were frequently—although not exclusively—addressed to
the god by the king himself rather than by the tomb-owner. The only
surviving copies of the famous Great Hymn to the Aten, the most com-
prehensive text on the main dogmas of the new religion (probably
composed by Akhenaten himself), are found in these tombs. This
THE AMARNA PERIOD AND LATER NEW KINGDOM 277
hymn and all other texts at Amarna were written in a newly created
official language that was much closer to everyday speech than the
classical Egyptian that had so far been used for official and religious
texts. The boundary between official and vernacular language did not
disappear completely, but the use of the latter for literary compositions
was greatly stimulated by this development and gave rise to a whole
new literature in the centuries after the Amarna Period.
Osiris, the most important god of the dead, appears to have been
proscribed from the very beginning of Akhenaten's reign. Even the
doctrine that viewed Osiris as the nocturnal manifestation of the sun-
god, well established in funerary religion long before Amarna, was
rejected by Akhenaten. The Aten was a god of life-giving light; during
the night he was absent, but it is unclear where he was thought to go.
Darkness and death were completely ignored instead of being
regarded as a positive, necessary state of regeneration. At night the
dead were simply asleep like every other living being and like the Aten
himself. They were not in the 'Beautiful West', the underworld, and
their tombs were not even physically located in the west but in the east,
where the sun rises. The 'resurrection' of the dead took place in the
morning, when the Aten arose. The Aten himself represented 'the
time in which one lives', as the Great Hymn expressed it. The mode of
existence of the dead was, therefore, one of a continual presence with
the Aten and the king in the temple, where they (or their fea-souls) fed
on the daily offerings. For this reason the Amarna private tombs were
full of representations of the temples of the Aten and of the king
driving along the royal road towards the temples and offering in them.
The temples and palaces of Akhetaten were the new hereafter; the
dead no longer lived in their tombs but on earth, among the living. The
tombs, therefore, served only as their nightly resting places. Mummifi-
cation persisted, because at night the ba returned to the body until
sunrise. For this reason, funerary rites, including offerings and tomb
equipment, appear to have continued as well, although most shabti
figures no longer have the chapter from the Book of the Dead tradition-
ally inscribed on them. It is difficult to be sure what the Amarna Period
private coffins and sarcophagi looked like, since no examples have ever
been found at Amarna. On Akhenaten's own large stone sarcophagus
the four winged goddesses who traditionally stood at the corners were
replaced by figures of Nefertiti, and some finds from other sites sug-
gest that private sarcophagi may also have been decorated with depic-
tions of members of the family of the deceased rather than funerary
deities. There was also no 'judgement of the dead' before the throne of
278 JACOBUS VAN DIJK
Osiris, which the deceased formerly had to pass through in order to
gain the status of a maaty ('righteous one'); instead, the king's officials
earned their life after death by following Akhenaten's teaching and by
being totally loyal to him during their lifetime. Akhenaten was the god
who granted life and a burial after old age in his favour; he embodied
maat and it was through loyalty to him that his subjects could become
maatyu. Without this there would be no life after death, and continued
existence upon earth depended on the king, who therefore monopo-
lized all aspects of the Amarna religion, including funerary beliefs.
Life outside Amarna during the Amarna Period
Most of our knowledge of Akhenaten's new religion derives from his
early monuments at Thebes and from the city at Amarna itself. What
happened in the rest of the country, especially after the king had
moved to his new city, is very much less clear. Akhenaten would
almost certainly have travelled outside Akhetaten; he even stipulates
(on the 'boundary stelae') that, if he were to die elsewhere, his body
should be brought back to Amarna and buried there. Apart from early
building activities in Nubia, we know of Aten temples in Memphis and
Heliopolis and there may have been others. Some Memphite blocks
display the late form of the Aten name (after regnal year 9), and a stray
block from Thebes also has this form; therefore, even after the radicali-
zation of Akhenaten's reform, construction work outside Amarna
obviously continued. What we do not know is the extent to which the
traditional cults were really abolished; our picture is very much col-
oured by a later description of the situation in the Restoration Decree
of Tutankhamun, the tenor of which is quite obviously propagandistic.
In everyday practice, the new religion probably only replaced the
official state cult and the religion of the elite; the majority of the people
must have continued to worship their own traditional, often local gods.
Even at Amarna itself there are a fair number of surviving votive
objects, stelae, and wall paintings that depict or mention gods such as
Bes and Taweret (both connected with childbirth); the harvest-goddess
Renenutet; the protective deities Isis and Shed ('the saviour', a new
form of Horus not found before Amarna); Thoth (the god of the
scribes); Khnum, Satet, and Anuket (the triad of Elephantine); Ptah of
Memphis; and even Amun of Thebes.
It is not always easy to decide whether tomb reliefs, stelae, and items
of burial equipment that mention the Aten together with traditional
gods such as Osiris, Thoth, or Ptah date from the beginning of the
THE AMARNA PERIOD AND LATER NEW KINGDOM 279
reign or later, or even from the time immediately after the Amarna
Period. Nor do we know whether the deceased buried in a necropolis
other than that of Akhetaten were supposed to partake of offerings in
the Aten temple at Amarna or in their home town, or how the dead
were thought to live on in places where there was no Aten temple at all.
Much further research is needed here, particularly in the necropolis of
Memphis, where many tombs of this period have yet to be discovered.
It is also unclear what happened to the civil administration during
the Amarna Period. Clearly Akhetaten had replaced Thebes as the
religious capital and the centre of the state cult, but did it also replace
Memphis as the administrative capital? One of the two viziers resided
at Amarna, but his northern colleague remained posted at Memphis.
It is probable that this city in fact retained its position of administrative
centre for the country throughout the Amarna Period. The situation
during the Saite Period may well afford a parallel: the 2 6th-Dynasty
kings greatly favoured Sais, their home town (although they were
originally of Libyan descent), which functioned as their capital, and
much of the state revenues went to the temple of its goddess Neith. Yet
Memphis remained the administrative centre of Egypt throughout the
Saite Period, a situation that persisted until the successor of Alexander
the Great removed the latter's mortal remains to Alexandria and made
this city the centre of Ptolemaic and Roman Egypt.
The Aftermath of the Amarna Period
Although the Amarna episode lasted barely twenty years, its impact
was enormous. It is perhaps the single most important event in
Egypt's religious and cultural history and it left deep scars on the col-
lective consciousness of its inhabitants. Superficially, the country
returned to the traditional religion of the time before Akhenaten, but
in reality nothing would ever be the same again. Some of the changes
can be detected in the burial arrangements of the elite, always a good
barometer of shifting religious attitudes. Most conspicuous are the
developments in tomb architecture. At Memphis in particular, free-
standing tombs appear that in all essential aspects resemble temples.
In Thebes rock tombs continue to be used, but their architecture and
decoration are adapted to the same new concept, that of the tomb as
the private mortuary temple for its owner, whose funerary cult is inte-
grated with the cult of Osiris. This god, who had been banned by
Akhenaten, was now universally seen as the nocturnal manifestation
of Ra, and his role in funerary matters increased dramatically as
280
JACOBUS VAN DIJK
Plan of a group of tombs in the New Kingdom necropolis at Saqqara, where many
important officials of the late i8th and i9th-2oth Dynasties were buried
compared to the days before the Amarna Period. In these tombs, the
solar symbol par excellence, the pyramid, previously a royal prerogative,
sat on the roof of the central chapel, usually with a capstone (pyramid-
ion) showing scenes of worship before Ra and Osiris. In the central
chapel itself the main stele, the focal point of the cult, often showed a
symmetrically arranged double scene comprising both of these gods
seated back to back. Statues that had previously been typically placed in
temples began to appear in private tombs, including images of various
deities and naophorous statues that show the deceased holding a
shrine with an image of a god.
The reliefs and paintings on the walls of the tombs were no longer
primarily concerned with scenes from the owner's career and profes-
sional occupation, although such scenes do not disappear completely,
but instead concentrate on showing him adoring Ra, Osiris, and a wide
variety of other gods and goddesses, wearing a long pleated linen cos-
tume (often wrongly called the 'dress of daily life') and an elaborate
wig. The same festive costume also appears on anthropoid sarcophagi
and shabtis, which hitherto had shown the deceased exclusively as a
THE AMARNA PERIOD AND LATER NEW KINGDOM 281
mummy. Apart from one or two examples from very early in the reign
of Tutankhamun, scenes in which the deceased is shown presenting
offerings to the king disappear completely; his place is now occupied
by Osiris enthroned. In general, religious scenes and texts, often taken
from the Book of the Dead, dominated the post-Amarna tomb decora-
tion. Illustrations and textual excerpts from various exclusively royal
funerary compositions such as the Litany ofRa and the so-called Books
of the Underworld began to appear on the walls of private tombs, first at
Deir el-Medina, but soon elsewhere as well. All of these features may
be explained as a reaction against Akhenaten's total monopolization of
the funerary cult of his subjects and the role that the Aten temples had
played in Amarna religion as the new 'hereafter'. The tomb-owners now
had their own temples in which they themselves worshipped the gods,
without the intervention of the king, whose role was thus minimalized.
The changes in funerary culture just outlined are symptomatic of
the totally different relationship between the gods and their worship-
pers, and the role played by the king in this relationship. In another
200 years, the ultimate consequence of this new world-view would be
shown by the realization of the so-called Theban theocracy, whereby
Amun himself was thought to rule as king of Egypt, governing his
subjects by means of direct intervention in the form of oracles. Before
we can discuss this development, however, we must return to the
political and dynastic history of Egypt following the end of the Amarna
Period.
Tutankhamun
The young Tutankhaten, still a child, had ascended the throne at
Amarna, but soon afterwards, perhaps as early as his first regnal year
or shortly afterwards, he abandoned the city founded by his father.
People continued to live in Akhetaten for some time, but the court
moved back to Memphis, the traditional seat of government. The old
cults were restored and Thebes once more became the religious centre
of the country. The king's name was changed to Tutankhamun and the
epithet 'ruler of southern Heliopolis', a deliberate reference to Karnak
as the centre of the cult of the sun-god Amun-Ra, was added to it. The
name of his great royal wife, his half-sister Ankhesenpaaten, was like-
wise altered to Ankhesenamun. Tutankhamun was by no means the
first ruler in the history of the dynasty to have ascended the throne as a
child. Both Thutmose III and Amenhotep III had been very young at
their accessions, but in both cases a senior female member of the royal
282 JACOBUS VAN DIJK
family (Hatshepsut and Mutemwiya, respectively) had acted as regent
during their early years. No such option was available now; therefore
the role of regent was played by a senior military official with no
bloodlinks with the royal family, the commander-in-chief of the army,
Horemheb. His titles as regent indicate that he gained the right to
succeed Tutankhamun if he were to die without issue. Horemheb
would in fact eventually become king himself, and in his Coronation
Text (a unique inscription giving an account of his rise to power,
carved on the back of a statue now in the Egyptian Museum, Turin), he
seems to suggest that it was he who advised the king to abandon
Amarna 'when chaos had broken out in the palace' (that is, after the
deaths of Akhenaten and his ephemeral successor). Obviously the
army had come to the conclusion that Akhenaten's experiment had
ended in disaster and had withdrawn its support from the religious
reforms they had initially helped to carry through, another tell-tale
sign of the important role played by the military in this whole affair.
The most important document of Tutankhamun's reign is the so-
called Restoration Stele, which presents an extremely negative descrip-
tion of the state in which Akhenaten's reforms had left the country.
The temples of the gods had become ruins, their cults abolished. The
gods had, therefore, abandoned Egypt; if one prayed to them, they no
longer answered, and, when the army was sent to Syria to expand the
boundaries of Egypt, it met with no success. The prominence of this
last phrase probably indicates why the army no longer supported the
Amarna policy. During Akhenaten's reign, Egypt's ally Mitanni had
been defeated by the Hittites, who were now the major power in the
north. This had prompted some of Egypt's vassals, notably Aziru of
Amurru, to try to establish an independent buffer state between the
two rival superpowers. Egypt was beginning to lose some of its
northernmost territory, and the army, restricted to limited police
actions in Syria, was obviously unable to do anything about it. With the
accession of Tutankhamun, these restrictions were evidently lifted,
since the reliefs in the inner courtyard of Horemheb's magnificent
Memphite tomb (decorated around this time) include the claim that
his name was 'renowned in the land of the Hittites', thus suggesting
that, early in Tutankhamun's reign, Horemheb must have been
engaged in military confrontations with the Hittites. These skirm-
ishes, as well as later ones, seem to have failed to establish a new
balance of power. On the other hand, simultaneous attempts to
reassert Egyptian authority in Nubia, documented by these same
reliefs, were probably more successful.
THE AMARNA PERIOD AND LATER NEW KINGDOM  283
In Egypt itself, a major campaign to restore the traditional temples
and to reorganize the administration of the country was set in motion.
The enterprise was led by the chief of Tutankhamun's treasury, Maya,
who was sent on a major mission to temples from the Delta to
Elephantine, in order to levy taxes on their revenues, which had pre-
viously been diverted to the Aten temples. Some of the measures later
described in Horemheb's Coronation Text and in his great Karnak
Edict may actually have been carried out during the reign of Tutan-
khamun. Maya was also responsible for the gradual demolishing of the
temples and palaces of Akhenaten, first at Thebes, but later at Amarna
as well. Most of the Theban talatat found their way into the founda-
tions and pylons of new construction works in Luxor and Karnak. As
overseer of works in the Valley of the Kings, Maya must have organ-
ized the transfer of Akhenaten's mortal remains to a small undecor-
ated tomb in the valley (assuming that the body found in KV 55 is
indeed Akhenaten's, as seems likely); later he was responsible for the
burials of Tutankhamun and his successor Ay (1327-1323 BC)and for
the reorganization of the workmen's village at Deir el-Medina when
work began on the tomb of Horemheb.
The Reigns of Ay and Horemheb
The events surrounding the death of Tutankhamun are still far from
clear. The king died unexpectedly in his tenth regnal year, at a time
when Egypt was engaged in a major confrontation with the Hittites
that ended in an Egyptian defeat at Amqa, not far from Qadesh. News
of this disaster reached Egypt at about the time of Tutankhamun's
death. We do not know whether Horemheb himself was leading the
Egyptian troops in this battle, but the fact that he does not appear to
have been involved in the burial arrangements for Tutankhamun,
despite his role as regent and heir presumptive, is highly suggestive.
Instead, Ay, a senior court adviser who had been one of Akhenaten's
most trusted officials and may have been a relative of Amenhotep Ill's
wife, Queen Tiy, conducted the obsequies and shortly afterwards
ascended the throne. Apparently he did so at first as a kind of interim
king, for Tutankhamun's widow, Ankhesenarnun, was trying to
negotiate a peace with the Hittites by writing to the Hittite king
Shupiluliuma to ask him for a son who could marry her and become
king of Egypt, in order that Egypt and Hatti should become 'one
country', an extraordinary step that may possibly have been instigated
by Ay. This request met with much suspicion in the Hittite capital and,
284 JACOBUS VAN DIJK
when Shupiluliuma was finally convinced of the Egyptian queen's
honorable intentions and sent his son Zannanza to Egypt, the un-
fortunate prince was murdered en route, perhaps by forces loyal to
Horemheb in Syria. The result was prolonged warfare with the
Hittites.
King Ay, who must have been fairly aged when he mounted the
throne, ruled for at least three full years. A fragmentary cuneiform
letter appears to suggest that he tried to make amends with the
Hittites, denying all responsibility for the death of the prince, but to no
avail. He also made a conscious effort to prevent Horemheb from
asserting his rights after his death, for he appointed an army com-
mander called Nakhtmin (possibly a grandson of his) as his heir. Des-
pite this, however, Horemheb succeeded in ascending the throne after
Ay's demise and soon set out to deface the monuments of his pre-
decessor and to destroy those of his rival Nakhtmin.
If Horemheb's path to the throne had been beset with difficulties,
his actual reign (1323-1295 BC) appears to have been relatively unevent-
ful. It should be borne in mind, however, that there are few inscrip-
tions from the later part of his reign. Even its length is still uncertain;
his highest attested date is year 13, but on the basis of Babylonian
chronology and two posthumous texts many claim that he reigned for
nearly twice as long as this. The unfinished state of his royal tomb in
the Valley of the Kings (KV 57), however, even if it was not begun
before his year 7, is difficult to reconcile with such a long reign.
Trouble with the Hittites over territories in northern Syria continued,
and around regnal year 10 the Egyptians appear to have made an
unsuccessful attempt to reconquer Qadesh and Amurru, although it is
typical of the reign that our sources for this confrontation are Hittite,
and not Egyptian texts. It is even possible that Horemheb finally came
to an agreement with his enemy, for a later Hittite text refers to a treaty
that had been in force before it was broken during the reigns of
Muwatalli and Sety I (1294-1279 BC).
At home, Horemheb embarked on a number of major building pro-
jects, including the Great Hypostyle Hall in Karnak. He may also have
begun the systematic demolition of the city of Amarna, still inhabited
at this time. Two stone fragments including a statue base bearing his
cartouches were found there. The reorganization of the country was
also taken in hand with great gusto. The Great Edict, which he pub-
lished on a stele in the temple of Karnak, enumerates a large number
of legal measures enacted in order to stamp out abuses such as the
unlawful requisitioning of boats and slaves, the theft of cattle hides,
THE AMARNA PERIOD AND LATER NEW KINGDOM 285
the illegal taxation of private farmland and fraud in assessing lawful
taxes, and the extortion of local mayors by officials organizing the
king's annual visit to the Opet Festival during the journey from Mem-
phis to Thebes and back. Other paragraphs deal with the regulation of
the local courts of justice, the personnel of the royal harem and other
state employees, and the protocol at court.
Perhaps the most salient feature of Horemheb's reign is the way
that he legitimized it; after all, he was of non-royal blood and was,
therefore, unable to claim a 'genealogical' link with the dynastic god
Amun. It is often maintained that his queen, a songstress of Amun
called Mutnedjmet, should be identified with a sister of Nefertiti of
that name, but this is not very likely as she appears to have become his
wife well before his accession, quite apart from the fact that the legit-
imizing force of such a royal marriage may well have been question-
able, given the circumstances. In his Coronation Text Horemheb does
not hide his non-royal background, but instead puts much emphasis
on the fact that, as a young man, he was chosen by the god Horus of
Hutnesu, presumably his home town, to be king of Egypt; he then goes
on to describe how he was carefully prepared for his future task by
being the king's (that is, Tutankhamun's) deputy and prince regent, a
claim largely substantiated by the inscriptions in his pre-royal tomb in
the Memphite necropolis. It is Horus of Hutnesu who finally presents
him to Amun during the Opet Festival procession, and who then pro-
ceeds to crown him as king. Horemheb thus owes his kingship to the
will of his personal god and to divine election during a public appear-
ance of Amun (that is, by means of an oracle). In this respect
Horemheb's coronation resembles that of Hatshepsut (1473-1458 BC),
who had also been elected by an oracle of Amun after having been
regent. However, Hatshepsut was at least able to claim to be of royal
blood herself and actually stressed that Amun had fathered her by the
queen mother, a subject that Horemheb carefully avoids in his
Coronation Text.
Rameses I
The principle of electing a non-royal heir was adopted by Horemheb
and the early Ramessid rulers, the first of whom was appointed by
Horemheb as prince regent during his lifetime with much the same
titles as he himself had held under Tutankhamun. This man, Para-
messu, acted as Horemheb's vizier as well as holding a number of
military titles including that of commander of the fortress of Sile, an
286 JACOBUS VAN D I J K
important stronghold on the landbridge connecting the Egyptian Delta
with Syria-Palestine. The role assigned to Paramessu once more
reveals Horemheb's preoccupation with the military situation in
Egypt's northern territories. Paramessu's family came from Avaris,
the former capital of the Hyksos, and the role of its local god Seth, who
had retained strong connections with the Canaanite god Ba e al, appears
to have been comparable with that of Horus of Hutnesu in Horem-
heb's career. In the light of this it is interesting to observe that Horem-
heb built a temple for Seth at Avaris. The Ramessid royal family
considered the god Seth to be their royal ancestor, and an obelisk
(originally from Heliopolis), recently discovered, on the seabed off the
coast of Alexandria, shows Sety I as a sphinx with the head of the Seth-
animal offering to Ra-Atum.
When Horemheb died, apparently childless, Paramessu succeeded
him as Rameses I (1295-1294 BC). With him began a new dynasty, the
19th, although there is some evidence to suggest that the Ramessid
pharaohs considered Horemheb as the true founder of the dynasty.
Rameses I must have been old when he mounted the throne, since his
son and probably also his grandson had already been born before his
accession. During his short reign (barely one year), and maybe even
before, his son Sety was appointed vizier and commander of Sile but
also held a number of priestly titles linking him with various gods
worshipped in the Delta, including that of high priest of Seth. In his
Coronation Text Horemheb had mentioned that he had equipped the
newly reopened temples with priests 'from the pick of the army',
providing them with fields and cattle. From other documents we know
that retired soldiers were often given a priestly office and some land in
their native towns, so Sety may also not have been particularly young
when his father mounted the throne.
Sety I and the 'Restoration'
Sety I must be credited with the bulk of the restoration of the tradi-
tional temples, continuing and surpassing the efforts of his predeces-
sors. Everywhere inscriptions of pre-Amarna pharaohs were restored,
and the names and representations of Amun hacked out by Akhenaten
were recarved. He also soon embarked on an ambitious building pro-
gramme of his own. Practically everywhere in the country, and particu-
larly in the great religious centres of Thebes, Abydos, Memphis, and
Heliopolis, new temples were erected or existing ones expanded.
Among the latter was the temple of Seth at Avaris, a city that was soon
THE AMARNA PERIOD AND LATER NEW KINGDOM 287
to become the new Delta residence of the Ramessid rulers. At Karnak,
Sety continued the construction of the Great Hypostyle Hall begun by
Horemheb, which was connected with his own mortuary temple at
Abd el-Qurna, directly opposite Karnak on the west bank of the Nile.
Together with Hatshepsufs temple at Deir el-Bahri, which he restored,
these buildings provided a splendid new setting for the important
annual Beautiful Feast of the Valley, during which Amun of Karnak
visited the gods of the west bank and people came to the tombs of their
deceased relatives to eat, drink, and be merry in their company. At
Abydos Sety I built a magnificent cenotaph temple for the god Osiris,
following Middle Kingdom and early 18th-Dynasty examples. The
famous king-list in this temple, a list of the royal ancestors partici-
pating in the offering cult for Osiris, provides the first evidence that the
Amarna episode was now completely obliterated from official records.
In the list Amenhotep III is directly followed by Horemheb, and other
sources indicate that the regnal years of the kings from Akhenaten to
Ay were added to those of Horemheb.
Sery's building programme was made possible because he reopened
several old quarries and mines, including those in Sinai, and also
because, like his predecessors, he raided Nubia for captives who could
be employed as cheap labour. Security was another reason for these
Nubian campaigns, for the finances for his building projects came
from the exploitation of gold mines both there and in the Eastern
Desert. The mines in the latter area in particular were worked on
behalf of Sety's great Osiris temple at Abydos; in regnal year 9 the road
leading to them was provided with a resting-place, a newly dug well,
and a small temple. In Nubia there was a failed attempt to sink a new
well to make the more profitable mines in some of the remoter areas
accessible.
Further resources had previously come from the Egyptian territor-
ies in Palestine and Syria and it was now essential to reassert Egyptian
authority over these areas. Sety began in his regnal year i with a rela-
tively small-scale campaign against the Shasu in southern Palestine,
soon followed by military expeditions further north. In a later war he
moved into territory held at the time by the Hittites and managed to
reconquer Qadesh, which in turn prompted Amurru to defect to the
Egyptian side. The result was a war with the Hittites during which both
vassal states were lost again, followed by a period of guarded peace.
Sety I was also the first king to have to face incursions by Libyan tribes
along the western border of the Delta. These tribes, who appear to have
been motivated primarily by famine, were to continue to cause prob-
288 JACOBUS VAN DIJK
lems throughout the rest of the New Kingdom, but little is known
about this first attempt to settle in Egypt, other than the fact that Sety's
campaign against them probably took place before his confrontation
with the Hittites.
The reliefs on the northern exterior wall of the Great Hypostyle Hall
documenting the Libyan and Syrian campaigns are in a new, much
more realistic style, which, despite a few precursors from the time of
Thutmose IV and Amenhotep III, was clearly influenced by the realism
of the Amarna style. More than the traditional scenes of slaying the
enemy with their strong symbolic content, these battle reliefs create the
feeling that we are looking at a real, historical event. An important role
in these reliefs is played by a 'group-marshaller and fan-bearer' called
Mehy (short for Amenemheb, Horemheb, or some similar name), who
accompanies Sety in a number of scenes. It is unlikely that this man was
ever more than a trusted military officer who perhaps conducted some
of the campaigns instead of the king himself, but Sety's successor
Rameses II (1279-1213 BC), eager to stress his own role on the battlefield
during the reign of his father, had Mehy's names and figures erased and
in some cases replaced by his own as crown prince.
Rameses II
Unfortunately it is not known how long Sety I occupied the throne. His
highest attested regnal year is his eleventh, but he may have ruled for a
few years more. Towards the end of his reign—we do not know exactly
when—he appointed his son and heir Rameses as co-regent while the
latter was still 'a child in his embrace'. The sources for this co-regency
all date from Rameses' reign as sole king, however, and he may well
have exaggerated its length and importance. It is nevertheless signifi-
cant that Rameses received the kingship in this way. Although clearly
a son of Sety I, he was almost certainly born during the reign of
Horemheb, before his grandfather ascended the throne, and at a time
when both Rameses I and Sety I were still simply high officials, a fact
later emphasized rather than disguised by Rameses II himself in
much the same way as Horemheb had done in his Coronation Text.
Although his father was obviously king when Rameses II was crowned
as co-regent, his election resembles that of Horemheb. Clearly the
succession of the crown prince was not a foregone conclusion and had
to be secured while his father was still alive. Only later, when Rameses
II ruled alone, did he revert to the old 'myth of the birth of the divine
king' that had legitimized the rulers of the i8th Dynasty.
THE AMARNA PERIOD AND LATER NEW KINGDOM 289
Very early in his reign, probably still as co-regent of his father, he
went on his first military campaign, a limited affair aimed at quelling a
'rebellion' in Nubia. Reliefs in a small rock temple at Beit el-Wali com-
memorating the event show the young king in the company of two of
his children, the crown prince, Amunherwenemef, and Rameses'
fourth son, Khaemwaset, who, although shown standing proudly in
their chariots, must have been mere striplings at the time. Throughout
the Ramessid Period the royal princes, who in the i8th Dynasty had
only occasionally been depicted in the tombs of their non-royal nurses
and teachers, would be prominent on the royal monuments of their
father, perhaps in order to emphasize that the kingship of the new
dynasty was now well and truly hereditary again. Almost without
exception every Ramessid crown prince held the title, honorific or real,
of commander-in-chief of the army, a combination first seen in the
case of Horemheb, the founder of their dynasty.
In his fourth regnal year Rameses II mounted his first major cam-
paign in Syria, as a result of which Amurru once again returned to the
Egyptian fold. This was not to last long, however, for the Hittite King
Muwatalli decided at once to reconquer Amurru and to try to prevent
further losses of territory to the Egyptians, with the result that the
following year Rameses again passed the border fortress at Sile, this
time in order to wage war directly against his rival. The battle of
Qadesh that followed is one of the most famous armed conflicts of
antiquity, perhaps not so much because it was significantly different
from earlier battles, but because Rameses, despite the fact that he was
unable to achieve his goals, presented it at home as a huge victory
described at large in lengthy compositions, which, in a propaganda
campaign of unprecedented proportions, were carved on the walls of
all the major temples.
In actual fact, Rameses had wrongly been led to believe that the
Hittite king was in the far north at Tunip, too scared to confront the
Egyptians, whereas in reality he was nearby on the other side of
Qadesh. Rameses had, therefore, made a quick advance to Qadesh
with only one of his four divisions and was then suddenly obliged to
face the huge army that the Hittite king had mustered against him.
Muwatalli first destroyed the advancing Egyptian second division,
which was about to join the first, then turned around to crush Rameses
and his troops. In his later descriptions of the battle Rameses tells us
that this was his true moment of glory, for, when even his immediate
attendants were ready to desert him, he called out to his father Amun
to save him, then almost single-handedly managed to fight off the
290 JACOBUS VAN DIJK
Hittite attackers. But Amun heard his prayers and rescued the king by
causing an Egyptian support force from the coast of Amurru to arrive
in the nick of time. These forces now attacked the Hittites in the rear
and, together with Rameses' division, severely reduced the number of
the enemy's chariots and sent the remainder fleeing, many of them
ending up in the river Orontes. With the arrival of the third division at
the close of the combat, followed by the fourth at sunset, the Egyptians
were able to reassemble their forces and were now ready to face their
enemy the next morning. But, despite the fact that the Egyptian
chariotry now outnumbered their Hittite counterparts, Muwatalli's
formidable army was able to hold its ground and the battle ended in
stalemate. Rameses declined a Hittite peace offer, although a truce was
agreed. The Egyptians then returned home with many prisoners of
war and much booty, but without having achieved their goal. In sub-
sequent years several other fairly successful confrontations in Syria-
Palestine took place, but each time the vassal states conquered on
these occasions quickly returned to the Hittite fold once the Egyptian
armies had gone home, and Egypt never regained Qadesh and Amurru.
In year 16 of Rameses II's reign, Muwatalli's young son Urhi-
Teshub, who had succeeded his father as Mursili III, was deposed by
his uncle Hattusili III and, two years later, after some failed attempts
to regain his throne with the help first of the Babylonians, then of the
Assyrians, he finally fled to Egypt. Hattusili immediately demanded
his extradition, which was refused, and so the Hittite king was ready to
wage war against Egypt again. Meanwhile, however, the Assyrians had
conquered Hanigalbat, a former vassal state that had recently defected
to the Hittites, and were now threatening Carchemish and the Hittite
empire itself. Faced with this menacing situation Hattusili had no
choice but to open peace negotiations with Egypt, which finally led to a
formal treaty in regnal year 21. Although the Egyptians had to accept
the loss of Qadesh and Amurru, the peace brought a new stability on
the northern front, and, with the borders open to the Euphrates, the
Black Sea, and the eastern Aegean, international trade soon flourished
as it had not done since the days of Amenhotep III. It also meant that
Rameses II could now concentrate on the western border, which was
under constant pressure from Libyan invaders, particularly on the
fringes of the Delta, where Rameses built a whole series of fortifica-
tions. In year 34 the bond with the Hittites was further strengthened by
a marriage between Rameses and a daughter of Hattusili, who was
received with much pomp and circumstance and was given the Egypt-
ian name Neferura-who-beholds-Horus (i.e. the King).
THE AMARNA PERIOD AND LATER NEW KINGDOM 291
This Hittite princess was only one of seven women who gained the
status of 'great royal wife' during Rameses' very long reign of sixty-
seven years. When he had become his father's co-regent he had been
presented with a harem full of beautiful women, but apart from these
he had two principal wives, Nefertari and Isetnefret, both of whom
bore him several sons and daughters. Nefertari was 'great royal wife'
until her death in about year 25, when the title passed on to Isetnefret,
who appears to have died not long before the arrival of the Hittite
princess. Four daughters of Rameses also held the title, Henutmira,
long believed to have been his sister rather than a daughter, Bintanat,
Merytamun, and Nebettawy. These were the most exalted among the
king's daughters, of whom there were at least forty in addition to some
forty-five sons. Many of them appear in long processions on the walls
of the great temples built by their father, who was to outlive several of
his children. They were buried one after the other in a gigantic tomb in
the Valley of the Kings (KV 5), which has recently been rediscovered. It
resembles the underground galleries that Rameses started to build at
Saqqara for the burial of the sacred Apis bulls of the god Ptah, which
had until then been placed in separate tombs.
During his long years on the throne, Rameses II carried out a vast
building programme. He began by adding a great peristyle courtyard
and pylon to the temple of Amun in Luxor, built by Amenhotep III and
completed by the last 18th-Dynasty kings. The courtyard was planned
at a curious angle to the rest of the temple, presumably in order to
create a straight line across the river to the site of the king's mortuary
temple, the Ramesseum, in much the same way as his father had done
with the Great Hypostyle Hall at Karnak and his Abd el-Qurna temple
on the west bank. Rameses also built a temple for Osiris at Abydos,
smaller than his father's, but equally beautiful. During the rest of his
reign he gradually filled the country with his temples and statues,
many of which he usurped from earlier rulers; there is hardly a site in
Egypt where his cartouches are not found on the monuments. Particu-
larly impressive is the astonishing series of eight rock temples in
Lower Nubia, including two at Abu Simbel, most of which must have
been built with a work force rounded up from among the local tribes,
as is attested in the case of Wadi es-Sebua, built for the king by Setau,
the viceroy of Nubia, after he had held a razzia in year 44.
Among the hundreds of statues of deities and kings that Rameses
usurped, those erected by Amenhotep III, the last king before the
Amarna Period, were particularly favoured, as were those made by the
kings of the i2th Dynasty, the great rulers of the classical period of
292 JACOBUS VAN DIJK
Egyptian history that served as a model for the new Egypt now taking
shape, after the radical break in the tradition constituted by the Amarna
Period. The same reflection on a great past is also evident from a
renewed interest in the classical writers of the Old and Middle king-
doms, especially the 'teachings' or 'instructions' of old sages such as
Ptahhotep and Kagemni, and descriptions of chaos such as those of
Neferti and Ipuwer. It was perhaps because Ramessid scribes felt that
these earlier works could not be equalled, let alone surpassed, that
contemporary literature, such as love poetry and folk tales and myth-
ical stories that sprang from an oral tradition, was written not in
classical Egyptian, but in the modern language first introduced in
inscriptions by Akhenaten.
Rameses II was also the king who expanded the city of Avaris
and made it his great Delta residence called Piramesse ('house of
Rameses'), the Raamses of biblical tradition. Its location has long been
disputed, but it has now been established beyond reasonable doubt
that it is to be identified with the extensive remains at Tell el-Dab c a and
Qantir in the eastern Delta. The city was strategically situated near the
road leading to the border fortress of Sile and the provinces in
Palestine and Syria and also along the Pelusiac branch of the Nile, and
it soon became the most important international trade centre and
military base in the country. Asiatic influence had always been strong
in the area, but now many foreign deities such as Ba e al, Reshep,
Hauron, Anat, and Astarte, to mention only a few, were worshipped in
Piramesse. Many foreigners lived in the city, some of whom became
high-ranking officials. One office that was more often than not held by
foreigners was that of'royal butler', a senior executive position outside
the normal bureaucratic hierarchy, the holder of which was often
entrusted with special royal commissions. As a result of the peace
treaty with the Hittites, specialist craftsmen sent by Egypt's former
enemy were employed in the armoury workshops of Piramesse to
teach the Egyptians their latest weapons technology, including the
manufacture of much sought-after Hittite shields. Indeed, by this date
the Egyptian army itself counted among its ranks many foreigners
who had come to Egypt as prisoners of war and had subsequently been
incorporated into the country's combat forces.
Many of Rameses' high officials lived and worked in Piramesse, but
most of them appear to have been buried elsewhere, particularly in the
necropolis of Memphis. About thirty-five tombs of the Ramessid Period
have so far been excavated there, some of them very large. These tombs
still took the form of an Egyptian temple, but, compared to the tombs
THE AMARNA PERIOD AND LATER NEW KINGDOM 293
of the late i8th Dynasty, the workmanship had declined. The earlier
tombs had walls built of solid mud-brick masonry with a limestone
revetment set against the interior faces, but now the walls consisted
entirely of a double row of limestone orthostats with the space in
between filled with rubble, and the same technique was used for their
pylons and pyramids. In addition, the quality of the limestone itself
was often not very good, and, rather than carefully making the blocks
fit against each other, a liberal amount of plaster was used to fill the
gaps between the blocks. Nor do the reliefs carved on them compare
favourably with those in the older tombs in the cemetery. This general
decline in the quality of the workmanship can be observed throughout
the country, even in the king's own temples; of the two main relief-
sculpting techniques, the superior, but more time-consuming and
more expensive, raised relief all but disappeared after the first years of
the reign, in favour of the common sunk relief. Generally speaking,
Rameses' monuments impress more by their size and quantity than by
their delicacy and perfection.
Rameses II was the first king since Amenhotep III to celebrate more
than one sed-festival. The first took place in year 30 and then another
thirteen followed, at first at more or less regular intervals of about
three years, and then, towards the end of his long life, annually.
Amenhotep III had become deified during his three jubilees, but in
this respect Rameses had less patience than his great predecessor, for
already by his eighth year we hear of a colossal statue being carved
which was given the name 'Rameses-the-god'. Colossal statues of the
king with similar names were set up in front of the pylons and by the
doorways of all the great temples and these received a regular cult as
well as being objects of public worship for the inhabitants of the towns
in which they stood. Inside the temples, Rameses-the-god had his own
cult-image and processional bark along with the other deities to whom
they were dedicated; in reliefs Rameses II is often shown offering to
his own deified self.
Among the king's many sons who held high positions, the second
son of Queen Isetnefret, Khaemwaset, must be singled out. He was
high priest of Ptah in Memphis and acquired a reputation as a scholar
and magician that would survive until Roman times. No other son
of Rameses II left so many monuments and many of these were
inscribed with learned, sometimes archaic, texts. Although, as we have
seen, the reign of Rameses II saw a marked revival of classical tradi-
tions, Khaemwaset must clearly have had a special interest in Egypt's
glorious past, for he also restored several pyramids of Old Kingdom
294 JACOBUS VAN DIJK
pharaohs in the Memphite necropolis, and in some of his own monu-
ments tried to copy the style of Old Kingdom tomb reliefs. As high
priest of Ptah, one of his duties was to see to the burial of the sacred
Apis bull and it is to Khaemwaset that the first galleries (rather than
individual tombs) of the Serapeum are due. He also travelled the
length and breadth of the country in order to announce his father's
first five sed-festivals, which were traditionally proclaimed from Mem-
phis. By year 52 of his father's reign, Khaemwaset was the eldest
surviving son and therefore became crown prince, but at that stage he
must have been in his sixties already, and he died a few years later,
around year 55. He was almost certainly buried in the Memphite
necropolis and not in the princely gallery tomb in the Valley of the
Kings (KV 5), but whether he was really interred in the Serapeum, as
many believe, is less certain.
After Khaemwaset's death Rameses II lived on for another twelve
years until he finally died in the sixty-seventh year of his reign, the
longest reigning monarch since Pepy I (2321-2287 BC) of the 6th
Dynasty. During the last years of his reign he had become a living
legend and he was clearly much admired (and envied) by his suc-
cessors. His memory would live on in later traditions both under his
own name and under that of Sesostris, in reality the name of several
Middle Kingdom rulers whose monuments he had so avidly usurped.
His twelve eldest sons had died before him, and it was Merenptah
(1213-1203 BC), the fourth son of Isetnefret and crown prince since the
death of Khaemwaset, who eventually succeeded him.
Rameses H's Successors
During the first years of his reign Merenptah, who must have been
fairly advanced in years already, sent several military expeditions
abroad, not only to Nubia, but also into Palestine, where he subdued
the rebellious vassals of Ashkelon, Gezer, and Yenoam; the 'victory
stele' that records these victories also contains the first reference in
Egyptian sources to Israel, albeit not as a country or city, but as a tribe.
The major event of Merenptah's reign occurred in his year 5, however,
and the victory stele really deals with this: a campaign against the
Libyans. They had been a problem even during his father's and grand-
father's reigns, but the fortresses Rameses II had built along the
western borders of the Delta were obviously unable to prevent the
invasion of a massive coalition of Libyan and other tribes led by their
THE AMARNA PERIOD AND LATER NEW KINGDOM 295
king, Mereye.
The previous decades had seen a great migration in the Aegean and
Ionian world that had probably been caused by widespread crop failure
and famine. According to a long inscription at Karnak (between the
Seventh Pylon and the central part of the temple), Merenptah had
actually sent grain to the starving Hittites, still Egypt's ally in the East.
Many important centres of Mycenaean Greece had been violently
destroyed and the western fringes of the Hittite empire had begun to
collapse. The marauding 'Sea Peoples', as they were soon to be called
in Egypt, had also reached the coast of North Africa between Cyrenaica
and Mersa Matruh, which in the Late Bronze Age was seasonally
occupied by foreign seafarers sailing from Cyprus via Crete to the
Egyptian Delta. In this area, the Sea Peoples joined the Libyan tribes
and with a force of some 16,000 men marched on Egypt; since they
brought their women and children with them, as well as cattle and
other belongings, they were obviously planning to settle in Egypt. They
had actually penetrated the western Delta and were moving south-
wards, threatening Memphis and Heliopolis, when Merenptah con-
fronted them and, in a battle that lasted for six hours, managed to
defeat them. The Libyans were destined to fail on this occasion
because, as Merenptah says on his victory stele, their king, Mereye,
had already been 'found guilty of his crimes' by the divine tribunal of
Heliopolis, and the god Arum, who presided over the tribunal, had
personally handed the sword of victory to his son Merenptah, making
the battle nothing less than a 'holy war'. Thousands of enemies were
killed, but great numbers were also captured and settled in military
colonies, especially in the Delta, where their descendants would
become an increasingly important political factor (see Chapter 12).
The rest of Merenptah's reign appears to have been peaceful, and
the king used it to build at least two temples and a palace in Memphis.
He must have realized that he did not have many years left, however,
for his mortuary temple on the Theban West Bank is constructed
almost exclusively from blocks removed from earlier structures, par-
ticularly the nearby temples of Amenhotep III. He died in his ninth
year. After his death, trouble over the succession broke out, for,
although the next king, Sety II (1200-1194 BC), was almost certainly
the eldest son of Merenptah, a rival king, Amenmessu, ruled for a few
years, at least in the south of the country. When exactly this happened
is still the subject of much controversy; it has been suggested that
Amenmessu deposed Sety II for some time between the latter's years 3
and 5, but others have the trouble set in at the beginning of the reign.
296 JACOBUS VAN DIJK
Whatever the truth may be, Sety ruthlessly erased and usurped all of
Amenmessu's cartouches and later texts refer to the rival ruler as 'the
enemy'.
When Sety II died, after a reign of almost six full years, his only son,
Saptah (1194-1188 BC), succeeded him. However, Saptah was not a son
of Sety's principal queen, Tausret (1188-1186 BC); instead he had been
born to him by a Syrian concubine called Sutailja. More importantly,
he was only a young boy who suffered from an atrophied leg caused by
poliomyelitis; his stepmother, Tausret, therefore remained 'great royal
wife' and acted as regent. She was not the only power behind the
throne, however, for a powerful official called Bay, described as the
'chancellor of the entire land', who was himself a Syrian, appears to
have been the true ruler of the country at this date. He is depicted
several times with Saptah and Tausret and in some inscriptions he
even claims that it was he who 'established the king on the throne of
his father', an extraordinary phrase normally reserved for the gods.
When Saptah died in his sixth regnal year Tausret reigned on as
sole ruler for another two years, doubtless with the support of Bay.
After Hatshepsut and Nefertiti she was the third queen of the New
Kingdom to rule as pharaoh. With her death the i9th Dynasty came to
an end.
Rameses III and the 2Oth Dynasty
How the next dynasty gained power remains unclear. The only indi-
cations of the political events at this date derive from a stele erected on
the island of Elephantine by its first ruler, Sethnakht (1186-1184 BC )>
and an account written down in the Great Harris Papyrus from the
beginning of the reign of Rameses IV (1153-1147 BC), some thirty years
later. On the stele, Sethnakht relates how he expelled rebels who on
their flight left behind the gold, silver, and copper they had stolen from
Egypt and with which they had wanted to hire reinforcements among
the Asiatics. The papyrus describes how a state of lawlessness and
chaos had broken out in Egypt because offerees from 'outside'; after
several years in which there was no one who ruled, a Syrian called Irsu
(a made-up name meaning 'one who made himself—that is, 'upstart')
seized power, and his confederates plundered the country; they treated
the gods like ordinary human beings and no longer sacrificed in the
temples, a description that resembles the one given of the Amarna
Period in the years of the Restoration. The gods then chose Sethnakht
to be the next ruler, just as they had Horemheb at the end of the i8th
THE AMARNA PERIOD AND LATER NEW KINGDOM 297
Dynasty, and he re-established order.
From these texts we may perhaps conclude that, after the death of
Tausret, Bay had tried to seize power and may even have succeeded for
a brief time until he was expelled by Sethnakht. The date of the
Elephantine stele is not Sethnakht's regnal year i, as one might expect
on a victory stele, but year 2, and this date is not given at the beginning
of the text, as was customary on stelae, but towards the end. It has,
therefore, been suggested that it represents the date of Sethnakht's
victory and at the same time the true date of his accession, having
counted in retrospect the time it took him to overcome his adversaries
as his first year. Be that as it may, he did not enjoy his newly gained
kingship for long, for he died soon afterwards and was succeeded by
his son Rameses III (1184-1153 BC).
Although the new king inherited peace and stability from his father,
he soon had his fair share of troubles as well. In year 5 he had to fight
off further advances by Libyan tribes, who had used the period of
internal struggle to penetrate into the western Delta as far as the
central Nile branch. By this time the Egyptians appear to have accepted
this peaceful immigration as inevitable, but, when a revolt against the
pharaoh broke out because he interfered in the succession of their
'king', Rameses III quickly responded and brought them back under
Egyptian control. A further Libyan campaign took place in year n. Far
more challenging, however, was the great battle against the Sea
Peoples in year 8.
Since the days of Merenptah, when some of the Sea Peoples had
first tried to enter Egypt from the west, their movements had turned
the whole of the Middle East upside down. They had destroyed the
Hittite capital Hattusas and swept away their whole empire; they had
conquered Tarsus and many of them had settled in the plains of Cilicia
and northern Syria, razing Alalakh and Ugarit to the ground. Cyprus
had also been overwhelmed and its capital Enkomi ransacked. Clearly
their ultimate goal was Egypt, however, and in year 8 of Rameses III
they launched a combined land and sea attack on the Delta. But the
Egyptians were well aware of the imminent danger and had moved a
large defence force to Djahy (southern Palestine, perhaps the Egyptian
garrisons in the Gaza strip) and fortified the mouths of the Nile
branches in the Delta. When the assault finally came, Rameses' troops
were well prepared for it and were able to beat the invaders back.
Although the Sea Peoples had changed the east Mediterranean world
for good, they never succeeded in conquering Egypt and their presence
in Syria-Palestine does not at first seem to have affected Egypt's sway
298 JACOBUS VAN DIJK
over its northern territories.
At home, Rameses III spent a lot of time and energy on his building
projects, foremost of which was his large mortuary temple at Medinet
Habu, begun shortly after his accession and finished by year 12; it still
stands today as one of the best preserved temples of the New Kingdom
(the decoration on its exterior walls including scenes from the battle
with the Sea Peoples). It was closely modelled on the Ramesseum of
his great predecessor Rameses II, whom Rameses III tried to emulate
in many other ways; his own royal names were all but identical to those
of Rameses II and he even named his sons after the latter's numerous
offspring. The building of Medinet Habu and other projects, including
the expansion of Piramesse, do not appear to have been hampered by
the various threats to Egypt's borders. We also hear of a major expedi-
tion to Punt, perhaps the first since the famous venture to that remote
land in the days of Hatshepsut, and another one to Atika, perhaps the
copper mines of Timna.
All was not well in Egypt, however. The period of turmoil before
Rameses' accession had led to corruption and various abuses, and he
was forced to inspect and reorganize the various temples throughout
the country. The Great Harris Papyrus enumerates the huge donations
of land he made to the most important temples in Thebes, Memphis,
and Heliopolis, and to a lesser extent to many smaller institutions as
well. By the end of his reign a third of the cultivable land was owned by
the temples and of this three-quarters belonged to Amun of Thebes.
This development upset the balance between temple and state and
between the king and the ever more powerful priesthood of Amun. An
overall loss of control over the state finances and economic crisis were
the result; grain prices soared and the monthly rations to the workmen
at Deir el-Medina, which had to be paid by the state treasury, were soon
in arrears, leading in year 29 to the first recorded organized strikes in
history. Things were made worse by repeated raids by groups of Libyan
nomads in the Theban area, which created a general sense of
insecurity.
This gradual breakdown of the centralized state may well have been
one of the reasons behind an attempt on the life of Rameses III, or, if it
was not, the general unrest and insecurity may at least have given the
conspirators the idea that they could count on general support if they
succeeded. The plot originated in the king's harem, presumably in
Piramesse, where one of the officials involved, the scribe of the harem,
Pairy, had a house. He was just one of several harem officials impli-
cated; the ringleaders were one of Rameses' wives, called Tiy, and
THE AMARNA PERIOD AND LATER NEW KINGDOM 299
some other women from the harem, as well as several royal butlers and
a steward; all of them were 'stirring up the people and inciting enmity
in order to make rebellion against their lord'. The ultimate goal was to
put Tiy's son Pentaweret on the throne instead of the king's lawful
heir. Apparently the plan was to murder the king during the annual
Opet Festival in Thebes, but included in the preparations were also
magical spells and wax figurines, which were smuggled into the
harem. The plot must have failed, however, for the king's mummy
shows no signs of a violent death, and his crown prince, Rameses IV,
and not Pentaweret eventually succeeded him. When all of this hap-
pened we do not know, but the records of the court hearings and the
sentences passed on 'the great criminals' (most of them were forced to
commit suicide) were written down at the beginning of the reign of
Rameses IV, who also compiled the Great Harris Papyrus, which con-
tains his father's 'testament', suggesting that the assassination
attempt took place towards the end of Rameses Ill's thirty-one-year
reign.
Rameses IV
All of the remaining 2 oth-Dynasty kings were called Rameses, a name
they adopted at their accession, adding it to their birth-name. They
were probably all related to Rameses III, although in some cases we do
not know exactly how. During their reigns, Egypt lost control over its
territories in Syria-Palestine and the importance of Nubia was rapidly
declining as well. Apart from the temple of Khonsu in Karnak, no
major temples were built even by those Ramessid rulers who reigned
long enough to do so. Rameses IV was the fifth son of his father and
had become crown prince around the latter's regnal year 22, after four
older brothers had died. The sons of Rameses III were not buried in a
gallery tomb in the Valley of the Kings like those of Rameses II, but in
separate tombs in the Valley of the Queens. Judging by the name of his
mother, Rameses Ill's Great Royal Consort Isis-Ta-Habadjilat, the
new king must have had at least some foreign blood running through
his veins. At the beginning of his reign he embarked on several build-
ing projects, especially his royal tomb and mortuary temple at Thebes,
for which he doubled the workforce of Deir el-Medina to 120 men.
Probably in connection with these projects he mounted several expedi-
tions to the quarries of the Wadi Hammamat, where little activity had
taken place since the days of Sety I, as well as to the turquoise and
copper mines in Sinai and Timna. None of his building plans came to
300 JACOBUS VAN DIJK
fruition, however, for he died after a reign of five (perhaps seven) years,
before he could complete any of them, despite his prayers on a large
stele in Abydos asking Osiris to grant him a reign twice as long as the
sixty-seven years of Rameses II.
During Rameses IV's reign, further delays in the delivery of basic
commodities at Deir el-Medina occurred; at the same time the influ-
ence of the high priest of Amun was growing. Ramesesnakht, holder
of that high office, was soon accompanying the state officials when
they went to pay the men their monthly rations, indicating that the
temple of Amun, not the state, was now at least partly responsible for
their wages. The highest state and temple offices were in fact in the
hands of the members of two families. Thus Ramesesnakht's son
Usermaatranakht was 'steward of the estate of Amun' and as such
administered the land owned by the temple, but he also controlled the
vast majority of the state-owned land in Middle Egypt. The holders of
the offices of'second and third priest' and of'god's father of Amun'
were all related to Ramesesnakht by marriage. This well illustrates the
marked tendency of these high positions, including that of high priest
itself, to become hereditary, and Ramesesnakht himself was to be
succeeded by two of his sons. The office became more and more
independent and the king had only nominal control over who was
appointed high priest.
The Final Reigns of the zoth Dynasty
Rameses IV was succeeded by his son, who became Rameses V (1147-
1143 BC) upon his accession. A major crime and corruption scandal
among the priesthood at Elephantine, which had in fact evolved during
the reign of his father, is the main event known from his reign,
although he also continued the latter's mining activities in Timna and
Sinai. After four years, Rameses V died of smallpox at a young age.
The next king, Rameses VI (1143-1136 BC), was a younger son of
Rameses III. He usurped the royal tomb and mortuary temple begun
by his nephew, whose burial had therefore to be delayed until an
alternative tomb had been found for him in Rameses VI's year 2. It
has, therefore, been concluded by some researchers that the succes-
sion was accompanied by civil unrest, especially as there are some
entries in a necropolis journal that state that the workmen of Deir el-
Medina, whose numbers were soon afterwards reduced to sixty again,
stayed at home 'for fear of the enemy'. This does not seem very
probable, however, although the mere fact that most officials remained
THE AMARNA PERIOD AND LATER NEW KINGDOM 301
in office from one reign to the next is hardly enough proof to the
contrary, for the same had been the case at the end of the i8th and iQth
Dynasties, when there had certainly been troubles. The 'enemy' men-
tioned in the journal is more likely to have been a group of Libyans,
who continued to be a nuisance in the area. Rameses VI reigned for
seven years; he is the last king whose name is attested in Sinai. During
the seven-year reign of Rameses VII (1136-1129 BC), grain prices
soared to their highest level, after which they gradually came down
again. His successor Rameses VIII was probably yet another son of
Rameses III, which might explain why his reign was so brief.
The exact family background of the last three Ramessid rulers is
unknown. The eighteen years or so of the reign of Rameses IX (1126-
1108 BC) were marked by increasing instability. In regnal years 8-15 we
regularly hear of Libyan nomads disturbing the peace in Thebes, and
there were also strikes again. It is, therefore, hardly surprising that this
period witnessed the first wave of tomb-robberies, known from a
whole series of papyri that record the trials of the thieves who had been
apprehended. However, the tombs in the Valley of the Kings were not
involved; in fact only one i7th-Dynasty royal burial in Dra Abu el-Naga
and a number of private tombs were robbed, and various thefts from
temples were also investigated. At the beginning of the reign,
Ramesesnakht (the high priest of Amun mentioned above) had died;
he was succeeded as high priest firstly by his son Nesamun, and then
by the latter's brother Amenhotep. In two reliefs at Karnak Amenhotep
had himself depicted on the same scale as Rameses IX, a fair indica-
tion of the virtual equality that now appears to have existed between the
king and the high priest of Amun. One of these scenes commemorates
an event in year 10, when Rameses rewarded Amenhotep for his
services to king and country with the traditional 'gold of honour'. The
many gifts bestowed upon him on this occasion must certainly have
been very impressive, but their quantities are nevertheless a revealing
illustration of the state of the economy, or at least of the king's wealth.
Among the gifts received by Amenhotep were 2 hin of a costly oint-
ment; some 200 years earlier, during the reign of Horemheb, one of
Maya's subordinates, a mere scribe of the treasury, had contributed
4 hin of the same ointment to the burial goods of his master.
Almost nothing is known about the reign of Rameses X, which
seems to have lasted for nine years. Rameses XI (1099-1069 BC), on
the other hand, ruled for thirty years, although certainly during the last
ten years the geographical extent of his power was virtually reduced to
Lower Egypt (that is, the Delta). During his reign, the crisis that had
302 JACOBUS VAN DIJK
gripped the Theban area in the previous decades deepened even
further: persistent trouble with Libyan gangs preventing the workmen
on the west bank from going to work, famine (the 'year of the hyenas'),
further tomb robberies and thefts from temples and palaces, and even
civil war. At some point, in or before year 12, Panehsy, the viceroy of
Nubia, appeared in Thebes with Nubian troops to restore law and
order, perhaps at the request of Rameses XI himself. In order to feed
his men in a city that was already suffering from economic malaise, he
was given, or perhaps usurped, the office of'overseer of the granaries'.
This must have brought him into conflict with Amenhotep, the high
priest of Amun, whose temple owned the bulk of the land and its
produce. The conflict quickly escalated and during a period of eight or
nine months (sometime between years 17 and 19) Panehsy and his
troops actually besieged the high priest at Medinet Habu. Amenhotep
then appealed to Rameses XI for help and this resulted in a civil war.
Panehsy marched north, reaching at least as far as Hardai in Middle
Egypt, which he ransacked, but probably actually pushing much
further north, until he was eventually driven back by the king's army,
which was almost certainly led by a general called Piankh. Eventually
Panehsy had to withdraw to Nubia, where trouble persisted for many
years, and where he was eventually buried.
In Thebes, General Piankh took over the titles of Panehsy as well as
styling himself vizier, and after the death of Amenhotep, who may or
may not have survived Panehsy's assault, he also became high priest of
Amun, uniting the three highest offices of the country in one person.
With Piankh's military coup begins the period of the wehem mesut, the
'renaissance', a term that had also been used by kings at the beginning
of the 12th and i9th Dynasties to indicate that the country had been
'reborn' after a period of chaos. In the Theban area documents were
now dated in years of the 'renaissance' rather than regnal years of the
king. Years i to 10 of the renaissance were identical with regnal years
19 to 28 of Rameses XI. After the death of Piankh, his son-in-law
Herihor took over all his functions, and after the death of Rameses XI
the former even assumed royal titles. In the north of the country
Smendes (1069-1043 BC) mounted the throne, and with these two
men the 2ist Dynasty begins.
After Rameses III the Egyptians finally lost their provinces in
Palestine and Syria, which after the invasion of the Sea Peoples and the
disappearance of the Hittite empire had broken up into several small
states. Problems in the north had been made worse by the gradual
sanding-up of the harbour of Piramesse owing to the slow but inexor-
THE AMARNA PERIOD AND LATER NEW KINGDOM 303
able eastward shift of the Pelusiac branch of the Nile. Nor did the kings
of the 2Oth Dynasty any longer have the power and the resources to
mount major expeditions to the gold mines in Nubia. Towards the end
of the dynasty the treasury of the temple of Amun sent some small-
scale expeditions to the Eastern Desert in search of gold and minerals,
but the quantities with which they came back were small. During the
years of the renaissance, Piankh and his successors, assisted by the
descendants of the workmen of Deir el-Medina who were now living at
Medinet Habu, began to tap a different source of gold and precious
stones: the very same tombs in the Valley of the Kings that their fathers
and grandfathers had carved and decorated, as well as many other
tombs both royal and private in the Theban necropolis. Over the next
century and later, the tombs were gradually despoiled of their gold and
other valuables; eventually they would be emptied out completely, and
even the mummies of the great pharaohs of the New Kingdom would
be unwrapped and stripped of their precious amulets and other trap-
pings and reburied together in an anonymous tomb in the Theban
cliffs. By some strange irony only two royal mummies would escape
this fate: that of Tutankhamun (in KV 62) and that of his father,
Akhenaten, the 'enemy of Akhetaten' (in KV 55).
The Historical and Social Repercussions of the Amarna and
Ramessid Periods
There can be no doubt that the great kings of the Ramessid Period
were immensely powerful rulers. Even Rameses XI was obviously still
able to mobilize an army that was strong enough to repel his oppo-
nent's troops all the way back to Nubia. And yet it is undeniable that
royal prestige had gradually eroded in the course of the iQth and 2Oth
Dynasties. As we have seen, political and economic developments,
which had led to the breakdown of the central government and the
concentration of ever more power in the hands of the high priests of
Amun, greatly contributed to this erosion. On the other hand, these
developments may themselves be seen as the result, or at least the
symptoms, of a much more fundamental change. At the root of this
change is yet again the Amarna Period.
Akhenaten had tried to remake society and had failed, even though
he had initially enjoyed the support of the army. What was worse,
however, is that in the eyes of all but a few of the Amarna elite he had
actually wrecked society. We have already seen how burial customs
after the Amarna Period reflect a totally different attitude towards the
304 JACOBUS VAN DIJK
king, as a reaction against the way Akhenaten had tried to monopolize
the funerary beliefs of his subjects. This monopoly was not limited to
life in the hereafter, however, but also deeply affected life on earth.
Traditionally, access to the god's cult image in the temple was
restricted to the king and the professional priesthood representing
him; for the vast majority of the population the only means of getting
in contact with the gods of their home town, without the intervention
of the state or the official temple cult, was during regularly held pro-
cessions, when the images of the gods were carried from one temple to
another on the occasion of a religious festival. These festivals, which
were quite frequent, were public holidays, and they played an enor-
mously important part in people's religious and social lives. Most
Egyptians had a strong emotional bond with their native town and its
god, the 'city-god', to whom they showed a life-long loyalty. The city-
god was also the god of the local necropolis, the 'lord of the burial' who
granted 'a goodly burial after old age' to his loyal servants.
Akhenaten had not only banned all gods other than the Aten and
abolished the daily rituals in their temples, but with them he had also
put an end to the festivals with their processions, and in doing so he
had undermined the social identity of his subjects. Instead, he had
claimed all devotion and loyalty for himself and the prosperity of the
country and the happiness of its inhabitants depended on him alone.
He was the 'city-god' not just of Akhetaten, but of the whole country,
and his daily chariot ride along the royal road at Amarna replaced the
divine processions. The history of the i8th Dynasty before the Amarna
Period had seen a clear development towards a more personal relation-
ship between the various deities and their worshippers. This develop-
ment came to a sudden halt when Akhenaten proclaimed a god who
could only be worshipped by his son, the king, whereas all individual,
personal devotion had to be diverted to the king himself. This total
usurpation of personal piety had seriously compromised the credibil-
ity of the dogma of divine kingship.
In the period after Amarna, the balance between god and king
underwent a dramatic change. The king lost for good the central posi-
tion he had occupied in the lives of his subjects; instead, the god
now acquired many traditional aspects of kingship. In the traditional
representative theocracy, the gods embodied the cosmic order that
they had created at the beginning of time, while the king, as their
intermediary, represented the gods upon earth, maintained cosmic
order by means of the temple ritual, and carried out their will by his
government. Only very rarely did the gods reveal themselves directly,
THE AMARNA PERIOD AND LATER NEW KINGDOM 305
and, when they did, they did so to the king.
After the Amarna Period, the problem of the unity and plurality of
the gods, which Akhenaten had tried to solve by denying the existence
of all but one sole god, was solved in a different way: Amun-Ra became
the universal, transcendent god, who existed far away, independent of
his creation; the other gods and goddesses were aspects of him, they
were his immanent manifestations. This situation is elegantly
expressed in a collection of hymns to Amun (preserved in a papyrus
now in Leiden), according to which Amun 'began manifesting himself
when nothing existed, yet the world was not empty of him in the
beginning'. This universal god was now the true king, and, although
the pharaoh's traditional titles—which were rooted in mythology and
express his divinity—did not change, he had in actual fact become
more human than ever before in the history of Egypt. The fact that Ay,
Horemheb, Rameses I, and even Sety I had all been commoners before
they mounted the throne may have had something to do with the speed
with which this change took place. The representative theocracy had
become a direct theocracy: no longer was the king the god's divine rep-
resentative upon earth who carried out his will; rather, the god revealed
his will directly to every human being and intervened directly in the
events of everyday life and in the course of history.
The new transcendent god had at the same time become a personal
god whose will determined the fate of the country and of the indi-
vidual. Texts express this by bridging the gap between the opposites of
being far away and yet nearby: Tar away he is as one who sees, near he
is as one who hears.' Amun-Ra looked down upon his worshippers
from afar, but at the same time he was near because he heard their
prayers and revealed himself in their lives by the manifestation of his
will, by his divine intervention.
This new form of religious experience, usually called 'personal
piety', was wholly characteristic of the Ramessid Period, although its
beginnings, suppressed by Akhenaten, went back to the mid-i8th
Dynasty. Penitential psalms, inscribed on votive stelae and ostraca by
literate members of the ordinary population, were one form in which
this piety was expressed. When an individual had commited a sin,
divine intervention could mean divine retribution, particularly when
this sin had gone undetected and unpunished by a human court of
justice. These penitential hymns attributed illness (often blindness,
although this word is probably used in a metaphorical sense) to the
state of being guilty of a hidden sin, which once revealed in the text on
a votive stele was no longer hidden, so that god would 'return' to his
306 JACOBUS VAN D I J K
worshipper and make him 'see' again. It was not only the individuals
who could sin, but also the country as a whole. In a text of this type
inscribed on a Theban tomb wall (TT 139) at the end of the Amarna
Period, Amun is begged to return, and in Tutankhamun's Restoration
Stele the gods are also said to have abandoned Egypt.
Another type of votive stele demonstrates that God was also thought
to be able to intervene positively in the life of his worshipper—for
example, by saving him from a crocodile or making him survive the
sting of a scorpion or the bite of a snake. Many gods received specially
made stelae or other objects as a thanksgiving for saving their worship-
pers; there is even a special god Shed, whose name means 'saviour',
and who, probably not by chance, appears for the first time at Amarna,
possibly in spite of official repression. Some people even went one step
further and put their whole lives in the hands of their personal patron
god or goddess, even to the extent of assigning all their possessions to
his or her temple.
Even the king might appeal to his god in his hour of need. When all
seemed lost and Rameses II was about to be captured or even killed by
his Hittite enemies at the Battle of Qadesh, he called out to his god
Amun, and the arrival of the king's support force at the critical
moment was interpreted as proof of the god's personal intervention.
This shows clearly that the king no longer represented god on earth,
but was subordinate to him; just like all other human beings, he was
subject to the will of god, even though in traditional mythological
terms he was still viewed as the divine pharaoh and on his monuments
this aspect would continue to be emphasized. Clearly the divide
between theological dogma and everyday reality had widened con-
siderably.
Once it had been recognized that god's will was the governing factor
in everything that happened, it became mandatory to know his will in
advance. Oracles, which had originally been consulted only by the
king, perhaps as early as the Old Kingdom (and which had during the
18th Dynasty been used to seek the god's approval of a king's accession
or a major trade or military expedition), began to be used in the
Ramessid Period to consult the god on all sorts of affairs in the lives of
ordinary human beings. Priests would carry the portable bark with the
god's image in procession out of the temple and a piece of papyrus or
an ostracon bearing a written question would be laid before him; the
god would then indicate his approval or disapproval by making the
priests move slightly forwards or backwards or by some other motion
of the bark. Appointments, disputes over property, accusations of
THE AMARNA PERIOD AND LATER NEW KINGDOM 307
crimes, and later even questions seeking the god's reassurance that
one would safely live on in the hereafter, were thus subjected to the
god's will.
All of these developments further minimalized the role of the king
as god's representative on earth; the king was no longer a god, but god
himself had become king. Once Amun had been recognized as the
true king, the political power of the earthly rulers could be reduced to a
minimum and transferred to Amun's priesthood. The mummies of
their royal ancestors were no longer considered the erstwhile incarna-
tions of god on earth, and so, with few scruples, their tombs could be
robbed and their bodies unwrapped.
II
Egypt and the Outside World
IAN SHAW
From the earliest times, expeditions concerned with trade, quarrying,
and warfare brought the Egyptians into repeated contact with foreign-
ers. The regions with which Egypt gradually fostered commercial and
political links can be grouped into three basic areas: Africa (primarily
Nubia, Libya, and Punt), Asia (Syria-Palestine, Mesopotamia, Arabia,
and Anatolia) and the northern and eastern Mediterranean (Cyprus,
Crete, the Sea Peoples, and the Greeks).
The Egyptians' African neighbours to the south included, over the
course of time, a number of different ethnic groups in Nubia (pri-
marily the A Group, the C Group, the Kerma civilization, the pan-grave
culture, the kingdom of Kush, the Ballana culture, and the Blemmyes),
and Ethiopia (the pre-Axumite cultures and the Axum civilization),
while to the north-east, beyond the Sinai peninsula, they encountered
many towns and villages in the hills and coastal plain of the Levant
(and, further to the north and east, a constantly changing mosaic of
kingdoms and empires in Anatolia and Mesopotamia). To the west, in
the Sahara, they came into contact with several different peoples
whom we now group under the general term 'Libyans'. Little archaeo-
logical evidence has survived concerning the latter, although it is
usually assumed on the basis of textual references that they were
nomadic or at least dependent on forms of pastoralism for their sub-
sistence, and it is only when they become part of Egyptian society in
the late New Kingdom and the Third Intermediate Period that aspects
of their culture can be glimpsed or hypothesized (see Chapter 12).
EGYPT AND THE OUTSIDE WORLD 309
The Racial and Ethnic Identity of the Egyptians
There are a number of different ways in which we can define the
ancient Egyptians themselves as a distinct racial and ethnic group, but
the question of their roots and their sense of their own identity has
provoked considerable debate. Linguistically, they belonged to the
Afro-Asiatic (Hamito-Semitic) family, but this is simply another way of
saying that, as their geographical position implies, their language had
some similarities to contemporary languages both in parts of Africa
and in the Near East.
Anthropological studies suggest that the predynastic population
included a mixture of racial types (negroid, Mediterranean, and Euro-
pean), but it is the question of the skeletal evidence at the beginning of
the pharaonic period that has proved to be most controversial over the
years. Whereas the anthropological evidence from this date was once
interpreted, by Bryan Emery and others, as the rapid conquest of Egypt
by people from the east whose remains were racially distinct from the
indigenous Egyptians, it is now argued by some scholars that there
may have been a much slower period of demographic change, prob-
ably involving the gradual infiltration of a different physical type from
Syria-Palestine, via the eastern Delta.
The iconography of the Egyptians' depictions of foreigners suggests
that for much of their history they saw themselves as midway between
the black Africans and the paler Asiatics. It is also clear, however, that
neither Nubian nor Syro-Palestinian origins were regarded as particu-
larly disadvantageous factors in terms of individuals' status or career
prospects, particularly in the cosmopolitan climate of the New King-
dom, when Asiatic religious cults and technological developments
were particularly widely accepted. Thus the demonstrably negroid
features of the high official Maiherpri did not prevent him from attain-
ing the special privilege of a burial in the Valley of the Kings at about
the time of Thutmose III (1479-1425 BC). In the same way, a man
called Aper-el, whose name indicates his Near Eastern roots, rose to
the rank of vizier (the highest civil office below that of the king himself)
in the late i8th Dynasty.
The Iconography of Warfare and Conquest: Textual and Visual
Evidence
The term 'Nine Bows' was frequently used to refer to the enemies of
Egypt, the specific identity of whom varied from one time to another,
310 IAN SHAW
although they usually included Asiatics and Nubians. They were
generally symbolized by depictions of rows of bows or bound captives,
the number of which varies, and the motif often decorated such royal
items as sandals, footstools, and daises, so that the pharaoh could sym-
bolically tread on his enemies. The depiction of nine bound captives
surmounted by a jackal, on the seal of the necropolis of the Valley of
the Kings, was evidently intended to protect the tomb from the
depredations of foreigners and other sources of evil.
Depictions of bound foreign captives frequently feature in Egyptian
art. Various prestige items of the late Predynastic and Early Dynastic
periods (such as the Narmer Palette) include scenes in which the king
dispatches or humiliates bound foreigners. The scene of the smiting
pharaoh is not only one of the most enduring aspects of pharaonic art
(appearing on temple pylons as late as the Roman Period) but also one
of the first recognizable icons of kingship, the earliest known instance
being a sketchy depiction painted on the wall of the late predynastic
Tomb 100 at Hierakonpolis in the late fourth millennium BC.
The excavations of the 5th- and 6th-Dynasty pyramid complexes of
Raneferef, Nyeuserra, Djedkara, Unas, Teti, Pepy I, and Pepy II at
Saqqara and Abusir have yielded a large number of statues of foreign
captives, which may perhaps have been placed in rows along the sides
of the causeway leading from the valley temple to the pyramid temple.
At a slightly later date, representations of bound captives were used in
cursing rituals, as in the case of five early i2th-Dynasty alabaster cap-
tive figures (now in the Egyptian Museum, Cairo) inscribed with
hieratic execration texts comprising lists of the names of Nubian
princes accompanied by insults.
Throughout the pharaonic and Graeco- Roman periods the depic-
tion of the bound captive was a popular theme in the decoration of
temples and palaces. The inclusion of bound captives in the decoration
of aspects of the fittings and furniture of royal palaces served to
reinforce the pharaoh's total suppression of foreigners and probably
also symbolized the elements of 'unrule' that the gods required the
king to control. There are, therefore, a number of depictions in Graeco-
Roman temples showing lines of gods capturing birds, wild animals,
and foreigners in clapnets.
The rekhyt-bird (a type of lapwing or plover with a distinctive crested
head) was often used as a symbol for foreign captives or subject
peoples, probably because, with its wings pinioned behind its back, it
roughly resembled the hieroglyph for a bound captive. The first depic-
tion of this bird is attested in the upper register of relief decoration on
EGYPT AND THE OUTSIDE WORLD 311
the late predynastic Scorpion Macehead (£.3100 BC), comprising a row
of lapwings hanging by their necks from ropes attached to the standards
representing early Lower Egyptian provinces. In this context the rekhyt
appears to be representing the conquered peoples of northern Egypt
during the crucial period when the country was transformed into a
single unified state. In the 3rd Dynasty (2686-2613 BC), however,
another row of lapwings were depicted in the familar pinioned form,
alongside the Nine Bows, crushed under the feet of a stone statue of
Djoser from his Step Pyramid at Saqqara. From that point onwards
there was a continual ambiguity in the symbolic meaning of the birds
(to modern eyes at least), since they could, in different contexts, be
taken to refer either to the enemies of Egypt or to the loyal subjects of
the pharaoh.
Where did the Outside World Begin?
The traditional physical borders of Egypt—the Western and Eastern
deserts, Sinai, the Mediterranean coast and the Nile cataracts south of
Aswan—were sufficient to protect Egypt's independence for thou-
sands of years. But perhaps the most intriguing issue in the geography
of ancient Egypt—especially with regard to attitudes to Africa and
Asia—is the question of the slowly changing Egyptian conception of
where the outside world began. To what extent, for instance, were
those areas outside the Nile Valley (but within the borders of modern
Egypt), particularly the Eastern Desert and the Sinai peninsula,
regarded as 'non-Egyptian' territory?
The Egyptians used two words to refer to a border: djer (an eternal
and universal limit) and task (an actual geographical frontier, which
might be set by people or deities). The latter was, therefore, essentially
movable, and all pharaohs were in theory entrusted with the responsi-
bility of'extending the borders' of Egypt, given that their royal names
and titles implied a potentially infinite area of political domination.
The furthest extent of the actual borders was evidently established in
the reign of Thutmose III in the i8th Dynasty, when triumphal stelae
were erected at the River Euphrates in Asia and at Kurgus (between the
fourth and fifth cataracts) in Nubia.
In the Early Dynastic Period and the Old Kingdom, the border with
Lower Nubia traditionally lay at Aswan, the modern name of which
derives from the ancient Egyptian word swenet ('trade'), clearly indi-
cating the commercial opportunities afforded by its location. The first
cataract, just a short distance to the south, represented a substantial
3 I2
IAN SHAW
obstacle to Nile boats, therefore all trade goods had to be carried along
the bank; this land route to the east of the Nile was protected by a huge
mud-brick wall, almost 7.5 km. long, probably primarily a i2th-Dynasty
construction.
By the i2th Dynasty, however, the border with Nubia lay much
further to the south, at the Semna gorge, the narrowest part of the Nile
Map of Egypt and the Ancient Near East, showing links and trade routes
EGYPT AND THE OUTSIDE WORLD 313
Valley. It was here, at this strategic location, that the i2th Dynasty
pharaohs built a cluster of four mud-brick fortresses, Semna, Kumma,
Semna South, and Uronarti. Several 'boundary stelae' erected by
Senusret III in the fortresses of Semna and Uronarti spelled out the
Egyptians' complete control over the region, including regulations
concerning the ability of Nubians to trade along the Nile Valley (see
p. 106).
From at least the beginning of the i2th Dynasty, the border with
Palestine in the eastern Delta was also defended by a line of fortresses,
known as the 'Walls of the Ruler' (inebu heka), and, at about the same
time, a fortress seems to have been established in Wadi Natrun in
order to protect the western Delta from the 'Libyans'. This policy was
maintained throughout the Middle Kingdom, and a number of new
fortresses were built in the New Kingdom, including the easterly sites
of Tell Abu Safa, Tell el-Farama, Tell el-Heir, and Tell el-Maskhuta,
and the westerly sites of el-Alamein and Zawiyet Umm el-Rakham.
Material Evidence of Early Contacts with Asia and Nubia
The evidence for commercial and diplomatic links between the emerg-
ing state of Egypt and its various neighbouring cultures and states
often survives in the form of exotic raw materials and products, as well
as the vessels in which they were carried. Although Egypt was clearly
self-sufficient in a wide diversity of rocks, plants, and animals, there
were nevertheless many much-prized materials that were not obtain-
able within the Nile Valley itself. Turquoise could be obtained only
from Sinai; silver probably from Anatolia or the north Mediterranean,
via the Levant; copper from Nubia, Sinai, and the Eastern Desert; and
gold from the Eastern Desert and Nubia, while such fine woods as
cedar, juniper, and ebony, as well as products such as incense and
myrrh, had to be imported from western Asia and tropical Africa.
One of the most well-travelled and sought-after materials was lapis
lazuli, a deep blue stone, streaked with glistening pyrite and calcite,
which was known to the Egyptians as khesbed. It was used for jewellery,
amulets, and figurines from at least as early as the Naqada II Period
(0.3500-3200 BC), but the principal ancient source seems to have been
located at Badakhshan in north-eastern Afghanistan (some 4,000 km.
from Egypt), where four ancient quarries have so far been identified:
Sar-i-Sang, Chilmak, Shaga-Darra-i-Robat-i-Paskaran, and Stromby.
Badakhshan lay at the centre of a wide commercial network through
which lapis lazuli was exported over vast distances to the early
314 IAN SHAW
civilizations of western Asia and north-east Africa, no doubt passing
through the hands of many middle men en route.
Some of the most important archaeological data for the earliest
Egyptian links with the outside world take the form of the pottery
vessels in which many commodities (usually food, drink, or cosmetics)
were transported to and from the Nile valley. The cache of about 400
Palestinian-style vessels that filled one chamber of Tomb U-j, in the
Naqada III Cemetery U at Abydos (see Chapter 4), shows that this elite
tomb-owner in ^.3200 BC—perhaps an early ruler—was able to exert
considerable commercial influence in order to obtain these grave
goods (probably wine jars). Very few of these vessels have been identi-
fied with pottery from contemporary sites in Palestine; therefore they
seem to have been types produced specifically for export. The same
tomb also contained Palestinian-style wavy-handled Egyptian vessels.
Another tomb (11-127) yielded fragments of ivory handles carved with
images apparently depicting rows of Asiatic captives and women
carrying pottery vessels.
Pottery found at early urban sites in southern Palestine itself sug-
gests that an Egyptian trade network may have been flourishing in this
region as early as the first phase of the Early Bronze Age. It has been
suggested that the expansion of the Naqada culture into the Delta
region in the late Predynastic Period may well have resulted from the
Upper Egyptian rulers' desire to gain direct commercial contact with
Palestine, rather than obtaining goods via the middlemen of Maadi
and other Lower Egyptian sites. By at least the ist Dynasty, the newly
unified Egyptian state had expanded beyond the Delta into southern
Palestine, with a thriving trade route passing through several hundred
encampments and way stations along the northern end of the Sinai
peninsula (see Chapter 4). Several of the Early Dynastic royal tombs at
Abydos contained fragments of Palestinian vessels, showing that the
rulers of Egypt included imported Asiatic commodities among their
funerary equipment.
At about the same time that Egyptians were first establishing com-
mercial links with the inhabitants of Early Bronze Age Palestine, they
were also making contact with the people of Lower Nubia (primarily in
order to gain access to the exotic products of tropical Africa, as well
as the mineral resources of Nubia itself). The archaeological traces of
these people, whom George Reisner named the 'A Group', have sur-
vived throughout Lower Nubia, dating from about 3500 to 2800 BC. The
grave goods often include stone vessels, amulets, and copper artefacts
imported from Egypt, which not only help to date these graves but also
EGYPT AND THE OUTSIDE WORLD 315
demonstrate that the A Group was engaged in regular trade with the
Egyptians of the Predynastic and Early Dynastic periods. Bruce
Williams has made the controversial suggestion that the chiefdoms of
the early A Group were actually reponsible for the rise of the Egyptian
state, but this has been refuted by most scholars (see Chapter 4).
The wealth and quantity of imported items appear to increase in
later A-Group graves, suggesting a steady growth in contact between
the two cultures. Sites such as Khor Daoud (comprising no settlement
remains but hundreds of silos containing Naqada-culture pottery
vessels that originally held beer, wine, oil, and perhaps cheese) were
evidently trading posts at which exchange of goods took place between
the late predynastic Egyptians, the A Group, and the nomads of the
Eastern Desert. Judging from some of the rich tombs at the Sayala and
Qustul cemeteries, which contain prestige goods imported from
Egypt, the elite within the A Group were able to profit substantially
from their role as middlemen in the African trade route. However, a
rock carving at the Lower Nubian site of Gebel Sheikh Suleiman (now
on display in the Khartoum Museum) appears to record a ist-Dynasty
campaign as far south as the second cataract, suggesting that contacts
with the A Group had by this date become somewhat more militaristic.
A process of severe impoverishment appears to have taken place in
Lower Nubia during the ist Dynasty, probably as a direct result of the
depredations of early Egyptian economic exploitation of the region. It
has been suggested that there might have been an enforced reversion
to pastoralism (perhaps partly due to environmental changes) or that
the local Nubian population might even have temporarily abandoned
the region, perhaps moving south and eventually returning as the so-
called C Group (once regarded as quite separate from the A Group, but
now seen to have a number of cultural features in common).
The C-Group people were roughly synchronous with the period in
Egyptian history from the mid-6th Dynasty to the early i8th Dynasty
(0.2300-1500 BC). Their principal archaeological characteristics inclu-
ded hand-made black-topped pottery vessels bearing incised decora-
tion filled with white pigment, as well as artefacts imported from
Egypt. Their way of life seems to have been dominated by cattle-
herding, while their social system was probably essentially tribal (until
they began to be integrated into Egyptian society). In the early i2th
Dynasty their territory in Lower Nubia was taken over by the
Egyptians, perhaps partly in order to prevent them from developing
contacts with the more sophisticated Kerma culture that had emerged
in Upper Nubia (see Chapter 8).
316 IAN SHAW
The Kingdom of Punt
Egyptian contacts with Africa gradually extended even further than
Lower and Upper Nubia, bringing them into contact with a region in
East Africa that they describe as Punt. Trading missions were sent
there from at least the 5th Dynasty (2494-2345 BC) onwards in order to
Map of north-east Africa during the pharaonic period, showing Nubian sites and
(inset) the probable location of the land of Punt
EGYPT AND THE OUTSIDE WORLD 317
obtain such products as gold, aromatic resins, African blackwood,
ebony, ivory, slaves, and wild animals (for example, monkeys and
cyncocephalous baboons). By the New Kingdom, such expeditions
were being depicted in temples and tombs, showing the inhabitants of
Punt as a people with a dark-reddish complexion and fine features;
they were shown with long hair in the earlier paintings, but from the
late i8th Dynasty onwards they had evidently adopted a more close-
cropped style. The last definite indications of expeditions to Punt date
to the time of the 2oth-Dynasty ruler Rameses III.
There is still some debate regarding the precise location of Punt,
which was once identified with the region of modern Somalia. A
strong argument has now been made for its location in either southern
Sudan or the Eritrean region of Ethiopia, where the indigenous plants
and animals equate most closely with those depicted in the Egyptian
reliefs and paintings.
It used to be assumed (primarily on the basis of the scenes at Deir
el-Bahri depicting Hatshepsut's expedition to Punt in the mid-18th
Dynasty) that the trading parties travelled by sea from the ports of
Quseir or Mersa Gawasis, but it now seems likely that at least some of
the Egyptian traders sailed south along the Nile and then took an over-
land route to Punt, perhaps making contact with the Puntites in the
vicinity of Kurgus, at the fifth cataract.
The Deir el-Bahri scenes include depictions of the unusual Puntite
settlements, comprising conical reed-built huts set on poles above the
ground, and entered via ladders. Among the surrounding vegetation
are palms and myrrh trees, some of the latter already in the process of
being hacked apart in order to extract the myrrh. The scenes also show
myrrh trees being loaded onto the ships so that the Egyptians could
produce their own aromatics from them (and it has been argued that
this in itself may be an argument for the combined Nile-overland
route from Punt to Egypt, given the fact that such plants might well
have died during the more difficult voyage northwards along the Red
Sea coast). These myrrh trees might even have been replanted in the
temple at Deir el-Bahri itself, judging from the surviving traces of tree
pits there.
'Imperialism' in the Middle and New Kingdoms
In the Middle and New kingdoms, Egypt gradually obtained a degree
of economic control over the regions of Nubia and Syria-Palestine.
Opinions differ, however, as to which of these territories can be said to
318 IAN SHAW
have been politically or socially 'colonized', or whether the situation
was much more erratic, perhaps characterized only by periodic raids
designed to safeguard trade routes and provide supplies of war booty.
Debate also centres on the question of the possible motivations for
ancient imperialism: were the Egyptian inroads into Nubia and the
Levant dictated by ideological imperatives, by economic necessity, or
by some other socio-political factor?
In practice, the answers to these questions are by no means straight-
forward and, not surprisingly, vary according to the specific place and
period. In the Middle Kingdom, for instance, the situation is in some
respects clearer: as far as Nubia is concerned, we know that the i2th-
Dynasty pharaohs used military force to control the region as far south
as the third cataract at least, building a chain of fortresses that would
have given them a stranglehold over Nile commerce. The fortresses
contained garrisons and extensive storerooms that would not only
have ensured a continuous military presence in Lower Nubia, but
would also have provided the potential to send campaigns further
south, when necessary to oppose any perceived or actual threat.
The enormous amount of space devoted to granaries at such fort-
resses as Askut, together with the traces of buildings interpreted by
Barry Kemp as royal 'campaign palaces' at Uronarti and Kor, all sug-
gest the use of the Lower Nubian fortresses as a i2th-Dynasty spring-
board into Africa rather than just a heavily defended border. The
storage space in the fortresses was also no doubt utilized to store the
materials and products imported by the Egyptians, while they were
en route for Thebes or Itjtawy.
In Palestine, however, there is very little evidence for any permanent
Egyptian presence during the Middle Kingdom. There were certainly
contacts both with the Levant and the Aegean during the i2th and i3th
Dynasties, but it remains unclear to what extent Egypt gained political
or economic control over any parts of the eastern Mediterranean. A
fragment of Amenemhat II's annals preserved at Memphis records at
least two invasions of the Levant during his reign, and the stele of
Khusobek (at Manchester Museum) records an expedition launched
against the Palestinian city of Shechem in the reign of Senusret III.
Apart from these references, however, the only other indications of
military designs on the Levant are to be found in elite epithets and
titles (which may well be primarily bombastic rather than historical),
or in descriptions of produce obtained from western Asia (which tend
not to specify whether the goods or livestock were obtained by force). A
reasonable archaeological case, however, can be made for a fairly
EGYPT AND THE OUTSIDE WORLD 319
strong and continuous Middle Kingdom Egyptian economic presence
in Palestine and Byblos (see below), probably periodically reinforced
by military pressure. The increasingly high numbers of Asiatics
known to have been living in Egypt during the Middle Kingdom (see
Chapter 7) suggests that at least some were being brought in as
prisoners of war.
Egyptian activities in the Levant during the New Kingdom are
attested in some detail both by archaeological and textual sources. The
latter consist not only of triumphal Egyptian 'victory stelae' and temple
reliefs, presenting glowing accounts of the spoils obtained by the king
on behalf of the gods, but also clay cuneiform tablets from a number of
sites (for example, Ta'anach, Kamid el-Loz, and Hattusas), document-
ing the diplomatic, administrative, and economic links between the
various states of the Near East. From an Egyptian point of view, the
most important of these 'archives' is a set of 382 tablets found at
Amarna in Middle Egypt, consisting mainly of correspondence between
foreign leaders and the Egyptian king in the mid-i4th century BC (the
late 18th Dynasty). The 'Amarna Letters' thus provide insights first
into the diplomatic relations between Egypt and the other great powers
(for example, Mitanni and Babylon) and, secondly, into the labyrinth-
ine politics of the small city states of Syria-Palestine, disputing and
allying among themselves as they slipped back and forth between the
spheres of influence of Mitanni, Egypt, and the Hittite kingdom.
The principal debate concerning Egyptian involvement in Syria-
Palestine during the New Kingdom centres on the question of the
degree to which Egypt maintained a permanent military and/or civilian
presence at the various towns and cities that they had conquered.
Some scholars argue that there is sufficient archaeological and textual
evidence to suggest that Egypt had effectively colonized some of the
towns of Palestine at least (perhaps initially inheriting the control of
this region when they pursued the defeated Hyksos into their home-
land at the end of the Second Intermediate Period (see Chapters 8 and
9)). According to this theory—primarily based on the Amarna Letters
and the presence of Egyptian artefacts at many Levantine sites—the
whole area of Syria-Palestine was divided up into three zones (north-
south: Amurru, Upe, and Canaan), each ruled by an Egyptian governor
via a number of small garrisons scattered among the local settlements.
Other scholars, however, argue that the material culture of Egyptian
sites in the eastern Delta is so clearly distinct from that of the nearest
towns in Palestine, just on the other side of Sinai, that it seems highly
unlikely that there were ever many Egyptians actually living among the
320  IAN SHAW
local populations (in contrast to the extensive architectural and arte-
facrual evidence of Egyptians colonizing Nubia in the New Kingdom).
The motivation for the significant New Kingdom Egyptian presence
in Lower Nubia may well have been primarily economic, but it has
been pointed out by a number of scholars that the archaeological and
textual evidence actually amount to a very complex web of information
concerning Egyptian attitudes to Nubia. To begin with, there is the
continuation, throughout the Middle and New kingdoms of the essen-
tially xenophobic ideology described above, whereby stereotypical bar-
baric Nubians were portrayed in official art and literature as worthless
representatives of chaos. This has to be contrasted, however, with two
important factors: first, that many foreigners (including Nubians and
Asiatics) were living happily alongside native Egyptians in many of the
towns in Egypt proper, and, secondly, that there is good evidence of a
deliberate New Kingdom policy of acculturation both in Nubia and the
Levant, so that the local elite were encouraged to adopt Egyptian cus-
toms and nomenclature, and their children were sometimes forcibly
removed to be 'educated' in Egypt, eventually returning to their home
countries fully indoctrinated with the Egyptian way of life.
The overall image of Egyptian 'imperialism', therefore, is multi-
faceted, the economic and political pragmatism of the pharaohs often
being cloaked in the hyperbole of royal rhetoric and piety. The debate
concerning ideology versus economics is difficult to resolve because
we rely primarily on a combination of royal religious and funerary texts
for our reconstruction of Egyptian behaviour in the outside world, yet
the real story probably lies in the more prosaic archival material that
has so rarely survived.
Byblos
The town of Byblos (or Jubeil) was located on the coast of Canaan
(about 40 km. north of modern Beirut). The principal settlement,
known in the Akkadian language as Gubla, has a long history extend-
ing from the Neolithic to the late Bronze Age, when the population
appears to have moved to a nearby site now covered by a modern
village. The importance of Byblos lay in its function as a port, and from
around the time of Egypt's unification it was used by the Egyptians as a
source of timber. The famous cedars of Lebanon, and other goods,
passed through it, and Egyptian objects are found there from as early
as the 2nd Dynasty (2890-2686 BC). The site included several
religious buildings, such as the so-called Obelisk Temple, dedicated
EGYPT AND THE OUTSIDE WORLD 321
to Ba'alat Gebal, the 'Lady of Byblos' (a local form of Astarte, who was
also identified with the Egyptian goddess Hathor), where one of the
obelisks was inscribed with hieroglyphs.
Egyptian culture of the Middle Kingdom had an especially strong
influence on the court of the Middle Bronze Age rulers of Byblos, and
among the objects found in the royal tombs of this period are several
bearing the names of the late i2th-Dynasty rulers Amenemhat III and
IV. Egyptian objects included ivory, ebony, and gold, while local imita-
tions used other materials and were executed in a less accomplished
style.
In the New Kingdom, the city features prominently in the Amarna
Letters, since its ruler, Ribaddi, sought military assistance from the
Egyptian pharaoh. On this occasion Byblos fell into enemy hands, but
was later regained. A sarcophagus found with objects of Rameses II
(1279-1213 BC) and showing Egyptian influence is important for its
later (tenth century BC) inscription for Ahiram, a local ruler, which is in
early alphabetic characters. Various Egyptian artefacts found at Byblos
itself attesting to strong royal diplomatic contacts between the pharaohs
and the rulers of Byblos include a vessel bearing the name of Rameses
II from the tomb of the above-mentioned Ahiram, inscribed door
jambs of Rameses II from a temple, and fragments of statues of
Osorkon I and II (the Osorkon I statue bearing a Phoenician inscrip-
tion and dating to the reign of Abibaal).
The archaeological evidence therefore suggests a peak of Egypto-
Byblian contact in the i9th Dynasty, followed by decline in the 2Oth
and 2ist Dynasties (documented by the Tale of Wenamun, a quasi-
historical description of a late 2 oth-Dynasty expedition to Byblos), and
finally a resurgence of links in the 22nd and 23rd Dynasties. After the
Third Intermediate Period, the importance of Byblos gradually appears
to have declined in favour of the neighbouring ports of Tyre and Sidon.
The Sea Peoples
In the i3th and i2th centuries BC, a series of major crop failures in the
northern and eastern Mediterranean appear to have triggered off large-
scale migrations through Anatolia and the Levant. These agricultural
problems evidently led the Egyptian 19th-Dynasty ruler Merenptah to
send grain to the famine-stricken Hittites (now in decline), and many
Mycenaean urban centres appear to have been destroyed at this date.
Among the new migrants in the Mediterranean region at this date
were a loose confederation of ethnic groups from the Aegean and Asia
322  IAN SHAW
Minor, known to the Egyptians as Sea Peoples. Some of these groups,
such as the Denen, Lukka, and Sherden, were already active by the
reign of Akhenaten (1352-1336 BC), while members of the Lukka,
Sherden, and Peleset are portrayed as mercenaries fighting for the
army of Rameses II (1279-1213 BC) at the Battle of Qadesh.
Later in the Ramessid Period, the Sea Peoples are described and
depicted on reliefs at Medinet Habu and Karnak as well as the Great
Harris Papyrus, a list of temple endowments in the reign of Rameses
III (1184-1153 BC). The latter sources indicate that the Sea Peoples were
not simply engaged in random acts of plundering but were part of a
significant movement of displaced peoples migrating into Syria-
Palestine and Egypt. It is clear that they planned to settle in the areas
that they attacked, since they are portrayed not merely as armies of
warriors but also as whole families bringing their possessions with
them in ox-drawn carts. Study of the 'tribal' names recorded by the
Egyptians and Hittites has shown that various groups of the Sea
Peoples can be linked either with particular homelands, or at least with
the places in which they finally settled. Thus, the Ekwesh and Denen
may possibly be correlated with the Achaean and Danaean Greeks of
the Iliad, the Lukka may have derived from the Lycian region of
Anatolia, the Sherden may have been connected with Sardinia, and the
Peleset are almost certainly to be identified with the biblical Philistines
(who gave their name to Palestine).
The Sea People's first attack on the Egyptian Delta, in alliance with
the Libyans, dates to the fifth year of the reign of Merenptah (1213-
1203 BC). The individual groups of Sea Peoples (in addition to the
Libyan Meshwesh) are named as the Ekwesh, Lukka, Shekelesh,
Sherden, and Teresh. According to Merenptah's reliefs on one of the
walls of the temple of Amun at Karnak and the text of a stele from his
funerary temple (the so-called Israel Stele), he successfully repelled
them, killing 6,000 and routing the rest. Moshe Dothan's excavations
at the Philistine city of Ashdod in 1962-9 uncovered a burnt layer
dating to the thirteenth century BC, which may perhaps correspond
either to the Levantine campaign of the pharaoh Merenptah or the
arrival of the Peleset themselves.
From an Egyptian point of view, the final confrontation with the Sea
Peoples took place in the eighth year of Rameses Ill's reign, by which
time the Sea Peoples had probably captured the Syrian cities of Ugarit
and Alalakh. They attacked Egypt by both land and sea, the latter con-
frontation being depicted in the celebrated sea-battle reliefs on the
external walls of Rameses' mortuary temple at Medinet Habu. This
EGYPT AND THE OUTSIDE WORLD 323
victory protected Egypt from overt invasion from the north, but ulti-
mately it was to be the more insidious infiltration of Libyan peoples
from the west that was more successful as a means of gaining control
of Egypt (see Chapter 12).
Conclusion
The history of Egypt's contact with the outside world is above all con-
cerned with power and prestige. In the earliest commercial links
between the Egyptians and their neighbours in Africa and the Near
East, the principal motivation appears to have been to obtain rare or
exotic materials and products that could serve to bolster the power base
of the individuals or groups concerned. Trade, whether inter-regional
or international, was an integral part of the formation and expansion of
the early Near Eastern states.
By the time that the full national administrative apparatus was
operating, in the Middle and New kingdoms, there were large sectors
of royal bureaucracy and military power dedicated solely to the process
of obtaining taxes and conscripted labour from the provinces of Egypt.
This efficient national economic system formed the ideal basis for the
process of exacting tribute (inu) and spoils from the lands outside
Egypt's borders. Both ideologically and economically, the acts of con-
quering and ruling were inseparable from the idea of absorbing new
wealth into the estates of the king and the major religious cults.
It was not, however, simply a question of importing materials and
commodities into Egypt. There also appears to have been a steady
influx of people, as well as linguistic and cultural influences, leading to
the creation of a distinctly cosmopolitan and multicultural society
from at least the New Kingdom onwards. The apparent tolerance of
foreigners within Egyptian society was nevertheless accompanied by a
tremendous continuity in terms of the core values and beliefs of the
indigenous population (so far as we can tell, given the bias of surviving
data towards the elite end of society). Egyptian culture was apparently
strong and flexible enough to survive long periods of Libyan, Kushite,
Persian, and Ptolemaic domination without the essence of the Egypt-
ians' identity as a nation being affected.
12
The Third Intermediate Period
(1069-664 BC)
JOHN TAYLOR
This 4<DO-year period, comprising the 2ist to 25th Dynasties (1069-
664 BC), may justly be regarded as marking a new phase in Egypt's
history. The period is characterized by significant changes in Egypt's
political organization, society, and culture. Centralized government
was replaced by political fragmentation and the re-emergence of local
centres of power; a substantial influx of non-Egyptians (Libyans and
Nubians) permanently modified the profile of the population, while
Egypt as a whole became more inward-looking, its contacts with the
outside world (and its impact on the Levant in particular) greatly
reduced in scale. These, and other, factors had important conse-
quences for the functioning of the economy, for the structure of society,
and for the religious attitudes and funerary practices of the inhabit-
ants. It is true that the period was marked by tensions over control of
territory and resources, leading on occasion to conflicts, but violence
was not endemic; the period as a whole was stable and represents far
more than a temporary lapse from traditional pharonic rule (an unfor-
tunate implication of the customary designation 'Intermediate'). Many
of the events and trends of these years were permanent in their effect
and played a crucial role in shaping the Egypt of the later first millen-
nium BC.
A sound historical framework for these centuries has proved more
difficult to establish than for any other major period of Egyptian
history. No pharaonic king-lists include the 2ist-25th Dynasties, and
the Egyptologist is thus forced to rely more heavily than is strictly
THE THIRD I N T E R M E D I A T E P E R I O D 325
desirable on the often garbled excerpts from the history of Manetho
(itself derived chiefly from Delta-based sources and thus offering, at
best, an incomplete picture). Careful collation of the Manethonian lists
with the scattered inscriptions of kings and local dignitaries of the
period and cross-references to Near Eastern sources has yielded a
chronology that is accepted in its main points by most scholars, but
some areas remain subjects for debate (notably the relationships and
spheres of influence of some of the provincial rulers who adopted royal
status during the late ninth and eighth centuries BC). With the excep-
tion of sites such as Tanis, the survival of evidence from this period in
the Delta is, as always, relatively poor, and, while Thebes has yielded a
very large quantity of artefacts, private statuary and funerary equip-
ment tend to predominate, whereas economic sources such as admin-
istrative papyri are very rare. Since it was in the north that many of the
most significant developments were taking place at this time, a
balanced picture of the country as a whole is difficult to achieve.
Historical Outline
The Third Intermediate Period was inaugurated by a major political
upheaval and a weakening of the economy. The civil war fomented by
Panehsy, the viceroy of Kush, shook the country, and his subsequent
defeat and expulsion beyond the southern frontier amounted to only a
partial victory for the government. Military action against Panehsy by
General Piankh failed to re-establish Egyptian authority in Nubia, and
control over the resources of the southlands—the gold mines and the
lucrative trade in the products of sub-Saharan Africa—was lost. Hence,
at the very outset of the period, Egypt suffered a serious reduction in
revenue from its former dependencies; as the Tale of Wenamun (a
narrative describing an expedition supposedly sent to Byblos by
Herihor) hints, the new Egyptian rulers may also have lacked the
prestige in the Levant that their predecessors had enjoyed.
Following the death of Rameses XI, 0.1069 BC > me 2om Dynasty—
and with it the Renaissance era—came to an end, but the foundations
of a new power structure were already in place, and transition to a new
regime occurred smoothly. Under the 2ist Dynasty Egypt was—to
outward appearances—politically united, but in reality control was
divided between a line of kings in the north and a sequence of army
commanders, who also held the post of high priest of Amun, at
Thebes. Smendes (1069-1043 BC), an influential figure of unknown
origin, founded the dynasty in the north, with his power base at the
326 JOHN TAYLOR
eastern Delta site of Tanis, a new city whose chief monuments were
constructed largely of reused materials brought from Piramesse and
other northern sites. Tentamun, probably Smendes' probable wife, is
thought to have been a member of the Ramessid royal family. While
this connection may have been a factor in the new ruler's rise to power,
the growing influence of the cult of Amun and its officials was
undoubtedly also significant. During this period, the government of
Egypt was in effect a theocracy, supreme political authority being
vested in the god Amun himself. In a hymn to Amun on a papyrus
from Deir el-Bahri, which has been dubbed the 'credo of the theocracy',
the god's name is written in a cartouche and he is addressed as the
Plan of the remains of the temples and tombs at the eastern Delta of This
THE THIRD INTERMEDIATE P E R I O D 327
superior of all the gods, the fountainhead of creation, and the true king
of Egypt. The pharaohs were now merely temporal rulers who were
held to be Amun's appointees and to whom the god's decisions were
communicated via oracles. The workings of the theocratic government
are explicitly documented at Thebes, where oracular consultations
were formalized by the institution of a regular Festival of the Divine
Audience, held at Karnak. The same principles are implied for the
north as well—Smendes and Tentamun are described in Wenamun
as 'the pillars which Amun has set up for the north of his land', while
the city of Tanis was developed as a northern counterpart to Thebes,
Amun's principal cult centre. Temples to the Theban triad were erected
there and Tanis's role as a holy city was enhanced by the siting of the
tombs of the 2ist-Dynasty kings within the temple precinct. To what
extent Tanis was really a political power base at this time may be
questioned, since excavations have so far revealed no dwellings, private
monuments (other than a few reused blocks from courtiers' tombs), or
donation stelae (that is, records of the bestowal of cultivable land on
the gods of local temples) in the area. There is evidence, however, that
Memphis functioned as a residence of the northern kings—a decree of
Smendes is recorded as having been issued there—and the ancient
city may once more have served as a major administrative base.
The activities of the northern rulers during the 2ist Dynasty are
poorly documented. Building works at Tanis and Memphis by Psusen-
nes I (1039-991 BC) and Siamun (978-959 BC) are the most promi-
nent remains within Egypt itself, and relations with the Levant seem to
have been sporadic and unadventurous. The marriage of a royal prin-
cess (perhaps a daughter of Siamun) to Solomon of Israel is a striking
testimony to the reduced prestige of Egypt's rulers on the world stage.
At the height of the New Kingdom, pharaohs regularly took to wife the
daughters of Near Eastern princes, but refused to permit their own
daughters to be married off to foreign rulers.
The most prominent of the southern commanders at the inception
of the Third Intermediate Period was the chief general Herihor.
Through his assumption of the office of high priest of Amun—and
even, on occasion, the titles and trappings of a pharaoh—supreme
civil, military, and religious authority was combined in the hands of a
single individual. However, it was to the family of Herihor's colleague
General Piankh that long-term control of Upper Egypt subsequently
passed. All of these men held the offices of chief general and high
priest of Amun. Under the auspices of the theocracy they derived their
executive powers from oracles of Amun, Mut, and Khons, by whom
328 JOHN TAYLOR
clerical appointments and major policy decisions of the rulers were
sanctioned. Although the temporal authority of the Tanite kings was
formally recognized throughout the whole of Egypt, and the Theban
commanders made only limited pretensions to royal status, it was
none the less they who were in effective control of Middle and Upper
Egypt. A formal frontier between the two regions was fixed at Teudjoi
(el-Hiba), south of the entrance to the Faiyum. Here, and at other sites
along the Nile, the southern rulers erected a series of fortresses. Other-
wise the principal activity documented in the south during the 2ist
Dynasty was the systematic dismantling of the New Kingdom royal
burials in the Theban necropolis. The Valley of the Kings ceased to be
the royal burial ground, the tomb-builders' community of Deir el-
Medina was disbanded, the contents of the tombs were appropriated
and the mummies secreted in caches.
After the reigns of Smendes and his successor Amenemnisu
(1043-1039 BC), the throne in the north passed to Psusennes I, son of
the Theban commander Pinudjem I, and control of Upper Egypt to his
brother Menkheperra. Thus for a time the same Theban line governed
all Egypt, and amicable relations between the north and the south were
maintained through the intermarriage of several members of the
rulers' extensive families. Yet the division of the realm persisted—an
indication that decentralization was tolerated by these rulers. About
984 BC a new family took control in the Delta, with the accession of
Osorkon the Elder (984-978 BC), the son of the Chief of the Meshwesh
Sheshonq, a ruler whose name and parentage proclaim his Libyan
origins. The Theban commanders dropped all claims to royal status,
and more openly made use of the names and date lines of the northern
monarchs in documents. Nevertheless, the Theban high priest
Psusennes ultimately became king in the north as Psusennes II
(959-945 BC), last ruler of the 2ist Dynasty.
By this time, Libyans constituted a substantial and influential pres-
ence in Egypt. Although major incursions of Meshwesh and Libu had
been repulsed by Merenptah and Rameses III, the settlement of immi-
grants, war captives, and garrison troops continued, particularly in the
Delta and in the area between Memphis and Herakleopolis; it has been
suggested that by the end of the New Kingdom the Egyptian army
was almost entirely made up of Libyan mercenaries. The incipient
decentralization of government during the 2ist Dynasty facilitated the
growth of provincial power-bases, and local dynasties of Libyan chief-
tains, descended from the settlers of the late New Kingdom, were able
to increase their autonomy; the ruling families in both north and south
THE THIRD I N T E R M E D I A T E P E R I O D 329
during the 2ist Dynasty included individuals who bore patently Libyan
names—and since some form of acculturation was doubtless practised
(see below), many more are probably disguised in the record under
Egyptian names. It was, therefore, only the culmination of an estab-
lished trend when, at the end of the 2ist Dynasty, the throne in Tanis
passed to the Chief of the Meshwesh Sheshonq (King Sheshonq I
(945-924 BC)). He belonged to a family settled at Bubastis, whose
members had, through judicious marriages with the royal family and
links with the high priests at Memphis, become highly influential in
the Delta. The transfer of power from Psusennes II appears to have
been accomplished with a minimum of opposition—it was undoubt-
edly eased by the fact that Sheshonq was the nephew of the earlier
Tanite king Osorkon the Elder, while his own son, the future Osorkon
I (924-889 BC), was married to Psusennes II's daughter Maatkara.
Sheshonq's reign (945-924 BC) stands out as a high point in the
Third Intermediate Period. Rejecting the internal divisions of the
2ist Dynasty in favour of New Kingdom models of pharaonic rule,
Sheshonq sought to re-establish the political authority of the king. The
theocracy continued to function but in a modified form—oracular
consultations still occurred, but no longer feature as a regular instru-
ment of policy. The new reign was marked by a change in the attitude
of the throne towards the integrity of the country, the adoption of an
expansionist foreign policy, and an ambitious royal building pro-
gramme.
The attempt to exert direct royal control over the whole of Egypt
involved curtailing the virtually independent status of Thebes. To
achieve this, the post of high priest of Amun was handed to one of
Sheshonq's sons, Prince luput, who was also army commander—a
policy followed by subsequent pharaohs. Other members of the royal
family and supporters of the dynasty were also appointed to important
offices, and loyalty on the part of local power-holders was encouraged
by marriages to daughters of the royal house.
After more than a century of passivity on the part of Egyptian rulers,
Sheshonq I intervened aggressively in the politics of the Levant to
reassert pharaonic prestige there. His Karnak inscriptions record a
major military expedition £.925 BC against Israel and Judah and the
principal towns of southern Palestine, including Gaza and Megiddo.
The Old Testament records the same event, stating (i Kgs. 14: 25—6)
that, in the fifth year of Rehoboam, 'Shishak, king of Egypt' seized the
treasures of Jerusalem, and adding (2 Chr. 12: 2-9) that he came with
"1,200 chariots and an army that included Libyans and Nubians. These
330 JOHN TAYLOR
sources indicate that the campaign was launched in support of Jero-
boam, an exile in Egypt who claimed the throne of Judah. However, if
this was meant to be the first stage of a programme to re-establish
Egyptian authority in Palestine, it remained only a flash in the pan.
Sheshonq died soon after his return to Egypt and under his successors
relations with the Levant appear to have reverted to purely commercial
contacts—notably the reopening of relations with Byblos. Sheshonq
Fs building programme included plans for a great court in the temple
of Amun at Karnak, but this remained unfinished at the king's death.
The gateway, known as the 'Bubastite portal'—the only section com-
pleted—was inscribed with a record of Sheshonq's victories in
Palestine, which is one of the most valuable historical sources for the
entire period.
Efforts to consolidate the unity of the realm continued under
Sheshonq's successors, but the growing power of provincial rulers led
to the weakening of royal control and a consequent fragmentation of
the country. The post of high priest of Amun and other key offices were
once more permitted to become hereditary, and this facilitated the
development of independent power bases. The appointment of close
relatives of the kings to important posts in major centres such as
Memphis and Thebes failed to halt the growing independence of the
provinces, and indeed probably accelerated the process. In an inter-
esting inscription on a statue from Tanis, Osorkon II (874-850 BC)
petitions Amun to confirm the appointment of his children to various
high civil and religious offices, with the significant proviso that 'brother
should not be jealous of brother'. From the mid-ninth to the mid-
eighth century BC the process of decentralization continued and the
power of the 22nd Dynasty diminished, as provinces ruled by royal
princes and Libyan chiefs became increasingly autonomous. At Thebes,
the high priest Harsiese declared himself king, and was buried at
Medinet Habu in a falcon-headed sarcophagus in clear imitation of the
funerary traditions of the Tanite rulers. Eventually, northern attempts
to impose authority over Thebes led to violence. A long inscription of
Prince Osorkon, son of Takelot II (850-825 BC), carved on the Buba-
stite portal at Karnak (the so-called Chronicle of Prince Osorkon),
describes a series of conflicts that arose as he sought to implement his
authority as high priest of Amun in Thebes against a rival group.
During the reign of Sheshonq III (825-773 BC) and in the years that
followed, numerous local rulers—particularly in the Delta—became
virtually autonomous and several declared themselves kings. The first
of these was Pedubastis I (818-793 BC), who may have been related to
THE THIRD I N T E R M E D I A T E P E R I O D 331
the royal family of the 22nd Dynasty. The location of his power base is
uncertain, but at Thebes it was his authority and that of his successors
that was recognized, in preference to the rule of Tanis. While these
local kings are assigned by some scholars to the 23rd Dynasty, it
remains unclear which, if any, of them are to be equated with the '23rd
Dynasty' as recorded by Manetho, which was perhaps composed of
successors to the 22nd Dynasty at Tanis. By ^.730 BC there were two
kings in the Delta (at Bubastis and Leontopolis), one at Hermopolis,
and one at Herakleopolis in Upper Egypt; besides those in the Delta,
and virtually independent, were a 'Prince Regent', four Great Chiefs of
the Ma, and a Prince of the West in Sais. This last, Tefnakht (kings
727-720 BC), had taken over all the territories of the western Delta and
Memphis, and was expanding into the northern reach of Upper Egypt.
This illuminating snapshot of the political geography of Egypt is to
be found on a stele set up at Gebel Barkal near the fourth cataract by
the Nubian ruler Piy (747-716 BC). During the second half of the
eighth century BC, the rulers of Kush had emerged as strong con-
tenders for power over Egypt. Following initial assertions of authority
by Kashta, Piy (Kashta's son) launched a military expedition into
Egypt—ostensibly to halt the expansionist policies of Tefnakht of Sais.
Piy's troops appear to have taken Thebes without a struggle, perhaps
owing to a previous agreement with the local representatives of the
23rd Dynasty, and the towns and cities of northern Upper Egypt
rapidly capitulated or were besieged and captured. Memphis offered
resistance and was taken by assault, after which the dynasts submitted
to Piy, acknowledging him as their overlord.
After this show of strength, Piy returned to Nubia, leaving the politi-
cal situation in Egypt virtually unchanged. During the following decade
Tefnakht assumed the status of king; he and his successor Bakenrenef
(Bocchoris) constitute the 24th Dynasty. Although based at Sais,
Bakenrenef s authority was soon acknowledged throughout the Delta
and as far south as Herakleopolis. But the Nubians, having once tasted
power in Egypt, were not prepared to countenance its loss. In c.ji6 BC
Piy's successor Shabaqo (716-702 BC) launched a new invasion. On
this occasion Egypt was formally annexed to Kush and Shabaqo and
his successors—Shabitqo, Taharqo, and Tanutamani—were recog-
nized by later historians as the 25th Dynasty. According to Manetho,
Bakenrenef was executed, but fully centralized government was not
restored. Instead, the Kushite monarchs ruled as overlords and per-
mitted the dynasts to remain in control of their fiefs. In order to be
recognized as authentic Egyptian pharaohs, they displayed special
332 JOHN TAYLOR
respect for Egyptian religious and cultural traditions, and intentionally
sought an ideological link with the great eras of Egypt's past—in
particular with the Old Kingdom. To this end, Memphis was promoted
to become the Kushites' preferred residence in Egypt, and nascent
archaizing tendencies were boosted, leading to a revival of artistic,
literary, and religious trends drawing inspiration from earlier ages. In
the south, Thebes retained its pre-eminent status, but the power of the
high priest of Amun was eclipsed. In its place the office of'god's wife
of Amun' grew in importance; this celibate priestess was usually a
royal princess, and each 'god's wife' adopted her successor from among
the junior members of the royal family, eliminating the possible emer-
gence of a Theban-based subdynasty to threaten the king's political
authority.
The Nubian rulers also pursued an aggressive policy with regard
to the former Egyptian dependencies and commercial partners in
Palestine. Intervention in the politics of this region during the early
seventh century BC led, unfortunately, to direct confrontation with the
might of Assyria, which was in the process of exerting its control over
this area of the Levant. In consequence, much of the reign of Taharqo
(690-664 BC) was occupied by increasingly desperate struggles to
defend Egypt against Assyrian aggression. Finally, after the sack of
Thebes by Ashurbanipal's forces (663 BC), the last Kushite monarch
was permanently expelled from Egypt, and it was left to Psamtek of
Sais (who had been installed by the Assyrians as a vassal ruler) to
recover Egypt's independence.
The zist to 24th Dynasties: The Libyan Period
The Libyans who settled in Egypt before and during the Third Inter-
mediate Period were drawn predominantly from the Meshwesh (or
Ma) and the Libu, the principal groups who had threatened Egypt's
security during the New Kingdom. Their homeland appears to have
been Cyrenaica, where they followed an economy based mainly on
pastoral nomadism, although there is also some evidence for settle-
ments. A low degree of infiltration by these people along Egypt's
western fringe was probably endemic; its culmination in large-scale
migrations under Merenptah and Rameses III seems to have been a
consequence of displacement of populations in Cyrenaica, perhaps on
account of local food shortages and the incursions of the Sea Peoples
along the North African coast. Possibly an additional factor was the
development of more concrete political cooperation and military
THE THIRD I N T E R M E D I A T E P E R I O D 333
organization among the Libyans of the later New Kingdom, which
might have prompted a more constructive impetus towards settlement
in Egypt. Under the successors of Rameses III a steady influx con-
tinued. The existence of different population groups among the Libyans
and their semi-nomadic lifestyle doubtless meant that numerous
groups, large and small, moved into Egypt independently. Some of
these Libyans were captives or mercenaries who were settled in mili-
tary communities as a policy of the 2 oth-Dynasty kings, but there were
probably many smaller groups that settled without coming under
official control.
The Libyan element in Egyptian society
Numbers of these Libyans were settled in the area between Memphis
and Herakleopolis, and in the oases of the Western Desert, but by far
the largest concentration of them was in the western Delta. Settlement
here was facilitated by the natural proximity of the area to the Libyans'
homeland, and by the relatively unimportant status of this part of
Egypt in the eyes of the pharaohs; thinly populated and of low agri-
cultural productivity, it was mainly used for grazing cattle.
An account of the growing military and political efficiency of the
Libyans towards the end of the New Kingdom, their chiefs were able to
secure positions of local influence. There had already arisen in Egypt a
class of ex-military men who had been rewarded for their services with
land and who could rise to high office in the bureaucracy. The chiefs of
Libyan mercenary groups were probably no less able to take advantage
of this situation, and in this way a number of principalities developed,
each based at an important town and each controlled by a Libyan
chief—and this not only in the Delta but at strategic points along the
Nile Valley, notably at Memphis and in the area around Herakleopolis.
Unfortunately the sparsity of evidence for the 2ist Dynasty conceals
the precise stages by which these chieftains rose to power, but Libyans
with high military rank are attested in the Herakleopolis area from the
beginning of the Third Intermediate Period, and the appearance of a
ruler named Osorkon on the throne at Tanis in the second half of the
2ist Dynasty is the clearest indication that they had attained the first
rank of Egyptian society.
The Libyans' consolidation of power was probably achieved in a
variety of ways. The development of the theocratic form of government
in the 2ist Dynasty doubtless helped to render their rule palatable in
the crucial transitional stage, by lending divine authority to their
policies. Integration into Egyptian society could have been further
334 JOHN TAYLOR
enhanced by acculturation. Although increased contact with foreign
lands and customs during the New Kingdom had made Egypt a cos-
mopolitan society with a mixed population, foreign settlers still under-
went a process of Egyptianization, the main manifestations of which
were the adoption of Egyptian names, dress, and burial customs. Evi-
dence for acculturation of the Libyans can be adduced, though it is by
no means conclusive. There is no trace of a characteristic material
culture for the Libyans in Egypt, though, in view of the scanty archaeo-
logical documentation of both the Nile Delta and the Libyans' home-
land of Cyrenaica, this picture may yet be transformed by further
investigations. More significantly, the Libyans of the 2ist-24th Dyn-
asties do not figure as 'foreigners' in the Egyptian graphic or textual
record. The distinctive ethnic features associated with Libyans in New
Kingdom art (yellow skin, sidelocks, tattoos, feathered headgear, penis
sheaths, and decorated robes) no longer appear, though this is not
altogether surprising, since the Libyans were distinguished from the
Egyptians in such depictions for ideological reasons rather than as a
faithful reflection of their appearance. In the same way, the depiction
of kings and officials of Libyan origin with traditional Egyptian dress,
attributes, and physical characteristics was probably a conciliatory
measure to encourage acceptance of their authority by the Egyptian
populace; it does not necessarily mean that total integration had been
achieved. There are, in fact, several indications that the Libyans retained
a large measure of their ethnic identity. Their distinctive and very un-
Egyptian names—Osorkon, Sheshonq, Takelot, and others—survived
for centuries after the arrival of the Libyans in Egypt, whereas in earlier
periods foreigners usually adopted or were given Egyptian names
within one or two generations. In the same way, Libyan chiefdom titles
were retained long after settlement in Egypt, and a feather worn in the
hair survived as a distinguishing mark of chiefs of the Meshwesh and
Libu. Long genealogies on statues and funerary objects are one of the
most characteristic features of Libyan-Period texts, yet are not usual in
Egyptian inscriptions before the late 2ist Dynasty. The increase in
such records apparently reflects a new importance attached to kinship
and the preservation of extensive lines of descent—it is a class of
evidence very much based on oral tradition, and tends to be an impor-
tant feature of non-literate societies such as that of the Libyans.
The Libyans and the Egyptians had very different cultural back-
grounds—the Libyans non-literate and semi-nomadic with no tradi-
tion of permanent building; the Egyptians literate, sedentary, and with
a long tradition of formal institutions and monumental construction.
THE THIRD I N T E R M E D I A T E P E R I O D
335
Kings and dynasts of Libyan origin controlled all or most of Egypt for
about 400 years, and some continued to hold power under the Kush-
ites. It is, therefore, very likely that several of the major changes in the
administration, society, and culture of Egypt that occurred during this
period may have had their origins in this mixing of societies.
Power structures and political geography
The most characteristic feature of Egypt during the Third Intermediate
Period is the political fragmentation of the country. This decentraliza-
tion was a consequence of major changes in the government of Egypt,
which distinguish the Third Intermediate Period from the New King-
dom. Important factors are the long-term survival of Libyan chiefs in
powerful positions, and the weakening of the authority of the king.
Particularly significant was the king's policy of granting exceptional
powers to kinsmen and provincial rulers, which created an impetus
towards regional independence and a tension over access to and con-
trol of economic resources.
In the New Kingdom, the majority of royal relatives had been care-
fully excluded from effective administrative and military power, thereby
neutralizing a potential threat to the authority of the king. But in the
Map of Egypt showing the
principal sites and political
divisions at the beginning of the
Third Intermediate Period
(^.1069 BC)
33 6
JOHN TAYLOR
Third Intermediate Period kings' sons were given unprecedented
administrative powers and were placed in charge of major settlements
that enjoyed considerable autonomy—chief among these being Mem-
phis, Herakleopolis, and Thebes. Until the pontificate of Harsiese
(c.86o BC) all 22nd-Dynasty high priests at Thebes were sons of the
Map of Egypt showing the major cities and dynastic centres in
the late Third Intermediate Period (^.730 BC)
THE T H I R D I N T E R M E D I A T E P E R I O D 337
reigning king, and since many of these local princes also had military
power at their disposal, this had major implications for the political
course of events.
Equally telling was the royal policy of permitting offices in the
bureaucracy, clergy, and military to become hereditary benefices of
provincial families. High office had sometimes been passed from
father to son in the New Kingdom, but the process was by no means
automatic. In the Third Intermediate Period the practice became
endemic; already under the 2ist Dynasty the posts of high priest of
Amun and chief general were controlled by a single family. An attempt
by the early rulers of the 22nd Dynasty to circumvent the debilitating
effects of this monopoly by appointing king's sons as high priests at
Thebes and other king's men to other high offices did not halt the
trend; the former actually promoted decentralization, and in the case
of the latter the hereditary principle soon reasserted itself The effects
of the practice are clear at Thebes, where genealogical inscriptions on
funerary objects and temple statues show the descent of important
posts in the administration and priesthood through many generations
of local families. The appearance in genealogies at this period of the
phrase mi nen ('the like-titled') before the names of ancestors is an
indication that the passing of offices to successive generations had
become commonplace. These families strengthened their own posi-
tions by intermarriage with other office-holding clans, creating power-
ful local elites who controlled provincial centres. Officials of traditional
centralized government, such as the vizier and overseers of the treasury
and granaries—who in the New Kingdom had constituted a check on
the independence of the provinces—now wielded only local influence,
or, as in the case of the southern viziers, were themselves members of
the dominant provincial aristocracy.
Under these conditions, the independence of regional centres and
the rise of collateral dynasties was virtually inevitable. The process of
decentralization was most marked in the Delta. Here several provin-
cial centres came under the control of Libyan chiefs, and some of
these—notably Sais and Leontopolis—eventually eclipsed the pre-
eminence of the 22nd Dynasty, whose sphere of influence was ulti-
mately reduced to a small area focused around Tanis and Bubastis.
The situation in Upper Egypt was analogous, althought this part of the
country retained greater territorial cohesion than the north. Thebes
was predominant throughout this entire period, its importance founded
on its status as the main cult centre of Amun, and on its being the
focus of the most powerful local elite.
338 JOHN TAYLOR
The attitude of the kings to this progressive fragmentation is of key
importance. In the First and Second Intermediate periods the division
of power within Egypt among two or more rulers was definitely
perceived as unacceptable; in the Third Intermediate Period, however,
decentralization was not regarded consistently in a negative light. The
long-term appointment of royal relatives to positions of power and the
marrying of kings' daughters to important provincial governors may
be viewed as measures to bolster the authority of the king; yet both
produced the opposite effect, promoting decentralization by strength-
ening the power base of local rulers. It has been suggested that King
Sheshonq III (825-773 BC), concerned at the waning authority of the
22nd Dynasty, intentionally established a collateral royal line, the 23rd
Dynasty, as a means of retaining a measure of control over provincial
elites. This is highly questionable, particularly in view of the debatable
status of the 23rd Dynasty. A clearer picture emerges if it is assumed
that decentralization was not only accepted but institutionalized as
a form of government. The political picture that emerges as the Third
Intermediate Period progresses is one of a federation of semi-
autonomous rulers, nominally subject (and often related) to an
overlord-king. This is perhaps an example of the impact of the Libyan
presence on the administration, since such a system can be seen as
consistent with the patterns of rule in a semi-nomadic society such as
theirs. In favour of this interpretation it should be noted that—despite
the prominence of military titles and fortified settlements during this
period—explicit references to internal conflict are limited and should
not be interpreted as signs of a slide into anarchy.
Consideration of the political geography of Egypt in the Third Inter-
mediate Period reveals indications of a north-south divide. Control of
the north was almost entirely in the hands of the Libyans. Their influx
was in fact crucial in the settlement and cultivation of the Delta: the
Meshwesh occupied the principal towns of the eastern and central
zones (Mendes, Bubastis, Tanis). The main influx of the Libu perhaps
occurred later than that of the Meshwesh, and hence they settled on
the less profitable western edge, around Imau. They ultimately founded
the dynasty of Sais. Another group, the Mahasun, are found towards
the south. The chronological and spatial distribution of 'donation
stelae' perhaps reflects the progressive utilization of cultivable land,
proceeding from the eastern and western edges of the Delta towards
the centre, as areas previously unoccupied or uncultivated were taken
over. The semi-autonomous status of centres such as Bubastis, Mendes,
Sebennytos, and Diospolis was probably established during the early
THE THIRD INTERMEDIATE P E R I O D 339
phase of Libyan settlement, and was retained throughout the succeed-
ing centuries.
Upper Egypt was less fragmented than the Delta. While centres
such as Hermopolis, Herakleopolis, el-Hiba, and Abydos were impor-
tant, Thebes retained its pre-eminent status throughout the Third
Intermediate Period. Southern resistance to the imposition of control
from the north was a recurring feature of the tenth to the eighth
centuries BC, with Thebes and its officials playing the leading role.
There are signs of this already at the outset of the 22nd Dynasty; in
inscriptions carved early in his reign, Sheshonq I holds the title 'chief
of the Ma' rather than 'king'. Subsequently, entitlement to the post of
high priest of Amun became a major source of contention. The claims
of Prince Osorkon, son of Takelot II, to the pontificate provoked fierce
resistance, the Thebans preferring to recognize the authority of the
23rd Dynasty kings Pedubastis I and luput I, and subsequently
Osorkon III and his successors, rather than that of the Tanite phar-
aohs. Still later, the southern rulers made an alliance with Kush and
continued to date inscriptions by the reigns of the Kushite monarchs
even after their expulsion, indeed as late as the first years of Psamtek I
(664-610 BC) of Sais.
Underlying the political north-south divide was an ethnic division.
The evidence of names, titles, and genealogies reveals the population
of the north as predominantly Libyan and that of the south as Egyptian.
Reflections of this can also be detected in material culture. After the
New Kingdom the evolution of the hieratic script used in business
documents produced two divergent forms, demotic in the north and
'abnormal' hieratic at Thebes—an indication that the administration
of the north had no appreciable impact on Thebes. Other linguistic
changes confirm the indications of a breakdown of New Kingdom
traditions; scribes of the Libyan period employed grammatical con-
structions and phonetic spellings that reflected current usage rather
than tradition, and hieratic script was increasingly used in place of
hieroglyphs in monumental inscriptions. These developments, partic-
ularly the last, are more common in the north and may reflect a lack
of concern for tradition on the part of Libyans grappling with an
unfamiliar idiom.
The ideology of kingship
The subordination of the temporal ruler to Amun, which was a key
aspect of the theocracy, may have recommended itself to the Libyan
rulers of the 2ist Dynasty as a politically expedient means of securing
340 JOHN TAYLOR
divine sanction for their new regime. As noted in Chapter 10, the
relationship between Amun and the king changed during the late New
Kingdom. With the establishment of the theocracy in the 2ist Dynasty,
the political independence of the king reached its lowest point, and his
executive authority scarcely exceeded that of the high priests. Indeed,
while three of the Theban pontiffs adopted kingly titles, the pharaoh
Psusennes I also appears as high priest of Amun, indications that the
offices were closer to equality than ever before. The Thebans' assump-
tion of royal attributes was restricted, for, although Herihor and
Pinudjem I were depicted with kingly prerogatives (equal in stature to
the gods, adorned with royal costume, and with names in cartouches),
Herihor was shown so only in temple reliefs and on the funerary papy-
rus of his wife Nodjmet, while his royal prenomen is merely the title
'high priest of Amun'. The commander Menkheperra, son of Pinud-
jem I, merely used cartouches occasionally and was once depicted in
kingly costume. Only Pinudjem I displayed fuller pretensions to
pharaonic status and was buried with royal honours. This sporadic
kingship may have been assumed mainly for cult purposes: since it
was the king who was the point of contact between the world of men
and that of the gods, a practically independent state such as that of
Upper Egypt required someone to fulfil that role. By the beginning
of the 22nd Dynasty, the Libyans were firmly entrenched in power,
and hence the theocratic character of government was toned down.
Sheshonq I and his successors re-emphasized the political authority of
the king, but, when this weakened after £.850 BC, it was first the high
priests at Thebes, and subsequently the 'god's wives of Amun' and
their officials, rather than Amun himself, who wielded power.
Throughout the eleventh to eighth centuries BC the Libyan rulers
made use of many of the outward manifestations of traditional
pharaonic rule to assert their status as true Egyptian kings. They were
depicted in pharaonic costume, with full fivefold titularies; the scene of
the king smiting enemies before Amun (attested for Siamun and
Sheshonq I) symbolized the traditional role of preserving maat (the
ordered universe) by defeating Egypt's foes, and the holding of the
sed-festival linked them to past generations of rulers. The sed-festival
held at Bubastis in year 22 of Osorkon II (874-850 BC) was com-
memorated in reliefs on a specially built red granite gateway that show
great adherence to ancient tradition in the forms of the ceremonies
depicted. To lend greater legitimacy to the rule of foreigners, the royal
ideology was developed along carefully selected lines. One of these
developments is the more frequent assimilation of the king to the child
THE THIRD INTERMEDIATE P E R I O D 341
Horus, son of Osiris and Isis, which is alluded to in the titularies of
several Libyan kings from Sheshonq I onwards and finds a parallel in
depictions of the pharaoh as a child suckled by a goddess. These phe-
nomena were doubtless intended to reconcile the indigenous popula-
tion to foreign rule; the Hyksos, Persians, and Ptolemies all found
such assimilation politically useful. But, as noted above, the Libyans
were never fully Egyptianized and, in spite of their pharaonic trapp-
ings, the kings preferred different patterns of rule to those of their New
Kingdom precursors.
A clear instance of this is the Libyans' apparent tolerance of two or
more 'kings' simultaneously, each entitled 'king of Upper and Lower
Egypt', irrespective of their actual spheres of influence. This is not the
only sign that the Libyans had adopted the trappings of Egyptian
kingship without fully understanding it; in the New Kingdom great
importance was attached to the composition of the royal titulary, which
was different for each king and reflected a carefully devised pro-
gramme for the reign. The titularies of the Libyan rulers, however, are
characterized by a monotonous repetition of prenomina and royal
epithets that frequently hinders the correct attribution of royal monu-
ments of this period.
Not only is it more difficult to distinguish king from king; there is
also a blurring of the distinction between the king and his subjects.
The power structure in Egypt around 730 BC, as revealed by the 'victory
stele' of Piy, shows chiefs of the Meshwesh on a footing of equality
with kings, although without royal titles. A few decades later, at the end
of Kushite rule, Assyrian records (the Rassam Cylinder) reveal a com-
parable situation, with all local governors grouped together irrespect-
ive of their titles. These include a 'king' (Nekau I (672-664 BC)), a
'great chief, a governor, and a vizier. The loss of the unique status of
the king is manifested in numerous ways. In art, non-royal persons are
depicted performing acts previously reserved for the king: a Libyan
chief is depicted in a statuette kneeling and offering to a god; a relief
shows another chief consecrating 'choice cuts' of meat on the altar to
the gods of Mendes; a high priest of Amun and a priest of lower rank
offer the image ofmaat on stelae. The same phenomenon is reflected
in economic sources, notably 'donation stelae'. In the New Kingdom
such donations were undertaken only by the king; in the Third Inter-
mediate Period numerous stelae record temple donations, and, while
the donor is occasionally a king, in the majority of cases it is a Libyan
chief or private individual. Even personal names can be revealing:
Ankh-Pediese, named on a Serapeum stele as grandson of the Great
342 JOHN TAYLOR
Chief of the Meshwesh Pediese, bears a name that means 'May
Pediese live', a commemoration of a Libyan chief in a context where
usually only a royal person (king or 'god's wife of Amun') is named. Per-
haps most remarkable of all is the intrusion of members of the king's
retinue into their masters' burial place; the interment of General
Wendjebauendjed in a chamber of the tomb of Psusennes I at Tanis
would have been unthinkable in the New Kingdom, but now the king
had more the character of a feudal overlord, supported by a network of
close kinsmen and retainers whose ties with their master are prom-
inent even in the grave.
The military in the Libyan period
After the New Kingdom, military power rather than bureaucratic con-
trol was a major basis of authority in Egypt. The new order was
founded by army commanders, and the rulers of the southern prin-
cipality throughout the whole of the 2ist Dynasty were primarily
generals. The appointments of the 22nd-Dynasty rulers ensured that
most provincial governors were army commanders, and the fact that
these titles were not purely honorific is demonstrated by references to
fortresses and garrisons under their command.
The building of fortresses is one of the best-documented activities of
this period. Few of these are attested archaeologically by more than a
few traces, but the locations of many are known from finds of bricks
stamped with the names of the dedicators. This evidence shows that a
whole series of fortresses was built in Upper Egypt during the 2ist
Dynasty (notably under Pinudjem I and Menkheperra). There was a
particular concentration of these installations on the east bank of the
Nile in northern middle Egypt: at el-Hiba, Sheikh Mubarek, and Tehna
(Akoris). From these strongholds a careful watch could be maintained
over Nile river traffic, and any local insurrections quickly crushed.
El-Hiba was more than just a lookout point and garrison. It was a
frontier fort and the northern headquarters of the Upper Egyptian
rulers during the 2ist Dynasty. Papyrus letters of the period, mention-
ing the generals Piankh and Masaharta, have been found there, and
the papyri bearing the literary compositions Wenamun and the Tale of
Woe, as well as the Onomasticon ofAmenemope, are probably from the
same vicinity. The site continued to function as an important military
headquarters under the 22nd Dynasty; a temple was built there by
Sheshonq I and added to by Osorkon I. Still later, the place was used as
an operational base by Prince Osorkon in his conflict with his Theban
opponents.
THE THIRD I N T E R M E D I A T E P E R I O D 343
Civilian settlements also appear to have acquired the character of
military strongholds in the Third Intermediate Period. The adminis-
tration of the Theban west bank took refuge in the fortified temple
enclosure of Medinet Habu during the troubles at the end of the New
Kingdom, and this apparently remained the high priests' residence
during the 2ist Dynasty. Nor was this an isolated instance. The account
of Piy's campaign in 0.730 BC shows that cities such as Hermopolis and
Memphis were fortified and sufficiently strong to withstand a siege.
The lifestyle of the Egyptians had evidently become habitually defen-
sive in outlook.
The heavy concentration of troops along the Nile may have had its
origins in the Libyan chiefs' determination to enforce their rule over
Egypt. This, together with Thebes' well-documented resistance to
external control, probably accounts for the siting of 21 st-Dynasty fort-
resses at such southerly locations as Qus and Gebelein, where they
could scarcely have served to guard against attack from outside the
Nile Valley. A rebellion in the Theban area occurred during the reign
of Pinudjem I, but its nature is obscure. Indeed, it is only known from
the stele set up by the high priest Menkheperra to commemorate the
pardoning of some of the miscreants and their recall from the oasis to
which they had been exiled as punishment. The struggles of Prince
Osorkon against Theban rebels over a century later demonstrated the
continued need for military force to retain authority in this area.
The relatively unadventurous foreign policy of Egypt's rulers in the
Third Intermediate Period can be seen as the logical counterpart to the
internal situation. Under a progressively decentralized regime, and
with a substantial part of the available military force required to keep
order within Egypt, the concentration of military effort and economic
resources necessary to pursue a consistent expansionist policy abroad
probably could not be achieved.
Economy and control of resources in the zist-z^th Dynasties
The period covered by the 2ist to 24th Dynasties is conspicuous for the
sparsity of large-scale royal stone monuments of the kind erected
during the New Kingdom. Except for those at Tanis, royal building
works were mainly confined to minor additions and repairs to existing
structures. This reduced level of activity coincides with the extensive
recycling of monuments and materials, a phenomenon particularly
obvious at Tanis, where much of the stonework—blocks, columns,
obelisks, statues—was brought from Piramesse and other sites, and
reinscribed or simply re-erected without modification. Judged against
344 JOHN TAYLOR
the productions of other periods, these factors could be considered as
signs of a weak economy. There is indeed no doubt that the Third
Intermediate Period began at a time of economic stress, and, as far as
can be discerned, revenues from the Levant and the African interior
were much reduced during this period by comparison with what had
been available during the New Kingdom.
There are, however, a number of indications that Egypt's economy
did not remain seriously weak throughout this entire period. The
unambitious nature of royal building projects and the high depend-
ence on reused materials in the Third Intermediate Period can be
plausibly explained by the politically fragmented state of the country.
Without a centralized administration under a single ruler it was no
longer possible to manage Egypt's resources efficiently or to mobilize
huge labour forces of the kind that had built the Memphite pyramids
or the temples of Karnak. It is significant that the relatively brief
phase of strong centralized government (the reigns of Sheshonq I to
Osorkon II) coincided with the erection of several of the most sub-
stantial royal monuments of the time: the Bubastite Portal at Karnak
and the 'festival hall' of Osorkon II at Bubastis.
Concerning the state of the agricultural economy at this period,
information is very limited. A few papyri (including Papyrus Rein-
hardt) and the donation stelae are the only sources. These latter, how-
ever, are very interesting; the majority date from the 22nd and 23rd
Dynasties, and they record the assignation of land to temples in order
to establish endowments for funerary cults. The large numbers
of these stelae that have been found in the north indicate that the
productivity of agricultural land remained sufficient to enable a sur-
plus to be available for such purposes. As noted above, the distribution
of these stelae also indicates that substantial areas of the western and
central Delta were being newly brought under cultivation.
There is also evidence that other forms of wealth were not lacking.
The burial goods of the 2ist- and 22nd-Dynasty kings found in the royal
tombs at Tanis included substantial quantities of gold and silver, while
an inscription from Bubastis recording the dedication by Osorkon I of
statues and cult utensils to the temples of Egypt lists the equivalent of
over 391 tons of gold and silver objects—all apparently presented
during the first four years of the king's reign. A proportion of this, it
has been postulated, may represent the booty from Sheshonq I's
Palestinian campaign of a few years earlier, while some of it was per-
haps recycled material extracted from New Kingdom tombs. Never-
theless, an economy in which so much bullion could be economically
THE THIRD I N T E R M E D I A T E P E R I O D 345
neutralized through consecration to deities could only have been a
healthy one.
The recycling of resources undoubtedly played a part in keeping the
state's coffers full. This was probably the principal reason (rather than
pious regard for the dead) for the dismantling of the New Kingdom
royal burials at Thebes during the 2ist Dynasty. The mummies of the
kings and their wives and families were removed from their tombs,
stripped of almost all their valuables, and reburied in groups in
unobtrusive and easily guarded caches. The hieratic dockets on coffins
and shrouds that record these actions show that they were carried out
on the authority of the ruling generals, while hundreds of rock graffiti
written by the necropolis scribe Butehamun and his colleagues testify
to the systematic searching-out and clearing of old tombs. Much
precious metal was doubtless melted down for reuse, but some items
seem to have been appropriated for the burials of the Tanite kings;
pectorals found on the mummy of Psusennes I bear a strong resem-
blance to New Kingdom examples such as those from the tomb of
Tutankhamun, and there are traces of altered names in some of the
cartouches. Items of substantial size were also recycled. A granite
sarcophagus was extracted from the tomb of Merenptah and trans-
ported to Tanis to be reinscribed for the burial of Psusennes I. The
wooden coffins of Thutmose I were refurbished and used again to
house the mummy of Pinudjem I. In this instance mere thrift may
have been of less import to Pinudjem than the opportunity offered to
associate himself directly with one of the great kings of Egypt's past,
thereby lending ideological support to his own somewhat unorthodox
claim to pharaonic status. Curiously, what may have begun as a
prerogative of the Theban rulers alone was soon widespread; in the
2ist Dynasty a high proportion of coffins used for burials at Thebes
were reinscribed and reused within a short time of the original
burial—probably illicitly: a docket written on a coffin in the British
Museum records its restoration to the true owner after necropolis
workers had been caught in the act of usurping it.
Kushite Rule (the 25th Dynasty, 747-664 BC)
For events in Nubia between the end of the New Kingdom and the
early eighth century BC, evidence is extremely meagre. Although the
suggestion that Lower Nubia was depopulated during this period is
probably an exaggeration, the population may have been less prosper-
ous than in earlier times and perhaps reverted to a semi-nomadic
346 JOHN TAYLOR
economy or migrated to the more prosperous south. Sporadic refer-
ences to viceroys of Kush during the 2ist-23rd Dynasties indicate that
some Egyptian pretensions to authority there were maintained, and
elements of royal titularies and formal epithets from temple inscrip-
tions in Egypt have been adduced as supportive evidence for an aggres-
sive policy to regain Upper Nubia—but, if this were the case, there was
no lasting effect.
The rise of Kush
There is no evidence from Nubia itself for any provincial government
or campaign at this time. In fact, inscriptions from Nubia suggest that,
following the withdrawal of Egyptian authority at the end of the New
Kingdom, several local power groups arose, perhaps maintaining a
degree of continuity at the pharaonic administrative and religious
installations. It was probably these groups that were responsible for a
small number of hieroglyphic inscriptions and reliefs in the Egyptian
iconographic tradition, apparently dating from this period; the reliefs
of Queen Karimala in the New Kingdom temple at Semna are a case in
point.
The most important of these indigenous polities arose in the area
downstream of the fourth cataract. The earliest of their rulers were
buried at el-Kurru. Although the exact sequence of the tombs is
uncertain, a clear evolution in the arrangements for the burials is
apparent. The earlier tombs are strongly Nubian in character, having a
circular tumulus or mastaiba-style superstructure over a burial pit,
which contained the corpse laid on bed. Later tombs are characterized
by more Egyptian-inspired features (mastaba superstructure accom-
panied by offering chapel, all within an enclosure wall). El-Kurru may
have been the original power base of these rulers, since a settlement
with defensive walls has been identified there, but by the late eighth
century BC their political and religious focus had shifted to Napata,
close to the great rocky outcrop of Gebel Barkal. During the New
Kingdom, this had been the centre of the cult of Amun in Nubia, and
the worship of the state god of Egypt became a defining feature of the
Kushite ruling elite. By the mid-eighth century the chieftains of Napata
had become overlords of Nubia and were already entertaining preten-
sions to rule Egypt as well.
The Kushite takeover of Egypt
Direct contact with Egypt was renewed around 750 BC. Kashta, the first
ruler of Kush of whom contemporary records survive, appears to have
THE THIRD INTERMEDIATE P E R I O D 347
been recognized as king throughout Nubia and as far north as Aswan,
where a stele was erected showing him as 'King of Upper and Lower
Egypt'. The inward-looking nature of Egypt's government probably
facilitated these advances. Under Piy, son of Kashta, some form of
agreement was perhaps reached with the 23rd Dynasty rulers recog-
nized in the Theban area. Piy's authority was acknowledged and his
sister Amenirdis I was adopted by the 'god's wife of Amun' Shepen-
wepet I as her successor. These preliminary steps were followed
around 730 BC by a more overt demonstration of power in the form of a
Kushite military expedition. According to the vivid description pro-
vided by Piy's triumphal stele from Gebel Barkal, the campaign was
prompted by the rapid territorial expansion of Tefnakht of Sais.
Having taken control of the entire western Delta and the Memphite
area, this powerful prince was extending his influence over the towns
and cities of northern Upper Egypt. Nimlot, petty king of Hermopolis,
joined forces with Tefnakht, but another 'king', Peftjauawybast,
having declared loyalty to Piy, was besieged in his city of Herakle-
opolis. Piy's forces advanced down the Nile, pausing at Thebes to offer
homage to Amun, before relieving Peftjauawybast and capturing
Hermopolis. Most of the other towns and cities along the river capitu-
lated, but Memphis offered stubborn resistance and had to be taken by
storm. Piy, however, with conspicuous reverence for the religious
traditions of Egypt, took care that the temples there should be pro-
tected from looting or desecration. Having adored the gods at Mem-
phis and Heliopolis, Piy received the homage of the provincial rulers,
who acknowledged his authority over Egypt as well as Kush.
Piy spent the remainder of his reign in Nubia and at his death was
buried at el-Kurru in a tomb of strongly Egyptian character, with a
pyramid superstructure and a burial outfit including shabti-figures.
Quite un-Egyptian, however, was the interment nearby of a team of
chariot horses, a feature also associated with the burials of Piy's suc-
cessors, and evidently a distinctively Kushite practice. In the years that
followed, the situation in the Theban area probably remained stable.
The installation of Amenirdis as 'god's wife of Amun'—doubtless with
the support of a Kushite retinue—lent weight to the influence of the
Nubian rulers in the area. In the north, however, the local dynasts were
left in control of their provinces, and under Tefnakht of Sais and his
successor Bakenrenef the 24th Dynasty resumed its territorial expan-
sion. Thus provoked, the new Kushite ruler Shabaqo reconquered
Egypt in about 716 BC, and forcibly imposed his authority over the
provincial governors.
348 JOHN TAYLOR
The rule of the Kushite monarchs
The fundamental basis of Kushite rule was military power. Close links
between the king and his army are apparent throughout the 25th
Dynasty. The devotion of Piy's troops to their master is constantly
stressed in the text of his triumphal stele, and physical prowess and
military training were held to be of importance both to the rulers them-
selves and to their soldiers. Hence the young Taharqo was present in
person at the battle of Eltekeh (701 BC), while a stele from Dahshur
recounts details of a gruelling military exercise organized by the same
king in the desert between Memphis and the Faiyum. However, in
spite of the strength of their armed forces, the Kushite kings perhaps
felt unequal to the task of controlling both their native land and a
unified Egypt. This may have influenced their toleration of a decentral-
ized administration within Egypt, since the principalities that had
enjoyed near-autonomy under the Libyan pharaohs retained their indi-
viduality throughout the rule of the Kushites. Hence in the early
seventh century BC Tanis was still ruled by local princes, some of
whom boasted royal titles, a situation that is reflected in the demotic
cycle of stories centring on a King Pedubast of Tanis—what con-
nection (if any) these Tanite rulers had with the old royal line of the
22nd Dynasty remains unknown. The Saite principality too survived,
ultimately to reunite Egypt under Psamtek I. At Thebes the office of
'god's wife of Amun' steadily increased in importance, constituting a
valuable support for the authority of the king; other traditionally
powerful offices, such as that of vizier, continued but were deprived of
effective power. The post of high priest of Amun, so often a source of
tension in previous years, had apparently remained vacant during the
later eighth century but was now reinstated, and assigned once more
to a king's son. Significantly, however, the holder wielded little or no
civil or military power. Local influence in Upper Egypt increasingly
devolved on those who held the office of governor of Thebes or
belonged to the retinue of the 'god's wife'. In the early phase of Kushite
rule Nubian retainers of the royal house were appointed to some of
these major posts in the civil and religious administration at Thebes—
only to be replaced after a few years by scions of local families.
Under the Kushites, the ideology of kingship was modified. Small
but significant changes were made to royal iconography: a double
uraeus was regularly depicted on the king's headband; the blue crown
ceased to be shown, while the cap crown became common in depic-
tions, both in its basic form and with additional bands—a distinctively
Kushite headgear. Innovations are also apparent in the mode of
THE THIRD I N T E R M E D I A T E P E R I O D 349
transmission of kingship; whereas in Egypt royal succession had been
patrilinear, in Kush a king was not necessarily succeeded by his son,
but sometimes by his brother. Such a system certainly operated during
the 25th Dynasty, both Piy and Shabitqo (702-690 EC) being suc-
ceeded by brothers. In spite of these divergences from Egyptian norms,
however, the Kushite rulers sought to strengthen their legitimacy by
posing as champions of ancient tradition. Hence Memphis became
their chief royal residence; a stele from Kawa records that Taharqo was
crowned at Memphis, and Shabaqo, Shabitqo, and Taharqo are all
known to have carried out building works there. This made excellent
political sense (Tanis being too remote geographically to serve as the
focus for a united Egypt), but there were also sound ideological reasons
for boosting the importance of the Memphite area, for in this way the
Kushite pharaohs could associate themselves directly with the great
rulers of the Old Kingdom. Royal tombs in Kush were constructed in
pyramid form. Scenes in temple T at Kawa were copied by Memphite
artists from Old Kingdom royal mortuary temples at Saqqara and
Abusir (the inclusion at Kawa of a scene of Taharqo as a sphinx
defeating Libyan foes—although based on Old Kingdom models—
may well have been intended to emphasize the Kushites' triumph over
Egypt's former rulers).
The verbose and monotonous royal titularies of the Libyan Period
were replaced by simpler ones recalling the style of the Old King-
dom—the prenomen of Taharqo (Khunefertemra) also assimilating
the king to the Memphite god Nefertem. The high status of the god
Ptah was also reaffirmed through the preservation of the cosmological
text known as the Memphite Theology of Creation. This inscription,
allegedly copied from a decayed papyrus on the orders of Shabaqo, was
carved on a basalt slab now in the British Museum; the text gives
primacy to Ptah as creator of the universe. At the same time, the
devotion to Amun which was so conspicuous a feature of the Kushite
monarchy continued to be emphasized, with extensive renovations
and additions made to the temples of Thebes and promotion also of
Amun's role as creator-god, as emphasized in the form and decoration
of a remarkable structure erected by Taharqo close to the sacred lake at
Karnak.
Cross-cultural links: Egypt and Kush
The Kushite rulers had already absorbed a measure of Egyptian culture
before Piy, as is shown by the design of the later tombs at el-Kurru. The
sources of this influence in the early stages of the kingdom are
350 JOHN TAYLOR
unknown, but commercial contacts, together with some survival of
Egyptian cult practices at Gebel Barkal, may have been significant.
These tendencies developed further after the intensification of contacts
during the eighth century, and by the time of Kashta a strong Egyptian-
ization of the ruler is apparent in iconography. Throughout the 25th
Dynasty the rulers and the elite were depicted in Egyptian dress,
adopted Egyptian burial practices, and professed devotion to Egyptian
gods. This acculturation remained a key component of Kushite culture
for centuries after the Nubians had relinquished control of Egypt.
The Kushite absorption of Egyptian material culture is most
apparent in royal monuments. Both in Egypt and Nubia, temples were
constructed according to Egyptian architectural traditions, with careful
observance of the appropriate artistic canons and use of hieroglyphic
script and Egyptian language in inscriptions. Although buried in their
homeland, the rulers constructed tombs in the Egyptian style, each
with a pyramid superstructure, an offering chapel to the east, and a
vaulted burial chamber adorned with scenes and texts from the New
Kingdom repertoire of'books of the netherworld'. Their bodies were
mummified and provided with anthropoid coffins, canopic jars, and
sfoabti-figures.
As was the case with the Libyans, the effects of acculturation prob-
ably conceal the origin of many Kushites living in Egypt at this period,
yet they too retained features of their ethnic identity. The rulers
retained their Kushite birth names, despite adopting Egyptian names
for the remainder of their titulary. Distinctive un-Egyptian names
(Irigadiganen, Kelbasken) also mark out several other officials of the
period as Kushites, while some took Egyptian names and retained
their original Nubian names as well. Kushite ethnic features, includ-
ing distinctive southern physiognomies, dark skin-colouring, and
'bobbed' feminine hairstyles are sometimes depicted in sculpture,
painting, and relief. The cultural exchange, however, was almost
entirely a one-way process, for very little that was Kushite was taken
over into Egyptian material culture, and even that little was not
retained permanently. The characteristic regalia of the Kushite rulers
disappeared after the 25th Dynasty, as did other innovations such as
the occasional depiction of the goddesses I sis and Nephthys with a
close-cropped 'Nubian' hairstyle on funerary monuments.
The 2$th Dynasty as a period of renewal
As part of their drive to obtain legitimacy as pharaohs, the Kushite
rulers evinced deep respect for Egyptian religious traditions. They
THE THIRD INTERM EDIATE P E R I O D 351
remodelled the ideology of the king—drawing inspiration from the
distant past, as seen in their royal titularies, their burial style, and their
promotion of the city of Memphis—and they made deliberate refer-
ence to the Old Kingdom. These associations were part of a much
wider and deep-rooted revival of things past that affected many aspects
of Egypt's court culture, religion, script, literature, art, architecture,
and burial practices during the first millennium BC. Such 'archaism'—
a turning to classic ages of the past as a source of new creative energy—
was not new; it is a recurrent feature in Egyptian culture. In this
instance it had its origins in the later Libyan period, having begun
during the first half of the eighth century BC. Already in the late 22nd
and 23rd Dynasties royal titularies show a progressive simplification,
and imitation of Old and Middle Kingdom models begins to be
apparent in royal iconography and funerary practices. The Kushites
(perhaps lacking suitable indigenous traditions in their homeland)
took up this trend actively. Archaism thus accelerated during the
late eighth and early seventh centuries, becoming fully synthesized in
the 26th Dynasty, with which period the trend is more usually asso-
ciated.
By the 25th Dynasty, artists had revived the Old Kingdom canon of
proportions for representing two-dimensional figures, with a reduc-
tion in the size of the squares in the grid system used by draughtsmen.
Statues, both royal and private, also imitated older models; hence,
among the many sculptures commissioned for the Theban governor
Mentuemhat are examples copying the striding male figures of the Old
Kingdom and the seated cloaked statuettes typical of the Middle King-
dom. In burial customs, the funerary assemblage, which had been
simplified under the 2ist and 22nd Dynasties (see below), was enriched
in the second half of the eighth century, with the revival of older
features, and notably the return—in a revised form—of the Book of the
Dead, as well as the introduction of new iconographic features (often
incorporating archaic elements) for coffins and tombs.
As noted above, the escalation of archaism in the eighth and seventh
centuries probably owes something to the aim of foreign rulers to be
accepted as Egyptians. An additional factor, however, was a desire to
preserve the past through copying earlier monuments. The most
explicit reference to this is the introduction to the Memphite Theology of
Creation, on the 'Shabaqo Stone', which relates how the king found the
text on a worm-eaten papyrus and ordered it to be transcribed for
posterity. Whether or not this statement is literally true, the intention
to preserve the integrity of an ancient text was reflected through
352 JOHN TAYLOR
conscious imitation of the format, wording, and spelling of early docu-
ments.
The widespread reuse of older material in the 2ist and 22nd Dyn-
asties had enabled craftsmen to study and copy earlier models, and the
greater productivity in temple and tomb construction fostered through-
out Egypt by the 25th Dynasty rulers provided an opportunity to
express these new tendencies more fully. This was doubtless one of
the main methods by which older models were transmitted, although
the possibility exists that 'pattern books', copied repeatedly over the
centuries, may also have played a part. Direct slavish copying was,
however, rare. Even when a 25th-Dynasty relief can be compared with
an Old Kingdom model, as in the sphinx scene of Taharqo (mentioned
above), there are some elements of innovation, and one cannot rule out
the hypothetical role of lost intermediary copies in the transmission of
such scenes over a long time span. As the example of Mentuemhat's
statues demonstrates, the revivalism of the 25th Dynasty and later
periods was characterized by an eclectic approach to sources. Many
works of art mingle elements drawn from models of different periods,
the 25th Dynasty showing a preference for the Old and Middle king-
doms, rather than New Kingdom models. This melding of different
influences is apparent even within individual works: statues of
Taharqo and Tanutamani (664-656 BC) from Gebel Barkal have the
strongly modelled bodies and simple costumes typical of the Old King-
dom, while their torsos display the median line characteristic of
sculptures created in the Middle Kingdom.
Kush and Assyria
Although centralized government had not been restored in Egypt by
the Kushite monarchs, their authority as overlords enabled them to
adopt a more active policy with regard to the Levant than had any of the
Libyan kings since Sheshonq I. This led to conflict with Assyria, whose
forces had taken over Babylonia and sections of the east Mediterranean
coast during the eighth century BC. Although Kushite interference in
Palestine was to lead ultimately to the Assyrian conquest of Egypt, a
threat to the country's independence certainly existed. Fighting began
when an army composed of Egyptians and Nubians advanced into
southern Palestine in support of Hezekiah of Judah, and clashed with
the forces of Sennacherib at Eltekeh in 701 BC. The Egyptian army
was defeated, but this did not deter provincial rulers in Egypt from
supporting other foreign princes in their resistance to Assyria. Thus
provoked, the Assyrian king Esarhaddon turned his attention to the
THE THIRD I N T E R M E D I A T E P E R I O D 353
conquest of Egypt. A first invasion attempt in 674 BC was repulsed; a
second, led by Esarhaddon in person, succeeded. Memphis was cap-
tured, and Taharqo fled to Nubia, leaving his wife and son as captives
in the hands of the conquerors. Rather than attempt to govern Egypt
themselves, the Assyrians withdrew, having first required the rulers of
the Delta principalities to swear oaths to support Assyrian authority
and to prevent any attempt by the Kushites to regain control of Egypt.
Among these vassals was Nekau (Necho) of Sais, whose son Psamtek
(the future Psamtek I) was conducted to Nineveh to receive instruction
in Assyrian customs, before being returned to act as ruler of Athribis.
None the less, Taharqo quickly recovered control of Egypt. A resur-
gence of Egypto-Kushite power (with the possibility of further inter-
ference in Palestine) could not be tolerated by the Assyrians, and in
667 BC Ashurbanipal, son and successor of Esarhaddon, invaded
Egypt. Taharqo again fled to Nubia, and the Egyptian dynasts sub-
mitted to the Assyrians. A subsequent plot to reinstate Taharqo failed
and the Egyptian vassals who had been involved in it were executed.
Nekau of Sais had abstained from supporting the Kushites, and his
position was further strengthened by his appointment as governor of
Memphis.
Taharqo died in Nubia in 664 BC and was buried beneath a pyramid
tomb at Nuri, a new royal necropolis located opposite Gebel Barkal.
His successor, Tanutamani, promptly invaded Egypt and defeated the
Delta vassals who supported Assyria. This action brought very strong
retaliation from Nineveh. A large army was dispatched to Egypt; the
entire northern part of the country was quickly subdued, and the
Assyrians advanced as far as Thebes, which they sacked and plun-
dered. Tanutamani was expelled and returned to Nubia. The Kushite
rulers, although maintaining nominal claims to authority over Egypt
for several generations, were never afterwards able to put them into
effect. However, the bloodshed and destruction that followed from the
Kushite opposition to Assyria proved to be a cloud with a silver lining:
it emphasized the necessity for military and civil cooperation by the
rulers of the principalities, if independence was to be regained, and
brought to power an exceptional individual who possessed the
resources and abilities to liberate Egypt and lead it into a new phase.
Psamtek I of Sais, son of Necho, was among the vassal rulers left by
the Assyrians to control the provinces. During his long reign he threw
off the Assyrian yoke and succeeded, where the Kushites had failed, in
reuniting the whole of Egypt under his sole command. It is only at this
point that the Third Intermediate Period can be said to be at an end,
354 JOHN TAYLOR
with Egypt poised once more to reap the benefits of a centralized
government controlled by a strong king.
Religion and Material Culture in the Third Intermediate Period
While there appears to have been considerable continuity in the prac-
tice of the temple cult throughout the pharaonic period, two factors
distinguish its performance in the Third Intermediate Period—the
diminished importance of the king and the prominence of women in
cult activities. One aspect of the loss of the unique status of the king
(see above) was that the performance of the temple ritual—essential to
the preservation of the ordered universe—was no longer his sole
prerogative; from the end of the New Kingdom it was increasingly the
clergy who carried out this task. This, together with the hereditary
character of priestly office throughout this period, contributed greatly
to the solidarity of this section of society. Full-time priests were now
usual, and pluralism enabled them to amass lucrative offices. The
culmination of this trend was the unprecedented prominence of the
high priest of Amun during the 2ist-23rd Dynasties, at which period
his power was augmented by civil and military authority. However, as
has been noted above, the excessive influence of this individual had
a destabilizing effect on the country, and the primacy of the post
was eclipsed in the eighth century BC—religious authority at Thebes
becoming increasingly centred on the 'god's wife of Amun', while civil
and military powers were distributed to others.
Temple cult and personnel
The prominence of women in the temple cult was already well estab-
lished in the 2ist Dynasty, when several important religious offices
were fulfilled by the wives and daughters of the high priests at Thebes.
The most important of these posts was that of'first great chief of the
musical troupe of Amun'. While the precise religious significance of
this office is unclear, it is not a coincidence that these high-ranking
women also held titles relating to the importance of goddesses such as
Mut and Hathor, each of whom were held to be instrumental in
Amun's perpetuating the creative process, and hence in ensuring the
continuation of the COSMOS.
The post of 'chief of the musical troupe' disappeared during the
22nd Dynasty, and in its place there was a major development of the
office of'god's wife of Amun' (or divine adoratrice). Her principal
religious function was to stimulate the god's procreative urges, and
THE THIRD I N T E R M E D I A T E P E R I O D 355
thereby to ensure the fertility of the land and the cyclical repetition of
creation. In the Third Intermediate Period this post was usually filled
by the daughter of a king or high priest, who was installed at Thebes. In
contrast to the situation in the New Kingdom, when the office could be
held by the king's wife, the 'god's wives' of the Third Intermediate
Period were expected to be celibate, an innovation perhaps associated
with the establishment of the theocratic state. As noted above, there
was undoubtedly a political dimension to this. The rise of the 'god's
wife' coincided with the decline of the power of the high priest of
Amun, and may have been promoted as a measure to solve the
'problem' of Theban secessionism, for, while the 'god's wife' enabled
the distant royal house to be represented at Thebes, her celibacy meant
that no subdynasty could arise (successors being adopted). Conse-
quently, the status of the 'god's wife' continued to rise and the adop-
tion system persisted until the end of the 26th Dynasty.
The increase in importance of the 'god's wife' during the Third Inter-
mediate Period is clear: from the 23rd Dynasty her status begins to
approach that of the king, and in the 25th Dynasty she appears with
greater prominence on monuments. The iconography extends beyond
the traditional depiction of the 'god's wife' as shaking sistra. In the
reliefs at the Karnak Osiris chapels and those of the 'god's wives' them-
selves at Medinet Habu they are seen in roles previously reserved for the
king: offering to deities (including presenting maat), being embraced by
gods, libating the image of the god, conducting foundation ceremonies,
and receiving attributes of kingship from the gods. Thus Amenirdis I
receives jubilee symbols from Thoth, Shepenwepet I has her headdress
adjusted by Amun, is suckled by a goddess, and is even shown wearing
two double crowns simultaneously, a unique depiction. As fragmentary
reliefs from North Karnak show, the 'god's wife' could even celebrate
the sed-festival, otherwise only attested for the king.
The 'god's wife' was the head of a 'domain of the divine adoratrice'.
This employed a substantial personnel, including 'singers of the
interior [chambers] of Amun' (celibate priestesses who were some-
times of high rank); inscriptions mention one who was a daughter of
Takelot II and another whose father was a Delta Libyan chief. The
domain also included priests and scribes and was headed by a 'chief
steward'. Through the escalating importance of the 'god's wife' and her
entourage, these stewards became powerful and influential figures at
Thebes towards the end of the 25th Dynasty (as their elaborate tombs
in the Asasif testify), and they were to play a key role in the reintegra-
tion of the south into a fully unified Egypt under the 26th Dynasty.
356 JOHN TAYLOR
It is no coincidence that the prominent role played by high-ranking
women in religious cults in the 2ist Dynasty was often in connection
with child-gods such as Horpakhered or Khons. Among their many
titles these ladies were 'nurses' or 'divine mothers' of these gods, and
the Third Intermediate Period marks the early stages in the growth of
emphasis on the mother-child relationship in Egyptian religion,
which was to become one of the pervading aspects of life in Egypt
during the remainder of the first millennium BC. A major manifesta-
tion of this 'mammisiac' religion is the importance increasingly
attached to divine triads, with the child-god (identifiable with the king)
as offspring of two other deities. Two of the most prominent of these
triads were those composed of Isis, Osiris, and Horus and of Amun,
Mut, and Khons, both already important in the Third Intermediate
Period. The rise in importance of Osiris at this time is clear from the
development of cult places dedicated to him at Thebes. Among the
most familiar images from ancient Egypt which first come to prom-
inence in the Third Intermediate Period are those of Isis nursing
Horus, and the child Horus standing on crocodiles, triumphing over
harmful forces (found chiefly on the magical stelae known as dppi).
The rise in the importance of these deities—and particularly myths
about the childhood of Horus in the marshes of the Delta—may be due
in part to the dominant influence of Delta-based rulers at this period.
Indeed, close links between mammisiac religion and the kingship are
apparent; several rulers from Sheshonq I to Taharqo were depicted in
temple reliefs and on small objects as nude children suckled by a
goddess (such as Hathor or Bastet)—a scene that symbolized the
transfer of kingship to a new ruler, rebirth being considered an appro-
priate metaphor for this rite of passage.
Throughout the Third Intermediate Period the cult of the Apis bull
of Memphis was maintained irrespective of the repeated changes in
authority over the city, as the burials in the Serapeum at Saqqara, with
their abundant votive stelae, testify. It is at this time, too, that the asso-
ciation of certain animals with other deities first becomes marked—a
trend that was to culminate in the animal cults of the Late Period with
their legacy of vast numbers of bronze votive statuettes, and catacombs
filled with millions of animal and bird mummies.
Burial practices
The political and cultural developments occurring in Egypt during this
period are amply reflected in the manner of treatment accorded to the
dead. Particularly noticeable are changes in the location of burials and
THE THIRD I N T E R M E D I A T E PERIOD 357
in the types of tombs. For the elite, the old physical isolation of the
necropolis was replaced by burial within the enclosure of a cult temple.
Since the royal tombs at Tanis are the earliest (and best-documented)
examples, this trend may have been an innovation by the 2ist-Dynasty
kings, and was perhaps motivated partly by the intention to make
Tanis a northern counterpart to Thebes. While the practice is most
conspicuous for kings, it also extended to persons of high rank—the
high priests at Memphis, whose tombs were constructed on the edge
of the precinct of the temple of Ptah; Queen Kama, buried at Leon-
topolis near Bubastis; a high official buried adjacent to the enclosure
wall of the temple at Tell Balamun. Whether or not the tendency had a
Delta origin, it was soon manifested at Thebes, where high officials
began to be buried within the precincts of Medinet Habu and the
Ramesseum. These locations, besides offering greater security from
robbery, were a means of establishing closer proximity to the gods. The
siting of the burials of'King' Harsiese and the later 'god's wives' at
Medinet Habu may also have been influenced by local cult activities:
during the Third Intermediate Period, the Small Temple there became
closely associated with the 'Mound of Djeme', where rituals relating to
the creative powers of Amun took place.
The tombs themselves were much simpler structures than those of
the New Kingdom. The period witnessed an interruption of the tradi-
tion of expending substantial resources on elaborate superstructures
and labyrinthine rock-cut sepulchres. Both royal and elite tombs were
reduced to small subterranean burial chambers with modest chapels
directly above. Private tomb chapels are not well documented archaeo-
logically, and seem to have been rare. Some have doubtless dis-
appeared through poor preservation, yet there is little evidence for
their existence outside the principal centres of Tanis, Memphis, and
Thebes. The rarity of individual chapels coincides with a rise in the
number and scale of communal burials, usually situated in older
tombs or disused religious structures. The gathering of the New
Kingdom royal mummies and the 2ist Dynasty priests of Amun into
caches in older tombs during the eleventh and tenth centuries BC
seems to mark the beginning of this pattern. Throughout the period,
persons of all ranks were buried in groups at sites throughout Egypt
(there are examples at Saqqara, Herakleopolis, Akhmim, Thebes, and
Aswan), and, where prosopographical data exist, as at Thebes, such
grouping can often be shown to be family orientated.
There was a significant reduction in the quantity and range of burial
paraphernalia. The fittings of the tomb chapel (such as statues and
358 JOHN TAYLOR
offering tables) all but disappeared, as did household furniture, cloth-
ing, tools and weapons and occupational equipment, musical instru-
ments, games, and stone and pottery vessels. Apart from a small stele,
usually of painted wood, the burial equipment was generally confined
to a narrow range of purely funerary items—coffins, canopic contain-
ers (usually dummies), amulets, shabtis, and funerary papyri (one
usually secreted within an Osirian statuette). The period is also
marked by a steady decline and, eventually, a break in the tradition of
providing funerary texts. While elite burials at Thebes in the 2ist
Dynasty continued to use the Book of the Dead, and even added the
Amduat and Litany ofRa to the non-royal repertoire, these traditions
were allowed to atrophy in the 22nd Dynasty. Funerary papyri ceased
to be produced and texts on coffins were reduced to little more than
repetitious offering formulas and speeches of gods, with a correspond-
ing simplification of the iconographic repertoire.
These factors seem to reflect major changes in attitudes to death and
burial in the Libyan Period. The lack of imposing tomb superstruc-
tures (even the most elaborate could have been built quickly) indicates
that burial was no longer so carefully anticipated and prepared for. The
ad hoc nature of tomb construction (roughly pieced together, often
from reused blocks) supports this, and, significantly, this description
applies especially to tombs in the Libyan-dominated north and in
middle Egypt: at Tanis, Memphis, Leontopolis, and Herakleopolis.
Substantial items of burial equipment such as stone sarcophagi were
almost confined to royalty, and even these few examples were mostly
reused from earlier periods. This recycling of funerary objects
extended to less costly items—notably in the 2ist Dynasty, when exten-
sive reuse of coffins occurred at Thebes. Yet Egypt did not lack material
wealth, and the decentralization of the land in no way brought a
decline in the skills of sculptors, painters, and metalworkers (see
below). The changed attitude to the dead that these factors suggest may
perhaps be more directly associated with the presence of the Libyans in
society. Constructing an elaborate physical environment for the dead
and a focus for mortuary cults was doubtless not a major feature of
semi-nomadic societies such as theirs. Significantly, it was only with
the imposition of authority by the Kushite rulers—whose devotion to
the ancient traditions of Egypt was of a rather purist kind—that a
revitalization of burial practices along traditional lines occurred.
This shift of emphasis away from the physical housing of the dead
brought with it a greater concentration on the body itself and its imme-
diate trappings. Mummification reached its peak in the 2ist Dynasty
THE THIRD I N T E R M E D I A T E P E R I O D 359
and high standards of preparation were maintained throughout the
succeeding period. Among the innovations were the use of subcu-
taneous packing to restore the shrunken features to lifelike form, more
elaborate cosmetic treatments, with hair carefully arranged and finger
nails meticulously preserved, and a more careful preservation of the
viscera, which were individually wrapped and returned to the body
cavity (canopic jars continued to be provided but were often mere
dummies). These techniques manifest a desire to make the body as
complete and perfect as possible. Its status as an idealized image of the
transfigured deceased was developed, and its security further ensured
by an increase in the number of coffins per burial—at least two and
sometimes as many as four.
The decline in the production of individual tomb chapels with
elaborate wall decoration led to a relocation of essential funerary
images and texts on the surfaces of the coffin and on papyri. Hence
21st-Dynasty coffins were covered inside and out with a densely packed
profusion of images. The priests of Thebes created a rich new reper-
toire of funerary iconography promoting the concept of rebirth
through the combined mythologies of Osiris and the sun-god, and the
images were devised with a view to concentrating multiple levels of
meaning in a single complex scene. In keeping with the caching of
burials and the general impermanence of the resting place at this time,
the coffin took on the religious functions of the tomb, as it had under
similar circumstances in the First Intermediate Period. By the end of
the Third Intermediate Period the evolution of the surface imagery
had given still greater prominence to the concept of the coffin as a
miniature universe, with the deceased at the centre, identified (through
the texts and imagery of the coffin) as the creator-god, and hence the
source of his own resurrection.
Burial practices also lend support to the notion of a north-south
division in the population and material culture of Egypt during this
period. Although Delta sites (apart from Tanis) have yielded few
burials datable to these centuries, evidence from the Memphite and
Faiyum areas may be usefully compared with the more abundant
material from the south. Of the limited range of burial goods provided
in Third Intermediate Period graves, only coffins were used con-
sistently throughout. Study of these hints at interaction between north
and south, notably at the beginning of the 22nd Dynasty, when a major
change in coffin style is attested at Thebes. This is apparent in the
abandonment of the style in vogue in the 2ist Dynasty, with its horror
vacui and multi-level images, and the rapid establishment in its place
360 JOHN TAYLOR
of a new range of types—polychrome cartonnage cases enclosed in
wooden coffins of much simpler design. These show an impoverish-
ment of the iconographic repertoire, with greater concentration on
symmetrical arrangements of gods, yet with a bolder use of colour.
There are indications that these features derived from the north, as
burials from the Memphite necropolis and from cemeteries around
the entrance to the Faiyum testify. The evident importation of north-
ern burial practices into Upper Egypt seems to coincide with the
imposition of stronger royal authority over the south during the reigns
of Sheshonq I and his successors. Yet during the succeeding period,
distinctively northern and southern styles of coffin seem to emerge,
probably reflecting the progressive decentralization of Egypt and
perhaps also the social division hinted at by other evidence.
Towards the end of the Third Intermediate Period there was a
marked return to older established traditions, coupled with innova-
tions. Elaborate tombs for the elite began once more to be constructed.
The Theban necropolis shows an evolution from tombs with modest
superstructures in the late eighth century to the gigantic complexes
built for Mentuemhat and his contemporaries at the end of the 25th
Dynasty. These have free-standing superstructures and elaborate sub-
terranean apartments, and the scale and craftsmanship of the monu-
ments indicate that preparations for death were once again being
taken seriously. The range of burial equipment increased; the develop-
ment of coffin styles produced new types, combining the revival of old
features with innovations—rectangular outer cases represent a shrine
or the tomb of Osiris, inner coffins project a new image of the trans-
figured deceased, closely resembling a statue, with back pillar and
pedestal. Shabtis followed a parallel course of development and statu-
ettes of the composite deity Ptah-Sokar-Osiris (also in this shape)
entered the funerary assemblage, ultimately to become one of the com-
monest features of Late Period burials. Functional canopic jars also
returned and, more importantly, funerary literature enjoyed a revival.
A revised Book of the Dead in the new so-called Saite recension (actually
an achievement of the 25th Dynasty) was inscribed on papyri and
coffins, while the archaizing fervour of the period led to the copying of
passages from the Pyramid Texts and their addition to the current
repertoire. This last excepted, Thebes seems to have been a major
centre for these innovations, which spread northwards during the
seventh century BC. This is not to deny that comparable developments
may have been occurring in other areas, but the local chronology at
sites such as Memphis is less clear.
THE THIRD INTERMEDIATE PERIOD 361
Artistic developments and technology
Despite the decentralization of Egypt, the products of the craftsmen
show no appreciable reduction in skill or expertise. It is true that stone
sculpture on a large scale remained rare throughout, but work of
unparalleled excellence was produced on a more modest scale, as new
emphasis was given to craftsmanship in the old but underdeveloped
media of metal and faience. All media reflect the progressive archaiz-
ing tendencies alluded to above, with the consequence that the influ-
ence of Old, Middle, and New Kingdom models becomes increasingly
apparent with the passage of time.
There was a reduction in the range of sculptural types. Royal statues
in stone are particularly rare—those of the 2ist Dynasty kings were
usurped from earlier rulers, and, although original works were pro-
duced in the 22nd and 23rd Dynasties most of the surviving examples
are of modest size. It was only under the Kushites that substantial and
powerful royal sculpture returned: the granite head of Taharqo in
Cairo, and the sphinx from Kawa in the British Museum, are among
the most striking examples. However, during the 22nd to 25th
Dynasties large numbers of statues of officials were dedicated in
temples, some of which are of exceptionally fine work. Among these
the block statue was notably popular, as were those forms in which the
subject is represented supporting a shrine, stele, or image of a deity
(naophorous and stelophorous statues). The fine reliefs of Sheshonq I
from el-Hiba and of Osorkon II from Bubastis show that two-
dimensional work of high quality was still being produced, although
the subject matter of the scenes was largely derivative. Painting also
flourished, and at Thebes the rich New Kingdom tradition of tomb
decoration was replaced by work of high standard on coffins, stelae,
and funerary papyri.
Perhaps the most lasting contribution of the Third Intermediate
Period to the arts and crafts lay in the field of metal working. The silver
coffins of kings Psusennes I and Sheshonq II and the wide range of
gold and silver vessels and jewellery from the Tanite royal tombs testify
to the continued expertise of Egyptian metalworkers, although foreign
influence is occasionally apparent in the shapes and decoration of
vessels. Of greater significance was the huge expansion of the range
and technical excellence of metal sculpture that occurred during this
period, some of it in gold and silver, but the major part in bronze.
These pieces were often exquisitely finished, and brilliant effects were
achieved through the embellishment of the surfaces with strands of
precious metal hammered into channels in the bronze. Solid-cast
362 JOHN TAYLOR
statuettes were frequent, and now began the tradition of small bronze
deity figures that led to the production of thousands of examples
during the succeeding centuries. More important were the large
bronze statues, hollow-cast using the lost-wax technoque, which were
dedicated as votive offerings or mounted on the portable barques of the
gods. The figure of the 'god's wife' Karomama in the Louvre is the
supreme example of the type, although a series of bronze Osiris
statues, now represented only by decayed and incomplete specimens,
may originally have been equally imposing. These statues, made in the
ninth to seventh centuries BC, represent the earliest known attempts to
create large bronze figures by the hollow-cast process, and were to
serve as important influences on early Greek bronzeworking. Classical
authors state that Samian craftsmen used Egyptian techniques in
creating the earliest large hollow-cast metal sculptures in the Greek
world, and this view is vindicated by the discovery of Egyptian bronzes
of this period on Samos itself.
Scarcely less vigorous was the production of faience. While glass
technology declined after the New Kingdom, that of faience boomed.
The majority ofshabtis of the period were made of this material, but a
great many were crudely fashioned. Much finer were the lotiform
chalices with relief scenes showing country life or the king in battle.
The shape of these chalices evokes the notion of rebirth, and the scenes
on them and on a related series of faience openwork spacer beads
reflect aspects of the mythology of creation. Equally typical of the
period are magical figurines designed to provide protection during
childbirth and nourishment for the young—these are of blue-green
faience, often with spots and details added in brown, and typically
show the household god Bes, a monkey, or a nude female holding a
vase or musical instrument, or sometimes suckling. Although they
occur as far south as el-Kurru in Nubia, the concentration of finds of
these figures at north-east Delta sites indicates that this was their main
area of production.
Conclusion
As noted at the beginning of this chapter, the pejorative implications
of the term 'intermediate' do little justice to the developments that took
place in Egypt between 1069 and 664 BC. Although the power struc-
ture within the country was very different from that which obtained in
the New Kingdom the towns and cities of Egypt flourished and the
economy of the country was generally healthy. Though decentralization
THE THIRD I N T E R M E D I A T E P E R I O D 363
of the government led to occasional power struggles, the system adop-
ted by the Libyan pharaohs and modified by the Kushites was generally
effective. Large-scale royal constructions may have been restricted, but
artistic continuity was maintained via other media (small sculpture,
metalwork, faience).
In a large degree, the Third Intermediate Period constitutes a dis-
tinct cycle in Egypt's history, defined by a passage from the loss of
unity at the end of the New Kingdom to the restoration of centralized
authority under Psamtek I. Valuable lessons were learned from the
fragmented politics of the period (particularly from the Assyrian
invasions). These provided the impetus needed to restore centralized
authority, and proved the ideological worth of archaism and the
political value of institutions such as that of the 'god's wife of Amun' in
fostering a stable and less turbulent state. The related changes in the
status of the king and the prominence given to new trends in religion
were also adumbrations of the future. Thus this period laid the founda-
tions for the last great phase of ancient Egypt's prosperity.
The Late Period (664-332 BC)
ALAN B. LLOYD
Egyptologists have generally been dismissive of the Late Period,
regarding it all too often as the last gasp of a once great culture. Such
views seriously devalue the historical achievement of these centuries
as well as the remarkable vitality that pharaonic civilization continued
to display. The student of this age has also a unique advantage. In
earlier periods we have to rely largely or exclusively on Egyptian
evidence, with all its inherent distortions, but the historian of the Late
Period disposes of a much broader range of written evidence, which
offers unparalleled potential for cross-reference and thereby provides
insights into the workings of Egyptian political and military institu-
tions stripped of the propagandist veneer invariably applied to histor-
ical narrative by native Egyptian scribes.
The centuries under discussion break down into four clearly defined
phases: the Saite Dynasty (664-525 BC); the First Persian Occupation
(525-404 BC); a period of independence (404-343 BC); and the Second
Persian Occupation (343-332 BC).
The Saite Dynasty: The Resurgence of Egypt's Power
The Saite reunification of Egypt in the mid-65os BC reversed a long-
running trend in the country's history in that all recent precedents
pointed imperiously to continued fragmentation punctuated by bouts
of foreign domination. The years following the end of the 2Oth Dyn-
asty had brought the disintegration of the kingdom under a variety of
pressures: the weakness of the last Ramesside rulers provoked the
collapse of centralized government; the development of the power of
13
THE LATE PERIOD 365
the priesthood of Amun-Ra at Thebes created a formidable rival to
royal authority; and the infiltration of the country by Libyans rapidly
led to their ascendancy in the social and political hierarchy. Not sur-
prisingly, vigorous Libyan princelings had experienced little difficulty
in getting their hands on the royal office, thus creating a sequence of
dynasties of varying efficiency. Later, the tangled web of the 25th
Dynasty—characterized by intermittent Nubian domination—covered
the best part of 100 years. Although the 25th Dynasty started well, it
ended with the country suffering severely from the Assyrian invasions
of 671 and 663 BC.
The founder of the 26th Dynasty, heir to this legacy, was, therefore,
confronted by several problems: the ancient ideal of Egypt as a unified
kingdom had been severely eroded by the rivalry of opposing power
blocks in the form of the priesthood of Thebes and Libyan dynasts; this
diffusion of power generated economic weakness and was, at the same
time, aggravated by it; finally, the ambitions of Asiatic enemies and
Nubian kings to regain control of Egypt posed a recurrent external
threat. Any attempt to recreate a powerful and united Egyptian state
was dependent on the eradication, or at least neutralization, of these
factors. In this the 26th Dynasty achieved spectacular success, which
was to be crowned with nothing less than the resurgence of Egypt as a
major international power.
The credit for reunifying Egypt falls to Psamtek I (664-610 BC),
whose father Nekau I (672-664 BC) had previously ruled at Sais under
Assyrian protection and had been killed for his pains by the Nubian
King Tanutamani (664-656 BC) in 664 BC. Psamtek succeeded to his
father's position with Assyrian support, initially controlling about half
the Delta with his main centres of power at Sais, Memphis, and
Athribis, as well as close religious links with Buto. The Assyrians evi-
dently saw this development as a continuation of the old system of rule
through local princes, but the sands were swiftly running out for such
power as Nineveh had in Egypt. Given their pressing commitments
elsewhere in the Empire, the Assyrians simply did not have the
military strength to maintain their position indefinitely so far west.
With typical Saite strategic acumen, it did not take Psamtek long to
exploit this situation, so that relations with Assyria quickly took a very
different turn, and in about 658 BC we find him receiving support from
Gyges of Lydia in emancipating himself from Assyrian control, an epi-
sode that may well be linked with Herodotus' tradition that Psamtek
employed Carian and Ionian mercenaries in his efforts to strengthen
and extend his authority. In addition to military power, our sources
366 ALAN B. LLOYD
highlight a further dimension to his strategy: strengthening his eco-
nomic base by developing trade links with Greeks and Phoenicians. It
was evidently firmly grasped by this formidable ruler that all power
must be based on a sound exchequer.
By 660 BC Psamtek had control of the entire Delta, and from this
potent military base he was able to gain mastery of the rest of the
country by 656 BC, mainly, it would seem, by diplomatic means,
although the wheels of diplomacy were certainly oiled by the obvious
availability of a substantial well-equipped military force of none-too-
scrupulous foreign mercenaries. He also benefited substantially from
the well-honed pliability of local princes such as the Shipmasters of
Herakleopolis Magna and Mentuemhat of Thebes, who quickly saw
the advantages of coming to an accommodation. At least equally press-
ing was the problem of gaining control of the powerful priesthood of
Amun-Ra at Thebes, which had been a significant factor in weakening
royal authority since the late New Kingdom. Here a major step in
resolving the difficulty was taken when Psamtek arranged for his
daughter Nitiqret to be appointed as heiress to the 'god's wife of Amun',
thereby initiating a process intended to place the major southern
repository of ecclesiastical power firmly in the hands of the dynasty.
Power gained is one thing; power maintained is quite another, but
the process of consolidation was carried out with triumphant success.
A major contribution was made by the mercenaries who had played
such a significant role in the conquest of the country. Our documenta-
tion lays much emphasis on those of Greek and Carian extraction, but
we also hear of Jews, Phoenicians, and possibly Shasu Bedouin. These
troops had two functions. In the first place, they were intended to
guarantee Egypt's security from external attack in the face of a series of
enemies, initially Assyrians and subsequently Chaldaeans (Babyloni-
ans) and Persians. However, they also undoubtedly provided a
counterweight within the country to the power of the machimoi, the
native Egyptian warrior class, who were, in origin, Libyans and posed a
significant potential internal threat to royal authority.
Herodotus informs us that stratopeda ('camps') were established
between Bubastis and the sea on the Pelusiac branch of the Nile. He
claims that these camps were occupied without a break for over a
century until the mercenaries were moved to Memphis at the begin-
ning of the reign of Ahmose II (570-526 BC), but the archaeological
evidence presents a rather more complex picture. At Tell Defenna
(Greek Daphnae) the earliest king exemplified is certainly Psamtek I,
but the vast majority of the material dates to the time of Ahmose II—
THE LATE P E R I O D 367
that is, the distribution contradicts the Herodotean tradition. We also
know of another camp 20 km. from Daphnae, a little to the south of
Pelusium, where sixth-century Greek pottery has been found in quan-
tity. The most plausible explanation for the contradiction between our
literary and archaeological evidence is that the troops were pulled out
of the camps at the beginning of Ahmose's reign as the result of an
anti-Greek backlash (see below), but reintroduced at a later stage to
counter the growing menace of Persia. As for their integration into the
Egyptian army, the famous Greek inscription on the leg of one of the
colossi at Abu Simbel, as well as later practice, indicates that the
mercenaries, under Egyptian command, formed one of the two corps
in the army whose supreme commander was also Egyptian. It has to be
said that these troops were not consistently reliable, and we do have
evidence of a revolt of mercenaries at Elephantine during the reign of
Apries (589-570 BC).
Petrie's work at Tell Defenna has provided a vivid and probably
typical picture of the character of the permanent bases of such troops in
the Saite period. The site is located on a large plain covered with pottery
and dominated by the remains of a mud-brick platform constructed on
the standard honeycomb principle consisting of casemates many of
which were filled with sand. Its original height was estimated by Petrie
to have been about 30 feet (c.io m.), and he believed that it had been
surmounted by a fort. This structure, which was certainly built by
Psamtek I, seems to have functioned as a keep within an enclosure
demarcated by a massive oblong mud-brick wall, but this had been
eroded to ground level by Petrie's time. Outside the wall lay the civilian
settlement, mainly to the east. Excavation yielded a substantial quantity
of Greek infantry equipment, but the site was also a naval base from
which Greek-style war galleys could operate, a situation reflecting the
important role played by the mercenaries in the Egyptian navy.
Not surprisingly, the preference shown to these foreign troops was
far from welcome to the machimoi. According to Herodotus, a group of
them mutinied and withdrew from Egypt to a site that may well have
lain somewhere in the vicinity of the Blue Nile and Gezira area near
Omdurman, if we can trust his topographical data. By the time of
Apries, things had got far worse and eventually reached a disastrous
level when we find the king being swept from the throne by a machimoi
backlash against the privileged position of Greeks and Carians in the
military establishment. The spark that lit this powder keg was a dis-
astrous defeat sustained by a force of machimoi sent against the Greek
city of Gyrene, which provided the opportunity for Ahmose to use
368 ALAN B. LLOYD
these troops to defeat Apries' mercenaries at Momemphis in 570 BC
and usurp the throne of Egypt.
The economy was an equally important focus of Saite policy in
reconstructing Egypt. The foundation of a sound economy in the
country was, and always has been, sound agriculture, and by Ahmose's
time this had been raised to a spectacular level of success. Herodotus
(2.177. i) comments, 'It is said that it was during the reign of Ahmose
II that Egypt attained its highest level of prosperity both in respect of
what the river gave the land and in respect of what the land yielded to
men and that the number of inhabited cities at that time reached in
total 20,000.'
Trade was also greatly encouraged. In our textual sources, Greek
relations play a major role, although it would be as well to remember
that most of the sources are themselves Greek. Within Egypt itself
we hear of trading stations such as 'The Wall of the Milesians' and
'Islands' bearing such names as Ephesus, Chios, Lesbos, Cyprus, and
Samos, but their precise relationship to the Crown or other Greek
centres in the country is quite unclear for the earliest period. However,
by far the best-documented trading centre is Naukratis, established on
the Canopic branch of the Nile not far from the capital, Sais, and with
excellent communications for internal and external trade. Although
the city was founded by Milesians in the mid- or late seventh century
BC, members of other East Greek cities were also firmly established
there, as well as traders from the island state of Aegina in the Saronic
Gulf south of Athens. Excavation has revealed a series of sacred enclos-
ures dedicated to Greek cults, a scarab factory producing material for
export, and a typical Late Period honeycomb platform comparable to
that at Tell Defenna, which may have been military in purpose but
could equally well have had civilian, administrative functions.
It is difficult to determine to what extent trade was regulated in the
early years of the foundation. It may be that from the very beginning
the model of Mirgissa in Nubia during the Middle Kingdom applied.
This system is summarily described in the stele of the eighth year of
the reign of Senusret III as follows:
The southern frontier made in regnal year 8 under the majesty of the King of Upper
and Lower Egypt Khakaure (may he live for ever and ever) in order to prevent it
being passed by any Nubian journeying north by land or in a kai-boat as well as any
livestock belonging to Nubians, with the exception of a Nubian who shall come to
traffic at Mirgissa or on an embassy, or on any matter which may lawfully be done
with them; but it shall be forbidden for any fca/-boat of the Nubians to pass north-
wards beyond Semna for ever.
THE LATE P E R I O D 369
Be that as it may, there is no doubt that Naukratis became the
channel through which all Greek trade was required by law to flow
from £.570 BC. However, there is evidence of even more strenuous
efforts to promote trade; we know that Nekau II (610-595 BC) at the
very least began to construct a canal running from the Nile to the Red
Sea, an activity that must indicate a revival of economic activity in the
Red Sea area, which had been a major focus of commercial concern in
earlier dynasties. It is also reasonable to regard the existence of the
implausible Herodotean narrative of a circumnavigation of Africa
instigated by Nekau II as a further reflection of interest in this quarter.
Impressive and even spectacular though these measures may have
been, we must never lose sight of the simple fact that big battalions
and a full exchequer can never be a sufficient basis for lasting power.
There must always be an ideological underpinning that is acceptable to
the subject people. In Egypt the basis for this had always been the
concept of divine kingship that gave the pharaoh a clearly defined and
universally accepted role, not only in the governance of the kingdom
but in the very maintenance of the cosmos itself. This agenda had to be
accepted and rigorously observed; to be a legitimate pharaoh it was
essential to act legitimately. I have summarized this pharaonic ideal
elsewhere as follows:
The basic elements are: pharaoh ascends the throne as Horus, champion of cosmic
order (maat) and vanquishes the forces of darkness; in continuation of this role he
then ensures the well-being of Egypt in economic terms by organizing the irrigation
system and in military terms by maintaining its military forces and defeating its
external foes; the pax deorum is ensured by supplying temples with all their require-
ments and by constructing monuments both for the gods and for himself (statues
and mortuary installations); expeditions will be sent to Punt, Sinai and other
canonical sources of raw materials and in the course of these operations the gods
will indicate their approval of the king by biayt, 'marvels', which may consist both of
the conspicuous success of the enterprise and of any signs or omens which the gods
may choose to provide. The result of all this will be long life for the king and the
realization of the will of the gods in the establishment of the cosmic order on
earth. (Herodotus Book II. Commentary 2.16-17)
Psamtek I was well placed here, but, at the same time, burdened
with a heavy responsibility. He was undertaking one of the most crit-
ical roles of kingship, donning the mantle of Menes and Mentuhotep
II: he was unifying the country and restoring the proper order of
things, the state of being that the Egyptians called maat. This emerges
with crystal clarity at the beginning of the preserved section of the
Nitiqret Adoption Stele, the longest surviving royal inscription of his
reign:
370 ALAN B. LLOYD
I [Psamtek] have acted for him as should be done for my father. (2) I am his first-born
son, one made prosperous by the father of the gods, one who carries out the rituals of
the gods; he begat him for himself so as to satisfy his heart. To be 'god's wife' have I
given him my daughter, and I have endowed her more generously than those who
were before her. Surely he will be satisfied with her adoration and protect the land of
(3) him who gave her to him ... I will not do that very thing which ought not to be
done and drive out an heir from his seat inasmuch as I am a king who loves (4)
truth—my special abomination is lying—the son and protector of his father, taking
the inheritance of Geb, and uniting the two portions while still a youth. (11.1-4)
This devotion to the gods could not be confined to statements of
intent. Both Psamtek and his successors engaged in architectural work
on sacred installations to express their devotion and maintain the
goodwill and support of the gods. Saite buildings are poorly preserved
in the archaeological record, to a considerable extent because they were
constructed in the Delta, where conditions for survival are much less
favourable than in Upper Egypt. Nevertheless, enough information is
preserved in Herodotus, inscriptions, and the building fragments to
demonstrate that the Saite rulers did everything they could to fulfil this
part of the agenda of kingship. It is claimed that Psamtek I constructed
the south pylon of the temple of Ptah at Memphis and also built on
behalf of the Apis bull in the same shrine; his successor Nekau II is
known to have been responsible for monuments in honour of Apis in
the same city, and there is inscriptional evidence of his endeavours in
the limestone quarries in the Mokattam Hills, where Psamtek II
(595-589 BC) has also left signs of quarrying work. Ahmose II was also
extremely active in Sais, the home of the dynasty, where he erected a
pylon for the temple of Neith, set up colossal statues, and manu-
factured human-headed sphinxes for a processional way. Indeed, the
evidence leaves us with a powerful impression of the ecclesiastical
splendours of this city in the Late Period that must have owed much to
the work of these Saite kings. The chief focus was the sacred enclosure
of Neith, which contained the main cult centre (the 'Mansion of Neith')
and provision for a host of associated gods (Osiris, Horus, Sobek,
Arum, Amun, Bastet, Isis, Nekhbet, Wadjet, and Hathor). There was,
in particular, a burial place of Osiris and a sacred lake on which the
rituals of the Festival of the Resurrection of Osiris were celebrated, and
the site was richly embellished with features such as obelisks of which
the sad ruins of Sais give little hint today.
The city of Sais was, however, just one recipient of 26th Dynasty
largesse. We also hear, for instance, of Ahmose setting up colossi at
Memphis (two of granite), building a temple of Isis in the same city,
THE LATE P E R I O D 371
and working at Philae, Elephantine, Nebesha, Abydos, and the oases,
while he also made contributions to earlier structures on many other
sites, including Karnak, Mendes, the Tanta area, Tell el-Maskhuta,
Benha, Sohag, el-Mansha, and Edfu. This intense building activity is in
turn reflected in quarry inscriptions at Tura and Elephantine.
The ideology of kingship not only encompasses the world of the
living but also gives the king a critical function beyond the grave: the
living king is the embodiment of Horus and rules the living; the
deceased king is Osiris, king of the dead, but, at the same time, since
Osiris in this context was assimilated to Ra, the king expected to
participate in the cycle of cosmic action. In order to propel the king
into his life beyond the grave and maintain him there, an elaborate
programme of ritual was devised, the most spectacular surviving illus-
trations of which are the pyramids of the Old and Middle kingdoms
and the New Kingdom tombs in the Valley of the Kings with their
attendant cult temples. The rulers of the 26th Dynasty built no funer-
ary monuments as spectacular as these but operated firmly within Late
Period tradition. From the end of the New Kingdom, kings had been
buried in chapel tombs in temple courtyards, partly, no doubt, for
security reasons, but also possibly as a reflection of a sense of depend-
ence on and devotion to the deities in question. Following this practice,
the kings of the 26th Dynasty were interred in chapel tombs in the
courtyard of the temple of Neith at Sais. None of these structures has
survived, but there is no difficulty in reconstructing them from the
description of Herodotus and obvious earlier parallels at Medinet Habu
and Tanis. They consisted of two elements: above ground a mortuary
chapel was constructed that was entered by way of a double door from a
columned portico. The walls of this structure were probably decorated
with painted relief sculpture relating to the mortuary cult of the
deceased king that was celebrated in the chapel. Beneath was the burial
vault containing the royal sarcophagus, and this too was probably
decorated. Grave goods, to judge from Tanite precedents, would have
been relatively restricted, but certainly included the traditional royal
shabtis and canopic jars.
To date in this chapter we have concentrated largely on Saite policies
and actions within Egypt, but, given the grim history of recurrent inva-
sion in the 25th Dynasty, we cannot be far wrong in assuming that the
major issue for the rulers of this period was the task of keeping the
frontiers of Egypt free from foreign invaders. The most critical area
was Asia, where initially the problem was the defence of Egypt's border
against a possible renewal of Assyrian attempts to gain control of
372 ALAN B. LLOYD
Egypt, but difficulties much closer to their homeland made this impos-
sible for the Assyrians to achieve. While evidence of Egyptian military
activity in Asia at this stage is far from plentiful, Psamtek's operations
clearly met with considerable success, despite the setback of a horde
invasion of the Near East by Cimmerian barbarians in ^.630 BC, which
he countered with the eminently sensible expedient of buying them
off. We hear of a successful, if protracted, siege of Ashdod (probably
^.655-630 BC), and late in his reign we encounter Egyptian forces
operating in Asia even further afield than in the heady days of the i8th-
Dynasty rulers Thutmose I and III. This startling phenomenon was
the consequence of the double threat to Assyria's very existence posed,
on the one hand, by the rise of the Chaldaeans in southern Iraq and, on
the other, by the growing menace of Media to the east in Iran. This
speedily led to an abrupt Assyrian volte-face in relation to Egypt, in the
form of an alliance between the two nations as a result of which we find
Egyptian forces operating against the Chaldaeans inside Iraq itself in
616 BC. Henceforth, until the last decades of the 26th Dynasty, it was
the Chaldaeans who were the major enemy of Egypt.
Psamtek's successor, Nekau II, continued his father's policy in the
north. Initially things went well, and again we are confronted with the
spectacle of Egyptian forces campaigning east of the Euphrates against
the Chaldaeans, defeating en passant Josiah of Judah in 609 BC. The
result was that the Egyptians were able to establish themselves on the
Euphrates for a short while, but this position was soon lost in 605 BC as
a result of their catastophic reverse at Carchemish, which was followed
by a brusque retreat to the eastern frontier of Egypt. The Egyptians
kept the Chaldaeans at bay, and on this occasion the border was not
breached. A small recovery seems to have been made in the reign of
Psamtek II, who was certainly able to mount some sort of expedition
into Palestine during the fourth year of his reign. In addition, his
diplomacy helped foment a general Levantine revolt against the
Babylonians that involved, amongst others, Zedekiah of fudah. Hero-
dotus makes it clear that the Near Eastern operations of these rulers
were by no means entirely land orientated, indicating that Nekau
constructed a fleet of ramming war galleys that may have been an early
type of trireme and some of which were used in the Mediterranean and
others in the Red Sea. Indeed, it may be that the abortive Red Sea canal
was intended, in part, to facilitate the transfer of naval forces from the
Red Sea to the Mediterranean as circumstances required.
Apries addressed himself vigorously to the Chaldaean problem.
Initially he undertook large-scale operations against the Chaldaeans in
THE LATE P E R I O D 373
conjunction with Phoenician cities and Zedekiah of Judah. These
activities led to disaster and possibly invasion of Egypt in the late
5805 EC. Subsequently a strategically well-conceived series of cam-
paigns was directed against Cyprus and Phoenicia (c.574-570 BC) in
which good use was made of the fleet. Ahmose II, who succeeded
Apries, was nothing if not lucky. He was able to defeat a Chaldaean
invasion of Egypt in the fourth year of his reign, and after that the
Chaldaeans had sufficient problems within the empire to keep them
fully occupied for the early part of his reign. In due course, however, he
was faced with a much more dangerous enemy created by the rise of
Persia under Cyrus the Great, who ascended the throne in 559 BC. To
deal with this menace a grand alliance of threatened nations was
created, which consisted of Egypt, Croesus of Lydia, Sparta, and the
Chaldaeans. With consummate strategic skill Cyrus knocked out the
link between the scattered allies by destroying Lydia in 546 BC. He then
turned on the Chaldaeans and took their capital Babylon in 538 BC,
leaving Ahmose with no major Near Eastern allies. Ahmose reacted by
developing a policy of cultivating close relations with Greek states to
strengthen his hand against the impending onslaught, and again he
was lucky. He died in 526 BC before the storm broke, leaving his son
Psamtek III (526-525 BC) to face the Achaemenid assault.
The south was not such an acute threat as the north, but the
Nubians could not be ignored, not least because they had certainly not
given up their ambitions to rule Egypt. There is no firm evidence of
military action against them in the reign of Psamtek I—indeed, the
introduction to the Nitiqret Adoption Stele suggests that he was
prepared to forget his differences with the Nubians, which included
the death of his father in battle against them, and that he adopted a
conciliatory policy. This stance may well have persisted to the end of
his reign, but we should be wary of assuming too much, given the
highly defective nature of our evidence. The situation was certainly
different in the reign of Nekau, who at some undefmable date was
forced to turn his attention to what a fragmentary text indicates was a
rebellion in Nubia; but the best-known Saite military commitment by
far is that of Psamtek II, who dispatched a great expedition in the third
year of his reign. This operation, which was designed to forestall a
Nubian assault on Egypt, seems to have taken the Egyptian army at
least to the fourth Nile cataract. It appears to have been successful, and
we hear nothing more in the dynasty of major military operations to the
south, although a demotic papyrus of the reign of Ahmose II describes
the king as sending into Nubia a small expedition, the character of
374 ALAN B. LLOYD
which is quite unclear, and there is archaeological evidence of an
Egyptian garrison at Dorginarti in Lower Nubia during the Saite and
Persian periods.
Relations with the Libyans were not consistently good during the
Saite Dynasty. The Saqqara Stele of the eleventh year of the reign of
Psamtek I, despite its damaged state, provides evidence of problems
with Libyan tribes to the west. These he seems to have defeated, and
they do not appear to have been a problem subsequently—quite the
contrary! About 571 BC we find the Libyans asking for Egyptian assist-
ance in dealing with the expansionist policy of Gyrene, a Greek colony
that had been founded in their territory about 630 BC. At the end of the
reign of Apries this city embarked on a programme of expansion that
brought them into collision with Egyptian interests, and in the
ensuing war Egypt was catastrophically defeated. Ahmose II adopted a
totally different approach to the Gyrene problem. As early as 567 BC we
find him forming an alliance with them against the Chaldaeans, and
this diplomatic link was cemented by marriage to a Cyrenean woman
who was alleged by some of Herodotus' sources, with considerable
plausibility, to have been a princess. This alliance stood the test of time
surprisingly well and was still in place at the time of the Persian
invasion in 525 BC.
The First Persian Period
Egypt's confrontation with Persia came to a head with the invasion of
Egypt in 525 BC, which led to the defeat and capture of Psamtek III by
Cambyses (525-522 BC) at the Battle of Pelusium. Cambyses' activities
in Egypt present a totally contradictory image in our sources, the com-
ments in classical authors being extremely unfavourable, whereas the
Egyptian evidence depicts a ruler anxious to avoid offending Egyptian
susceptibilities and presenting himself as an Egyptian king in all
respects. This aspect comes through particularly strongly in the
inscriptions on the statue of Udjahorresnet, where at least three major
points emerge: in the first place, Cambyses had assumed at least the
forms of Egyptian kingship; secondly, he was perfectly prepared to
work with and promote native Egyptians to assist in government; and,
thirdly, he showed a deep respect for native Egyptian religion. This
latter point also emerges in his burial of an Apis bull with all the
ancient rituals.
None of this prevented the outbreak of a revolt in Egypt when
Cambyses died in 522 BC, but the independence gained was short lived,
THE LATE P E R I O D 375
since Darius (522-486 BC) was able to regain complete control of the
country in 519/18 BC. With this reign, Egypt settled into a pattern the
beginnings of which are already clearly visible in the reign of Cam-
by ses. The head of the government was the Great King whose position
was legitimized for Egyptian purposes by the only means possible—
that is, by defining him as pharaoh on the same terms as a native
Egyptian ruler. Cambyses' policy of massaging Egyptian ideological
susceptibilities also continued under Darius both in religious matters
and administration: the building or restoration of temples was a prom-
inent feature—the medical school at Sais was restored, the building (or
rebuilding) of the temple of Amun of Hibis in the Kharga Oasis was
begun, and work was carried out at Busiris and the Serapeum at
Saqqara, and possibly also at Elkab. Darius is also credited with a
programme of law reform.
However, not all Persian kings showed the same delicate touch, and
Xerxes (486-465 BC) received a particularly bad press for his impious
disregard of temple privilege. As for administration, the Persians, like
the Ptolemies after them, had the good sense to realize that the
Egyptian system for running the country was the best that could be
devised, and maintained it with only the minimum of Persian admin-
istrative overlay needed to integrate the province into the Achaemenid
imperial organization. This primarily amounted to the insertion of a
satrap at the top. The satrap, who was effectively a viceroy, was drawn
from the cream of the Persian aristocracy, but his activities were none
the less carefully monitored by the imperial network of inspectors or
informers holding titles such as 'king's eye' or 'listeners'. He ran the
central administration through a chancellory that was controlled by a
chancellor assisted by a 'scribe'. The language used in the chancellory
was Aramaic, a situation that required the employment of a staff of
Egyptian translators. Below this level, the Persians showed a marked
disinclination to innovate. The legal system remained Egyptian, and
we can identify a series of Egyptians occupying positions of import-
ance, if not power, throughout the period.
At the same time, we can see an uncompromising determination to
keep firm control of the province, a policy that did not stop short of
inserting non-Egyptians into Egypt and Egyptian institutions, as and
when the Persians thought fit. They also ensured a substantial military
presence for the maintenance of external and internal security, and
Egypt was also expected to play its full part as a satrapy of the Persian
empire. Between c.$io and 497 BC Darius completed the construction
of a canal begun under Nekau II running from the Pelusiac branch of
376 ALAN B. LLOYD
the Nile through the Wadi Tumilat to the Bitter Lakes and the Red Sea,
a project that was clearly part of a policy of locking Egypt into the
imperial network of communication. Not only were Egyptian crafts-
men used for building operations as far afield as Persia, but also the
military resources of the country were exploited to the full to advance
Persian imperial expansion—Egyptians were involved in the naval
assault on Miletus that brought the Ionian Revolt to an end in 494 BC,
and Egyptian military and naval resources played a major role in the
great assaults of Darius and Xerxes on Greece in 490 and 480 BC. The
Egyptians supplied ropes for Xerxes' bridge of boats across the Helles-
pont and assisted in its construction, while the fleet of Xerxes used
against the mainland Greek states in 480/79 BC contained 200
Egyptian triremes under the command of Achaemenes, the brother of
Xerxes himself, as against the 300 supplied by the Phoenicians, indi-
cating that Egypt was no mean naval power at this period. This
contingent performed particularly well at Artemisium, where it cap-
tured five Greek ships with their crews, although this record does not
seem to have been maintained at Salamis. Finally, we should note that
the fiscal obligations of a satrapy were also laid upon Egypt, but these
were not unduly oppressive.
Overall, the impression created by such sources as we have is that
the Persian regime in Egypt was far from oppressive, and more than a
few Egyptians found it perfectly possible to come to terms with it.
Indeed there is indisputable evidence of a slow Egyptianization of the
conquerors themselves. Nevertheless, there are obvious areas where
tensions might arise. While the Great King might be presented for
ideological purposes as pharaoh, he was an absentee landlord based in
Iran and could not fail to appear to many as a token pharaoh only.
Secondly, the conquest by the Persians did not allay the ambitions of
native dynasts to rule the country, and they would have watched care-
fully for any opportunity to assert Egyptian independence and realize
their own ambitions. Furthermore, Egyptian xenophobia, highlighted
by Herodotus in the fifth century BC, will hardly have promoted inte-
gration between Persians and Egyptians, and this could be aggravated
by religious considerations, as illustrated by an episode in the reign of
Darius II (424-405 BC) involving mercenaries settled at Elephantine
and the local population. Here we find the priests of the ram-headed
god Khnum locked in a conflict with Jewish mercenaries that ended in
the destruction of the temple of lao (Yahweh). Given such flashpoints,
it is hardly surprising that the history of the First Persian Period is
punctuated by revolts. However, all these efforts came ultimately to
THE LATE PERIOD 377
nought until, 0.404 BC, the younger Amyrtaios successfully raised the
flag of insurrection to inaugurate the last extended period of independ-
ence under native rulers that pharaonic civilization was to enjoy.
Egyptian Independence (404-343 BC)
Most of the detailed evidence for the political and military history of
this period derives from Greek sources, which inevitably means that
they reflect the interests of classical observers and readers. They paint
a convincing picture of a period dominated by two recurrent issues:
instability at home and the ever-present spectre of aggressive Persian
power abroad. The grizzly panorama of intra- and inter-familial strife
between aspirants to the throne emerges with stark clarity in the case
of the 29th and 3oth Dynasties. In the murky history of these two
families we are confronted with a situation that we can only suspect for
earlier Egyptian history but that, we can be confident, was not infre-
quently lurking behind the ideological mirage projected by pharaonic
inscriptional evidence. Classical commentators, writing from quite a
different perspective, reveal without compunction the complex inter-
action of individual ambition untrammelled by loyalty or ideological
factors whereby ambitious political figures seize any opportunity for
advancement provided by the sectional interests of the native Egyptian
warrior class, Greek mercenary captains, and, less obviously, the
Egyptian priesthood. For the 29th Dynasty our evidence is far from
full, but it demonstrates unequivocally that almost every ruler had a
short reign and suggests that all of them, with the exception of Hakor
(393-380 BC), may have been deposed, sometimes probably worse.
The classical sources are particularly revealing for the succeeding dyn-
asty. The founder, Nectanebo I (380-362 BC), a general and apparently
a member of a military family, almost certainly came to the throne as
the result of a military coup, and we are unlikely to be guessing badly if
we suspect that this experience motivated him in establishing his
successor Teos (362-360 BC) as co-regent before his own death in
order to strengthen the chances of a smooth family succession. Ulti-
mately, this availed him nothing, because Teos was deposed in circum-
stances of which we are graphically informed. Indeed, nothing could
give us the flavour of the politics of this period better than Plutarch's
version of these events:
Then, having joined Tachos [i.e. Teos], who was making preparations for his cam-
paign [against Persia], he [Agesilaus] was not appointed commander of the entire
force, as he was hoping, but only given command of the mercenaries, whilst Chabrias
378 ALAN B. LLOYD
the Athenian was put in charge of the fleet. Tachos himself was commander-in-
chief. This was the first thing which vexed Agesilaus; then, whilst he found the
prince's arrogance and empty pretensions hard to bear, he was compelled to put up
with them. He even sailed with him against the Phoenicians, and, setting aside his
sense of dignity and his natural instincts, he showed deference and subservience,
until he found his opportunity. For Tachos' cousin Nectanabis [i.e. the future Necta-
nebo II], who commanded part of the forces, rebelled, and, having been proclaimed
king by the Egyptians and having sent to Agesilaus begging him for help, he made
the same appeal to Chabrias, offering both men great rewards. Tachos presently
learned of this and begged them to stand by him, whereupon Chabrias tried by
persuasion and exhortation to keep Agesilaus on good terms with Tachos. . . . The
Spartans sent a secret dispatch to Agesilaus ordering him to see to it that he did what
was in Sparta's best interests, so Agesilaus took his mercenaries and transferred his
allegiance to Nectanabis  Tachos, deserted by his mercenaries, took to flight, but
meanwhile another pretender rose up against Nectanabis in the province of Mendes
and was declared king. (Plutarch, Life of Agesilaus 36-9)
Egyptian evidence, though far from copious, provides intriguing
insights into the self-perception of these last native rulers. If we con-
sider the titularies of the rulers of the 29th Dynasty, we find that
Nepherites I bears a Horus name borrowed from Psamtek I and a
Golden Horus name taken from Ahmose II, while Hakor uses the
Horus and nebty names of Psamtek I and the Golden Horus name of
Ahmose II. These phenomena demonstrate unequivocally that both of
these pharaohs were determined to associate themselves with the great
rulers of the 26th Dynasty, the most recent 'golden age' in Egypt's
history.
Service to the gods is also a recurrent feature: Nepherites I has left
evidence of work at Mendes, Saqqara, Sohag, Akhmim, and Karnak
(where his son Psammuthis was also active), and Hakor's building
operations can be identified throughout the country. In the 3Oth
Dynasty, efforts were particularly spectacular: Nectanebo I built at
Damanhur, Sais, Philae, Karnak, Hermopolis (where he significantly
set up a stele before a pylon of Ramesses II), and Edfu, and we have
evidence of Nectanebo II's personal participation in the burial of an
Apis at Saqqara, as also of his role in raising the status of the Buchis
bull of Armant to that of the Apis bull of Memphis; there is also
inscriptional evidence of acts of piety to Isis of Behbeit el-Hagar, for
whom he began the construction of an enormous temple. The cyni-
cism of modern scholars has frequently led them to argue that these
activities were very much the result of a determination to keep the
support of the priests, and there is probably some truth in this, but it
would be a mistake to deny that there was also genuine religious
THE LATE P E R I O D 379
fervour. In the Hermopolis stele of Nectanebo I the traditional recip-
rocal relationship between gods and the king is asserted: the king
makes offerings to Thoth and Nehmetawy in return for the support
that he believes they gave him in gaining control of the kingdom; the
king also makes the traditional claim that his work in the temple
restored what he found in ruins—in other words, he is reaffirming the
old doctrine of the 'cosmicizing' role of pharaoh. In the Naukratis stele
of this same ruler we find him attributing his success to Neith, the
great goddess of Sais (again an affinity with the 26th Dynasty), insist-
ing that wealth is the gift of the goddess, and emphasizing that he is
preserving what his ancestors had done. There is surely no reason to
argue that these ancient concepts had lost any of their force to motivate
a ruler or to deny the sincerity of the gratitude expressed by recipro-
cating the beneficence of the gods.
When we turn to foreign policy, the dominant consideration is
Persia, for which the loss of Egypt was never—and could not be—an
accomplished fact. Fortunately for these last native pharaohs, pressing
Persian concerns nearer home meant that the recovery of Egypt made
it difficult for the Great King to give such a distant province his
undivided attention until 374/3 BC, when Artaxerxes II (405-359 BC)
embarked on the first major attempt to recover the country. The
Egyptian approach to the Achaemenid threat oscillated between using
diplomatic means to keep the Persians at bay and having recourse to
large-scale military operations. Since Egypt's preferred role was gener-
ally that of paymaster, direct military intervention by units of the army
or navy is infrequent and occurs only when prompted by necessity or
invincible ambition. The ease with which this policy could be con-
ducted is explained by the fact that it unfolded as part of a much greater
game, since all this Egyptian activity took place against the backdrop of
the struggle for independence of other western provinces of the
Achaemenid empire and the long-standing rivalry between Sparta and
Persia to define their respective spheres of influence in the Aegean,
Asia Minor, and the eastern Mediterranean. This created a lethal inter-
play of move and counter-move in which Egypt never had any difficulty
in finding enthusiastic support. Indeed, its prominence in these
operations was such that, even if the Persians had been prepared to let
sleeping dogs lie, they could not have done so, since an independent
Egypt would always have been a threat to the strategic equilibrium of
the western empire. It is, therefore, hardly surprising that Artaxerxes
III (343-338 BC) organized no fewer than three major assaults to
recover this lost but highly dangerous province.
380 ALAN B. LLOYD
We are fortunate in knowing a great deal of the organization and
character of the military operations of these sixty years of confronta-
tion. At this time Egyptian military resources were made up of three
main elements. In the first place, we frequently encounter Greek
mercenaries, Egypt's rulers having, in the main, a keen perception of
reality marked, amongst other things, by the firm conviction that
Egypt's interests were best served by paying others to do its fighting for
it. We therefore find Hakor putting together a large force of such
troops in the 3805 BC and Teos employing 10,000 picked mercenaries
in 361/0 BC, while Nectanebo II is said to have had 20,000 when
Artaxerxes III invaded the country in 343/2 BC. These troops showed a
clear superiority over the native Egyptian machimoi (militia) in the civil
war between Nectanebo II and Teos, but proved unreliable during the
successful Persian invasion of Egypt in 343/2 BC. In addition to these
troops we hear on a number of occasions of large forces of machimoi.
Plutarch describes them at one point in somewhat disparaging terms
as 'a rabble of artisans whose inexperience made them worthy of
nothing but contempt', but they were certainly capable of effective
military action: Diodorus comments on the effectiveness of their
skirmishing tactics against the forces of Artaxerxes in 374/3 BC, while
in the civil war of 360 BC they initially performed well against Agesilaus
and Nectanebo II, even if they were ultimately both outgeneralled and
outfought by their Greek opponents. On the negative side, that conflict
also clearly demonstrates that they were of unpredictable loyalty and
far from averse to playing the kingmaker, particularly if the promised
rewards were right.
The third ingredient in Egyptian military resources was allied
troops: the assets of the rebel Persian admiral Glo (in fact an Egyptian)
brought a significant increment to the forces of Hakor in 380 BC; the
Spartans were allies of Teos in 361/0 BC and sent 1,000 heavy infantry
with Agesilaus to Egypt, though they subsequently switched their sup-
port to Nectanebo; the Phoenicians appear as allies of Nectanebo II
in his struggle against Artaxerxes III; and Nectanebo availed himself
of the services of £.20,000 Libyans in the same context. The troops
featured in our Greek sources are generally infantry, but cavalry are
also mentioned explicitly on one occasion. As we should expect, we
have evidence of considerable Egyptian skill at military engineering in
exploiting the defensive possibilities of the terrain. Nectanebo I is
described as fortifying the coast and the north-east Delta very elab-
orately in 374/3 BC. All entrances were blocked off by land and sea: at
each of the seven mouths there was a town with large towers and a
THE LATE P E R I O D 381
wooden bridge dominating the entrance; Pelusium had a ditch around
it with fortified points of access by water that were blocked by moles,
and all the land approaches were flooded, whilst the town at the Men-
desian mouth had both a surrounding wall and a fort inside. The
Egyptians' expertise in this area also emerges in their operations
against Agesilaus and Nectanebo in 360 BC, and in the measures taken
by Nectanebo II to counter the assault of Artaxerxes III in 343/2 BC. Too
often, however, it was the high command of the Egyptian army that
proved the Achilles' heel, jealousy between Egyptian and foreign
generals easily becoming a flashpoint. Whilst Hakor hired the Athenian
Chabrias as general 0.385 BC without untoward results, Teos' undip-
lomatic arrangements in 360 BC were not so happy, in that Agesilaus
was given command of the Greeks only whilst Teos controlled the
Egyptian troops and also retained overall command of the army. Martial
failings on the part of the pharaoh could also be critical and eventually
lost Egypt its freedom, for our sources make it clear that the major
factor here was the ineptitude and cowardice of Nectanebo II himself.
These military confrontations were not confined to operations by
land. Naval activity features prominently, as indeed it was bound to do,
since one of the classic strategic techniques used by the Persians was,
where possible, to shadow the movements of their armies by fleet
movements along their flank. The best-known example of this is the
invasion of Greece by Xerxes in 480 BC, but any large-scale attack on
Egypt would present a perfect opportunity for such two-pronged
operations. The Egyptians, therefore, needed to be able to counter
Persian fleet movements as well as those of forces coming south by
land. However, even beyond this specific context it should be remem-
bered that the possession of effective naval units greatly strengthened
the strategic and tactical mobility of Egyptian forces in the east Med-
iterranean theatre. Fleets are, therefore, a frequent matter of comment
in our sources: for example, in 400 BC we find a rebel Persian admiral
called Tamos (certainly Egyptian!) taking refuge in Egypt with his fleet
and promptly being murdered by an enigmatic Egyptian ruler (prob-
ably Amyrtaios) specifically to gain possession of his naval assets, and
in 361/0 a substantial fleet is prepared alongside the army to partici-
pate in the general revolt of the western provinces of the Persian
empire. The technical sophistication of these forces was evidently high.
Whenever Egyptian warships are mentioned they are called triremes:
ramming war galleys propelled by three superimposed banks of oars,
the classic first-rate battleship of the Mediterranean world at this
period. In 396 BC we find Nepherites sending Agesilaus of Sparta the
382 ALAN B. LLOYD
equipment for 100 triremes—clearly he had enough and to spare in
his arsenals. We are told that the Cypriot rebel Evagoras acquired fifty
triremes from Hakor in 381 BC; and in 361-360 BC we are told that Teos
prepared a fleet of 200 such warships which were very well equipped.
Although the Egyptian ships are always described as triremes, we
should note that the Persian fleet collected for operations against Egypt
in 374 BC consisted of 300 triremes and 200 triakontors (single-
banked galleys with thirty oarsmen), and the Egyptian navy must also
have contained such lighter units. That native Egyptian commanders
could achieve the rank of admiral in the Persian fleet is a sufficient
testimony to their quality, but the Egyptian navy at this time could
recognize ability wherever it lay, and Teos had no hesitation in
appointing the superb Athenian admiral Chabrias to command his
naval units in 361 BC.
The re-establishment of Persian control in Egypt, which was com-
pleted no later than 341 BC, was attended by plundering of temples
and a policy of consolidation that took the form of demolishing the
defences of major cities and setting up once more a Persian provincial
administration staffed in part by local Egyptian officials such as Som-
tutefnakht. Evidently the intention was a return to the arrangements of
the previous occupation, but the outcome was a regime of recurrent
viciousness and incompetence that soon raised the level of disaffection
to the point of armed rebellion. It is surely here, perhaps about
339/8 BC, that the uprising of the much-discussed Khababash must be
placed, a rebellion so successful that it gave him at least partial control
of the country and a claim to the pharaonic office. In 333 BC there is an
equally signal example of disaffection in the enthusiasm with which
the appearance of the Macedonian rebel Amyntas was welcomed in
the country. It comes as no surprise, therefore, that, when Alexander
the Great invaded the country late in 332 BC, he had no difficulty in
quickly terminating the hated rule of Persia.
Culture in Continuum
Up to this point our discussion has been dominated by the political,
military, and economic vicissitudes of Egypt from the beginning of the
Saite period to the Macedonian conquest. Although it is impossible
to deny the vigour and skill with which the Egyptians met these chal-
lenges, our survey might easily create the impression of a nation sub-
jected for generations to considerable discontinuity. When, however,
we turn to cultural phenomena, a very different picture emerges. The
THE LATE P E R I O D 383
visual arts are paradigmatic. While, on the one hand, they show a
determination to draw on the traditions of the Old, Middle, and New
Kingdoms, as well as the Kushite Period, they display anything but the
arid archaism of which they are still too often accused. On the contrary,
the assertion of continuity with older tradition is combined with the
exercise of considerable invention and originality both in materials
and iconography, producing some of the most remarkable sculpture in
the entire pharaonic corpus. For other spheres of cultural activity there
is sometimes an unnerving lacuna in extant material—there are, for
example, no literary texts securely dated to this period. For all that,
close analysis of such evidence as we do possess confirms that
Egyptian society and civilization as a whole were characterized by the
same traits as the visual arts. We routinely encounter features with
which the student of earlier periods will be completely familiar.
Mortuary contexts continue to reveal the intense importance of
family ties, sometimes in a spectacular form: the tomb of the vizier
Bakenrenef at Saqqara of the reign of Psamtek I appears to have been
used for the burial of members of the family for the best part of 300
years, and the tomb of Petosiris at Tuna el-Gebel contained burials of
five generations of his family running from the 3Oth Dynasty into the
Ptolemaic Period. Non-mortuary epigraphy points in the same direc-
tion: the Wadi Hammamat inscription of Khnumibra shows a com-
parable awareness of family lineage in the 2yth Dynasty, purporting to
record his genealogy for over twenty generations as far back as the iQth
Dynasty, though we must be cautious about the historical precision
of this document. Such material also demonstrates the continued
importance of continuity of office within the family: Petosiris' family
occupied the office of High Priest of Thorn at Hermopolis over five
generations, whilst Khnumibra's ancestors are alleged to have had
something approaching a stranglehold over the offices of vizier and
overseer of works for centuries.
Local loyalties are, if anything, even stronger than of old:
Udjahorresnet insisted at the beginning of the 2yth Dynasty on the
sterling service that he had done for his native city, while the fourth-
century inscription of Somtutefnakht, set up in the temple of
Harsaphes in his home town of Herakleopolis Magna, indicates that
such service was transmuted into devotion to the local god, an easy and
natural formulation that was commonplace at this time. Such devotion
to local gods is easily paralleled earlier, but its prominence in the Late
Period is very marked, originating, no doubt, in the political frag-
mentation that was endemic after the collapse of the New Kingdom. A
384 ALAN B. LLOYD
corollary of this situation is the marked tendency for the main focus of
personal devotion to become the main city deity, who thus acquires the
omnipotence and omniscience of the traditional great gods of the
pantheon. This phenomenon generated, in turn, an intense sense of
the imminence of the divine presence, which is probably a major factor
in the development of animal cults, one of the distinctive religious
features of the Late Period. Devotion to this immediately present deity
was naturally accompanied by a powerful conviction of the depend-
ence of man on divine favour, which is frequently expressed in sculp-
ture through statues of individuals supporting and offering images of
their local god.
Biographical inscriptions further reveal that the factors leading to
success in life were perceived in essentially traditional terms: royal
favour was still regarded as a prerequisite of success; it was also con-
sidered essential to lead one's life on the basis ofmaat, the order of the
universe, both physical and moral, which came into existence at the
creation of the world and is definitive—that is, incapable of improve-
ment. Living in accordance with maat is described in the tomb of
Petosiris as 'The Way of Life', and a frequently mentioned stimulus to
follow this path is divine influence operating on the heart of the indi-
vidual—that is, on the source of his moral being. Once again, this con-
cept is not difficult to parallel earlier (for example, the old concept of
the netjer imy.k, 'the god who is in you'), but it is much more system-
atically developed in the texts of the Late Period. To follow 'The Way of
Life' under the guidance of god brought success in this world and also
beyond the grave, where yet another sanction lay in wait. The day of
judgement in the Hall of the Two Truths was set for all, and no dis-
tinction was made between rich and poor. However, this strong con-
viction that justice would ultimately be done did not prevent the
expression of a carpe diem philosophy, revealing that the Egyptians had
lost little of their love of life, and it is not surprising to find the appear-
ance of the occasional protest at the unfairness of an early death that
has prevented the enjoyment of all that life has to offer. Here again,
however, we are not confronted with a complete novelty; for the fragil-
ity of Egyptian certainties about life after death is eloquently expressed
in such earlier texts as the Song of the Blind Harper and chapter 175 of
the Book of the Dead. As for the principles of the mortuary cult, they
remained the same in the Late Period, if less elaborately developed in
practice, and old convictions such as the benefits to be gained by the
recitation of formulas and the performance of funerary rituals retained
much of their strength.
THE LATE P E R I O D
385
Plan of the tomb of Mentuemhat. It shows the arrangement of the structures below
ground-level, which are entered by a descending passage to the east. This gives
access via two columned halls to a great sun-court excavated in the rock but open
to the sky which is flanked by chapels to north and south. This leads to another
open court giving on to the subterranean part of the tomb which ends with the
sarcophagus chamber at the extreme west of the installation. The walls were
richly decorated with relief which shows a mixture of traditional elements, from
such sites as the tombs of Menna and Rekhmira and the Deir el-Bahri complex,
as well as contemporary features
The concept of the prerequisites of the afterlife presented a some-
what contradictory picture, but again it was a question of working with
and developing older ideas. Much effort was again spent by those who
could afford it on the production of tombs, some of which are spec-
tacular instances of conspicuous display. The mortuary complex of
Mentuemhat at Thebes is the most impressive non-royal site in that or
any area, and many a New Kingdom vizier would have envied the tomb
constructed for Bakenrenef looking out over the valley from the east
escarpment at Saqqara.
In the Saite Period, particular ingenuity was expended on building
unrobbable tombs that were filled solid with sand after interment, and
had precisely the desired effect, but grave goods were no longer as
plentiful or as rich as they had been in the New Kingdom, even though
gold or gilt-silver masks and jewellery could still be buried with the
deceased. This paucity of grave goods means that vaults are small—
often little larger than the sarcophagi themselves. As far as low-status
burials are concerned, we are better informed for this period than most
others, particularly at Saqqara, where excavations have revealed bodies
with little or no mummification interred in the poorest of coffins,
frequently no more elaborate than palm-leaf mats, and deposited in a
386 ALAN B. LLOYD
pit in the sand distinguished above ground, if at all, by nothing more
than a simple marker to guide the poor attentions of a relative anxious
to perform whatever minimal service could be afforded for the
deceased. All this chimes well enough with indications from earlier
periods to prove that at this level too the Late Period was continuing the
ancient ways.
Biographical inscriptions reveal yet another shift of emphasis in the
clear narrowing of the gap between the pharaoh and his subjects, and
this is echoed by the ease with which non-royal persons were able to
requisition ancient royal funerary literature: in several Saite tombs at
Saqqara (including those of the vizier Bakenrenef, the commander of
the royal fleet Tjanenhebu and the physician Psamtek), the Pyramid
Texts were employed, and fourth-century coffins also exemplified this
development. The tomb of Petosiris shows a parallel phenomenon in
that Petosiris himself claims at one point in his biographical inscrip-
tion to have performed the old royal foundation ritual of stretching the
cord. In all this, however, we are again not confronted with something
totally new, given that the I2th Dynasty, for example, already provides
ample demonstration of a willingness to concede the humanity of the
supposed god-king. It is all too easy to ignore the fact that in every
period of Egyptian history the relationship between the ideology of
kingship and the practicalities of life was ultimately defined by
historical experience, and the narrowing of the gap in these late
sources reflects nothing less than the realities of the distribution of
power in Late Period Egypt.
To conclude: the three centuries preceding the invasion of Egypt by
Alexander the Great (332-323 BC) were centuries of no mean achieve-
ment. Although the country was twice subjected to Persian domina-
tion, it still succeeded in maintaining its independence for long
periods against powerful enemies, and made a major impact on the
course of the interminable Near Eastern power struggle as well as
reasserting its interests on the Upper Nile. In the Saite Period several
factors interacted to create the basis for success. In the first place, a
family of rulers appeared who were both ideologically acceptable,
politically streetwise, and militarily highly astute.
However, the Saites were also lucky in that for most of the dynasty
the dynamics of imperialism in the Near East ran very much in their
favour. Empires expand as long as their institutional structures and
the will of their leaders can support such expansion. When the
Assyrians and Chaldaeans attempted to incorporate Egypt into their
empires, they were both operating at the outer limits of their capacity.
THE LATE P E R I O D 387
Even a slight deterioration within their territory inevitably meant a
diminution of resources that could be brought to bear against Egypt, to
the extent that effective action and control became quite impossible. It
is hardly surprising, therefore, that Assyrian rule was intermittent and
very low key, whilst all the Chaldaeans could achieve was to threaten,
invade, and withdraw.
The danger posed by the Persians was of a different order, since they
possessed much greater assets in wealth and manpower, and initially a
much more vigorous impetus to conquest derived ultimately from
Cyrus. However able a pharaoh might be, if the Persians operated at
the peak of their potential, the land of Egypt must fall. Yet the laws of
grand strategy were the same for the Persians as their predecessors,
and the marginal geographical position of Egypt in relation to the
Achaemenid empire meant that it would inevitably be difficult to
maintain permanent control and that the potential for successful
revolt would always be there.
Against this background, the panorama presented by the fifth and
fourth centuries BC of oscillation between rebellion, independence,
and occupation becomes immediately intelligible. Yet none of this
furious endeavour leads to any abatement in the vitality of Egyptian
cultural life. Certainly we suffer badly from the severe loss of the art,
architecture, and literary work of these years, but more than enough
survives to reveal a society that was powerfully aware of its past while
exploring new ways or, at least, insisting on finding its own points of
cultural emphasis. Wherever we look, we are confronted by a powerful
current of continuity united with a vital evolutionary dynamic that pro-
vides the obvious underpinning for and explanation of the very con-
siderable achievements of the age of the Ptolemies that followed.
14
The Ptolemaic Period (332-30 BC)
ALAN B. LLOYD
Ptolemaic Egypt is a tale of two cultures. Differing in ethos, focus, and
aspiration, these cultures initially maintained a wary coexistence, in
which convenience and the balance of power generated a viable degree
of cooperation usually sufficiently effective to mask their mutual dis-
taste. From the end of the third century BC, even this collaboration was
increasingly eroded by the divisive pressures exerted by dynastic
schism, maladministration, economic crisis, and Egyptian resent-
ments. Not the least fascinating aspect of this complex relationship is
the fact that, despite all its inner tensions, Egypt of the Ptolemies was
in many ways spectacularly successful, whether we consider the
achievements of the Graeco-Macedonian elite or those of the Egyptian
cultural milieu.
Prelude
It is most appropriate to begin the study of Ptolemaic Egypt with the
arrival of Alexander the Great in 332 BC, thus bringing to an end the
Second Persian Period, the passing of which was lamented by no one.
Before Alexander resumed his conquests in 331 BC, he was obliged to
address the problem of how to administer his new province.
The foundation of Alexandria was clearly an innovation intended to
create a new base for governing the country, but in other respects
Egypt's ancient ways prevailed. If we can trust the Alexander Romance
(a semi-mythicizing biography written anonymously under the pseu-
donym of Callisthenes in about the second century AD or earlier),
Alexander had himself crowned in the temple of Ptah at Memphis,
THE PTOLEMAIC P E R I O D 389
thereby firmly asserting that he was assuming the mantle of an
Egyptian pharaoh, but there is no doubt at all that he was conceptual-
ized in those terms by the Egyptians, who gave him a standard royal
titulary, and that he showed great respect for Egyptian religious sus-
ceptibilities. Keenly aware of the intrinsic strategic dangers latent in
Egypt's wealth and geographical position, he evidently fought shy of
concentration of power: the administration of the country was com-
mitted to an Egyptian called Doloaspis; the collection of tribute was
entrusted to Kleomenes of Naukratis; the army was placed under the
command of two officers, Peukestas and Balakros; and the navy was
allotted a separate commander in the form of Polemon. Kleomenes
was subsequently appointed governor of the entire province, which he
administered with a high degree of corruption.
On Alexander's death in Babylon in June 323 BC, his mentally unpre-
dictable half-brother Arrhidaeus (323-317 BC) was declared king, with
Perdiccas as regent, on the understanding that, if the child yet to be
born to Alexander's Bactrian wife Roxane were male, that child should
be joint king. Major sections of the empire were at this point allocated
by Perdiccas to Alexander's marshals, and in this division Ptolemy,
son of Lagos, acquired Egypt, Libya, and 'those parts of Arabia that lie
close to Egypt' with Kleomenes as his second in command.
The settlement of Perdiccas could not hold. It merely set the scene
for the Wars of the Successors, which inevitably broke out to deter-
mine whether Alexander's empire would survive intact. This complex
series of operations falls into two phases: the first, which ran from 321
to 301 BC, was fought out between the 'Unitarians' (above all Perdiccas
himself, Antigonus 'the one-eyed', and his son Demetrius 'the
besieger'), who attempted to preserve the unity of the empire, and the
'separatists' (pre-eminently Ptolemy, Seleucos, and Lysimachos), who
were determined to carve out their own kingdoms. Ptolemy's ambition
speedily brought him to the fore as the major headache for the unitar-
ians, who paid him the compliment of two invasions of Egypt, the first
by Perdiccas in 321 BC and the second by Antigonus in 306 BC, both of
which were defeated by Egypt's geography rather than by Ptolemy
himself. The unity issue was resolved by the defeat and death of Anti-
gonus at Ipsus in 301 BC, which decided this phase of the conflict in
favour of the separatists. By that time all the major protagonists,
including Ptolemy, had already anticipated this outcome by declaring
themselves kings.
390 ALAN B. LLOYD
High Summer of a Kingdom
The second phase of the Wars of the Successors ran from 301 to 280 BC
and is characterized by struggles between the separatists to establish,
maintain, or increase their kingdoms. It came to an end with the death
of Lysimachos at Corupedium in 281 BC, and the subsequent assas-
sination of his conqueror, Seleucus, later in the same year. The out-
come of these events was critical for the subsequent history of the
hellenistic world in that it yielded three great kingdoms: Macedon,
with pretensions to rule neighbouring states that were sometimes
realized, sometimes not; the Seleucid empire, based on Syria and
Mesopotamia; and the empire of the Ptolemies, the core of which
was Egypt and Cyrenaica. With these kingdoms we are confronted
with the protagonists in a power game that was to dominate the
Map of the Mediterranean region during the Ptolemaic Period (332-30 BC)
THE PTOLEMAIC P E R I O D 391
eastern Mediterranean and the Levant until Egypt was brought under
Roman control in 30 BC.
It is important to grasp that the rivalry between these kingdoms was
not confined to matters of political or military control, important
though these issues were. The underlying psychological motivation lay
where we should expect it to lie in any Graeco-Macedonian context—
that is, in an invincible impetus to self-assertion that would, in turn,
generate prestige. Cutting a fine figure in the great arena of Graeco-
Macedonian activity—even beyond—and placing the opposition firmly
in the shade were ultimately the most important issues. Certainly,
military conquest was a major means of achieving this, but the creation
of a kingdom of unequalled splendour was equally important and
could absorb an enormous amount of effort and resources. In this
battle for power and prestige the Ptolemies were beyond doubt the
outright winners, in the third century at least.
To all three kingdoms the key issue of high politics and grand
strategy was to expand their empires at the expense of their rivals by
whatever means they could, but the history of their conflicts is far from
simple. It is clear that the ambitions of the early Ptolemies were such
that they posed a serious threat to the aspirations of both the other
major players, who found it convenient to pool their resources against
the common enemy. Not surprisingly, therefore, in the early 2705 BC
we find a peace being concluded between Macedon and the Seleucids,
which was to become one of the very few constants in the history of the
third century BC.
For the Ptolemies there were two main areas of expansionist activity:
(i) the ancient centres of Greek culture in the eastern Mediterranean,
and (2) Syria-Palestine. As for the first, it is important to grasp that the
rulers of all these Hellenistic kingdoms felt themselves to be Macedo-
nians with Macedonian traditions and a close and deep affinity with
Greek culture. Therefore, the arena in which they above all wished to
make their mark was the mainland of Greece, the Aegean, and the
Greek cities on the coast of Asia Minor. For the Ptolemies of the third
century BC this meant in political and military terms a long struggle for
the hegemony of Greece against Macedon, which had acquired control
of a large part of the area in the time of Philip II and regarded it
unequivocally as Macedonian by right of conquest. This struggle, in
turn, enmeshed the Ptolemies in supporting the major political forces
in the Greek world, above all Epirus, the Aetolian and Achaean leagues,
Athens, and Sparta, who inevitably looked to Egypt for help against the
common enemy, but it also entailed efforts to maintain bases on and
392 ALAN B. LLOYD
in the Aegean and along the south coast of Asia Minor, control of
Cyprus, and a requirement to maintain an alliance with the strategic-
ally and economically important island of Rhodes. Inevitably, Ptolemaic
ambitions in Asia Minor brought them into stark conflict with Seleucid
interests in that area.
Despite the challenge of two great kingdoms, the first three Ptolem-
ies were initially highly successful in realizing their ambitions in the
Aegean. Reviewing their achievements in that quarter, Polybius writes
as follows:
their sphere of control included the dynasts in Asia and also the islands, as they were
masters of the most important cities, strongholds and harbours along the whole
coast from Pamphylia to the Hellespont and the region of Lysimachia. They kept
watch on affairs in Thrace and Macedonia through their control of Aenus and
Maronea and of even more distant cities, and, in this way, having extended their
reach so far and having shielded themselves at a great distance with these client
kings, they never worried about the safety of Egypt. That was why they rightly
devoted much attention to external affairs ... (Polybius, 5.34)
However, we should read these words with care. Polybius does not
say that these kings held an empire with clearly defined frontiers and a
coherent imperial administration. The passage reveals—and this is
confirmed by other evidence—that this 'empire' was, in truth, a thing
of nuances, an amalgam of bases, alliances, protectorates, and friendly
factions or individuals, frequently bought with Egyptian gold, forming
a network of nodes through which the Ptolemies were able to exert
political and military power. Nor, indeed, was the sphere a static one
even in these early years. In the struggles generated by these ambi-
tions, the early Ptolemies enjoyed mixed fortunes, but ultimately the
Macedonians and Seleucids prevailed. By the end of the third century
BC, Ptolemaic influence in Greece was gone as a significant force,
although a garrison was maintained on Thera in the south Aegean
until 145 BC. As for Asia Minor, the triumphs of Antiochus III in that
area during the Fifth Syrian War precipitated the end of Ptolemaic
hegemony on the west and south coast by £.195 BC.
The pattern of initial expansion giving way to severe recession by the
early second century BC was repeated in Syria-Palestine. The deter-
mination to bring Coele-Syria and the Phoenician cities into the
Ptolemaic kingdom surfaced early. The area had, of course, been a
traditional focus of concern in pharaonic times, but there were better
reasons than precedent for the Ptolemies to wish to hold it: strategic-
ally, its occupation facilitated the defence of Egypt as well as the
Ptolemaic province of Cyprus; control of Phoenicia gave the Ptolemies
THE PTOLEMAIC P E R I O D 393
access to Phoenician naval resources; finally, the occupation also
yielded major economic benefits both in fiscal terms, and with regard
to access to major trade routes (including the great commercial centre
of Petra), and, in particular, the ability to exploit the timber resources
of the Lebanon, which was a significant source of shipbuilding timber
for the Ptolemaic fleet. Not surprisingly, therefore, Ptolemy I (305-285
BC) made repeated efforts to gain control of this area: he held it in the
period 320-315 BC and briefly after the Battle of Gaza in 312, but in
301 BC he occupied Syria-Palestine probably as far as the Eleutherus
River, despite the fact that this territory had been allocated to Seleucus
after Ipsus. The determination of the Seleucids to maintain their
claims gave rise to no fewer than six Syrian Wars beginning in the
reign of Ptolemy II (285-246 BC) and ending with that of Ptolemy VI
(180-145 BC), although the issue was decided to all intents and pur-
poses by the Egyptian defeat at Panion in 200 BC, as a result of which
Ptolemy V (205-180 BC) conceded the claims of the Seleucids to Syria
and Phoenicia in ^.195 BC.
These Ptolemaic military successes and ultimate failure were linked
to a number of prerequisites: an effective army and navy; an adminis-
trative system at home that provided the basis, above all the economic
infrastructure, to fund expansion; conditions within the kingdom that
made it possible to concentrate such efforts on foreign enterprises;
and rulers with the vision and capacity to carry them forward.
Military Might
The Ptolemaic army, like all its Hellenistic counterparts, was the army
of Alexander, modified in the light of experience and necessity. Alex-
ander's forces consisted of a variety of complementary units that
reflected a tactical concept based on pinning the enemy down by
infantry pressure along much of the line and delivering the crucial
assault at a selected point by means of heavy cavalry. This meant that
the major tactical elements were a phalanx of heavy infantry armed
with pikes of considerable length (5.5 m., later increased) and a strike
force of heavy cavalry made up of squadrons of Macedonians, Thessal-
ians, and allies. The gap inevitably arising in action between these
elements was plugged by elite light infantry called hypaspists, 3,000
strong. These field forces, on which victory in general actions
depended, were supplemented by a wide range of light troops, both
horse and foot, and largely mercenary, and complemented by a highly
sophisticated siege train.
394
ALAN B. LLOYD
When we turn to the armies of the Ptolemies, we encounter much
that Alexander would have found immediately recognizable. At Gaza
in 312 BC the Ptolemaic assault was delivered by a force of 3,000 cavalry
armed with swords and the traditional Macedonian cavalry pike or
xyston. This succeeded in turning the flank of the opposing cavalry
force, which broke and fled the field, exposing the enemy phalanx to
an assault on its left flank. Faced with this threat, they quickly turned
tail and fled in confusion. Almost a century later the tactical thinking
at Raphia (217 BC) was very similar: Ptolemy IV's cavalry on the left
wing was driven from the field by its Seleucid opposite numbers,
whilst the Ptolemaic cavalry on the right wing reciprocated by
defeating the Seleucid horsemen facing them. In this battle, however,
victory was decided by Ptolemy's phalanx, which, on the king's per-
sonal encouragement, levelled pikes and charged the opposing phalanx
which quickly collapsed. In 200 BC, Panion provides yet another
example of cavalry as the striking wing, here very much to the dis-
advantage of the Ptolemaic army, since the Seleucid cavalry was able to
demolish its left wing, drive it from the field, and then return to
threaten the rear of the Ptolemaic phalanx, which had no alternative
but to withdraw.
Despite the underlying tactical similarity to the armies of Philip II
and Alexander, there was one crucial innovation that featured in all
three actions: the use of war elephants, which was a tactic learned from
the Indians. The elephants were employed as an ancient equivalent of
the tank, in order to assault and disrupt the enemy line. One solution
Fleet dispositions at the Battle of
Salamis exemplify the principle of a
concentrated heavy assault on part of
the enemy line much used in land
warfare after the Battle of Leuctra in
371 BC. Demetrius' left wing, which he
led personally, was the heavy wing on
which he relied to shatter its opposite
number and roll up the line against
the shore, the centre and the right
wing were relatively lightly held.
Ptolemy placed the weight of his
attack on the left wing. Ptolemy's left
was victorious, but Demetrius' heavy
contingent routed the opposing right
wing and then induced the collapse of
the Ptolemaic centre
THE PTOLEMAIC P E R I O D 395
to such an assault was to prevent them reaching the line in the first
place, and this was achieved brilliantly by Ptolemy I at Gaza by throw-
ing out in front of his army a screen of men armed with iron-covered
stakes that were fixed into the ground to block the advancing elephants
of Demetrius. Another remedy, clearly generally adopted, was to attack
the elephants and their drivers with highly mobile light troops armed
with javelins or bows. This meant, in turn, that any force using
elephants could not advance without its own light-armed troops in
attendance to neutralize those of the opposition. The major problem of
the Ptolemies in using the elephant was that of getting an adequate
supply of good quality animals—that is, Indian elephants. We hear
of none in the army of Ptolemy I at Gaza, but after the defeat of
Demetrius' elephant force he captured the survivors. The Ptolemaic
attempt to solve the problem in the long term was to use African
elephants, and hunts for these animals are mentioned on several
occasions in our sources. Unfortunately, the only trainable elephants
in Africa are the forest variety which is smaller than the Indian, so that
it is not surprising to find that Ptolemy IV's elephants at Raphia
quickly turned tail and fell back on his lines, with serious, though not
disastrous, consequences for the army as a whole. We hear of no
Ptolemaic elephants at Panion, though our one surviving source on
this action is highly defective, but it is interesting to note that Seleucid
elephants are claimed to have panicked the Aetolian cavalry on the
crucial Egyptian left wing, and elephants are also mentioned as par-
ticipating in the final encircling movement against the Ptolemaic
phalanx that sealed the defeat of the entire army.
One of the most notable changes in the Ptolemaic army in the
fourth and third centuries BC is the progressive dilution of its
Macedonian element, initially in favour of mercenaries but ultimately
leading to the incorporation of the Egyptian machimoi, or warrior class.
As early as Gaza in 312 BC Diodorus describes the army as containing
18,000 infantry and 4,000 cavalry, partly Macedonian, partly mercen-
ary, but we are also informed that there were numerous Egyptians in it,
some employed as baggage-carriers, others as soldiers, presumably
auxiliaries. By the time we get to Raphia these trends have gone much
further. Here Ptolemy IV disposed of an elite cavalry force 3,000
strong, of which over 2,000 were Libyans or Egyptians. Similarly, in
a phalanx of probably 45,000 men no fewer than 20,000 were
Egyptian. Ptolemy fielded, in addition, 2,000 mercenary cavalry, both
Greek and non-Greek, 3,000 Cretans, 3,000 Libyans, and 4,000
Thracians and Galatians. Indeed, it is highly improbable that
396 ALAN B. LLOYD
Macedonians and their descendants formed more than a small
proportion of this army.
The cost of funding such a large mercenary force was clearly a heavy
drain on the resources of the Crown, which could be met only if the
economy of the country was functioning properly, but the internal
disruptions that rose thick and fast after the death of Ptolemy IV were
bound to impair the ability of the rulers of Egypt to maintain such
troops. The problem of guaranteeing an adequate supply of soldiers
drawn from ethnic groups traditionally exploited by Macedon was
addressed at an early stage by the Ptolemies through the creation of a
large military reserve stationed in settlements throughout the country.
In these places they were given land allotments whose size was deter-
mined both by rank and type of unit. These plots they often did not
farm themselves but simply used as a source of income, but they
received them on the understanding that, whenever they were needed,
they would be called up for service, as in the case of the 4,000
Thracians and Galatians mentioned in the build-up to the Raphia cam-
paign. It is, however, intriguing that this is the only contingent in this
category mentioned by Polybius at that juncture, and the fact that it
formed a relatively small part of the army fielded for this operation
indicates that cleruchs (military settlers to whom the king gave allot-
ments of land called kleroi) were not regarded as the ideal source for
the bulk of the army.
Another obvious solution to the military manpower problem was
the Egyptian militia or machimoi, a remedy first tried apparently at
Gaza that fell into abeyance for many years, probably through a keen
awareness of its possible political disadvantages. Ultimately, short-
term necessity swept long-term considerations imperiously away, and
we find this group being exploited with spectacular success at Raphia,
where the bulk of the phalanx which gave Ptolemy the victory con-
sisted of Egyptian soldiers. The growing reliance on this class created
by the increasing difficulty in acquiring troops from traditional
Ptolemaic sources led to a critical shift in the balance of power within
the country, which is sharply highlighted by Polybius:
As for Ptolemy, his war against the Egyptians followed immediately on these events.
For the aforementioned king, by arming the Egyptians for his war against Antio-
chus decided on a course of action which was appropriate to the immediate
circumstances, but ignored the future consequences. For the soldiers, exalted by
their victory of Raphia, were no longer inclined to obey orders, but were casting
around for a leader and figurehead, thinking themselves capable of looking after
themselves. In this they finally succeeded, not long afterwards. (Polybius 5.107)
THE PTOLEMAIC P E R I O D 397
The army, however, was not the only requirement. The realization
of Ptolemaic ambitions in the Aegean and Eastern Mediterranean was
also dependent on the maintenance of a powerful battle fleet. This
force was not only a means of establishing and maintaining a Ptolemaic
presence in the area but also served as a weapon in the propaganda
battle for prestige and status. As in more modern times, large and
powerful naval units could be used to generate a sense of power even
when direct armed confrontation was not at issue. The critical strategic
importance of the fleet was grasped from the very beginning of the
Ptolemaic period, and its rise and decline are an unfailing barometer
of Lagid imperial and political fortunes in the Greek world.
Tactically, naval warfare developed to a marked degree in the late
fourth century BC. The trends emerge clearly in the best recorded of
Ptolemaic sea fights, the Battle of Salamis, which was fought off the
east coast of Cyprus in 306 BC and ended in the catastrophic defeat of
the Egyptian fleet. The action arose from an attempt by Ptolemy to
relieve his brother Menelaos, who was being besieged in Salamis on
land and sea by Demetrius, son of Antigonus. Ptolemy had approx-
imately 140 warships, facing perhaps 180 of the enemy. Diodorus, our
fullest source, unfortunately for our purposes, gives more information
on the fleet of Demetrius than that of Ptolemy, but there can be little
doubt that these details apply equally well to the opposition. A number
of points emerge: in the first place, we hear of soldiers being embarked
and of much action involving them; secondly, Demetrius equipped his
ships with ballistae and catapults capable of firing bolts three spans
(£.0.5 m.) in length, which were used to good effect; thirdly, ships of
various rates were engaged—for example, Demetrius' powerful left
wing contained 30 'fours', 10 'fives', 10 'sixes', and 7 'sevens', though
the bulk of his fleet consisted of 'fives'. The Ptolemaic fleet, on the
other hand, was made up entirely of 'fives' and 'fours'; furthermore,
both fleets appear to have drawn up for battle as three blocks of ships—
a centre with a wing on either side—but Demetrius made his seaward
wing particularly powerful while Ptolemy did likewise on the landward
side; finally, we should note that the fleets employed a primitive sys-
tem of signalling.
This summary reveals several important features. In the first place,
naval warfare has clearly been powerfully influenced by warfare on
land. While ramming manoeuvres were still being executed, the
emphasis has shifted from fighting battles of manoeuvre to conducting
land battles at sea, which placed a heavy premium on developing ever
bigger units capable of carrying large forces of marines who force a
398 ALAN B. LLOYD
decision by slogging it out toe-to-toe with the enemy. Athenaeus'
description of Philadelphus' fleet demonstrates the point perfectly: not
only does he state that it contained 2 'thirties', i 'twenty', 4 'thirteens', 2
'twelves', 14 'elevens', 30 'nines', 37 'sevens', 5 'sixes', 17 'fives' (as well
as a force of ships rated as 'fours' to 'one-and-a-halfs' which was
numerically double the rest), but he also describes a monstrous 'forty'
of Ptolemy IV that he makes a point of saying was capable of carrying
no fewer than 2,850 marines. The structure of these heavy ships has
been much misunderstood, older literature interpreting the terms
used to designate them as referring to banks of oars. This is quite
impossible. These vessels were propelled mainly, if not completely, by
multiple-rower sweeps and would never have had more than three
banks of oars, and the 'rating' must refer to the number of oarsmen in
a unit of rowers. The largest ships are now known to have had a
catamaran structure that would obviously greatly increase the deck
space available for marines, making such ships a particularly formid-
able proposition in a land-battle-at-sea. The militarization of naval
warfare is also illustrated by the mounting of artillery aboard ship, a
practice that obviously reflects the greatly enhanced importance of
artillery for both siege warfare and field use in the army of Philip II and
Alexander. The use of a heavy wing as a strike force by both pro-
tagonists is another case of adapting land warfare to the sea, since the
employment of that principle was a fundamental tactical device in the
Macedonian army. The use of signals will also emanate from the same
source.
Powerful and effective though the Ptolemaic fleet was in the first
half century of the dynasty, their shipbuilding efforts in themselves
could not guarantee consistent success, and in the mid-third century
BC their fleets suffered three hammer blows that presaged the gradual
unravelling of Ptolemaic sea power in the area: at Ephesus (probably in
258 BC) a Ptolemaic fleet suffered a reverse at the hands of the Rhodian
admiral Agathostratus, probably, in this case, being outmanoeuvred by
superior seamanship rather than outfought in a struggle between
marines; apparently about the same time the Ptolemies suffered a
second major reverse off Cos at the hands of Antigonus Gonatas, king
of Macedon, in which a powerful three-banked ship played a critical
role in bringing the Macedonians victory; subsequently, apparently
£.245 BC, Antigonus, although outnumbered, inflicted another defeat
on the Ptolemaic navy at Andros, this time probably by outfighting the
Ptolemaic marines.
THE PTOLEMAIC P E R I O D
399
The Land of Egypt
The Ptolemies' intense competitive spirit of self-assertion did not
confine itself to military conflict. There were other weapons in the
struggle for status and prestige in the cockpit of the hellenistic world,
which included their capital city of Alexandria. Founded by Alexander
in 331 BC, this city became the Ptolemaic capital and was vigorously
exploited from the beginning of the period as the major showcase for
Ptolemaic wealth and splendour and by the same token as the most
The city of Alexandria. Its commercial pre-eminence was based on three main
harbours: the deep Great Harbour formed by Cape Lochias and Pharos Island
which was joined to the mainland by the artificial Heptastadium (also an aqueduct)
and capable of taking the largest ships: Eunostos Harbour to the west: and the
harbour on Lake Mareotis which received cargoes from inland which fed the
export trade. The city's streets were designed on a chessboard pattern with the
main thoroughfare (3om. wide) running east-west from the Canobic Gate to
the Gate of the Moon. The main quarters of the city were (from west to east)
the Necropolis (famed for its gardens), Rhakotis (the Egyptian area), the
Royal Quarter, and the Jewish Quarter
400 ALAN B. LLOYD
significant non-military means by which the Ptolemies could vie with
and surpass their rivals. It quickly became the most spectacular city in
the hellenistic world. Strabo, who visited the city just after the demise
of the Ptolemaic dynasty, had no doubt of the importance of con-
spicuous display in Ptolemaic building on the site: he describes the
palace quarter in the northern part of the city as follows:
The city has most beautiful public enclosures and the palaces, which cover a fourth
or even a third of its entire area. For just as each of the kings from love of splendour
used to add some ornament to the public monuments, so also would he invest him-
self at his own expense with a residence in addition to those already in existence so
that now, to quote the Poet [Homer], 'there is building after building'. All, however,
are connected both with each other and with the harbour, even those that lie outside
the harbour. (Strabo, Geography 17. i. 8)
But there was much more than that. Closely associated with these
installations was the Sema, the burial place of the Ptolemaic kings, also
containing the body of Alexander the Great himself, which had origin-
ally been enclosed in a gold sarcophagus, though this was subse-
quently replaced by one of glass. The possession of this body was, in
itself, one of the greatest propaganda assets that the Ptolemies enjoyed
and was the result of an astute hijacking operation carried out by
Ptolemy, son of Lagus, when the corpse was being transferred to
Macedon for burial in the royal necropolis at Aegae. The most spec-
tacular of all Alexandria's buildings was, of course, the lighthouse on
the east end of Pharos island. Yet another renowned feature of the city
was the Mouseion, of which the world-famous library formed part.
This institution was founded by Ptolemy I as part of a policy of making
Alexandria the centre of Greek culture. The Mouseion was modelled
on the schools of Plato and Aristotle at Athens and, like them, was a
centre of research and instruction. Great efforts were expended to get
volumes for this library, and Ptolemy I's agent Demetrius of Phalerum
dispatched searchers all over the Greek world to obtain the required
texts. So successful were the efforts of the Ptolemies in this respect
that by the end of the period the library appears to have held no fewer
than 700,000 volumes, and the entire installation provided a superb
context for the pursuit of scholarship and scientific enquiry, so that
Alexandria quickly became the major centre for these activities, boast-
ing such figures as Eratosthenes of Gyrene (^.285-194) in science,
Herophilus of Chalcedon (£.330-260 BC) in medicine, Zenodotus of
Ephesus (born £.325 BC) and Aristarchus of Samothrace (^.217-145 BC)
in literary scholarship, and Apollonius of Rhodes and Callimachus of
Gyrene (both third century BC) in creative writing.
THE PTOLEMAIC P E R I O D 401
Alexandria also offered potential as a venue for great panhellenic
events that attracted participants from the entire Greek world, who
were thus able to marvel at the city that became the Ptolemies' master-
piece. Ptolemy II went so far as to establish a four-yearly festival called
the Ptolemaieia (probably in 279/8 BC), which was intended to honour
his father and, at the same time, the dynasty that he founded. Our
sources are quite clear that this festival was intended to be equal in
status to the Olympic Games themselves. We are particularly well
informed on a spectacular piece of grand theatre organized on behalf
of Ptolemy II, which illustrates the lengths and expense to which these
rulers would go to impress their Graeco-Macedonian audience. Cal-
lixeinus of Rhodes describes in very great detail a pompe, 'procession',
performed in the city's stadium and, as a preamble, tells of a remark-
able pavilion constructed in the palace area that was intended to house
a great symposion, 'drinking party', for the most distinguished guests.
This structure was remarkable for its size and splendour and con-
tained many extraordinary features: enormously expensive and lavish
fittings and equipment, a remarkable variety of animal pelts of
unusual size, rich floral embellishment that would not have been
possible anywhere else in the world, and sculptures and paintings of
the highest quality and value. In addition, this structure was designed
to make statements about Ptolemaic kingship: it combined at several
points Greek and Egyptian motifs, gave prominence to the Ptolemaic
heraldic eagle linking the family with Zeus, insisted on the military
aspects of Ptolemaic kingship, and asserted links with Dionysus and
Apollo. The procession of Dionysus, which this remarkable structure
was meant to service, continues the same propaganda line: the dyn-
astic agenda emerges strongly in the association of Ptolemy I,
Berenice, and Alexander the Great himself; the marked Dionysiac
dimension to the procession asserts the dynasty's affinities with the
god; the wealth of the kingdom is heavily emphasized in the copious
references to the valuable commodities to which it had access such as
frankincense, myrrh, saffron, and gold, as well as to Egypt's agri-
cultural productivity. Access to remarkable animals in great quantities
is also a major feature; we have a foreign-policy reference in the
symbol representing the strategically critical city of Corinth in the
procession; and the military might of Egypt is powerfully impressed
on the spectator by the participation of a force of no fewer than 57,600
infantry and 23,200 cavalry. In all this activity at Alexandria, archi-
tectural and otherwise, the overwhelming cultural emphasis is on
things Graeco-Macedonian, but the Ptolemies were strongly aware of
402 ALAN B. LLOYD
the fascination that pharaonic civilization had long held for the Greek
world and were far from averse to adding a touch of exotic spice drawn
from that quarter. It is not surprising, therefore, to find evidence of the
large-scale removal of Egyptian monuments to Alexandria or to
identify examples in the city of colossal statues of Ptolemaic kings and
queens represented in traditional Egyptian style.
The expense of maintaining these military operations and dynastic
pretensions was enormous and presupposed a highly effective infra-
structure capable of maximizing the potential of the Egyptian economy
both internally and externally. The most effective methods for running
the land of Egypt had been devised by the ancient Egyptians them-
selves. This the Ptolemies knew full well and contented themselves
essentially with refining this ancient system in the interest of maxi-
mum economic return. The key principle of government was king-
ship, but a kingship rather more complex than that of the Ptolemies'
Egyptian predecessors: the Ptolemies were not simply pharaohs but
also Macedonian kings ruling a Graeco-Macedonian elite within the
country, as well as subject peoples beyond. In the eyes of the Macedo-
nians the king's claim to Egypt and its dependent provinces lay in the
fact that it was 'spear-won' territory—that is, his right to rule was the
right of conquest, and by that right the kingdom became his estate to
administer as he thought fit. Initially, this kingship was exercised by
Alexander the Great, then Arrhidaeus, his brother, and Alexander IV
(317-310 BC), Alexander's son, while Ptolemy, son of Lagus, was tech-
nically only the governor of the province, but in 305 BC Ptolemy him-
self assumed the crown, and a crown that had to be held fully within
Macedonian tradition. In Macedon, to make good a claim to the throne
two things were traditionally necessary: that Argead blood should flow
through the veins of the claimant and that the army should formally
approve the accession. The problem of satisfying the first condition
was neatly solved by the claim that Ptolemy I was not the son of his
historical father Lagus at all but of Philip II himself, who had already
impregnated Ptolemy's mother before she was given to Lagus. As for
acclamation by the army at Alexandria, it is not conspicuous in our
sources but it was clearly long a recognized principle.
The process of validating Ptolemaic kingship in non-Egyptian con-
texts did not stop at such traditional Macedonian principles, as indeed
it could not, because very quickly Macedonians lost their importance
in the kingdom to the myriad Greeks who offered their services to
Egypt or simply featured as subjects in the far-flung foreign domains
that fell initially under Ptolemaic authority. From the time of Ptolemy
THE PTOLEMAIC P E R I O D 403
II we find the claim being made that the king and his wife were
themselves gods, a notion that quickly developed into the concept that
the king belonged to a hiera oikia or 'sacred family' consisting of the
living king and all dead rulers of the dynasty, including Alexander,
through whom Ptolemies could derive their ancestry from Zeus him-
self (if the claim of direct descent from Philip was not accepted). These
concepts also brought with them a claim of descent from Heracles and
Dionysus that played a prominent role in the Ptolemaic propaganda of
kingship. This body of concepts was associated with an offering cult in
honour of the king and his consort that was essentially a ruler cult pro-
viding an opportunity to Greek subjects for the corporate acknowl-
edgement and reaffirmation of the Ptolemies' political position—that
is, we are confronted with a clear case of the use of cult activity as a
support for a political system, a mechanism whose merits were sub-
sequently not lost on Roman emperors. This development has fre-
quently been claimed to have been inspired by Egyptian concepts, but
anyone familiar with the development of fourth-century Greek
thought on the relationship between human and divine and the clear
blurring of the dividing line between man and god will have no diffi-
culty in identifying the hellenic antecedents of this notion.
A very remarkable development within this royal house was the
establishment of full brother-sister marriage as a recurrent, though
not consistent, practice. This usage, initiated by Ptolemy II, who
married his full-sister Arsinoe II, has frequently been claimed to have
evolved on the basis of Egyptian historical precedent, a notion that
has persisted into recent literature, despite the total lack of reliable
pharaonic evidence that full brother-sister marriage was ever prac-
tised by Egyptian kings. It is possible that the mythological brother-
sister marriage between I sis and Osiris had some influence in moving
the Ptolemies in this direction, and a parallel was certainly drawn, but
brother-sister marriage has an obvious Greek mythological prototype
in the marriage of Zeus and Hera, which was easy to invoke for a
family that claimed Zeus as an ancestor. Be that as it may, the under-
lying rationale for introducing the practice is likely to have had a
severely practical dimension. Arsinoe II was a woman of formidable
ability and strength of character, like so many Graeco-Macedonian
women of rank—it is no coincidence that the best-known Ptolemy is
Cleopatra VII (51-30 BC)—and the marriage guaranteed, or helped to
guarantee, that she worked for, not against, him. Furthermore, it
ensured that she did not marry a possible rival whose position would
thereby have been powerfully enhanced. Above all, the union ensured
404 ALAN B. LLOYD
Ptolemaic control of the major assets at her disposal from her previous
marriage. The precedent, once set, was followed by many Ptolemaic
rulers and was far from an unalloyed asset. Most obviously, by giving
an institutional basis for the exercise of royal power by royal women at
the very highest level, the Ptolemies impaired the position of the
monarchy itself and contributed significantly to the long history of
dynastic instability that crippled the family. The inherent dangers of
the practice were further aggravated by the Ptolemaic taste for poly-
gamy, which could not but create disastrous rivalries for the suc-
cession.
As for the Egyptians, they cast the Ptolemies in the role of pharaoh,
the only form of legitimization of supreme political power that they
knew. The first Ptolemy known to have been crowned pharaoh in the
traditional manner was Ptolemy V, but there is a tradition that Alex-
ander underwent this ceremony, and the balance of probability must
lie heavily in favour of the assumption that it was standard practice.
Certainly they were all treated as pharaohs on Egyptian monuments
from the Macedonian conquest itself.
Below the king we find an administrative structure that has all the
hallmarks of the pharaonic system made sharper. The overriding con-
cern of the Ptolemaic system at all levels was fiscal, a fact that is
reflected in the activities of the dioiketes, 'manager', the major officer of
state whose chief concern was the financial administration of the
kingdom. He was assisted by a veritable army of subordinates, includ-
ing the eklogistes, 'accountant', and, at a later stage, the idios logos, 'privy
purse', who was responsible for the private resources of the king. This
economic focus is also in evidence when we turn to local government,
which was based on the traditional system of 'nomoi' (the Greek term
for ancient Egyptian sepatu), comprising about forty administrative
districts comparable to modern British counties. Within these provin-
ces agricultural production was the key focus. All land technically
belonged to the Crown, but for practical purposes it was carefully
divided into two categories: basilike ge, 'royal land', worked by 'royal
farmers' holding their land on lease and paying a yearly rent, and ge en
aphesei, 'remitted land', which fell into a number of categories: hierage,
'temple land', allocated to temples as their economic base; klerouchike
ge, 'land held by cleruchs', parcels of which could be found throughout
the country and consisted ofkleroi, 'allotments', assigned to soldiers in
return for military service as required; ge en doreai, 'land held in gift',
assigned to servants of the Crown as a stipend for exercising govern-
ment office and tied to that function; idioktetos ge, 'private land'—that
THE PTOLEMAIC P E R I O D 405
is, land which was de facto, if not de iure, held by private individuals;
and, finally, politikege, 'city land', assigned to the very small number of
Greek-style cities in Egypt. However, whatever the land title, agri-
cultural activity was meticulously controlled by central government
down to the smallest detail with the simple aim of maximizing the
return to the royal treasury. The following extract from an adminis-
trative papyrus is typical of the uncompromising and pervasive rigour
of this system:
Audit the revenue accounts, if possible, village by village—and we think it not to be
impossible, if you devote yourself zealously to the matter. If this is not possible, [do
it] by toparchies, approving in the audit nothing but payments to the bank in the
case of money taxes, and in the case of corn dues or oil-bearing produce only deliv-
eries to the corn-collectors. If there is any deficit in these, compel the toparchs and
the tax-farmers to pay into the banks, for the arrears in corn, the values assigned in
the ordinance, for those in oil-bearing produce, liquid produce according to each
kind. (Papyri Tebtunisyo^. 117-34)
The same level of state control is equally visible in all other forms of
economic activity—the exploitation of mines and quarries, the produc-
tion of papyrus, the operation of the novel banking system, currency
control, and also foreign commerce, in which Philadelphus was con-
spicuously active, not only opening or maintaining lucrative foreign
trading connections but facilitating it by large-scale engineering enter-
prises such as the completion of the Pharos lighthouse, the improve-
ment of the Koptos road joining the Nile Valley to the Red Sea, and the
reopening of the old Persian canal joining the Pelusiac branch of the
Nile to the Gulf of Suez.
The relationship between the Graeco-Macedonian elite and their
Egyptian subjects in the earlier phase of Ptolemaic rule is not always
clear and, where it is, it shows some inconsistency. An inscription at
Akhmim appears to refer to a Ptolemaic princess who had married a
prince of the 30111 Dynasty, and the old Egyptian aristocracy was
certainly not relegated to impotence: members of the 3oth-Dynasty
royal line seem to have retained high military office into the Macedo-
nian Period; in the reign of Ptolemy II we find a man called
Sennushepes as overseer of the royal harem and holding high office in
the Koptite nome; evidence from the same reign also places Egyptians
in high administrative and military positions within the Mendesian
nome. These and other cases justify the strong suspicion that the
Egyptian Dionysius Petosarapis, who appears with high court rank in
Alexandria in the i6os BC, had more precedents in the early Ptolemaic
Period than we are often inclined at present to concede.
406 ALAN B. LLOYD
Evidence is much fuller for the large class of priests and temple
scribes, although we should not fall into the trap of regarding them as a
closed group. Priestly office was compatible with secular office, so that
we cannot maintain a firm distinction between a secular aristocracy of
rank and office, on the one hand, and ecclesiastical status, on the other.
The priests were based at numerous temples, which were frequently
rebuilt or embellished in Ptolemaic times and still constitute some of
the most spectacular and complete remains of pharaonic culture. One
of the best examples is the temple of Horus the Behdetite at Edfu, which
is virtually completely Ptolemaic, being a focus of building activity
from 237 until 57 BC, although it is highly significant that the Ptolemies
chose to retain for the holy-of-holies the shrine of Nectanebo II, thus
affirming their continuity with Egypt's past. Another major focus of
Ptolemaic temple-building activity was Philae, where again we see
close links affirmed with the last native Egyptian Dynasty. These and
all other temples in the land continued to perform their ancient func-
tion as the power houses of Egypt, the interface between the human
and divine in which pharaoh, through his proxy, the local high priest,
conducted the critical rituals of maintenance for the gods, and the
gods, in turn, channelled their life-giving power through pharaoh into
Egypt.
One of the distinctive features of major state temples in the
Ptolemaic and Roman periods was the provision of a small peripteral
temple, invariably placed at right angles to the main temple, for which
Champollion coined the term mammisi (an invented Coptic word
meaning 'birth house'). The Ptolemaic mammisis were usually sur-
rounded by colonnades with intercolumnar screen walls, and they
were used to celebrate the rituals of the marriage of the goddess (Isis or
Hathor) and the birth of the child-god. There appear to have been
earlier counterparts of the mammisi in the form of 18th-Dynasty reliefs
describing the divine birth of the king at Deir el-Bahri and Luxor. The
temple of Hathor at Dendera includes two mammisis, one dating to
the Roman Period and the other to the time of Nectanebo I (380-362
BC), the latter evidently being used for the enactment of thirteen-act
'mystery plays' concerning the births both of the god Ihy and of the
pharaoh.
However, the temples were far from being simply cult centres. They
were also important foci of economic activity whose resources were
provided by the produce from land ceded to them by the Crown,
although this land did not become their absolute property, and they also
benefited from dues such as tithes and state grants. They produced
THE PTOLEMAIC PERIOD 407
manufactured goods for secular purposes, particularly cloth, and were
major sponsors of artistic works such as sculptures, which would be
created in their hut-nebu or 'houses of gold', or through their building
programmes, which generated an enormous market for the skills of
sculptors and painters. The work of these artists is of very great inter-
est, since it provides the clearest Ptolemaic evidence of an attempt at
cultural accommodation between Greek and Egyptian in that their
work was patently taking them in two different directions. In the first
place their determination to continue the traditions of Late Period
Egypt is particularly evident in the relief sculpture that survives in
enormous quantities in Ptolemaic temples, but it also shows through
in numerous examples of sculpture in the round, some of it quite
unsurpassed in the entire canon of Egyptian sculpture. There is, how-
ever, an increasing tendency for the influence of classical sculpture to
make an impact, so that works in a rather incongruous mixed style
become more and more common, a trend that was destined ultimately
to have serious consequences for traditional Egyptian art.
The priests enjoyed considerable political power, not least because
their good will was evidently seen by the Ptolemies as the key to the
acquiescence of the Egyptian population, and some of them, like
Manetho of Sebennytus, played a major role in Ptolemaic cultural
politics. The High Priests of Memphis were particularly important
from this point of view, both because they were the most significant
figures in the second city in the kingdom and because they were the
supreme pontiffs of Egypt at the time, with wide-ranging contacts and
influence in the country as a whole. The Ptolemies did everything they
could to ensure this support, but they spread their blandishments
much more widely than that, as is indicated in such well-known
expressions of priestly gratitude as the Canopus and Roserta decrees.
Indeed, a sensitive reading of such texts reveals an ever greater care on
the part of the Ptolemies to keep the priests on the side of the govern-
ment as the political and military power of the state declined.
The priests and scribes were the pre-eminent repositories and
exponents of traditional Egyptian culture, a role in which they were
clearly spectacularly successful in Ptolemaic times. If we consider text-
ual material produced for use in the temple cult, such as The Legend of
Horus of Behdet and the Winged Disc carved on the inside of the west
enclosure wall at Edfu, we encounter a profound knowledge of ancient
tradition combined with an impressive capacity for narrative and an
ability to write surprisingly good classical Egyptian, despite some con-
tamination from Late Egyptian and Demotic stages of the language, and
408 ALAN B. LLOYD
an exuberant development of the potential of the hieroglyphic script
that would have made the text frequently unintelligible to any reader of
the high Middle or New kingdoms. In other contexts we find the old
genres continuing to flourish—for example, tomb biographies and
cognate mortuary texts, pseudepigrapha, ritual texts, stories, and wis-
dom texts. The old compositional principles retain their currency, and
the conceptual world is unequivocally that of late pharaonic culture.
In the Ptolemaic concept of the afterlife, the judgement of the dead
was still central, as was the conviction that the verdict of the tribunal
(before which all must come in the underworld) depended on a virtu-
ous life. Negative attitudes to death could certainly emerge, in that
there was a willingness to complain of the injustice of an untimely end
to life and the helplessness of man in the face of death, and this could
lead, in turn, to a conviction that man should make the most of life
while it was possible to do so. However, in relation to both death and
life, there was the overriding conviction that the gods were maintain-
ing a moral order and that it was of critical importance to determine
their will and abide by it. This order was clearly seen as a definitive
framework of long standing that could not be changed, the structure
and workings of which could be determined by looking to the past, and
in particular to the ancient texts described in one passage as 'the Souls
of Re'. There was a very strong sense of dependence on the will of the
gods and a conviction that they would exact retribution for unaccept-
able behaviour. There was much talk of something that we translate as
'Fate', but it is evident that this could become coterminous with the
will of the gods. However, the Egyptians were not left completely in the
dark as to what that will might be, for they were convinced that the
gods frequently communicated with man, particularly by means of
dreams.
There was also an increased taste in Ptolemaic times for apocalyptic
literature, which was believed to give a direct insight into the workings
of the divine order. There continued to be a strong conviction that
there were experts who could break outside the normal range of
human capacity through their knowledge of words and actions of
power (heka) that could create changes, often spectacular, in the physi-
cal world. As for the concept of the make-up of man, this had not
changed, and the view of his social relations contains nothing surpris-
ing. Thus the Egyptians continued to see themselves in a social context
that transcended the present to embrace both ancestors and descend-
ants whose good report was a significant part of the immortality which
Egyptians craved. There was also a clear sense of social hierarchy and a
THE PTOLEMAIC P E R I O D 409
recognition that a person's position within that structure determined
his authority. In day-to-day living, family solidarity and the interests of
the local town were emphasized, as was the time-honoured paternal-
istic principle and active concern for those less well-off than oneself.
On the other hand, the wisdom literature could express a hard-headed
practicality and circumspection that left little room for trusting one's
fellow man; it could also betray a misogyny that had much to do with
the recognition of the sexual power of woman.
As of old, much weight was placed on self-control and restraint as
cardinal virtues, and in political relations the pharaoh could still be
seen as a divine benefactor whose support was essential to success,
although there was a greater willingness to concede even his depend-
ence on the gods and the possibility that he could act in a manner
unacceptable to them that would bring retribution on him and the
kingdom. Finally, we should not ignore the important point that there
was one aspect of this vital culture that made a deep and lasting
impression on Egypt's Hellenic masters—religion, where the success,
in particular, of Isis and the Egyptianizing cult of Serapis constitute a
remarkable example of cultural syncretism.
Below the large group of Egyptian scribes engaged in temple duties
was a significant number of scribes who functioned as civil servants
and secretaries. Indeed, there were ample opportunities in local and
provincial government if they were prepared to learn enough Greek to
act as intermediaries between Egyptians and the Graeco-Macedonian
elite. Lower in the social hierarchy were the artisans and craftsmen
who might express their talents in the temples, but there was certainly
scope in Ptolemaic Egypt for the independent entrepreneur, particu-
larly in the larger centres of population where we encounter numerous
small businessmen and even businesswomen producing for the retail
trade. Below them again we encounter the machimoi or militia who
were largely Egyptian and functioned as soldiers or policemen (see
Chapter 13). Having their origins in pharaonic times, the machimoi
continued into the Ptolemaic period, and, after their success at Raphia
in 217 BC, their importance in the military establishment increased.
Their economic and social status was not, however, high, since the land
allotments that they received were significantly smaller than those of
their non-Egyptian counterparts, typically 5 or 7 arourai (i aroura = 0.7
acre) as against the 20, 30,70, or even more allocated to Greek deruchs.
The productivity of these allotments was such that there was no mar-
gin to employ assistant labour, so that, if machimoi were called away for
military service, they could run into severe economic difficulties.
410 ALAN B. LLOYD
Not much lower than the machimoi was the great mass of the
Egyptian peasantry engaged in the agricultural production that formed
the basis of the economy. This involved the back-breaking task of
creating and maintaining the irrigation system in addition to the
normal agricultural activities of cereal and fodder production, arbori-
culture, and stock rearing. The peasantry might carry out these tasks as
labourers or tenants on Crown and temple land or on great estates, and
the more enterprising and successful might rent additional acreage
from landholders such as ckruchs who had no taste for the agri-
cultural life themselves. Some of them were also perfectly capable of
making the most of any additional opportunities for supplementing
their income—for example, acting as transport agents as required by
government or local centres of economic production. Indeed, it is clear
that some tenants of Crown land were in quite a good line of business,
but in most cases the peasantry was evidently operating at the level of
marginal subsistence, and its lot could easily become intolerable,
particularly in times of internal political disruption, which were
increasingly common from the end of the third century BC.
A Long Decline
The erosion of the Aegean and Syrian possessions of the Ptolemies in
the late third and early second centuries BC was to leave them with only
two foreign provinces: Cyrenaica and Cyprus. Polybius blames the rot
squarely on the character deficiencies of Ptolemy IV himself, but the
decline of Ptolemaic power lay deeper than the iniquities of a single
ruler. In the first place, dynastic schism, which had its roots in the very
institutional character of the monarchy itself, became a recurrent
feature of Ptolemaic history, generating murderous bouts of inter-
necine strife that at best were enervating and at worst raised instability
in the kingdom to a disastrous level. These problems were often aggra-
vated by the fury of the Alexandrian mob, which first surfaced at the
death of Ptolemy IV in the lynching of his minister Agathocles—
indeed, nothing gives a better picture of their unbridled and vicious
temper than Polybius' description of the murder of Agathocles' rela-
tives and associates:
All of them were then handed over together to the mob, and some began to bite
them, others to stab them, others to gouge out their eyes. As soon as any of them fell,
the body was torn limb from limb until they had mutilated them all; for the savagery
of the Egyptians is truly appalling when their passions are aroused. (Polybius, 15.33)
THE PTOLEMAIC P E R I O D 411
Their predilections as king-makers are subsequently demonstrated
in numerous episodes. Thus the long conflict between Ptolemy VI and
VIII frequently involved the actions of the mob, and in 80 BC they
excelled themselves by assassinating Ptolemy X himself. Finally in
48/7 BC their anarchic propensities reached a crescendo that culmin-
ated in the summary destruction of their power by none other than
Julius Caesar. The effects of these inherent weaknesses at the centre of
the kingdom were compounded on many occasions by the self-seeking
ambition of high-ranking Greeks, military and civilian, who were
determined to do anything to further their personal interests.
In Egypt outside Alexandria the political situation rapidly deterior-
ated from the late third century BC onwards, as the country seethed
with internal discord. These circumstances certainly facilitated the
elevation of some of the more able and enterprising Egyptians, and
there is clear evidence that they were succeeding in closing or even
eliminating the gap that normally existed between Greek and Egyp-
tian, gaining estates of some considerable size and even attaining the
rank of provincial governor (strategos) or governor-general (epistrategos).
The recurrent civil unrest has often been seen as a nationalistic,
ethnically motivated reaction by Egyptians against the hated Greek,
but the situation was clearly much more complex than that and is
probably better read as the natural outcome of the weakening of royal
authority that created a context where ancient rivalries and aspirations
of various kinds were no longer contained by central authority. These
might be hostilities between Egyptian cities, as when Hermonthis
(Armant) and Theban Crocodilopolis went to war against each other in
the time of Ptolemy VIII (170-116 BC). Again, when, between 205 and
186 BC, an independent state was established in the Thebaid, governed
in succession by two native kings called Haronnophris and Chaon-
nophris, we may well be seeing a resurgence of the ancient political
ambitions of the priesthood of Amun, and it is worth noting that, in
the final battle in 186 BC, Nubian troops fought in Chaonnophris'
army—that is, we may also have evidence of a resurgence of the
ancient interest in Thebes by Nubian devotees of the god. However,
since religiously determined xenophobia is a soundly documented
phenomenon in the Late Period, it would be extremely suprising if it
were totally absent from Egyptian motives in this move to independ-
ence.
There are many other signs, large scale and otherwise, of disaffec-
tion among the Egyptian population—strikes, flight (sometimes to the
point where whole settlements were abandoned), brigandage, attacks
412 ALAN B. LLOYD
by desperadoes on villages, despoliation of temples, and frequent
recourse to the temples' right of asylum. These are indisputably the
reactions of people pushed beyond the limits of endurance by famine,
rampant inflation, and an oppressive and vicious administrative sys-
tem operated by officials who were all too often corrupt and beyond the
effective control of central government. Against such men, the lower
strata of society, who were largely Egyptian, were, in reality, defence-
less and, therefore, easy targets. Uprisings by these people might easily
be construed as nationalistic, given the close congruence between
economic status and ethnic origin, and we can be confident that they
acquired that dimension explicitly from time to time, but at the most
fundamental level the uprisings were those of the oppressed against
the establishment regarded as responsible for that oppression, and
that establishment could just as easily be perceived as the Egyptian
priesthood and their temples as Graeco-Macedonian officialdom.
Whatever the motivation, however, the corrosive economic effects of
these disruptions struck a deadly blow at the economic infrastructure
at precisely the time when alternative sources of wealth had largely
dried up.
All these internal events were played out against the backdrop of
growing interventionism by Rome in the Eastern Mediterranean.
Sometimes solicited, sometimes not, this process led progressively to
the elimination of the kingdom of Macedon (167 BC), the acquisition of
the kingdom of Pergamum in 133 BC, the gradual erosion of Seleucid
power culminating in the annexation of the rump of the kingdom in
64 BC, and eventually the demise of the kingdom of the Ptolemies
itself. The last event was long in coming and was the last episode in
Ptolemaic relations with Rome that went back to the early years of the
dynasty and evolved through several phases. Starting on a basis of
equality in the reign of Ptolemy II with diplomatic courtesies between
equals expressed in an embassy to Rome in 273 BC, we move in the
early second century BC to a situation where Rome became the
guarantor of Egyptian independence.
Polybius' description of C. Popilius Laenas' removal of Antiochus
IV from Egyptian territory in 168 BC perfectly illustrates the ensuing
shift in power. On handing the king the Senate's decree:
Popilius did something which seemed insolent and arrogant to the highest degree.
With a vine stick which he had in his hand he drew a circle around Antiochus and
told him to give his reply to the message before he stepped out of that circle. The
king was astounded by the arrogance of this action and hesitated for a short time
and said he would do everything the Romans asked. (Polybius, 29. 27)
THE PTOLEMAIC PERIOD 413
From this it was a natural progression to become the mediator in
dynastic disputes: during the long-drawn-out quarrel between the
brothers Ptolemy VI and VIII, Rome was the arbitrator; Ptolemy XI
(80 BC) owed his kingdom to Rome and allegedly left it to his bene-
factor by will; in the dispute between the Alexandrians and Ptolemy
XII (80—51 BC), Rome played a decisive role; and Rome's involvement
in the murderous conflicts between Cleopatra VII and her brothers
Ptolemy XIII and XIV ushered in the last phase of Ptolemaic kingship.
In this maelstrom, and improbably, Cleopatra was able briefly to
resurrect past glories 0.36 BC when, through the largesse of Mark
Antony, we see the fleeting resurgence of Ptolemaic control in south-
ern Asia Minor and Syria-Palestine, but this ran counter to the general
trend that featured Rome as the sole beneficiary of the long decline of
the dynasty: Cyrenaica was acquired in 96 BC, Cyprus in 58. Finally, it
was Egypt's turn. In 30 BC, through a struggle as spectacular and
dramatic as anything that antiquity can offer, this brilliant and ancient
kingdom fell to Rome, thereby initiating the final long-drawn-out
chapter in the history of pharaonic culture.
The Roman Period (30 BOAD 395)
DAVID PEACOCK
There can be few historical events better known than the love affair
between Mark Antony, triumvir of Rome, and the beautiful and
talented Queen Cleopatra VII of Egypt. His association with Cleopatra
may not have been without political motives, for there was much to be
gained by Rome fostering good relations with Egypt, the wealth of
which was proverbial. Ultimately, however, his relationship brought
him into conflict with his astute, single-minded, brother-in-law, Octa-
vian. The issue was finally settled in the battle of Actium, fought in
September 31 BC, and a year later Octavian, who in 27 BC changed his
name to Augustus, entered Egypt for the first and last time. Egypt, the
land of the pharaohs and their hellenistic successors, the Ptolemies,
was now part of the Roman empire.
Egypt was a land apart—an exotic and distant part of the empire,
perhaps more bizarre than any other province. Here, pharaonic cul-
ture thrived and a visitor to Roman Egypt would have found himself in
a time capsule, for the sights, sounds, and customs of Roman Egypt
would have had more in common with pharaonic civilization than
with contemporary Rome. Temples were still built in traditional style.
The hieroglyphic script continued to be used, and Egyptian was spoken
by the common people, although the lingua franca was Greek. Cleo-
patra was, as far as we know, the only Graeco-Roman ruler of Egypt to
learn Egyptian, and then it was one of a multitude of languages in
which she was proficient. Further indications of the depth of the all-
pervading pharaonic culture is the persistence of mummification as a
burial rite and continuing reverence for Egyptian gods. The special
nature of Roman Egypt is undeniable, although there is a growing
i5
THE ROMAN P E R I O D 415
body of scholars who consider the 'Romanity' of Egypt to be a more
significant aspect.
Whether this is the case or not, cultural differences existed and it is
hardly surprising that Rome adopted a somewhat hostile and sus-
picious attitude to Egypt. Roman senators were forbidden to enter the
country and native Egyptians were excluded from the administration.
It is significant that the only Egyptian town founded by Rome was
Antinoopolis, on the Nile in Middle Egypt. The force behind this estab-
lishment was Hadrian, one of the few emperors ever to visit the
country. His own love affair with Egypt is reflected in his great villa at
Tivoli, where he attempted to recreate a Nilotic landscape in the Cano-
pus garden.
Despite its unique aspect, Egypt has a special role to play in our
understanding of the Roman empire as a whole. The dry climate has
led to the preservation of a wealth of evidence that is lacking in more
temperate regions. It is, for example, a repository of written evidence
that is seldom preserved elsewhere. Best known are the papyri, which
give an unrivalled insight into the business affairs and daily life of
Roman Egypt. One of the most famous and productive sites is the town
of Oxyrhynchus near the Nile, about 200 km. south of Cairo. In 1897
two Oxford scholars, Grenfell and Hunt, began to quarry the rubbish
of the ancient town (sebakh in Arabic) for papyri. Their work proved to
be a windfall for papyrology, for the documents so far published occupy
nearly sixty volumes and there is almost the same quantity awaiting
study.
Egypt is also the most important country for ostraca, documents
written on potsherds in place of papyrus. Between 1987 and 1993 exca-
vations at the Mons Claudianus fort in the Eastern Desert yielded over
9,000 ostraca, the largest collection from anywhere in the ancient
world. For the first time they document quarry operations and give us a
unique insight into the provisioning and logistics of a major Roman
enterprise in the desert.
Documentary evidence apart, Egyptian town sites and tombs often
yield organic matter that is seldom available elsewhere. Textiles are
often beautifully preserved, as are basketry, leather, or food remains.
Unfortunately the potential of this material has yet to be fully explored,
as all too often it has been discarded in favour of the written evidence.
Thus, Grenfell and Hunt seem to have thrown this material aside to be
used as fertilizer by the fellahin. Recent excavations, such as those at
Mons Claudianus, are beginning to rectify this imbalance.
416 DAVID PEACOCK
Administration
Roman Egypt was divided into about thirty administrative units called
'nomes', a system inherited from the preceding Ptolemaic era. Each
had a governor or strategos, appointed by and answerable to the Prefect
or governor of Egypt, via one of four epistrategoi, the regional adminis-
trators. The Prefect was assisted by procurators responsible for finance
and by other officials.
Each of the nomes had a capital town or metropolis, where the seat of
local government was located. Unfortunately we do not know very
much about these, as the urban topography of Roman Egypt has been
little studied. The two best understood are Oxyrhynchus and Arsinoe,
whence the evidence is derived from papyri. It appears that they were
places of some sophistication and wealth. Thus, Oxyrhynchus had a
gymnasium, public baths, a theatre, and about twenty temples, while
Arsinoe had running water supplied by two reservoirs into which
water was pumped from an arm of the Nile.
During the first two centuries AD, the nomes and their metropolises
enjoyed little in the way of self-government, but in AD 200 Septimius
Severus ordered the creation of town councils in each nome, a step
towards upgrading the metropolises to municipia (a municipium being,
in essence, a self-governing borough). This, however, led to consider-
Diagram showing the bureaucratic structure of Roman Egypt
THE ROMAN P E R I O D 417
able resentment, for with increased responsibility came increased
financial burdens to the holders of office.
Under Roman rule, all males between the ages of 14 and 60 were
obliged to pay a poll tax annually. Roman citizens were exempt, but
these probably only formed a minor part of the population. The upper
classes, the 'metropolites', paid at a reduced level. Class was, thus, a
subject of some consequence and at the age of 14 a metropolite boy
would be required to present his credentials.
The Army
As in other provinces, the main agent of control was the army. The epi-
graphic and papyrological evidence that Egypt provides furnishes an
unrivalled picture of the functioning of a provincial army, to which can
be added the archaeological evidence of the forts from which the army
operated. Many of these, preserved by the desert, still stand to their
wall tops.
One of the major early historical sources on the disposition of troops
was Strabo (17.1.12), who, in a much-cited passage, states:
There are three legions of soldiers, one in the city and the others in the chora. In
addition there are nine Roman cohorts, three in the city, three on the border with
Ethiopia at Syene, as guard for those places and three elsewhere in the chora. There
are three horse units which are likewise positioned in important places.
The city is, of course, Alexandria, where the fort of Nikopolis, about
5 km. east of the centre, stood until the late nineteenth century. Today
a few fragments remain in the Khedival palace that was built on the
site and all but obliterated it. Another legion seems to have been
stationed in the fortress of Babylon (fragments of which can still be
seen in the grounds of the Coptic museum in Cairo), while the third
had the task of guarding the Thebaid. The legions deployed include the
XXII Deiotariana, the III Cyrenaica, the II Traiana, and the XV
Apollinaris.
Strabo is much less specific about the auxiliary units, but here it is
possible to fill in the detail from a variety of sources within and outside
Egypt. The evidence includes dedications, diplomas, tombstones, and
other inscriptions, as well as papyri and ostraca, the latter two more or
less restricted to Egypt itself. During the first three centuries AD there
seem to have been, on average, three to four alae (cavalry units)
stationed in the country, as well as about eight cohorts, which accords
remarkably well with Strabo's claims.
418 DAVID PEACOCK
The units changed and moved from one part of the empire to
another, and between different places within Egypt, and in some cases
it is possible to reconstruct their history. Thus, the ala Vocontorium is
one of the earliest and best-attested auxiliary units in Egypt. Prior to
AD 60 it seems to have been based in the Koptos area and there is also
evidence for its presence in the fort at Babylon in AD 59. During the
Flavian period it may have served on the German frontier, returning to
Egypt by AD 105. It was later deployed in the Eastern Desert at Mons
Porphyrites (AD 116), then again in the Nile Valley, until it disappeared
from the records in AD 179.
Another example is the cohors II Ituraeorum, which is attested in
Syene (Aswan) in AD 28 and AD 75 and later at various other places in
the Syene area, before ending up at Mons Claudianus in AD 223-5.
The tasks that the army had to perform were multifarious. Defence
of the empire was obviously important. According to Strabo, the areas
to the south and east of Egypt were peopled by tribes largely identified
to the Romans by their eating habits. There is little doubt that the
troops stationed at Syene (Aswan) would have been charged with
securing the southern limits of the state. Equally, desert security might
have been, in some measure, the responsibility of units based along
the Nile in Upper and Middle Egypt. There were certainly forts in parts
of both the Eastern and Western Deserts, but they seem as much
related to mineral exploitation and the promotion of trade as to
security.
However, the army based in Egypt played a major role in most of the
eastern military campaigns, such as the annexation of Arabia in
AD 106 and Trajan's Parthian War. It was also called on to quell the
Jewish revolts of the first and second centuries AD. Here the legions at
Nikopolis and the units stationed at Pelusium in northern Sinai would
have played a significant part, as they could have moved with relative
rapidity to eastern trouble spots. Alexandria was undoubtedly the key
military base. The legions based nearby would have been charged with
controlling the unruly Alexandrian mob, securing this jewel of a city
against attack, policing the countryside, and playing a part in the wider
problems of the empire.
In fact, a major role of the army everywhere was to act as a police
force. There is a substantial number of ostraca, principally referring to
the Eastern Desert, that specify guard duties and the manning of
skopeloi or watch towers. It appears that the guards were organized into
dekanoi, which were controlled by curatores, who in turn were respon-
sible to centurions. Movement along desert roads seems to have been
THE ROMAN P E R I O D 419
very strictly controlled, with need for permits, written on an ostracon,
or perhaps sometimes papyrus. Undoubtedly this was a measure to
limit the banditry for which Egypt was notorious. This enduring prob-
lem must have been a major preoccupation of the army, with units of
soldiers under the command of the strategos hunting down both the
bandits and their sympathizers in the general population. Banditry
would have been particularly prevalent in the mountainous parts of the
Eastern Desert, where there would have been ample opportunity to
hide, and rich picking to be had from the caravans of oriental luxuries
travelling from Berenice or Myos Hormos (Quseir el-Qadim) on the
Red Sea coast to the Nile. This undoubtedly accounts for the string of
forts between Berenice and Koptos and particularly for the forts and
numerous watchtowers on the road between Quseir el-Qadim and
Koptos.
The army seems to have been involved in many other activities, such
as the supervision of grain boats travelling down the Nile to Alex-
andria, guarding the ever unpopular tax collectors while executing
their duties, and supplying and supervising quarrying and mining
enterprises in the desert. Here, the evidence from Mons Claudianus
suggests that they lived alongside civilians and were an integral part of
the extractive system. They were charged, amongst other things, with
supervising the skopeloi, with guarding valuables such as iron tools,
and perhaps with the maintenance of structures.
The Economy
There are three interrelated aspects of the economy of Roman Egypt.
The most important is the agricultural production of the Nile Valley
and the Delta. The fecundity of Egypt was well known and the city of
Rome relied heavily on the Alexandrian grain ships to feed its teeming
population. A second facet is the mineral extraction focused largely,
but by no means exclusively, on the Eastern Desert. Here gold had
been exploited since pharaonic times, but during the Roman period it
was also a source of exotic stones such as the granito del foro and
imperial porphyry. The red granite of Aswan has a long history of
exploitation and it is not surprising that it was also one of the most
important decorative stones used by the Romans.
The third aspect of the economy is the role that Egypt played in
articulating Roman trade. Alexandria was, of course, one of the great
trading cities of the ancient world, but Egypt is uniquely placed, with
access to both the Mediterranean and the Red Sea, which itself leads to
420 DAVID PEACOCK
the Indian Ocean and beyond. The country thus played a major role in
Rome's trade with the Orient: with India in particular, but through it
contact was made with Malaysia and possibly even China.
To many people today Egypt is a long thin ribbon of land expanding
to a triangle in the form of the Delta. This is where the population lives
and works and this is where the food is grown. Today, as in the past,
the fertile land produces a surplus. The cause of this fertility is not, of
course, the climate, for rainfall is negligible, but the river Nile. Before
the building of the first Aswan dam, the river would burst its banks
annually, depositing a fresh layer of rich silt on the surface of the fields.
So important were these floods that their height was measured with
specially constructed Milometers, Roman examples of which can be
seen at, for example, Aswan and Luxor, with a fine medieval one at
Cairo. The level of taxation was adjusted to the height of the water: a
good flood would mean a good crop, and the populace would be able to
stand higher taxes. Pliny (Historia Naturalis, 5.58) is quite specific
about the importance of flooding:
An average rise is one of seven metres. A smaller volume of water does not irrigate
all localities and a larger one, by retiring too slowly, retards agriculture; and the
latter uses up the time for sowing because of the moisture of the soil while the
former gives no time for sowing because the soil is parched. The province takes
careful note of both extremes: in a rise of five-and-a-half metres it senses famine and
even at one of six metres it begins to feel hungry, but six-and-a-half metres brings
cheerfulness, six-and-three-quarters complete confidence and seven metres
delight, (trans A. Bowman)
Rome's reliance on Egyptian grain has a long history, stretching
back to the Ptolemies, when, as early as 211 or 210 BC, Rome requested
a consignment of grain from Ptolemy IV. The arrival of the Alex-
andrian grain ships was to become an important element in the
economy of Rome, upon which the fates of emperors might hang.
Under Augustus it may have reached 20 million modii (well over
i million tons). The corn trade was part of the annona, the tax in kind
levied by Rome on the producing provinces. There is some evidence to
suggest that even the cost of transport from the estate to the Nile had to
be met by the producers.
The supply of grain from the growing areas to the warehouses of
Alexandria was a carefully regulated operation. The shipment was
carried out by the sitologos (corn official) aided by the antigrapheus
(clerk) and a financial assistant.
A sealed sample or deigma would be entrusted to the boat's captain
for delivery with the consignment. This would be a check against
THE ROMAN P E R I O D 421
adulteration or substitution of the cargo with one of lower quality
during the voyage. In any case, it seems to have been normal practice
for a soldier to be present on board during the journey. On arrival at
the great granaries in Alexandria, the corn would be in the care of
special Roman procurators who, with their staff, were responsible for
its safe-keeping and condition.
The corn ships generally left Alexandria in May or June and the
journey to Rome, against the prevailing northerly winds, could take a
month or perhaps even two. The route would be along the North
African coast or north to Cyprus, then hugging the south coast of
Turkey. The return with a tail wind took about a fortnight, the ships
travelling 'with the speed of racehorses', as the emperor Gaius
claimed. Either way, the journey was not without its hazards, as St
Paul's shipwreck on Malta vividly illustrates.
Archaeologically, we know very little about the estates that produced
this corn, but the papyri known as the Heroninos archive permit the
detailed reconstruction of the working of one of them operating during
the third century AD, the Appianus estate in the Faiyum. It appears that
the owner of the estate, Aurelius Appianus, was a landowner of some
standing with holdings comparable with those of Roman senators. His
central administrators, bound by patronage, were recruited from the
town councillors and landowners of the nome, and below them were
the phrontistai or production managers, probably recruited from
wealthy rural families, who perhaps worked for several estates simul-
taneously. The labour was provided by a nucleus of full-time workers
supplemented by extra labourers when needed. It seems that the
supply of paid labour from the poorer classes in rural Egypt made it
unnecessary and uneconomic to seek slave labour.
There were three categories of full-time labourers: the paidaria,
oiketai, and metrematiaioi. The first two categories seem to have been
employed for life and perhaps provided with free accommodation,
while the metrematiaioi were independent villagers contracted to work
for a varying set number of years. Casual labourers came from many
different backgrounds sometimes outside the village.
The primary aim of the unit was the production of wine for external
sale. The other crops were grown to provide food for the employees,
fodder for the estate draught animals, and grain for tax. All of these
were necessary to permit the economic functioning of the estate. It is
thus apparent that the grain for which Egypt was renowned was pro-
duced as part of a complex and sophisticated system of farming, which
made profits in other ways.
422 DAVID PEACOCK
The mineral resources of the Eastern Desert were known and
exploited during pharaonic times. For example, the amethyst mines of
Wadi el-Hudi have yielded a stele recording the use of the army in
mines operated under Senusret I of the Middle Kingdom. Further-
more, the New Kingdom temple of Sety I at Abydos was granted rights
at the gold mines in the Eastern Desert, a gang of workmen to bring
back the gold, and a settlement at the mines themselves. These may
well be the mines at Umm el-Fawakhir in the Wadi Hammamat,
which still in use at the end of the twentieth century. A remarkable
papyrus map in the Turin Egyptian museum is thought to depict this
area.
Interest in the mineral resources, particularly gold, persisted
through the Ptolemaic period into Roman times. Finds of black gloss
pottery at sites such as Abu Zawal, about 20 km. west of Mons
Claudianus, suggest that it, and probably other mines, were estab-
lished before the Roman conquest, although they undoubtedly con-
tinued to operate after it.
Gold working sites have been little studied, but their appearance is
distinctive. There is usually a cluster of small huts surrounded by
stone heaps, and everywhere there is evidence of the apparatus used to
crush the quartzite from which the ore would be extracted. The prin-
cipal tool seems to be a well-made type of curved saddle quern with a
heavy upper stone shaped like Napoleon's hat, the 'brim' of which
formed the handles. Water would be needed in considerable quantity
to separate the pay dirt from the gangue, and some sites, of which Abu
Zawal is characteristic, have a substantial well forming the core of the
complex. In other cases the crushed rock would have been taken away
and separated elsewhere.
The method of extracting the gold was observed by the Greek geo-
grapher Agatharchides, who visited the mines in the second century
BC. His original work has been lost, but fortunately his description has
been preserved in the writings of Diodorus Siculus. He tells us that the
rock was broken by fire setting and the use of hammers. It was then
crushed in large stone mortars to the size of a pea, after which it would
be ground to a fine powder in hand mills before being washed with
water on a sloping surface to separate the gold and country rock.
Presumably, the saddle querns, now so much in evidence on these
sites, were used in the final grinding.
Stone quarrying also has a long ancestry in Egypt. The most cele-
brated example must be the great complex at Aswan, now unfortu-
nately much disturbed and built over by the expansion of the modern
THE ROMAN P E R I O D 423
town. Aswan produced a variety of granitic rocks, the most celebrated
of which was the red- or rose-coloured granite. During the pharaonic
period it was used for sarcophagi, for obelisks, and as capping for the
great pyramids of Giza, perhaps because its reddish colour suggests
the sun. During the Roman period, the quarries continued unabated,
and columns carved from Aswan granite are found in quantity around
the shores of the Mediterranean. It is, in fact, one of the 'big three'
decorative rocks of the Roman world, on a par with the granito violetto
from the Troad and Cipollino from Greece.
The success of Aswan clearly results from its location on the banks
of the Nile. The products could easily be loaded onto barges and floated
to Alexandria, where they would be transferred to the lapidariae naves,
the special stone ships used for transporting heavy loads across the
Mediterranean. Other successful quarries such as those for sandstone
at Gebel el-Silsila or those for 'Egyptian alabaster' (or 'calcite alabaster')
in Middle Egypt, were also situated within easy reach of the Nile
(although the principal calcite-alabaster quarries at Hatnub were at
least half a day's journey away, and presumably somewhat longer
when hauling large blocks). At Aswan, the quarries seem to have had a
long life, the Romans continuing a tradition of quarrying of several
thousand years.
For obvious logistical reasons, the large-scale quarrying of remote
desert stone (for use in buildings or sculpture) was eschewed by the
pharaohs, with the exception of bekhen, a greywacke sandstone from
Wadi Hammamat, and even more remarkably, the so-called 'Cheph-
ren diorite', an anorthosite-gneiss from Gebel el-Asr in the Western
Desert about 200 km. south-west of Aswan. During the Roman
Period, however, an attempt was made to exploit the very considerable
lithic resources of the desert more thoroughly, and the focus was the
Eastern Desert, where a range of hard basement rocks was exploited,
comprising mainly porphyry and varieties of diorite.
The centre that articulated most of this activity seems to have been
Mons Porphyrites (Gebel Dokhan), about 70 km. north-west of Hur-
ghada. Ostraca from Mons Claudianus state that the men working
there were part of the numerus of Porphyrites and the arithmos of
Claudianus. Similarly, the workers at nearby Tiberiane (Barud), the
source of the granito bianco e nero, seem to have been of the numerus of
Porphyrites and the arithmos of Tiberiane. To this can be added the
scatter of tiny chips of porphyry found on most quarry sites in the
Eastern Desert, suggesting that men who had worked the porphyry
were being sent to other quarries.
424 DAVID PEACOCK
A recently discovered inscription documents the discovery of this
area in a remarkable way. It tells us that the resources were found by
Gaius Cominius Leugas, who must have been the Roman equivalent
of a field geologist, on 23 July AD 18. He appears to have discovered por-
phyry, black porphyry, multi-coloured stones, and knekites fsafflower
stone'), which has yet to be geologically defined.
The dating of the earliest quarrying at Mons Porphyrites to the reign
of Tiberius (AD 14-37) i s confirmed by a further inscription, and it
appears to have persisted until the late fourth or possibly even the early
fifth century AD, if the pottery dating is correct. Purple had been worn
as a mark of nobility in the Mediterranean region for many thousands
of years and no doubt the discovery of a purple rock would have been a
major event of considerable interest to the emperor personally. The
operation itself has been described, with some justification, as the
most remarkable manifestation of Roman activity to be seen anywhere
in the empire. It was necessary to supply the quarries with food, to dig
wells tapping the fossil water (which, contrary to popular belief,
abounds in the desert), and to construct forts for the military and
villages for the workers. While the two might have cohabited to some
extent, the quarries are on the tops of mountains and it was convenient
to post workers nearer to their place of labour. The site seems to have
begun as a series of scattered mountain villages, which were later, in
the second century AD, to be controlled by a fort at wadi level. In the late
Roman period convicts may have been used, and a passage in the
writings of Eusebius refers to a group of Christians (almost certainly
quarry-workers) who had their eyes gouged out and their hamstrings
cut before being deported to Palestine—presumably for trying to
proselytize the garrison. However, for much of the time the operation
was probably manned by civilians and soldiers working together,
which was certainly the case at Mons Claudianus. Even Christianity
was generally tolerated, as a number of inscriptions attest.
Mons Claudianus, some 50 km. to the south of Mons Porphyrites,
was the source of a grey granodiorite used mainly for columns. This is
now the most intensively studied of the Roman quarry sites in the
Eastern Desert. The complex comprises a fort of Domitianic date, and
an earlier one that has produced an ostracon of Nero, with 130 small
quarries scattered within a radius of about i km. around; each was
connected to the main wadi bed by a slipway, which terminated in a
loading ramp—the place where the products would be transferred
from rollers or sledges to carts for the i2o-km. journey across the
desert to the Nile. Some of these carts must have been very large, for a
THE ROMAN P E R I O D 425
2o-m. column would weigh over 200 tonnes. Here, it is pertinent to
note that an ostracon refers to a twelve-wheeled cart and, in the Naq el-
Teir plain, tracks have been observed with a span of 3 m.
It used to be thought that the rock of Mons Claudianus, also known
as the granito del foro, from its frequent occurrence in the Roman
forum, had a pan-Mediterranean distribution. However, a programme
of chemical and petrographic analysis during the 19905 has shown
that its distribution is virtually restricted to some of the finer monu-
ments in Rome. It appears that Mons Claudianus lay outside the
normal orbit of Roman trade and may have been more or less the
personal property of the emperor. It is interesting to note that other
grey rocks of similar appearance were exploited in more accessible
outcrops on the islands of Elba and Giglio, and also in western Turkey.
The rock of Mons Claudianus was special, not because of its proper-
ties, but because of where it came from. It was a product from the
utmost end of the empire and could be won only by extraordinary
efforts. This could be the secret of the whole quarrying enterprise in
the Eastern Desert, which makes little sense in rational economic
terms.
The importance of Egypt to the Roman economy went beyond pro-
duction. Perhaps one of the strangest and most bizarre aspects of taste
among the Roman nobility was the predilection for oriental luxuries:
pearls, pepper, silks, frankincense, and myrrh, as well as various other
spices and exotic medicines. Egypt articulated this trade, for these
goods were brought by ship across the Indian Ocean and thence to the
western shores of the Red Sea. Here they were offloaded and dragged
across 150 km. of desert to the Nile, whence they were floated to
Alexandria and then on to Rome. India benefited from this trade, for in
return it received glass, textiles, wine, grain, fine pottery, and precious
metals as well as human cargoes, such as singing boys and maidens
for the pleasure of Indian potentates.
It might be thought advantageous for the ships to sail up the Red
Sea and to cross the isthmus now occupied by the Suez Canal. Indeed,
there was a project, begun under Ptolemy II and improved by various
successors, particularly Trajan and Hadrian, that connected the Nile
with the Bitter Lakes. However, it was not extensively used, at least in
the first centuries BC and AD, largely because of the severe northerly
wind that blows down the Red Sea for 80 per cent of the year. This would
have been a major hazard to Roman shipping and it was preferable to
make a more southerly landfall and to take goods overland to the Nile.
The two ports established by Ptolemy II Philadephus (285-246 BC) to
426 DAVID PEACOCK
facilitate this trade were Berenice, named after his wife, and Myos
Hormos. It appears that Myos Hormos was pre-eminent during the
second century BC and that Berenice began to rise in importance
during the first century BC and became dominant in the first century
AD, although Myos Hormos continued in use. The India trade was thus
developed in Ptolemaic times and the Romans merely took over and
perhaps expanded a well-established concern. The Red Sea would also
have been known to pharaonic traders, for it undoubtedly gave access
to the mysterious East African land of Punt (see Chapter 11), from
whence came exotic plants and animals.
The site of Berenice is well established and has been equated with
the ruins near Ras Banas in southern Egypt, since its discovery by
Belzoni in 1818. Myos Hormos has been the subject of extended
debate, most writers siting it at Abu Sha e ar, 20 km. north of Hurghada,
because this accords with the latitude and longitude given in Ptolemy's
Geography. However, the 19905 excavations of the little fort on the site
demonstrated that it is a late Roman and Byzantine foundation, with
no evidence of earlier settlement. However, the site of Myos Hormos is
described in some detail in the Classical literature, and study of satel-
lite images suggests that the closest correspondence is with the site of
Quseir el-Qadim at the end of the fortified road from Koptos on the
Nile. This diagnosis has recently been confirmed by excavations at
el-Zerqa about halfway along the route, for these have produced
ostraca demonstrating beyond reasonable doubt, that the port at the
end of the road was indeed Myos Hormos.
The nature of this trade can be filled out from both literary and
archaeological sources. The main document is the Periplus Maris
Erythraei, a sailing guide to the Red Sea, the Gulf of Aden, and the
western Indian Ocean, compiled in the first century AD. It is sup-
plemented with references in Tamil poems to 'cool fragrant wines
brought by the Yavana in their ships' or again 'the thriving town of
Muziris, whither the beautiful large ships of the Yavana come bearing
gold, making the waters white with foam and return laden with
pepper'. It appears that the best time to leave Egypt was July, when the
south-western monsoons would drive the ships across the Gulf of
Aden and the Indian Ocean, while the return would be delayed until
November to take advantage of the north-eastern monsoons.
The south-western monsoons are some of the most ferocious winds
on earth, and the ships must have been immensely large and strong to
withstand such a voyage, perhaps akin to those on the Alexandria-
Rome run, which were up to 60 m. long and had a displacement of
THE ROMAN P E R I O D 427
Maps efforts in the Eastern Desert and the routes from the Red Sea
ports Berenice and Myos Hormos (Quseir el-Qadim) to the Nile
in the Roman Period (30 BC-AD 395)
around 1,000 tons. Certainly the profits would have made the risks
worthwhile: a recently published papyrus describes a shipment of nard
(aromatic plants), ivory, and textiles from Muziris in India to Alex-
andria; this consignment had a value of 131 talents, enough to buy
2,400 acres of the best farmland in Egypt.
Archaeology can also help in understanding this trade. Long ago, Sir
Mortimer Wheeler excavated the Roman settlement of Arikamedu on
the Coromandel coast of India, where he found amphorae that would
have contained the best wine of Campania, and fine red pottery of
Tiberian date, produced in the workshops of Lyons, Pozzuoli, and Pisa.
In Egypt, an excavation project during the 19908 at Berenice promises
to furnish equivalent information on the Egyptian end. In the late
19705 and early 19805 small-scale excavations at Quseir el-Qadim,
428 DAVID PEACOCK
then thought to be the port of Leucos Limen, produced interesting
material, including a sherd with a Tamil inscription on it.
The overland routes from Berenice and Myos Hormos across the
desert have been thoroughly studied. The one from Berenice runs in
a north-westerly direction for over 350 km. and is equipped with
hydreumata (watering places) every 20-30 km. Its destination is Koptos,
but about halfway along there is a branch to the west that leads to
Apollinopolis Magna (Edfu). The route from Myos Hormos again
leads to Koptos, and Strabo tells us that the journey took six or seven
days, the route being furnished with hydreumata dug to a great depth.
Two of these (el-Mweih and el-Zerqa) were excavated in the 19905,
producing new documentary evidence in the form of ostraca, the pub-
lication of which is awaited.
The final leg of the trade, from Alexandria to Rome, may well have
been intimately connected with the annona (the tax in kind, mentioned
above), since shippers who served the state were able to carry some of
their own goods free of tolls. However, this is by no means the whole
story. Alexandria has produced many more examples of Baetican oil
amphorae than any other major city in the eastern Mediterranean, a
single example that serves to emphasize its role as a major port for
inter-regional trade of all sorts and in all directions. To Strabo it was
the greatest port in the world, and, of course, its Pharos or lighthouse
was one of the wonders of the ancient world.
Religion
There can be no aspect of Roman Egypt more complex or more diffi-
cult to understand than religion. In effect, Rome inherited pharaonic
religion, on which a classical gloss had been superimposed, largely
during the preceding Ptolemaic period. Visitors to the ancient temples
of Egypt usually think that they are looking at masterpieces of the
Dynastic era, but in many cases—Dendera, Edfu, Kom Ombo, Esna
and Philae, for example—the extant structures are substantially
Ptolemaic and Roman.
Although the first and most striking aspect of Egyptian religion is
polytheism, there were a number of overriding beliefs (for further dis-
cussion, see the section on New Kingdom religion at the beginning of
Chapter 10). Thus, such gods as Ra, the sun, Geb, the earth, and Nut,
the sky, seem to have been worshipped almost everywhere in Egypt.
There was, however, also a tendency towards monotheism. Ra was the
THE ROMAN P E R I O D 429
source of everything, Ptah is described as 'the heart and tongue of the
gods', and in the mid-fourteenth century BC Akhenaten decreed that
Aten was the one god that should be worshipped. Another readily
observed feature of Egyptian religion is the partiality for animal cults.
For example, Horus is represented by a falcon and Hathor by a cow. It
was not, however, the animals themselves that were the focus of the
worship, but the gods that chose to take on their forms. From this
arose the custom of mummifying animals, often by the thousand:
crocodiles, baboons, cats, the Oxyrhnchus fish, and so on.
Each of this plethora of gods had his or her own role to play, but the
situation is far from simple, because their roles changed through time,
and gods could merge together so as to become all but indistinguish-
able from one another. Thus Horus, the falcon, shown with a sun disc,
is often indistinguishable from the sun-god Ra. Amun was originally
the god of water and air, but later became the god of physical repro-
duction, the giver of life.
After Alexander's conquest in 332 BC, Greek culture became
implanted, not only in the Greek cities of Alexandria, Naukratis, and
Ptolemais, but also in the Greek communities scattered throughout
the land. The Greeks identified their own gods within the Egyptian
spectrum. Thus, Horus was equated with Apollo, Thoth with Hermes,
Amun with Zeus, Hathor with Aphrodite, and so on. How the beauti-
ful Athene would have reacted to being equated with the hippo-
potamus-goddess Taweret, we do not know.
A good example of this process of hellenization is the god Pan. He
was equated with Amun-Min, the god of sexual reproduction, who had
an important sanctuary at Koptos. The city is at the end of desert roads
leading to the east. Amun-Min thus became the god of the east and was
shown with an incense burner, perhaps symbolizing the spices and
perfumes of the orient. From these beginnings, during the Roman
period, Pan became the god of the Eastern Desert, the capricious
guardian of the desert routes. He is shown not as the Pan of Greek
mythology, but as the ithyphallic Min, his erection clearly inherited
from his previous life.
During Ptolemaic times, a new god called Serapis was invented with
the object of giving a greater degree of political and religious unity.
Unlike the traditional pharaonic-period deity Osirapis, from whom he
derived, he is shown not as an animal but as a bearded man, not unlike
Zeus: of all the Egyptian gods, he is the most similar to a Graeco-
Roman god. Serapis became immensely popular at Memphis, the old
capital of Egypt, and then at Alexandria, when the seat of government
430 DAVID PEACOCK
was moved there. Eventually the cult gained adherents in Sabratha and
Lepcis, Rome, and later Ephesus and the Danube provinces.
Another very popular god in Roman Egypt was Isis, sometimes
identified with Hathor. She was both wife and sister to Osiris, who was
judge and ruler of the dead and supreme god of the funerary cult. Her
role was that of a prototype for motherhood and the faithful wife. She
was much adored by women, to whom she was queen of heaven and
earth, of life and death. She looked favourably on all women's activities
to such an extent that she was at one time the goddess of prostitutes as
well. As in the case of Serapis, Isis' worshippers were to be found all
over the empire, particularly in Spain. The rituals associated with her
cult changed little from pharaonic times: at dawn her statue would be
uncovered and adorned with jewels while the sacred fire was lit—all to
the accompaniment of sacred music.
Just as the gods of Roman Egypt were essentially Egyptian gods, so
temple architecture forms a continuum with Dynastic and Ptolemaic
temples. The exception is the Paneion, which because of Pan's special
role in the desert may be situated away from habitation in remote
spots. Often they were no more than a rock on which travellers would
write their dedications. A fine example of this is to be seen in the Wadi
Hammamat.
The temple of Hathor at Dendera provides a good idea of the appear-
ance of a late Ptolemaic-Roman temple. The propylon (north gate) is
the work of Domitian and Trajan, but the main focus of the complex,
the beautifully preserved temple of Hathor, was constructed between
125 BC and AD 60. The front of the building has a massive facade
marked by six columns with Hathor-headed capitals surmounted by a
cornice. The entrance leads to a hypostyle hall, built in the twenty-first
year of Tiberius by Aulus Evilius Flaccus, with the aid of the inhabit-
ants of the town and district, and its roof is supported by Hathor-
headed columns. The hall leads through to an inner hypostyle hall and
two Vestibules', the inner of which contains the sanctuary, sur-
rounded by a number of chapels. The ornament is characteristically
Egyptian, but many of the subjects are Roman emperors. Thus we see
Tiberius before the gods, Claudius making an offering to Hathor and
Ihy, and representations of Augustus and Nero. The whole complex is
a strange experience for a student brought up on classical scholarship.
Another fine example of a Roman temple is Trajan's kiosk at Philae,
preserved on an island between Aswan and the High Dam. This
elegant and finely proportioned building has fourteen columns with
bell capitals and screen walls, two of which are decorated with scenes
THE ROMAN P E R I O D 431
representing Trajan making offerings to Isis, Osiris, and Horus. The
symbolism of all these temples must have had a very special message
to the population of Roman Egypt. Here there is no question of the
emperor as god; he is seen as a supplicant to the great gods of old
Egypt.
However, from the mid-first century AD onwards a new phenom-
enon appeared on the religious scene: Christianity. It seems to have
taken root in Alexandria, whence it spread to the rest of the country. No
doubt, with so many cults in existence, one more could be accepted
and absorbed. However, Christianity was an uncompromising
religion that did not see itself on a par with the others and actively
sought to win converts from paganism. The old order was threatened,
and from the mid-third century onwards persecution began in a
sporadic way until its culmination in the great purges of Diocletian,
begun in AD 303.
In the third century AD there emerged a new trend in religious
practice that was to sweep the world. The desert is a religious testing
ground, away from the hurly-burly of ordinary life where survival
depends on reliance on God. Christ had already set the scene when he
spent forty days and forty nights in the desert undergoing the tempta-
tions of the devil. In the late third century, according to tradition, two
young rich men, Paul the first hermit and Anthony the first monk,
each separately left their homes in the Nile Valley to live in the solitude
of the wilderness. How they survived is not really a mystery, because
holy men everywhere are treated with respect and fed by people they
encounter. Since they both settled by springs, no doubt they would
have been visited by Bedouin who would have known of the water
source and had rights there. Eventually, despite his isolation, the fame
of Anthony spread and even the emperor Constantine wrote to him
asking for prayers. He was visited by his old disciples, various dig-
nitaries, pilgrims, and, of course, curious sightseers. The coming and
going of visitors led to the establishment of a caravanserai, which
eventually developed into a monastery—the most significant monas-
tery in Christendom, from which all others derived.
Burial customs are, of course, intimately connected with religious
practices. It is not surprising, therefore, that the practice of mummifi-
cation persisted alongside paganism—in some cases as late as the
fourth century AD. The poor might receive the simplest burial as
plainly bandaged mummies, but the rich would be given an elaborate
mummy case, as pharaonic tradition dictated. During the Roman
period encaustic portraits painted on board were set into the head of
432 DAVID PEACOCK
the mummy case. These minor works of art are some of the most vivid
and realistic to be seen anywhere in the Roman world. No doubt they
would be commissioned from a highly skilled artist and, as they have
an almost photographic degree of realism, they appear to have been
executed while the individual was still alive. It has been suggested that
they were painted during the prime of life and success, and were then
kept for their eventual funerary use.
In Alexandria, there is evidence for an alternative style of burial,
perhaps reflecting a different taste amongst the wealthy inhabitants of
Greek origin. In the Kom el-Shugafa (the hill of the potsherds) is a
complex of catacombs dating to the second century AD. It comprises a
circular stairwell leading to a complex of burial chambers and a
banqueting hall where mourners visiting the tombs could dine in close
proximity to the deceased. While it was originally designed for the
wealthy, it seems to have been extended to the poorer classes, for there
are many small niches to accommodate unpretentious burials.
Artistically the decoration is of some interest, deriving elements from
both the Greek and the Egyptian canons. There are false sarcophagi
decorated with masks, ox skulls, and festoons, but elsewhere are reliefs
depicting deities such as Anubis or Thorn.
Crafts and Trades
Minor arts and crafts are in abundant evidence in Roman Egypt.
Almost every site of this period is littered with pottery, glass, and
faience as well as organic materials, which are not normally seen in
more temperate climates, such as basketry, textiles, and leather.
Because of the architectural richesse of Egypt and the wealth of written
evidence, everyday crafts have received less attention than they merit.
Their potential for the analysis of trade, chronology, and technology
has yet to be realized, but since the 19805 particularly the systematic
study has commenced and is beginning to show interesting results.
Pottery is widely acknowledged to play a vital role in many aspects of
archaeological enquiry. Imports to Roman Egypt such as wine jars
from Italy and France, oil jars from Spain, fine red wares from North
Africa, or lamps from Italy can be recognized and dated. Their import-
ance is undeniable and they are beginning to throw light on trading
contacts with the rest of the Mediterranean. However, our knowledge
of the local Egyptian wares is still relatively limited. Most assemblages
are dominated by jars made from 'Nile silt', a dull dark-brown clay
characteristic of the Nile floodplain. There is every reason to believe
THE ROMAN P E R I O D 433
that these were being produced at many potteries along the Nile Valley
and in the Delta, but there is a marked archaeological lacuna and we
know of only a few kiln sites—all of them situated on the southern
shore of Lake Mareotis near Alexandria and all discovered through the
researches of one man, Jean-Yves Empereur. These Alexandria kilns
appear to have been producing a type of amphora that is not closely
datable and that appears on a majority of Roman sites in Egypt. In the
third century, the kilns may have been producing imitations of Koan
amphorae, presumably because they were destined to contain Koan-
style wine, which was a medicinal variety made with sea water.
At the other end of Egypt, pottery with a red slip or wash was made at
Aswan, and it is again found widely throughout the country, particu-
larly in first- and second-century contexts. However, this is most
certainly only part of the story and there must have been many other
establishments along the Nile Valley producing jars or fine table wares
such as the Egyptian 'red slip ware' first defined by John Hayes.
Among the papyri from Oxyrhnchus are three that are leases for
potteries. It appears that production was closely linked to the estate.
The lessor, presumably the estate-owner, agrees to provide the pottery
building, the storeroom, the wheel, the kiln, the clay, and the fuel for
firing, in return for which the lessee must provide his own workforce
and supply the lessor with a very large number of jars, in one case in
excess of 15,000, which must have been destined to contain the
produce of the estate. It is unfortunate that it is not possible to link this
fascinating documentary evidence of estate production to the actual
pottery or even to the type of pots produced.
Throughout most of the Roman world, fine table wares take the
form of red gloss wares, produced in Gaul, Italy, or the East. While
these are also found in Egypt, their place is taken by brilliant blue or
green faience vessels. Faience is not pottery but a glazed quartz frit
formed by grinding quartz and mixing it with an alkali salt and a
colourant such as a copper salt. There are several ways of making
faience, all of which produce much the same end result: for example, a
core of fine quartz cemented with alkali can be packed into a glazing
mixture of plant ash, copper oxide, and lime, or the frit can be prepared
and painted onto the fashioned core. Alternatively, as the quartz dries,
the colourant is drawn to the surface so that, on firing, it fuses to
produce the characteristic glaze. Faience cannot be thrown and was
usually formed by moulding: it is thus more suited to the production of
beads and figurines, but in the Roman period it was used for plates,
dishes, and drinking cups. We know little about the production of
434 DAVID PEACOCK
Roman faience and it is unfortunate that the one kiln site known, at
Memphis, was excavated early this century before modern techniques
of observation and recording had been developed.
Glass is another common component of Roman rubbish deposits.
Much of it is of surprisingly fine quality, often thin walled and clearly
very accomplished. Even on desert sites the vessels may be blown,
mould-blown, or with multi-coloured ornament or cut decoration. At
present it is unclear how much of this was imported from the great
glasshouses of Syria or how much was locally produced. Alexandria is
described by Strabo and other later writers as a great centre for glass-
making, perhaps making some of the finest polychrome vessels, but
archaeologically we know very little about it. There were certainly other
glasshouses, judging from the guild of glass workers mentioned in the
Oxyrhnchus papyri.
The production of flour was an important trade closely connected
with subsistence. Rotary querns were certainly in use, but the type of
mill most commonly encountered is the lever or 'Olynthian' mill. It
comprises a slab of stone about 50 sq. cm. with a slot, which forms the
hopper, in the middle. A lever is fixed across the top of the stone, which
is oscillated to and fro around a pivot. Examples have been found in the
Greek settlement of Naukratis, but also at Tanis, in the Faiyum, at
Quseir el-Qadim, and in the forts of Tiberiane (Barud) and Mons
Porphyrites. It is almost certain that this type of mill was introduced by
the Greeks, where the type continued in use until at least the third
century BC. However, in Egypt they certainly persisted into the Roman
period and the example from Quseir belongs to the first century AD,
while those from the forts are certainly of first- or second-century AD
date. The fort at Badia, in the Mons Porphyrites complex, has produced
the components of segmented mills in lava probably from the Greek
island of Nisyros. The type is known from Delos, although the
examples from Badia could be of late Roman date.
In the ancient world it seems that Egypt was renowned for its tex-
tiles, and significant collections, largely of the later Roman period,
have been recovered from the towns of Antinoopolis and Panopolis,
where there may have been woollen mills. Again Alexandria seems to
have been important, supporting a linen trade and the reworking of
oriental silks.
Other crafts that might be mentioned are the growing and manu-
facture of papyrus, the manufacture of drugs and medicines, the pro-
duction of jewellery, leather working and metalworking, all of which
are still inadequately studied.
THE ROMAN P E R I O D 435
Demography
The demography of Roman Egypt during the first three centuries AD is
well documented, for we have about 300 papyri recording census
returns. These returns detail not only members of families living in
the Nile Valley, but also their lodgers and slaves.
Estimates of the population of Roman Egypt are fraught with diffi-
culty, not least because the two principal historical sources contradict
one another. Diodorus Siculus puts the population in the first century
BC at 3 million, while Josephus, writing in the first century AD, gives a
figure of 7.5 million exclusive of Alexandria. On the whole, modern
scholars find the figure given by Diodorus to be the more credible.
Alexandria, one of the most populous cities in the ancient Med-
iterranean, is said by Diodorus to have a population of 300,000, which
is not so far removed from the modern estimates of around 500,000.
It can also be argued that the rural population was distributed over
2,000-3,000 villages, each with an average population of around
1,000-1,500, giving a total figure of 3 million, which accords well with
the probable rural population before the nineteenth century. Such
calculations by modern scholars give a total population of 4.75 million,
of which 1.75 million lived in the towns.
The census returns enable us to flesh out these bare figures. It
seems that around two-thirds of households comprised conjugal
families (with their siblings) or multiple families linked by kinship,
while most of the remaining households were occupied by solitary
persons or by families extended by the presence of co-resident kin.
Lodgers seem to have been comparatively rare. Slaves, on the other
hand, constitute about n per cent of the total population. Since the
returns give ages, it is possible to estimate death rates. Among women,
it seems that very few lived to their sixties, and female life expectancy at
birth was probably in the mid- to low twenties. Male life expectancy, on
the other hand, was at least twenty-five years. The sex ratio of the 1,022
persons whose sex can be adduced was 540 males to 482 females, but
among slaves it is reversed (thirty-four males to sixty-eight females).
Marriage in Roman Egypt was a legal status that had consequences
for the offspring, but weddings and divorces were private matters in
which the state did not intervene. The wife would nearly always live in
the husband's household, often with his extended family. About a
sixth of all marriages were those between brothers and sisters. Most
women would have married by their late teens and virtually all by their
late twenties, but only half of all men had married by the age of 25. The
436 DAVID PEACOCK
average age of women at maternity was around 27 years. The demo-
graphic picture of Roman Egypt thus corresponds closely with that of a
typical pre-industrial Mediterranean population.
The Nature of Roman Egypt
All Roman provinces were an amalgam between the influence of
Rome and indigenous culture. In most cases, the former more or less
subsumed the latter. Thus, in Roman Britain or Gaul, for example,
traces of the pre-existing Iron Age persist, but the most marked aspect
is the change to a Mediterranean style of life. Only in Egypt, and per-
haps to some extent in the Greek lands of the north-eastern Med-
iterranean, is the Roman period an essay in continuity with what went
before. At least one of the reasons for this must lie in pharaonic archi-
tecture. The creation of a landscape dominated by buildings made of
massive blocks of stone, which were not easily swept away, must have
been a major factor. They served exactly their intended purpose: to
remind people of the greatness of pharaonic civilization and to be a
constant witness to the beliefs and values of that period of Egyptian
greatness. This may not be the only reason, but it must have been a
contributory factor.
It would be wrong to suggest that the Roman era was one of stag-
nation or that there was no change at all during the seven centuries
that lay between the death of Cleopatra on i2th August 30 BC and the
Arab conquest of AD 642. However, the major cultural change took
root in the third century AD, when Christianity gained widespread
acceptance, as it did throughout the empire generally. Monasticism
had its roots in the Egyptian desert led by people such as St Paul and
St Anthony. Even here pharaonic culture was not without its influence,
for Anthony started his religious life living in an old tomb near his
village on the Nile and it is here that he wrestled with demons and wild
animals, before making his journey into the wilderness.
EPILOGUE
In the Book of Gates (a series of funerary texts and images used to decorate
late New Kingdom tombs), the Egyptians represented the infinity of time
with an apparently endless snake or with a doubly twisted rope being spun
from the mouth of a deity (stars above the twists in the rope serving as indi-
cations of the passage of units of time). In this image, time is evidently
conceived as a phenomenon emerging from the original depths of creation
and eventually falling back into the same depths. It is this universal,
cyclical nature of time that pervades the ancient Egyptians' sense of their
own history. In this history we have untangled some of the twists and
caught glimpses of events as they vanish back into the mouth of the snake,
but we can ultimately reach a true understanding of Egyptian history only
if we combine archaeological and textual evidence into a complete
patchwork of material culture and politics, as the contributors to this
volume have attempted to do. All ancient history tends to be more or less
fragmentary and elusive but the sheer diversity of Egyptian sources
occasionally allows certain historical episodes or ways of life to spring very
sharply and vividly into focus.
FURTHER READING
Abbreviations
African Archaeological Review
American Journal of Anthropology
Annales du Service des Antiquites d'Egypte
Bulletin of the American Schools of Oriental Research
Bulletin de I'lnstitut Franc,ais d'Archeologie Oriental
Bibliotheca Orientalia
Bulletin de la Societed'Egyptologie de Geneve
Bulletin de la Societe Franc, aise d'Egyptologie
Chronique d'Egypte
Gottinger Miszellen
Journal of the Anthropological Institute
Journal of the American Research Center in Egypt
Journal of African Studies
Journal of Egyptian Archaeology
Journal of Field Archaeology
Journal of Near Eastern Studies
Journal of Roman Archaeology
Journal of Roman Studies
Journal of the Society for the Study of Egyptian Antiquities
Journal of World Prehistory
Lexikon der Agyptologie, ed. W. Helck et al.
(Wiesbaden, 1975-86)
Mitteilungen des Deutschen Archaologischen Instituts,
Abteilung Kairo
Memoires de I'lnstitut Franc.ais d'Archeologie Oriental
Metropolitan Museum Journal
Orientalia Lovaniensa Periodica
Proceedings of the Prehistoric Society
Revue d'Egyptologie
Studien der altagyptischen Kultur
Varia Aegyptiaca
World Archaeology
Zeitschrift der fur agyptische Sprache und Altertumskunde
AAR
AJA
ASAE
BASOR
BIFAO
BiOr
BSEG
BSFE
CdE
GM
JAI
JARCE
JAS
JEA
JFA
JNES
JRA
JRS
JSSEA
JWP
LA
MDAIK
MIFAO
MM]
OLP
PPS
RdE
SAK
VA
WA
ZfS
FURTHER R E A D I N G 439
General
As far as histories and general reference works are concerned, the follow-
ing are well worth consulting: I. E. S. Edwards et al. (eds.), Cambridge
Ancient History, vols. 1-4, 3rd edn. (Cambridge, 1971-3); Bruce Trigger et
al, Ancient Egypt: A Social History (Cambridge, 1983); Barry Kemp, Ancient
Egypt: Anatomy of a Civilization (London, 1989); Nicolas Grimal, A History
of Ancient Egypt (Oxford, 1992); Donald Redford, Egypt, Canaan and Israel
in Ancient Times (Princeton, 1992); Jean Vercoutter, L'Egypte et la vallee du
Nil, i. Des origines a la fin de Vancien empire (Paris, 1992); Claude Vander-
sleyen, L'Egypte et la vallee du Nil, ii. De la fin de I'ancien empire a la fin du
nouvel empire (Paris, 1995); Ian Shaw and Paul Nicholson, The British
Museum Dictionary of Ancient Egypt (London, 1996); Serge Donadoni (ed.),
The Egyptians (Chicago, 1997); Judith Lustig (ed.), Anthropology and Egypt-
ology: A Developing Dialogue (Sheffield, 1997), and Regina Schulz and
Matthias Seidel (eds.), Egypt: The World of the Pharaohs, English edn. ed.
Peter Der Manuelian (Cologne, 1998).
There are numerous general works on the art and literature of Egypt and
the following are only a small selection: Cyril Aldred, Egyptian Art in the
Days of the Pharaohs (London, 1980); Bernard Bothmer et al., Egyptian
Sculpture of the Late Period 700 BC to AD 100 (New York, 1960); Raymond
Faulkner, The Ancient Egyptian Pyramid Texts (Oxford, 1969); Miriam
Lichtheim, Ancient Egyptian Literature, 3 vols. (Berkeley and Los Angeles,
1973-80); Antonio Loprieno (ed.), Ancient Egyptian Literature: History
and Forms (Leiden, 1996); Jaromir Marek, Egyptian Art (London, 1999);
Richard Parkinson, Voices from Ancient Egypt (London, 1991); William
Peck, Egyptian Drawings (New York, 1978); Georges Posener, Litterature et
politique dans I'Egypte de la XII dynastie (Paris, 1956); Gay Robins, The Art of
Ancient Egypt (London, 1997); Heinrich Schafer, Principles of Egyptian Art
(Oxford, 1978); Donald Spanel, Through Ancient Eyes: Egyptian Portraiture
(Birmingham, Ala., 1988), and William Stevenson Smith, The An and Archi-
tecture of Ancient Egypt, rev. W. Kelly Simpson (Harmondsworth, 1981).
Many articles in periodicals have tackled various aspects of Egyptian
religion and ideology, but only a few monographs have dealt with this
crucial area of Egyptian culture. Some earlier works still have a great deal
to offer, e.g. Henri Frankfort, Kingship and the Gods (Chicago, 1948), and
Siegfried Morenz, Egyptian Religion (London, 1973), but the best of the
works published over the last twenty years are: Jan Assmann, Egyptian
Solar Religion in the New Kingdom: Re, Amun and the Crisis of Polytheism
(London, 1995); Erik Hornung, Conceptions of God in Ancient Egypt (Ithaca,
NY, 1982), and Idea into Image: Essays on Ancient Egyptian Thought (New
York, 1992); Stephen Quirke, Ancient Egyptian Religion (London, 1992);
John Romer, Valley of the Kings (London, 1981); A. I. Sadek, Popular
440 FURTHER READING
Religion in Ancient Egypt during the New Kingdom (Hildesheim, 1988);
Byron E. Shafer (ed.), Religion in Ancient Egypt: Gods, Myths and Personal
Practice (London, 1991), and W. Kelly Simpson (ed.), Religion and Phil-
osophy in Ancient Egypt (New Haven, 1989).
There is no shortage of books on Egyptian funerary practices, but the
following are probably the most up-to-date and widely available recent pub-
lications: Sue D'Auria et al., Mummies and Magic: The Funerary Arts of
Ancient Egypt (Boston, 1988); I. E. S. Edwards, The Pyramids of Egypt
(Harmondsworth, 1985); Erik Hornung, The Valley of the Kings (New York
1990); M. Lehner, The Complete Pyramids (London, 1997); Nicholas Reeves,
The Valley of the Kings: The Decline of a Royal Necropolis (London, 1990);
Nicholas Reeves and Richard Wilkinson, The Complete Valley of the Kings
(London, 1996), and Jeffrey Spencer, Death in Ancient Egypt (Harmonds-
worth, 1982).
Apart from the books on funerary practices mentioned above, a sur-
prisingly small number of works deal with the overall development of
Egyptian architecture. Somers Clarke and Reginald Engelbach, Ancient
Egyptian Masonry: The Building Crajt (Oxford, 1930; repr. New York, 1990,
as Ancient Egyptian Construction and Architecture), is still useful but has
been very much replaced by Dieter Arnold, Building in Egypt: Pharaonic
Stone Masonry (Oxford, 1991). Alexander Badawy, A History of Egyptian
Architecture (Berkeley and Los Angeles, 1968), and Stevenson Smith, Art
and Architecture (see above under the topic of art), are both useful on the
more religious and aesthetic aspects of the topic, respectively.
Settlements, society, and material culture are discussed by the follow-
ing: Manfred Bietak, 'Urban archaeology and the "Town Problem" in
ancient Egypt', in K. Weeks (ed.), Egyptology and the Social Sciences (Cairo,
1979), 95-144; Morris Bierbrier, Tomb-Builders of the Pharaohs (London,
1982); Barry Kemp, 'The Early Development of Towns in Egypt', Antiquity,
51 (1977), 185-200, and Ancient Egypt: Anatomy of a Civilization (London,
1989); Dominique Valbelle, Les Ouvriers de la tombe: Deir el-Medineh a
I'epoque ramesside (Cairo, 1985), 261-317; Gay Robins, Women in Ancient
Egypt (London, 1993), and Paul Nicholson and Ian Shaw (eds.), Ancient
Egyptian Materials and Technology (Cambridge, 2000).
Major books and articles on ancient Nubia include Bruce Trigger,
History and Settlement in Lower Nubia (New Haven, 1965); Fred Wendorf
(ed.), The Prehistory of Nubia, 2 vols. (Dallas, Tex., 1968); William Adams,
Nubia: Corridor to Africa (London, 1977); Brigitte Gratien, Les Cultures
Kerma: Essai de classification (Lille, 1978); Steffen Wenig, Africa in Antiquity:
The Arts of Ancient Nubia and Sudan, 2 vols. (New York, 1978); Charles
Bonnet, Kerma, royaume de Nubie (Vienna, 1992), and 'Excavations at the
Nubian Royal Town of Kerma: 1975-91', Antiquity, 66 (1992) 6nff; Stuart
FURTHER R E A D I N G 441
Tyson Smith, Askut in Nubia (London, 1995); Peter Shinnie, Ancient Nubia
(London, 1996); Derek Welsby, The Kingdom of Kush: The Napatan and
Meroitic Empires (London, 1996), and Dieter Wildung (ed.), Sudan: Ancient
Kingdoms of the Nile (Paris, 1997), the latter being a well-illustrated exhibi-
tion catalogue.
i. Introduction
As far as the chronology of ancient Egypt is concerned, there are many
different sources, but three that provide very different perspectives on the
way in which the dating system has been constructed from an elaborate
combination of astronomical observations, king-lists, and genealogies are
Richard Parker, The Calendars of Ancient Egypt (Chicago, 1950); Kenneth
Kitchen, 'The Chronology of Ancient Egypt', WA 23 (1991), 201-8, and
Donald Redford, Pharaonic King-Lists, Annals and Day-Books: A Contribu-
tion to the Egyptian Sense of History (Mississauga, 1986). For the idea of the
study of cultural and social change during various periods, as opposed
to conventional political history, see David O'Connor, 'Political Systems
and Archaeological Data in Egypt: 2600-1780 BC', WA 6 (1974), 15-38;
Stephan Seidlmayer, 'Wirtschaftliche und gesellschaftliche Entwicklung
im Ubergang von alten zum mirtleren Reich', in J. Assmann and W. V.
Davies (eds.), Problems and Priorities in Egyptian Archaeology (London,
1987), 175-217; Barry Kemp, Ancient Egypt: Anatomy of a Civilization
(London, 1989); Kathryn Bard, 'Toward an Interpretation of the Role of
Ideology in the Evolution of Complex Society in Egypt', Journal of Anthro-
pological Archaeology, u (1992), 1-24, and Robert Wenke, 'Anthropology,
Egyptology and the Concept of Cultural Change', in J. Lustig (ed.), Anthro-
pology and Egyptology (Sheffield, 1997), 117-36.
For Manetho, see Manetho, Aegyptiaca, ed. and trans. W. G. Wadell
(London, 1940); for the Royal Turin Canon, see Alan Gardiner, The Royal
Canon of Turin (Oxford, 1959), and Jaromir Malek, 'The Original Version
of the Royal Canon of Turin', JEA 68 (1982), 93-106; and, for the Palermo
Stone, see Heinrich Schafer, Ein Bruchstuck altagyptischer Annalen (Berlin,
1902), and Georges Daressy, 'La Pierre de Palerme et la chronologic de
1'ancien empire', BIFAO12 (1916), 161-214.
The complex question of the links between Sothic heliacal risings and
Egyptian chronology is discussed in numerous books and articles, includ-
ing the following: Parker, 'Sothic Dates and Calendar "Adjustment"',
RdE 9 (1952), 101-8; Jacques Vandier, Manuel d'archeologie egyptienne, I
(Paris, 1952), 842-3; Jaroslav Cerny, 'Note on the Supposed Beginning of a
Sothic Period under Sethos I', JEA 47 (1961), 150-2; M. F. Ingham, 'The
Length of the Sothic Cycle', JEA 55 (1969), 36-40; Laszlo Kakosy, 'Die
442 FURTHER READING
Mannweibliche Natur des Sirius in Agypten', Studia Aegyptiaca, 2 (Buda-
pest, 1976), 41-6; G. Clerc, 'Isi-Sothis dans le monde remain', Hommages
a Maarten J. Vermaseren (Leiden, 1978), 247-81; Christiane Desroche-
Noblecourt, 'Isis Sothis—le chien, la vigne—et la tradition millenaire',
Livre du Centenaire, IFAO1880-1980 (Cairo, 1980), 15-24, and Rolf Krauss,
Sothis- und Mondaten: Studien zur astronomischen und technischen Chron-
ologie Altagyptens (Hildesheim, 1985).
For Flinders Petrie's system of seriation for the Predynastic, see his own
expositions of the technique: 'Sequences in prehistoric remains', ]AI, NS 29
(1899), 295-301, and Diospolis Parva (London, 1901), but, for more recent
approaches, see D. G. Kendall: 'A Statistical Approach to Flinders Petrie's
Sequence-Dating', Bulletin of the International Statistics Institute, 40 (1963),
657-80; Barry Kemp, 'Automatic Analysis of Predynastic Cemeteries: A
New Method for an Old Problem',/EA 68 (1982), 5-15, and Toby Wilkinson,
State Formation in Ancient Egypt: Chronology and Society (Oxford, 1996).
The ongoing debate concerning co-regencies can be explored by con-
sulting W. Kelly Simpson, 'Studies in the Twelfth Egyptian Dynasty: I-II',
JARCE 2 (1963), 53-63; William Murnane, Ancient Egyptian Coregencies
(Chicago, 1977), and David Lorton, 'Terms of Coregency in the Middle
Kingdom', VA2 (1986), 113-20.
A number of problems relating to the social and political history of
Egypt (many relating to the nature of the 'intermediate periods') are dis-
cussed in the following: Barbara Bell, 'The Dark Ages in Ancient History:
1. The First Dark Age in Egypt', A/A 75 (1971), 1-26, and 'Climate and the
History of Egypt: the Middle Kingdom', A/A 79 (1975), 223-69; Kenneth
Kitchen, 'The Basics of Egyptian Chronology in Relation to the Bronze
Age', in Paul Astrom (ed.), High, Middle or Low: Acts of an International
Colloquium in Absolute Chronology Held at the University of Gothenburg
20-22 August 1987 (Gothenburg, 1987), 37-55; P. James et al, Centuries of
Darkness: A Challenge to the Conventional Chronology of Old World Archae-
ology (London, 1991); Manfred Bietak (ed.), Agypten und Levante III: Acts of
the Second International Colloquium on Absolute Chronology (Vienna, 1992);
William Ward, "The Present Status of Egyptian Chronology', BASOR 288
(1992), 53-66, and Leo Depuydt, 'On the Consistency of the Wandering
Year as Backbone of Egyptian Chronology', JARCE 32 (1995), 43-58.
2. Prehistory
Full bibliographic references for this period, with topographic and
thematic indexes, as well as maps locating prehistoric sites, can be found
in Stan Hendrickx, Analytical Bibliography of the Prehistory and the Early
Dynastic Period of Egypt and Northern Sudan (Leuven, 1995), to which yearly
FURTHER READING 443
additions appear in the journal Archeo-Nil. Most of the recent synthetic
works are part of general works on Egypt's early history, such as Beatrix
Midant-Reynes, The Prehistory of Egypt: From the First Egyptians to the First
Pharaohs (Oxford, 2000), and Michael A. Hoffman, Egypt before the
Pharaohs (London, 1980), the revised edition of which (Austin, Tex., 1991)
includes an extra chapter summarizing recent discoveries.
An excellent recapitulation of the state of research is provided by Frank
Klees and Rudolph Kuper (eds.), New Light on the Northeast African Past
(Cologne, 1992), which includes contributions by leading specialists. The
work is particularly important with regard to the syntheses of the different
stages of the Palaeolithic Neolithic of the Western Desert.
The basic publications for the work of the Combined Prehistoric Expedi-
tion on the Nubian and Egyptian prehistory, are Wendorf (ed.), The Pre-
history of Nubia, 2 vols. (Dallas, Tex., 1968), and Wendorf and Schild (eds.),
The Prehistory of the Nile Valley (New York, 1976).
For the prehistory of the Western Desert, the first systematic study is
that on the Kharga Oasis, by Gertrude Caton-Thompson, The Kharga Oasis
in Prehistory (London, 1952). More recent studies related to particular
regions are those by Schild and Wendorf: for Dakhla, The Prehistory of
Dakhla Oasis and Adjacent Desert (Wroclaw, 1977), and for Bir Sahara, The
Prehistory of an Egyptian Oasis: A Report of the Combined Prehistoric Expedi-
tion to Bir Sahara, Western Desert, Egypt (Warsaw, 1981). The evidence
concerning the so-called radar channels is summarized in two articles in
JFA 14 (1987) and 15 (1988), by respectively William P. McHugh and
Wendorf and their associates.
The Middle Palaeolithic of Bir Tarfawi and Bir Sahara is presented at
large in Wendorf, Schild, Close, et al, Egypt during the Last Interglacial: The
Middle Paleolithic of Bir Tafawi and Bir Sahara East (New York, 1993). For
an extensive study on the Levallois technique, using mainly examples from
Egypt and Nubia, there is Philip Van Peer, The Levallois Reduction Strategy
(Madison, 1992).
The most recent overview of Middle and especially Late Palaeolithic
chert mining in Egypt is given by Vermeersch, Paulissen, and Van Peer, in
Archaeologia Polona, 33 (1995), while Vermeersch et al., 'Une miniere de
silex et un squelette du paleolithique superieure a Nazlet Khater',
L'Anthropologie, 88 (1984), 231-44, consists of a summary of the finds at
Nazlet Khater, including the Late Palaeolithic burial. The even older burial
from Taramsa Hill can be found in Vermeersch et al., 'A Middle
Palaeolithic Burial of a Modern Human at Taramsa Hill, Egypt', Antiquity
72 (1988), 475-84. Egypt's earliest rock art is presented by Dirk Huyge,
'Hilltops, Silts and Petroglyphs: The Fish Hunters of El-Hosh', Bulletin des
Musees royaux d'Art et d'Histoire 69 (1998), 1-17.
444 FURTHER READING
The Late Palaeolithic site E 71X12 near Esna is described in Wendorf,
Schild, Baker, Gautier, Longo, and Mohammed in A Late Paleolithic Kill-
Butchery Camp in Upper Egypt (Dallas, Tex., and Warsaw, 1997). The sites
from the same period at Wadi Kubbaniya are published as Wendorf,
Schild, and Close (eds.), The Prehistory of Wadi Kubbaniya I: The Kubbaniya
Skeleton (Dallas, Tex., 1986), and Wendorf, Schild, and Close (eds.), The
Prehistory of Wadi Kubbaniya II: Stratigraphy, Palaeoeconomy and Environ-
ment (Dallas, Tex., 1989), and Wendorf, Schild and Close (eds.), The Pre-
history of Wadi Kubbaniya III: Late Palaeolithic Archaeology (Dallas, Tex.,
1989). The same authors are also responsible for the excavation report on
the Early Neolithic at Bir Kiseiba in the Western Desert: Cattle- Keepers of
the Eastern Sahara: The Neolithic of Bir Kiseiba (Dallas, Tex., 1984). A first
summary of the megaliths found at Nabta Playa is given by Wendorf and
Schild in Sahara, 5 (1992-3).
The work by the Besiedlungsgeschichte der Ost- Sahara project is summar-
ized by its director Rudolf Kuper in CRIPEL 17 (1995). The full results of
the work in the Wadi el-Akhdar can be found in Werner Schon,
Ausgraburgen im Wadi el-Akhdar; GilfKebir (SW-Agypten) (Cologne, 1996).
A summary of the Neolithic in the Dakhla Oasis is provided in Mary M. A.
McDonald, 'Early African Pastoralism: View from Dakleh Oasis (South
Central Egypt)', Journal of Anthropological Archaeology 17 (1998), 124-42.
The Epipalaeolithic Elkabian was published as Vermeersch, Elkab II:
L'Elkabien, Epipaleolithique de la Vallee du Nil Egyptien (Leuven, 1978).
Interpretations of the Faiyum cultures are given in Fekri Hassan,
'Holocene Lakes and Prehistoric Settlements in the Western Fayum, Egypt',
JAS 13 (1986), 483-501, and Boleslaw Ginter and Janusz K. Kozlowski,
'Kulturelle und palaoklimatische Sequenz in der Fayum-Depression: Eine
zusammensetzende Darstellung der Forschungsarbeiten in den Jahren
1977-1981', MDAIK 42 (1986). Ginter, Kozlowski, and Barbara Drob-
niewicz published the evidence on the Tarifian in Silexindustrien von El
The different stages of cultural development at Merimda have been pub-
lished as Josef Eiwanger, Merimde — Benisaldme l-lll, 3 vols. (1984-92).
For the culture of el-Omari, see Fernand Debono and Bodil Mortensen, El-
Omari (Mainz am Rhein, 1990).
The three principal excavation reports for the Badarian are: Guy Brun-
ton and Caton-Thompson, The Badarian Civilisation and Prehistoric
Remains near Badari (London, 1928), and two volumes by Brunton, Mosta-
gedda and the Tasian Culture (London, 1937), and Matmar (London, 1948).
Among the studies dealing with various aspects of the Badari culture,
Renee F. Friedman, Predynastic Settlement Ceramics of Upper Egypt: A Com-
parative Study of the Ceramics of Hemamieh, Naqada and Hierakonpolis
FURTHER R E A D I N G 445
(1994), presents a most important study regarding the settlement ceram-
ics of both the Badari and Naqada culture. An equally important study for
the lithic industries has been made by Diane L. Holmes, The Predynastic
Lithic Industries of Upper Egypt: A Comparative Study of the Lithic Traditions
of Badari, Naqada and Hierakonpolis (Oxford, 1989). For Badarian social
stratification, see W. Anderson, 'Badarian Burials: Evidence of Social
Inequality in Middle Egypt during the Early Predynastic Era', JARCE 29
(1992), 51-66. Recent work on Badarian sites in the Badari area and at
Mahgar Dendera is presented respectively by Holmes and Friedman in
PPS 60 (1994), and Hendrickx and Midant-Reynes, 'Preliminary Report
on the Predynastic Living Site: Maghara 2 (Upper Egypt)', OLP19 (1988),
5-16. For an alternative view on the links between the Neolithic cultures in
the Delta and the Badarian culture, see the article by Werner Kaiser, 'Zur
Siidausdehnung des vorgeschichtlichen Deltakulturen und zur fruhen
Entwicklung Oberagyptens', MDAIK^i (1985), 61-88.
3. The Naqada Period
These suggestions for further reading are by no means exhaustive; Stan
Hendrickx, Analytical Bibliography of the Prehistory and the Early Dynastic
Period of Egypt and Northern Sudan (Leuven, 1995), gives some idea of the
wealth of published material concerning the Predynastic Period, and the
list below is intended to give only the fundamental works, which should
allow the reader to gain a deeper knowledge of the various topics.
Apart from the pioneering work of Flinders Petrie and James Quibell
(e.g. Petrie and Quibell, Naqada and Ballas (London, 1896), and Petrie,
Prehistoric Egypt (London, 1920)), there are more recent syntheses of the
Predynastic cultures by Lech Krzyzaniak, Early Farming Cultures on the
Lower Nile: The Predynastic Period in Egypt (Warsaw, 1977); Michael Hoff-
man, Egypt before the Pharaohs: The Prehistoric Foundations of Egyptian
Civilization (Austin, Tex., 1991), and Beatrix Midant-Reynes, The Prehistory
of Egypt: From the First Egyptians to the First Pharaohs (Oxford, 2000). Jean
Vercoutter, L'Egypte et la vallee du Nil, i: Des origines a la fin de I'ancien
empire (Paris, 1992), devotes over 200 pages to the question of the begin-
nings of Egyptian culture. Also recommended are the excellent articles by
Fekri Hassan, 'The Predynastic of Egypt', JWP 2 (1988), 135-86, and
Kathryn Bard, 'The Egyptian Predynastic: A Review of the Evidence',
JFA 21/3 (1994), 265-88.
For Predynastic chronology since Petrie, the crucial work undertaken by
Werner Kaiser (e.g. 'Zur inneren Chronologie der Naqada-Kultur', Archeo-
logia Geographica 6 (1957), 69-77) has been continued by Hendrickx in his
doctoral thesis, a clear summary of which was published as one of the
446 FURTHER READING
papers in Jeffrey Spencer (ed.), Aspects of Early Egypt (London, 1996). For
discussion of the radiocarbon dates from Predynastic sites, see Fekri
Hassan, 'Radiocarbon Chronology of Neolithic and Predynastic Sites in
Upper Egypt and the Delta', AAR 3 (1985), 95-16.
Comparatively little recent work has dealt with the Naqada culture, but
the available funerary data are thoroughly exploited in }. J. Castillos, A
Reappraisal of the Published Evidence on Egyptian Predynastic and Early Dyn-
astic Cemeteries (Toronto, 1982), and Kathryn Bard, From Farmers to
Pharaohs: Mortuary Evidence for the Rise of Complex Society in Egypt
(Sheffield, 1994). For the French excavations at el-Adaima, which have
begun to clarify many aspects of Naqada funerary practices, see the
preliminary reports published by Midant-Reynes et al. in BIFAO from
1990 onwards. The American work at Hierakonpolis is discussed in Hoff-
man, The Predynastic of Hierakonpolis: An Interim Monograph (Giza-
Macomb, 111., 1982), and Renee Friedman and Barbara Adams (eds.), The
Followers ofHorus: Studies Dedicated to Michael Hoffman (Oxford, 1992).
With regard to the northward expansion of the Naqada culture into the
Delta, see Alexander Scharff, Das vorgeschichtliche Graberfeld von Abusir el
Meleq (Leipzig, 1926), concerning the excavation of the cemetery of Abusir
el-Melek, and, for the recent German work at the eastern Delta site of
Minshat Abu Omar, see Karla Kroeper and Dietrich Wildung, Minshat Abu
Omar: Ein vor-und fruhgeschichtliche Friedhofim Nildelta, I, Graber 1-114
(Mainz, 1994), as well as Kroeper, Tombs of the fiite in Minshat Abu
Omar', in Edwin van den Brink (ed.), The Nile Delta in Transition: ^th-yd
Millennium BC, (Jerusalem, 1992), 127-50, which includes precious infor-
mation relating to the northward expansion of Upper Egyptian cultures.
With regard to the south, and contacts with the Nubian A Group, the
results of the Scandinavian expeditions are reported in H. Nordstrom,
Neolithic and A-Group Sites (Uppsala, 1972), and the American work has
been published by Bruce Williams, Excavations between Abu Simbel and
the Sudan Frontier: Part I: A Group Royal Cemetery at Qustul: Cemetery L
(Chicago, 1986). The latter includes important funerary remains at Qustul,
which Williams, 'The Lost Pharaohs of Nubia', Archaeology 33 (1980),
13-21, has claimed as proof of Nubian roots for the Egyptian pharaonic
civilization. William Adams's opposition to this suggestion, in 'Doubts
about Lost Pharaohs',JNES 44 (1985), 185-92, and Williams's subsequent
response ('The Forebears of Menes in Nubia: Myth or Reality?', JNES 46/1
(1987), 15-26) give some idea of the fierce debates that can sometimes rage
in the Egyptological world.
For the Maadi/Buto culture, see the German publications of I. Rizkana
and Jiirgen Seeher, Maadi, 3 vols. (Mainz, 1987-9), and the synthetic article
published by Seeher: 'Maadi—eine pradynastische Kulturgruppe zwis-
FURTHER R E A D I N G 447
chen Oberagypten und Palastina', Prahistorische Zeitschrift, 65/2 (1990),
123-56. In addition there is a description of current Italian research by
Isabela Caneva, 'Recent Excavations in Maadi', in L. Krzyzaniak and M.
Kobuciewicz (eds.), Late Prehistory of the Nile Basin and the Sahara
(Poznan, 1989). K. Kohler, The State of Research on Late Predynastic
Egypt: New Evidence for the Development of the Pharaonic State?', GM
147 (1995), 79-92, puts forward a modification to the usual view of the
two cultural groupings, Naqada and Maadi, which she suggests are too
schematic, arguing instead that there were a number of regional differ-
ences. See Diana Holmes, The Predynastic Lithic Industries of Upper Egypt:
A Comparative Study of the Lithic Traditions ofBadari, Nagada and Hiera-
konpolis (Oxford, 1989), for evidence of such regional distinctions in the
lithic industries of Upper Egypt. Kohler's hypothesis, however, is vehe-
mently contested by Werner Kaiser, Trial and Error', GM 149 (1995), 5-14.
Finally, it is worth stressing the importance of the proceedings of a col-
loquium held in Cairo: Edwin van den Brink (ed.), The Nile Delta in Trans-
ition: tfh-yd Millennium EC (Jerusalem, 1992), which provides a good
summary of current research concerning this part of the Nile Valley, which
was long considered hostile and uninhabited during the Predynastic
Period, including a number of papers dealing with early contacts between
the Delta and the Near East.
4. The Emergence of the Egyptian State
The most complete listing of bibliographical material for the Predynastic
and Early Dynastic periods can be found in Stan Hendrickx, Analytical
Bibliography of the Prehistory and the Early Dynastic Period of Egypt and
Northern Sudan (Leuven, 1995), which also has detailed maps of early sites
in Egypt, the Eastern and Western Deserts, Nubia, Sinai, and southern
Palestine. Although somewhat outdated in terms of tomb identifications,
Bryan Emery, Archaic Egypt (Harmondsworth, 1967), is an important con-
tribution to these studies, and he published several volumes on his exca-
vations at North Saqqara (Cairo, 1938, 1939, 1949, London, 1954, 1958).
More recent synthetic works on the subject include Jeffrey Spencer, Early
Egypt (London, 1993), Toby Wilkinson, Early Dynastic Egypt (London,
1999), and Michael Rice's somewhat controversial study, Egypt's Making
(London, 1991).
A major contribution concerning the intellectual foundations of the
early state in Egypt and early cults is found in Barry Kemp, Ancient Egypt:
Anatomy of a Civilization (London, 1989). The early state in Egypt is also
discussed in Jac. Janssen, "The Early Egyptian State', in J. M. Claessen and
P. Shalnik (eds.) The Early State (The Hague, 1978). The environmental
448 FURTHER  READING
parameters are discussed in Karl Butzer, Early Hydraulic Civilization in
Egypt (Chicago, 1976). For an overview of Predynastic cultures, see
Kathryn Bard, 'The Egyptian Predynastic: A Review of the Evidence/ JFA
21 (1994), 268-88. For a study of the late Predynastic and Early Dynastic
periods, including a list of cemeteries, see Bodil Mortensen, 'Change in
the Settlement Pattern and Population in the Beginning of the Historical
Period', Agypten und Levante, 2 (1991), 11-37.
The Followers of Horns (Oxford, 1992), is an excellent collection of
articles compiled by Renee Friedman and Barbara Adams in memory of
the late Michael Hoffman, who directed excavations at Hierakonpolis. See
especially the articles by David O'Connor on Early Dynastic temples,
Thomas von der Way on architecture at Buto, and Harry Smith on con-
nections between Egypt, Susa, and Sumer.
On Egyptian state formation, see Bruce Trigger, 'Egypt: A Fledgling
Nation', JSSEA 17 (1990), 58-66, and 'The Rise of Egyptian Civilisation',
in Trigger et al., Ancient Egypt: A Social History (Cambridge, 1983), 1-70.
See also Jiirgen Seeher, 'Gedanken zur Rolle Unteragyptens bei der
Herausbildung des Pharaonenreiches', MDAIK, 47 (1991), 313-18, and E.
Christiana Kohler, 'The State of Research on Late Predynastic Egypt: New
Evidence for the Development of the Pharaonic State?', GM 147 (1995),
79-92. For a theoretical perspective, see Robert Wenke, 'Egypt: Origins of
Complex Societies', Annual Review of Anthropology, 18 (1989), 129-55, an< ^
'The Evolution of Early Egyptian Civilization: Issues and Evidence', JWP
5/3 ( I 99 I )> 2 79~3 2 9- Werner Kaiser has written a number of important
articles on the subject in MDAIK (1958, 1985, 1990), and in ZAS (1956,
1959,1960,1961,1964). Another important reference is Wolfgang Helck,
Untersuchungen zur Thinitenzeit (Wiesbaden, 1987). The question of the
location of the Early Dynastic city of Memphis is discussed by David
Jeffreys and Anna Tavares in 'The Historic Landscape of Early Dynastic
Memphis', MDA/K5O (1994), 143-74.
The origins and initial form of writing in late Predynastic and Early
Dynastic Egypt are considered by John Ray in 'The Emergence of Writing
in Egypt', WA 17/3 (1986), 390-8, while the social implications of the use
of writing are discussed by John Baines in 'Literacy and Ancient Egyptian
Society', Man, 18 (1983), 572-99. Baines also provides a stimulating analy-
sis of the relationship between early Egyptian art and writing: 'Communi-
cation and Display: The Integration of Early Egyptian Art and Writing',
Antiquity, 63 (1989), 471-82. The social and economic context from which
hieroglyphs emerged is discussed in comparison with early Mesopo-
tamian, Chinese, and Mesoamerican languages by Nicholas Postgate, Tao
Wang, and Toby Wilkinson in 'The Evidence for Early Writing: Utilitarian
or Ceremonial?', Antiquity, 69 (1995), 459-80.
FURTHER READING 449
The ceremonial palettes and maceheads of the late Predynastic Period
have been the subject of numerous books and articles. Nicholas Millet
seeks to understand their decorative schemes and cultural significance by
comparing them with other items such as the labels attached to funerary
equipment (The Narmer Macehead and Related Objects', JARCE 27
(1990) 53-9), while others adopt more speculative historical and art-
historical approaches (e.g. W. A. Fairservis Jr., 'A Revised View of the
Na c rmr Palette', JARCE 28 (1991), 1-20, and W. Davis, Masking the Blow:
The Scene of Representation in Late Prehistoric Egyptian Art (Berkeley and
Los Angeles, 1992)).
For the early temple on Elephantine Island, see Kaiser's articles in
MDAIK, 32 (1976) and 33 (1977), and Kaiser et al. in MDAIK 51 (1995).
Giinter Dreyer, Elephantine VIII: Der Tempel der Satet (Mainz, 1986), is a
very thorough publication of these excavations. Excavations at Hierakon-
polis were reported for the Egyptian Research Account in J. E. Quibell and
Flinders Petrie, Hierakonpolis I (London, 1900), and Quibell and F. W.
Green, Hierakonpolis II (London, 1902). Barbara Adams has published
studies of the early excavations, Ancient Hierakonpolis (Warminster, 1974),
and a Supplement (Warminster, 1974), to this, as well as The Fort Cemetery
at Hierakonpolis (Warminster, 1987). Michael Hoffman published numer-
ous articles on his fieldwork there, and two important books, The Predyn-
astic of Hierakonpolis: An Interim Monograph (Cairo, 1982), and Egypt before
the Pharaohs, 2nd edn. (Austin, Tex., 1991). Three reports on excavations at
Hierakonpolis were published: Walter Fairservis, Occasional Papers in
Anthropology (Poughkeepsie, 1983,1983,1986).
Petrie published his Egypt Exploration Fund excavations at the royal
cemetery at Abydos in two volumes: The Royal Tombs of the First Dynasty
I-II (London, 1900-1). For a discussion of the debate concerning the
location of the royal and non-royal tombs at Abydos and Saqqara, see
Kemp's articles in JEA 52 (1966), and Antiquity, 41 (1967). For David
O'Connor's recent work at Abydos, see "The Earliest Pharaohs and the
University Museum', Expedition, 29/1 (1987), 27-39, an< ^ 'New Funerary
Enclosures (Talbezirke) of the Early Dynastic Period at Abydos',/ARC£26
(1989), 51-86; 'Boat Graves and Pyramid Origins: New Discoveries at
Abydos', Expedition, 33/3 (1991), 5-17, and 'The Earliest Royal Boat Graves',
Egyptian Archaeology 6 (1995), 3-7. For the recent excavations of the
German Archaeological Institute at Umm el-Qa'ab, see articles in MDAIK
by Kaiser (1981, 1987), Kaiser and Dreyer (1982), Kaiser and Grossman
(1979), and Dreyer (1987,1990,1991,1993).
An overview of Egyptian relations with Palestine is given in Donald
Redford, Egypt, Canaan, and Israel in Ancient Times (Princeton, 1992). For
more specific archaeological information, see Edwin van den Brink (ed.),
450 FURTHER READING
The Nile Delta in Transition: ^th-yd Millennium BC (Jerusalem, 1992),
including articles on recent excavations in the Delta at Mendes, Minshat
Abu Omar, Tell el-Fara e in/Buto, Tell el-Farkha, and Tell Ibrahim Awad.
Also in this volume are important articles by Israeli archaeologists on
Egyptian evidence in southern Palestine.
5. The Old Kingdom
No specialized history of the Old Kingdom has yet been written, therefore
it is necessary to resort to the relevant sections of the more general
histories of Egypt. W. Helck, Geschichte des Alien Agypten (Leiden, 1981),
may be beginning to show its age, but remains by far the best concise
history of ancient Egypt published in the past two or three decades. For an
update on some latest ideas, it is useful to consult more recent publica-
tions of the same type, such as Nicolas Grimal, A History of Ancient Egypt
(Oxford, 1992), and especially Jean Vercoutter, L'Egypte et la vallee du Nil, i:
Des origines a la fin de I'Ancien Empire (Paris, 1992). For more general back-
ground information, read Barry Kemp, Ancient Egypt: Anatomy of a Civili-
zation (London, 1989). Jaromir Malek's more popularly oriented In the
Shadow of the Pyramids (London, 1986) provides information on Old King-
dom economy, administration, religion, and arts and contains a spec-
tacular collection of photographs of Old Kingdom monuments by Werner
Forman. Dietrich Wildung, Die Rolle agyptischer Kdnige im Bewufetsein ihrer
Nachwelt (Berlin, 1969), gives a more unusual view of the history of the
Old Kingdom.
Aspects of Old Kingdom economy, administration, and foreign policy
have been discussed by W. Helck in a number of books and articles. The
originality of his contribution surpasses that of any other scholar, and at
least his Wirtschajtsgeschichte des Alten Agypten im 3. und 2. Jahrtausend vor
Chr. (Leiden, 1975), Untersuchungen zu den Beamtentiteln des agyptischen
Alten Reiches (Gliickstadt, 1954), and Die Beziehungen Agyptens zu Vordera-
sien im^. und 2. Jahrtausend v. Chr, 2nd edn. (Wiesbaden, 1971), must be
mentioned. K. Baer, Rank and Title in the Old Kingdom: The Structure of the
Egyptian Administration in the Fifth and Sixth Dynasties (Chicago, 1960), is
another classic that has lost little of its interest. Hans Goedicke introduced
many new ideas in his Konigliche Dokumente aus dem Alten Reich (Wies-
baden, 1967) and Die privaten Rechtsinschriften aus dem Alten Reich
(Vienna, 1970).
More recent specialized studies include Naguib Kanawati, The Egyptian
Administration in The Old Kingdom: Evidence on its Economic Decline
(Warminster, 1977), and Governmental Reforms in Old Kingdom Egypt
(Warminster, 1980), although they are not entirely uncontroversial. See
F U R T H E R R E A D I N G 451
also Nigel Strudwick, The Administration of Egypt in the Old Kingdom: The
Highest Titles and their Holders (London, 1985). E. Martin-Pardey, Unter-
suchungen zur agyptischen Provinzialverwaltung bis zum Ende des Alten
Reiches (Hildesheim, 1976), deals with a difficult but very important topic.
P. Posener-Krieger, Les Archives du temple funeraire de Neferirkare-Kakai
(Les Papyrus d'Abousir): Traduction et commentaire, 2 vols. (Cairo, 1976), is
of fundamental importance. R. M tiller-Wollermann, Krisenfaktoren im
agyptischen Staat des ausgehenden Alten Reichs (Tubingen, 1986), is an
extremely thought-provoking publication. K. Zibelius, Agyptische Sied-
lungen nach Texten des Alten Reiches (Wiesbaden, 1978), is a very useful
survey of the problem. B. L. Begelsbacher-Fischer, Untersuchungen zur
Gotterwelt des Alten Reiches im Spiegel der Privatgraber der IV. und V.
Dynastie (Freiburg, 1981), sets out the main parameters for the study of
Old Kingdom religion.
Old Kingdom monuments and art have been discussed, with varying
degrees of originality, in many publications. The following may be recom-
mended without reservation: I. E. S. Edwards, The Pyramids of Egypt, 5th
edn. (Harmondsworth, 1993); R. Stadelmann, Die Agyptischen Pyramiden:
vom Ziegelbau zum Weltwunder (Darmstadt, 1991), and W. Stevenson
Smith, A History of Egyptian Sculpture and Painting in the Old Kingdom, 2nd
edn. (Boston, 1949), but there are many others. Yvonne Harpur, Decora-
tion in Egyptian Tombs of the Old Kingdom: Studies in Orientation and Scene
Content (London, 1987), is noteworthy for its thoroughness, while Nadine
Cherpion, Mastabas et hypogees d'Ancien Empire: Le Probleme de la datation
(Brussels, 1989), presents an admirably audacious challenge to current
thinking. Problems that currently preoccupy art historians of the Old
Kingdom can be found in Kunst des Alten Reiches. Symposium in Deutschen
Archaologischen Instituts Kairo am 29, und 30. Oktober 1991 (Mainz, 1995),
N. Grimal (ed.), Les Criteres de datation stylistiques a I'Ancien Empire (Cairo,
1998) and the exhibition catalogue Egyptian Art in the Age of the Pyramids
(New York, 1999). For some recent developments in Old Kingdom archae-
ology, one may consult Miroslav Verner, Forgotten Pharaohs, Lost Pyramids:
Abusir (Prague, 1994).
Alessandro Roccati, La Litterature historique sous I'Ancien Empire egyptien
(Paris, 1982), is an excellent survey of textual sources. Elmar Edel's
contribution to our knowledge of Old Kingdom texts has been remark-
able—for example, in his Hieroglyphische Inschrijten des Alten Reiches
(Opladen, 1981). Raymond Faulkner's translation of The Ancient Egyptian
Pyramid Texts (Warminster, 1986) is a rich seam that will be mined for
many years to come. Henry Fischer's contribution to the study of the Old
Kingdom has been outstanding, and at least his Dendera in the Third
Millennium EC (Locust Valley, 1968) must be pointed out. These books will
452 FURTHER READING
provide references to primary publications of material and to articles in
specialized periodicals that cannot be listed here because of lack of space.
6. The First Intermediate Period
As a general survey of the political history of the First Intermediate Period
based mainly on contemporary epigraphic sources, W. C. Hayes, 'The
Middle Kingdom in Egypt: Internal History from the Rise of the Heracleo-
politans to the Death of Ammenemenes III', in I. E. S. Edwards et al. (eds.),
Cambridge Ancient History, 1.2 (Cambridge, 1971), 464-531, is still useful.
Translations of nearly all of the relevant texts are to be found in W.
SchenkePs invaluable volume, Memphis, Herakleopolis, Theben: Die epi-
graphischen Zeugnisse der j.-ii. Dynastie Agyptens (Wiesbaden, 1965). The
chronological problems are summarized in Stephan Seidlmayer, 'Zwei
Anmerkungen zur Dynastie der Herakleopoliten', GM157 (1997), 81-90,
while R. Muller-Wollermann provides a multi-faceted discussion of pos-
sible reasons for the collapse of the Old Kingdom in her Krisenfaktoren im
agyptischen Staat des ausgehenden Alten Reiches (Tubingen, 1986).
The archaeology of the First Intermediate Period is surveyed by Seidl-
mayer, Graberfelder aus dem Ubergang vom Alten zum Mittleren Reich, Stud-
ien zur Archaologie der Ersten Zwischenzeit (Heidelberg, 1990), and
'Wirtschaftliche und gesellschaftliche Entwicklung im Ubergang vom
Alten zum Mittleren Reich', in W. V. Davies et al. (eds.), Problems and
Priorities in Egyptian Archaeology (London, 1987), 175-218. Information on
decorated coffins and the early history of the Coffin Texts is to be found in
Harco Willems, Chests of Life (Leiden, 1988). The early history of button
seals and their implications for First Intermediate Period popular culture
and popular religion are studied in A. Wiese, Die Anfdnge der agyptischen
Stempelsiegel-Amulette (Freiburg, 1996). For the importance of Egyptian
popular traditions in general, see Barry Kemp, Ancient Egypt: Anatomy of a
Civilization (London, 1989), 64-107.
First Intermediate Period prosopography and the development of pro-
vincial administration in the individual regions of Upper Egypt are treated
in Henry Fischer's admirable works Dendera in the First Millennium BC
(Locust Valley, 1968) and Inscriptions from the Coptite Nome (Rome, 1964),
as well as in a series of excellent articles by Edward Brovarski, including
'Ahanakht of Bersheh and the Hare Nome in the First Intermediate
Period', in W. Kelly Simpson and W. Davies (eds.), Studies in Ancient Egypt,
the Aegean, and the Sudan (Festschrift D. Dunham) (Boston, 1981), 14-30;
'The Inscribed Material of the First Intermediate Period from Naga-ed-
Der', A/A 89 (1985), 581-4; 'Akhmim in the Old Kingdom and First Inter-
mediate Period', in P. Posener-Krieger (ed.), Melanges Gamal eddin
FURTHER R E A D I N G 453
Mokhtar (Cairo, 1985), 117-53; 'Abydos in the Old Kingdom and First
Intermediate Period: part i', in C. Berger et al. (eds.), Hommages a Jean
Leclant, i (Cairo, 1994), 99-121, and 'Abydos in the Old Kingdom and First
Intermediate Period: part 2', in D. Silverman (ed.), For his Ka, Essays
Offered in Memory of Klaus Baer (Chicago, 1994), 15-44- These questions
are further addressed in F. Gomaa, Agypten wahrend der Ersten Zwischen-
zeit (Wiesbaden, 1980), and Naguib Kanawati, Akhmim in the Old Kingdom
(Sydney, 1992).
The tomb of Ankhtifi at el-Mo e alla, with its famous inscription, was
published by Jacques Vandier in Mo'alla (Cairo, 1950). In Vandier, La
Famine dans I'Egypte ancienne (Cairo, 1936), the evidence for famines from
ancient Egyptian historical sources is assembled. Barbara Bell, 'The Dark
Ages in History I: The First Dark Age in Egypt', A/A 75 (1971), 1-26,
provides an interpretation of the reasons for the First Intermediate Period
referring to climatic change (although, from a historian's point of view, the
methodological basis of this exceedingly influential article seems
questionable). The development of irrigation during the First Intermedi-
ate Period is studied in W. Schenkel, Die Bewasserungsrevolution im Alien
Agypten (Mainz, 1978).
The royal tombs of the nth Dynasty were re-excavated and studied by
Dieter Arnold, Graber des Alien und Mittleren Reiches in el-Tdrif (Mainz,
1976). For the excavations at the Kom Dara, it is still necessary to refer to
R. Weill, Dara, Campagnes de 1946-1948 (Cairo, 1958).
For the small amount of evidence on Herakleopolitan dynastic history,
see J. von Beckerath, 'Die Dynastie der Herakleopoliten', ZAS 93 (1966),
13-20. The important inscriptions of the Asyut nomarchs were restudied
by Elmar Edel in Die Inschriften der Grabfronten der Siut Graber (Opladen,
1984) and by Detlef Franke in 'Zwischen Herakleopolis und Theben:
Neues zu den Grabern von Assjut', SAK14 (1987), 49-60. Recent studies
(including translations) of the literary works that have been dated to the
Herakleopolitan Period are Richard Parkinson, The Tale of the Eloquent
Peasant (Oxford, 1991), and J. F. Quack, Studien zur Lehre fur Merikare
(Wiesbaden, 1992).
The problem of using later literary texts as historical sources for the First
Intermediate Period was addressed by G. Bjorkman in 'Egyptology and
Historical Method', Orientalia Suecana, 13 (1964), 9-33, and by F. Junge in
'Die Welt der Klagen', in J. Assmann (ed.), Fragen an die altagyptische
Literatur, Gedenkschrijt E. Otto (Wiesbaden, 1977), 275-84. The lasting
impact that the experience of the First Intermediate Period had on Egyp-
tian thought was set out by J. Assmann in his outstanding book Agypten:
eine Sinngeschichte (Munich, 1996), 122-34.
454 FURTHER  READING
7. The Middle Kingdom Renaissance
To understand the major difficulties of Middle Kingdom chronology, a
start can be made with the 'standard chronology' given both by W. F.
Edgerton, 'Chronology of the Twelfth Dynasty', JNES i (1942), 307-14,
and by R. A. Parker, The Calendars of Ancient Egypt (Chicago, 111., 1950),
and 'The Sothic Dating of the Twelfth and Eighteenth Dynasties', in J. H.
Johnson and E. F. Wente (eds.), Studies in Honor of G. R. Hughes (Chicago,
111., 1976), 177-89). A clear account of the problems involved with any
chronology and with the co-regency theory is presented in W. Kelly
Simpson, 'Studies in the Twelfth Egyptian Dynasty: I-IF, JARCE2 (1963),
53-63. For 'revised chronologies' modifying this mainstream view, see, for
example, Rolf Krauss, Sothis und Munddaten (Hildesheim, 1985), and
Detlef Franke, 'Zur Chronologic des Mittleren Reiches I and II',
Orientalia, NS 57 (1988), 113-38.
In Die chronologische Fixierung des agyptischen Mittleren Reiches nach dem
Tempelarchiv von Illahun (Vienna, 1992), I. Luft has proposed fixed dates
for the Middle Kingdom based on the el-Lahun papyri. In Sesostris ler, etude
chronologique et historique du regne (Brussels, 1995), Claude Obsomer re-
examines the previous theories and the evidence (translations and line
drawings of texts) and offers a number of cogent reasons why the co-
regency theory should be questioned in regard to Amenemhat I, II, and
Senusret I. Josef Wegner, 'The Nature and Chronology of the Senusret
III-Amenemhat III Regnal Succession: Some Considerations Based on
New Evidence from the Mortuary Temple of Senusret III at Abydos', JNES
55 ( T 996)> 2 49~79» conveniently reviews the chronology debate (although
he virtually ignores the arguments mounted by Obsomer) and produces
new and vital evidence for the reign of Senusret III. In 'Amenemhat I and
the Early Twelfth Dynasty', MM] 26 (1991), 5-48, Dorothea Arnold has
strongly questioned the dating of tombs normally attributed to the period
of Mentuhotep III, her views setting the transfer of government to Lisht in
about year 20 of the reign of Amenemhat I.
Herbert Winlock, The Rise and Fall of the Middle Kingdom (New York,
1947), still deserves to be read, not least because Winlock did so much of
the original investigation for this period. Edouard Naville, The Xlth Dynasty
Temple at Deir el-Bahari, 3 vols. (London, 1907-13), is indispensable.
Dieter, Dorothea, and Felix Arnold's reinvestigations are more important
for insights into the meaning and purpose of the architecture and pottery
of this period: Mentuhotep Tempel des Konigs Mentuhotep von Deir el Bahari
(Mainz, 1974), Der Pyramidenbezirk des Konigs Amenemhet III (Mainz,
1987), and The South Cemeteries of Lisht, i: The Pyramid ofSenwosret I (New
York, 1988). Gyoro Voros's interesting discovery of the nth Dynasty
temple of Mentuhotep III, as well as his sed- festival building and his prob-
FURTHER READING 455
able tomb, on Thoth Hill at Luxor, is published in Temple of the Pyramid of
Thebes (Budapest, 1998).
A number of very detailed and thoughtful studies of individual reigns
have been published, beginning with Labib Habachi's study of Mentu-
hotep II: 'King Nebhepetre Mentuhotpe: His Monuments, Place in History,
Deification and Unusual Representations in the Form of Gods', MDAIK
19 (1963), 16-52. Winlock, The Slain Soldiers of Neb-hepet-Re Mentu-hotpe
(New York, 1945), looks at warfare in Mentuhotep II's time; Alan
Gardiner, 'The First King Mentuhotpe of the Eleventh Dynasty', MDAIK
14 (1956), 42-51, solves the problem of the various names of the king. For
the i2th Dynasty there is Ronald Leprohon, The Reign of Amenemhat I
(Toronto, 1980), and Robert Delia, A Study of the Reign ofSenusert III (New
York, 1980). Obsomer's quite splendid Sesostris ler (cited above), which,
among other things, puts forward a persuasive argument for discounting
the theory of co-regencies for the i2th Dynasty, has set new standards for
such works. For the 'Tod treasure', see Fernand Bisson de la Roque et al,
Le Tresor de Tod (Cairo, 1953). Dietrich Wildung, Sesostris und Amenemhet,
Agypten im Mittleren Reich (Freiburg, 1984; French trans., L'Age d'or) offers
a penetrating artistic analysis for the entire Middle Kingdom. Two chap-
ters of Gay Robins's The Art of Ancient Egypt (London, 1997), survey Middle
Kingdom art and architecture at a more general level, but offer more
insights than the earlier studies of Edward Terrace, Egyptian Paintings of
the Middle Kingdom (New York, 1967).
Essential reading on administration during the i3th Dynasty is Stephen
Quirke's pithy monograph The Administration of Egypt in the Late Middle
Kingdom (New Maiden, 1990), while his Middle Kingdom Studies (New
Maiden, 1991) includes some useful discussions concerning various pro-
cesses of change in both the the i2th and 1301 dynasties. Miriam Lich-
theim, Ancient Egyptian Autobiographies chiefly of the Middle Kingdom
(Freiburg, 1988), provides vital source material.
Torgny Save-Soderbergh, Aegypten und Nubien (Lund, 1941), is the
basic volume for Middle Kingdom activities in Nubia, supplemented by
Paul Smither, 'The Semnah Dispatches', JEA 31 (1945), 3-10, which pro-
vides fascinating insights into life in the Nubian forts. Bryan Emery, Lost
Land Emerging (New York, 1967), gives a popular account of the excavation
of Nubia in more recent times. Extremely helpful information on the
workings of the Egyptian administration in Nubia are provided by Stuart
Tyson Smith in various articles, as well as in Askut in Nubia (London,
1995)-
For information on life within the palace during the Middle Kingdom,
see Alexander Scharff s article on Papyrus Bulaq 18: 'Ein Rechnungsbuch
des koniglichen Hofes aus de 13. Dynastie (Papyrus Boulaq Nr. 18)', ZAS
456 FURTHER READING
56 (1920), 51-68, as well as Quirke's Administration volume (mentioned
above), and Manfred Bietak (ed.), Haus und Palast im Alien Agypten
(Vienna, 1996), which contains a wealth of information on Middle King-
dom houses and palaces (most articles in English). The second part of
Barry Kemp, Ancient Egypt: Anatomy of a Civilization (London, 1989), pro-
vides an informative, lively account of the organization and daily lives of
bureaucrats and townspeople who lived at this time.
The letters of Hekanakhte were translated by T. G. H. James in The
Hekanakhte Papers and other Early Middle Kingdom Documents (New York,
1962). Middle Kingdom literature has been made available in many texts,
such as Miriam Lichtheim, Ancient Egyptian Literature, i (Los Angeles,
1973), and Richard Parkinson, Voices from Ancient Egypt (London, 1991),
while the classic interpretation of links between Middle Kingdom litera-
ture and politics is Georges Posener, Litterature etpolitique dans I'Egypte de
la Xlle dynastie (Paris, 1956).
Numerous articles and monographs have been written on Egyptian reli-
gion, such as Stephen Quirke, Ancient Egyptian Religion (London, 1992),
but no single book has yet been written on the religion of the Middle
Kingdom in particular. In the meantime, Quirke's essay on Middle King-
dom religion in W. Forman and S. Quirke, Hieroglyphs and the Afterlife
(London, 1996), has gone some way towards addressing this deficiency.
Raymond Faulkner, The Ancient Egyptian Coffin Texts, 3 vols. (London,
1972-8), contains essential primary source material, while two books by
Harco Willems, Chests of Life (Leiden, 1988), and The Coffin of Heqata
(Groningen, 1994), discuss the evidence for religious beliefs and practices
of the Middle Kingdom.
8. The Second Intermediate Period
New evidence concerning the Second Intermediate Period is emerging so
quickly that many publications are out of date before they are printed, but a
few of them contain sufficient basic documentation to ensure that they will
remain important for a long time. Others are useful for the fresh eye they
cast upon well-worn facts. I have limited the Further Reading to these
categories.
I have used the Kamose texts as a guide through the maze of evidence
relating to the Second Intermediate Period. The best translation of them is
that given by H. S. Smith and A. Smith, 'A Reconsideration of the Kamose
Texts', ZAS 103 (1976), 48-76. The most wide-ranging discussion of the
period is provided by the contributors to Eliezer D. Oren (ed.), The Hyksos:
New Historical and Archaeological Perspectives (Philadelphia, 1997).
J. von Beckerath, Untersuchungen zur politischen Geschichte der zweiten
FURTHER READING 457
Zwischenzeit in Agypten (Gliickstadt, 1965), is still the best introduction to
chronological questions, but it needs to be updated with D. Franke, 'Zur
Chronologic des Mittleren Reiches Teil II: Die sogenannte "Zweite
Zwischenzeit" Altagyptens', Orientalia, 57 (1988), 245-74. Further discus-
sions can be picked up from his bibliography. A fresh and highly specu-
lative review of the written sources is given by D. Redford, Egypt, Canaan
and Israel in Ancient Times (Princeton, 1992). The most up-to-date and
comprehensive study of the contemporary written sources is Kim Ryholt,
The Political Situation in Egypt during the Second Intermediate Period
(Copenhagen, 1997). Ryholt's reconstruction of the Turin Canon papyrus,
the most important source for the period, has been followed in the writing
of Chapter 8 but his chronology and his integration of those kings who are
known only from scarabs into the Dynastic series has not been accepted.
For an important review of Ryholt, see Daphne Ben-Tor, Susan Allen, and
James Allen, 'Seals and Kings', BASOR 315 (1999), 47-74.
The evidence for Asiatics in Egypt during the Second Intermediate
Period is discussed by Georges Posener, 'Les Asiatiques en Egypte sous les
XII etXIII Dynasties', Syria, 34 (1957), 145-63; D. Arnold, F. Arnold, and
S. Allen, 'Canaanite Imports at Lisht, the Middle Kingdom Capital of
Egypt', Agypten und Levante, 5 (1994), 13-32, Kenneth Kitchen in 'Non-
Egyptians Recorded on Middle Kingdom Stelae in Rio de Janeiro', in
Stephen Quirke (ed.), Middle Kingdom Studies (New Maiden, 1991), 87-90,
Daphne Ben-Tor, The Historical Implications of Middle Kingdom
Scarabs found in Palestine bearing Private Names and Titles of Officials',
BASOR 294 (1994), 7-22; The Relations between Egypt and Palestine in
the Middle Kingdom as reflected by contemporary Canaanite Scarabs', IE]
47 (1997) 162-89; Rolf Krauss, 'An examination of Khyan's place in W. A.
Ward's seriation of royal Hyksos scarabs', Agypten und LevanteVll (1998),
39-42.
As far as the Delta-site of Avaris (Tell el-Dab e a) is concerned, everything
which M. Bietak writes is 'work in progress', so that every publication
contains new information. The most recent information will be found
in the journal Agypten und Levante edited by Bietak. Comprehensive
summaries of the Tell el-Dab e a finds are M. Bietak, Avaris: The Capital of
the Hyksos (London, 1996), and 'Egypt and Canaan during the Middle
Bronze Age', BASOR 281 (1991), 27-72. A more detailed report of the final
phase of Hyksos power is Perla Fuscaldo, Tell el-Daloa X. The Palace District
of Avaris. The Pottery of the Hyksos Period and the New Kingdom (Areas H/III
and H/VI). Part I: Locus 66 (Vienna, 2000). For different perspectives
on Tell el-Dab e a, see Oren (ed.), The Hyksos (cited above), and W. Vivian
Davies and Louise Schofield (eds.), Egypt, the Aegean and the Levant
(London, 1995); Patrick McGovern The Foreign Relations of the 'Hyksos'. A
458 FURTHER READING
neutron activation study of Middle Bronze Age pottery from the Eastern
Mediterranean (Oxford, 2000).
For the Delta in general, see Bietak, 'Zum Konigreich des '3-zh-R'
Nehesi', SAKu (1984), 59-75; Jean Yoyotte, 'Le Roi Mer-djefa-Re etle dieu
Sopdu: Un monument de la XIV Dynastie', BSFE114 (1989), 17-63, and
J. S. Holladay, Jr., Tell el-Maskhuta (Malibu, 1982), and Mohamed Abd El-
Maksoud, Tell Hebona (19^1-1991) Enquete archeologique sur la Deuxieme
Periode Intermediate et le Nouvel Empire a I'extremite orientale du Delta.
(Paris, 1998).
For the study of the Memphis region in the Second Intermediate Period,
see Dorothea Arnold, 'Keramikbearbeitung in Dahschur 1976-1981',
MDA/K38 (1982), 25-65; Dieter Arnold, The South Cemeteries ofLisht I:
The Pyramid of Senwosret I (New York, 1988); Janine Bourriau, 'Beyond
Avaris: The Second Intermediate Period in Egypt outside the Eastern
Delta', in Oren (ed.), The Hyksos (cited above); W. C. Hayes, 'Horemkhauef
of Nekhen and his trip to It-Towe', JEA 33 (1947), 3-11; and Quirke, 'Royal
Power in the i3th Dynasty', in Quirke (ed.), Middle Kingdom Studies (cited
above), 123-39.
On administrative titles, see Quirke, 'The Regular Titles of the Late
Middle Kingdom', RdEyj (1986), 107-30.
For a discussion of the boundary between the Asiatic and Egyptian Nile,
see J. Bourriau, 'Some Archaeological Notes on the Kamose Texts', in A
Leahy and J. Tait (eds.), Studies on Ancient Egypt in Honour ofH. S. Smith
(London, 1999), 43-48.
For Thebes, see Herbert Winlock, 'The Tombs of the Kings of the Seven-
teenth Dynasty at Thebes', JEA 10 (1924), 217-77 which is now comple-
mented by new evidence discussed by Michel Dewachter, 'Nouvelles
Informations relatives a 1'exploitation de la Necropole Royale de Drah
Aboul Neggah', #^£36 (1985), 43-66 and Daniel Polz 'The Ramsesnakht
Dynasty and the Fall of the New Kingdom. A New Monument in Thebes',
SAK25 (1998), 257-293; P. Vernus, 'La Stele du roi Sekhemsanktaowyre
Neferhotep lykernofret et la domination Hyksos', ASAE 68 (1982), 129-35,
and 'A propos de la stele du pharaon Mntw-htpi', RdE^i (1990), 22.
For funerary texts, see P. Vernus, 'Sur les graphics de la formule
"L'offrande que donne le roi" au Moyen Empire et a la Deuxieme Periode
Intermediate', in Quirke (ed.), Middle Kingdom Studies (cited above),
141-52, and Parkinson and Quirke, 'The Coffin of Prince Herunefer and
the Early History of the Book of the Dead', in A. B. Lloyd (ed.), Studies in
Pharaonic Religion and Society (London, 1992), 37-51.
For Sobekemsaf s expedition to the Wadi Hammamat, see Annie Gasse,
'Une expedition au Ouadi Hammamat sous le regne de Sebekemsaf I',
BIFAO 87 (1987), 207-18.
FURTHER R E A D I N G 459
With regard to the remains at Elephantine and the Second Cataract
Forts, see Detlef Franke, Das Heligtum des Heqaib auf Elephantine (Heidel-
berg, 1994); Cornelius von Pilgrim, Elephantine XVI11: Untersuchungen in
der Stadt des Mittleren Reiches und der Zweiten Zwischenzeit (Mainz, 1996);
Stuart Tyson Smith, Askut in Nubia (London, 1995), and Janine Bourriau,
'Relations between Egypt and Kerma during the Middle and New
Kingdoms', in W. V. Davies (ed.), Egypt and Africa: Nubia from Prehistory to
Islam (London, 1991), 129-44.
For the principal excavations relating to the Kingdom of Kush, see
Charles Bonnet, Kerma, Royaume de Nubie (Geneva, 1990) and Edifices et
rites funeraires a Kerma. (Paris, 2000).
For the history of the war against the Hyksos and subsequent reunifi-
cation of Egypt, see Claude Vandersleyen, Les Guerres d'Amosis, Fondateur
de la XVIII Dynastie (Brussels, 1971); Peter Lacovara, Deir el Ballas: Prelim-
inary Report on the Deir el Ballas Expedition 2980-1986 (Winona Lake,
1990); M. C. Wiener and James Allen, 'Separate Lives: the Ahmose Tem-
pest Stela and the Theran eruption', JNES 57/1 (1998), 1-28, and W. J.
Eastwood, N. J. Pearce, J. A. Westgate, and W. T. Perkins, 'Recognition of
Santorini (Minoan) Tephra in Lake Sediments from Golhisar Golii, South-
west Turkey by Laser Ablation ICP-MS', Journal of Archaeological Science,
25/7 (July 1998), 677-87.
9. The i8th Dynasty before the Amarna Period
For the i8th Dynasty generally, see the excellent volume by Claude Van-
dersleyen, L'Egypte et la vallee du Nil, ii: De la fin de I'ancient empire a la fin
du nouvel empire (Paris, 1995) [see p. 284 n. i for a detailed bibliography for
the temples of Hatshepsut and Thutmose III at Deir el-Bahri]. See also two
volumes by Donald Redford: History and Chronology of the Eighteenth
Dynasty, Seven Studies (Toronto, 1967), and Egypt, Canaan, and Israel in
Ancient Times (Princeton, 1992). On Karnak, see Jean-Claude Golvin and
Jean-Claude Goyon, Les Bdtisseurs de Karnak (London, 1987).
For the reign of Ahmose, see Manfred Bietak, Avaris: The Capital of the
Hyksos: Recent Excavations at Tell d-Daloa (London, 1996), and W. V.
Davies and Louise Schofield (eds.), Egypt, the Aegean, and the Levant. Inter-
connections in the Second Millennium BC (London, 1995); particularly
important are the stelae published by Claude Vandersleyen, 'Une tempete
sous le regne d'Amosis: Deux nouveaux fragments de la stele d'Amosis
relatant une tempete', RdEic) (1967), 123-59, RdE 20 (1968), 127-34. O n
the Donation Stela, see Michel Gitton, in, for example, Les Divines Epouses
de la i8e dynastie (Paris, 1984). For discussion of the confusion of royal
coffins and grave goods, see Marianne Eaton-Krauss, 'The Coffins of
460 FURTHER READING
Queen Ahhotep, Consort of Seqeni-en-Re and Mother of Ahmose', CdE 65
(1990), 195-205.
For the reigns of Amenhotep I, Thutmose I, and Thutmose II, see
Franz-Jiirgen Schmitz, Amenophis I (Hildesheim, 1978); Catherine
Graindorge and Philippe Martinez, 'Karnak avant Karnak: Les Construc-
tions d'Amenophis ler et les premieres liturgies amoniennes', BSFE 115
(1989) 36-64; James Romano, 'Observations on Early Eighteenth Dynasty
Royal Sculpture', JARCE 13 (1976), 97-111; Ingegerd Lindblad, Royal
Sculpture of the Early Eighteenth Dynasty in Egypt (Stockholm, 1984), and
Helen Jacquet-Gordon, Le Tresor de Thoutmosis ler: La Decoration, 2 vols.
(Cairo, 1988). Anthony Spalinger, Three Studies of Egyptian Feasts and their
Chronological Implications (Baltimore, 1992), includes discussion of the
jambs from the Third Pylon at Karnak bearing inscriptions concerning
Amenhotep I's religious festivals. For detailed discussion of the reign of
Thutmose II, see two articles by Luc Gabolde: 'La Chronologic du regne de
Thoutmosis II, ses consequences sur la datation des momies royales et
leurs repercutions sur 1'histoire du developpement de la Vallee des Rois',
SAK 14 (1987), 61-87, and 'La "Cour des fetes" de Thoutmosis II a
Karnak', Karnak, 9 (1993), 1-82.
For Hatshepsut and Thutmose III, see Peter Dorman, The Monuments of
Senenmut (London, 1988), and The Tombs of Senenmut: The Architecture
and Decoration of Tombs 71 and 353 (New York, 1991); Suzanne Ratie, La
Reine Hatchepsout (Leiden, 1979); Donald Redford, in LA vi (Wiesbaden,
1988), and Guido P. F. van den Boorn, The Duties of the Vizier: Civil
Administration in the Early New Kingdom (London, 1988). For discussion of
the implications of the texts from the tombs of User in terms of the royal
prerogatives assumed by members of the court of Hatshepsut and Thut-
mose III, see Erik Hornung, 'Die konigliche Dekorarion der Sargkam-
mer', in Eberhard Diziobek (ed.), Die Graber des Vezir User-Amun Theben
Nr. 61 und 131 (Mainz, 1994), 42-7.
For Amenhotep II and Thutmose IV, see Peter der Manuelian, Studies
in the Reign of Amenophis II (Hildesheim, 1987); Charles Van Siclen III,
Two Monuments from the Reign of Amenhotep II (San Antonio, 1982), The
Alabaster Shrine of King Amenhotep II (San Antonio, 1986), and 'The
Building History of the Tuthmosid temple at Amada and the Jubilees
of Tuthmosis IV, VA 3 (1987), 53-66; Hourig Sourouzian, 'A Bust of
Amenophis II at the Kimbell Art Museum', JARCE 28 (1991), 55-74; Betsy
M. Bryan, The Reign of Thutmose IV (Baltimore, 1991), and 'Portrait sculp-
ture of Thutmose IV, JARCE 24 (1987), 3-20, as well as Bernadette
Letellier, 'La Cour a peristyle de Thoutmosis IV a Karnak', BSFE 84 (1979),
33-49, and 'Thoutmosis IV a Karnak', BSFE 122 (1991), 36-52.
For Amenhotep III, see Eric Cline and David O'Connor (eds.),
FURTHER R E A D I N G 461
Amenhotep III: Perspectives on His Reign (Ann Arbor, 1998), with chapters
referred to above by David O'Connor and William J. Murnane; Arielle
Kozloff and Betsy M. Bryan, Egypt's Dazzling Sun: Amenhotep III and his
World (Cleveland, Oh., 1992); Lawrence Berman (ed.), The Art of Amenho-
tep III: Art Historical Analysis (Cleveland, 1990) [including W. Raymond
Johnson, 'Images of Amenhotep III in Thebes: Styles and Intentions'],
and 'The Deified Amenhotep III as the Living Re-Horakhty: Stylistic and
Iconographic Considerations', International Association of Egyptologists,
Congress 6: Atti, ii (Turin, 1993), 231-6; Claude Vandersleyen, 'Les Deux
Jeunesses d'Amenhotep III', BSFEui (1988), 9-30; Dorothea Arnold, The
Royal Women ofAmarna (New York, 1996); William L. Moran, TheAmarna
Letters (Baltimore, 1992); William Murnane, Ancient Egyptian Coregencies
(Chicago, 1977), and The Road to Kadesh: A Historical Interpretation of the
Battle Reliefs of King Sety I at Karnak, 2nd edn. (Chicago, 1990), and Mario
Liverani, Prestige and Interest: International Relations in the Near East, ca.
1600-1100 BC (Padua, 1990).
10. The Amarna Period and the Later New Kingdom
There is a vast literature on all aspects of the Amarna Period; over 2,000
titles are listed in Geoffrey Martin, A Bibliography of the Amarna Period and
its Aftermath: The Reigns ofAkhenaten, Smenkhkare, Tutankhamun and Ay
(c. 1350-1321 BC) (London, 1991). There is also a journal devoted entirely to
the period: Amarna Letters: Essays on Ancient Egypt ca. 1390-1310 BC, i (San
Francisco, 1991). Translations of all the relevant texts are now available in
William Murnane, Texts from the Amarna Period in Egypt (Atlanta, 1995).
Two classic studies are Cyril Aldred, Akhenaten, King of Egypt (London,
1988), and Donald Redford, Akhenaten: The Heretic King (Princeton, 1984);
both have been compared and contrasted in a highly informative review by
Marianne Eaton-Krauss, 'Akhenaten versus Akhenaten', BiOr 47 (1990),
541-59. A recent book on Akhenaten and his new religion is Erik Horn-
ung, Echnaton: Die Religion des Lichtes (Zurich, 1995). The city of el-
Amarna is brilliantly treated by Barry Kemp, Ancient Egypt: Anatomy of a
Civilization (London, 1989), 261-317.
For the political developments at the end of the dynasty, the pre-royal
career of Horemheb, and the role of Maya, see Jacobus van Dijk, The New
Kingdom Necropolis of Memphis: Historical and Iconographical Studies
(Groningen, 1993), 10-83. Egypt's foreign policy during the Amarna Period
and the early i9th Dynasty is admirably treated in William Murnane, The
Road to Kadesh: A Historical Interpretation of the Battle Reliefs of King Sety I at
Karnak (Chicago 1985; 2nd rev. edn., 1990). For the Memphite necropolis,
see Geoffrey Martin, The Hidden Tombs of Memphis: New Discoveries from the
462 FURTHER READING
Time of Tutankhamun and Ramesses the Great (London, 1991), and Jacobus
van Dijk, 'The Development of the Memphite Necropolis in the Post-Amarna
Period', in A.-P. Zivie (ed.), Memphis et ses necropoles au Nouvel Empire:
Nouvelles Donnees, nouvelles questions (Paris, 1988), 37-46. On the Delta resi-
dence of the Ramessid pharaohs, see Manfred Bietak, Avaris and Piramesse:
Archaeological Exploration in the Eastern Nile Delta. (London, 1979).
Studies on the dynastic history of the Ramessid Period include Murnane,
'The Kingship of the Nineteenth Dynasty: A Study in the Resilience of an
Institution', in David O'Connor and David Silverman (eds.), Ancient Egyptian
Kingship (Leiden, 1995), 185-217; Kenneth Kitchen, Pharaoh Triumphant:
The Life and Times of Ramesses II, King of Egypt, 3rd edn. (Warminster, 1985);
Labib Habachi, Features of the Deification of Ramesses II (Gliickstadt, 1969);
Hourig Sourouzian, Les Monuments du roi Merenptah (Mainz, 1989);
Rosemarie Drenkhahn, Die Elephantine-Stele des Sethnacht und ihr historischer
Hintergrund (Wiesbaden, 1980); Pierre Grandet, Ramses HI: Histoire d'un
regne (Paris, 1993); A. J. Peden, The Reign of Ramesses IV (Warminster, 1994),
and Kitchen, 'Ramesses VII and the Twentieth Dynasty', JEA 58 (1972),
182-94. For the role played by the Libyans in Later New Kingdom Egypt, see
the essays collected in M. Anthony Leahy (ed.), Libya and Egypt ^00-750 JBC
(London, 1990). A new and highly attractive reconstruction of the events at
the end of the Twentieth Dynasty is given by Karl Jansen-Winkeln, 'Das Ende
des Neuen Reiches', ZAS 119 (1992), 22-37, an< ^ 'Die Pllinderung der
Konigsgraber des Neuen Reiches', ZAS 122 (1995), 62-78.
For the economic history of the New Kingdom, see the fundamental
studies by Jac Janssen, 'Prolegomena to the Study of Egypt's Economic
History during the New Kingdom', SAK$ (1975), 127-85, and Commodity
Prices from the Ramessid Period (Leiden, 1975); see also Kemp, Ancient
Egypt: Anatomy of a Civilization (cited above), 232-60. Three recent studies
by Jan Assmann on the social and religious history of the New Kingdom
are Egyptian Solar Religion in the New Kingdom: Re, Amun and the Crisis of
Polytheism, trans. A. Alcock (London, 1995), Agypten: Theologie und From-
migkeit einerfruhen Hochkultur (Stuttgart, 1984), 221-85, and ^Sgypten: Eine
Sinngeschichte (Munich, 1996), 223-315; see also P. Vernus, 'La Grande
Mutation ideologique du Nouvel Empire: Une nouvelle theorie du pouvoir
politique: Du demiurge face a sa creation', BSEG 19 (1995), 69-95; A. M.
Gnirs, Militar und Gesellschaft. Ein Beitrag zur Sozialgeschichte des Neuen
Reiches (Heidelberg, 1996), and M. Romer, Gottes- und Priesterherrschaft in
Agypten am Ende des Neuen Reiches: Ein religionsgeschichtliches Phanomen
und seine sozialen Grundlagen (Wiesbaden, 1994).
FURTHER R E A D I N G 463
ii. Egypt and the Outside World
For general discussions of Egyptian contacts with the outside world, see
Dominique Valbelle, Les Neuf Arcs (Paris, 1990); Donald Redford, Egypt,
Canaan, and Israel in Ancient Times (Princeton, 1992); Edda Bresciani,
'Foreigners', in S. Donadoni (ed.), The Egyptians (Chicago, 1997), and
E. Uphill, 'The Nine Bows', Jaarbericht van het Vooraziatische-Egyptisch
Genootschap Ex Oriente Lux, 19 (1965-6), 393-420.
With regard to textual and visual sources for Egyptian racial caricatures
and ethnic designations, see J. Osing, 'Achtungstexte aus dem Alten Reich',
MDAIK 32 (1976), 133-85; Georges Posener, 'Achtungstexte', in LA i
(Wiesbaden, 1975), 67-9; G. Posener, Cinq figures d'envoutement (Cairo,
1987), and Martin Bernal, Black Athena: The Afro-Asiatic Roots of Classical
Civilization, 2 vols. (London, 1987-91).
For views of the problem of ethnicity with regard to ancient Egyptians and
their neighbours, see John Baines, 'Contextualizing Egyptian Representa-
tions of Society and Ethnicity', in J. S. Cooper and G. M. Schwartz (eds.), The
Study of the Ancient Near East in the Twenty-First Century: The William Foxwell
Albright Centennial Conference (Winona Lake, 1996), 339-84; Henry Fischer,
'Varia Aegyptiaca', JARCE 2 (1963), 17-51; Anthony Leahy, 'Ethnic Diversity
in Ancient Egypt', in J. M. Sasson (ed.), Civilizations of the Ancient Near East
(New York, 1995), 225-34, and F. M. Snowden, Jr., 'Ancient Views of Nubia
and the Nubians', Expedition, 35 (1993), 40-50 [and for Ptolemaic ethnicity,
see the further reading for Chapter 14].
For the history of Egyptian contacts with Nubia, see W. B. Emery, Egypt in
Nubia (London, 1965); Bruce Trigger, Nubia under the Pharaohs (London,
1976); William Adams, Nubia: Corridor to Africa, 2nd edn. (London, 1984);
W. Vivian Davies (ed.), Egypt and Africa: Nubia from Prehistory to Islam
(London, 1991); David O'Connor, Ancient Nubia: Egypt's Rival in Africa
(Philadelphia, 1993); T. Celenko (ed.), Egypt in Africa (Indianopolis, 1996),
and (especially from a visual point of view) Dieter Wildung (ed.), Sudan:
Ancient Kingdoms of the Nile (Paris, 1997).
For the people and cultures of pre-Greek Libya, see W. Holscher, Libyer
und Agypten (Gluckstadt, 1937); Kenneth Kitchen, The Third Intermediate
Period in Egypt (Warminster, 1986), 287-361; M. A. Leahy, "The Libyan
Period in Egypt: An Essay in Interpretation', Libyan Studies, 16 (1985),
51-65, and (ed.) Libya and Egypt, c.1300-750 BC (London, 1990), and
Anthony Spalinger, 'Some Notes on the Libyans of the Old Kingdom and
Later Historical Reflexes', JSSEA 9 (1979), 125-60.
For the much-debated question of the location of the kingdom of Punt,
the means by which Egyptians travelled there, and the products that they
were seeking, see Louise Bradbury, 'Kpn-Boats, Punt Trade and a Lost
Emporium', JARCE 33 (1996); David Dixon, 'The Transplantation of Punt
464 FURTHER READING
Incense Trees in Egypt', JEA 55 (1969), 55-65; R. Fattovich, 'The Problem of
Punt in the Light of Recent Fieldwork in the Eastern Sudan', in S. Schoske
(ed.), Akten Munchen 1985, iv (Hamburg, 1991), 257-72; R. Herzog, Fount
(Gllickstadt, 1968); Kitchen, 'The Land of Punt', in Thurstan Shaw et al.
(eds.), The Archaeology of Africa: Food, Metals and Towns (London, 1993),
587-608, 'Further thoughts on Punt and its neighbours', in A. Leahy and J.
Tait (eds.), Studies in Ancient Egypt in Honour of H. S. Smith (London, 1999)
173-8, and William Stevenson Smith, 'The Land of Punt', JARCE i (1962),
59-60.
For Egyptian social, political, and economic involvement in Syria-
Palestine, Turkey, and Mesopotamia, see Raphael Giveon, The Impact of
Egypt on Canaan (Gottingen, 1978); W. Helck, Die Beziehungen Agyptens zu
Vorderasien im3. und2.Jahrtausendv. Chr. (Wiesbaden, 1962); Barry Kemp,
'Imperialism and Empire in New Kingdom Egypt', in P. Garnsey and C. R.
Whittaker (eds.), Imperialism in the Ancient World (Cambridge, 1978),
7-58; Jean Leclant, Les Relations entre I'Egypte et la Phenicie du voyage
d'Ounamon a I'expedition d'Alexandre (Beirut, 1968), and William Ward,
Egypt and the East Mediterranean World (Beirut, 1971).
For the contacts between Egyptians and the inhabitants of the north
Mediterranean islands (and the Greek mainland), see Jean Vercoutter,
L'Egypte et le monde egeenprehellenique (Cairo, 1956); John Barns, Egyptians
and Greeks (Oxford, 1966); John Boardman, The Greeks Overseas (Har-
mondsworth, 1964); H.-J. Thissen, 'Griechen in Agypten', in LA iii (Wies-
baden, 1977), 898-903; Barry Kemp and Robert Merrilees, Minoan Pottery
from Second Millennium Egypt (Mainz, 1981); Naphthali Lewis, Greeks in
Ptolemaic Egypt (Oxford, 1986), and W. Vivian Davies (ed.), Egypt, the
Aegean and the Levant (London, 1995).
For the Sea Peoples, see T. and M. Dothan, People of the Sea: The Search
for the Philistines (New York, 1992); Redford, Egypt, Canaan and Israel (cited
above), 285-394, and Nancy Sandars, The Sea Peoples (New York, 1985).
12. The Third Intermediate Period
For the period as a whole the basic source is still Kenneth Kitchen, The
Third Intermediate Period in Egypt (1100-650 BC) (Warminster, 1973; 2nd
edn. with Supplement, 1986; 3rd edn., with new preface, 1995). The Sup-
plement and 1995 preface summarize and criticize studies on chronology
and political geography published since 1973. Textual sources for the
period are scattered, and up-to-date studies of many key texts are still
awaited. Translations of some basic texts are provided by Miriam Lich-
theim, Ancient Egyptian Literature, iii (Berkeley and Los Angeles, 1980).
For a brief but well-chosen selection of texts, see Pascal Vernus, 'Choix
FURTHER READING 465
de textes illustrant le temps des rois tanites et libyens', in J. Yoyotte (ed.)
Tunis, I'ordes Pharaons (Paris, 1987), 102-111.
Genealogies and prosopography of official families are discussed in M.
L. Bierbrier, The Late New Kingdom in Egypt (c. 1300-664 BC) (Warminster,
1975); P.-M. Chevereau, Prosopographie des cadres militaires egyptiens de la
Basse Epoque (Antony, 1985), and G. Vittmann, Priester und Beamte im
Theben der Spatzeit (Vienna, 1978).
Papers discussing the royal and official families and the chronology
of the period are numerous; among the most significant modifications to
the structure outlined by Kitchen are (i) a reappraisal of events at the
beginning of the Third Intermediate Period: K. Jansen-Winkeln, 'Das
Ende des Neuen Reiches', ZAS 119 (1992), 22-37; 'Die Pliinderung der
Konigsgraber des Neuen Reiches', ZAS 122 (1995), 62-78; (2) the his-
torical status and sphere of influence of the 23rd Dynasty: D. A. Aston,
'Takeloth II—A King of the "Theban Twenty-Third Dynasty"?', JEA 75
(1989), 139-53; M. A. Leahy, 'Abydos in the Libyan Period', in M. A. Leahy
(ed.), Libya and Egypt, c.1300-750 BC (London, 1990), 155-200. Revi-
sionist theories by various authors, proposing major contraction of the
chronology of the Third Intermediate Period, have not been generally
accepted.
On ideological aspects of kingship in the Third Intermediate Period, see
M.-A. Bonheme, Les Noms Royaux dans I'Egypte de la Troisieme Periode
Intermediate (Cairo, 1987). For a well-documented study of the society,
administration, and culture of the period, see David O'Connor, 'New
Kingdom and Third Intermediate Period, 1552-664 BC', in B. Trigger et al.,
Ancient Egypt: A Social History (Cambridge, 1983), 183-278. For a general
summary of the period with emphasis on Tanis, see J. Yoyotte (ed.), Tanis,
I'or despharaons (Paris, 1987).
The chronology, culture, and society of the Libyan Period are discussed
in M. A. Leahy (ed.), Libya and Egypt, 01300-750 BC (London, 1990), while
the Libyan character of the 2ist Dynasty is covered in K. Jansen-Winkeln,
'Der Beginn der Libyschen Herrschaft in Agypten', Biblische Notizen, 71
(1994), 78-97. For the impact of Libyan immigration on Egyptian culture
and society, see M. A. Leahy, 'The Libyan Period in Egypt: An Essay in
Interpretation', Libyan Studies, 16 (1985), 51-65. For Libyan Period his-
torical and biographical inscriptions, see R. A. Caminos, The Chronicle of
Prince Osorkon (Rome, 1958), and K. Jansen-Winkeln, Agyptische Bio-
graphien der 22. und 23. Dynastie (Wiesbaden, 1985). For discussion of the
Delta principalities, see J. Yoyotte, 'Les Principautes du Delta au temps de
1'anarchie libyenne (etudes d'histoire politique)', MIFAO 66 (1961), 121-81,
pis. I-III, and F. Gomaa, Die libyschen Furstentumer des Deltas (Wiesbaden,
1974). For donation stelae, see D. Meeks, 'Les Donations aux temples dans
466 FURTHER READING
TEgypte du ler millenaire avant J.-C.', in E. Lipinski (ed.), State and Temple
Economy in the Ancient Near East, ii (Leuven, 1979), 605-87.
The Kushite Period (25th Dynasty) is dealt with in Laszlo Torok, The
Birth of an Ancient African Kingdom: Kush and her Myth of the State in the
First Millennium EC (Lille, 1995), and Jean Leclant, 'KuschitenherrschafV,
in LA iii (Wiesbaden, 1980), 893-901. For the campaign of King Piye, see
N. Grimal, Le Stele triomphale de Pi-(ankh)y au Musee du Caire (Cairo,
1981), and E. R. Russmann, The Representation of the King in the XXVth
Dynasty (Brussels, 1974).
A number of articles and monographs discuss the religion and material
culture of the Third Intermediate Period. H. Kees, Die Hohenpriester des
Amun von Karnak von Herihor bis zum Ende des Athiopienzeit (Leiden,
1964), still contains useful material though now superseded on specific
issues of identification and genealogy of the priests of Amun during the
Third Intermediate Period; see also J.-M. Kruchten, Les Annales despretres
de Karnak (XXI-XXIIIemes dynasties) et autres textes contemporains relatifs a
I'initiation despretres d'Amon (Leuven, 1989). For oracles, see J.-M. Kruch-
ten, Le Grand Texte oraculaire de Djehoutymose, intendant du Domaine
d'Amon sous le pontificat de Pinedjem II (Brussels, 1986); for the role of
women in temple cult, see S.-A. Naguib, Le Clerge Feminin d'Amon thebain
a la 2i e Dynastie (Leuven, 1990); for the God's Wife of Amun, see E. Graefe,
Untersuchungen zur Verwaltung und Geschichte der Institution der Gottes-
gemahlin des Amun vom Beginn des Neuen Reiches bis zur Spatzeit (Wies-
baden, 1981); for religious iconography, see Richard Fazzini, Egypt:
Dynasty XXII-XXV (Iconography of Religions xvi(io); Leiden, 1988).
Burial customs are described in Pierre Montet, La Necropole royale de
Tanis, i-iii (Paris, 1947-60), and coffins are discussed in A. Niwinski, 2ist
Dynasty Coffins from Thebes: Chronological and Typological Studies (Mainz
am Rhein, 1988), and R. van Walsem, The Coffin of Djedmonthuiufankh
in the National Museum of Antiquities at Leiden (Leiden, 1997). Funerary
papyri are covered in A. Niwinski, Studies on the Illustrated Theban Funerary
Papyri of the nth and loth Centuries BC (Freiburg, 1989).
The sculpture of the Third Intermediate Period is discussed in Karol
Mysliwiec, Royal Portraiture of the Dynasties XXI-XXX (Mainz am Rhein,
1988). Pottery is described by David Aston, Egyptian Pottery of the Late New
Kingdom and Third Intermediate Period (Twelfth-Seventh Centuries BC)
(SAGA 13) (Heidelberg, 1996); faience figurines are described by J. Bulte,
Talismans Egyptiens d'heureuse maternite (Paris, 1991); and metalworking is
discussed by Christiane Ziegler, 'Les Arts du metal a la Troisieme Periode
Intermediate', in J. Yoyotte (ed.), Tanis, I'or des pharaons (Paris, 1987),
85-101; R. S. Bianchi, 'Egyptian Metal Statuary of the Third Intermediate
Period (Circa 1070-656 BC), from its Egyptian Antecedents to its Samian
FURTHER READING 467
Examples', in M. True and J. Podany (eds.), Small Bronze Sculpture from the
Ancient World (Malibu, 1990), 61-84.
Theban tombs of the later Third Intermediate Period are described in
D. Eigner, Die Monumentalen Grabbauten der Spatzeit in der Thebanischen
Nekropole (Vienna, 1984).
13. The Late Period
Studies of the Late Period as a whole feature in all general Egyptian
histories, such as Emile Drioton and Jacques Vandier, L'Egypte, 5th edn.
(Paris, 1975); Alan Gardiner, Egypt of the Pharaohs (Oxford, 1961), Bruce
Trigger et al, Ancient Egypt: A Social History (Cambridge, 1983), and
Nicolas Grimal, A History of Ancient Egypt (Oxford, 1992), but the best
book dedicated to this period is F. K. Kienitz, Die politische Geschichte
Agyptens vom 7. bis zum ^.Jahrhundert vor der Zeitwende (Berlin, 1953).
For the Saite Period in particular, consult Kenneth Kitchen, The Third
Intermediate Period in Egypt (1200-650 BC) (Warminster, 1973); T. G. H.
James, 'Egypt: The Twenty-Fifth and Twenty-Sixth Dynasties', in J. Board-
man et al (eds.), The Cambridge Ancient History, iii(2), 2nd edn. (Cambridge,
1991), 677-750, and Anthony Leahy, "The Earliest Dated Monument of
Amasis and the End of the Reign of Amasis', JEA 74 (1988), 183-99.
For the First Persian Occupation, see Georges Posener, La Premiere
Domination perse en Egypte (Cairo, 1936); Edda Bresciani, 'La satrapia
d'Egitto', Studi classici e orientali, 7 (1958), 153-87, and John Ray, 'Egypt:
525-404 BC', in Boardman et al. (eds.), The Cambridge Ancient History, iv,
2nd edn. (Cambridge, 1988), 254-86. For the independence period and
the Second Persian Occupation, see Alan Lloyd in Boardman et al. (eds.),
The Cambridge Ancient History, vi (Cambridge, 1994), 337ff
For discussions of kingship in the 26th~3oth dynasties, see Eberhard
Otto, Die biographischen Inschrijten der agyptischen Spatzeit (Probleme der
Agyptologie 2, Leiden, 1954); Peter Kaplony, 'Bemerkungen zum agyp-
tischen Konigtum, vor allem in der Spatzeit', CdE^6 (1971), 250-74; Janet
Johnson, 'The Demotic Chronicle as an Historical Source', Enchoria 4
(1974), 1-17, and Leahy, 'Royal Iconography and Dynastic Change,
750-525 BC: The Blue and Cap Crowns', JEA 78 (1992), 223-40.
With regard to social structure during the Late Period, see E. Meyer,
'Gottestaat, Militarherrschaft und Standewesen in Agypten', Sitzungs-
berichte der preussischen Akademie der Wissenschaften und Philosophische-
historiche Klasse, 28 (1928), 495-532, and Alan Lloyd, in Trigger et al,
Ancient Egypt (cited above), 299-301. For discussion of the economy and
systems of administration, see Lloyd, in Trigger et al, Ancient Egypt (cited
above), 325-37. The machimoi and their Libyan ancestors are discussed by
468 FURTHER READING
Kitchen, The Third Intermediate Period (cited above), F. Gomaa, Die liby-
schen Furstentumer des Deltas vom Tod Osorkons II. bis zur Wiedervereinigung
Agyptens dur Psametik I (Wiesbaden, 1974), and Lloyd, Herodotus Book II, 3.
i84ff. For commercial and military contacts between the Greeks and the
Egyptians during the Late Period, see John Boardman, The Greeks Overseas
(Harmondsworth, 1964); Whitney Davis, 'The Cypriotes at Naukratis',
GM 41 (1980), 7-19; William Coulson and A. Leonard, Jr., Cities of the
Delta, i. Naukratis: Preliminary Report on the 1977-1978 and 1980 Seasons
(Malibu, 1981); Lloyd, Triremes and the Saite Navy', JEA 58 (1972),
268-79, an d Boardman, 'Settlement for Trade and Land in North Africa:
Problems of Identity', in G. R. Tzetskhladze and F. Angelis (eds.), The
Archaeology of Greek Colonization: Essays Dedicated to Sir John Boardman
(Oxford, 1984), 137-49.
For discussion of the priesthood during the Late Period, see Hermann
Kees, Das Priestertum im agyptischen Staat (Leiden, 1953); Serge Sauneron,
Les Pretres de I'ancienne Egypte (Bourges, 1957), and G. Vittmann, Priester und
Beamte im Theben der Spatzeit (Vienna, 1978). For religion in general during
this period, see Otto, Die biographischen Inschriften (cited above), passim.
For Late Period art, see Bernard Bothmer et al., Egyptian Sculpture of
the Late Period 700 EC to AD 100 (New York, 1960); Richard Fazzini, Images
for Eternity: Egyptian Art from Berkeley and Brooklyn (New York, 1975); Cyril
Aldred, Egyptian Art in the Days of the Pharaohs (London, 1980), chs. 16-17,
and William Stevenson Smith, The Art and Architecture of Ancient Egypt,
rev. W. Kelly Simpson (Harmondsworth, 1981), ch. 21. For detailed study
of archaism in Late Period art and literature, see H. Brunner, 'Archaismus',
in LA i (Wiesbaden, 1975), 386-95; John Cooney, 'Three Early Saite
Tomb Reliefs', JNES 9 (1950), 193-203, and Peter Der Manuelian, Living in
the Past: Studies in Archaism of the Egyptian Twenty-Sixth Dynasty (London,
1994).
A comprehensive study of Late Period mortuary archaeology is lacking,
but relevant material appears in Jeffrey Spencer, Death in Ancient Egypt
(Harmondsworth, 1982). See also David Aston, 'Dynasty 26, Dynasty 30,
or Dynasty 27? In search of the Funerary Archaeology of the Persian
Period', in M. A. Leahy and W. J. Tait (eds.), Studies on Ancient Egypt in
Honour of H. S. Smith (London, 1999).
14. The Ptolemaic Period
Useful general studies of the period are C. Preaux, Les Grecs en Egypte
d'apres les archives de Zenon (Brussels, 1947); W. Tarn and G. T. Griffith,
Hellenistic Civilisation, 3rd edn. (London, 1952); E. Will, Histoire politique
du monde hellenistique, 2 vols. (Nancy, 1966-7); C. Preaux, Le Monde
FURTHER R E A D I N G 469
hellenistique, 2 vols. (Paris, 1978); Hartwig Maehler and V. M. Strocka, Das
ptolemaische Agypten: Akten des Internationalen Symposion 27.-29. Sept 1976
in Berlin (Mainz am Rhein, 1978); F. W. Walbank, The Hellenistic World
(Glasgow, 1981); J. Boardman et al. (eds.) The Cambridge Ancient History,
vii(i), 2nd edn. (Cambridge, 1984), viii, 2nd edn. (Cambridge, 1989), ix,
2nd edn. (Cambridge, 1993), x, 2nd edn. (Cambridge, 1996); N. G. L.
Hammond, The Macedonian State (Oxford, 1989); G. Holbl, Geschichte des
Ptolemaerreiches (Darmstadt, 1994); J. Whitehorne, Cleopatras (London,
1994), and S. Vleeming (ed.), Hundred-Gated Thebes (Papyrologica
Lugduno-Batava 27; Leiden, 1995). For discussion of Alexandria, see P. M.
Fraser, Ptolemaic Alexandria, 3 vols. (Oxford, 1972).
For the military history of the Ptolemies, see F. Adcock, The Greek and
Macedonian Art of War (Berkeley and Los Angeles, 1957), and Leo Casson,
Ships and Seamanship in the Ancient World (Princeton, 1971).
For Ptolemaic kingship, see Janet Johnson, The Demotic Chronicle as
an Historical Source', Enchoria, 4 (1974), 1-17; E. E. Rice, The Grand Pro-
cession of Ptolemy Philadelphus (Oxford, 1983); K. Bringmann, 'The King as
Benefactor: Some Remarks on Ideal Kingship in the Age of Hellenism', in
A. Bulloch et al (eds.), Images and Ideologies: Self-definition in the Hellenistic
World (Berkeley and Los Angeles, 1993) [including the article by L.
Koenen, The Ptolemaic King as a Religious Figure']; W. Huss, 'Das Haus
des Nektanebis und das Haus des Ptolemaios', Ancient History, 25 (1994),
111-17; J. K. Winnicki, 'Carrying Off and Bringing Home the Statues of the
Gods: On an Aspect of the Religious Policy of the Ptolemies towards the
Egyptians', Journal of Juristic Papyrology, 24 (1994), 149-90. On brother-
sister marriage, see R. S. Bagnail and B. W. Frier, The Demography of
Roman Egypt (Cambridge, 1994).
For the economic and social history of the Ptolemaic period, see J. N.
Svoronos, Die Munzen der Ptolemaer (Athens, 1908); M. Rostovtzeff, A
Large Estate in Egypt in the Third Century EC: A Study in Economic History
(Rome, 1967; repr. of 1922 edn.), and The Social and Economic History of the
Hellenistic World, 3 vols. (Oxford, 1953); D. J. Crawford, Kerkeosiris: An
Egyptian Village in the Ptolemaic Period (Cambridge, 1971); N. Davies and
C. M. Kraay, The Hellenistic Kingdoms: Portrait Coins and History (London,
1973); S. B. Pomeroy, Women in Hellenistic Egypt from Alexander to Cleo-
patra (Detroit, 1984); Dorothy Thompson, Memphis under the Ptolemies
(Princeton, 1988); R. A. Hazzard, Ptolemaic Coins: An Introduction for Col-
lectors (Toronto, 1995), and Dominic Montserrat, Sex and Society in Graeco-
Roman Egypt (London, 1996).
On Ptolemaic priests, temples, and religion, see Serge Sauneron, Les
Pretres de I'ancienne Egypte (Bourges, 1957); F. Dunand, 'La Classe sacerdo-
tale et sa fonction dans la societe egyptienne a 1'epoque hellenistique', in J.
470 FURTHER READING
Margueron et al. (eds.), Sanctuaires et Clerges (Paris, 1985), 41-59; Eleni
Vassilika, Ptolemaic Philae (Leuven, 1989); W. Huss, Der makedonische
Konig und die agyptischen Priester: Studien zur Geschichte des ptolemaischen
Agypten (Stuttgart, 1994), and R. Merkelbach, his Regina, Zeus Serapis. Die
griechisch-agyptische Religion nach den Quellen dargestellt (Stuttgart, 1995).
On ethnicity during the Ptolemaic Period, see C. Preaux, 'Esquisse
d'une histoire des revolutions egyptiennes sous les Lagides', CdE u (1936),
522-52; Naphthali Lewis, Greeks in Ptolemaic Egypt (Oxford, 1986); K.
Goudriaan, Ethnicity in Ptolemaic Egypt (Amsterdam, 1988), and P. Bilde et
al., Ethnicity in Hellenistic Egypt (Aarhus, 1992).
On the art and literature of the period, see Bernard Bothmer et al.,
Egyptian Sculpture of the Late Period 700 BC to AD 100 (New York, 1960);
Richard Fazzini, Images for Eternity: Egyptian Art from Berkeley and Brooklyn
(New York, 1975); Fazzini and Robert Bianchi, Cleopatra's Egypt (New
York, 1981), and Miriam Lichtheim, Ancient Egyptian Literature, iii. The
Late Period (Berkeley and Los Angeles, 1980).
15. The Roman Period
Until recently there was a dearth of books on Roman Egypt, a situation that
is rapidly being rectified. The best overall introduction is undoubtedly
Alan Bowman, Egypt after the Pharaohs (London, 1986). Other general
works worth reading include Naphthali Lewis, Life in Egypt under Roman
Rule (Oxford, 1983), and J. G. Milne, A History of Egypt under Roman Rule
(London, 1924). On the Romanization of Egypt, see Lewis 'The Romanity
of Egypt: A Growing Consensus', Atti del XVII Congresso Internazionale di
Papirologia (Naples, 1984), 1077-84. The special place of papyri and
ostraca is now discussed by Roger S. Bagnall, Reading Papyri, Writing
Ancient History (London, 1995). The administration of Roman Egypt is a
complex issue, but good summaries will be found in the books by Bowman
and Lewis cited above.
The role of the army is evaluated by R. Alston in Soldier and Society in
Roman Egypt (London, 1995), but for a recent review of the army in the
Eastern Desert see Valerie Maxfield, 'Eastern Desert Forts and the Army in
Egypt during the Principate', in Donald Bailey (ed.), Proceedings of the
British Museum Conference on Roman Egypt, published as JRA supplement
J 9 ( I 996)> 9 -I 9- Much of J. Lesquier, L'Armee romaine de I'Egypte
d'Auguste a Diocletian (Cairo, 1918), is still valid.
The grain trade has been the subject of much debate, but a fundamental
work is G. Rickman, 'The Grain Supply under the Roman Empire', in}. H.
D'Arms and E. C. Kopff (eds.), The Seaborne Commerce of Ancient Rome:
Studies in Archaeology and History (Rome, 1980), 261-76. For the
FURTHER R E A D I N G 471
Appianus estate, see Dominic Rathbone, Economic Rationalism and Rural
Society in Third Century AD Egypt (Cambridge, 1991).
The stone resources of the Eastern Desert are discussed by David
Peacock, Rome in the Desert: A Symbol of Power (Southampton, 1992), and
in Peacock and Maxfield, Survey and Excavation at Mons Claudianus
1987-1993, i. The Topography and Quarries (Cairo, 1996). The distribution
of Mons Claudianus rock is discussed in Peacock et al., 'Mons Claudianus
and the Problem of the granito del foro: A Geological and Geochemical
Approach', Antiquity, 68 (1994), 209-30. For the site of Myos Hormos,
see Peacock, The Site of Myos Hormos: A View from Space', JRA 6
(1993), 226-32. Desert routes are discussed by J.-C. Golvin and M. Redde,
'Du Nil a la Mer Rouge: Documents anciens et nouveaux sur les routes du
desert oriental d'Egypte', Karthago, 21 (1987), 5-64; Steven Sidebotham et
al., 'Survey of the Abu Sha'ar-Nile Road', A/A 95 (1991), 571-622, and R.
Zitterkopf and S. Sidebotham, 'Stations and Towers on the Quseir-Nile
Road',/£A 75 (1989), 155-89. On trade, see also L. Casson, The Periplus
Maris Erythraei (Princeton, 1989), and Sidebotham, Roman Economic
Policy in the Erythra Thalassa (Leiden, 1986).
Aspects of religion in Roman Egypt are covered in H. I. Bell, Cults and
Creeds in Graeco-Roman Egypt (1953); David Frankfurter, Religion in Roman
Egypt: Assimilation and Resistance (Princeton, 1998), and R. Witt, Isis in the
Graeco-Roman World (London, 1971). For Christianity and monasticism,
see Colin Walters, Monastic Archaeology in Egypt (Warminster, 1974), and
Bagnall, Egypt in Late Antiquity (Princeton, 1993). For mummy portraits,
see Euphrosyne Doxiadis, The Mysterious Fayum Portraits (London, 1995);
Susan Walker and Morris Bierbrier (eds.), Ancient Faces: Mummy Portraits
from Roman Egypt (London, 1997), and Bierbrier (ed.), Portraits and Masks
in Roman Egypt (London, 1997).
For pottery, see Jean-Yves Empereur, 'Un atelier de Dressel 2-4 en
Egypte au I He siecle de notre ere', Bulletin de Correspondence Hellenique,
suppl. 13 (1986), 599-608, and Empereur and M. Picon, 'A la recherche
des fours d'amphores', Bulletin de Correspondence Hellenique, suppl. 13
(1986), 103-24. For papyrological evidence, see H. Cockle, Tottery Manu-
facture in Roman Egypt', JRS 71 (1981), 87-97. For faience and glass
manufacture, see Paul T. Nicholson, Egyptian Faience and Glass (Princes
Risborough, 1993), and D. B. Harden, Roman Glass from Karanis (Ann
Arbor, 1936).
The nature of society in Roman Egypt is discussed in both R. S. Bagnall
and B. W. Frier, The Demography of Roman Egypt (Cambridge, 1994), and
Dominic Montserrat, Sex and Society in Graeco-Roman Egypt (London,
1996).
GLOSSARY
Acheulean stone tool industry, characterized by roughly symmetrical
bifacial handaxes and cleavers, which is linked with the appearance of
Homo erectus and also early Homo sapiens
akh one of the five principal elements that the Egyptians considered
necessary to make up a complete personality (the other four being the
ka, ba, name, and shadow); it was believed to be both the form in which
the blessed dead inhabited the underworld, and also the result of the
successful reunion of the ba with its ka
Amarna Letters set of cuneiform tablets from the city at Amarna, most of
which derive from the 'Place of the Letters of Pharaoh', a building identi-
fied as the official 'records office' in the central city at Amarna; all but
thirty-two of the 382 documents are items of diplomatic correspondence
between Egypt and many of the rulers of Western Asia
'anatomically modern' humans The first hominids (a) to resemble mod-
ern humans (in anatomical terms) and (b) to belong to the subspecies
homo sapiens sapiens; the term is in fact rather misleading, since the early
examples (who have brow ridges and larger teeth) are quite different
from genuinely modern humans such as ourselves
ankh hieroglyphic sign denoting 'life', which takes the form of a cross sur-
mounted by a loop; the sign was eventually adopted by the Coptic church
as its unique form of cross
Apis bull sacred bull who served as the ba (physical manifestation) of the
god Ptah, the cult of which dates back to the beginning of Egyptian
history; the bulls were buried in the Serapeum at Saqqara
Aten deity represented in the form of the disc or orb of the sun, the cult of
which was particularly promoted during the reign of Akhenaten
Aterian Palaeolithic industry (named after the site of Bir el-Ater in eastern
Algeria) that was characterized by a distinctive type of tanged stone
point (implying the use of hafting)
ba, Z>a-bird aspect of human beings that resembles our concept of'person-
ality', comprising the non-physical attributes that made each person
unique; it was often depicted as a bird with a human head and arms,
and was also used to refer to the physical manifestations of certain
gods
GLOSSARY 473
bark, bark shrine type of boat used to transport the cult images of Egyp-
tian gods from one shrine to another. As well as the principal shrines in
the temples, there were also small 'bark shrines' (also described as
'resting places', or 'way stations') along the routes of ritual processions
benben stone sacred stone (perhaps a lump of meteoric iron) at Heliopo-
lis, which symbolized the primeval mound and perhaps also the petri-
fied semen of the sun-god Atum-Ra; it served as the earliest prototype
for the obelisk and possibly even the pyramid
block statue type of sculpture representing an individual in a very com-
pressed squatting position, with the knees drawn up to the chin, thus
reducing the human body to a schematic blocklike shape
Book of the Dead funerary text known to the Egyptians as the 'spell for
coming forth by day', which was introduced at the end of the Second
Intermediate Period and consisted of about 200 spells (or 'chapters'),
over half of which were derived directly from the earlier Pyramid Texts
and Coffin Texts; the text was usually written on papyrus and placed in
the coffin, alongside the body of the deceased
BP abbreviation for 'before present', which is most commonly used for
uncalibrated radiocarbon dates or thermoluminescence dates; 'present'
is conventionally taken to be AD 1950
canopic jars four stone or ceramic vessels used for the burial of the viscera
(liver, lungs, stomach, and intestines) removed during mummification;
specific elements of the viscera were placed under the protection of four
anthropomorphic genii known as the Sons of Horus
cartonnage material consisting of layers of linen or papyrus stiffened with
plaster and often decorated with paint or gilding; it was most commonly
used for making mummy masks, mummy cases, anthropoid coffins,
and other funerary items
cartouche (shenu) elliptical outline representing a length of knotted rope
with which certain elements of the Egyptian royal titulary were sur-
rounded from the 4th Dynasty onwards
cataracts, Nile the six rocky areas of rapids in the middle Nile Valley
between Aswan and Khartoum
cippus type of protective stele or amulet on which the naked child-god
Horus was portrayed standing on a crocodile and holding snakes, lions,
or other animals. It was probably used to heal snake bites or scorpion
stings, but probably also had a more general prophylactic purpose
Coffin Texts group of over 1,000 spells, selections from which were
inscribed on coffins during the Middle Kingdom
demotic (Greek: 'popular (script)') cursive script known to the Egyptians
as sekh shat, which replaced the hieratic script by the 26th Dynasty; initi-
ally used only in commercial and bureaucratic documents, by the
474 GLOSSARY
Ptolemaic Period it was also being used for religious, scientific, and
literary texts
'divine adoratrice' (duat-netjei) religious title held by women, which was
originally adopted by the daughter of the chief priest of the god Amun in
the reign of Hatshepsut; from the reign of Rameses VI onwards it was
held together with the title 'god's wife of Amun'
donation stele slab of inscribed stone recording the granting of areas of
cultivable land to the gods of local temples
dromos processional way interconnecting different temples
encaustic painting technique, employing a heated mixture of wax and pig-
ment, which was particularly used for the Faiyum mummy portraits of
Roman Egypt
Epipalaeolithic chronological term usually applied to the last phase of the
Palaeolithic period in North Africa and the Ancient Near East; the
Egyptian and Lower Nubian Epipalaeolithic is characterized mainly by
its innovative lithic technology (microlithic flake tools) and its chrono-
logical position between the Nilotic Upper Palaeolithic and Neolithic
(i.e. £.10,000-5200 BC)
faience glazed non-clay ceramic material widely used in Egypt for the
production of such items as jewellery, shabtis, and vessels
false door stone or wooden architectural element comprising a rectangu-
lar imitation door placed inside Egyptian private tomb chapels, in front
of which funerary offerings were usually placed
foundation deposits buried caches of ritual objects placed at crucial points
under important structures such as pyramid complexes and temples;
the offering of model tools and materials was believed to maintain the
building magically for eternity
'god's wife of Amun' (hemet-netjer nt Imen) religious title first attested in
the early New Kingdom that later became closely associated with the
'divine adoratrice'. She played the part of the consort of Amun in religious
ceremonies at Thebes. From the late 2oth Dynasty onwards, she was
barred from marriage and adopted the daughter of the next king as heiress
to her office. In the 2501 and 26th Dynasties, the 'god's wife' and her adop-
ted successor played an important role in the transference of royal power
hieratic (Greek: hieratika, 'sacred') cursive script used from at least the
end of the Early Dynastic Period onwards, enabling scribes to write
more rapidly on papyri and ostraca, making it the preferred medium for
scribal tuition. An even more cursive form of the script, known as
'abnormal hieratic', began to be used for business texts in Upper Egypt
during the Third Intermediate Period
hieroglyphics (Greek: 'sacred carved (letters)') script consisting of picto-
grams, ideograms, and phonograms arranged in horizontal and vertical
GLOSSARY 475
lines, which was in use from the late Gerzean Period (0.3200 EC) to the
late fourth century AD
Horus name the first royal name in the sequence of five names making
up the Egyptian royal titulary, usually written inside a serekh (see below)
hypostyle hall (Greek: 'bearing pillars') large temple court filled with col-
umns and lit by clerestory windows in the roof; the columns were often
of varying diameter and height, but those along the axial route of the
temple were usually tallest and thickest
instruction (Egyptian: sebayt, wisdom texts, didactic literature) type of
literary text (e.g. The Instruction ofAmenemhat I) consisting of aphor-
isms and ethical advice, the earliest surviving example of which is said to
have been composed by the 4th Dynasty sage, Hardjedef
ka the creative life force of any individual, whether human or divine; rep-
resented by a hieroglyph consisting of a pair of arms, it was considered
to be the essential ingredient that differentiated a living person from a
dead one
kiosk small chapel without a roof, which was used to contain cult statues
of deities during festivals
Maat Goddess symbolizing justice, truth, and universal harmony, usually
depicted either as an ostrich feather or as a seated woman wearing such
a feather on her head. Small figurines depicting Maat were frequently
offered to deities by Egyptian rulers, thus indicating the king's role as
guarantor of justice and harmony on behalf of the gods
mammisi ('birth place', 'birth house') Coptic term invented by Champol-
lion to describe a building in major temple complexes of the Late Period
and Graeco-Roman Period, in which the rituals of the marriage of the
goddess (Isis or Hathor) and the birth of the child-god were celebrated; it
was placed at right angles to the main temple axis
mastaba-tomb (Arabic: 'bench') type of Egyptian tomb, the rectangular
superstructure of which resembles the low mud-brick benches outside
Egyptian houses; it was used for both royal and private burials in the
Early Dynastic Period but only for private burials from the Old Kingdom
onwards
Medjay Nubian nomadic group from the eastern deserts of Nubia, who
were often employed as scouts and light infantry from the Second Inter-
mediate Period onwards; they have been identified with the archaeo-
logical remains of the so-called pan-grave people (see below)
microlith type of stone tool, comprising a small blade or fraction of blade,
usually less than 5 mm. long and 4 mm. thick, which is regarded as the
archetypal tool of the Mesolithic Period, although it is now also recog-
nized in some Palaeolithic industries. Single microliths were some-
times used as the tip of an implement, weapon, or arrow, while multiple
476 GLOSSARY
examples were evidently hafted together to form composite cutting
edges on tools
Mnevis bull sacred animal regarded as the ba (physical manifestation) of
the sun-god at Heliopolis. Each Mnevis bull was required to be totally
black and was usually represented with a sun-disc and uraeus between
its horns. Because of his close connections with the sun-god, the Mnevis
was one of the few divine entities recognized by Akhenaten
Mousterian one of the key stone tool industries of the Middle Palaeolithic,
based on flakes produced from carefully prepared cores using the Leval-
lois technique, which gradually replaced the heavier handaxes of the
Acheulean industry (see above)
Nilometer device for measuring the height of the Nile, usually consisting
of a series of steps against which the increasing height of the annual
indundation, as well as the general level of the river, could be measured
nome, nome symbols Greek term used to refer to the forty-two traditional
provinces of Egypt, which the ancient Egyptians themselves called sepat;
for most of the Dynastic Period, there were twenty-two Upper Egyptian
and twenty Lower Egyptian nomes
nomen (birth name) royal name introduced by the epithet sa-Ra ('son of
Ra'), which was usually the last one in the sequence of the royal titulary;
it was the only one to be given to the pharaoh as soon as he was born
offering formula (hetep-di-nesu, 'a gift which the king gives') prayer
asking for offerings to be brought to the deceased, which formed the
focus of food offerings in private tombs; the formula is often accom-
panied by a depiction of the deceased sitting in front of an offering table
heaped with food
Opening of the Mouth ceremony funerary ritual by which the deceased
and his funerary statuary were brought to life
ostracon (Greek: ostrakon; pi. ostraka; 'potsherd') sherds of pottery or
flakes of limestone bearing texts and drawings, commonly consisting of
personal jottings, letters, sketches, or scribal exercises, but also often
inscribed with literary texts, usually in the hjeratic script
palace facade architectural style comprising a sequence of recessed niches,
which was particularly characteristic of the external walls of Early Dyn-
astic funerary buildings at Abydos and Saqqara
pan-grave culture material culture of a group of semi-nomadic Nubian
cattle-herders who entered Egypt in the late Middle Kingdom and Second
Intermediate Period; well attested in the Eastern Desert, their character-
istic feature being the shallow circular pit grave in which they buried
their dead
peret ('coming forth') Egyptian term for the spring season. The Egyptians
divided the year into twelve months and three seasons: akhet (the
GLOSSARY 477
inundation itself), peret (when the crops began to emerge), and shemu
(harvest-time). Each season consisted of four 3<D-day months, and each
month comprised three lo-day weeks
playa plain characterized by a hard clayey surface and intermittently sub-
merged beneath a shallow lake
prenomen (throne name) one of the five names in the Egyptian royal titu-
lary, which was introduced by the title nesu-bit: 'he of the sedge and the
bee', which is a reference both to the individual mortal king and the
eternal kingship (not 'king of Upper and Lower Egypt', as it is some-
times erroneously translated)
pylon (Greek: 'gate') massive ceremonial gateway, called bekhenet by the
Egyptians, which consisted of two tapering towers linked by a bridge of
masonry and surmounted by a cornice; it was used in temples from at
least the Middle Kingdom to the Roman Period
Pyramid Texts the earliest Egyptian funerary texts, comprising some 800
spells or 'utterances' written in columns on the walls of the corridors
and burial chambers of nine pyramids of the late Old Kingdom and First
Intermediate Period
rekhyt-bird Egyptian term for the lapwing (Vanellus vanellus), a type of
plover with a characteristic crested head, often used as a symbol for
foreigners or subject peoples
'reserve head' type of Memphite 4th Dynasty funerary sculpture, con-
sisting of a limestone human head, usually with excised (or unsculpted)
ears and enigmatic lines carved around the neck and down the back of
the cranium
royal titulary classic sequence of names and titles held by each of the
pharaohs consisting of five names (the so-called fivefold titulary), which
was not fully established until the Middle Kingdom; it consisted of the
Horus name, the Golden Horus name, the Two Ladies name (nebty), the
birth name (nomen; sa-Ra) and the throne name (prenomen; nesu-bit)
sacred lake artificial pool in the precincts of many Egyptian temples from
the Old Kingdom to the Roman Period
sa/Ptomb type of rock-cut tomb used in the el-Tarif area of western
Thebes by the local rulers of the Theban nth Dynasty
satrapy province in the Achaemenid empire
scarab type of seal found in Egypt, Nubia, and Syria-Palestine from the
nth Dynasty until the Ptolemaic Period; its name derives from the fact
that it was carved in the shape of the sacred scarab beetle (Scarabaeus
sacer)
sed-festival (heb-sed; royal jubilee) royal ritual of renewal and regenera-
tion, which was intended to be celebrated by the king only after a reign
of thirty years had elapsed
478 GLOSSARY
Serapeum term usually applied to buildings associated with the cults of
the Apis bull or the syncretic god Serapis. The Memphite Serapeum at
Saqqara, the burial place of the Apis bull, consists of a series of cata-
combs to the north-west of the Step Pyramid of Djoser
serdab (Arabic: 'cellar'; Egyptian: per-twt, 'statue-house') room in
mastaba-tombs of the Old Kingdom, where statues of the ka of the
deceased were usually placed
serekh rectangular panel (perhaps representing a palace gateway) sur-
mounted by the Horus falcon, within which the king's 'Horus name'
was written
shabti (ushabti, shawabti) funerary figurine, usually mummiform in
appearance, which developed during the Middle Kingdom out of the
funerary statuettes and models provided in the tombs of the Old King-
dom; the purpose of the statuettes was to perform menial labour for
their owners in the afterlife
shaduf irrigation tool comprising a long wooden pole with a vessel at one
end and a weight at the other, by means of which water could be trans-
ferred between rivers and canals
sistrum (Egyptian: seshesht, Greek: seistron) musical rattling instrument
played mainly by women, except when the pharaoh was making offer-
ings to the goddess Hathor
solar boat (solar bark) boat in which the sun-god and the deceased pharaoh
travelled through the netherworld; there were two different types of
bark: that of the day (mandet), and that of the night (mesektet)
speos (Greek: 'cave') type of small rock-cut temple
sphinx mythical beast usually portrayed with the body of a lion and the
head of a man, often wearing the royal nemes headcloth, as in the case of
the Great Sphinx at Giza; statues of sphinxes were also sometimes given
the heads of rams (criosphinxes) or hawks (hierakosphinxes)
talatat blocks small sandstone relief blocks dating to the Amarna Period,
the name for which probably derives from the Arabic word meaning
'three hand-breadths', describing their dimensions (although the word
may also have stemmed from the Italian word tagliata: 'cut masonry'
throne name seeprenomen
triad group of three gods, usually consisting of a divine family of father,
mother, and child worshipped at particular cult centres
Two Ladies name (nebty) one of the royal names in the 'fivefold titulary';
the term derives from the fact that this name was under the protection of
two goddesses: Nekhbet and Wadjet
uraeus serpent-image that protruded just above the forehead in most
royal crowns and headdresses; the original meaning of the Greek word
uraeus may have been 'she who rears up'
GLOSSARY 479
viceroy of Rush (Ring's son of Rush) the Egyptian official governing the
whole of Nubia (Wawat and Kush) in the New Kingdom
vizier term used to refer to the holders of the Egyptian title tjaty, whose
position is considered to have been roughly comparable with that of the
vizier (or chief minister) in the Ottoman empire; the vizier was therefore
usually the next most powerful person after the king
C H R O N O L O G Y
This chronology has been compiled on the basis of a number of different
criteria, ranging from the interpretation of ancient texts to the radiocarbon
dating of excavated materials. The dates from 664 BC to AD 394 are precise
(deriving primarily from Classical sources), whereas those for prehistory
(£.700,000-3000 BC) are approximations based on a combination of
stratigraphic information, seriation of artefacts, radiocarbon dates, and
thermoluminescence dates.
The dates for the majority of the Pharaonic Period (i.e. 0.3000-664 BC)
are based mainly on ancient king-lists, dated inscriptions, and astronom-
ical records. In the New Kingdom and Third Intermediate Period the
margin of likely error is about a decade, but this tends to increase as we
move further back in time, so that in the Old Kingdom it might be about
fifty years, and in the ist Dynasty it might be as high as 150 years.
When the dates for two or more dynasties overlap (principally in the
Second and Third Intermediate periods), this is because their rule was
accepted in different parts of the country. Overlapping dates for reigns
within dynasties usually indicate co-regencies (i.e. periods when a king
and his successor ruled simultaneously). When there are apparent gaps in
the chronology (particularly at the end of dynasties), this is usually because
there are one or two extremely poorly documented rulers, whose regnal
dates are unknown or difficult to assess.
By the beginning of the Old Kingdom, Egyptian rulers had five names;
the oldest of these was the so-called Horus name, and this is the one that
we have usually cited for kings of the ist~3rd dynasties (except in the case
of Djoser, whose Horus name, Netjerikhet, is given in parentheses). From
the 4th Dynasty onwards, we have usually given one or both of the so-
called cartouche names (i.e. the 'nesu-bit' and 'son of Ra' names), and we
have also sometimes added the Greek form of the name, especially when
this is the name by which a ruler is better known to modern readers (e.g.
Cheops for Khufu). Note that the existence and chronological position of
the 3rd-Dynasty ruler Nebka are currently a matter of debate.
C H R O N O L O G Y 481
Palaeolithic Period 1
Lower Palaeolithic
Middle Palaeolithic
Transitional Group
Upper Palaeolithic
Late Palaeolithic
Epipalaeolithic
0.700,000-7000 BP
C.7OO/5OO, OOO-25O,OOO BP
C.25O,OOO-7O,OOO BP
7O,OOO-5O,OOO BP
£.50, OOO-24,OOO BP
€.24, 000-10, 000 BP
C.IO,000-7000 BP
Saharan Neolithic Period
Early Neolithic
Middle Neolithic
Late Neolithic
C.8800-4700 BC
0.8800-6800 BC
0.6600-5100 BC
0.5100-4700 BC
Predynastic Period
Lower Egypt 2
Neolithic
Maadi Cultural Complex
€.5300-3000 BC
0.5300-4000 BC
(or 0.6400-5200 BP)
0.4000-3200 BC
Upper Egypt
Badarian Period 3 0.4400-4000 BC
Amratian (Naqada I) Period 0.4000-3500 BC
Gerzean (Naqada II) Period  0.3500-3200 BC
After 0.3200 BC the same chronological sequence applies to the whole of Egypt
Naqada III/'Dynasty o' 0.3200-3000 BC
Early Dynastic Period
ist Dynasty
Aha
Djer
Djet
Den
Queen Merneith
Anedjib
Semerkhet
Qa e a
C3OOO-2686 BC
0.3000-2890
482 CHRONOLOGY
2nd Dynasty
Hetepsekhemwy
Raneb
Nynetjer
Weneg
Sened
Peribsen
Khasekhemwy
2890-2686
Old Kingdom
yd Dynasty
Nebka
Djoser (Netjerikhet)
Sekhemkhet
Khaba
Sanakht?
Huni
4th Dynasty
Sneferu
Khufu (Cheops)
Djedefra (Radjedef )
Khafra (Chephren)
Menkaura (Mycerinus)
Shepseskaf
$th Dynasty
Userkaf
Sahura
Neferirkara
Shepseskara
Raneferef
Nyuserra
Menkauhor
Djedkara
Unas
6th Dynasty
Teti
Userkara [a usurper]
Pepy I (Meryra)
2686-2l6o BC
2686-2613
2686-2667
2667-2648
2648-2640
2640-2637
2637-2613
2613-2494
2613-2589
2589-2566
2566-2558
2558-2532
2532-2503
2503-2498
2494-2345
2494-2487
2487-2475
2475-2455
2455-2448
2448-2445
2445-2421
2421-2414
2414-2375
2375-2345
2345-2181
2345-2323
2323-2321
2321-2287
CHRONOLOGY
Merenra
Pepy II (Neferkara)
Nitiqret
jth and 8th Dynasties
483
2287-2278
2278-2184
2184-2181
2181-2160
Numerous kings, called Neferkara, presumably in imitation of Pepy II
First Intermediate Period
gth and loth Dynasties
( H er akleopolitan)
Khety (Meryibra)
Khety (Nebkaura)
Khety (Wahkara)
Merykara
nth Dynasty (Thebes only)
[Mentuhotep I (Tepy-a: 'the ancestor')] Intef I (Sehertawy)
Intef II (Wahankh)
Intef III (Nakhtnebtepnefer)
Middle Kingdom
nth Dynasty (all Egypt)
Mentuhotep II (Nebhepetra)
Mentuhotep III (Sankhkara)
Mentuhotep IV (Nebtawyra)
i2th Dynasty
Amenemhat I (Sehetepibra)
Senusret I (Kheperkara)
Amenemhat II (Nubkaura)
Senusret II (Khakheperra)
Senusret III (Khakaura)
Amenemhat III (Nimaatra)
Amenemhat IV (Maakherura)
Queen Sobekneferu (Sobekkara)
ijth Dynasty
Wegaf(Khutawyra)
Sobekhotep II (Sekhemra-khutawy)
lykhernefert Neferhotep (Sankhtawy-sekhemra)
2l6o-2O55 BC
2l6o-2025
2125-2055
2I25-2II2
2112-2063
2063-2055
2055-1650 BC
2055-1985
2055-2004
2004-1992
1992-1985
1985-1773
1985-1956
1956-1911
1911-1877
1877-1870
1870-1831
1831-1786
1786-1777
1777-1773
1773-after 1650
484 CHRONOLOGY
Ameny-intef-amenemhat (Sankhibra)
Hor (Awibra)
Khendjer (Userkara)
Sobekhotep III (Sekhemra-sewadjtawy)
Neferhotep I (Khasekhemra)
Sahathor
Sobekhotep IV (Khaneferra)
Sobekhotep V
Ay (Merneferra)
iflh Dynasty
Minor rulers probably contemporary with the i3th or i5th Dynasty
Second Intermediate Period
i$th Dynasty (Hyksos)
Salitis/Sekerher
Khyan (Seuserenra)
Apepi (Aauserra)
Khamudi
i6th Dynasty
Theban early rulers contemporary with the i5th Dynasty
iyth Dynasty
Rahotep
SobekemsafI
Intef VI (Sekhemra)
IntefVII(Nubkheperra)
Intef VHI(Sekhemraherhermaat)
SobekemsafI I
Siamun (?)
Taa (Senakhtenra/Seqenenra)
Kamose (Wadjkheperra)
New Kingdom
i8th Dynasty
Ahmose (Nebpehtyra)
Amenhotep I (Djeserkara)
Thutmose I (Aakheperkara)
Thutmose II (Aakheperenra)
1773-1650
1650-1550 BC
1650-1550
c.i6oo
^1555
1650-1580
£.1580-1550
£.1560
1555-1550
1550-1069 BC
1550-1295
1550-1525
1525-1504
1504-1492
1492-1479
CHRONOLOGY 485
Thutmose III (Menkheperra)
Queen Hatshepsut (Maatkara)
Amenhotep II (Aakheperura)
Thutmose IV (Menkheperura)
Amenhotep III (Nebmaatra)
Amenhotep IV/Akhenaten (Neferkheperurawaenra)
Neferneferuaten (Smenkhkara)
Tutankhamun (Nebkheperura)
Ay (Kheperkheperura)
Horemheb (Djeserkheperura)
Ramessid Period
i<)th Dynasty
Rameses I (Menpehtyra)
Sety I (Menmaatra)
Rameses II (Usermaatra Setepenra)
Merenptah (Baenra)
Amenmessu (Menmira)
Sety II (Userkheperura Setepenra)
Saptah (Akehnrasetepenra)
Queen Tausret (Sitrameritamun)
2Oth Dynasty
Sethnakht (Userkhaura Meryamun)
Rameses III (Usermaatra Meryamun)
Rameses IV (Heqamaatra Setepenamun)
Rameses V (Usermaatra Sekheperenra)
Rameses VI (Nebmaatra Meryamun)
Rameses VII (Usermaatra Setepenra Meryamun)
Rameses VIII (Usermaatra Akhenamun)
Rameses IX (Neferkara Setepenra)
Rameses X (Khepermaatra Setepenra)
Rameses XI (Menmaatra Setepenptah)
1479-1425
1473-1458
1427-1400
1400-1390
1390-1352
1352-1336
1338-1336
1336-1327
1327-1323
1323-1295
1295-1069 BC
1295-1186
1295-1294
1294-1279
1279-1213
I2I3-I203
1203-1200?
I2OO-II94
II94-II88
Il88-Il86
1186-1069
1186-1184
1184-1153
II53-II47
II47-II43
1143-1136
1136-1129
II29-II26
II26-II08
II08-I099
1099-1069
Third Intermediate Period 1069-664 BC
2ist Dynasty 1069-945
Smendes (Hedjkheperra Setepenra) 1069-1043
Amenemnisu (Neferkara) 1043-1039
Psusennes I [Pasebakhaenniut] (Akheperra Setepenamun) 1039-991
Amenemope (Usermaatra Setepenamun) 993~9&4
4 86
CHRONOLOGY
Osorkon the Elder (Akheperra setepenra)
Siamun (Netjerkheperra Setepenamun)
Psusennes II [Pasebakhaenniut] (Titkheperura Setepenra)
22nd Dynasty
Sheshonq I (Hedjkheperra)
Osorkon I (Sekhernkheperra)
Takelot I
Osorkon II (Usermaatra)
Takelot II (Hedjkheperra)
Sheshonq III (Usermaatra)
Pimay (Usermaatra)
Sheshonq V (Aakheperra)
Osorkon IV
23 rd Dynasty
Kings in various centres, contemporary with the later 22nd, 24th,
dynasties, including:
Pedubastis I (Usermaatra)
Input I
Sheshonq IV
Osorkon III (Usermaatra)
Takelot III
Rudamon
Peftjauawybast
Input II
984-978
978-959
959-945
945-715
818-715
and early 25th
24th Dynasty
Bakenrenef (Bocchoris)
2$th Dynasty
Piy (Menkheperra)
Shabaqo (Neferkara)
Shabitqo (Djedkaura)
Taharqo (Khunefertemra)
Tanutamani (Bakara)
727-715
720-715
747-656
747-716
716-702
702-690
690-664
664-656
Late Period
26th Dynasty
[Nekau I
Psamtek I (Wahibra;
664-332 BC
664-525
672-664]
664-610
CHRONOLOGY
Nekau II (Wehemibra)
Psamtek II (Neferibra)
Apries (Haaibra)
Ahmose II [Amasis] (Khnemibra)
Psamtek III (Ankhkaenra)
2 /th Dynasty (ist Persian Period)
Cambyses
Darius I
Xerxes I
Artaxerxes I
Darius II
Artaxerxes II
28th Dynasty
Amyrtaios
2gth Dynasty
Nepherites I [Nefaarud]
Hakor [Achoris] (Khnemmaatra)
Nepherites II
joth Dynasty
Nectanebo I (Kheperkara)
Teos (Irma atenra)
Nectanebo II (Senedjemibra setepenanhur)
2nd Persian Period
Artaxerxes III Ochus
Arses
Darius III Codoman
Ptolemaic Period
Macedonian Dynasty
Alexander the Great
Philip Arrhidaeus
Alexander IV 4
Ptolemaic Dynasty
Ptolemy I Soter I
Ptolemy II Philadelphus
487
610-595
595-589
589-570
570-526
526-525
525-404
525-522
522-486
486-465
465-424
424-405
405-359
404-399
404-399
399-380
399-393
393-380
£.380
380-343
380-362
362-360
360-343
343-33 2
343-338
338-336
336-332
332-30 BC
332-305
332-323
323-317
317-310
305-285
285-246
CHRONOLOGY
Ptolemy III Euergetes I
Ptolemy IV Philopator
Ptolemy V Epiphanes
Ptolemy VI Philometor
Ptolemy VII Neos Philopator
Ptolemy VIII Euergetes II
Ptolemy IX Soter II
Ptolemy X Alexander I
Ptolemy IX Soter II (restored)
Ptolemy XI Alexander II
Ptolemy XI I Neos Dionysos (Auletes)
Cleopatra VII Philopator
Ptolemy XIII
Ptolemy XIV
Ptolemy XV Caesarion
246-221
221-205
205-180
180-145
145
170-116
116-107
107-88
88-80
80
80-51
51-30
51-47
47-44
44-30
Roman Period 5
Augustus
Tiberius
Gaius (Caligula)
Claudius
Nero
Galba
Otho
Vespasian
Titus
Domitian
Nerva
Trajan
Hadrian
Antoninus Pius
Marcus Aurelius
Lucius Verus
Commodus
Septimius Severus
Caracalla
Geta
Macrinus
Didumenianus
Severus Alexander
Gordian III
30 BC-AD 395
30 BC-AD 14
AD 14-37
37-41
41-54
54-68
68-69
69
69-79
79-81
81-96
96-98
98-117
117-138
138-161
161-180
161-169
180-192
193-211
198-217
209-212
217-218
218
222-235
238-242
488
CHRONOLOGY
Philip
Decius
Callus and Volusianus
Valerian
Gallienus
Macrianus and Quietus
Aurelian
Probus
Diocletian
Maximian
Galerius
Constantius
Constantine I
Maxentius
Maximinus Daia
Licinius
Constantine II
Constans (co-ruler)
Constantius II (co-ruler)
Magnetius (co-ruler)
Julian the Apostate
Jovian
Valentinian I (west)
Valens (co-ruler, east)
Gratian (co-ruler, west)
Theodosius (co-ruler)
Valentinian II (co-ruler, west)
Eugenius (co-ruler)
489
244-249
249-251
251-253
253-260
253-268
260-261
270-275
276-282
284-305
286-305
293-311
293-306
306-337
306-312
307-324
308-324
337-340
337-350
337-361
350-353
361-363
363-364
364-375
364-378
375-383
379-395
383-392
392-394
1 The dates for the Palaeolithic Period are primarily based on uncalibrated radiocarbon
dates, therefore they are given as radiocarbon years BP (before present) rather than as BC
dates. In order to establish a secure link between the BP and BC dates, the overall range
for the Neolithic is cites in terms of both BP and BC. All other dates are BC or AD.
2 The Term 'Lower Egypt' here denotes the Delta, the Faiyum, and an area stretching as
far as zookm. south of Cairo.
3 The Badarian may have been a culture restricted to the Badari region near Asyut in
Middle Egypt, rather than being a chronological phase throughout the whole of
southern Egypt.
4 Alexander IV was only the nominal ruler in 310-305 BC.
5 The overall dates given here for the Roman Period begin with the official establishment
of Egypt as a Roman province (on 31 August 30 BC) and end with the final division of the
empire into western and eastern sections in AD 395 (i.e. the beginning of the Byzantine
Period, which is usually described as the Coptic or Christian Period in Egypt).
This page intentionally left blank
I N D E X
A-Group culture 63, 64, 73, 97, 308,
314-15
Aamu people see Asiatics: Hyksos
Aat, Queen 157
Aata 203, 218
Aauserra Apepi see Apepi
Abbassiya 17
Abbott Papyrus 193-4
Abka rock art 26-7
absolute dating 2, 44, 138
Abu Simbel 291, 367
Abusir, and Fifth- Dynasty kings 99,
IOO, IO2
Abydos 12, 64, 170
Early Dynastic Period 73, 76, 77,
78-80, 314
Eighteenth Dynasty 206, 211—12,
214, 222
First Intermediate Period 125-6,
134
Middle Kingdom 150, 155, 168
Naqada 50, 52, 57, 60, 314
Osiris temple 287, 291
Predynastic Period 3,5,7
Achaemenes (brother of Xerxes) 376
Acheulean period 17-18
Adaiima site (Naqada II) 50, 51
administration
Amarna Period 279
Dynasty o 75
First Dynasty 64
First Intermediate Period in,
117-20
First Persian Period 375
Herakleopolitan Dynasty 131-2
Middle Kingdom 141, 162-4
New Kingdom 242, 249, 261-4,
284-5
Old Kingdom 93-5, 100-1, 106-7
Ptolemaic Period 404, 411-12
Ramessid Period 298-300
Roman Period 416-17
Third Intermediate Period 324,
329,330,335-8
trade 73-4
writing 74-5
Admonitions of an Egyptian Sage 134
Afghanistan 62, 313
Afian industry 25
Africa 16-17, 22» IQ6, 234
Africanus 5, 179
afterlife
democratization of the 150, 158,
168
Early Dynastic 68
expressed in tomb reliefs 98
Fourth Dynasty 87
gallery tombs for the 80
kingship and 92-3, 371
Ptolemaic Period 407
Pyramid Texts and 102-3
sed-festivals 89
Agatharchides 422
Agathocles (minister to Ptolemy IV)
410
Agathostratus, Admiral 398
492
INDEX
Agesilaus of Sparta 377-8, 380,
381-2
agriculture
Badarian 39
C-Group 315
Early Dynastic period 65-6, 315
Libyan period 332, 344
Maadian 55
Middle Kingdom 150-1
Naqada 58
Neolithic 28-9, 30, 33, 34, 36
New Kingdom 253
Ptolemaic Period 4 04, 405, 410
and pyramid construction 94
Roman Period 419, 420-2
Aha (ist Dynasty) 67
Ahhotep I, Queen 208, 210, 211,
218-19
Ahmarian industry 23
Ahmose, Queen (wife of Thutmose I)
221
Ahmose, son of Ibana 200, 201, 207,
213-14, 216, 223, 224
Ahmose-ankh, Prince 218, 219
Ahmose I (i8th Dynasty) 176, 197-8,
203-6, 213, 214
construction by 207-11
female relations 218,219
mummy of 211, 212
Ahmose II (26th Dynasty) 367-8,
37°-i> 373' 374
Ahmose-Nefertari, Queen 210, 213,
214,215,218,219,221
Ahmose Pennekhbet 213, 216, 223,
224, 227, 228
Ain Besor, Palestine 73-4
Akhenaten see Amenhotep
IV/Akenhaten
Akhetaten (Amarna) 165, 255,
269-70, 274, 281, 284
Alexander IV 402
Alexander the Great 13, 279, 382,
386,399
army of 393, 394
burial place of 400
coronation of 388-9
succession 389-90, 402
Alexandria 9, 388
and grain trade 420-1
Ptolemaic Period 399-400, 401,
402,410
Roman Period 417, 418, 428, 432,
433' 434> 435
Serapis cult 429-30
Amarna see Akhetaten
Amarna Letters 250, 261, 270, 319,
321
Amarna Period
art and architecture in 272-6
funerary beliefs 276-8, 304
impact of 279-81
life outside of Amarna 278-9
religion 269-70
royal women 271-2
Amelineau, Emile 67, 80
Amenemhat, Prince 229
Amenemhat (high priest) 237
Amenemhat I (i2th Dynasty) 135,
146-8, 163, 185
Amenemhat II (i2th Dynasty) 151-2,
3 i8
Amenemhat III (i2th Dynasty)
156-8, 183, 185
Amenemhat-itj-tawy 146-7
Amenemhat IV (i2th Dynasty)
158-9, 176
Amenemhat (nomarch) 151
Amenemhat (vizier) 145, 146
Amenemheb 262
Amenemnisu (2ist Dynasty)
328
Amenemopet (vizier) 263
I N D E X
493
Amenhotep (high priest of Amun)
301,302
Amenhotep I (i8th Dynasty) 9, 182,
208, 212-14, 220
monuments of 214-16
Amenhotep II (i8th Dynasty) n, 231,
237, 241-6
athleticism 241-2
building programme 242-4
in the Levant 244-6
military associations 262
Amenhotep III (i8th Dynasty) n,
244, 253
administration 264
building programme 256-9
deification of 254-5, 266, 268, 293
international relations and 260-1
and Queen Tiye 259-60
tomb of 257
Amenhotep IV/Akhenaten (i8th
Dynasty) 254, 255, 259, 260,
303-4,429
and Amarna 269-70
funerary beliefs and 277-8
and Karnak 267-9
reburial of 283
representational art of 272-4
and royal wives 271-2
Amenhotep (royal steward) 234, 264
Amenirdis I 347, 355
Amenmessu (igth Dynasty) 295
Amratian (Naqada I) phase 43, 45-9,
5 2 ~3
Amun 122, 144, 346
and 'god's wife' 32, 252
Ahmose-Nefertari 210, 215, 219,
221
Amenirdis I 347
Hapshepsut 226, 228
Merytra 244
Nefrura 229
Nitiqret366
Satamun 220
Shepenwepet I 347
Third Intermediate Period 354-5
New Kingdom 266-7
Ahmose 209-10
Amenhotep III 256, 258
Horemheb 285
Rameses III 298
Rameses IV 300
Thutmose III 237
Thutmose IV 229
Ptolemaic Period 411
subordination of temporal ruler to
298, 300, 301, 304-5, 339-40
Third Intermediate Period 326-7,
329,332,354
see also high priests; priests
Amyntas (Macedonian rebel) 382
Amyrtaios (28th Dynasty) 377
ancestor worship 7, 226, 243
Andros, battle of 398
Anedjib (ist Dynasty) 67, 74
animal cults 356, 384, 429
animal husbandry 33, 36, 39, 49, 52,
55
Ankh-Pediese 341-2
Ankhesenamun (Ankhesenpaaten)
(wife of Tutankhamun) 272,
281
Ankhtifi (nomarch) 118-23
Anthony the monk 431, 436
Antigonus Gonatas 398
Antigonus ('the one-eyed') 389
Antinoopolis 415
Antiochus IV 412
Antoninus Pius, Emperor 9
Apepi (i5th Dynasty) 180, 182, 183,
188, 198, 199
Apis-bull burials 257, 291, 294, 356,
370' 374
494
INDEX
Apollonius of Rhodes 400
Apries (26th Dynasty) 367-8, 372-3
Arabia 418, 436
archaism
Late Period 378, 379, 383
Ramessid Period 291-2
Third Intermediate Period 351-2,
361,371
Arikamedu excavation 427
Aristarchus of Samothrace 400
aristocracy
First Intermediate Period 111-12,
113, 131-2
Middle Kingdom 163, 169-70
see also elite
New Kingdom 234
Arkinian sites 27
army 262
Amarna Period 270, 282
First Persian Period 375-6
Late Period 366-7, 373, 380-1
Libyan period 342-3
Ptolemaic Period 389, 393-6, 409
Ramessid Period 286, 292
Roman Period 417-19 see also
mercenaries; navy
Arnold, Dieter 143, 155
Arnold, Dorothea 147
aromatic oil 69
Arrhidaeus (brother of Alexander)
402
Arrhidaeus (Macedonian Dynasty)
389
arrowheads, Neolithic 31, 34
Arsinoe II (sister and wife of Ptolemy
II) 403-4
art
Amarna style 272-4
archaism 351-2, 361
depictions of bound captives 310
Eighteenth Dynasty 204, 242
First Intermediate Period 114
Libyan period 341
Mesopotamian motifs 62
Middle Kingdom 143-4, 149-50,
151-2, 153
Naqada 46-7, 51
Predynastic Period 75
rock 26-7, 63
Roman Period 431-2
temple 76
Theban nth Dynasty 127
Third Intermediate Period 361
Upper Egypt 122-3
victory representations 46-7,
60-1, 105
xenophobic 320
Artatama, king of Mitanni 250, 251
Artaxerxes II (27th Dynasty) 379
Artaxerxes III (Second Persian
Period) 379, 381
Artemisium 376
Ashurbanipal, king of Assyria 332,
353
Asia 146, 147, 149, 151-2, 155, 157,
167, 372
Asia Minor 3 92
Asiatics 174-5, 310, 319
assassinations 298-9, 390, 411
Assyria 251, 290, 332, 386, 387
conquest of Egypt 352-4, 365
renewed invasion threat by 371-2
astronomy
chronology and 8-10
observatories 216
pyramid postitioning 91
religion and 87, 102
Aswan 103, 418, 422-3, 433
Asyut cemetery 131, 134
Aten
and Amenhotep III 254-5
and death 277
INDEX
495
temples 275-6, 278
see also Amenhotep I V/Akhenaten
Aterian industry 22
Athenaeus 398
Athens 391
Augustus (Octavian), Emperor 414,
420,430
Aurelius Appianus 421
Avaris
Eighteenth Dynasty 208, 234
Middle Kingdom 160
Ramessid Period 286-7, 2 9 2
Second Intermediate Period 172,
174-82, 195
Thebes war 197, 199-202
Awibra Hor (i3th Dynasty) 160, 185
axes 199
Ay (i8th Dynasty) 253, 283-4
ba (spiritual force) 169, 277
Baal Zephon, weather-god 177
Babylon (Sangar) 167, 245, 251
Babylonia 352,372
Badarian Period 36-40,43,45-6
Bakenrenef (24th Dynasty) 331
Bakenrenef (vizier), tomb of 383, 385
Ballanan—Silsilian industry 25
banditry 411-12,419
Bashendi culture 30-1
Battlefield palette 61
Bay ('chancellor of the entire land')
296, 297
beards, as symbol of power 47-8
Beautiful Festival of the Valley 144,
231-2,233,287
Beersheba, Palestine 53
Beni Hasan (Speos Artemidos) 230
Berenice 426,428
Bietak, Manfred 176-7,178,180,
208
biographical inscriptions 101,
118-19, 120-1, 126, 131, 135-6,
384,386
Bir Kiseiba (Neolithic site) 28, 30
Bir Sahara oasis 18, 19
Bir Tarfawi oasis 18, 19
Birket Habu harbour 257
blade production 20, 23, 48
boat burials 70, 71
boats and ships
in the afterlife 89
Epipalaeolithic 32
grain 420-1
in Naqada art 51
sea-going 58, 145, 317, 426-7
stone transportation 423
war galleys 372, 381-2
see also navy
Boker Tachtit site 21
Bonnet, Charles 196
Book of Gates 437
Book of the Dead 193, 277, 281, 351,
358.384
Book of Two Ways 170
Book of what is in the Netherworld 237,
262,358
border police 106
borders 175
Egyptians changing conception of
311-13
Middle Kingdom 154-5, 156
New Kingdom 220
Saite Dynasty 371-3
Third Intermediate Period 328
Upper and Lower Egypt 188-90
see also boundaries
brick moulds 166
bronze 361-2
Brunton, Guy 43, 154
Bubastis 337, 340, 344
Buhen 73, 195, 199, 203, 229
496
INDEX
building projects
Amenhotep I (i8th Dynasty)
214-16
Amenhotep II (i8th Dynasty)
242-4
Amenhotep III (i8th Dynasty)
256-9
Amenhotep IV/Akhenaten (i8th
Dynasty) 274-6
Eighteenth Dynasty 208-9
First Intermediate Period no
Hatshepsut (i8th Dynasty)
229-34, 237
Horemheb (i8th Dynasty) 284-5
Late Period 378
Middle Kingdom 142-4, 150,
156-7, 169
New Kingdom 265
Old Kingdom 85-7, 97
Osorkon II (22nd Dynasty) 344
Rameses II (i9th Dynasty) 291,
292-3
Rameses III (2oth Dynasty) 298
Rameses IV (2oth Dynasty) 299
Saite Dynasty 370-1
Sety I (i9th Dynasty) 286-7
Sheshonq I 330
Third Intermediate Period 327
Thutmose I (i8th Dynasty) 222-3
Thutmose II (i8th Dynasty) 226-7
Thutmose III (i8th Dynasty) 236
Thutmose IV (i8th Dynasty) 248-9
Bull palette 61
burials
A-Group culture 63
Amarna Period 277
Apis-bull 257, 291, 294, 356, 370,
374
Badarian 37
boat 70, 71
chariot horses 347
communal 177, 357
elite 59, 60
family 383
First Intermediate Period 116
Kushite 350, 351
late Middle Palaeolithic 21
Late Palaeolithic 26
Late Period 385-6
Maadian 55-6
Middle Kingdom 143, 168-9
Naqada 41, 45, 50, 52-3, 58, 59, 60
Neolithic 34, 35, 36
Qarunian 32
recycled 345, 358
Roman Period 431-2
Second Intermediate Period 175,
178, 186-7, X 95' J 97
Third Intermediate Period 356-60
Upper Palaeolithic 23 see also
Abydos; cemeteries; coffins;
grave goods; mummification;
tombs
Buto (Predynastic site) 54, 56, 59, 62,
65
Byblos 74, 96, 101, 106, 151, 160, 167,
320-1
C-Group culture 73,308,315
Caesar, Gaius Julius 411
Callimachus of Gyrene 400
Callixeinus of Rhodes 401
Cambyses (27th Dynasty) 374,375
Canaanite blades 54
canals 152-3,369,372,375-6,425
canopic jars 158,358,360,371
Canopus decree 407
captives 97,152,154,155,175, 227,
238,239,240,244-5, 2 5°>
287,295
symbolized depictions of 310
I N D E X
497
caravan routes 106, 149, 176, 428
Carchemish, battle of 372
Carian mercenaries 365, 366
cartonnage masks 114, 115, 129, 170
Caton-Thompson, Gertrude 32, 33
cattle censuses (hesbet) 5
cattle herding 28, 315
cavalry
Ptolemaic Period 393, 394, 395
Roman Period 417
cemeteries
A-Group 63
desecration of 134
First Intermediate Period
provincial 112-13
Herakleopolitan era 130, 131
late i7th and early i8th Dynasty
211-12
Maadian 53, 55
Middle Kingdom 142
Naqada 58, 59
New Kingdom 187
North Saqqara 70-3
Second Intermediate Period 185,
187, 189-90, 195
Theban 124-5 see a^so Abydos
Cemetery T 59
census 5, 88, 95, 435
centralization
First Intermediate Period in
Libyan period 344
Middle Kingdom 147, 155
Old Kingdom 95, 107
Rameses III 298
Chabrias, Admiral 381, 382
Chaldaea (Babylonia) 372-3, 374,
386,387
Champollion, Jean Francois 406
Chaonnophris (Theban king) 411
chariots 202, 225, 239, 290
Cheops see Khufu
chert 20, 22,35,39
China 420
Christian quarry-workers 424
Christianity 431, 436
chronology 1-15
Eighteenth Dynasty 205
First Intermediate Period 109
Naqada 42-5
New Kingdom 220
Old Kingdom 84-5
Second Intermediate Period
176-7, 178-80, 186-7, 1 9°
Third Intermediate Period 325
Twelfth Dynasty 138, 154
Cimmerian barbarians 372
civil unrest in Ptolemaic Period
410-11
Claudius, Emperor 430
cleavers 18
Cleopatra VII 403, 413, 414
cleruchs 396, 404, 409, 410
climate change 25, 27, 315
First Intermediate Period 119
Middle Palaeolithic period 19-20
Old Kingdom 107
Upper Palaeolithic 23
co-regencies 10-11
Amarna Period 267, 269, 271
Eighteenth Dynasty 213, 234, 241,
2 47> 2 55
Late Period 377
Middle Kingdom 138, 149, 154
Ramessid Period 288, 291
Coffin Texts 115, 168
coffins
Early Dynastic 68, 72
First Intermediate Period 115
Middle Kingdom 168-9, 1 7°
Naqada II 50
New Kingdom 219
recycled 345
INDE X
Second Dynasty 81
Theban 193
Third Intermediate Period 359-60
colossi 76-7, 139, 156, 183, 230, 232,
253,265,293
and Amenhotep III 257, 258-9
Ptolemaic Period 402
Saite Dynasty 370
see also Great Sphinx
Complaints ofKhakheperraseneb 135
conscription 82, 148
Constantine, Emperor 431
Copaic Basin of Boeotia 153
copper 96, 101, 105, 156, 313
Abydos royal tombs 66
Maadian 55
Naqada II 51
corvee system 161, 170
cosmetics 55 see also palettes
craftsmanship
First Intermediate Period 114
Naqada 48, 51-2
Ptolemaic Period 406-7, 409
Ramessid Period 293
regionalism 143-4
Roman Period 432-4
royal sponsorship of 66
sacred 186
Third Intermediate Period 361-2
Upper Egypt Predynastic 57, 58
Crete 166, 167, 204-5, 2O ^> 2 34> 2 35>
295
cult centres
Early Dynastic Period 76-8
Eighteenth Dynasty 204, 209, 213,
222, 230
Saite Dynasty 370
Twenty- Second Dynasty 178
culture 12-13
Asiatic 176
in Byblos 320
First Intermediate Period no,
in-12, 113-14
Greek 42 9
Hyksos 182
Kerma 196, 197
Kushite-Egyptian 349-50
Middle Kingdom 171
Ptolemaic Period 407-8
Second Intermediate Period 184
see also archaism
cuneiform writing 261,319
Amarna Letters 250, 261, 270,
321
Cusae, temple of 188, 190, 230
cylinder seals 62, 159, 176
,196
3*9>
Cyprus 167, 295, 297, 373, 392, 410,
4*3
Cyrenaica 332, 334, 390, 410, 413
Gyrene 367, 374
Cyrus the Great 234-5, 373
Dabban industry 23
Dahshur
Fourth Dynasty pyramids 87-8
Middle Kingdom pyramids 156
Twelfth Dynasty pyramids 152,
157-8^85
Dakhla Oasis 18,30
Darius I (27th Dynasty) 375
Darius II (27th Dynasty) 376
day-books (genut) 4,151-2
de Morgan, Jacques 41,43, 59,160
death
Amarna Period 277
kingship and 67, 92-3,371
Libyan Period 358
Ptolemaic Period 408
decans 9
decentralization 324,328,329,330,
335^37.338,361,362-3
498
INDEX
499
deification
ofkings
Middle Kingdom 140-1
New Kingdom 213, 254-5, 293
Rameses II 255, 293
of officials 194, 264
Deir el-Bahri 140, 143, 144, 147, 169,
192
Hatshepsut 231, 232-4, 317
New Kingdom 211, 215, 216, 229,
237, 244
Deir el-Ballas settlement 198
Deir el-Medina 165, 213, 283, 298,
299,300,328
Deir Rifa cemetery 189-90
Delta 62
Early Dynastic Period 72-3
graves in 56
Hyksos 157, 174-206
Late Period 366
Libyan period 333, 337
military colonies 295
Naqada culture 64
Neolithic sites 34-6
New Kingdom 215
Sea Peoples 297, 322
Tasian culture 37
Third Intermediate Period 328
Demetrius (son of Antigonus) 395,
397,400
demography 112, 309, 435-6
Demotic Chronicle 5
Den (ist Dynasty) 7, 67, 68, 74, 77
dendrochronology 3, 12
Denen 322
desecration and defacement
of cemeteries 134
Giza Sphinx temple 248
of Hatshepsut monuments 237,
241, 242
Ptolemaic Period 412
Dialogue between a Man Tired of Life
and his 'ba' 135, 169, 171
Diocletian, Emperor 431
Diodorus Siculus 380, 395, 397, 422,
435
Dionysius Petosarapis 401, 403, 405
diplomacy 96
with Byblos 321
Eighteenth Dynasty 208, 235, 240,
250, 260-1, 265
Late Period 379
Saite Dynasty 366, 372
Djary (military officer) 126
Djedefra (4th Dynasty) 90, 98
Djedkara (5th Dynasty) 102, 105
Djefahapy, mayor 164
Djehuty (royal butler) 230, 231, 234,
240
Djer (ist Dynasty) 9, 67, 68, 70, 74
Djet (ist Dynasty) 5, 67
Djoser (3rd Dynasty), Step Pyramid of
69, 71, 76, 79, 81, 85-6, 97, 311
dogs 52, 68
Doloaspis (administrator) 389
domestication of animals 28-9, 30,
32,36
donation stelae 338, 341, 344
Donation Stele, Karnak 210, 219
donkeys 55, 178
Dothan, Moshe 322
Dra Abu el-Naga 211-12, 230
Dreyer, Gunter 60
dwarfs 68
Dynasty o 58, 59, 60, 62, 74, 75
Early Dynastic Period 4, 13,
borders 311
cult centres 76-8
human sacrifices 50, 68
and Old Kingdom 83
J 4>57
500 INDEX
Second Dynasty 78-81
trade 314—15 see also First Dynasty;
Second Dynasty
Early Pleistocene Period 17
Eastern Desert
Early Dynastic Period 32, 40, 58
Eighteenth Dynasty 230
Old Kingdom 105
Ramessid Period 287
Roman Period 418-19, 422, 424
Ebla 106
economy
Badarian 39
First Intermediate Period 119
imperialism and 318, 319, 320
Libyan period 343-5
Middle Kingdom 157
Naqada 49, 52
New Kingdom 230, 242, 253, 265,
298
Old Kingdom 84-5, 93-5, 100-1
Ptolemaic Period 396, 404-5
Ptolemaic temples 406-7
Roman Period 419-28
Saite Dynasty 368
Third Intermediate Period 324
use of writing 75 see also
administration
Edfu 121-2, 191, 251, 406
Egyptian Independence (404-343 BC)
377-82
Egyptianization 334, 341, 376, 409
Eighteenth Dynasty 182, 184, 185,
200-1, 207-65
Abydos 206, 211-12, 214, 222
border expansion 311
co-regencies 213, 234, 241, 247, 255
cult centres 204, 209, 213, 222,
230
diplomacy 208, 235, 240, 250,
260-1, 265
religion 226
royal women 216—20, 246
state administration in the 261-4
Ekwesh 322
El-Hiba 342
el-Kurru, tombs at 346, 349
el-Omari culture 36, 53
el-Tarif (necropolis) 124, 125
Elephantine 10, 64, 88
astronomical observations 216
Early Dynastic Period 77, 78
First Intermediate Period 126-7
fortress at 140
New Kingdom 227, 236
Second Intermediate Period
194-6
elephants 223, 225
war 394-5
Eleventh Dynasty 137, 139-45
monuments 126-8, 142-4
elite
burials 57, 59, 60, 62, 70-3, 211,
357>36o
First Intermediate Period 111-12,
123
Graeco-Macedonian 405-6
A Group 315
Naqada 51, 52
New Kingdom 261-2
newly created 21 6
Old Kingdom 84
Elkab 31-2, 186, 187, 191, 198
Emery, Bryan 66, 81, 309
Empereur, Jean-Yves 433
Ephesus, battle of 398
Epipalaeolithic period 31-2
Eratosthenes of Gyrene 400
Esarhaddon, king of Assyria 352-3
estates 95, 164
Ethiopia 308
ethnic identity 309, 339, 350
I N D E X 501
Euphrates 372
Eusebius 5, 179, 424
Evagoras (Cyriot rebel) 382
execration texts 155
faience 77
Naqada 1 48
Roman Period 433-4
Third Intermediate Period 362
Faiyum B culture 32, 33, 44, 53
Faiyum irrigation scheme 152-4, 157
Fakhurian group 24
family
ancestor worship 7, 226
burials 115-16, 212, 357, 383
life in Middle Kingdom 150
famine 119, 332
Middle Kingdom 150-1
and migration 321
Ramessid Period 295
Famine Stele 85
fauna 19, 25, 27, 28-9, 32, 46, 49, 55
Federn, Walter 43-4
festival calendars 215, 216
Festival of the Divine Audience 327
Fifteenth Dynasty 179
Fifth Dynasty 99-101, 121
Palermo Stone from 4-5
religion 98-9
statutues of captive foreigners 310
figurines
Amratian 46-7
Badarian 38
Early Dynastic Period 76, 77
Middle Kingdom 159, 170
Neolithic 35
Third Intermediate Period 362
firesticks 166
First Dynasty 12, 61
Abydos royal burials 57, 64
cult centres 76-8
Early 63-78
First Intermediate Period 8, 64,
134-6, 175
capital and provinces 111-13
chronology 109
cultural and social development
113-15
Herakleopolitan Dynasty 128-34
pottery types 14, 113-14, 133
regional style 116-17
religion 115-16, 121-2, 169
society and government 117-20
Theban Dynasty 123-8, 133-4
fish traps 27
fishing 24, 25, 26, 27, 169
Badarian 39, 40
Epipalaeolithic 31-2
Neolithic 33, 34, 36
Flaccus, Aulus Evilius 430
flake technology 20-1, 22, 26, 33, 34
flora 24, 166
flour mills 434
fortifications
Avaris 180
Late Period 380-1
Libyan period 342
Middle Kingdom 140, 146, 148,
149,154-5,175,313,318
New Kingdom 260
Roman Period 417, 418, 419
Second Intermediate Period 195-6
Third Intermediate Period 328
Fourteenth Dynasty 177, 178, 179
Fourth Dynasty 87-92, 93
Franke, Detlef 10, 163
frescos (Eighteenth Dynasty) 204-5,
208
funerary cults 82
Early Dynastic Period 69-70
early writing and 76
502 INDEX
Fourth Dynasty 87, 89, 95-8
high officials 72, 164
integrated with Osiris 279
Kamose 199
Late Period 384
Second Intermediate Period 187
Twelfth Dynasty 156, 165
funerary objects see grave goods
Gabolde, Luc 220
Galatians 396
Gaza 237
Gaza, battle of (312 BC) 393, 394, 395
Gebel Barkal 346, 347, 350, 352, 353
Gebel el-Arak 46
Gebel Sheikh Suliman (rock carving)
73
Gerzean phase (Naqada II) 43,
49-53
Gisr el-Mudir, North Saqqara 85
Giza
Fourth Dynasty 90-1
mastaba-tombs 100
New Kingdom 222, 243, 248-9,
251 see also Great Sphinx;
pyramids
glass 239-40, 434
gods
during Amarna Period 270
foreign 240, 265, 292
hellenization of 429
local 121-2, 278
polytheism 428-30
gold
Eighteenth Dynasty 230, 251
Middle Kingdom 148, 149
Mitanni 240
Predynastic Period 51,58
Ramessid Period 287
Roman Period 422
Second Intermediate Period 195,
197
Third Intermediate Period 344
Golden Horus name 378
grain 420-1
grave goods 35
Badarian 37
Early Dynastic Period 68, 69
Eighteenth Dynasty 210
First Dynasty 71
First Intermediate Period 113, 114
A Group 314-5
Late Period 385
Libyan period 344-5
Middle Kingdom 140
Naqada45~9, 5 0 " 1 ' 57> 60, 62, 66
Old Kingdom 114
Saite Dynasty 371
Second Dynasty 80
Second Intermediate Period 187,
194
Third Intermediate Period 357-8
Great Edict 284-5
Great Harris Papyrus 296, 298, 299,
322
Great Hymn to the Aten 276, 277
Great Sphinx 90, 222, 243, 247-8
Greece 154, 295, 376, 377, 381, 391,
392,401,429
Late Period 368-9
Greek mercenaries 365, 366, 367, 380
Green, F. W. 61
Grenfell, Bernard and Hunt, Arthur
4 J 5
Gyges of Lydia 354
Gynaecological Papyrus 166
Hadrian, Emperor 415
Hakor (29th Dynasty) 377, 378, 380
Halfan artefacts 20
INDEX
5°3
handaxes, Acheulean 17-18
Hardjedef (son of Khufu) 90, 98
Harisiese (high priest of Amun) 330,
336,357
Haronnophris (Thebanking) 411
Harvey, Stephen 201, 211
Hathor
Middle Kingdom 140
New Kingdom 220, 230, 233
Ptolemaic Period 406
Roman Period 429-30
Third Intermediate Period 354
Hatshepsut, Queen 10-11, 188, 203,
221, 225, 226-7, 2 4-6
building projects 229-34
dishonouring of 237, 241, 242
election by Amun oracle 285
Punt expedition 234, 317
regency of 228-9
Harti 245
Hattusili III, king of the Hittites
290
Hawara, tomb of Amenemhat III
i 5 8
Hebenu 230
'Hekanakhte papers' 150-1
Hekarnehhe (royal nurse) 253
Heliopolis
Amarna Period 278
Middle Kingdom 149
New Kingdom 236, 248, 257
Old Kingdom 86
Ramessid Period 286, 295, 298
H el wan site 81
Hemen, god 122
Hemiunu (vizier to Khufu) 89
Hendrickx, Stan 44
Henenu (royal steward) 141-2, 145
Hepu (vizier) 263-4
Heqaib 194-5
Heracles 403
Herakleopolis Magna 108, 111-12,
128, 333, 347, 383
Herakleopolitan Dynasty 128-33
Thebans and 109, 131-2, 133-4,
139-40
Herihor, General 327, 340
Hermopolis 230, 347
Herodotus 89, 151, 154, 156, 158, 354,
366,367,368,370,372,374
Herodotus Book II 8
Heroninos archive 421
Herophilus of Chalcedon 400
Hesyra (official of Djoser) 98
Hetep-Senusret (Lahun) 165-6
Hetepdief, priest 79
Hetepheres, Queen 89
Hetepsekhemwy (2nd Dynasty) 78,
79
Hetepy (high official) 126
Hezekiah of Judah 352
Hierakonpolis 4
Early Dynastic Period 6 1, 63, 65,
77-8
elite tombs at 60
FortCemetry72, 79
Naqada49, 52
Predynastic Period 38, 39, 45
hieratic script 98, 186, 193, 339
hieroglyphs 193
earliest known 60
Early Dynastic Period 76
Old Kingdom 98
Predynastic period 74
Ptolemaic 60, 408
Third Intermediate Period 339, 346
high priests 237, 261, 300, 301
chief generals as 327-8, 329
hereditary 330
Ptolemaic Period 407
Third Intermediate Period 336-7,
339,340,348
5°4
INDEX
Hittites 321
and New Kingdom 270, 282,
283-4
and Ramessid Period 287,
289-90, 292, 295
Hoffman, Michael 52
Holocene period 27, 32
Homo erectus 17
Homo sapiens 22
Horemakhet, cult of 243, 249
Horemheb (i8th Dynasty)
as regent 244, 263
reign of 284-6
Horemkhauef (overseer of priests)
186,188
Hornung, Erik 8, 262
horses 202, 225
burial 347
see also cavalry
Horus 4, 6, 140, 141, 356, 378, 429
cult centre 60
of Hutnesu 285, 286
and kingship 7, 8, 90-1, 125, 222
politics and 121-2
temple of 6 1
see also Isis; Osiris
Horus Qahedjet 88
Horus the Behdetite temple, Edfu
406
houses
Amarna Period 274
Badarian 39-40
Early Neolithic 29
Maadian 53
Naqada 49, 52
Neolithic 34, 35
New Kingdom 213
Puntite 317
Second Intermediate Period 177
human sacrifice 50
Huni (3rd Dynasty) 88, 90
Hunter's Palette 46
hunting 169
Amratian 46
Epipalaeolithic 32
kill-butchery camp sites 25
Middle Palaeolithic period 19
Naqada period 52
Neolithic 29, 31, 34
Hyksos 157
Avaris culture 174-82
chronology 176-7, 178-80
Memphis 182-7
Second Intermediate Period 319
and Theban war 197-203
Hymn to Hapy 171
Hymns to Senusret III 164
I ah (moon-god) 209
Ihnasya el-Medina cemetry 134
Iliad 242
Imhotep (architect) 86-7
imperialism, of Middle and New
Kingdoms 317-20
incense 145, 317
India 420, 425
Ineni (high official of Thutmose I)
225,228
Intef I (nth Dynasty) 123-5
Intef II (nth Dynasty) 125-6, 133-4
Intef III (nth Dynasty) 139
Intef VI (i7th Dynasty) 192, 193, 218
Intef VII (i7th Dynasty) 192, 193
Inten, ruler of Byblos 160
Ionian Revolt 376
Iput, Queen 103, 104
Iran 63, 372
Iraq 239, 372
Iri-Hor (Dynasty o) 67
irrigation 131
Early Dynastic period 65
INDEX
5°5
Faiyum scheme 152-4, 157
Ptolemaic Period 410
Isetnefret (wife of Rameses II) 291
Isis (goddess) 356
Late Period 370, 378
New Kingdom 259, 278
Ptolemaic Period 406, 430-1
Roman Period 430
Islamic period 13
Isnan industry 26
Israel 28, 33, 294, 327, 418
Itjtawy 146-7, 159, 177, 185, 187
luput I (23rd Dynasty) 339
lykhernefert Neferhotep see
Neferhotep
lykhernefertneferhotep (i3th
Dynasty) 160
Jeffreys, David 64
Jeroboam 329-30
jewellery 154, 313
Jewish mercenaries 376
Johnson, Janet 6
Johnson, W. Raymond 254-5
Joseph 155
Josephus, Flavius 5, 179, 182,
435
Josiah of Judah 372
.257
2OI-2,
ka-chapels 77
Ka (Dynasty o) 67
Kadesh see Qadesh
Kaiser, Werner 44, 57, 67
Kamose (i7th Dynasty) 172, 183, 188,
193, 195-6, 197, 199-200, 218
Karimala, Queen 346
Karnak temple complex 136
and Ahmose 209-10, 214
and Amenhotep 1215
and Amenhotep II 242, 243-4, 245
and Amenhotep III 256, 258
and Amenhotep I V/Akhenaten
267-9
First Intermediate Period 126
and Hatshepsut 230-1, 234
Middle Kingdom 144, 169
and Sety I 287
Sheshonq I 330
and Thutmose I 223
andThutmose II 226-7
and Thutmose III 236-7, 245
and Thutmose IV 249, 250
Kashta, king of Kush 346-7, 350
Kawa (son of Khufu) 90
Keminebu, Queen 152
Kemp, Barry 257, 318
Kenamun (steward) 262-3
Kenemun, tomb of 239
Kenisa 222
Kerma (capital of Kush) 196-7,
198-9, 203, 260, 308, 315
Khaba (3rd Dynasty) 87
Khababash rebellion 382
Khaemwaset (son of Rameses II)
293-4
Khafra (4th Dynasty) 90, 97
Khakaura Senusret III see Senusret
III
Khakheperra Senusret II see Senusret
II 152-4
Khamudi (i5th Dynasty) 180, 200
Kharga Oasis 18, 106
Khasekhemra Neferhotep I see
Neferhotep I
Khasekhemwy (2nd Dynasty) 69, 70,
74, 77, 78, 79-80, 83
Khenti-Amentiu, god 77
Khentkawes, Queen 91, 99-100
Khentykhetaywer (king's official) 152
Kheperkara Senusret I see Senusret I
506 INDEX
Khety, House of 128, 139
Khety (vizier) 139, 141
Khnum, god 85, 88, 127, 376
Khnurnetneferhedjetweret, Queen
J 53
Khufu (Cheops) (4th Dynasty)
88-90, 93
Khuy, King 133
Kiman Faras temple 156
king-lists 4, 5-7, 57, 123, 149, 192,
287, 324
see also Turin Canon
kingship 4
and afterlife 67, 92-3, 371
and association with royal females
251-2
and divinity 7-8, 222, 231, 369
Early Dynastic Period 65
emulation of 215
expressed in tomb reliefs 98
and external policy 97
in First Intermediate Period 121
ideology of 66-7, 75
Kushites 348-9, 351
Late Period 378-9, 386
mammisiac religion and 356
Middle Kingdom 164-5
Ptolemaic Period 401, 402-3
royal succession 91, 349
smiting pharaoh iconography 310,
340
and solar theology 251, 266-7
subordination to Amun 304-7,
339-40
in Third Intermediate Period 121,
339-42, 354
titles 6, 7, 90, 91, 104, 125
veneration of 236
Kiya (wife of Amenhotep
IV/Akhenaten) 271
Kleomenes of Naukratis 389
knapping 18, 20, 26
Kom Dara (mastaba-tomb) 131-2
Kom el-Shugafa catacombs 432
Konosso Stele 251
Koptos colossi 76-7
Krauss, Rolf 10
Kubbaniyan group 24-5
Kush (Upper Nubia), kingdom of
149,152,188,195,308
and Assyria 352-4
conquest of Egypt (24th Dynasty)
347-5 2
New Kingdom 214, 223, 227, 260
Second Intermediate Period 196-7
Third Intermediate Period 331-2,
3 6i
labels, funerary 5, 14, 75, 76
Lacovara, Peter 198
Lahun 154, 165-6, 167
Lahun letter 9, 139
land
allocation 404-5
endowments 99, 101, 298, 344,
404, 409
reclamation 152-3, 209
language
Amarna Period 276-7, 292
Aramaic 375
Early First Dynasty 64
ethnic identity and 309, 339
Ptolemaic Period 407-8, 414
lapis lazuli 313
Late Palaeolithic period 23-7
Late Period 12
Egyptian Independence (404-343
BC) 377-82
First Persian Occupation 374-7
religion 378-9, 383-4
Saite Dynasty 364-74
INDEX 507
Second Persian Occupation 382
sources 5-6, 364, 365, 374, 377-8,
380
Lebanon 58, 74, 81, 141, 238, 320, 393
legal system 284-5, 375
Legend ofHorus ofBehdet and the
Winged Disc, The 407
Lehner, Mark 243
Leontopolis 337
letters and documents see papyri
Leugas, Gaius Cominius 424
Levallois method 18, 19, 20, 21
Levant 33, 235,308
Amenhotep II in the 244-6
Middle Kingdom 318
New Kingdom 319
Saite Dynasty 372
Third Intermediate Period 329,
33 2 >35 2
Thutmose III and 237-41
Thutmose IV and 250
timber 58
libraries 400
Libu 332, 338
Libya and Libyans 15, 148, 287, 294,
295, 297, 308, 322, 328-9,
333' 3 6 5> 374> 3 8 °
Libyan period (2ist to 24th
Dynasties) 332-45
life expectancy 43 5
Lisht 137, 138, 172, 185-6, 188
Litany ofRa 237
literacy 151, 171
literature
and Amarna Period 276-7
Ankhtifi's inscription 118-23
First Intermediate Period 134-6
Herakleopolitan or First
Intermediate Period 129, 134
Middle Kingdom 134-6, 146, 148,
149, 164-5, 17 1
Old Kingdom 90, 98
Ptolemaic Period 400
Ramessid Period 292
lithics 1 8, 24, 26
Badarian 38, 39
Faiyum or Neolithic 33
Neolithic 28, 30, 34, 36
Lloyd, Alan 8
Lovell, Nancy 68
Lower Egypt 62
Early Dynastic Period 72-3
graves in 56
Hyksos 157, 174-206
kings of 4
Late Period 366
Libyan period 333, 337
military colonies 295
Naqada culture in 64
Neolithic sites 34-6
New Kingdom 215
Sea Peoples 297, 322
Tasian culture 37
Third Intermediate Period 328
Lower Nubia see Nubia
Lower Palaeolithic 17-18
Lukka 322
Luxor 250, 252
Amenhotep III 258
Hatshepsut 231
Opet Festival 267
Rameses II 291
luxury goods 63, 69, 101, 106, 231,
234,238,239-40,313,317,
425
Lysimachos 389, 390
Maadian Complex 53-6, 59, 6 1
Maakherura Amenemhat IV see
Amenemhat IV
Maat 257, 259, 269, 278, 340, 384
508 INDEX
Macedonian Empire 388-9, 390-1,
392,393,402,412
maceheads 14-15
Amratian 45
ceremonial 3
Maadian 54
Naqada48, 51
Narmer 4, 61, 75, 77
Scorpion 61, 77, 311
machimoi (warrior class) 366, 367,
3 8o >395> 39 6 >409
Mahasun 338
Mahgar Dendera (Badarian
settlement) 40
Maksoud, Mohammed 175
Malkata temple complex 257
mammisiac religion 356
mammisis (peripteral temples) 406
Manetho, Aegyptiaca i, 4, 5, 6, 7,
11-12, 84, 88, 98, 103, 104,
io7' *54
First Intermediate Period 108, 109
Herakleopolitan Dynasty 128
king-lists 7
Middle Kingdom history 148
Second Intermediate Period 173,
177, 179
Third Intermediate Period 325, 331
Manetho of Sebennytus 407
Mark Antony 413, 414
marriage 216-20
brother-sister 403-4, 435
diplomatic 240, 260, 290-1, 327,
374
within families 328, 337
father-daughter 252, 259-60
officials and royal family 216-17,
329,338
Roman Period 435-6
Martin, Geoffrey and Raven, Martin
212
Masaharta, General 342
Masara culture 30
mass consumption, First
Intermediate Period 114, 115
mastaba-tombs 80-1
First Intermediate Period 114, 115,
116, 130, 132-3
Middle Kingdom 154, 156, 170
Old Kingdom 85-6, 89, 91, 93,
IOO, IOI
Third Intermediate Period 346
Mastabat el-Fara e un tomb 91
mathematics 181
Maya (treasurer to Tutankhamun)
283
mayors 193, 194, 263
medical instruments 166
Medinet Habu 15, 226, 237, 298,
302, 303, 343, 357
megaliths 30
Megiddo, battle of 237, 238
Mehy (military officer) 288
Meidum pyramid 88, 93
Meketra (high official) 147
Memphis 64, 72, no, 130
captured by Assyrians 353
chief royal residence of Kushite
pharaohs 349, 351
Eighteenth Dynasty 208-9, 2 ^6
excavation of 13
Hatshepsut 230
High Priests of 407
Hyksos rule 183
necropolis 279, 292-3, 294
Ramessid Period 292-3
Serapis cult 429
Third Intermediate Period 327, 331,
332, 347, 349
and Tutankhamun 281
Memphite Theology of Creation 349,
35 1
I N D E X 509
Mendes 341
Menes (ist Dynasty) 4, 7, 43, 91, 103
Menkauhor (5th Dynasty) 100, 102
Menkaura (4th Dynasty) 84, 90-1
Menkheperra, General 340, 343
Menkheperraseneb (high priest) 237
Mennefer 104
Mentuemhat, tomb of 385
Mentuhotep, General 149
Mentuhotep, Queen 193
Mentuhotep I (nth Dynasty) 123-4
Mentuhotep II (nth Dynasty) 109,
122, 125, 131, 134, 139-44, 162,
167, 232
mercenaries 190
First Persian Period 376
Late Period 365, 366-8, 380
Libyan 328, 333
Nubian 106, 120
Ptolemaic Period 395-6
Sea Peoples 322
Merdjedefra (i4th Dynasty) 178
Merenptah (i9th Dynasty) 198,
294-5, 321, 322
Merenra (6th Dynasty) 1 06, 107
Mereye, king of Libya 295
Merimda Beni Salama (Neolithic
site) 34-6, 53
Meritaten (daughter of Nefertiti)
269, 271
Merneferra Ay (i3th Dynasty) 186,
i8 7
Merneith, Queen 67
Mertiotes, Queen 90
Meru (chancellor) 141
Merykara (loth Dynasty) 128, 140
Merymose, viceroy of Kush 260
Merytamun, Princess 218, 219-20
Merytra (mother of Amenhotep II)
241, 244, 246
Meshesher 15
Meshwesh 332, 338, 341
Mesopotamia 62, 65, 74, 78, 390
metalwork 48, 51, 55, 361-2
Met jen (4th- Dynasty official) 95
microlithics 32, 33
Middle Bronze Age 176, 177, 180,
184, 202
Middle Kingdom 2, n, 109, 137-71
Abydos tombs 67
border fortifications 313
Coffin Texts 115
culture 171
and First Intermediate Period 136
fivefold titulary 6
imperialism 317-18
literature 134-6
political system 161-4
references to Asiatics 175
religion 168-70
royal cementries 142-3
Royal courts 164-5
trade and commerce 149, 151-2,
160,161,166-8,368
urban life 1 6 5-6
Middle Palaeolithic period 18-22
Middle Pleistocene 17
migration 16—17, 27
Asiatic 175
Early Dynastic 73
Libyans 332-3
Naqada43, 5&~9
Ramessid Period 295
Sea Peoples 321, 322
Sixth Dynasty 106
Miletus 376
military decorations 240
Millet, Nick 3-4
minerals 58, 147, 419, 422
mining 22, 55, 156-7, 194
Minmose (overseer of works) 236
Minshat Abu Omar 58, 72
5io INDEX
Mit Rahina annals 151, 152
Mitanni 239, 240
Amarna Period 270, 282
and Amenhotep II 245, 265
and Thutmose I 223, 224, 225
andThutmose III 237-8, 240
and Thutmose IV 250-1
mobiliaryart3, 14
monasticism 431, 436
Mond, Robert 44
monotheism 266, 428-9
Mons Claudianus 415, 418, 423, 424,
425
Mons Porphyrites 423-4
Montet, Pierre 167
Montu, temple to 122, 244
Morris, Sara 242
mortuary cults see funerary cults
Mostagedda cemetery 189-90
Mouseion 400
mummies and mummification
First Intermediate Period 114, 115,
129
Middle Kingdom 170
Naqada II 50
New Kingdom 211, 212, 303, 307,
345
Roman Period 431-2
Second Dynasty 81
Twenty-first Dynasty 358-9
Munich papyrus 263
murder 148, 284, 299, 410-11
Murnane, William n, 264
Mursili III, king of the Hittites 290
Mut (mother-goddess) 231, 252, 258,
354
Mutemwiya (mother of Amenhotep
111)252,253,258
Mutnefret (wife of Thutmose I) 221,
227-8
Muttuy (daughter of Sennefer) 263
Muwatalli, king of the Hittites 284,
289-90
Myers, Oliver 44
Myos Hormos 426, 428
myrhhtrees3i7
mystery plays 406
mythology 79
Nabta Playa (Neolithic) 28, 29, 30
Nag Ahmed el- Khalifa, near Abydos
J 7
Nahrin 238-9, 240, 245, 250
Nakhtmin (army commander) 284
Napata region 222, 227, 245, 346
Naqada culture 36, 37, 38, 39, 43, 56,
57
and A-Group culture 63
burials 60
cemetries 81
chronology 42-5
contact with Palestine 61-2
formal art styles 66
northward expansion of 58-9, 64
phases of 43, 45-53
pottery 13
writing 74
Naqada ('gold town') 52, 58
Narmer (Dynasty o) 67
Narmer Macehead 6 1, 75, 77
Narmer Palette 6 1, 66, 75, 77
Naukratis stele 379
Naukratis (trading centre) 368
navy
Late Period 372, 376, 379, 381-2
Ptolemaic Period 393, 397-8
Second Intermediate Period 201
Nazlet Khater site 20, 22
Nazlet Safaha 20
Nebamun (standard-bearer) 250
Nebererau I (i7th Dynasty) 192
I N D E X
Nebetta (royal wife) 246
Nebhepetra Mentuhotop II see
Mentuhotep II
Nebka (3rd Dynasty) 84, 85, 87
Nebsenra (i4th Dynasty) 178
Nebtawyra Mentuhotep IV (nth
Dynasty) 145
Nectanebo I (3oth Dynasty) 6, 377,
380,381
Nectanebo II (3Oth Dynasty) 378,
380, 381,406
Neferhotep I (i3th Dynasty) 160, 168,
195
Neferhotep (i3th Dynasty) 191
Neferirkara (5th Dynasty) 84,
99-100
Nefermaat, Prince 89
Nefertari (wife of Rameses II) 291
Nefertiry 252
Nefertiti (wife of Amenhotep
IV/Akhenaten) 268-9, 271"2
Neferu, Queen 143
Neferuptah, Princess 158
Nefret, Queen 153
Nefrubity, Princess 221
Nefrura, Princess 228, 229
Nefrusi, sack of 199
Negev 21
Nehesy (i4th Dynasty) 177-9
Nehesy (treasurer) 233
Nehmetawy 379
Nehry, Count 162-3
Nehy (viceroy) 234
Neith, temple of 370, 371
Neith (goddess) 76, 279, 379
Nekau, king of Sais 353, 365
Nekau II (26th Dynasty) 369, 370,
372, 373
Neolithic period 3
Nile Valley 33-6, 53
Western Desert 27-31
Nepherites I (2 9th Dynasty) 378, 381
Nero, Emperor 424, 430
nesu-bit titles 7, 104, 125
New Kingdom n, 12, 335
autobiographical tomb texts 15
beginning of the 207-10
cemeteries 187
co-regencies in 10
imperialism 319-20
religion 266-7, 34^
royal women 216-20 see Amarna
Period; Eighteenth Dynasty;
Nineteenth Dynasty;
Twentieth Dynasty
trade with Punt 317
Newberry, Percy 166
Nile
cataracts 311-12
cultural development and 52
Epipalaeolithic 31-2
floodplain 40, 63
Hyksos control of 183
inundation 5, 28, 103, 119, 157, 420
mythical status 51
Neolithic period 33-6
prehistory of 1 6
Wild Nile stage 25-6
see also Delta
Nimaathap, Queen 83
Nimaatra Amenemhat III see
Amenemhat III
Nimlot, ruler of Hermopolis 347
'Nine Bows' 309-10, 311
Nineteenth Dynasty 285-96, 321
Nitiqret, Queen 107
Nitiqret Adoption Stele 369-70,
373
Nitiqret (daughter of Psamtek) 366
Niy225
nobles see aristocracy; elite
nomads 31, 32, 107, 315, 332
5 1 1
512 INDEX
nomarchs 108, 117, 118-20, 123, 155,
191
Middle Kingdom 141, 162-4,
169-70
New Kingdom 230
nomes 416-17
North Saqqara 64, 70-3
Northern cultures 53-6
Nubia 18, 19, 20, 23, 63
Early Dynastic Period 73
Egyptian imperialism and 317-18
Kush 196
Late Period 373
mercenaries 120
Middle Kingdom 140, 148, 149,
152, 154-5, 160-1
New Kingdom 207-8, 213-14,
229,234, 251,320
Amenhotep III 260
Thutmose I 221, 223
Old Kingdom 97
Ramessid Period
Rameses II 289-90
Rameses XI 302
Setyl287
Second Intermediate Period 194
Sixth Dynasty 106
Third Intermediate Period 325,
331-2
Twenty-fifth Dynasty 346
see also Kush, kingdom of
Nubkaura Amenemhat II see
Amenemhat II
Nubkheperra Intef VII see Intef VII
Nynetjer (2nd Dynasty) 79
Nyuserra (5th Dynasty) 99, 100
Obelisk Temple 320
Obsomer, Claude 149
O'Connor, David 66, 69, 70, 78, 257
Octavian see Augustus, Emperor
officials
carrying weapons 140
corrupt 412
court 165
Eighteenth Dynasty 261-4
hereditary 383
marriage 216-17, 3 2 9
Middle Kingdom 141, 162
Old Kingdom 95
Ptolemaic Period 404, 405, 412
Ramessid Period 292
satraps 375
seals of 74
Second Intermediate Period 191-2
Theban 126
Third Intermediate Period 329,
335-8, 355
tombs of 64, 70-3, 80, 211, 309
oils 69
Old Kingdom 4, n, 81, 294
chronology 5, 84-5
compared with First Intermediate
Period 112, 114, 130
decline of 1 06-7
economy and administration 93-5
kingship and afterlife 92-3
large-scale building projects
85-7
pottery types from 14
Pyramid Texts 102-3
Royal Funerary cults 95-8
state administration 117
sun-temples 98-9
Old Testament 329-30
Opening of the Mouth funerary
ceremony 48
Opet Festival 231, 233, 258, 267, 285,
299
oracles 6, 285, 306-7, 327, 329
oral tradition 334
Osiris 7, 67, 143
Abydos 211, 222, 287, 291
Amarna Period 277
Middle Kingdom 142, 168
New Kingdom 279-81
Pyramid Texts 102-3
and Ra 266
Saite Dynasty 370, 371
and Senusret 150
Third Intermediate Period 356
Osorkon, Prince 342, 343
Osorkon I (22nd Dynasty) 329, 342,
344
Osorkon II (22nd Dynasty) 330, 340
Osorkon the Elder (2ist Dynasty)
328,329,333
ostraca 415, 418-19, 423
overseer of priests 122, 123, 131, 134,
186, 230
Oxyrhynchus papyri 415, 433, 434
Painted Tomb 50
Pakhet 230
palaces 83, 91-2
Amarna Period 275
Eighteenth Dynasty 208
Middle Kingdom 165
Ramessid Period 292
Second Intermediate Period 198
White Wall 104
Palermo Stone 4-5, 57, 85, 96
Palestine 197, 203
border with Egypt 313
Early Dynastic Period 73
Late Period 372
and Maadian culture 53, 54, 55
Middle Kingdom 155, 318
New Kingdom 235, 245, 250-1
Predynastic Period 61-2
Ramessid Period 294
Third Intermediate Period 330, 352
trade 60, 314
palettes 14, 38, 54
ceremonial 60-1
Naqada 41
Naqada I 48
Naqada II 51
Narmer 4, 66, 75, 77
Neolithic 30, 34
Two Dog 77
votive 3
Pan, god 42 9
pan-grave people 189-90, 193, 308
Panehsy, and civil war 302, 325
Paneion 430
Panion, battle of (200 BC) 393, 395
papyri 434
Bulaq 18 165
Ebers 9
'Hekanakhte papers' 150-1
Heroninos archive 421
Lahun papyri 9, 139
Libyan period 342
Old Kingdom 100
Oxyrhynchus 415, 433, 434
Pyramid Texts 102-3, 115, 386
Sallier 1 173
Semna dispatches 155
Westcar 98, 99, 100, 171
Paramessu see Rameses I
Parthian War 418
pastoralism 31, 315, 332
peasants 410
Pedubast, king of Tanis 348
Pedubastis I (23rd Dynasty) 330-1,
339
Peftjauawybast, ruler of
Herakleopolis 347
Peleset (Philistines) 322
Pelusium, battle of 374
Pepy I (6th Dynasty) 104-5, IO &> 294
INDEX
5 J 3
Pepy II (6th Dynasty) 77, 106, 107,
no
Perdiccas 389
Pergamum, kingdom of 412
Peribsen (2nd Dynasty) 78, 79
Periplus Maris Eiythraei 426
Persia 373, 379, 381, 387
occupation of Egypt 374-7, 382
Petosiris, tomb of 383, 384, 386
Petrie, Flinders 2, 3, 45, 52, 69
Abydos 67, 68, 73, 80
Hawara 158
Koptos 76
Lahun 154, 165, 166
Naqada culture 41, 42-4, 45, 50,
52,58
Tarkhan 72
Tell Defenna 367
Pharos lighthouse, Alexandria 400
Philadelphus 398, 405
Philae4o6,43O-i
Philip II 391, 394, 398, 402
Phoenicia 373, 376, 380, 392-3
Piankh, General 302, 303, 325, 327,
342
pilgrims 243
Pinudjem I (2ist Dynasty) 340, 343,
345
Piramesse 292, 298, 303, 326, 343
Piy (25th Dynasty) 331, 343, 346, 347,
348, 349
Pliny 158
Plutarch 377-8, 380
police
border 106
Roman Period 418-19
political system
First Intermediate Period 117,
121-3
Middle Kingdom 155, 161-4
Third Intermediate Period 324,
328,329,330,335-8,361,
362-3
political unification 59, 64
Polybius 392, 396, 410, 412
polygamy 404
polytheism 428-30
Polz, Daniel 211
Popilius Laenas, C. 412
porphyrites 423-4
pottery
Abydos Ware 73
Amratian 45-6
Badarian 37-8, 39
First Intermediate Period 13-14,
113-14, 133
Herakleopolitan era 130
Kerma ware 196, 198
Maadian 53-4, 56
Middle Kingdom 166-7
Naqada 13, 41, 42-4, 50-1, 62
Neolithic 29, 30, 31, 33, 34, 35, 36
pan-grave 190
Roman Period 432-3
trade in 314
power
iconography 47-8, 51, 75-6
and royal mortuary cults 69-70
Predynastic Period 2, 14
artefacts 3, 13
trade 314-15
Upper Egypt 57
priestesses 210, 215, 219, 251
see also Amun, and 'god's wife'
priests 300
of Amun 237
funerary cults 89
genealogy 181
nominal 100-1
oracles and 306-7
overseer of 122, 123, 131, 134, 186,
230
514
INDEX
INDEX
515
provincial 122
Ptolemaic Period 40 6, 407
retired soldiers 286
Third Intermediate Period 354, 355
see also high priests
Prophecy ofNeferti 134, 135, 146, 148
Proto-Moeris Lake 32
provinces
First Intermediate Period 111-13,
119-20
Ptolemaic Period 411
Roman Period 416-17
Psamtek I (26th Dynasty) 322, 339,
348, 353, 365-6, 370, 372
Psamtek II (26th Dynasty) 370, 372,
373
Psamtek III (26th Dynasty) 373, 374
Psusennes I (2ist Dynasty) 327, 328,
340, 342
Psusennes II (2ist Dynasty) 328, 329
Ptah, god
Apis-bull burials 180, 182, 183,
188,198,199
New Kingdom 230, 236, 257, 291,
293,294
Old Kingdom 86, 104
Roman Period 429
Second Intermediate Period 183
Third Intermediate Period 349
Ptahshepses 101
Ptolemaic Period 2, 13, 388-90
absolute chronologies 8
administration 404-5
cities of the 399-400
civil unrest 410-13
imperialism 391-3
kingship 402-4
libraries 400
military might of 393-8
Ptolemaieia festival 401
religion 429
Rome's interventionism 412-13
society 405-10
sources 392, 395, 396, 397, 398,
400,401,410,412
Wars of the Successors 389-90
Ptolemy, general 13
Ptolemy, son of Lagus 400, 402
Ptolemy (geographer) 426
Ptolemy I 389, 393, 395, 400
Ptolemy II 393, 401, 403, 425
Ptolemy II Philadephus 425
Ptolemy IV 394, 395, 398, 410
Ptolemy VI 393, 411, 413
Ptolemy VIII 411, 413
Ptolemy X Alexander I 411
Ptolemy XI Alexander II 413
Ptolemy XII Neos Dionysos 413
Ptolemy XIII 413
Ptolemy XIV 413
Ptolomy V Epiphanes 85
Punt, kingdom of 316-17
Middle Kingdom 145, 152, 168
New Kingdom 231, 233, 234, 235,
298
Old Kingdom 101, 105
Pusch, Edgar 257
Pyramid Texts 102-3, 115, 386
pyramids
construction 93-4
Fifth Dynasty 99, 102-3
Fourth Dynasty 87-92, 93
Great Pyramid at Giza 88-90, 93
Kushite 349, 350
Middle Kingdom 156, 271
Old Kingdom 371
proto 69
restoration of 294
Seventeenth Dynasty 211
Seventh/ Eighth Dynasty 107
Sixth Dynasty 104
as solar symbol 280
5 i6
INDEX
Step Pyramid of Djoser 69, 71, 76,
79, 81, 85-6, 97, 311
temples 87, 97-8, 100, 105
Twelfth Dynasty 152, 154, 157-8
Qa e a (ist Dynasty) 67, 69,78
Qadan industry 26
Qadesh (Kadesh)
battle of 289-90,306,322
ruler of 237, 238
Qakara Iby (7th/8th Dynasty) 107
Qarunian culture 32
Qena region 17,19, 22
quarries 96
Quibell, J. E. 4,45, 52, 61, 80, 81
Quirke, Stephen 159,160
Ra, god
included in royal titles 90, 91, 92
New Kingdom 257, 259, 266
Old Kingdom 86, 87, 98-9, 103,
104
Roman Period 428
see also Osiris
Ra-Horakhty 233, 242, 244, 256, 266
racial groups in Egypt 309, 324, 334
radiocarbon dating 2-3, 12, 34, 36, 44
Rahotep (i7th Dynasty) 192
Rameses I (191!! Dynasty) 285-6
Rameses II (i9th Dynasty) 7, 217,
257, 288-94, 306
and Byblos 321
deification of 255, 293
Rameses III (2oth Dynasty) 297-9
final confrontation with Sea
Peoples 322-3
mortuary temple of 15
Rameses IV (2Oth Dynasty) 299-300
Rameses IX (2Oth Dynasty) 301
Rameses V (2oth Dynasty) 300
Rameses VI (2oth Dynasty) 300
Rameses VII (2oth Dynasty) 301
Rameses VIII (2oth Dynasty) 301
Rameses X (2Oth Dynasty) 301
Rameses XI (2oth Dynasty) 301-2
Ramose (son of Amenhotep I) 220
Raneb (2nd Dynasty) 79
Raneferef (5th Dynasty) 100
Raphia, battle of (217 BC) 394, 395,
396,409
rebellions and uprisings 79-80, 161,
203-4, 22 7
First Persian Period 374-5, 376-7
Khababash 382
Levant 244-5
Libyan period 343
mercenaries 367
Nubia 270
in Ramessid Period 296-7, 300-1,
302
Red Pyramid 88
Red Sea 369, 425-6
Red Sea canal 372, 375-6
Redford, Donald 4, 5, n, 75, 191, 254
regicide 411
Reisner, George 72, 198, 314
rekhyt-birds 310-11
relief carvings
Eighteenth Dynasty 201, 206, 215,
244
Middle Kingdom 139, 144-5,
149-50, 160, 169
New Kingdom 258
Old Kingdom 97-8, 99, 100
Ptolemaic Period 406-7
Ramessid Period 288, 322
Third Intermediate Period 355,361
religion
Amarna Period 269-81
ancestor worship 7-8, 226, 243
INDEX
517
Fifth Dynasty 98-9, 101, 102-3
First Intermediate Period 115-16,
121-2, 169
Fourth Dynasty 87
under Kushite rule 349
Late Period 378-9, 383-4
Middle Kingdom 168-70
New Kingdom 266-7
personal piety 305-6
post-Amarna 279-81
Ptolemaic Period 429
Roman Period 428-32
Third Intermediate Period 354-60
remuneration system 100-1, 106-7
Restoration Decree 278
Restoration Stele 282
reunification of Egypt 122, 131, 140,
142, 203-6, 207, 364
Rhind Mathematical Papyrus 173,
181, 200, 205
Rhodes 392
Rhomboidal (or Bent) Pyramid 87
Ribaddi, king of Byblos 321
rock art 26-7, 63
rock carvings 73, 139
rock tombs 114, 118, 124, 131, 140,
170, 279
Roman interventionism (Ptolemaic
Period) 412-13
Roman Period 2, 13
absolute chronologies 8
administration 416-17
army 4 17-1 9
astronomy 9
craftsmanship 432-4
demography 435-6
economy 419-28
religion 428-32
sources 414-15, 417, 418, 421, 422,
424,426,433,434,435
Rosetta Stone 407
royal butlers 230, 231, 234, 292
royal nurses 247, 252, 263
royal titles 7-8, 104, 125
Ryholt, K. S. B. 178, 179, 191
sacrifice
human 50
for royal burials 67-8
Safahan artefacts 20
Saff Dawabai24
so^-tombs 116, 124-5, *4 2
Saharan Neolithic (Ceramic) 27-31
Saharo-Sudanese pottery 54
Sahathor (i3th Dynasty) 160
Sahura (5th Dynasty) 99-100, 101
Sais 322, 331, 337, 338, 339, 370
Saite Dynasty 279, 348, 360, 364-74,
385,386
Salamis, battle of (306 BC) 394, 397
Salitis see Sekerher
Sankhkara Mentuhotep III (nth
Dynasty) 144-5
Saptah (i9th Dynasty) 296
Saqqara 69
Fifth Dynasty pyramids 102-3
Fourth Dynasty pyramids 87-8
funerary labels from 5, 76
New Kingdom 211-12
private tombs at 233
Second Dynasty 79
Second Intermediate Period 184
tombs of high officials 70-3, 80
Saqqara Stele 374
Satamun (daughter of Tiye) 220,
259
Satet, goddess 77, 127
Sathathoriunet, Princess 154
Satire of the Trades 171
Satkamose, princess 218
satraps 375
5 i8
INDEX
Sa e wawporti68
scholarship 400
Schulman, Alan 74
science 400
Scorpion, King 60
Scorpion Macehead 61, 77, 311
scribes 151
Dynasty o 74
Eighteenth Dynasty 264
hieratic script 186
patron of 86
Ptolemaic Period 406, 407, 409
Ramessid Period 292
Second Intermediate Period
180-1, 193
Third Intermediate Period 355
Sea Peoples 295, 297-8, 321-3, 332
sea trade 74, 317
seals 74, 75, 79, 87, 254
Sebilian industry 26
Second Dynasty 78-81
Second Intermediate Period
172-206, 210
chronology 186-7
Theban-Hyksos 197-203
Thebes 191-4
Turin Canon from 7
serf-festivals (royal jubilees) 89, 99
'god's wife of Amun' 355
Middle Kingdom 138, 141, 149
New Kingdom 215, 243-4, 2 54»
267, 268
Rameses II 293, 294
Third Intermediate Period 340
Sehertawy Intef I (son of
Menruhotep I) 124
Sekerher (i5th Dynasty) 179, 180
Sekhemkhet (3rd Dynasty) 86-7
Sekhemra Intef (i7th Dynasty) 193
Sekhemra-khutawy Sobekhotep II see
Sobekhotep II
Sekhemra-sewadjitawy Sobekhotep
III see Sobekhotep III
Sekheperenra (i4th Dynasty) 178
Seleucid empire 390-1, 392, 393,
394' 395' 4 12
Seleucus 389, 390
self-deification 140-1
Sema 400
Semerkhet (ist Dynasty) 67, 69, 74
Senenmut (steward of Hatshepsut)
228-9, 2 3°' 2 3 J > 2 33> 2 ^ 2
Seni (viceroy) 234
Seniseneb, Queen 221
Sennefer (mayor of Thebes) 263
Senusret I (i2th Dynasty) 123, 136,
148-50, 155, 167, 194, 215, 422
Senusret II (i2th Dynasty) n, 138,
i6 3
Senusret III (i2th Dynasty) 9, n, 138,
154-6, 163-4, 168, 185, 194,
318,368
Septimus Severus, Emperor 416
SeqenenraTaa (i7th Dynasty) 192,
193, 197, 198, 199, 212, 218
sequence-dating system 2,3, 42-4
Serapis cult 409, 429
serdab (chamber for statutes of
deceased) 69
serekhs 6, 75, 76, 79
'Sesostris' 151, 154
Seth, god 79, 177, 202
Sethnakht (2Oth Dynasty) 296-7
settlement patterns
Early Dynastic Period 65, 72
First Intermediate Period 113
Naqada culture 52
northern Sinai 61-2
Old Kingdom 94-5
Sety I (i9th Dynasty) 7, 244, 284,
286-8
Sety II (i9th Dynasty) 295-6
INDEX
519
Seventeenth Dynasty 179, 191, 192
royal tombs 193
Shabaqo Stone 351
Shabaqo (25th Dynasty) 331, 347
Shabitqo (25th Dynasty) 349
shabtis 143, 170, 265, 280, 350, 360,
362
Sharuhen 237
Shasu people 227
Sheikh Muftah culture 30-1
Shekelesh 322
Shepenwepet I, Queen 355
Shepseskaf (4th Dynasty) 91, 98
Shepseskara (5th Dynasty) 100
Sherden 322
Sheshonq I (22nd Dynasty) 329-30,
339' 340
Sheshonq III (22nd Dynasty) 330, 338
ships see boats and ships; navy
Shipwrecked Sailor, The 171
Shunet el-Zebib (funerary enclosure)
6 9
Shupiluliuma, King of the Hittites
283-4
Shuwikhatian industry 23
Siarnun (2ist Dynasty) 327
silver 51, 240, 313
Sinai 40, 61-2,73
Sirius (dog-star) 8-9, 216
Sitiah (wife of Thutmose III) 240-1
Sixteenth Dynasty 179 see also Second
Intermediate Period
Sixth Dynasty 103-6, 310 see also Old
Kingdom
slaves 152, 154, 155, 435
Smendes (2ist Dynasty) 302, 325-7
Smenkhera (i3th Dynasty) 183
Smenkhkara (Neferneferuaten) (i8th
Dynasty) 272
Sneferu (4th Dynasty) 90
funerary cult of 96
nesu-bit titles 7
Nubian campaigns 97
pyramids 87-8, 93
Snodgrass, Anthony 12
Sobek cult 257
Sobekemsaf I (i7th Dynasty) 192
Sobekemsaf II (i7th Dynasty) 210
Sobekhotep II (i3th Dynasty) 160,
193-4
Sobekhotep III (i3th Dynasty) 160,
168
Sobekhotep IV (i3th Dynasty) 160-1,
177
Sobekhotep (treasurer to Amenhotep
01)253,264
Sobekhotep V (i3th Dynasty) 160
Sobekmose (treasurer to Amenhotep
111)264
Sobekneferu, Queen 157, 158-9,
228
society
Amarna Period 304
Early Dynastic Period 64, 76
First Intermediate Period 122,
134-5
Libyan period 333-4
Old Kingdom 103
Predynastic 47
Predynastic Period 56, 57
Ptolemaic Period 408-9
Third Intermediate Period 324
Sodmein Cave, near Quseir 19-20,
3 2 >34
Solomon of Israel 327
Sopdet (Sothis) 9
Soped, god 178
Sothic cycle 9
Spain 430
Spalinger, Anthony 215
Sparta 379, 380, 391
Speos Artemidos 230
520 INDEX
sphinxes 90, 222, 230, 243, 247-8,
370
Spiegelberg, Wilhelm 6
staircases, tomb 68, 71
state administration see
administration
statues
bronze 362
colossi 76-7, 139, 156
of foreign captives 310, 311
function of tomb 97
Kushite archaism 351, 361
private 233-4
Ptolemaic Period 407
Ramessid 291-2
royal 153
tomb 175
unusual black-skinned 143
statuettes see figurines
stelae 15
Amarna 255
double-dated 138
Early Dynastic Period 68, 76
Eighteenth Dynasty 209-10, 214
Famine 85
funerary 3
Late Period 369-70, 374
Middle Kingdom 154-5, 168, 169
New Kingdom 218, 219, 227, 245,
247-8, 251, 282, 311
Second Intermediate Period 178,
181, 182, 186-7, 1 9 1 ' J 99
Third Intermediate Period 338, 341
Step Pyramid of Djoser (3rd Dynasty)
69, 71, 76, 79, 81, 85-6, 97,
3"
stone quarrying 145, 299
Eighteenth Dynasty 230
Middle Kingdom 149, 157
New Kingdom 209, 214, 287
Old Kingdom 96, 105
Roman Period 419, 422-5
Second Intermediate Period 192
see also building projects;
pyramids; relief carvings
Story (or Tale) ofSinuhe 148, 165,
171
Strabo 158, 400, 417, 418
strikes 298, 411
Sudan 37, 54
Suez Canal 425
suicide 169
sun
gods 251, 254, 256, 258, 266
symbolism 244
temples 98-9, 100, 233, 249 see
also Amun; Ra
Swadjenra Nebererau I (i6th
Dynasty) 185
Swaserenra Bebiankh (i6th Dynasty)
193
Syncellus, George 5
Syrene (Aswan) 418
Syria 62, 149, 151-2, 167, 201, 235,
238,239,390,392,393
Amenhotep II in 244-5
Amenhotep IV (Akhenaten) in 270
New Kingdom 319
Rameses II in 289
Thutmose I and 221, 223-5
Thutmose IV and 250
Syrian Wars 392, 393
Tachos, King 6
Taharqo (25th Dynasty) 322, 348,
349> 352, 353
Takelot II (22nd Dynasty) 330, 355
Tale of the Eloquent Peasant 129
Tamos, Admiral 381
Tanis 326, 327, 329, 331, 337, 342,
343' 357
INDEX 521
Tanutamani (25th Dynasty) 352, 353,
365
Taramsa-i 20, 21
Tarifian culture 33
Tarkhan 72
Tasian culture 37
Tausret, Queen (i9th Dynasty) 296
taxation 64, 75
Middle Kingdom 161
Old Kingdom 94, 95
Roman Period 417, 419, 420
Teaching ofAmenemhat I, The 164
Teachings for King Meiykara 129, 134,
149, 164
Tefnakht, king of Sais 331, 347
Tell Defenna 367
Tell el-Dab e a 138-9, 174-82, 184, 185,
187, 202
Eighteenth Dynasty at 208
frescos 204-5, 2O ^
Tell el-Farkha 59
Tell el-Habua (Tjaru) 175, 200-1, 202
Tell el-Iswid (Predynastic site) 54
Tellel-Rub e a59
Tell Ibrahim Awad 59
Tell Ibrahim Awad (Predynastic site)
54
Tell Mardikh 96
Tern, Queen 143, 144
Tempest Stele 209-10
temples
administration 164
of Amun 122
Aten 275-6, 278
Early Dynastic Period 76, 77-8
First Persian Period 375
Kushite 350, 351, 352
Mentuhotep II 142-3
New Kingdom 256-8, 298
post-Amarna Period 283
provincial 122, 164
Ptolemaic Period 406
pyramid 87, 97-8, 100, 105, 158
'Restoration of 286-7
Roman Period 430-1
Saite Dynasty 370
sun 98-9, 100, 233
terrace 232-3
Theban nth Dynasty 126-7
triple 144
Twelfth Dynasty 156
valley 89, 226
see also individual temples
Tentamun (wife of Smendes) 326,
327
Teos (Tachos) (3oth Dynasty) 377-8,
380,381,382
Teresh 322
Teti (6th Dynasty) 103, 104, 184
Teti (vizier to Kagemni) 103
Teti-an 203, 218
Tetisheri, Queen 218
textiles 434
Thebes 118, 149, 187, 195
conflict in 302
First Intermediate Period 123-6,
J 33-4
Middle Kingdom 139-45, 146, 147
New Kingdom 209, 226, 228, 230,
2 33~4> 2 35> 2 57> 266-7
nomarchs 108-9
Ptolemaic Period 411
Second Intermediate Period
191-4, 197-203
Third Intermediate Period 328,
330-1,332,336-7,339,343,
348, 360
theocracy 305, 326-7, 329, 333, 340
Thera volcano eruption 205
thermoluminescence dating 2, 14,
3 6 >44
Thessalians 393
522 INDEX
Third Dynasty
funerary architecture 85-7
tombs 93 see also Old Kingdom
Third Intermediate Period 12, 302,
324-63
invasion by Assyria 353-4
Kushite rule (24th Dynasty)
347-52
Libyan period (2ist to 24th
Dynasties) 332-45
metal work 361-2
religion 354-60
Thirteenth Dynasty 137, 159-61, 168,
177, 186
Hyksos violation of monuments of
183
pan-grave people 189-90
Thirtieth Dynasty 377-8, 378, 380,
381,382
Thoth 144, 379
Thracians 396
Thutmose I (i8th Dynasty) 219,
220-6,232,345
Thutmose II (i8th Dynasty) 221,
226-8
Thutmose III (i8th Dynasty) 10-11,
185, 202, 205, 212, 215, 228,
235-41, 246, 256
administrators 261-2
border expansion 311
in the Levant 237-41
Thutmose IV (i8th Dynasty) 237
administration 263-4
building projects 236, 248-9
and Great Sphinx 247-8
kingship and royal women 251-2
Tiaa (mother of Thutmose IV) 247,
252
Tiberius, Emperor 424, 430
timber 81, 238, 313, 320, 393
Tivolivilla4i5
Tiye, Queen 253, 259-60
Tjehenu palette 61
Tjenuna (chief steward) 263
Tjetjy (treasurer) 126
Tod temples 145, 151, 167
tomb chapels 357
tomb-robberies 89
Predynastic Period 51
Ramessid Period 301, 302
Thebes 192-3
Third Intermediate Period 303
tombs
atAbydos 66-70
Amarna Period 276-7, 279
autobiographical texts in 101,
118-19, 120-1
Early Dynastic Period 72
family 115-16
of high officials 64, 70-3, 89, 101,
147, 261-4, 292-3
Kushite 196, 346, 352
Late Period 383, 385
Maadian 55—6
Middle Kingdom 144, 151-2, 169
Naqada 59, 60
New Kingdom 233-4, 235, 239,
257
post- Amarna 279-81
provincial aristocracy 111-12, 113,
163, 169-70
regional styles 116-17
rock 114, 118, 124, 131, 140, 170,
279
royal wives and daughters 143,
260, 291
saffu6, 124-5, 1 4 2
Saite Dynasty 371
Second Dynasty 78-80, 79
Second Intermediate Period 175
soldiers 140
Third Dynasty 93
INDEX 523
Third Intermediate Period 327,
357-8, 360
of Thutmose I and Thutmose II
225-6
see also mastaba-tombs; pyramids
towns and cities
Middle Kingdom 163, 165-6
Ptolemaic Period hostility between
411
Roman Period 416-17
see also urbanization
trade and commerce 314-17, 320-1
A-Group people 73
Badarian 40
Fifth Dynasty 101
and funerary cults 96
luxury goods 69
Maadian 54-5, 56
Middle Kingdom 149, 151-2, 160,
161,166-8
Naqada 51, 58
New Kingdom 234-5
Predynastic Period 34, 61-2
Ptolemaic Period 393
Roman Period 419-28
Saite Dynasty 368-9
Second Dynasty 81
Second Intermediate Period
(Avaris) 182, 197
Sixth Dynasty 105-6
state organized 73-4
Trajan, Emperor 418
Tree Shelter site (Wadi Sodmein) 32
triremes 372, 381-2
see also navy
Tunip, city of 152, 250, 289
Turin Canon 7, n, 84, 85, 88, 91, 103,
104, 107, 108, 109
Middle Kingdom 154, 159, 160
Second Intermediate Period 173,
177, 178, 179, 180, 191
turquoise 96, 101, 105, 156, 158, 214,
299,313
Tushratta, king of Mitanni 259
Tutankhamun (i8th Dynasty) 253,
271, 272, 278, 281-3, 3°3
Twelfth Dynasty 137, 145-59
borders 312-13
co-regencies 10, 138, 154
Faiyum irrigation scheme 152-4,
157
pottery 166-7
representations of captive
foreigners 310
trade 168
Twentieth Dynasty 12, 296-302
tomb-robbers 193-4
Twenty-fifth Dynasty
Assyrian invasions 322, 352-3
Kushite rule 345-54
Twenty-first Dynasty 302, 325-9
burials 358-9
Libyan period 333
Twenty-ninth Dynasty 377, 378
Twenty-second Dynasty 329, 339
military 342
Twenty-sixth Dynasty 279
funerary monuments 371
reunification of 365
Twenty-third Dynasty 330-1, 338, 339
Two Dog Palette 77
Ty 101
U-j, Abydos3
Udjahorresnet 374, 383
Umm el-Qa c ab (at Abydos) 12
Unas (5th Dynasty) 80, 100, 102,
unification of Egypt 59, 6 1, 64
Upper Egypt
agriculture 58
art 122-3
103
524
INDEX
Badarian culture 36-40
burials 57
First Intermediate Period 108,
109, 118-20
Herakleopolitan era 129-30
links with Maadian culture 54
Middle Kingdom 142
pottery styles 116
provincial cemeteries in 112-13
Second Intermediate Period 192,
2IO
temples 105
Third Intermediate Period 339
Upper Palaeolithic period 22-3
urbanization
Early Dynastic period 65, 78
First Intermediate Period 113
Middle Kingdom 165-6
Naqada 52, 62
Uruk culture 62, 63
Userkaf (5th Dynasty) 98, 99
Userkara (6th Dynasty) 104
Valley of the Kings 144
New Kingdom 212, 225, 232, 237,
263, 284, 371
Ramessid Period 291, 309
Third Intermediate Period 303,
328
Valley of the Queens 299
Van Siclen, Charles 243
Vandersleyen, Claude 218, 221
viziers 139, 141, 145
Amarna Period 279
Late Period 383
Middle Kingdom 162
New Kingdom 261, 263-4, 264,
309
Old Kingdom 89, 95
Sixth Dynasty 103
von der Way, Thomas 62
Voros, Gyoro 144
Wadi Hammamat 7, 40, 96, 430
Wadi Hof-Helwan sites 36
Wadi Kubbaniya 24
Wadi Shatt el-Rigal 139
Wakankh Intef II see Intef II
Walls-of-the-Ruler 146, 148, 161,
313
war galleys 372
Ward, William 10
warfare and conflict
Assyrians invasion 353
First Intermediate Period 120-1,
125-6
Herakleopolitan-Theban 131-2,
!33-4> *39-40
Kushite conquest of Egypt 347
Late Period 379-80
Middle Kingdom 147-8, 149,
151-2, 154
New Kingdom 214, 218, 236
Hittites 282, 283-4
Kush 227
Levant 237-41, 244-6, 250-1
Nubia 223, 270
Old Kingdom 97, 105
Predynastic Period 61
Ptolemaic Period 393, 394-8, 410
Ramessid Period 294
Hittites 287
Nubia 289-90
Sea Peopld 295, 297-8
Roman Period 418
Saite Dynasty 371-4
Theban-Hyksos 197-203
Third Intermediate Period 329-30,
330-1
Wars of the Successors 389-90
INDEX
5 2 5
Wars of the Successors (Ptolemaic
Period) 389-90
Washshukanni 250
weapons
as grave goods 140
Hittite technology 292
Theban 202
Webensenu, Prince 248
Weeks, Kent 291
Wegaf Khutawyra (i3th Dynasty)
159-60,168
Wegner, Josef 138
wells 28-9
Wendjebauendjed, General 342
Weni (high official) 104-5
Wepwawetemsaf (i3th Dynasty) 191
Western Desert 18, 19, 22, 23
Badarian culture 39
Maadian culture 54
Neolithic culture 27-31
Old Kingdom 106
quarrying 96
Wheeler, Sir Mortimer 427
White Pyramid at Dahshur 152
White Wall (royal residence) 83, 91,
104
Wild Nile Stage 25-6
Willems, Harco 2
Williams, Bruce 63
Wilson, John 150
wine production 421
Winlock, Herbert 193
wisdom texts 129, 134, 409
women 354-6
literacy of 151
priestesses 210, 215, 219, 251
Roman Period 435-6
royal 158, 216-20, 240-1, 246,
251—2
Amarna Period 271-2
Ptolemaic Period 403-4
as rulers
Cleopatra VII 403, 413, 414
Nefertiti 272
Sobekneferu 157, 158-9, 228
Tausret 296
see also Amun, and 'god's wife'
writing 9, 151
cuneiform 261
First Dynasty 64
hieratic 98, 186, 193, 339
Middle Kingdom 171
Ptolemaic Period 406-7
Second Intermediate Period
180-1
Theban 193
Third Intermediate Period 339
see also hieroglyphs
xenophobia, Egyptian 320, 376,411
Xerxes I (27th Dynasty) 375,381
Zawiyet el- Aryan 87
Zedekiah of Judah 372, 373
Zenodotus of Ephesus 400
Zeus 401, 403
Zivie, Alain 264
