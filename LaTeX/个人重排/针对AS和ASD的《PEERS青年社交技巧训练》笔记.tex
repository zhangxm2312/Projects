\documentclass[UTF8]{article}
\usepackage{ctex,geometry,enumitem}
\usepackage{datetime}
\usepackage[compact]{titlesec}
\usepackage[colorlinks=true,linkcolor=blue,urlcolor=blue]{hyperref}
\geometry{a4paper,left=1cm,right=1cm,top=0.8cm,bottom=1.2cm}
\setenumerate[1]{itemsep=0pt,topsep=0pt,partopsep=0pt,parsep=\parskip}
\setenumerate[2]{itemsep=0pt,topsep=0pt,partopsep=0pt,parsep=\parskip}
\setitemize[1]{itemsep=0pt,topsep=0pt,partopsep=0pt,parsep=\parskip}
\setitemize[2]{itemsep=0pt,topsep=0pt,partopsep=0pt,parsep=\parskip}
% \setdescription{itemsep=0pt,partopsep=0pt,parsep=\parskip,topsep=5pt}
\newcommand{\ind}{\mbox{\hspace{-1pt}}}
\newcommand{\indList}{\vspace{-6pt}}
\newcommand{\Exer}{\subsection{练习和作业}}
\newcommand{\exer}{\paragraph*{家庭作业}}
\newcommand{\prac}{\paragraph*{学员行为演练}}
\newcommand{\warn}{\subparagraph*{注意:}}
\newcommand{\Dot}{•}
\newcommand{\dash}{–}

\title{针对AS\footnote{阿斯伯格症候群(Asperger syndrome)是广泛性发育障碍(PDD,Pervasive Developmental Disorder)中的一种综合症,属于自闭症谱系障碍(ASD)。 其重要特征是社交与非言语交际的困难,同时伴随著兴趣狭隘及重复特定行为,但相较于其他自闭症谱系,仍相对保有语言及认知发展。症状:社交互动有困难、自我局限且重复的行为。}\,/ASD\footnote{自闭症谱系障碍(Autism Spectrum Disorder)。其患者必须存在下列两个症状:(1)缺乏社交沟通与社交互动(或社交及沟通上的广泛性异常);(2)局限的、重复的行为、兴趣或活动(或异常局限性的兴趣、高度重复性的行为)。 自闭症症候群的共同特征包含:社交缺乏、沟通困难、刻板的或重复的行为和兴趣、五官的感受力不灵敏、认知发展较慢。}\,\,的《\href{https://book.douban.com/subject/30403423/}{PEERS® 青年社交技巧训练}\footnote{Laugeson,Elizabeth. PEERS®青年社交技巧訓練:幫助自閉症類群與社交困難者建立友誼. 簡意玲,譯. 臺北:心靈工坊,2018. ISBN: 9789863571308.}\,\,》\href{https://www.douban.com/group/topic/153935186/}{笔记}}
\author{作者:\href{https://www.douban.com/people/terest/}{晨海};编辑及排版:\href{https://www.douban.com/people/zhangxm2312/}{章小明}}
% \date{ver.1.1.20210806.\xxivtime}
\date{}

\begin{document}
\maketitle

\subparagraph*{笔者注:} 因为这本书是面向当教练的父母/老师,所以书中有大量的实例和给教练/老师的家庭作业(问答)部分,为了简略篇幅,这边主要挑一些觉得重要的,具体内容可以看原书. 另,成人可以从中学习到一些经验; 另一本《交友的科学》是这本的缩略版,没有和异性交往的部分,故而 这本更好一些; 如果是成人,没人照顾者 或者想为照顾者省钱的情况下,也推荐使用FriendMarker APP,即虚拟社交教练(仅限于IOS,不免费).

\paragraph*{可以预见的成效:} 在2009年《自闭症与发展疾患期刊》的一篇文章中,作者研究比较33位年龄均在13-17岁间的自闭症类群青少年,并将其分为家长辅导的PEERS治疗组,和(暂候治疗的)对照组.治疗课程后,老师和家长都在之后的报告中写到``在社交知识/技巧上有显著提升''.在2012年同期刊上的28位自闭症类群青少年研究也验证了前一个研究的成果. 临床上,PEERS课程也可以用于ADD\footnote{注意力不足过动症(Attention Deficit Disorder),即注意缺陷障碍。最主要的症状是频繁且不自觉地走神。},抑郁症,焦虑症,与其他发展疾患,并预期可以得到相应的成效.

\paragraph*{课程结构:} 先花50分钟的时间验收和讨论上周的家庭作业(一般是问答),花20分钟讲授课程,花10分钟指定家庭作业,最后花10分钟的时间进行回顾,分发作业,和为下周作业进行私下协调,这其中还会包括一些角色扮演和行为演练.

\warn 讲授课程的过程中,一些内容更推荐用提问的方式,真正让学员参与其中,而不是只是坐在那听. 比如第一章的讲授课程中的\emph{良好友谊的特征}(\ref{1.1}),可以陈述完一个点后问:``为什么这是好的?'';而知道这些之后,就可以通过提问的方式讲授\emph{友谊的类型:普通朋友拥有的良好友谊特征多吗?}(\ref{1.2}).这还可以避免时间一长,刚说的忘记了.最后下课前的一遍是第三遍,作业是第四遍,一周后的作业检查是更有针对性的第五遍.通过这五遍及在日常生活的实践,这些知识就被牢牢地掌握住,并变成社交技能了.

\vspace{3pt}
书中内容会据点做非常细致的讲述.比如在\emph{良好友谊的特性: 分享共同兴趣}(\ref{1.1})处,书中会叙述``朋友有相同的兴趣,喜欢的事物与活动和共同兴趣,可以让你们一起讨论,一起去做'',以便于理解.但本文中除特别词汇外不会加入这种延伸,请自己通过经验或词典补充.

这种社交技能的东西重在理解和使用,切不可让学员死记硬背.所以同书中一样,没有加入序号\footnote{为示事物间顺序,编者为某些内容填上序号了.}.对于台湾与大陆不同的词汇描述,笔者也做了自己的修改.

\tableofcontents

\newpage
\section{交换信息和开启交谈}

\subsection{良好友谊的特征\label{1.1}}
\indList
\begin{itemize}
    \item[\Dot] 分享共同兴趣
    \item 善意与关心
    \item 支持
    \item 互相了解
    \item 守信用与忠心.忠心就是用行动支持和忠诚.
    \item 诚实与信任
    \item 平等
    \item 能够自我揭露.即分享个人的想法/感受/过去经历.
    \item 解决冲突.通过关怀/守信用/信任的方式.
\end{itemize}
\indList

\subsection{朋友的类型\label{1.2}}
\indList
\begin{itemize}
    \item 点头之交(Acquaintances and Known names):知道名字或见过几面
    \item 普通朋友(Friends):知道名字,有互动,但不常花时间在社交上
    \item 较熟的朋友(Best Friends and Good friends):会花更多时间社交
    \item 最好的朋友(Intimates, Close Friends):在小圈子内活动,大部分时间都在一起
\end{itemize}
\indList

\subsection{交换信息的规则\label{1.3}}
\indList
\begin{itemize}[label=\dash]
    \item[1.] 问别人问题(兴趣,或周末喜欢做什么)
    \item[2.] 回答自己提出的问题.如分享与自己有关的事.
    \item[3.] 找出共同兴趣
    \item[4.] 接着问(在相同主题内)相关的问题
    \item[5.] 分享对话.有时需要暂停一下,使得别人有机会问问题或者做评论.
    \item 不要垄断对话:让别人有机会说话.
    \item 不要像在访问:不要一个接着一个问题,而不分享自己的回答.
    \item 不要一开始就问太私人的问题
\end{itemize}
\indList

\subsection{开启交谈的步骤\label{1.4}}
\indList
\begin{enumerate}
    \item[\dash] 不要上来就说``Hi''或介绍自己:这并不是实际可行的做法,因为很突兀
    \item 轻松的看一下对方:先表示你对ta有兴趣.注意不要瞪着.
    \item 同时利用手边的物品:看起来像专心做其他事的过程中看了一眼,避免突兀.
    \item 找出共同兴趣:通过观察,找到可能的共同兴趣.这样有可以交谈的理由.
    \item 提起共同兴趣:通过评论/问题/称赞找到交谈的理由.
    \item 交换信息:问相关问题,或谈和自己有关的信息.这样就可以彼此认识了.
    \item 评估对方和我说话的兴趣:如果看起来没有兴趣的话,应该放弃.
    \item 介绍自己(我是某某某,很高兴认识你)
\end{enumerate}
\indList

\Exer
\prac 抢答游戏``最喜欢什么'': 随便挑一个学员,问它最喜欢什么.要和它对话过或者电子通话过的学员抢答.
\exer (1)安排团体内打电话或者视频聊天的时间,并在打电话后询问他们``你们的共同兴趣是什么?''.若已知,则问他们``如果要一起消磨时光,可以做什么?''.
(2)学员和教练之间练习\emph{开启并维持交谈,交换信息,找出共同兴趣}(\ref{1.4})的步骤,问题同上.
\warn 要给学员赞美,不要直接指出/指责学员的错误,而是在每个步骤给出积极反馈:如``下次,我们更注意,不要垄断对话/问朋友喜欢什么,怎么样?''. 使用关键词,避免说教.

\newpage
\section{交换信息和维持交谈}

\subsection{交谈主题}
\indList
\begin{center}
\begin{tabular}{c|c|c}
    \textbf{学校或者工作的八卦}& \textbf{电脑游戏和网络信息} &\textbf{课程或爱好}\\\hline 
    朋友问题& 电脑/科技& 考试/论文/功课\\ 
    家庭问题& 漫画/动漫& 教授/指导老师\\ 
    男女朋友& 电影& 申请大学与工作\\ 
    约会& 综艺/节目& 体育赛事\\ 
    派对/聚会& B站视频/抖音& 单车/摩托车/汽车\\ 
    周末活动& 喜欢的网站& 打工/旅行\\ 
    见面会& 音乐/音乐会& 社交名人\\ 
    社团/活动& 书籍& 流行时尚\\ 
    美食& 新闻/媒体/政治& 购物\\ 
    兴趣/嗜好& 网络课程& 化妆/发型\\
\end{tabular}
\end{center}
\indList\indList
\warn 应避免谈到政治和宗教等容易产生纠纷的话题.

\subsection{交换信息并维持交谈}
先回顾 \ref{1.3} 交换信息的规则.
\begin{itemize}
    \item[\dash] 不要一直重复
    \item 倾听你的朋友说话
    \item 提出开放式语句:即不要提出一些只能回答``是''``不是''``对''``不对''的问题.但开放式提问不能太多,否则像采访.
    \item[\dash] 不要自夸
    \item[\dash] 避免好辩.喜欢和别人争辩显得非常不礼貌.
    \item[\dash] 不要纠正别人,尤其在有他人在场的场合.避免别人尴尬.
    \item[\dash] 不要挖苦别人
    \item 控制适当音量,不要太大声.
    \item 维持适当的身体界限.大概维持在一双手臂长的距离.
    \item 维持适当的眼神接触.不要盯着别人,也不要不看它,会显得没有兴趣聊下去.
\end{itemize}

\Exer \prac 提出开放式问句:针对Jeopardy!\footnote{危险边缘(Jeopardy!)是由Merv Griffin在1964年创办的美国电视智力竞赛节目。 和同类节目一样,它涵盖了历史、语言、文学、艺术、科技、流行文化、体育、地理、文字游戏等多方面内容。 但与众不同的是,它采取一种独特的问答形式:参赛者须根据以答案形式提供的各种线索,以问题的形式作出正确的回答。}主题(最喜欢的音乐/周末活动/运动/游戏/电影/节目/书/食物)在小组中问问题并交换信息.
\exer 同上周.
\warn 从这一课开始,带领者和教练要给每种规则做错误示范. 比如说话声音小,站太近等. 示范后提问``这段对话中,我\emph{做错}了什么?'',最后再做一次正确示范.

\newpage
\section{找朋友的来源}
我们不需要也不能和每一个人做朋友,对他人而言亦然.既然朋友是一种选择,那么在哪能让学员找到一些交朋友的好选择?

\indList
\subsection{社交群体}
\begin{table*}[h]\centering\small
\indList\indList\indList
\begin{tabular}{r|ccccccc}
\normalsize\textbf{政治群体}&LGBTQ&宗教团体&政治社团&历史社团&军事社团&动物保护者&环境保护者\\
\normalsize\textbf{STEM爱好者}&计算机/技术&数学/物理&化学/生物&机器人&医学&&\\
\normalsize\textbf{运动爱好者}&健身&篮球/足球&羽毛球/乒乓球&滑冰/滑板/冲浪&自行车/机车&&\\
\normalsize\textbf{一般交际}&辩论队&学生会&同学/同事/邻居&兄弟会/联谊会&狂欢者&&\\
\normalsize\textbf{音乐爱好者}&组建乐团&音乐剧&乐器演奏&合唱/伴奏&摇滚&Hiphop&\\
\normalsize\textbf{游戏爱好者}&PC游戏&主机游戏&社交游戏&音游&氪金手游&&\\
\normalsize\textbf{娱乐}&追星/八卦&汉服/JK/Lolita&手账/手作&妆饰&ACG爱好者&cosplay&bjd娃娃\\
\normalsize\textbf{艺术爱好者}&电视剧&戏剧/话剧&电影&创作&文学/文青&绘画&街舞\\
\normalsize\textbf{其他}&新闻&建筑&棋类&集邮&旅行&读书会&\\
\end{tabular}\footnote{此处由编者根据内地情况进行了改动}
\indList\indList\indList
\end{table*}

\subsection{可能的社交活动}
\indList
\begin{center}
\begin{tabular}{c|l}
\textbf{兴趣}&\textbf{相关社交活动}\\ \hline
电脑/科技&电脑课,电脑科技部门的有关活动,科技聚会/发布会,科技/电脑俱乐部\\
电玩&和朋友一起玩游戏,游戏大会,游戏商店,游戏见面会,游戏俱乐部\\
科学&科学博物馆活动,科学课,科学见面会,俱乐部,机器人俱乐部\\
漫画&漫画展,漫画书店,上漫画课,漫画/动漫见面会,漫画/动漫俱乐部\\
棋类&有人下棋的游戏商店;各种棋的竞标赛,见面会,俱乐部\\
cosplay&参加漫画展,上缝纫课学做服装,参加cosplay活动,俱乐部,见面会\\
电影&见面会,俱乐部,电影的贴吧类网络社团\\
体育&参加体育,在社区和公园运动,参加体育的社交联盟和赛事,见面会,俱乐部\\
汽车&看车展,参观体育博物馆,上汽车商店课程,见面会,俱乐部\\
音乐&听音乐会,参加大学乐团,上音乐课,见面会,俱乐部\\
\end{tabular}
\end{center}
\indList\indList

\subsection{社交群体共同兴趣的线索}
\indList
\begin{itemize}
    \item 外表,服装,发型
    \item 兴趣(这是谈论的主题)
    \item 在某地/某时主要会做什么
    \item 和谁在一起
    \item 常参加的活动
\end{itemize}
\indList

\subsection{被社交群体接纳与否的特征}
\indList
\begin{center}
\begin{tabular}{r|l}
\textbf{被接纳的特征}&\textbf{不被接纳的特征}\\ \hline
有个别人找你或找你一起出去&没人找你出去\\
它们和你说话,或是回应你的话&忽略你,不回应你尝试想说的话\\
留给你联络方式&没有给你联络方式\\
跟你要联络方式&没有跟你要联络方式\\
通过各种联络方式想和你聊天&没有通过联络方式和你聊天\\
在通讯中回应你&没有接电话或在通讯中回应你\\
邀请你一起做一些事情&没有邀请你一起做一些事情\\
接受你一起做一些事情&没有接受你一起做一些事情,拒绝你的邀请\\
把你拉入/加入它们的社交媒体群组&忽略你在社交媒体上的交友请求\\
对你说一些好话,并夸赞你&嘲笑或捉弄你\\
\end{tabular}
\end{center}
\indList\indList

\Exer \exer 学员共同讨论并决定社交活动,然后参加这些活动(要求:感兴趣,一至两周一次,年龄相近且互相接纳). 以及同样的学员间打电话或者视频聊天,下周带一件自己喜欢的私人物品来交换.
\prac 通过Jeopardy!来进行抢答

\newpage
\section{网络通讯}

\subsection{交换联络方式的具体步骤}
\begin{enumerate}
    \item 多次交换信息(我记得你说过你喜欢科幻电影)
    \item 找出共同兴趣(我也喜欢科幻电影,你看了上周新片了吗?)
    \item 把共同兴趣当成联络的的一种方法(听说不错, 也许我们该一起去看看)
    \item 评估对方和我联络的兴趣:如果有兴趣的就同意了(这样可以约个时间),\\ 如果没有时间/犹豫/找借口忙,可以回去继续谈共同兴趣(没关系,不过你真的可以去看看).
    \item 建议交换联络方式(我们应该留个电话号码/你有用QQ或微信吗?)
\end{enumerate}

\subsection{打电话中通话的步骤}
\begin{enumerate}
    \item 说出要找的人(如果是家庭电话或对方声音不像朋友的话,可以说``喂,请问小明在家吗?'')
    \item 说明自己是谁(喂,小明,我是小可啊)
    \item 询问是否方便通话(你现在方便说话吗?)
    \item 问候(最近怎么样?)
    \item 给出一个打电话的理由(我打过来是想知道你最近怎么样)
\end{enumerate}

\subsection{结束通话的步骤}
\begin{enumerate}
    \item 等到对话稍微停下来的时候(尽量不要打断)
    \item 给一个必须离开的说法(唉,我最好赶快回去工作)
    \item 告诉对方聊得很愉快(和你聊天很愉快)
    \item 告诉对方你之后会再找对方聊天(之后再找你聊)
    \item 说再见
\end{enumerate}

\subsection{说法范例\label{4.4}}
\indList
\begin{tabular}{c|c}
\textbf{打电话的理由}&\textbf{必须挂电话的理由}\\ \hline
只是打来看看你最近怎么样&得走了\\ 
只是打来看看你最近好吗&不耽误你的时间了\\ 
是想请教学习/工作上的问题&我要去念书了\\ 
有一阵子没找你唠嗑了&该吃晚饭了\\ 
想知道你现在再干嘛呢&要回去工作/做作业了\\ 
\end{tabular}
\indList

\subsection{语音留言步骤}
\begin{enumerate}
    \item 说明自己是谁
    \item 说你想找谁
    \item 打电话的时间
    \item 打电话的原因
    \item 留下联系方式
    \item 说再见
\end{enumerate}

\subsection{网络通讯的一般规则}
\begin{itemize}
    \item 说明自己是谁
    \item 接触不熟的人时,找一个理由
    \item[\dash] 不要在早九点前和晚九点后找对方
    \item[\dash] 不要谈太私人的事情,否则有泄露隐私的风险.
    \item[\dash] 两次信息定律:如果没回,不可以连续传信息超过两次
    \item[\dash] 避免不请自来的电话,它就像电话推销和邮箱病毒广告.
    \item 注意网络安全,小心被骗以及暴露个人信息.
\end{itemize}

\subsection{网友见面的安全诀窍}
\begin{itemize}
    \item 只在公共场所,人越多越好.
    \item[\dash] 不要单独和团体成员或者网友去任何地方
    \item[\dash] 不要搭顺风车
    \item 以自己的交通方式前往和离开见面
    \item 让家人或者好友知道去哪,和谁在一起,什么时候去
    \item 参加见面的之前和之后都要和朋友/家人联系
    \item 尽可能和其他朋友一起参加见面
\end{itemize}

\Exer \exer 同上周.团体成员在通话过程中,使用上面的开始结束通话知识(\ref{4.4})并记下,并在下周课开始时回答这个问题.
\warn 为了保护学员的信息安全,需要查看学员在社交媒体平台(如贴吧,豆瓣,知乎,微信)及个人信息界面上有没有泄露个人信息.

% \newpage
\section{善用幽默}

\subsection{使用幽默的规则}
\begin{itemize}
    \item 初次认识某人的时候,需要正经些,否则会让人感到不舒服和困惑.
    \item[\dash] 不要重复别人已经听过/讲过的笑话
    \item[\dash] 避免羞人的笑话,否则会伤害到他人
    \item[\dash] 避免说黄色笑话,这会给自己带来坏名声,并让听的人不自在.
    \item[\dash] 避免说只有部分人才听得懂的小团体笑话,这会让听不懂的人感到被排除在外和受伤.
    \item[\dash] 不要跟权威人士说笑话,这会被认为很不尊重他人,而且还会惹上麻烦.
    \item[\dash] 不要无缘无故的发笑,否则你会被认为你在笑身边的人,或是觉得你怪异.
    \item 幽默要切合年纪.不要说太过幼稚的笑话,或需要上了些年龄的人才听得懂的笑话.
    \item 幽默要切合情境.比如不要对不懂机械工程的人说这个领域内的笑话.
    \item 说笑话的时机需要恰当:自由时间,午休,别人在说笑时合适;考试,工作,上课,或别人难过时不合适.
    \item 别人说笑时需要礼貌性的笑笑
    \item 注意幽默回馈,根据反馈来判断这个笑话怎么样,要不要继续说.\\ 反应有(1)没笑;(2)礼貌的笑;(3)取笑你;(4)跟着你的笑话笑.其中只有最后一种才应该继续讲.
\end{itemize}

\subsection{幽默回馈特征}
\indList
\begin{center}
\begin{tabular}{c|c}
\textbf{取笑你}&\textbf{跟着你的笑话笑}\\ \hline
翻白眼&笑出声,并有微笑的表情\\ 
跟着别人笑才笑&称赞你的笑话或你的幽默感\\ 
没讲完就笑&—边笑一边点头\\ 
过了很久才笑&要你说另一个笑话\\ 
一边笑一边做奇怪表情&说出``这还不错''并微笑\\ 
一边笑一边指着你&说出``你还蛮风趣的''并微笑\\ 
一边笑一边对其他人摇头&``我要把这个笑话记下来"\\ 
讽刺的评论&也开始说起笑话来\\ 
\end{tabular}
\end{center}
\indList

\subsection{分辨与笑话不同关系的人}
\indList
\begin{enumerate}
    \item 喜欢笑话的人:占大部分.可能本身并不是很有幽默感,但可以享受幽默.
    \item 爱说笑话的人:小部分不停的说笑话,\ind 自认为是搞笑或卖丑的人.\ind 人们也会因此跟着笑或取笑它们,\ind 但后者是不对的!
    \item 讨厌笑话的人:不要和它们说笑话,就像强行纠正别人一样.
\end{enumerate}
\indList

\Exer \exer 同上周,但要注意幽默反馈(而非开始和结束对话),演练中也是.

\newpage
\section{加入一群人交谈}

\subsection{加入一群人交谈的步骤\label{6.1}}
\begin{enumerate}
    \item 聆听对话,以知道目前聊天的主题是什么.
    \item 维持一个距离观察.不要太引人注目,更不要瞪着这群人.
    \item 利用手边物品.类似于\emph{开启交谈的步骤}(\ref{1.4}),好像在专心做什么.避免光在听,这样别人会不自在.
    \item 辨识主题
    \item 在主题上找出共同兴趣
    \item 走进它们:加入交谈,通常与他人保持一到两个手臂的距离是合适的.
    \item 等候谈话停顿.只要不是太明显的打断他人谈话就可以,因为可能很难找到``完美停顿''
    \item 提起主题.做个评论/问问题/称赞主题,这是加入谈话的理由.
    \item 评估它们和我说话的兴趣:在和我说话吗?在看着我吗?面对着我吗?(若否,则对方圈子封闭)
    \item 介绍自己.(刚忘了说了/我们没有见过面,我是某某某)
\end{enumerate}
\Exer \exer 同上.但和教练/同伴一起练习加入一群人交谈,并在其中注意幽默反馈.

% \newpage
\section{退出交谈}
平均一个人加入一群人会话,有五成左右的概率会被拒绝,所以有时会候失败,但不要放弃在其他团体中尝试.
\subsection{谈话中不被接纳的处理}
\begin{center}
\begin{tabular}{c|c}
    \textbf{交谈中不被接纳的理由}&\textbf{下次可以有什么不同的做法}\\ \hline
    它们想私下谈&等一下再尝试,加入前先聆听\\
    它们粗鲁或恶意&尝试不同的谈话群体\\
    你没有遵守\emph{加入交谈的规则}(\ref{6.1})&等一下再尝试,先按上面步骤来\\
    你谈了太私人的事情&尝试不同谈话群体,并避免私人话题\\
    它们组成一个小社交圈,不喜欢新朋友&尝试不同的谈话群体\\
    它们谈的事情你并不知道&找主题了解的对话群体\\
    你在它们中有不好的名声&找不知道或不在意你名声的群体\\
    并不知道你想加入&等一下再尝试,按上面步骤来\\
\end{tabular}
\end{center}
\subsection{被接纳后被排斥的退出步骤}
\begin{enumerate}
    \item 保持冷静,不要生气,或表现出生气的样子.这样会导致不好的名声.
    \item 看向别处,好像被其他事物吸引.可以看左右或者自己的物品,表现出不再对谈话感兴趣.
    \item 等待短暂的谈话停顿
    \item 为离开给一个简短的说法(我得走了/保重/再见).因为这种情况下可能它们不在乎.
    \item 走开.不用等待回应,即使是它们说再见也不用等说完.
\end{enumerate}
\subsection{完全被接纳时的退出步骤}
\begin{enumerate}
    \item 等待谈话停顿
    \item 给出离开的一个具体的理由(我得回家了/我的休息时间结束了).具体的理由不代表具体的说明.
    \item 说下次见.让朋友们知道你是还想和它们在一起,离开是不得已的.
    \item 说再见.这样比较礼貌.较熟的朋友可以不用.如果有朋友要拥抱或者碰拳,挥手之类的,完成这个礼仪.
    \item 走开
\end{enumerate}
\warn 上面的每一个例子都需要做示范,并问出观点``感觉怎么样?''

\Exer \exer 同上,使用这次学到的退出谈话的知识.

\newpage
\section{朋友聚会}
\subsection{规划聚会(5W)\label{8.1}}
\begin{itemize}
    \item Who:哪些人在场
    \item What:想做什么
    \item Where:在哪里
    \item When:何时聚会
    \item How:如何进行聚会,尤其在交通和付费方面(谁来开车,要不要买票).
\end{itemize}
\subsection{聚会常见活动}
\begin{center}
\begin{tabular}{cccccc}
\textbf{公共场所活动}&\textbf{室内活动}&\textbf{游乐园}&\textbf{聚餐}&\textbf{双人运动}&\textbf{团体运动}\\\hline
电影院&电动游戏&电动游戏中心&餐厅&双人自行车&水球\\
购物中心&电脑游戏&游戏机(投篮机)&美食街&步行&足球\\
体育赛事&上网&真人射击游戏&快餐车&攀岩&棒球\\
夜店/舞厅&看B站&迷你高尔夫&晚餐聚会&网球&曲棍球\\
漫画展&社交网络&水上乐园&冰淇淋店&冲浪&排球\\
游戏大厅&听音乐&棒球&糖水店&游泳&羽毛球\\
科学博物馆&看比赛&迷你赛车&批萨&划船&篮球\\
战棋游戏&观赏颁奖典礼&摩托车赛&外卖&高尔夫&袋棍球\\
音乐会&看电影&车展&自带&滑板/溜冰鞋&橄榄球\\
节日庆祝活动&看电视节目&野生动物园&自助烧烤&桌球&武术\\
角色扮演&桌游&水族馆&寿司&篮球&有氧课程\\
保龄球&下棋&博物馆&做饭&溜冰&舞蹈课程\\
宠物店&益智问答&科学展览馆&野餐&飞盘&街舞\\
公园&飞镖&艺术/美术馆&烘培&钓鱼&漂流\\
海滩/河边&手工艺&书城(可以聊天)&美食节&健身&慢跑\\
\end{tabular}
\end{center}
\subsection{准备聚会}
\begin{enumerate}
    \item 提醒并确认计划(在聚会前一两天).可以打电话或发消息.
    \item 确认聚会空间已经被清理.如果在家里/开车,最好提前一两天.
    \item 准备好一些点心/饮料,可以分享
    \item 收拾好不想和别人分享或让别人看到/碰到的个人物品.因为说``不可以看/碰''会显得非常没有礼貌.
    \item 准备好其他活动.活动需要是你们共同感兴趣的.
\end{enumerate}
\subsection{开始与朋友的聚会}
\begin{enumerate}
    \item 朋友来了打招呼问候
    \item 邀请进门
    \item 介绍不认识的人
    \item 带它们看环境
    \item 招待点心饮料
    \item 问它们想做什么.比如朋友不舒服,就商量后面的计划.
\end{enumerate}
\subsection{朋友聚会中的规则}
\begin{itemize}
    \item 以共同兴趣的活动为基础
    \item 由客人选择在你家的活动.但也应事先有想法,并有所准备.
    \item 从善如流,即不管发送什么预期之外的事情,都可以随机应变.
    \item[\dash] 不要未预期地邀请其他人来聚会,这不尊重原来的朋友.
    \item[\dash] 不要冷落你的朋友.尽量对每个朋友都时常关心,与其聊天.
    \item[\dash] 不要嘲笑你的朋友.男性之间的互相嘲笑是有风险的.
    \item 支持你的朋友.如果你的聚会中有朋友嘲笑另一个朋友,需要支持那个被嘲笑的朋友,并尽力让双方保持和气.
    \item[\dash] 不要和你的朋友起争执.应在聚会中从善如流,避免冲突.
    \item[\dash] 不要纠正你的朋友
    \item 要有好的风度,比如玩游戏或者运动时不要输不起或好胜.
    \item 如果觉得无聊,可以做一些改变.不要直接说很无聊或直接走了,比如可以建议玩一些其他的.
    \item 至少用一半的时间交换信息(如交流共同兴趣).若否,就不会更了解彼此.
    \item 前几次聚会避免时间太长.先保持简短而愉快,不要着急.
\end{itemize}
记住:\emph{朋友是一种选择.}如果你的朋友从不听你的建议,或者没有顾及到你的感受,未必要与之聚会.
\subsection{结束与朋友的聚会}
\begin{enumerate}
    \item 等待活动中的停顿
    \item 给结束聚会一个说法(明天一早还要上班)
    \item 送朋友到门口
    \item 谢谢朋友来聚会
    \item 告诉朋友你度过非常愉快的时光
    \item 说再见/下次见
\end{enumerate}
\Exer \exer 和其他学员用上面的知识组织聚会,练习加入退出一群人交谈,以及观察幽默回馈.

\newpage
\section{约会礼仪:让某人知道你喜欢它}
\subsection{选择适当的人约会}
\begin{itemize}
    \item 和朋友一样, 约会的对象也是一种选择
    \item 同时符合:
    \begin{itemize}[label=*]
        \item 学员喜欢
        \item 对学员感兴趣
        \item 共同兴趣
        \item 年龄相近
        \item 可能答应约会
    \end{itemize}
    \item[\dash] 避免:
    \begin{itemize}
        \item 捉弄/冷落/占便宜/利用学员
        \item 曾拒绝过学员
        \item 已有伴侣
    \end{itemize}
\end{itemize}
\subsection{约会对象来源}
\begin{tabular}{|ccccc|}\hline
    朋友的朋友&派对/聚会&学校&公园&图书馆\\ 
    家人的朋友&工作&社交活动&体育活动&咖啡馆\\ 
    同学的朋友/家人&邻居&俱乐部/健身房&成人课程进修&书店\\ \hline
\end{tabular}
\subsection{让对方知道你喜欢它\label{9.3}}
\begin{itemize}
    \item[1.] 对共同的朋友说.先让朋友询问对方是否有伴侣了,再说喜欢它.最后问朋友,``你觉得它可能会愿意和我约会吗?''.如果朋友要去问它,要朋友不要说是你.
    \item 用眼神传情:眼神接触, 微笑, 移开眼神, 反复数次.
    \item[2.] 问对方是否和人与之约会
    \begin{enumerate}[label=(\arabic*)]
        \item 交换信息找共同兴趣(听说你上次去滑雪, 我也喜欢滑雪)
        \item 问和共同兴趣有关的社交活动(平常和谁一起去滑雪?)
        \item 轻松把内容转向约会(是和男/女朋友去的吗?).这一步可以分辨出对方对你有没有好感.
        \item 给出一个为什么这么问的理由(我的朋友们都是和男/女朋友一起去滑雪)
        \item 谈话转回共同兴趣(所以你滑雪滑的不错?)
    \end{enumerate}
    \item 赞美\begin{itemize}[label=*]
        \item 对不熟的人要具体(你笑起来真好看/那个笑话真的好好笑)
        \item 对比较熟的可以笼统一些(你很漂亮/你真有趣)
        \item[\dash] 避免太多关于外貌的赞美,赞美应该只限于脖子以上
    \end{itemize}
    \item 表现出对对方的兴趣.交换信息以寻找共同兴趣.
    \item 跟着对方的笑话而笑.不要过于夸张, 也不要过于礼貌.
\end{itemize}
\Exer \exer 和朋友聚会.如果有感兴趣的对象,可以练习让对方知道你喜欢它.练习加入退出一群人交谈.

\newpage
\section{约会礼仪:提出约会邀请}

\subsection{提出约会邀请之前}
\begin{itemize}
    \item 告诉你们共同的朋友
    \item 做到\emph{让对方知道你喜欢它}(\ref{9.3})中的全部步骤
\end{itemize}

\subsection{提出约会邀请\label{10.2}}
\begin{itemize}
    \item 等适当的时候提出.不要在有旁人或对方在做事时.要在都有空,面对面且对方心情不错时.
    \item 交换信息(为什么约它?)
    \item 提起共同兴趣.这也是约会的主要内容.
    \item 问对方在某个时间点要做什么(这周末/下周五你要做什么?)
    \item 评估对方对我的兴趣:\begin{itemize}[label=*]
        \item 说有空并微笑/说有事并表现失望是\emph{好的}的迹象
        \item 无论有事或有空,但并看起来不太自在/提到自己男女朋友/转移话题,都是\emph{不好的}的迹象
    \end{itemize}
    \item 用共同迹象当作一起出去的理由,如果表现得感兴趣就邀请.\\ (比如共同兴趣是科幻电影,就说``一起去电影院看刚出的科幻电影''.)
    \item 交换联络方式.若无则可以索要(我可以加你微信吗?).
    \item 告诉对方你会再联络确认\begin{itemize}
        \item 一般在约好后两天内:当天会吓到对方,超过三天对方可能会失去兴趣,认为你诚意不够.
        \item 用\emph{5W方法}(\ref{8.1})规划约会
        \item 约会前再确认
    \end{itemize}
\end{itemize}

\subsection{邀请被拒绝}
\begin{enumerate}
    \item 保持冷静
    \item 以一个轻松的声明表达接受(好的,没关系)
    \item 话题转回共同兴趣
    \item 用一个说法来结束交谈.如友善地说``如果你改变心意,请告诉我'',不要问原因.
\end{enumerate}

\subsection{拒绝别人}
\begin{enumerate}
    \item 保持冷静
    \item 礼貌拒绝(我想不太适合/我不是很想约会)
    \item 为拒绝给一个说法(我觉得我们只是朋友)
    \item 谢谢对方提出邀请(不管怎样,还是要谢谢你,我觉得很荣幸.)
    \item 话题转为共同兴趣
    \item 用一个理由结束交谈.
    \item[\Dot] 友善
    \item[\dash] 不要为避免拒绝别人/抱歉而答应
    \item[\dash] 不要取笑或作弄对方
    \item[\dash] 不要将这件事告诉别人
\end{enumerate}
\Exer 同上.用到这次学到的知识.

\newpage
\section{约会礼仪:前往约会}

\subsection{规划约会}
\begin{itemize}
    \item 按计划,应邀后两天确认
    \item 按\emph{5W方法}(\ref{8.1})确认约会
    \item 即将约会之前确认计划
\end{itemize}
同\emph{提出约会邀请}(\ref{10.2})最后一条.

\subsection{为约会做准备}
\begin{itemize}
    \item 确定你的约会地点是可以见人的.如果约会的地方是你的家或要坐你的车子, 要做好清洁.
    \item 把不想在约会时被看/触摸到的东西收好
    \item 保持良好卫生
    \item 穿着合适\begin{itemize}[label=*]
        \item 运动活动可以穿普通衣服/运动装
        \item 吃晚餐/看电影要穿好看点
        \item 看戏剧/交响乐时要穿着讲究
    \end{itemize}
    \item[\dash] 避免穿着过于性感撩人,避免对方认为你是随便的人.
    \item 试着表现自己最好的一面.这是对对方的尊重,也可能会增加对对方的吸引力.
\end{itemize}

\subsection{约会的安全}
\begin{itemize}
    \item[\dash] 如果不是特别熟,不要一开始就给对方家里电话/全名/住址.
    \item 见面前网络搜索对方.可以是全名/网名/照片,以免被骗.
    \item 让家人和朋友知道你在哪里,和谁在一起
    \item 自己用交通工具赴约与离开
    \item 在公共场所约会
    \item[\dash] 避免一开始就和对方去任何地方
    \item 赴约前后告诉家人朋友
\end{itemize}

\subsection{开始约会的步骤}
\begin{itemize}
    \item 问候约会对象
    \item 如果在自己家:\begin{enumerate}[label=(\arabic*)]
        \item 邀请对方进门
        \item 介绍对方不认识的人
        \item 看看环境
        \item 招待点心饮料
    \end{enumerate}
    \item 询问对方有关你提的计划(我们还是去原来预约的餐厅吗?)
\end{itemize}

\subsection{约会过程}
\begin{itemize}
\item 表示出对约会对象的兴趣:问对方问题,听对方说想说的,微笑,眼神接触.
\item 用至少一般的时间来交换信息
\item 跟着对方的笑话而笑
\item 保持礼貌并尊重ta:\begin{itemize}
    \item 扶门/拉椅子
    \item 等对方的食物/饮料来了才吃喝
    \item[\dash] 不要讲脏话或诅咒
    \item[\dash] 不要争辩/纠正/嘲笑对方
\end{itemize}
\item 问你的约会对象想做什么,不要自己做所有决定.
\item 从善如流,灵活应对计划外的情况.
\item 赞美你的约会对象,同邀请时一样.
\item[\dash] 不要对别人传情,经常/过久地看别人也是.
\item[\dash] 不要邀请意料之外的人加入你们的约会
\item[\dash] 不要冷落你的约会对象.不要和其他人说太多话,玩手机.就算有重要的电话或短信,应该先告知.
\item 如果你或者你的约会对象觉得无聊,应提议转换活动.
\item[\dash] 避免有风险的谈话内容,如性,宗教,政治.
\item[\dash] 避免有风险的地方,如夜店/电音派对/海鲜餐厅/寿司店/音乐会/嘈杂的餐厅/可能发生亲密接触的地方.
\item 准备好付账.不是每次都要你付账,但是你应当永远做好付账的准备,并表示要付账.
\end{itemize}

\subsection{结束约会的步骤}
\begin{enumerate}
\item 等待约会中的空档
\item 给约会结束给一个好的说法(有点晚了,我想我最好送你回家)
\item 谢谢对方跟你出来/约你出来
\item 告诉对方你度过了愉快的时光
\item 陪对方走出去,不管送不送回家.
\item 提议下次在一起出来(我们找个时间再约一次吧?)
\item 告诉对方你何时再联络(``我过几天再打给你'',并一定要记得打给ta)
\item 说再见
\item[\Dot] 任何身体接触都要得到许可.如果想拥抱对方或者亲吻对方,需要先问``我可以拥抱你/和你吻别吗?''.
\end{enumerate}

\subsection{约会之后}
\begin{enumerate}
    \item 隔天发信息或者打电话联络
    \item 谢谢对方跟你出去/约你出去
    \item 告诉对方觉得很愉快
    \item 再次约定下次约会的内容
\end{enumerate}

\newpage
\section{约会礼仪:该做与不该做的事}

\subsection{约会该做的事情}
\begin{itemize}
    \item 要记得\emph{约会是一种选择}\begin{itemize}[label=*]
        \item 互相喜欢
        \item 有相同兴趣爱好
        \item 相处感觉良好
    \end{itemize}
    \item 如果对方不感兴趣了,要放下
    \item 要维持礼貌和尊重
    \item 要诚实且诚恳
    \item 要保持联络
    \item[\dash] 任何身体接触一定要获得许可
\end{itemize}

\subsection{约会不该做的事情}
\begin{itemize}[label=\dash]
    \item 不要一开始谈太私人的事情.包括谈太多个人信息(比如自己有ASD),会吓跑人家.
    \item 不要谈你的约会史.包括缺乏约会经验/之前不好的约会经验.
    \item 不要一开始就谈感觉.太急于表现会吓跑约会对象.
    \item 不要让关系发展太快,理由同上.
    \item 不要假设你们就是一对.容易之后产生误解和失落,而且也会吓跑对方.
    \item 不要大肆宣扬关系中的私密细节,尤其身体接触相关.
    \item 不要当个情感玩家(渣男/女),即同时约会多个人且互相不知道其他人的存在.
    \item 不要给对方压力(身体接触相关)
\end{itemize}

\subsection{处理伴侣提出的有关性要求的压力}
\begin{enumerate}
    \item 保持冷静
    \item 告诉对方(我不想要/这让我感觉不舒服)
    \item 给对方一个理由(我们还不够了解彼此/我还没准备好发生关系 )
    \item 用``我''开头的陈述告诉伴侣你的感受(我喜欢你,不过我觉得需要更多时间/我需要更多空间)
    \item 改变话题
    \item (如果对方仍然持续施压)给一个理由并离开
\end{enumerate}
永远记住\emph{约会/伴侣}是一种选择,而不是一种契约或者义务.
\warn 为什么共同兴趣如此重要,不只是因为有这个可以有话题,更多是在日常相处中,一方面一方总是假装对什么感兴趣很累(即扮演另一个人),另一方面这种扮演对关系大有害处.通过扮演获得的亲密关系最多只能保持到结婚前,之后的关系中更多的是被欺骗一方中的痛苦,这段关系很大可能是不幸的.

\Exer \exer 尝试使用上面的知识练习

\newpage
\section{处理意见相左}

\subsection{意见相左时的回应步骤}
\begin{enumerate}
    \item 保持冷静(深呼吸,从一数到十)
    \item 聆听对方的意见
    \item 重复对方说的,以表明同理心,且表示自己有认真听对方说.
    \item 用``我''开头的陈述解释自己的想法(而不只是直接告诉对方``你错了'')
    \item 说你很抱歉.即使没有做错的地方,也应该道歉(我很抱歉发生这件事/闹的不愉快).
    \item 尝试解决问题.比如:\begin{itemize}
        \item 告诉对方你会采取不同做法
        \item 问对方希望你怎么做
        \item 建议对方你希望对方做的事情(而非要求对方按照你的想法做)
    \end{itemize}
    \item[\Dot] 就算不能解决问题,也一定要保持冷静,接受别人的不同意.
\end{enumerate}

\subsection{提出不同意见的步骤}
\begin{enumerate}
    \item 等待合适的时机和场合
    \item 保持冷静
    \item 提议私下谈(我能找你私下谈点事情吗?)
    \item 用'我'开头的陈述解释想法
    \item 聆听对方的意见
    \item 重复对方的话
    \item 告诉对方你需要对方怎么做\begin{center}
    \begin{tabular}{l|l}
        \textbf{对方做法}&\textbf{表达方式}\\ \hline
        \textbf{未聆听}&``我需要你给我一个解释的机会.''\\
        \textbf{未重复我的话}&``你可以从我的观点来理解吗?''\\
        \textbf{未解释看法}&``你可以帮助我了解为什么你这样做吗?''\\
        \textbf{未抱歉}&``我想,如果你也感觉有点抱歉,我会感觉好很多.''\\
    \end{tabular}
    \end{center}
    \item 尝试解决问题\begin{center}
    \begin{tabular}{l|l}
        \textbf{自身愿望}&\textbf{表达方式}\\ \hline
        \textbf{告诉对方你会有不一样的做法}&``我需要你给我一个解释的机会.''\\
        \textbf{希望对方理解你这么做}&``你可以从我的观点来理解吗?''\\
        \textbf{建议对方你希望对方做的事情}&``你可以帮助我了解为什么你这样做吗?''\\
    \end{tabular}
    \end{center}
    \item 就算不能解决问题, 也一定要保持冷静, 接受别人的不同意见
\end{enumerate}
如同约会一样,永远记得\emph{朋友/伙伴}是一种选择.你不必和每个人做朋友,更不用讨每个人的喜欢.
\Exer \exer 练习处理意见相左.

\newpage
\section{处理直接霸凌}
有些人们喜欢嘲笑/欺负别人,是为了看你/周围其他人的反应.如果辩解或生气/羞愤/不反抗的话,会更容易被霸凌;而忽略/走开/告诉别人/嘲笑/打回去通常没有效果.

\subsection{处理嘲笑}
\begin{enumerate}
    \item[\dash] 不要忽略/走开/告诉其他人/嘲笑回去.男生之间也不要互相挖苦.
    \item 表现得它们所说的对你没有任何影响
    \item 表现得仿佛它们说的没有说服力或是愚蠢的
    \item 简短的口头反驳\begin{center}
        \begin{tabular}{|ccccc|}\hline
            随便你&是哦,所以呢&然后呢&你的重点是什么&我需要在乎吗\\ 
            为什么我需要在乎&面无表情的说``好好笑哦''&好严重噢&又怎么样&谁在乎\\
            哪有什么大不了的&有这么严重吗&随你怎么说&&\\\hline
        \end{tabular}
    \end{center}注意语调要显得无聊,找到合适的语句反驳.
    \item 非口语的反驳:翻白眼/耸肩/摇头
    \item 反驳嘲笑后离开.可以是移开眼神,或是慢慢离开.
    \item[\Dot] 问题改善以前,嘲笑可能会更严重
    \item[\Dot] 预期嘲笑者会再次尝试
    \item[\dash] 避免对有身体攻击性的人用嘲笑反驳
    \item[\dash] 避免对权威者用嘲笑反驳
\end{enumerate}

\subsection{处理尴尬的回馈}
\begin{center}
\begin{tabular}{c|l}
\textbf{尴尬范例}&\textbf{利用尴尬回馈的范例}\\\hline
\textbf{衣服}&考虑改变衣服,尝试往社交群体样式.规律洗换衣物,接受教练/家人/朋友的建议.\\
\textbf{体味}&每天使用除臭剂,规律沐浴洗发,并使用洗护剂.香水少量即可.\\
\textbf{头皮屑}&使用针对头皮屑的洗发露,或就诊.\\
\textbf{口腔卫生}&每天刷牙,使用牙线,清洁舌苔,使用漱口水和口香糖.\\
\textbf{幽默感}&注意幽默回馈,考虑少说一点笑话.对刚认识不久(半个月内)的人要正经一点.\\
\textbf{不寻常行为}&建议考虑改变或者停止那些行为.\\
\end{tabular}
\end{center}

\subsection{处理身体霸凌}
\begin{itemize}
    \item[\dash] 避开霸凌者,如果可能.
    \item 规划你的路线,比如避开霸凌者经常出现的地方.
    \item 霸凌者就在附近的时候,要保持低调.
    \item[\dash] 不要尝试和霸凌者做朋友,否则反而更注意到你,从而更经常霸凌你.
    \item[\dash] 不要挑战霸凌者,比如作弄/嘲笑/指出错误/给它添加麻烦.
    \item 和其他人在一起.落单的人最容易成为霸凌者欺负的对象.
    \item 当霸凌者在附近时,和权威者(如老师/管理人/老板/共同上级)在一起.可以不来往,但是可以有意无意的在旁边.
    \item 提出申诉.如果上面的都没有效果的话,向老师/部门主任/大学行政人员/监管人员(如宿管)/管理人/老板/人力资源部门/(如果极端的话)执法者提出.而且这时的一个重要的决定,一般需要一些协助.其实如果是负责的学校和老师的话,霸凌本身就不会明目张胆的发生(后果是严重的).但如果学校和老师不负责,最好直接转学校.
\end{itemize}

\newpage
\section{处理间接霸凌}

\subsection{处理网络霸凌}
\begin{itemize}
    \item[\dash] 避免征战,不要有反应.它们就是想要挑起你的反应,破坏你的心情,即使``随便你''这种都不要回.
    \item 有朋友撑腰,比如在多人的网络聊天过程中.同现实霸凌一样,没什么朋友的人容易被视为霸凌对象.
    \item 暂时离开社交网络,暂时也不要回其他人,免得对方继续攻击.
    \item 封锁(block)网络霸凌者,如屏蔽对方的邮箱及删除对方等.
    \item 保存证据.即保存任何威胁/污蔑/骚扰/侮辱的内容,因为之后可能要作为证据通报.如果是朋友圈这种,可以截图后删除内容.
    \item 举报,给如网络服务者/网络社区管理者/大学/主管等.
\end{itemize}

\subsection{处理谣言和闲话}
\begin{itemize}[label=\dash]
    \item 避免和好说闲话的人做朋友/为敌,避免告诉对方隐私/个人信息和激怒对方.
    \item 对谈话保存中立态度,即尽量和对方无关.比如在不给自己带来损失的前提下,不妨碍它们).
    \item 避免散播别人的谣言,因为可能会伤害到朋友和潜在朋友.
    \item 不要去证明谣言是错的,或表现出恼怒.这正中对方下怀.
    \item 表现出谣言对你没有作用,即并不会伤害到你.
    \item 不要质问散播谣言的人.这会导致争斗,对方可以更理直气壮的散步谣言.\\ 永远不要和这种散步谣言,有强盗逻辑的人争辩.
    \item[\Dot] 对于有人回在乎或者相信闲话的人表示惊讶.(潜台词:居然有人会在乎/相信这种事情 ,这些人很蠢.)
\end{itemize}
\subsection{散播和自己有关的谣言(反向传播)}
\begin{enumerate}
    \item 找一位支持你的朋友
    \item 找一个听众,让它好像不经意听到.
    \item 对朋友提起谣言(有听过和我有关的那个谣言吗?)
    \item 对有人在乎/相信谣言的事情表示惊讶(竟然回有人信那种话,真是见了鬼)
    \item 和其他支持你的朋友重复上面的步骤(增加荒谬的可信度)
\end{enumerate}
\warn 无论在现实还是网络中,面对谣言时,直觉中的回击都是不太正确的.

\section{向前迈进与毕业}
\subsection{向前迈进}
\begin{itemize}
    \item 通过前面的兴趣与社交活动 群体接纳与否的征兆, 可以帮助你更好的加入社交团体.
    \item 规律的朋友聚会.这样可以发展亲近的友谊,一般一周至少需要有一次聚会.
    \item 记住友谊和约会是一种选择.
    \item 持续进行社交指导.这可能需要教练的提醒,鼓励,支持.
\end{itemize}
\subsection{恭喜!}\emph{你已经通过了这门课程, 但是PEERS的技巧可以一直持续使用.}
\end{document}