\documentclass[UTF8]{article}
\usepackage{ctex,geometry}
\geometry{a4paper,left=1.5cm,right=1.5cm,top=1.5cm,bottom=1.5cm}
\title{禅与摩托车维修艺术}
\date{}
\author{Robert M. Pirsig}
\begin{document}
\maketitle
\section*{第一部分}
\subsection*{1}

\par 左手都不用从车把上抬起来,我低头看了一眼表,现在是早上八点半。虽然车速高达六十迈,但是迎面而来的风依旧潮热难忍。我不禁想,这一大早就已经这么闷热,到了下午可该如何是好啊!
\par 我们现在的位置是中部大草原,路旁的沼泽飘来刺鼻的气味。这些沼泽满布四周,大大小小数以千计,正适合猎鸭。我们正由Minnesota的Minneapolis朝西北的Dakotas前进。目前走的是双车道的旧公路,自从几年前有一条平行的四线干道通车后,这条路上的车辆就少多了。车子经过沼泽,空气突然变得清凉起来,而不一会儿过了沼泽,又恢复了原来的闷热。
\par 能骑摩托车来走一遭的确是件乐事,虽然这里不是什么名山大川,也没有寻幽览胜之处,但这正是它迷人之处。
\par 从这里走过,紧绷的神经便都松弛下来了,颠簸的水泥路两边是草坡和水烛\footnote{正式学名为香蒲,为生于水边的多年生草本植物,因茎的前端会生出圆柱状的小花繁生,形似蜡烛,故通称为水烛},并且长着水草的沼泽和更茂盛的水烛一直在前方绵延。有的时候四周又是一片开阔的水域,只要仔细瞧瞧就会远远看见在水烛边上栖息的野鸭,此外还有乌龟……你看,那儿有一只红翅膀的乌鸫\footnote{又名百舌,为一种生活于北美大草原的鸣禽}。
\par 我拍了拍Chris的膝盖,指给他看。
\par “什么事?”他大声嚷道。
\par “有一只乌鸫!”
\par 他嘟囔了句什么,我没有听见,就大声喊回去,“你说什么?”
\par 他一把掀开我头盔的后半部,喊道:“我已经看过好多只了,老爸。”
\par “喔!”我大声回应,然后点点头,的确十一岁大的孩子对红翅乌鸫是不会有什么感觉的。
\par 要对这事儿有感觉,需要上点儿年纪,对我而言,这感觉里面掺杂着许多他不曾有过的回忆。很久以前,那些寒风瑟瑟的早晨,沼泽中的水草都已枯黄,水烛在冷风的吹拂中摇曳,我们穿着高筒靴站在沼泽里,等待日出,等待猎鸭时候的到来,而四周踩过的烂泥正散发出一股刺鼻的气味。冬天的时候,沼泽结冰了,我踩在冰上,四周是枯萎的水烛,在我面前除了蒙蒙的天空,只剩下一片死寂和酷寒,这时候不会有乌鸫的踪迹。然而现在是七月,它们都回来了,处处显得生机勃勃,沼泽里面是一片唧唧的虫鸣和小鸟啁啾的欢闹之声,不知有多少生命正在我们周围呈现着盎然的生机,生生不息,代代相传。
\par 骑摩托车旅游和其他的方式完全不同。坐在汽车里你只是被局限在一个小空间之内,因为已经习惯了,你意识不到从车窗向外看风景和看电视差不多。
\par 你只是个被动的观众,景物只能呆板地从窗外飞驰而过。
\par 骑摩托车可就不同了。它没有什么车窗玻璃在面前阻挡你的视野,你会感到自己和大自然紧密地结合在了一起。
\par 你就处在景致之中,而不再是观众,你能感受到那种身临其境的震撼。脚下飞驰而过的是实实在在的水泥公路,和你走过的土地没有两样。它结结实实地躺在那儿,虽然因为车速快而显得模糊,但是你可以随时停车,及时感受它的存在,让那份踏实感深深印在你的脑海中。
\par 我和Chris以及那些骑在前面的朋友,正准备到Montana一游,或许还可以骑得更远一点也说不定。我们刻意避免按照固定的行程前进,宁可随心所欲地走走停停,因为旅游本身远比赶赴某一个目的地更加惬意。现在我们在度假,想走一走支线,石子铺的乡间小路是最好不过的选择了。然后才是州际干道,下下之选才是高速公路。我们打算好好欣赏一下沿途的风光景致,所以要好好享受旅游的过程,不会干那种在很短时间之内游览几个景点的煞风景的事。这样一来我们心情大好,崎岖的山路虽然漫长,但是骑摩托车却是一种享受——倾斜的身体可以顺着山势忽左忽右,不像在车厢里被晃得东倒西歪。要是一路上车子少那就更好了,同时也比较安全。
\par 我认为路边要是没有广告牌或是休息站什么的,景色一定更美:不论是路旁的树丛,地上的小草或是园里的果树都长到齐肩高,沿途时不时还有小孩向你挥手,也有大人从屋里走到廊前看看是谁经过。一旦你停车问路或是想了解什么当地的情况,你得到的回答往往出乎意料:他们会问你打哪儿来,已经骑了多久等等,热情而又滔滔不绝地和你神侃半天,简直比你还要兴奋。
\par 我们夫妻俩和一些老友迷上这种乡间小路已经有好些年了。当初为了调剂一下或是为了通往另一条干道而走捷径,都不免要骑上一段。每次我们都会惊讶于景色的美丽,骑回原路时便有一种轻松愉悦的感觉。我们经常这么骑,后来才明白道理其实很简单:这些乡间小路和一般的干道迥然不同,就连沿线居住的居民的生活步调和个性也不一样。他们一直都没有离开过本地,所以可以很悠闲地和你寒暄问候、谈天说地,那感觉好极了。反而是那些早就搬到城市里的人和他们的子子孙孙迷失了,忘记了这种情怀。这实在是我一个宝贵的发现。我在想,为什么我们这么久之后才会着迷。我们早已看过却仿佛没有看到,或者说是环境使我们视而不见,蒙骗了我们,让我们以为真正的生活是在大都市里,而这里只不过是落后的穷乡僻壤。这的确是件令人迷惘的事,就好像真理已经在敲你的门,而你却说:“走开,我正在寻找真理。”所以真理掉头就走了。哎,这种现象真是让人不解。
\par 然而我们一旦迷上这种旅游方式,就再也忘不了那些风景宜人的小路,忘不了那些消磨了很多个周末、夜晚和假日的美好时光。我们成了真正的乡野骑士迷,只要骑到那里就会有值得一看的景物。
\par 我们已经学会了如何在地图上目测出好的旅游路线。比如说,如果地图上路线很曲折那就对了,因为这表示可能有山丘在此。如果是由乡镇通往都市的干道那就糟了。最好的路线是前不着村、后不着店的那种,而且有一条便捷的副线。如果你出了一座大镇预备往东北走,那么肯定不可能一出城就走上好长一段路,往往你会先朝北走一阵子,然后再往东走,之后再往北走,然后就到了一条当地人才走的小路。
\par 走乡间小路最怕走迷了路。这些路往往只有当地人在走,他们都很熟悉路况,即使没有路标也不会有人迷路,所以就很少设置路标。就算设了,也只是小小的一块牌子放在草丛中,毫不起眼。
\par 而且往往只标示一次,错过了,那就算你倒霉。更过分的是,干线地图上所标示的小路经常出错,你会发现自己原先骑在双线道上,不久就变成单线道,最后竟来到一片草原,而前面已经没有路了;要不然你就被稀里糊涂地引到一个农家后院。
\par 所以我们得到的指引其实很少,只能靠着图示自己摸索。为了预防阴天时看不到阳光,我就随身携带一个罗盘,然后把地图用特殊的包装裹住,放在油箱上面。这样一来我就能知道离上一个岔口有多远,而前面的路又该怎么走。
\par 有这些工具的辅助,也没有什么目的地的压力,我们这一路行来顺畅得很,没有遇到什么麻烦事情。我们可以说几乎把整个美国大地都揽入怀中了。
\par 在劳动节和阵亡将士纪念日的周末,我们骑在路上,没有看到其他车辆的踪迹。没想到路过一条州干道的时候,竟然看到车子一辆接着一辆,一直排到很远的地方。车子里的人愁眉苦脸,在后排坐着的孩子早已不耐烦地大哭起来。我真希望能告诉他们一些事,但是他们只是绷着脸,一副十分匆忙的模样,所以也只好作罢。
\par 我已经看过这些沼泽不知多少回了,但是对我来说,每一次都是新鲜的。
\par 如果你以为沼泽大部分时候都是静谧温驯的,那你可就错了。你也可以说它们有些残忍和冷酷,这些都算是它们的特质。但是实际的情况却往往和你想的大相径庭。你看,那儿有一大群红翅乌鸫被我们的声音吓着了,从水烛里的鸟巢飞了出来。我又拍了拍Chris的膝盖……
\par 然后才突然想起他已经看过了。
\par “什么事?”他又嚷道。
\par “没事。”
\par “究竟是什么事?”
\par “只是看看你还在不在。”我回喊道,之后就不再说什么了。除非你很喜欢大声喊叫,否则一路上便很少说话,主要的精力都花在观赏风景和沉思上,想想自己看到了什么,听到了什么,看看天色如何,或是回忆一下往事,偶尔也看看摩托车的状况,欣赏一下我们来到的乡野。日子就是这样随意,忘掉时间,没有人会催促你,也不会担心浪费时间。
\par 接下来我想要谈谈我的想法。我们常常太忙而没有时间好好聊聊,结果日复一日地过着无聊的生活,单调乏味的日子让人几年后想起来不禁怀疑,究竟自己是怎么过的,而时间已悄悄溜走了。
\par 现在我们的确空下来了,我想谈一些我自己觉得颇为重要的事。
\par 我心里想的有一点类似于Chautauqua\footnote{Chautauqua,19 世纪末期美国的教育改革运动,起自于纽约的Chautauqua一地。由卫理公会的牧师Dr.John H. Vincent 及Ohio州的制造商Lewis Miller 倡导,于暑期时在野外举行教育集会,提供宗教和成人教育的课程方式,举凡娱乐、演戏、音乐、讨论、报告均有。每年约有5 万人参加。它的贡献在于促进函授教育的发展和暑期学校的兴起。1921 年时曾扩增至12000个社团,但与原发起组织无关,并有500万人参加过此活动。后来因为汽车、收音机、电影的崛起而消失}——这是我想到的惟一的名称——就像美国19 世纪末兴起的暑期野外学校。就在我们现在所身处的美国,借着一连串谈古论今的表演来寓教于乐,让大家的生活更有深度,有更多的领悟。不过Chautauqua因为收音机、电影和电视的出现而没落了,在我看来这种改变不见得是一种进步,虽然全美的思想交流更加快速便捷,但也似乎变得更浅陋。原先的河道已无法再承担这样的流量,它只有另觅新的出路。然而这样它就为两岸带来了更多的灾难。在这次Chautauqua当中,我不打算在脑海里挖掘任何新的河道,只想把旧的想法疏通一番,因为它已经被腐败发臭的思想和陈旧观念堵塞。“有什么新鲜事儿?”这是一个人们最感兴趣的问题,但是也最不着边际,可以没完没了地问下去。如果认真探讨它的答案,所得的只不过是一堆琐碎的跟风事物,这些都是将来的淤泥。我宁可问这样的问题:“什么是最好的?”这个问题能疏通河道而非拓宽它。人类历史中有些时代,思想的河道挖凿得太深,以至于无法修改,从而再也无法出现任何新气象,这时追求“最好的”就成了僵化的教条——但我们的现状并非如此。目前的普遍思想似乎早已漫过两岸,丧失了主要的目标和方向,淹没了低洼地区,把高地孤立起来,切断了它和其他地区的联系。除了河水本身浪费精力的躁动外,像这样到处流溢并没有任何意义,所以目前似乎真的到了需要疏通的时候了。
\par 骑车走在前面的是John Sutherland和他太太Sylvia,他们已经驶入路边的野餐区。是该伸展一下身体了。我把车子停在他们旁边,Sylvia正拿下头盔,把头发甩开,而John则在一旁拉起他那辆宝马的脚架。我们都没说什么,在一起旅游这么久了,彼此已经太熟悉了,只要交换个眼神就知道对方在想什么。现在,我们只是静静地四处望望。
\par 一大早野餐区不见半个人影,只有我们在此,仿佛这么辽阔的空间都属于我们了。John走过草丛,来到一座铁铸的水泵前打水上来喝。Chris则从树下走过,越过一座长满杂草的小土墩,走到小溪旁,而我只顾着四下眺望。
\par 不一会儿,Sylvia坐到野餐桌旁的木板凳上,伸直双腿,交替着慢慢地举起来,但是却低着头,沉默不语,似乎心情不好。我问她怎么了,她抬起头看了看我,又低下去。
\par “都是那些迎面而来的车子里的人,”她说,“头一个脸上的表情看起来这么难看,第二个也是。一个接一个,每一个人都很不高兴。”
\par “他们只是开车去上班啊。”她观察得很仔细,但是这似乎也没有什么不对劲。“你知道,为了工作嘛。”我重复了一遍。“星期一早上总是睡眼惺忪的,有谁上班还会咧着嘴笑啊?”
\par “我是指他们看起来失魂落魄的,”她说,“好像全都是行尸走肉,怎么像是去奔丧一样!”说完她便把两脚放下,不动了。
\par 我了解她的意思,但是她并没有说出一番道理。人工作就是为了要活下去,原本就是这么回事儿。“我正在看沼泽。”我说。
\par 过了一会儿,她抬起头来说:“你看到了什么?”
\par “那儿有一大群红翅乌鸫。我们经过的时候它们突然全部飞起来了。”
\par “哦。”
\par “真高兴再看到它们。你知道,它们让我回想起好多事情。”
\par 她想了一会儿,站了起来。看到身后那些绿阴深浓的树,她笑了。她明白我话里的意思,她确实是个善解人意的女人。
\par “的确,”她说,“它们真美。”
\par “多看看它们吧。”我说。
\par “一定。”
\par John回来了,他检查了一下摩托车发动的情形,然后又调整车上绑东西的绳索;再打开车上的行李袋,在里面乱翻了一阵,然后拿出一些工具放到地上,“你们如果要用绳子过来拿,别客气,”他说。“老天,我带的东西太多了,是我需要的五倍。”
\par “现在还不用。”我答道。
\par “火柴,”他一边说一边还在翻,“防晒油、梳子、鞋带……鞋带?我们要鞋带做什么?”
\par “先不提这个。”Sylvia说,他们面无表情地看了看对方,然后又一起朝我望来。
\par “鞋带随时会断。”我一本正经地说,他们笑了,但不是对着彼此笑。
\par Chris很快就回来了,大家该起程上路了。Chris整装就座的时候,他们已经发动车子,Sylvia朝我们挥挥手,大家又骑上干道,不一会儿,只见他们远远地骑在了前头。
\par 让这趟旅行带有Chautauqua的意味和他们两位有关。虽然在好几个月以前,可能连我自己也不清楚,这一切是受了他们之间隐隐暗藏的摩擦所影响。
\par 我想在任何婚姻里摩擦都免不了,但是他们的情形比较不幸,不过这是对我而言。
\par 他们之间不是个性不合,而是别的原因。双方都没有错,但是都没有办法解决,就连我也不一定有化解的方法,只有些个人的看法。
\par 这些看法始于我和John对一件小事有了不同的意见:一个人保养车子究竟应该到什么程度?对我来说,尽量使用买摩托车时附送的小工具箱和使用手册,然后自己保养,是一件再自然不过的事;但是John反对这么做,他认为应该让师傅负责修理和保养才不会出错。
\par 这两种看法都很平常,如果我们没有骑摩托车一起旅行,没有坐在乡村路旁的野店一起喝啤酒,或是随兴闲聊,那么这点意见上的分歧就不会扩大。只要我们谈的内容是天气、路况、民情、往事或是新闻,谈话自然就很愉快。然而一提到车况,话就说不下去了。大家都保持缄默。就好像是两个老友,一个是天主教徒,另一个是基督徒,两人一起喝啤酒,享受人生,只要一谈到节育,谈话马上中断。
\par 当然在你发现有这种状况的时候,就好像发现自己补好的牙又脱落了,你绝对不会袖手不管的,你会到处寻找,找到了再塞进去,塞紧了还要好好想想是怎么掉的。你会花这么多时间,并不是因为这件事有趣,而是因为它萦绕在你心头让你放心不下。只要我一谈到摩托车保养的问题,他就会坐立不安。这样一来只会使我想更进一步地探索下去,并不是故意想激怒他,而是因为他的不安似乎象征了某些隐而未显的问题。
\par 当你谈到节育的时候,横梗在你们中间的并不是人口多寡的问题,那只是表象,真正起冲突的是信心。基督教看重的是实际的社会问题,而天主教徒则认为那是亵渎天主的权威。你可以滔滔不绝地辩解计划生育的重要性,一直到你自己都厌烦了,然而仍无法说服对方,因为他并不认为符合社会实际的需要有何好处,他自有比实用更重要的价值观。
\par John的情形就是这样,我可以滔滔不绝地讲解摩托车保养的实际效用,一直说到我喉咙沙哑,但是John仍然无动于衷,只要一谈到这方面,他就一脸茫然,不是改变话题就是看到别处去。他不想听我说下去。
\par 在这方面Sylvia倒是和他意见一致,反应甚至更激烈。在她比较体贴的时候,她会说:“这根本是风马牛不相及的两件事。”脾气来的时候就说:“简直是胡说八道。”他们根本不想了解,连听都不想听。我越想深入了解为什么我如此被技术工作所吸引,而他们却如此憎恨,原因就变得越模糊不清。结果原本只是小小的歧见,最后却演变成一道鸿沟。
\par 很明显,他们并不是能力不足,夫妻俩都属于聪明之辈,只要他们肯花心思,在一个半钟头之内就学得会如何靠听发动机的声音修理车子,这样不但能省下大量的时间和金钱,更不必时时刻刻担心车子会出状况。他们应该知道这一点,或许也可能不知道,我不清楚。
\par 我们从来没有讨论过这个问题,最好还是顺其自然吧。
\par 但是我记得有一次在Minnesota的Savage,当时天气差点把我热昏了,我们在酒吧里待了大约一个钟头,出来的时候摩托车晒得几乎没法骑上去。我先发动好准备上路,但是John仍然在用脚踩发动器,我闻到一股汽油味,就像炼油厂传出来的一样,便告诉了他,以为这样足以提醒他是发动机湿了,所以无法发动。
\par “对,我也闻到了。”他边说边继续踩,不停地用力踩,有时还跳起来踩,我不知道该说什么。一直到他踩得气喘如牛,汗流浃背,再也踩不动时,我才建议他不妨把火花塞拿出来晾干,让汽缸通通风,然后我们可以回去喝杯啤酒再出来。
\par 喔,我的天,真糟糕,他根本不拿工具修理。
\par “什么工具?”
\par “就是把那一堆工具拿出来。”
\par “它没有理由发动不起来。这是一台全新的摩托,而且我也完全照手册上说的去做。你看,我照他们说的把阻风门拉到底。”
\par “阻风门拉到底?”
\par “手册上是这么说的。”
\par “那是发动机冷的时候才这么做!”
\par “我们至少进去了半个钟头。”他说。
\par 我听了暗吃一惊,“但是John,你知道今天天气有多热。”我说,“即使是大冷天也得半个多钟头才能散热到可以发动。”
\par 他抓抓头,“那为什么不在手册里说明呢?”他打开阻风门,再一踩就发动了。“这就对了。”他很高兴地说。
\par 就在第二天,仍在附近地区,同样的情况又发生了一次。这回我决定什么也不说,我太太催我过去助他一臂之力,但是我摇摇头,我告诉她,除非他真正感觉需要别人的帮助,否则别人的介入只会引起他的厌烦。所以我们就走到一旁,坐在阴凉的地方等。
\par 在他发动不了的时候他对Sylvia特别客气,这表示他已经愤怒到极点了,而Sylvia在一旁露出“天啊,又来了”的表情。其实只要他问我一句,我一定会立刻上前帮助他,但是他并没有这么做。他大约花了十五分钟才把车子发动。
\par 后来我们在米尼通卡湖畔喝啤酒,大伙儿都围着桌子喝酒的时候,只有他一言不发。我看得出来,他是为刚才的事耿耿于怀。过了好一阵子,他的心情稍微放松了,才说:“你知道……刚才发动不了的时候还真是……让我火冒三丈;心想非把它发动起来不可。”开口说话似乎让他轻松了一些。他又说:“他们店里只剩下这一台破车。他们也不知道该拿它怎么办,是退回工厂,还是随便卖掉,结果看到我进店里去,正巧我身上带了一千八百元,就这样做了他们的替死鬼。”
\par 我几乎是半请求地希望他试着去听发动机的声音,结果他试得很辛苦,但问题还是一样,他干脆回去和大伙儿再喝一杯,话题就到此为止。
\par 他并不是固执的人,心胸也不狭窄,既不懒惰也不愚蠢,所以这件事要解释起来还挺不容易的,有些神秘感,因为在没有答案的地方穷打转是很荒谬的。
\par 我曾经想过,是不是我在这方面比较特别,但是这个说法并不成立,大部分骑摩托车旅游的人都知道如何调整发动机。开汽车的人通常不会去碰发动机,不论多小的城镇都会有一间修理店,提供车主昂贵的、专门的工具和诊断用的设备,这些都是一般车主不会购买的。
\par 同时汽车的发动机比摩托车复杂多了,一般人也不易了解,所以不自备修理工具还有情可原;但是John骑的是宝马R60,我敢打赌由这里至盐湖城不会有任何修理店,假如他的指针或是火花塞烧坏了,他就完了。我知道他没有多预备一套,他根本连它是什么也不知道,万一在South Dakota或Montana用坏了,我真不晓得他该怎么办,或许把车子卖给印第安人吧。现在我知道他在做什么,他在小心谨慎地避免谈起这方面的问题,他想宝马的车子最有名的就是很少在路上发生机械方面的故障,这就是他的如意算盘。
\par 起初我认为,这只是他们在对待摩托车时特有的态度,但是后来才发现情形并非如此……有一天我在他们家等着一起上路,我注意到水龙头在滴水,我记得上次就已经滴了,事实上已经滴了很久。我提醒他这件事,John告诉我,他换过新的皮圈但还是滴水,他说了这些就不再提了,也就是说事情到此为止。
\par 如果你试过修理水龙头,但是情况依旧,那就表示你命中注定有个会滴水的水龙头。
\par 我很惊讶,水龙头这样日复一日、年复一年地滴滴答答地响,他们难道不会神经衰弱吗?然而我发现他们一点都不担心,也不去注意这件事。所以我的结论是他们不怕被水龙头打扰。有些人的确如此。我不记得是什么改变了这个判断……好像是Sylvia正要说话,而滴水声又特别大,无意中引起她情绪上的变化。她的声音一向很轻柔,有一天她想大声说话压过滴水声,这时候孩子们走进来打断她的话,她不禁发起脾气来,仿佛是因为滴水声引起的。事实上是这两件事引起的,而让我惊讶的是她并没有怪罪到水龙头上,她甚至有意不去怪罪它。其实她早已注意到水龙头的问题,只是刻意压制自己的怒气,那个该死的水龙头几乎要把她逼疯了!但是她仿佛有隐情,不肯承认这个问题有多严重。
\par 我很奇怪,为什么要对水龙头压抑自己的怒火?
\par 我想起摩托车的问题,再加上我头顶上方坏掉的灯泡,啊,事情明白了!
\par 问题不在于摩托车,也不在于水龙头,问题在于他们无法忍受高科技的产物,这样一来,发生的各种状况便明朗起来了,我知道是因为科技的关系。Sylvia曾经很不喜欢一个朋友,因为对方认为电脑程序设计是很有创意的东西。
\par 而他们夫妻的绘画和相片里完全没有跟科技有关的景物。当然,我想她还不至于对水龙头大发脾气。通常你很容易对深深厌恶的对象压抑自己一时的怒气。
\par 而John只要一碰到修理车子的问题就会沉默下来,即使他已经很明显地在为此受苦。你只要稍加注意就会明白,这些都是科技惹的祸。这就是为什么他们要骑着摩托车到乡野去享受阳光和新鲜空气。而我总是把他们不愿意去面对的问题拉到台面上来,因此使他们二人十分尴尬。只要我们一谈到这方面的问题,谈话就会中断。
\par 还有其他的事情也解释得通。谈到痛苦的字眼时,他们是用“它”或“它们”来代替,比如说:“避不开它的。”
\par 如果我问:“避开什么?”他们就会回答我:“整个环境”或是“整个组织结构”,甚至是“整个体系”。Sylvia有一次甚至带着保护自己的口吻说:“当然,你知道如何驾驭它。”她这么说让我得意了一下,但是我有些不好意思地问她什么是“它”,心里有些困惑,我以为是比科技更神秘的东西。但是现在我知道,她所指的“它”虽不是全部,但也主要是指科技。然而这么说也不完全对,它应该是指来自于科技的一股力量,没有明确的定义,而且缺乏人性、机械化、了无生气,是一头瞎了眼的怪兽,一股死气沉沉的力量。他们夫妻俩觉得很恐怖,因而试图尽量避开它,却又明知那是不可能的。我的用词严重了些,但是实际情况的确如此。虽然总会有人了解它驾驭它,但那些人是工程师。他们在描述自己的工作时用的是非人性的语言,不论你听过多少回,也无法了解其中的意义。而和科技有关的怪物正吞噬了大片的土地,污染了空气和湖泊,人类既无法打击它们,也无法逃避得了。
\par 这种态度不难理解,在你经过大城市的工业区时,你就会看到整片所谓的科技区。门前围了高高的铁丝网,大门紧锁,告示牌上写着“禁止跨越”。在一片污浊的空气之后,你看到的是奇形怪状而又丑陋的金属物和砖块,也不知用途为何。它的主人你永远也见不着,它为什么在那儿也没有人知道,所以你感受到的只是一股莫名的疏离感,仿佛你并不属于那儿。它的主人和知其来由的人可不希望你在附近闲逛,这些工厂让你在自己的土地上竟有陌生的感觉。它特殊的形状、外观还有神秘感,一切都在叫你“滚开”。你知道这一切总有解释,而且它们毫无疑问地对人类间接地有些益处,但是这些益处你没看见,你只看见“禁止跨越”和“保持距离”的牌子,你只看见人们像蝼蚁一样为这些庞然巨物做工。于是你想,即使我是它们的一分子,也不过是另一只做苦役的蝼蚁罢了。这种感觉十分可怕,我想这就和他们夫妻俩无以名状的态度有关。任何和阀门、轴心、扳手沾上边的东西,都属于非人的世界,所以他们宁可不去想它,甚至不愿和它有任何关连。
\par 如果情形真是如此,那么他们并不是惟一有这种想法的人。毫无疑问地,他们只是忠于自己的感觉,而没有刻意模仿别人。但是其他的人也是忠于自己的感觉,没有模仿别人。所以如果你以记者的角度来看此事,就会发现有一场不知来源的群众运动正在逐渐成形。他们打着反科技的旗号,高喊:“科技滚蛋,搬到别处去。”然而在人们的脑海里仍然残存着一丝理智,没有工厂就没有工作,就没有相当的生活水准。但是,人们头脑中有太多的力量胜过了理智,只要憎恨科技的情绪超过它,那么残存的一丝理智便会瓦解。
\par 有人封这种反科技的人为“披头士”或是“嬉皮”,但是人并不会因为这样一个封号就产生归属感,John夫妇如此,大多数人也是如此。何况做这样渺小的一分子正是他们所厌恶的。科技正是贬低他们的帮凶,所以他们厌恶科技。截至目前为止,还仅限于被动的排斥,尽可能地逃到郊外去,但是情况不一定非如此被动不可。
\par 在摩托车维修方面我并不同意他们的看法,并不是我没有同情心,而是我认为他们的逃避和厌恶只是一种自欺的行为。佛陀或是耶稣坐在电脑和变速器的齿轮旁边修行会像坐在山顶和莲花座上一样自在。如果情形不是如此,那无异于亵渎了佛陀——也就是亵渎了你自己。这就是我想在这次Chautauqua旅程当中讨论的主题。
\par 我们已经离开沼泽区了,但是空气湿度仍然十分高——高到你可以直接看到太阳周围那圈黄色的光晕,就好像雾天看到的一样。但是我们现在是在乡间的绿野,农舍显得很干净,洁白而又清新,并没有出现一点雾气。
\subsection*{2}
\par 行经的路曲折复曲折……我们偶尔停下来休息,吃顿午餐,顺便聊一聊,然后再专心地骑下去,摆在眼前的是条漫漫长路。到了下午开始有些倦意,正好与第一天早上的兴奋相抵。目前我们行进的速度不快也不慢。
\par 迎面吹来的是西南风,我们的车子斜切进风里,仿佛要感受一下风的威力。
\par 最近我觉得这条路有些怪异,总有些令我们担心,好像有人在监视或跟踪我们。
\par 然而前头一辆车也没有,后面只有远远落后的John夫妇。
\par 我们尚未进入Dakota,但是辽阔的田野告诉我们近了。有些田里种着亚麻,蓝色的花朵随风摇曳,远远望过去像是起伏的波浪。山丘的广袤也是少见的,视线所及除了大地就是高远的苍天。
\par 远处的农舍小到几乎消失在视线之外,一路行来,越来越觉得天地开阔起来。
\par 在中部大草原和大草原之间并没有明显的界限,就在你不知不觉中已经改变了。就仿佛你由波涛拍岸的港口出发,不一会儿只觉得海浪深深地起伏着,回首一望,已经不见陆地的踪影。这一带的树也比较少,我忽然发现它们都是人工种植的,围着房舍,成排地立在田野间用来防风。没有种树的地方只长草,有的时候还夹杂着野花和野草,既没有灌木也没有小树。现在我们到达草原了。
\par 我有一种感觉,我们之中没有人知道七月里在草原待上四天会是个什么情景。如果是开车旅行的话,脑海中的印象只是一片平坦和空旷,极为单调乏味,一连开了几个小时之后,仍然看不见要往何处去,一路上都是笔直的道路,不禁令人怀疑究竟还要开多久才会有人烟。
\par John有些担心Sylvia会不适应这种状况,想要她搭飞机直接飞到Montana的比林斯,但是Sylvia和我都劝他打消这个主意。我认为只有在情绪不对的时候,身体上的不适才更加明显,那时你就会把不适的原因归咎于环境。但是如果情绪很正常的话,身体上的不适就无关紧要了。看看Sylvia,我不觉得她有任何不快。
\par 而且如果搭飞机抵达Loki山,你只会觉得景致很美,但是如果你是经过几天辛苦的旅程,通过这一片大草原,才抵达Loki山,那么你会从另一个角度来看它,那里仿佛是你的目标,是你的应许之地。如果John、Chris和我到达的时候是这种感受,而Sylvia又是另外一种,那么会引起摩擦,它比我们一路上从Dakota所感受到的酷热和单调还严重。反正我喜欢和她说话,我也是为自己着想。
\par 我这么想,在我凝视这些草原的时候可以指点她一起看,我想她会接受的。
\par 希望她能感受到我已经放弃告诉别人的事,就是那些其他都不存在,只有它存在它受到注意的事。她一向住在城里,似乎常会因为单调乏味的生活而郁闷,然而我希望她能接受这种单调,这种来自于一望无际的草原和风的单调——就在这里,而我无以名之。
\par 现在我看到了天边一些别人没有发现的东西。在远远的西南边——你只能从这边的山顶看得见——天际有一道黑边。暴风雨要来了,或许一直使我惴惴不安的就是这件事,我刻意不去想它,但是我早就知道在这种湿度和风速下,暴风雨极可能会来。真糟糕,第一天上路就碰上恶劣的天气。不过我以前提过,骑摩托车旅游要的就是身临其境,而不是冷眼旁观,暴风雨自是不可避免的一环。
\par 如果只是雷雨云或是狂风还可以骑一阵子,但是这次来的不是,那条黑长的云前面没有任何卷云,所以是冷锋。
\par 而冷锋打从西南来的时候特别强烈,通常会伴有飓风。飓风来的时候,最好找个地方避一下,等它过了再出来。它们来的时间不会很长,走了之后会带来凉爽的空气,骑起摩托车十分舒畅。
\par 最糟糕的莫过于暖锋,它们一来就好几天。我记得几年前Chris和我曾骑车到加拿大一游,走了一百三十英里的时候遇上了一道暖锋,虽然事前有许多征兆,但是我们当时并不明白。那次旅游的情形真可说是难以言表而且十分凄惨。
\par 当时我们骑的是六匹半马力的摩托车,载着超重的行李,旅游的常识又十分欠缺。车子只能跑四十五英里,而且是迎着风走,再加上它不是专门旅游用车,所以骑起来十分吃力。头天晚上我们骑到北部森林中的一座大湖边,就在风雨交加的情形下搭起帐篷。大雨下了一整晚。我忘了沿帐篷边挖上一道沟,结果凌晨两点的时候雨水涌了进来,浸湿了我们的睡袋。到了第二天早上,我们全身都湿透了,加上睡眠不足,心情很坏。我以为继续上路之后不久雨就会停,结果并没有这么好运,到了早上十点,天色暗到所有的车子都把车灯打开,然后又狠狠地下了一阵大雨。
\par 我们穿的是斗篷,昨晚曾用来搭帐篷,现在它们被风吹得像船帆一样,使车速慢到了三十英里。路上的积水有两英寸深,响雷和闪电就在我们身旁呼啸而过。我还记得有一辆车经过,坐在里面那位妇人吃惊地望着我们,不知道在这种天气里我们还骑车做什么。
\par 车子慢下来了,先是二十五英里,然后是二十英里,一直到它开始出现劈里啪啦的响声,然后车速降到五六英里。
\par 我们来到一座废弃的加油站,旁边是一座林场,树木早已被砍光了,我们赶忙进去躲雨。
\par 那个时候我就和现在的John一样,对摩托车的维修所知不多,我还记得我把斗篷举到头上,以防雨水滴到油箱中,然后用两腿摇车子,里面似乎还有汽油。
\par 我又检查了一下火花塞,看看仪表和汽化器,然后再踩发动器,一直到我筋疲力尽。
\par 进了加油站,里面还有啤酒屋和餐厅,我们吃了一份全熟的牛排之后,出来再试着发动车子。Chris在一旁不知轻重地一直问问题,问得我火冒三丈。
\par 最后我看发动不了就算了,结果冲他而来的怒气也就消了。我小心地告诉他玩完了,这次度假我们不准备骑车上路了。
\par Chris建议我检查一下汽油的存量——这我已经做过了,或是去找修理师傅,但是附近根本没有任何修理店,只有砍下来的松树、灌木和大雨。
\par 我们坐在路旁的草丛里,沮丧极了。
\par 我两眼呆呆地望着一旁的树和灌木,耐心地回答Chris所有的问题,幸而他问得越来越少。最后他终于明白我们没法再继续骑下去了,于是大哭起来。我想那个时候他有八岁了。
\par 我们搭便车回到城里,租了一辆拖车联结在我们的车子后面,回到原地把摩托车载回来,然后开汽车重新开始旅行。但是感受却不一样了,而且也没能真正享受旅游的乐趣。
\par 假期结束后两个礼拜,有一天下班后,我又把汽化器拿出来研究,想看看问题究竟出在哪里,但是仍然看不出个所以然。然后我打算清洗汽化器,于是打开油箱塞,竟然没有半滴油流出来!
\par 我真的不相信会发生这种事,到现在还是不相信。
\par 因为这个疏忽,我责怪自己不下一百次,我想到了最后我还是不会原谅自己的。很明显,我听到的油箱里的声音其实是从备用油箱里发出来的;我没有仔细检查,因为我以为发动机熄火的问题是下雨造成的,那个时候我还没想到自己这样骤下结论有多么愚蠢。现在我们骑的是二十八匹马力的摩托车,而我非常认真地保养它。
\par John的车子突然超过我的,他向下摆手要我们停下来,于是我们把车速慢下来,在铺了碎石子的路边找了一块空地,准备把车子停下来,路边的水泥很粗糙,石子也铺得很松散,我对他突如其来的举动很不满意。
\par Chris问:“我们停下来做什么?”
\par John说:“我想我们错过岔路了。”
\par 我回头看看,什么也没有看见,我说:“我没有看见任何标示。”
\par John摇摇头说:“和谷仓的门一样大。”
\par “真的?”
\par 他和Sylvia都点点头。
\par 他靠过来,然后弯身研究我的地图,指了指该转弯的地方,还有上方的一条高速公路。“我们已经过了这条高速公路。”他说,我知道他说的没错,因此有些不好意思,我问:“究竟是要回头呢,还是要继续往前走?”
\par 他想了一下:“我想没有理由走回头路。好吧!我们继续往前走,反正我们总会走到那儿。”
\par 我跟在他们后面一直想,为什么会发生这种事呢?我几乎没有注意到高速公路,而且刚才我也忘了告诉他们暴风雨要来的事,事情有些乱了套。
\par 暴风雨的云带现在更宽了,但是并不如我想像中发展得那么快。这样一来就更不妙了,因为它们如果来得快便也去得快。但是一旦发展得较慢,很可能我们被困住的时间会更长。
\par 我用牙齿把一只手套咬下来,伸手去摸发动机边上的铝盖。目前的温度还算正常,虽然已经热到无法把手停留在上面,但是还不至于把手烫伤,所以这一切都还算是正常的。
\par 像这种气冷式发动机,如果过热的话会造成发动机的故障,这个发动机就曾经遇到过一次……事实上是三次,所以我经常检查它,就像检查有心脏病的人一样,虽然目前看起来仍然很正常。
\par 出毛病的时候,活塞因为过热而膨胀,很容易就卡住汽缸壁,有的时候甚至会熔化。它会卡住发动机和后轮,造成突然刹车。这辆车第一次出现这种问题的时候,害得我整个人都冲到前轮的上方,后面的人几乎趴在我身上,三十分钟之后,活塞活动自如了,车子才又能正常运转。但是我仍然在路边停下来,看看究竟出了什么问题。后面的人只会问:“你停下来做什么?”
\par 我耸了耸肩,和他一样茫然地站在那儿,傻傻地看着别人的车子从身旁呼啸而过。发动机当时非常热,周围的空气都受到传染而微微地震颤起来。我们几乎可以看到热力所发射出来的光芒。
\par 如果我将手指沾湿放上去,它一定会像碰到热铁一样嗞嗞地响起来。因此我们就只能慢慢地骑回家了。一听发动机的声音就知道是活塞出了问题,需要大修一番。
\par 我把这辆车送进了修理店,我可不想插手。很可能需要买其他的零部件或是专门的工具,然后再花上许多无谓的时间,我既然能在短时间之内让别人做好,就不需要自己做,这有些类似John的态度。
\par 这家店和我以前去过的那一家不同,里面的师傅和以前的也不同。以前的师傅看起来像是古代的战士,而现在的这些看起来则像小孩子。他们把收音机的音量开到最大,然后在四周蹦蹦跳跳地一边走来走去,一边聊着天,似乎并没有注意到我的存在。最后终于有一个人走过来,听我说是活塞的问题,他就说:“哦!是梃杆出了问题吗?”
\par 梃杆出了问题吗?那个时候我就应该知道会有怎样的下场。
\par 两个礼拜以后我付了一百四十美元的账,然后小心谨慎地低速行驶,骑了大约一千英里之后才恢复正常。但是一骑到时速七十五英里,毛病就又出现了;降骑到时速三十英里,又恢复了正常,情形和以前一样。于是我就把车子送回店里去修,但是他们反倒责怪我使用不当,争论了一阵儿之后,我们都同意打开检查,结果是,他们决定自己做一次高速的路试。
\par 毛病再次出现。
\par 在这次大修之后两个月,他们更换了汽缸,然后换上较大的主汽化器喷嘴,然后使运转的速度减慢,使发动机尽可能不会过热,然后告诉我不要骑得太快。
\par 发动机里面有不少的油脂,而且无法发动。我发现火花塞与高压电线松了,于是我把它们接上去,然后再启动,结果现在真的出现梃杆的杂音,他们并没有帮我调整梃杆。我把这个告诉他们,修车的小伙子就拿了一把可调整的扳手过来,结果他方法不对,很快就把铝制的梃杆盖子弄坏了。
\par 他说:“我们仓库里还有存货。”
\par 我点点头。
\par 他拿了一把榔头和錾子,要把它们敲松,然而他的錾子却把铝盖凿穿了,我看见錾子直接撞到了发动机头上,后来他的榔头没能打到錾子上,结果把两片散热片给打破了。
\par 我冷静地说:“不要再敲了。”心里觉得这简直是一场恶梦,“请你给我一些新的盖子,就让它这样好了。”
\par 我赶快离开这个地方,梃杆的杂音,梃杆的盖子也坏了,发动机里又都是油脂。骑回去的路上,我发现时速二十英里左右的时候就会有强烈的震动。我在路边停下,发现四个发动机接合螺钉中的两个不见了,还有一个的螺母丢了,所以整个发动机的接合螺钉就只剩下一个了。上盖凸轮的链条松紧控制器的螺钉也不见了,这就意味着调整梃杆也没有用了。这真是一场恶梦。
\par 我总是想到John把自己的宝马车子交给别人修理的事,这个问题我从来没跟他谈过,或许我应该和他谈谈了。
\par 几个礼拜之后,我找到故障的原因,在内部供油系统上有一根二十五分的销子被剪断了,以至于在速度高的时候,油没有办法流进来。
\par 为什么会发生这种事情呢?这个问题不断在我脑海中出现,这就是我想要写这本书的原因。为什么他们的动作这样粗鲁呢?他们不像John和Sylvia一样害怕科技,他们都是专门人员,然而做起事来却像猩猩一样,没有真正地投入,似乎没有明显的原因。我试着回想那间修理店,就是让我做恶梦的那个地方,想要找出问题的真正答案。
\par 那架收音机是一条线索,一边工作一边听音乐是没有办法真正思考的,或许他们并不认为自己的工作需要任何的思考,只不过是玩弄几把扳手罢了。如果你一边工作一边听音乐或许会更愉快一些。
\par 他们动作的速度是另外一条线索,他们把东西到处丢,而且也不记得丢在哪里。如果你不反省一番,你就不知道这样做往往会浪费时间,而且成效不佳——也就是说需要花更多的钱。
\par 但是最重要的线索似乎是他们脸上的表情。然而实在很难解释,虽然他们看起来很随和、友善、轻松自在,但是却没有投入工作之中,他们就像旁观者一样,你会觉得他们只是在那儿晃来晃去,然后接过别人递给他们的扳手。他们对自己的工作没有认同感,不会说:“我是师傅。”一旦到了下午五点,八个小时一满,你知道他们会立刻放下手中的工作,即刻离开,然后尽可能地不去想他们的工作。在这一方面,他们与John和Sylvia一样,虽然想运用科技的成果,但是却不愿和它发生任何关系。或者说他们之间的确有关系,但是他们都没有投身其中,而保持冷淡疏离的态度,他们参与了这方面的工作,但是却没有真正地关心它。
\par 这些修理师傅没有发现销子断了,那是前一个修理师傅在组合侧盖板的时候,不小心剪断的。我记得以前的车主说过,有一位修理员告诉他侧盖板很难盖好,这就是原因了。一般摩托车手册中都会提到这一点,但是他太匆忙而疏忽了。
\par 在我编辑电脑手册的时候,也在想这个问题。一年当中我有十一个月都在编写这方面的手册,我知道一般这方面的资料都充满了错误,以至于解释不清,而且漏掉了不少重要的资料。有的时候需要读上五六遍才能略微了解它们的意思。但是让我惊讶的是,这些手册编写者的态度和这些修理人员的态度一样,竟然都是旁观者,所以它们可以被称为旁观者的手册。在字里行间,你隐约可以嗅到这样的意味:“这是机器,它和周围环境中的一切都没有关系,和你也没有关系,你和它也没有关系;你只需要懂得操纵某些开关,维持电压的强度,检查某些毛病等等。”就这么一回事。修理人员对这些机器的态度就和这些手册所透露出来的态度是一样的,都是保持旁观者的立场。于是我联想到市面上没有一本手册谈到保养、维修摩托车究竟是怎么一回事,这是最重要的一点。人们认为关心自己所做的事一点都不重要,要么就视之为理所当然。
\par 在这次旅游当中,我想应该注意这一点,更深入地研究,看看是否能够了解究竟是什么把人和人的工作分离开来,进而了解二十世纪的人究竟是出了什么问题。我并不想仓促行事,因为仓促本身就是二十世纪最要不得的态度,当你做某件事的时候,一旦想要求快,就表示你再也不关心它,而想去做别的事。所以我想慢慢来,用我找到被剪断了的销子的态度,有了这种态度才能发现原因,这样才能仔细而且透彻地进行这件事,除此之外,别无他法。
\par 我突然注意到,大地现在变得一片平坦,没有小丘,甚至也没有任何凸起之处,这表示我们已经进入红河谷,很快就会到Dakota了。
\subsection*{3}
\par 在我们出了红河谷的时候,暴风雨的云层似乎就在我们左右。
\par John和我在Breckenridge讨论过,决定继续走下去,直到必须停下来为止。
\par 但是我们走不了太久了,太阳已经被遮住,迎面吹来的风很冷,我们笼罩在一片灰暗的雨云当中。
\par 暴风雨的云层似乎非常厚实,虽然整个草原辽阔无际,但是头上这一片正要袭来的雨云却更教人害怕。现在我们只能看它的脸色行驶。它什么时候下来,我们无法掌握,惟一能做的只是看着它愈来愈近。
\par 刚才我们曾经看到了一个小镇,镇子里一些小的建筑和一座水塔已经看不清了。暴风雨随时会来。现在四周再也看不到任何城镇,所以我们必须骑快些。
\par 我骑到John旁边,作了个加速的手势,他点点头。我让他骑在前面,然后紧紧地跟着他,车速由七十到八十到八十五。现在我们已经感受到大雨来前的强风了,我把头低下来,迎着风向前去。
\par 车速已经到九十了,车速表上的指针不断来回地摆动着,但是转速表仍然维持在九千。时速大约是九十五英里,我们就以这样的速度往前冲去。现在因为骑得太快,没有办法行在路肩上,我为了安全起见,就打开了车灯,反正天色也愈来愈暗了,必须这么做才行。
\par 这个时候我们飞驰过平坦的大地,四下看不见任何机动车,甚至连一棵树也没有。路面平直而且干净,发动机的转速也一直保持在非常高的状态,这就表示还没有出问题。天色愈来愈暗了。
\par 突然之间,天空劈过一道闪电,接着是一声巨雷,我不禁震动了一下。Chris把头抵着我的背。这时落下来几滴预警的雨,在这种速度之下,它们打在脸上好像针扎一样。
\par 第二次闪电和雷声又来了,照得整个大地一片光明。
\par 然后速度降到七十英里、六十英里,然后是五十五英里,之后就保持这个速度。
\par Chris大叫道:“我们为什么慢下来了呢?”
\par “太快了!”
\par “不快!”
\par 我点点头。
\par 这时候房子和水塔从我们身旁掠过,然后出现了一条小下水道,旁边有一个十字路口,路一直通往天边,没错,我想一点儿都没错。
\par Chris喊着:“他们已经远远赶在前面了。骑快点儿吧!”
\par 我摇摇头。
\par 他又叫着:“为什么呢?”
\par “危险!”
\par “他们都不见了!”
\par “他们会等的。”
\par “骑快点嘛!”
\par “不行。”我摇摇头,这只是一种感觉。这个时候你得信任车子,于是我就把速度保持在五十五英里。
\par 雨开始下了,但是我看见前面有小镇的灯光……我知道它就在那儿。
\par 当我们到达的时候,John和Sylvia在路旁的第一棵树下等我们。
\par “怎么回事?”
\par “开慢了一点。”
\par “我们知道,车子有什么毛病吗?
\par “没有,让我们避开这场雨。”
\par John说在城的那一边有一间汽车旅馆,但是我告诉他,如果向右转再过几个路口,在一排白杨树旁边有一间更好的。
\par 过了几个路口之后,我们来到白杨树边,这里的确有一间小型的汽车旅馆。
\par John在室内绕了一圈说:“这里的确很不错,你以前是什么时候来的?”
\par 我说:“我不记得了。”
\par “那你怎么知道的这里呢?”
\par “凭直觉。”
\par 他看了看Sylvia,摇摇头。
\par Sylvia已经默默注意了我好一段时间,她看到我签名的时候手有一些颤抖。
\par 她说:“你的脸色好苍白,是不是闪电吓着你了?”
\par “没有。”
\par “你好像看到了鬼一样。”
\par John和Chris都看着我,我转过身向着门。外面仍然下着雨,我们跑进房间。车子盖好了,我们要等暴风雨过去再去骑它。大雨初停,天空稍稍放亮,但是从汽车旅馆的院子里,我看到在白杨树后,夜晚正逐渐来临。然后我们走到城里,用过晚餐。就在回旅馆的路上,一整天下来的劳累突然侵袭而来,于是我们停下来休息,浑身酥软无力。坐在汽车旅馆院子里的铁椅上,John从冰箱里拿出来混着其他饮料的Whiskey,我们慢慢地啜饮,心旷神怡,白杨树排在道路两旁,晚风轻轻袭来,叶子沙沙作响。
\par Chris在想接下来我们应该做什么。他一点都不累,汽车旅馆的新鲜感让他十分兴奋,他希望我们就像他们在夏令营的时候一样来唱歌。
\par John说:“我们唱不好。”
\par Chris说:“那么我们来讲故事。”
\par 他想了一下,“你知道有什么好的鬼故事吗?我们小组的孩子,晚上都很会讲鬼故事。”
\par John说:“那你先给我们讲一些鬼故事好了。”
\par 于是Chris开始讲鬼故事,听起来十分有趣。其中有一些我在他这个年纪都还没有听过。他希望听我讲一些鬼故事,但是我一个都不记得了,过了一会儿,他说:“你相信鬼吗?”
\par 我说:“不相信。”
\par “为什么?”
\par “因为他们没有科学依据。”
\par 我的答案不禁让John笑了起来,我接着说:“他们的存在不占用任何空间,也没有能量,因此根据科学定理,他们只存在于人的心中。”
\par 这个时候,酒精、倦意和微风纠缠于我心中,一起影响着我,我又说道:“当然,科学定理也不占用任何空间,也没有能量,因此也只存在于人的心中,所以完全科学的态度就是既不相信鬼,也不相信科学,这样你就安全了。然而这样一来,你就没有多少可以相信的了,但是惟有这样才是科学的态度。”
\par Chris说:“我不知道你在说些什么。”
\par “我开了点儿玩笑。”
\par 我说话的时候Chris有些消沉,但是我不认为这会伤害他。
\par “在青年会的夏令营里面,有一个小孩子说他相信鬼。”
\par “他只是骗着你好玩罢了。”
\par “不是的,”他说,“如果埋葬一个人的方法不对,他的灵魂就会来骚扰活着的人,他真的这样相信。”
\par 我又说:“他只是骗你罢了。”
\par Sylvia问他:“他叫什么名字?”
\par “Tom White Bear(汤姆 白熊)。”
\par John和我交换了一下眼色,突然之间,我们都了解了是怎么一回事。
\par 他说:“是印第安人吗?”
\par 我笑着说:“我想我得再补充一句,我所说的是欧洲的鬼。”
\par “有什么不同呢?”
\par John大笑起来:“他盯上你了。”
\par 我想了一下说:“印第安人对事情的看法通常和别人不同,我并不是说他们全错,但是他们并不认为科学是印第安传统的一部分。”
\par “Tom White Bear说他父母叫他不要相信这些玩意儿。但是他祖母偷偷地告诉他这是真的,所以他就信了。”
\par 他这个时候面带恳求地看着我,有的时候他的确想要知道一些事情,所以如果我继续开玩笑下去,并不是个好父亲该有的态度,于是我调整了一下自己的心态,“当然,我也相信有鬼的存在。”
\par 这个时候,John和Sylvia用奇怪的眼神看着我。我明白这一次要脱身并不容易,势必要作一番解释。
\par “认为欧洲人或是印第安人相信鬼的存在是一种无知,这是非常自然的,从科学的角度来看,这样的人仍然处在非常原始的状态之中。所以今天有人表示,相信鬼神的存在就会被别人认为是无知,甚至是头脑有问题,因为很难想像有鬼存在的世界究竟是怎样的。”
\par John同意地点点头,然后我又继续说。
\par “我个人的看法是,其实现代人未必比以前的人聪明,人的智商并没有多大改变,那些印第安人和中古世纪的人跟我们都差不多,但是彼此所处的环境不同;在以前的环境中,他们认为鬼神是存在的,就像现代人认为原子、质子、光子和量子是存在的。从这个角度来说,我相信有鬼,也就是说,现代人也有属于他们的鬼神,你知道的。”
\par “这是什么意思?”
\par “比如说,物理定理、逻辑学……数的系统……几何代数等等,这些都是所谓的鬼魂,因为我们太相信了,所以它们看起来就是真的。”
\par John说:“我认为它们是真的。”
\par Chris说:“我不明白啊!”
\par 于是我又继续说:“比如说,有人假设地心引力在Newton发现之前就已经存在,这是一件非常自然的事,但是如果认为地心引力直到17 世纪才存在,那就很愚蠢了。”
\par “当然。”
\par “所以这种定理是在何时开始存在的呢?它一直都存在的吗?”
\par John皱了皱眉头,不知道我要说什么。
\par 我说:“我的意思是,在有地球之前,在日月星辰形成之前,在一切之初,地心引力就已经存在了。”
\par “当然。”
\par “地心引力也没有自己的质量,没有自己的能量,当时人尚未出现,所以也不存在于人的心灵之中。它也不在空间里,因为也没有空间存在,更不存在于任何地方——这个地心引力仍然存在吗?”
\par 现在John可就不那么肯定了。
\par 我说:“如果地心引力存在,那么说实在的,我就不知道什么是非存在了。
\par 我认为地心引力已经通过所有非存在的考验,你想不出地心引力有什么不符合非存在的条件,或是科学上有证明其存在的证据。然而一般人仍然认为它是存在的。”
\par John说:“我得好好地想一想。”
\par “我推测如果你继续想下去,你只会一直原地打转,一直原地打转,直到你想出惟一合理有意义的结论,那就是,在Newton诞生之前,地心引力并不存在。
\par 不会有其他合理的结论。
\par “我的意思是,”我在他打断之前接着说,“就是地心引力定理只存在于人的心里,这也是一种鬼魂!对于别人所相信的鬼魂,我们很容易无知而且自负地就进行攻击,但是对于我们自己心中的鬼魂,我们却非常无知而且盲目地信仰着。”
\par “那么为什么所有的人都相信地心引力的确是存在的呢?”
\par “大家被催眠了,用比较正统的说法是,大家受了教育。”
\par “你的意思是老师把学生催眠了,让他们相信地心引力的存在?”
\par “正是如此。”
\par “听起来很荒谬。”
\par “在教室里,你听说过视线接触的重要性吗?每一位教育家都强调这一点,但是没有人会向你解释。”
\par John摇摇头,然后又为我倒了一杯,他用手遮着嘴,小声地跟Sylvia说:“你看他大部分的时间看起来都是这么正常。”
\par 我回答他:“这是我几个礼拜以来所说的第一件正常的事,其他的时候,我的脑海中和你一样充满了20世纪的狂想,所以没有注意到我自己的想法。”
\par “但是我会再替你重复一遍。”我说,“我们相信,Newton的理论早在他出生之前的几十亿年,就已经存在于宇宙的混沌之中,而他奇迹般地发现了这个理论。它一直存在着,虽然没有应用于实践。后来这个理论逐渐成形了,而且为人所运用。事实上这些理论就形成了世界。John,这种说法太荒谬了。”
\par “而科学家所面临的矛盾是心。心既非物,也没有能量,但是他们并不能否认心存在于他们所做的一切之中。逻辑存在于心中,数字也只存在于心中。如果科学家认为鬼也只存在于人的心里,我不会反对这种说法。其中‘只’是一个关键词,科学只存在于你的心里,这种说法并没有错,鬼也是一样。”
\par 他们还是看着我,所以我继续说:“自然的法则是人类发明的,就像鬼的存在一样。逻辑学、数学也都是如此,所有值得赞美的事,也都是人类的发明。
\par 这个世界也是人类所想像出来的,整体来说也就是一种灵界的存在。在古代,我们所居住的这个美妙的世界就被如此视之,它由鬼神所统领,我们之所以能看到这个世界,就是因为鬼神让我们看见,他们是Moses、the Christ、the Buddha、Plato、Rousseau、Descartes、Jefferson、Lincoln等等,Newton是非常好的一位,可算其中最好的一位,所以我们的常识就是由过去成百上千的鬼神所构成的,他们企图在人的生命当中找到他们的地位。”
\par 这个时候John正沉思不语,但是Sylvia非常兴奋地说:“你这些念头是从哪儿来的?”
\par 我想要回答他们,但是又说不出口,我觉得已经说完了,甚至说过了头,是该结束的时候了。
\par 过了一会儿,John说:“再去看看山倒是不错。”
\par 我很同意:“没错,的确如此,让我们喝完这一杯吧!”
\par 我们结束聊天,各自回房。
\par 我看见Chris在刷牙,答应让他明天早上再洗澡。我以大人的身份决定睡在窗边的床上。熄灯之后,他说:“现在可以给我讲一个鬼故事了吧!”
\par “我刚刚不是讲过了吗?”
\par “我是说一个真正的鬼故事。”
\par “那是你听过的最真实的鬼故事。”
\par “你知道我的意思,是另外一种。”
\par 我努力回想传统的鬼故事,“Chris,小的时候我听过许多鬼故事,但是我都忘了。”我说,“现在是睡觉的时候了,明天还要早起呢。”
\par 这个时候,大地一片寂静,只有风吹动窗子的声音。草原上的风习习吹来,一想起这个,就不禁让我陶陶然。
\par 风一时起,一时落,不断地吹送过来……它们来自那么遥远的地方。
\par Chris问我:“你见过鬼吗?”
\par 我已经快睡着了,我说:“Chris,我曾经认识一个人,他花了一生的时间,什么事也不做,只是去追寻一个鬼魂,结果只是徒劳,赶快去睡吧。”
\par 我发现自己说错的时候为时已晚。
\par “他找到了吗?”
\par “他找到了,Chris。”
\par 我一直希望Chris能够听听风的声音,不要再问问题了。
\par “那么后来呢?”
\par “他把他给痛打了一顿。”
\par “然后呢?”
\par “然后他自己也变成了鬼。”我以为这样说会让Chris早点睡,但是却使我精神愈来愈好。
\par “他叫什么名字呢?”
\par “你都不认识。”
\par “究竟叫什么呢?”
\par “那不重要。”
\par “究竟叫什么呢?”
\par “Chris,他的名字,这不重要嘛,他叫Phaedrus,你没听过的名字。”
\par “我们骑摩托车遇上暴风雨的时候,你有没有看见他?”
\par “你为什么会这么问呢?”
\par “Sylvia说她以为你看到了鬼。”
\par “那只是一种形容罢了。”
\par “爸爸?”
\par “Chris,这是最后一个问题,要不然我就要生气了。”
\par “我只是想说,你所说的和别人说的很不一样。”
\par “Chris,我知道,”我说,“这是个问题。现在睡吧!”
\par “爸爸,晚安。”
\par “晚安。”
\par 半个钟头之后,他已经睡熟了,窗外的风依然十分强劲,而我却再也睡不着了。就在窗外的夜色中,冷风穿过道路吹进树林里,月光也在微微震颤的叶子上闪烁着——毫无疑问地,Phaedrus看到了这一切,我不知道他为什么在这里,我也永远不可能知道他为什么要以这种方式回来,但是是他引领着我们走到了这条奇怪的路上,他一直都与我们在一起,这是我们逃不了的。
\par 我希望我不知道他为什么会在这里,但是恐怕我必须承认我知道。刚才我提到的有关科学、鬼神以及下午的时候,说到的有关关心和科技方面的事,都不是我自己的思想——我已经有许多年没有新的思想了,这些都是从他那儿窃取来的,而他一直在一边观看着这一切,这就是他为什么在这里的原因。
\par 经过这样一番招认之后,我希望他让我好好地睡一觉。
\par 可怜的Chris,他会问我:“你知道什么鬼故事吗?”我应该给他讲一个鬼故事,但是一想起这一点,我就不禁悚然而栗。
\par 我真的该睡觉了。
\subsection*{4}
\par 每一位参加Chautauqua的人都应该保有一张清单,上面列着重要的事项,以备将来之需或是记下灵感。记载要详细。
\par 现在趁别人还在酣睡着浪费美丽的晨曦,我正好把它们列下来。
\par 现在我有一张清单,上面写着下次骑摩托车去Dakota旅游时所该准备的东西。
\par 天一亮我就醒了,Chris仍然在熟睡。我原本也想再多睡一会儿,但是听到外面的鸡啼,想到我们现在正在度假,实在没有多睡的必要。我听到隔壁John正在锯木头的声音……或者是Sylvia……
\par 不会,声音太大了,该死的链锯,听起来像是……
\par 我已经很厌倦旅行的时候忘记带东西,所以我列出这样的单子,放在家里的文件夹中,一旦要出发的时候好派上用场。
\par 大部分的东西都很普通,不需要额外的说明,有一些就比较特别,需要进一步解释,还有一些非常奇怪,需要更多的说明。这张单子分成四部分,包括衣服、个人用品、炊具和露营用具,以及有关摩托车的用品。
\par 第一部分有关衣服的部分很简单:1.两套内衣。
\par 2.一件长的内衣。
\par 3.我们两人各一套衬衫和裤子。
\par 我喜欢用军人穿的工作服,因为便宜、耐穿而且耐脏。我有一项列做“正式服装”,John在后面用铅笔补上一句“半正式宴会装”,其实我所想的只是在加油站之外可能想穿的衣服。
\par 4.毛衣和夹克各一件。
\par 5.手套,没有里子的皮手套最好,因为不但能够防止晒伤,而且能够吸汗,使你的手部保持干爽。如果你只是骑车出去一两个钟头,皮手套就不重要,但是如果你夜以继日地骑车就非常重要了。
\par 6.骑车专用的靴子。
\par 7.雨具。
\par 8.旧头盔和遮阳帽。
\par 9.流线型的防风罩,这个帽子让我觉得很闷,所以只有在下雨的时候才用,在高速度之下你会觉得雨打在脸上像针刺一样。
\par 10.护目镜,我不怎么喜欢使用挡风玻璃,因为会阻碍我的视线。由英国制造的玻璃镜片很薄,这样的护目镜就相当不错,能够挡掉不少风,如果是塑胶制品,就会被风刮破,视线也会扭曲。
\par 下面是个人用品:梳子、皮夹子、小刀子、小型随身记事本、笔、烟和火柴、手电筒、肥皂和塑胶肥皂盒、牙刷、牙膏、剪刀、头痛药、驱虫药和除臭剂(骑了一整天车子,就算你的朋友不告诉你,你也会知道那有多么臭)、防晒油(骑车的时候你不会注意到晒伤的问题,一停下来的时候就已经太晚了,所以尽早涂上它)、创可贴、卫生纸、毛巾(要放在塑料盒里面才不会把其他的东西弄湿)、面巾。
\par 书,我不知道其他的人是否会带书,因为十分占空间,不过我会随身带三本,然后夹一些空白纸做记录,三本书是:1.有关这部摩托车修理店的资料。
\par 2.一本普通的摩托车问题指南,包括所有我记不住的资料,书名是——《Chilton摩托车问题指南》,由Ocee Rich所著,Sears Roebuck公司出版。
\par 3.Thoreau的《Walden(瓦尔登湖)》……Chris从来没有听过这本书,但我可以读上一百次也不觉得累。通常我会选一本他不懂的书,作为我们以后对答之用,我先读一两个句子,然后等他一连串地发问,然后再回答他的问题,之后再读一两个句子,用这样的方法读古典作品很有用,它们一定是用这种方式写成的。
\par 有的时候整个晚上我们都不断在阅读、讨论,而往往只读了两、三页,这是一个世纪以前的阅读方式……当时Chautauqua非常流行,除非你也这样做,否则你就不知道究竟有多么愉快。
\par 我看Chris睡熟了,一点也没有平常不安的样子,我想还不应该把他叫醒。
\par 露营用具包括:1.两个睡袋。
\par 2.两个斗篷和一张铺在地上的布。
\par 这些可以组成一个帐篷,同时旅行的时候也可以挡雨防湿。
\par 3.绳子。
\par 4.美国哥代蒂克地图(U.S.Geodeytic Survey maps),上面有我们希望旅游的地点。
\par 5.弯刀一把。
\par 6.旧指南针一个。
\par 7.行军水壶一个。我们出发的时候没有找到,我想大概是孩子把它丢到哪里去了。
\par 8.两个军用杂物箱,其中放着刀叉和汤匙。
\par 9.可折叠的Sterno牌的炉子,加上一个中型的Sterno罐子。我先买来试用,目前还没用过,一旦下雨或是还没有到达森林地带,找木柴就会是个问题。
\par 10.一些铝制易开罐的罐子,可以装猪油、盐、牛油、面粉、糖等。这是我们好几年前,在一家登山店买的。
\par 11.清洁剂。
\par 12.两个铝架的背包。
\par 摩托车用品。一般都有一个工具箱放在座位下,里面装有:一把可调整的大型扳手、专用榔头、錾子、一把锥状打击器、一对轮胎骨、补胎的用具、打气筒、一罐润滑铁链用的二硫化钼喷剂(这种喷剂可以深入每一个链环的内部,有非常良好的润滑效果,一旦二硫化钼干掉之后,就应该补上SAE—30 的机油)、撞击螺丝刀、锉刀、探测用仪表、测试灯。
\par 零件包括:火花塞、节流阀、离合器、煞车绳、指针、保险丝、头灯和尾灯的灯泡、接合传动链用的环扣与扣钩、固定销、打包绳、备用传动链(这个链子是我换下来的用旧的,如果目前所用的坏掉,换上它以后还足够支撑到修理店)。
\par 就是这么多了,不包括鞋带。
\par 你很可能会怀疑我们的车子究竟有多大,是否要像拖车一样大才能装下这些东西,但是其实并不像想像中的那么多。
\par 我怕如果让这些家伙继续睡下去,他们可能整天都起不来了。外面的天空一片湛蓝,我们这样浪费时间实在是丢脸。
\par 于是我还是去把Chris摇醒,他睡眼惺忪地睁开了眼睛,然后不解地坐起来。
\par 我说:“该梳洗了。”
\par 我走到外面,空气十分新鲜,事实上——天啊!——简直有些冷飕飕的,我敲了敲John夫妇的门。
\par 门里面传来John懒洋洋的声音:“来了,来了。”
\par 今天的天气好似秋天,车上沾了不少露珠,没有下雨,但是很冷,大约只有摄氏十度左右。
\par 在等他们的时候,我检查了一下齿轮箱的机油量、胎压、螺丝和传动链条的松紧。那儿有一点松了,我拿出工具箱来,然后把它旋紧,我已经迫不及待地准备要上路了。
\par 我看Chris穿得颇为暖和,于是打好包就上路了。不过天气的确很冷,不出几分钟,衣服内的暖气就被风给吹光了,我不禁打了几个大寒颤。
\par 只要太阳出来久一点,就会比较暖和。大约半小时之后,我们就可以在Ellendale吃早点,今天的路都很直,所以可以走很远。
\par 要不是冷得要命,今天会有一段非常棒的旅程,朝阳照着田野上的露珠,晶莹闪亮,空气中有一些迷蒙的晨雾。
\par 一路上只有我们行来,别人似乎都还没有起,现在是六点半,我手上戴的这副旧手套上似乎出现了一层霜,但是我想可能是昨天晚上水浸过的痕迹。这真是一副好手套,但是天气实在是太冷,连皮手套也变硬了,硬得我几乎没有办法把手伸直。
\par 昨天我曾经谈过关心,我关心这副皮手套,我微笑着看它们被风吹拂,因为它们已经在那儿陪伴了我这么多年。
\par 它们已经磨损老旧了,但我却在它们身上发现了一种幽默感。整副手套都沾满了油渍、汗水、灰尘,而且还有地方发霉了。现在把它们放在桌上,即使天气不冷,它们也没有办法平平地躺着。它们似乎有属于自己的往事。虽然只值三块美金,而且已经补到无法再补,但是我仍然花了许多时间和精力去清理它们。我不能想像换戴一副新手套的感觉。
\par 这种想法似乎很不实际,但是手套并不仅仅需要实际,其他事情也是如此。
\par 我对这部摩托车也有同样的感情,我已经骑了两万七千英里,可算是一部旧车,尽管街上还有很多更老的摩托车在跑。我相信大部分的骑士都会同意,一旦一辆车陪伴过你许多时光,那么对你来说它就是独一无二的,是别的车子无法取代的。有一位朋友和我骑同一个牌子、型号甚至同一年生产的车子,有一次他骑来让我修理,当我骑上它的时候,我很难相信这部车子竟然和我的是同一个牌子。你会发现车子已经拥有了属于它自己的声音和节奏,与我的完全不同——不是不如我的,而是不同。
\par 我想你可以称之为个性,每一部摩托车都有它自己的个性,也可称之为你对这一部车子所有直觉的总和。这种个性常会改变,多会变得更糟,但常常也会变得出人意料地好,培养这种车子的个性正是维修保养的真正目的。
\par 新的车子就好像美丽的陌生人,按照它们所受的待遇,要不就很快会退化成别扭的人或是跛子,要不就变成健康、好脾气、长久的朋友,而这部车虽然遭受了那些所谓师傅的毒手,但是似乎已经完全修复了,而且愈来愈不需要修理。
\par Ellendale到了!
\par 在晨曦中我们看见一座水塔,还有几丛树林和其中的建筑物,我已经习惯了一路上的冷风。这时候是七点十五分。
\par 几分钟之后,我们把车子停在一座老旧的砖房前,John和Sylvia停在我们后面,我转身向他们说:“天气好冷!”
\par 他们只是呆呆地瞪着我。
\par “冻僵了?”我说,他们没有回应。
\par 我一直等到他们停好车,然后看见John准备卸下所有的行李,他有一个结打不开,最后干脆放弃了,我们走向餐厅。
\par 我又试了一次,走到他们面前,这时我觉得自己骑车骑得有一点儿神志不清。我绞着手笑着说道:“Sylvia!说话啊!”但是她脸上毫无笑意。
\par 我想他们真是冻僵了。
\par 他们眼也不抬地叫了早点。
\par 吃完早点后,我才又开口:“接下来该怎么办?”
\par John故意慢慢地说:“我们不打算离开这里,除非天气好转。”他的口吻好像是小镇上的警长,我想这就是最终决定了。
\par 于是John、Sylvia和Chris就在餐厅旁边饭店的大厅里坐着取暖,而我出去散散步。
\par 我想他们有点儿生我的气,为什么要一大早就把他们挖起来在冷风里赶路?如果彼此相处太久,个性上的不同是注定要显露出来的。我想起来了,我以前从来没有在下午一两点钟以前和他们一起骑车上路,虽然我认为一大早是一天中最适合骑车的时间。
\par 小镇非常干净而且空气清新,不像我们昨晚留宿的那个小镇。街上有一些人正一面打开店门说“早上好”,一面谈论天气有多么冷。在街背阴的那面有两个温度计,分别指向摄氏5.5和7.7度,而被太阳照到的另一个则是18.5度。
\par 经过了几个街区之后,大街分成两条泥泞的路通往田野。我经过一栋组合式的活动屋,里面装了一些农机和一些修理工具,最后来到尽头的田野,有一个人站在那儿用怀疑的眼光看着我,不知道我要做什么,或许是发现我正在观察活动屋里面的情形。我回到街上找了一张冰冷的椅子坐下来,呆呆地望着摩托车,什么事也不做。
\par 虽然天气很冷,但还不至于那么冷,John和Sylvia是怎么度过Minnesota寒冷的冬天的呢?我怀疑。从这里我们就可以发现明显的矛盾,几乎根本不需要思考就可以明白,如果他们不能忍受生理上的不适,而同时又无法接受科技的成果,他们就一定得做些让步。他们一面需要科技,一面又要诅咒它。我相信他们很明白这点,而这正是他们厌恶科技的原因。他们并未提出一个逻辑论题,只是做出直接的反应而已。现在有三位农夫进城了,开了一辆全新的卡车,我敢打赌他们进城另有目的。他们是来炫耀一下这辆车子,还有拖车和那台新的洗衣机。如果机器出了问题,他们有工具去修理,而且他们知道如何使用工具。
\par 他们珍惜科技,然而他们却是最不需要科技的一群人。如果明天所有的科技都消失了,这些人仍然可以活得好好的,日子可能不好过,但是他们可以活下来,而John、Sylvia、Chris和我可能在一个礼拜之内就死了。这样子诅咒科技是不敬的,但是情形就是如此。
\par 又钻进死胡同了。如果有人不懂心存感激,而你当面告诉他,那么你就等于是在骂他,这样你什么事都解决不了。
\par 半个小时之后,旅店门口的温度计显示现在的温度是11.5度,我在空旷的餐厅中找到他们。他们看起来一副睡眠不足的样子,不过比刚才要好。John愉快地说:“我准备把所有的衣服都穿上,我相信这样就不会有问题了。”
\par 他出去走到车子旁边,回来时说:“我真讨厌打开这些包裹,但是我又不希望像刚才那样继续骑下去。”他还说男厕所里面冷死了。餐厅里面一个人也没有,他便从我们坐的位子后面走过去,这时我正在和Sylvia聊天。然后我抬起头,看见John穿了一套淡蓝色长袖长腿的内衣。他不断嘲笑自己这副傻模样,我盯着他放在桌上的眼镜看了一会儿,然后就对Sylvia说:“你知道,我们刚刚才坐在这里和Clark Kent说过话……你看他的眼镜还在这儿……现在,突然之间……Louis,你不认为……?”
\par John大吼一声:“无敌超人来了!”
\par 他像穿了溜冰鞋一样滑过大厅的地板,翻了一个筋斗,然后又滑回来,他举起一只手,然后又缩回来,仿佛准备飞向空中,“预备起!”然后他摇摇头,“老天!我讨厌冲破那么好的天花板,但是我的X 光眼告诉我有人有麻烦了。”Chris在一旁咯咯地笑着。
\par Sylvia说:“如果你再不多穿一点衣服,我们都会有麻烦的。”
\par John笑着说:“我是暴露狂吗?我是Ellendale的救星。”他又得意洋洋地走了一阵子,然后穿上外衣,他说:“哦!他们不会这样做的,无敌超人和警察有着相当的默契,他们知道谁站在法律、真理和秩序的这一边。”
\par 我们上高速公路的时候,仍然感觉非常寒冷,但是已经好多了。我们又经过了几个城镇,几乎在不知不觉中,太阳温暖起来了,而我的情绪也跟着好了起来。这时疲倦的感觉已经完全消失,风和太阳让你觉得很舒服,让这一切显得很真实。温暖的太阳融和了马路、绿草原上的农庄,还有迎面而来的风。很快地就只剩下温暖的风,速度和太阳,最后的一丝寒意已经被太阳驱走了。只剩下迎面而来的风,暖洋洋的太阳和平坦的大道。
\par 这个夏天是如此的满眼绿意,空气是如此的清新。
\par 在一排老篱笆前的青草中,一些白色和黄色的雏菊摇曳着,草地上漫步着几只牛,远处有一片高地,上面有一些金黄色的东西:几乎看不清楚究竟是什么,反正我们也不需要知道。
\par 这时候有一点上坡,发动机的声音逐渐沉重起来。我们爬过了这个小坡,一片新的土地展现在眼前,路在逐渐下降,发动机的声音也轻快了许多,这里有一大片草原,沉静地躺在天地之间。
\par 后来我们停下来的时候,Sylvia的眼睛被风吹得流泪了,她伸开双臂说道:“天啊!真美!这么空旷的一片大地。”
\par 我教Chris如何把夹克铺在地上,然后将衬衫折起来当枕头,虽然他并不想睡,但是我告诉他先躺下来,他需要先休息一会儿。我把我的夹克铺开,吸收更多的热气,John拿出他的照相机。
\par 过了一会儿,他说:“这是天底下最难拍的了。你需要一个三百六十度的广角镜头,你看着这样一片风景,然后看看地上的草,一切都妙不可言。但是一旦你用框子框住,美感就都不见了。”
\par 我说:“我想这就是你在汽车里面所见不到的吧!”
\par Sylvia说:“大约在我十岁的时候,有一次也是像这样在路旁停了下来,我差不多照了半卷的相片,后来洗出来的时候,我哭了,里面什么都没有。”
\par Chris说:“我们什么时候再继续走?”
\par “你急什么?”我问。
\par “我就是想再继续走下去。”
\par “前面没有比这里更好的风景了。”
\par 他皱着眉沉默不语,“我们今天晚上要露营吗?”他问,他们夫妇俩担心地看着我。
\par 他又问:“要露营吗?”
\par 我说:“再看看吧!”
\par “为什么还要再看看呢?”
\par “因为我现在还不知道。”
\par “为什么你现在还不知道呢?”
\par “我就是不知道为什么我不知道。”
\par John耸耸肩表示没有关系。
\par 我说:“这里不是最适合露营的地方,这里既没有蔽荫也没有水源。”但是突然间我又添了一句:“好吧,我们今晚就找个地方露营。”我们以前曾讨论过这件事。
\par 我们又沿着这空旷的路继续骑下去,我不想拥有这些草原,或是把它们拍下来,我也不想改变它们,甚至也不想停下来,或是继续走下去。我们只是沿着空旷的路继续骑下去。
\subsection*{5}
\par 平坦的草原逐渐变成起伏的大波浪,篱笆愈来愈少,而满眼的绿意也变得苍白起来,一切改变都意味着我们已经接近高原地区。
\par 我们在Hague停下来加油,顺便问一问有没有路可通过Bismarck和Mobridge之间的Missouri河。服务生并不清楚,而今天又十分炎热,John和Sylvia到一边把长袖内衣脱下来。摩托车需要换油,链条也要润滑一下。我做的时候,Chris在旁边看着,他有一点不耐烦,这不是个好现象。
\par 他说:“我的眼睛疼。”
\par “为什么啊?”
\par “风吹的。”
\par “我们去买护目镜。”
\par 我们走进一间店铺买咖啡和面包,这里陈列的东西花样繁复,所以我们都不说话,只忙着观看。偶尔听到有些人在谈话,他们似乎都彼此认识,偶尔也会看看我们这些陌生人。之后,我们到街上买了一个温度计放在袋子里,又买了一副护目镜给Chris。
\par 店主也不知道如何渡Missouri河。John和我一起研究地图,我本希望能够找到私人的渡船,或者是人行桥,或者其他什么都好,但是很明显,那儿什么也没有。这主要是因为对岸没什么去的价值,那儿整片都是印第安人的保留地。
\par 于是我们决定往南走到Mobridge,然后从那儿渡河。
\par 往南走的路糟透了。崎岖狭窄,颠簸难行。我们一路顶风而行,向着太阳而去。大拖车通常都另择他路了。骑在这些过山车道一样的山路上,下坡的时候会突然地向下疾冲,然后又得慢慢地往上爬。这样一来我们就没有办法看得很远,因而变得有些紧张。骑到第一个坡道的时候,我有一点恐慌,因为我还没有准备好。现在我紧紧地握住摩托车的把手迎上前去,一点危险也没有,只是让你大吃一惊。这时候天气越来越热,越来越干燥。
\par 到Herreid之后,John独自走开去喝一杯,而Sylvia、Chris和我走到公园里找阴凉的地方想要休息一下。然而我觉得有些不安,因为冥冥之中似乎有一些变化。这个小镇的路非常宽,宽得不切实际。空中飘浮着灰尘,房屋之间有许多空旷的土地,野草丛生,一片荒芜。
\par 铁皮做的遮阳板和水塔跟前面城镇里的一样,但是分布的范围要大多了。这里的一切都像是已被人抛弃,外观十分机械化,同时杂乱无章地四散着。我逐渐明白是什么事不对劲了。如今已经没有人再关心保留地,这块土地没有多大的价值。我们现在已经置身在西部的小镇中了。
\par 我们在Mobridge的餐厅里吃了汉堡,喝了点麦乳精。然后慢慢地穿过了一条繁忙的街道,来到了山脚下。那儿就是Missouri河。这里的河水很奇怪,两岸长满了茂盛的野草,根本无法取到水,我转身看看Chris,但他似乎对这并不感什么兴趣。
\par 我们沿着河岸骑下山,找到了桥。
\par 我们在桥上看着河水很有节奏地流淌着,然后过了河。
\par 我们爬过了一条长长的山路,来到另外一个乡镇。
\par 这里完全没有篱笆。没有矮树丛,更没有树木。山势绵延,壮阔无比。远远看去,John的摩托车就像一只小蚂蚁,在草地上慢慢地爬行,在山坡上方,有一些岩石在断崖顶上探出头来。
\par 这里的一切都天生整齐有序。如果是已经荒废的土地,应该有许多破败之处,再加上不少老旧的建筑,还有上过油漆的碎片、电线、野草……然而这里却完全没有这种景象。不能说它保持得好,只是从来没有杂乱过。它本来就应该是这样,这就是保留区。
\par 在岩石的另外一边没有摩托车修理店,我在想我们准备妥当没有,如果路上出了什么问题那可就麻烦了。
\par 我用手去试发动机的温度,没有问题。我发动了一下,想听听它空转的声音,但是我听到了一种很有趣的声响,于是又发动了一次。过了一阵儿我才明白,那根本不是发动机的声音,而是从山谷里传来的回声。真有意思,我又发动了两三回。Chris问我出了什么问题,我叫他听回声,他却什么都没有说。
\par 这部旧车子的发动机有些金属声,仿佛里面有许多松散的叶片在劈啪作响,听起来很难听。其实这是气门正常的声音,一旦你习惯了这种声音,并且学会期待它的出现,那么当发动机的声音有所不同时,你很自然地就能听出来。
\par 如果你什么都听不到,那就最好。
\par 我想让John对那个声音感兴趣,但是根本不成,他所听到的只是噪音;他所看到的只是摩托车和我手中拿着沾满油污的工具,此外别无他物,这样当然引不起他的兴趣。
\par 他不了解发生了什么事,而且也没有兴趣去研究。他对事情的表象比较感兴趣,对于内涵就不然了。这一点很重要,因为这就是他看事情的方法。我花了好长的时间才发现我们之间的这种不同,所以在这次旅程当中,很重要的一件事就是要明确这种不同。
\par 我被他的拒绝弄得有点不好意思,想尽所有办法,试图引起他对机械的兴趣,但是始终不知从何开始。
\par 我想或许我应该等到他的摩托车出毛病的时候再帮他去修理,他才可能会感兴趣。但是我把事情弄糟了,因为我没想到他看事情的方法和我不同。
\par 他的把手变松了,他认为问题并不严重,只是在用力扭转的时候才有一点儿松。我提醒他在上紧螺丝的时候不要用可调整的扳手,因为很可能会伤到表面,然后就会生锈,他答应用我的工具。
\par 他把车子骑过来的时候,我拿出扳手,但是发现再怎么旋紧都没有用。
\par 我说:“你应该用薄铁片垫一下。”
\par “什么薄铁片?”
\par “就是一片扁平条状的薄铁片,把它塞在把手的缝隙里,这样就会使把手更紧。通常在修理各种机器的时候都会用到它。”
\par “喔,”他有点感兴趣,“很好,那么要到哪儿去买呢?”
\par “我这儿有,”我很高兴地说,拿起了一个啤酒罐。
\par 他一时明白不过来,然后说:“什么?就是这个啤酒罐?”
\par “没错,”我说,“世界上最好用的垫片。”
\par 我自认为这一点很聪明,省得他到处去找买垫片的地方,也节省了他的时间和金钱。
\par 但是我很惊讶的是,他竟然没有发现它的妙用。事实上他对这件事的态度一直很傲慢,找各种理由来搪塞我,后来我才发现他真正的态度。最后我们决定不修车把了。据我所知把手仍然会松。不过我知道当时他的确很生气,我竟敢用啤酒罐的薄片去修理他花一千八百美金买来的全新的宝马车!这辆车代表的是半个世纪以来德国人在机械上的精良水准。
\par Ach, du lieber!
\par 此后我们就很少提到维修摩托车的问题,现在回想起来,应该是根本就没有再谈过了。
\par 我应该这样向他解释,这个啤酒罐是铝做的,不但材质很软,而且附着性很好,在这种情况中最适合使用,而且它不会受潮氧化,说得更仔细一点,它的表面有一层氧化物,可以防止进一步的氧化。
\par 换句话说,任何一位拥有精良的机械技术的、真正优秀的德国技师,都会认为这个解决办法最好不过了。
\par 后来我想了一下,我应该偷偷地走到工作台,切下一部分啤酒罐,把上面的印刷除掉,然后回来告诉他,我们很幸运,只剩下一片了,还是由德国进口的。这样就成了。它是由德国Baron Alfred Krupp公司制造的,我以特价买到了。这样他就搞不清楚究竟是怎么一回事儿了。
\par 这个念头让我高兴了一阵子,但是我渐渐发现,这样仍然是行不通的。我又再一次产生了以前曾经提过的那种感觉,这件事所牵涉到的问题比看到的要严重得多。一旦你仔细研究彼此之间的小分歧,就会有重要的发现。这只是我的感觉,我要像以往一样地继续思考其中的因果关系,了解究竟是什么造成了John和我之间这样大的差异。在从事机械方面的工作时,常常会有这种情况出现,一旦遇到瓶颈,你只好停下来,仔细思考一番,看看是否有新的信息,然后出去逛逛,等你再回来时,原先隐而未显的原因就会浮现出来。
\par 这个逐渐浮现出来的原因就是:我从理智、知识的角度去看修理把手的问题,其中牵涉到金属的所有科学上的特性。而John却从直觉和当下的角度去看待它。我是从内涵着手,而他却是从物的表象开始。我看到的是这个铝片的意义,而他看到的却是这个铝片的外观。
\par 所以,如果你只看到铝片的外表,当然会沮丧,谁会喜欢在一台新买的摩托车上安装废铝片呢?
\par 我想我忘了提John是一名演奏家,他和城里的很多乐队合作,专门负责打鼓,所以收入相当不错。我想他就是以打鼓的方式去看事情——也就是说他并没有真正地思考。他只是做了,他对用啤酒罐来修理摩托车这件事的反应,就跟有人在打鼓时忘了拍子的反应一样。
\par 对他而言,这是一种拖累,所以他不希望有这种情况发生。
\par 一开始,这种差异似乎并不起眼,但是它逐渐……逐渐……逐渐地扩大,一直到我开始注意到为什么我会忽略它的存在。有些东西你忽略是因为它们非常细微,但是有些却是因为它们过于庞大。
\par 我们两个人讨论相同的事,思考相同的事,然而他的出发点却和我的完全不同。
\par 他的确关心科技,但是他的观点已经被扭曲了,所以虽然他想要接近它,但是因为缺乏理性的思考,任他怎么反覆运用,这一切对他来说都只不过是一种诅咒。他想不通这个世界上竟然会有这样令人难以置信的事。
\par 这就是他所处的角度,一种常规的角度。我一直都是从一个十分理性的角度来谈论一切有关机械的事物,因为机械是零部件、是各种关系、是分析、是组合、是明了事物的原委,但它并不真的在此处。它总是在别处,我们都以为别处即此处,但是实际上它却远在千里之外,这就是机械的本质。
\par John的这种角度上的差异也是六十年代文化变异的根基。我认为它至今仍然在影响着我国人民对事情的看法,代沟就是由此而来。“披头士”和“嬉皮”的名称也来自于此。而现在事实证明,这种角度不只流行于一时,还会一直延续下去,这种角度之所以仍然存在,是因为它是非常严肃而且重要的。它看起来似乎无法与理性、秩序和责任并存,但事实并非如此。现在我们已经接触到事情的根本。
\par 我的腿变得很僵硬,甚至开始有些疼,于是我伸出一条腿,尽可能向左右做最大幅度的摆动,虽然略有帮助,但一会儿支持腿伸出来的肌肉就又开始酸了。
\par 在这里我们看到在事实认定上的冲突,不论科学家如何说它,你此时此刻所看到的世界,就是你所谓的事实。John就是如此去看的。但是从科学的角度来观察世界,这也是一种事实,不论它的表象如何。所以像John这样的人,如果要坚持己见,必然会采取一些行动,而不仅仅是不予理睬。如果John原先的看法出现了问题,他就会发现这一点。
\par 这就是为什么那天他会因发动不了摩托车而生气,因为这侵犯了他的事实。
\par 这似乎是在他看事情的方式上凿了一个洞,他无法面对,因为这样很可能会威胁到他整个的生活方式。从某个角度来说,他和那种学科学的人一样,某些时候会对抽象艺术产生愤怒,因为抽象艺术也不适合他们的生活方式。
\par 在这里,你有两种事实,一种是你立刻感受到的艺术表象,另外一种是隐藏其中的科学道理,因为它们彼此不相融,所以彼此之间没有多少的关联。事实就是如此。所以你可以说,这里有点问题。
\par 在一条废弃的路上,我们发现一间杂货铺,于是我们停下来,坐在杂货铺后面的一些包装箱上喝罐装啤酒。
\par 现在我觉得有些疲倦,背也开始疼了。我把箱子推到一根柱子旁边,然后靠在柱子上坐着。
\par Chris的表情沮丧,我一看就知道他现在心情不好。这一天的确把我们给累坏了。我在Minnesota州的时候就告诉过Sylvia,当我们走到第二天或第三天的时候,精神会突然变得很差,没想到现在就来了,Minnesota州——那是什么时候呢?
\par 一辆车停在路边,一个喝得烂醉的女人走下来,想替别人买啤酒,但她不知道买哪一种牌子的。店主的太太等得火冒三丈,但是她还是没办法决定。这时她看到了我们,就东倒西歪地走过来,问我们是不是摩托车的车主,我们点了点头。然后她就说,希望我们能够载她一程。我走开了,让John去处理这件事。
\par 他很圆滑地把她给打发了,但是她一次又一次地回来,请求我们载她一程,还给了John一块钱。我跟John开了几个玩笑,但是并不好笑,只是让气氛更加凝重。我们从杂货铺出来,又一次置身于枯黄的草坡上。这时阳光笼罩大地。
\par 我们到达Lemmon的时候已经累坏了。
\par 在小酒馆里,我们听说往南走走有一个可以露营的地方。John想在Lemmon公园里露营,这个建议很奇怪,让Chris十分气恼。
\par 我已经许久没有这样疲劳过了,其他人也是一样。但是我们仍然拖着疲惫的身躯到超市胡乱买了些东西,然后有些困难地放到车上。太阳只剩下了最后的余晖,天色在一个钟头之内就会完全暗下来。我们似乎无法再往前行,我想我们可能得在这儿停下来了。
\par 我说:“Chris,我们走。”
\par “不要对我吼,我已经准备好了。”
\par 我们从Lemmon骑上一条乡间小路,似乎骑了好久好久,人已经累瘫了,但是实际上并没有多远,因为太阳还没有下山。露营的地方已经很久没有人来过了,这倒不错。但是还有不到半个钟头,太阳就会完全下山,而且我们的精力已经耗尽了,这是最大的问题。
\par 我用最快的速度把东西卸下来,但是我太疲惫了,以至于犯了一个大错:我把所有的东西都卸在了路边,没有注意这个地点有多么糟糕。后来我发现这里风太大,就是那种高原的风。这里似乎已经被弃置许久,所有的东西都被烧过,而且十分干旱。只有一个湖在我们下面,它只能算是个大储水池。风从天边吹向湖面,吹过来,吹向我们。凌厉如刀,已经很冷了。离小路二十码远的地方,有一些矮小的松树,我要Chris把东西都搬到那儿去。
\par 他没有照我的话做,而是走到湖边去了。我只好独自搬行李。
\par 这时候我看到Sylvia拖着疲惫的身子,很专心地在准备煮饭的用具。
\par 太阳完全下山了。
\par John找来一些木柴,但是都太大,而且风吹得这样急,很难点火。我们需要把它们劈开才能点着。我走到松树丛边,在星光下摸索着我那把弯刀,可是树林里实在太暗了,我找不到。
\par 于是我走过去把摩托车骑过来,把头灯打开,这样就可以找到手电筒了。
\par 我一样一样地翻,想找到手电筒,过了很长时间我才突然意识到,我不需要手电筒,我需要的是弯刀,而弯刀就在我眼前。拿着弯刀回来的时候,John已经把火点好了,于是我就用刀劈了一些较大的木柴。
\par Chris又出现了,他手里拿着手电筒。
\par 他抱怨地说:“我们什么时候才可以吃饭?”
\par 我告诉他:“我们很快就会做好了。
\par 把手电筒放在这儿。”
\par 他又不见了,随身带着手电筒。
\par 风太大了,吹得火呼呼作响,左摇右摆,我们没有办法做好牛排。于是我们从路旁找来大石头,想堆在火旁边把风隔开,但是天色实在是太暗了,我们无法看清自己的动作,于是就把两辆摩托车都骑过来,打开头灯,照着火堆,这时候我们看到火堆里冒出许多火花,然后消失在风中。
\par 身后突然传来一声巨响,我听到Chris在一旁咯咯地笑着。
\par Sylvia很生气。
\par Chris说:“我找到一些鞭炮。”
\par 我及时控制怒气,然后很严肃地告诉他:“现在是吃饭的时候。”
\par “我要一些火柴。”
\par “坐下来吃。”
\par “先给我一些火柴。”
\par “坐下来吃。”
\par 他坐下来了,我想用军用刀切牛排,但是牛排实在是太坚韧了,于是我就找了一把猎刀来切。摩托车的灯直射向我,在阴影中完全看不见刀子的方向。
\par Chris说他也切不动他的牛排,于是我说把我的刀子给他。正要拿给他的时候,他却把盘子打翻了。
\par 没有人说话。
\par 我并不是气他把盘子给打翻了,我气的是一切都被弄得油腻腻的,要一直忍耐到回家的时候。
\par “还有吗?”他问。
\par 我说:“把那个吃掉。它只是掉到了桌上。”
\par “太脏了。”他说。
\par “就只有这些了。”
\par 这时候大家都有些闷闷不乐,我只想去睡觉,但是Chris生气了,我想最好能够当众理论一下,我等着,果然,很快就开始了。
\par 他说:“我不喜欢这个味道。”
\par “没错,Chris,味道不是很好。”
\par “我不喜欢吃这个。我也不喜欢在这里露营。”
\par Sylvia说:“这是你出的主意,是你想要露营的啊!”
\par 她不该这样讲的,但是她肯定没想到。你一旦上了他的钩,他就会再给你另外一个饵,然后又来一个,一直到最后你想打他,这才是他要的。
\par 他说:“我不管。”
\par Sylvia说:“你应该明白这一点。”
\par “我不。”
\par 火爆的场面就要出现了,Sylvia和John看了看我,但是我仍然面无表情。
\par 对这种情形我感到很抱歉,但是我现在无能为力,任何争执只会把事情弄得更糟。
\par Chris接着说:“我不饿了。”
\par 没有人回答他。
\par “我的胃很痛。”Chris的话锋一转,然后就走到林子里去了,即将出现的火爆场面因而平息下来。
\par 用完餐之后,我帮Sylvia清理了一下,然后又坐了一会儿,我们把车灯关掉以节省电力,也太刺眼了。风小一些了,火里仍然有一些微光。过了一会儿,我的眼睛对黑暗就习惯多了,刚才生的气和吃的东西赶走了一部分倦意。Chris还没有回来。
\par Sylvia问我:“你想他会不会是故意在和我们过不去?”
\par 我说:“我想,虽然可能我说的不完全对,但是我很讨厌这种儿童心理学的分析,就当他是一个讨厌的家伙吧。”
\par John笑了笑。
\par 我说:“反正晚餐吃得不错,我很抱歉,他竟然表现得这样。”
\par “对他不会有害的。”
\par “你想他会不会在里面迷路了呢?”
\par “不会,如果他迷路了会大声喊。”
\par 这个时候Chris还没有回来,我们也没有别的事情可以做。我开始观察四周的环境,周遭听不到一点声音,这真是一个孤寂的草原。
\par Sylvia说:“你认为他真的胃痛吗?”
\par 我很确定地说:“是的。”我很不愿意继续讨论这个问题,但是似乎我需要做进一步的解释,因为他们肯定感觉到事情比他们看到的要复杂。所以最后我说:“我想他一定是真的痛,他曾经检查过许多次,有一次甚至严重到我们以为是盲肠炎……那个时候我记得我们正向北旅行,当时我刚处理完一份价值五百万美元的机械合约。那真是够折磨人的,我在一个礼拜之内就要赶出一份六百页的资料,当时我真想杀人。所以我们想最好到森林里走一遭。
\par “我记不得去了哪里,当时脑海里塞满了工程方面的资料,而Chris在一旁大声哭嚎,后来我才发现必须尽快把他送到医院,究竟是哪一所医院我记不得了,但是他们什么也没有发现。”
\par “什么都没有发现吗?”
\par “是啊,后来又发生过一次同样的情形。”
\par “难道没有一个医生知道是怎么回事吗?”Sylvia问我。
\par “今年春天的时候,他们诊断后认为是精神疾病的征兆。”
\par “什么?”John说。
\par 现在天色已经完全暗下来了,我看不见John和Sylvia的身影,甚至连山的线条也看不清;我想听听远方的声音,但是什么也听不见;我不知道该怎么回答,所以就沉默下来了。
\par 我努力观察的时候,可以看到天上的星星,但是眼前的营火却使它们黯然失色,夜色愈来愈浓了,烟已快抽完,所以我干脆把它熄了。
\par Sylvia说:“我不知道有这么回事,”
\par 她所有的怒气都消了,“我们都觉得奇怪,你为什么不带你太太来,而要带他来。”她说,“还好你告诉了我们这一点。”
\par John拿了一些还没有烧完的木头丢到火里。
\par Sylvia说:“你认为原因是什么呢?”
\par John想要打断我们的谈话,但是我回答:“我也不知道,因果似乎无法解释他的状况。因果逻辑是思想上的产物,我认为精神疾病先于人的思想。”我想他们并不懂我所说的。对我来说,也是如此,现在我已经太累了,不想动脑筋,所以就任由它去吧。
\par John问我:“精神医生怎么说呢?”
\par “什么也没有说,我没有让他继续治疗。”
\par “没有继续治疗?”
\par “是的。”
\par “这样做好吗?”
\par “我不知道,我没有很充分的理由认定治疗不好,只是我自己有心理障碍。
\par 我曾经想过去治疗,也试着找出所有应该治疗的理由,然后计划去拜访那些医生,甚至把他们的电话都找出来了,然后我心里突然觉得有问题,就好像门砰地关起来了一样。”
\par “听起来不对劲。”
\par “除了我大家都不这么想,我想我不能永远忍下去。”
\par “但是为什么?”Sylvia问。
\par “我不知道为什么……那只是……我不知道……他们不像自己人。”我很惊讶,竟然用这个词,我以前从来没有用过,不像自己人……好像是穷人的说法……就是不亲切……他们对他没有真正的关心,因为不是自己人……就是这种感觉。
\par 这个说法如此古老,几乎已经逃逸出了现代人的脑海。几个世纪以来,变化是如此之大。现在每一个人都能够对别人很友好,或者说大家认为每一个人都很友好。可是放在很久以前,友好的人都是天生如此,而不得不表现出来。
\par 事实上现在大部分的时候,这只是一种虚伪的态度,就像第一天上课的老师一样。但是那些不是自己人的人,又怎么会知道友好究竟是怎么一回事儿呢?
\par 这件事不断地在我的脑海中出现。
\par 我有一种很奇怪的感觉。
\par Sylvia问:“你在想什么?”
\par “一首Goethe写的诗,大约是在两百年以前写的,我很久以前读过,不知道为什么现在突然想起来了,除非是… … ”这种很奇怪的感觉又回来了。
\par Sylvia问:“诗里说了些什么?”
\par 我努力地去回想:“有一个人晚上在海边骑马,有风迎面吹来。父亲紧紧地把儿子抱在怀中,问儿子为什么看起来这样苍白,儿子回答他:‘爸爸!难道你没有看到鬼吗?’爸爸尽量地安慰儿子,告诉他他所看到的只是岸边的一层薄雾,他所听到的只是树叶在风中飒飒作响,但是儿子仍然认为有鬼。父亲只好尽快地在黑夜中骑回去。”
\par “结局呢?”
\par “结果孩子死了,鬼赢了。”
\par 风把炭火吹起来了,我看到Sylvia有一点儿吃惊地看着我。
\par 我说:“但是这件事是发生在别的地方,而且是在很久以前。现在我们相信人死如灯灭,根本没有鬼。我相信这一点。”我望着一片黑暗的原野,“虽然我不知道究竟是怎么一回事儿……这些天来,我对许多事情都有些不确定,或许这就是为什么我话说得这么多。”
\par 炭火快要熄了,我们抽完了最后一枝烟,这时Chris仍然在黑暗之中的某一个地方,不过我不打算把他找回来。
\par John小心谨慎地保持着沉默,而Sylvia也是如此。突然之间,我们各自沉浸到了自己的世界之中,不再有任何交谈。
\par 我们在火上浇了些水,把它弄熄了,然后走到林子里面去找睡袋。
\par 我发现我在松林里放睡袋的那一小方土地不太好,那儿既是我的避难所,也是从储水池那边飞过来的成千上万的蚊子的避难所。驱蚊剂根本不管用,于是我爬进睡袋,只留一个小孔用来呼吸,当Chris回来的时候,我几乎已经睡着了。
\par 他说:“那儿有一个大土堆。”一边说一边用脚踩地上的松针。
\par 我说:“好,快去睡觉。”
\par “你应该去看看,明天你要去看吗?”
\par “我们不会有时间的。”
\par “明天早上我可以到那儿去玩吗?”
\par “可以。”
\par 他把衣服脱掉,弄出不少响声,然后才爬进睡袋里。爬进去之后,他滚了一下,没有说话,然后又滚了一下,说:“爸爸!”
\par “什么事?”
\par “你还是小孩子的时候是什么样的情形?”
\par “Chris赶快睡!”一个人听什么话都是有限度的。
\par 后来我听到一阵啜泣。我知道他在哭,虽然我已经筋疲力尽,但是却睡不着了。这个时候如果我说几句安慰话,可能会有用。他只是想要对我表示友好,但是这些话因为某些原因就是说不出来。对陌生人或是病人比较需要说些安慰的话,对自己人就不是了,像这样小小的安慰,并不是他要的,我不知道他想要些什么,或是他在找些什么。
\par 一轮圆月慢慢地从松树梢头升起,它缓缓地行过天际,我半睡半醒地想着事。实在是太累了。月亮、奇怪的梦、蚊子的声音、过去片断的回忆,这一切混成了一片虚幻的废弃的风景。在这个模糊的梦里月亮十分皎洁,但是仍然有一层薄雾,我和Chris正骑着一匹马,它跳过海边的一条小溪,这条小溪流过沙滩,流到大海里去了。然后梦断了……
\par 然后又回来了。
\par 在雾中似乎出现了一个人的身影,我仔细看的时候他又不见了,当我把视线转开,他就又忽然出现在我的眼角下,我想要跟他说话,叫他的名字,但是我并没有这样做,因为我一旦用任何手势或是行动去和他接触,就等于把他给变得实在了。而他并没有实体,但是我认识他,他就是Phaedrus。
\par 他是邪灵,已经发狂了,从一个无所谓生死的世界而来。
\par 梦里的人影逐渐消失,我的情绪也平缓下来……毫不急促地……让他慢慢消逝……既不相信他,也不否定他……但是头发在后脑勺缓缓地飘着……他在叫Chris吗……是吗……
\subsection*{6}
\par 早上醒来的时候,已经九点钟了,天气热得无法再继续睡下去。爬出睡袋,太阳已经高高挂在天空中,早晨的空气十分清爽而又干燥。
\par 由于晚上睡在地上,醒来的时候眼皮有些浮肿,而且关节有些疼。
\par 我的嘴很干,有些裂了,脸上跟手上都被蚊子咬了,昨天早上晒伤的地方也在痛。
\par 在松树林的另外一边是晒干的野草,还有一堆堆的沙土,反射着太阳光,亮得你无法直视。四周的热气、沉寂而荒凉的山坡地、万里无云的蓝天,这些都让我觉得空气十分沉闷。
\par 天空没有一丝云朵,今天想必又是个大热天。
\par 我走出松树林,来到草地上一块光秃秃的沙地前,看了好一会儿,径自沉思着。
\par 我决定今天的Chautauqua要先探讨Phaedrus的世界,我原本只想重述他有关科技和价值观的思想,而不想去谈他这个人,但是昨天晚上我想到的一切却让我无法这么去做,不提他这个人似乎是在逃避不该逃避的事。
\par 天刚蒙蒙亮的时候,我想起了Chris的印第安朋友,他祖母提到的一些事情澄清了我的一些思绪。她说如果埋葬一个人时出了问题,他的鬼魂就会出现。
\par 这一点的确如此,Phaedrus没有得到安葬,这就是问题的根源。
\par 后来我转过身去,看到John也起来了,满脸狐疑地望着我。他还没有完全清醒,于是绕着圈子走,想要让脑子清醒一下。不久Sylvia也起来了,她的左眼也肿了。我问她是怎么回事,她说是被蚊子叮的。我开始收拾东西,准备装上车,John也开始收拾了。
\par 收好了之后,我们又生了一堆火,Sylvia打开一包包的火腿、蛋和面包,准备做早餐。
\par 吃完早餐后,我过去把Chris摇醒,他不想起来,我又叫了他一次,他还是不肯起来,于是我抓起睡袋的尾端,像抖桌布一样地把他给抖了出来,结果他躺在松针上一直眨眼睛,花了好一会儿时间,才知道究竟是怎么回事,而我已经开始叠睡袋了。
\par 他很不痛快地吃早餐,才吃了一口,他就说不饿,他的胃还在痛。我指着下面的湖让他看,在这儿出现这样的湖是一件很奇怪的事,但是他一点儿也不感兴趣,还是照样抱怨着。我不管他了,John和Sylvia也是一样。我很高兴他们已经知道了Chris的问题,不然会引起许多摩擦。
\par 我们静静地吃完早餐,我有点出奇地安静,大概与我决定谈谈Phaedrus有关。
\par 这个时候我们大约距离湖边一百英尺之远,望过去可以看到一片广袤无垠的西部地区,光秃秃的山坡地,既没有人烟,也没有一丝声响。但是像这样的地方,会略略地提起你的精神,让你以为情形会愈来愈好。
\par 在我们把东西装上车的时候,我突然发现后轮胎已经磨损得非常厉害,像昨天那样的车速,载那么重的东西,地上又那么热,轮胎一定会这样的。车链也松脱了,于是我拿出了工具来修理,然后我不禁叫了起来。
\par “怎么回事?”John说。
\par “链条调整器的螺丝松了。”
\par 我把调整链条的螺丝旋下来检查,“是我的错,没有松开车轴的螺帽就想一次调整好。螺丝还是好的。”我指给他看。“好像是里面的螺丝松了。”
\par John盯着轮胎看了好一会儿。“你认为可以骑到城里吗?”
\par “当然可以,你可以一直骑下去,只不过链条会变得很难调整。”
\par 他很仔细地看着我把后车轴的螺帽旋下来,然后用锤子从旁边敲,一直敲到调好链条的松紧。然后使出全身力气锁紧螺帽,以防日后松脱,再换一根开口销。摩托车的车轴螺帽和汽车不同,这种不会影响轴承的松紧度。
\par “你怎么知道要这么做?”他问。
\par “你就是要把它想出来。”
\par “我不知道要从哪里开始想。”他说。
\par 我想了一下,那的确是个问题,好吧!要从哪一个地方开始呢?为了让他明白我的想法,就必须向前追溯,愈向前追溯,你就愈需要继续追溯下去,一直到原先只是沟通上的一个小问题,最后变成哲学上的大问题。我想这就是为什么要有Chautauqua的原因。
\par 我把工具箱收拾好,然后合上侧盖。
\par 我想了一下,他还是值得我向他解释的。
\par 上路之后,刚才工作的时候所流的一点汗被蒸发了,所以觉得很舒服。然后就觉得天气很炎热,很可能有26.8 度以上。
\par 路上没有其他的车子,我们一路前行,这真是出门旅行的好天气。
\par 但是现在我想开始尽一点责任,我想提到一个人,他已经离开这个世界了,他有一些理想曾经公诸于世,可是没有人相信他,也没有人真正了解他,他已经被世人遗忘。我宁可他继续被人遗忘,个中原因很快就会明了。但我别无选择,只有将他再次提起。
\par 我并不完全了解他的一生,不会有人知道的,除了Phaedrus自己之外。但是他早已作古,我们从他的著作、别人对他的谈论以及我片断的回忆中,或许可以拼凑出他的理想的一些概要。由于这一次旅程的中心思想源自于他,所以我们会紧紧跟随他的脚步。我们采用的方法比较容易让人了解,不是使用完全抽象的方法。我们的目的并不是想为他辩解,当然,也不是想歌颂他,我们主要的目的就是希望能让他永远地安息。
\par 在Minnesota州的时候,我们曾经路过一些沼泽地,我曾经提到John夫妇畏惧科技缺乏人性的一面,现在我要探讨的是相反的方向,直接进入科技的核心。
\par 这样一来,我们也就是要进入Phaedrus的世界,是他惟一熟知的世界,其中的一切都要从基本的形式去了解。
\par 基本的形式是稀有的讨论题材,因为它本身就是一种讨论的模式。比如说,你从事情的表象来讨论,或是从它们基本的形式来讨论,当你想要讨论这些讨论的模式时,你所要面临的问题就是所谓的平台的问题。因为除了这些模式自身,你将没有平台可依。
\par 前面我曾经谈论过他的基本形式世界,或者从外部角度来说,谈论过基本形式的表象,科技。现在我想应该从基本形式本身的角度来看他的基本形式世界,我想要谈的是基本形式世界的基本形式。
\par 要谈论这个,我们首先要使用二分法,但是在我使用之前,我必须先说明二分法究竟是什么和它的含意。这是一个很长的故事,而我现在只想先用二分法,然后再解释。我想把人类的知识分成两种——古典的认知和浪漫的认知。
\par 从终极的真理来看,这种二分法没有多大的意义,但是如果我们想借用古典的方式去研究基本形式世界,势必要用到这种方法,而Phaedrus认为古典和浪漫的意义如下:古典的认知认为这个世界是由一些基本形式组成的,而浪漫的认知则是从它的表象来观察。如果你拿一部发动机或是机械图,或是电子仪表给浪漫的人看,他一定不感兴趣,因为他所看到的只是表象,枯燥无味,只是列出一大堆复杂的专有名词、线条和数字,没有让他觉得有趣的事。但是如果你把这些东西拿给一个偏向古典思想的人看,他会仔细地观察,然后就会着迷,因为他看到在这些线条和符号之后是丰富的基本形式。
\par 浪漫的模式主要有丰富的灵感、想像力、创造力和直觉。最主要的是情感而非事实。和科学相对的艺术往往就是很浪漫的,它的存在不依赖于理性或是法则,而是依赖于感情、直觉和美学。
\par 在北欧的文化当中,浪漫往往和女性有关,但这并不是必然的现象。
\par 相对的,古典的思想往往依赖于理性和法则——它们是思想和行为的基本形式,在欧洲的文化当中主要与男性有关,同时科学、法律、医药等各学科都受到了古典思想的影响,因此对大部分的女性来说毫无吸引力。所以虽然骑摩托车旅行是件很浪漫的事,但是要维修、保养摩托车却全然是古典的行为。修理车子的时候,必然会弄脏手,而且全身都是油污,这些基本形式往往和浪漫的精神相冲突,因而女性很不喜欢这样。
\par 虽然在古典的认知方式当中,它的表象通常是丑陋的,但是这不是天生的。
\par 浪漫的人往往会忽略古典的美感,因为它出现得非常微妙。古典的风格往往直截了当而且完全不加修饰,不情绪化,简洁,有严谨的比例,它的目的并不是要引发别人情绪上的波动,而是要从混乱中找出秩序,所以它的风格并不自由也不自然,反而要求的是规规矩矩,所有的一切都在控制之下,而它的价值标准在于控制技巧的高低。
\par 对于一个浪漫的人来说,这种古典的方式往往显得很沉闷,呆滞而且丑陋。
\par 就像保养车子一样,车子的一切都可以分解成零部件和它们之间的关系。所有的一切都必须经过测量和证明,这就给人一种沉重的压迫感,一种永无止尽的灰暗,这就是一股死亡的势力。
\par 然而对于一个古典的人来说,浪漫的人就很轻浮而没有理性,心情起伏不定,不值得信任,只对享乐感兴趣,是一种肤浅的人,就像寄生虫一样没有内涵,无法养活自己,是社会的负担。从这里我们就差不多可以看出他们彼此之间的冲突了。
\par 这就是问题的根源,人在思考和感觉的时候往往会偏向于某一种形式,而且会误解和看轻另一种形式。然而没有人会放弃自己所看到的真理,就我所知,目前还没有人可以真正融合两者,因为这两者之间根本就找不到交会点。
\par 所以在近代古典和浪漫的文化之间,产生了严重的冲突——这两个世界逐渐分离,互相仇视,所有的人都在怀疑是否要继续这样发展下去。事实上没有人希望如此——不论他的敌手如何想。
\par 在这种情况之下,Phaedrus的思想和言论才显得重要,然而在他的时代,没有人会注意他的言论,只觉得他很古怪,不受欢迎,有一些疯狂,还有人认为他完全是一个疯子。毫无疑问,他的确是疯了。但是我们从他当时的著作中可以看出,使他发疯的正是这些对他充满敌意的看法。他古怪的行为往往使他与人疏离,然而这样一来就会造成他更古怪的行为,从而恶性循环,一直到濒临某一点,直到最后被法院所派来的警察逮捕,然后永远与社会隔绝。
\par 我们正准备左转上12 号国道,John停下来加油,我在他的旁边停下来。
\par 加油站门口的温度计显示现在是33.3 度,我说:“天气又要很难熬了。”
\par 油加好了。我们穿过街道去喝咖啡。
\par 当然Chris也已经很饿了。
\par 我告诉他我等这顿饭已经等了很久了。他要么跟我们一起吃,要么就别吃,我并没有生气,只是在述说事实。他虽然在抱怨,但是知道自己该怎么做。
\par 我从Sylvia的脸上看出她松了一口气,很明显地她认为这个问题还没有结束。
\par 我们喝完咖啡出来的时候,外面的天气酷热无比,于是我们赶快骑摩托车离开。突然我们都觉得有一阵子凉爽,但是立刻就消失了。太阳照着枯草和沙地,一切都泛成了白色,我必须眯着眼睛才不会觉得刺眼。这条12 号国道已经很老了,路况非常差,柏油路面已经遍是坑洞,高低不平。我们看到了路标,知道前面必须绕道。道路两旁经常会出现一些破旧的房舍、木板屋,和路边的小摊子。现在交通流量非常大,而我正很高兴地想着Phaedrus那个注重理智和分析的古典世界。
\par 自古以来,他的这种理性就被用来把人从周遭烦闷的环境当中提升起来,而人们往往很难发现这一点,这是因为,一旦运用这种方法成功地摆脱了这种环境,浪漫的人就转而想逃离这种方法。
\par 而他的世界不容易为人观察清楚,并不是因为它很古怪,反而是因为它太平凡了。熟悉往往也会使一个人视而不见。
\par 他看事情的方法可以被称为是注重分析的,这可以说是古典的另一种特性,他是一个完全信奉古典精神的人。为了更完整地解释古典,我想更进一步地分析“分析”的本身。首先我要提出一个分析的例子,然后再作进一步的分析。
\par 摩托车就是一个最好的例子,因为它就是由古典的人发明的,所以:为了古典、理性的分析,摩托车可以从它的组件以及功能来讨论。
\par 如果从组件来说,可以分成两种,其一是动力产生系统,其二是运转系统。
\par 动力产生系统可以分成发动机和动力传送系统。首先我们来看发动机。
\par 发动机包括动力钢体结构、油气系统、点火系统、自动控制系统和润滑系统。
\par 动力钢体结构包括汽缸、活塞、连杆、曲轴和飞轮。
\par 油气系统是发动机的一部分,包括油箱、汽油过滤器、空气过滤器、化油器和排气管。
\par 点火系统包括交流发电机、整流器、蓄电池、高压线圈和火花塞。
\par 自动控制系统包括凸轮链、凸轮轴、梃杆以及配电盘。
\par 润滑系统包括机油水泵、通道——输送机油到各个部位。
\par 动力传动系统可以辅助发动机,它包括离合器、变速器和链条。
\par 支架结构系统包括骨架,其中有踏板、座位和挡泥板。驾驶系统包括前后防震器和轮子、控制杠杆以及传动钢绳、车灯、喇叭、车速表以及里程表等等。
\par 这是从组件来看一辆摩托车,要了解这些组件的作用,必须进一步地解释它的功能。
\par 摩托车可以分成一般发动机的运转功能和特别控制功能,一般的运转功能可以分成进气行程、压缩行程、动力行程和排气行程。
\par 我可以介绍这四个行程每一阶段运作的方式,然后再介绍特别控制功能运作的情形,但是也只能提纲挈领地介绍摩托车的基本形式,就像前面所介绍的一样简短和基本,几乎任何一个组件都可以无限地讨论下去。我曾经看过一本书,整本书专门讨论触点,它是配电盘中非常小而重要的一部分,而除了我们这里所讨论的单汽缸的奥图发动机之外,还有二行程的发动机、多汽缸的发动机、柴油发动机、回转式发动机等等——但是这个例子已经够了。
\par 从这个简短的描述中,我们知道了摩托车都有哪些组件,以及它们如何运作。在这里我们还需要借助于一个图表,知道它们都在哪儿,甚至需要从机械运作原理的角度来了解它们为什么如此运作。但是我的目的并不是要详细分析摩托车,而是以其作为一个开始,提出一个认知上的模式,作为我们分析的目标。
\par 初次听到我的介绍,谁都不觉得有什么好奇怪的,就像求学的第一堂课,读教科书的开头或者第一天工作时的介绍。但是其中特别之处在于不是用它作为讨论的模式,而是作为讨论的对象。
\par 如此,其中就有一些值得我们玩味深思之处。
\par 首先我们发现前面所记叙的这一段文字有一个特点,你必须先压制住自己的看法,否则你就无法读下去,它是一个比沟里的死水还要沉闷的东西,你会读到化油器、齿轮、压缩机等等,活塞、火花塞、进气等等,如果从浪漫的角度来看就会觉得非常沉闷、丑陋而且十分笨拙,浪漫的人很少能突破这一点。
\par 但是一旦你能控制最初的反应,就会继续发现其他的内涵。
\par 首先如果单凭上面这一段描述,你完全无法了解摩托车,除非你已经知道它怎样运作。对于了解来说十分必要的即时表面印象已经消失,只有基本的形式仍然存在。
\par 第二是其中不包括观察者。我的描写并没有说,你要打开汽缸的上盖,才能够看到活塞。我并没有提到你。甚至操控者就好像机器人一样,它的操作完全机械化。在这一段描述当中完全没有任何主观的字眼,只有客观的存在。
\par 第三是其中完全没有好与坏的价值判断,只有事实。
\par 第四是这里有一把刀子在舞动,一把非常锋利的刀子。它是知识的利器,锋利到有的时候你几乎看不到它的运作。你认为这些组件就是这样的,而且各有命名,但是它们也可以有完全不同的名字或是完全不同的功能,这就看如何运用这把刀子了。
\par 比如说,自动控制系统包括凸轮链、凸轮轴、梃杆和配电盘,之所以会这样划分,就是因为这把分析的小刀。如果你到一家摩托车用品店购买摩托车的自动控制系统,他们根本就不知道你在说什么,因为他们不是这样分类的。没有任何两家制造商的分类完全相同,而每一位修理师傅所熟悉的问题和你的认知也是不同的。
\par 所以了解这把小刀是非常重要的,不要因为它把摩托车划归某一类型,你就完全相信,因而受到愚弄,把精力集中在这把小刀的本身才重要,后面我会继续介绍如何有效地运用这把刀子,作为解决古典和浪漫冲突的依据。
\par Phaedrus就非常善于使用这把刀子,他不但使用得很灵巧,而且能够产生莫大的力量。他根据自己的想法,把这个世界分成许多部分,然后把这些部分再细分下去,然后愈分愈细,一直分到他理想中的程度。我们由古典和浪漫这两个词语就能了解他的功力有多高。
\par 如果他的功夫仅止于分析,那么我宁可不去介绍它。最特殊的是,他使用这把刀子的方式很奇特,而且很有意义。
\par 没有人了解这一点,我也不相信他自己会明白,或许这是我的幻觉。但是他使用这把刀子时就好像一名刺客,而不像差劲的外科医生。也许这两者没什么区别。然而他是看到了一种病象,所以才拿起这把刀子,一刀一刀地切下去,一直切到最深处。他一直在追寻着一样东西,那才是重要的。他有所追求而且使用这把刀子,因为这是他惟一的工具,但是他太过深入,最后竟把自己给牺牲了。
\subsection*{7}
\par 现在到处都非常炎热,我已经没有办法忽视它的存在。迎面而来的风就像由火炉里吹来的一样。由于戴了护目镜,所以眼睛才比脸上其他部位觉得清凉一点儿。我的手倒没觉得很热,但手套的表面已经被汗水洇湿了好几块,而且上面还有许多条汗水干掉后留下的白色痕迹。
\par 前面的路上有一只乌鸦,正拖着一块腐烂的肉,就在我们逐渐接近的时候,它才慢慢地飞了起来。那肉看起来好像是一条蜥蜴,已经干掉了,粘在沥青上面。
\par 地平线上出现了建筑物的影子,远远看过去有些闪动,我看着地图,心想那一定是鲍曼,这让我想起了冰水和冷气。
\par 在鲍曼的街道上几乎看不到人影,虽然路上停了许多车子,告诉我们的确有人在这儿,但是大家都躲在屋子里。
\par 我们把车子停好,车头向外,以便离去的时候只要骑上去就可以走了。有一位孤单的老人戴着一顶宽边的草帽,看着我们把车子停好,然后摘掉头盔和护目镜。
\par 他说:“很热是不是?”他的脸上毫无表情。
\par John摇摇头说:“天啊!热死了。”
\par 在帽子的阴影里,老人脸上的表情似乎要变成一丝笑容。
\par “现在是几度?”John问。
\par “38.8 度,这是我刚才看到的,可能会再升到40 度。”他说。
\par 他问我们打多远的地方来,我们告诉他,他赞赏地点点头。他说这一趟不算短,然后他又问了点车子的事情。
\par 虽然我们很想赶快进去喝一杯,然后享受一下冷气,但我们并没有离开他,而是在38 度的烈日底下和他说着话。他经营过一家牧场,已经退休了。他告诉我们许多年以前他有一辆汉德森的摩托车。在这种大太阳底下他竟然想谈他的车子,这让我很高兴,我们谈了一会儿,John、Sylvia和Chris都愈来愈不耐烦。
\par 最后我们互道再见的时候,他说他很高兴认识我们,虽然仍然面无表情,但是我们觉得他说的是真心话。在大太阳底下,他踏着凝重的步伐走开了。
\par 在餐厅里我试图提起这件事,但是没有人感兴趣,John和Sylvia呆呆地坐在那儿吹冷气,一动也不动,女侍过来问我们要点什么,这才使他们恢复了一点生气。但是他们还没想好,她又走开了。
\par Sylvia说:“我不想离开这儿。”
\par 我又想起外面那位戴宽边帽子的老人,我说:“想想这里还没有冷气的时候是个什么样子。”
\par 她说:“我会的。”
\par “路上这么热,而且我的后轮胎又不行了,我们不能够超过六十英里。”
\par 他们没有任何反应。
\par 和他们比起来,Chris似乎恢复了正常,机警而且四处观望。吃的东西刚一端上来,他就狼吞虎咽了一番,我们还没有吃到一半,他就已经把他的那份吃完了,于是我们又叫了一些,他在那儿吃,我们等他。
\par 后来一路上的热浪就更凶猛了,太阳眼镜和护目镜都无济于事,你需要戴焊接工所戴的面罩。
\par 高原因被侵蚀而变成了有峡谷的山坡,远远看去是一片淡褐色,一片荒凉,只有四处散布的野草、岩石和沙地。看看高速公路的黑色路面对眼睛来说是一种松弛,所以我定睛看着它。我看见左边的排气管冒出比以往还要蓝的烟,于是我在手套的尖端吐了一点口水,一碰排气管,竟然嘶嘶作响,这不是好现象。
\par 这时候重要的是要学着忍耐,不要想去克服它……我在学习控制自己。
\par 现在我应该谈谈Phaedrus的那把刀了,这样对我们所谈论的东西会比较容易了解。
\par 他用这把刀划分这个世界,架构自己的理念。几乎每一个人都在使用自己的刀子。我们观察周遭成千上万的事物——这些不断变化的形状、被太阳照得灼热的山坡、发动机的声音、节流阀的运作,每一块岩石、野草和篱笆,还有路旁的碎片——你知道有这些东西存在,但是你并没有全部注意到它们,除非出现某些奇特的或是我们容易观察到的事物。我们几乎不可能全部意识到这些东西,而且把它们记住。那样一来,我们的心里就会充满了太多无用的细枝末节,从而无法思考。从这些观察当中,我们必须加以选择,而我们所选择的和所观察到的,永远不一样,因为经由选择而产生了变化。我们从所观察到的事物当中选出一把沙子,然后称这把沙子为世界。
\par 一旦我们手中握着这把沙子,也就是我们选择出来认知的世界,接下来就要开始分辨。这就是那把刀子。我们把沙子分成许多部分,此地、彼岸;这里、那里;黑、白;现在、过去;也就是把我们所认知的宇宙划分成许多部分。但是我们看得愈久,就愈会发现它的不同。
\par 没有两粒沙是一样的,有一些在某些方面相同,有一些在另外一方面相似,而我们可以根据彼此之间的类似和差异,堆成不同的沙堆。我们也可以按照不同的颜色、颗粒,不同的大小、不同的形状或者是否透明来分。你认为这种划分一定会有尽头,但是事实却不然,你可以一直分下去。
\par 古典的认知法就是针对这些不同的沙堆以及分类法还有彼此之间的关系,而浪漫的认知则是针对分类之前的那把沙子。它们彼此互不相容,但是都是观察世界的方法。
\par 现在有一件很重要的事,就是如何把这两者融合为一,却不伤害到彼此,这种认知法不会拒绝分类,也不会拒绝不分类。这种认知法就是直接把重点放在沙子的来源,也就是无穷的景致之中,这就是我们这位悲惨的博士Phaedrus想做的。
\par 想要了解他究竟做的是什么,就需要观察风景当中的那个他,他无法从整个风景中分离出来。他正站在沙中,把沙分成不同的沙堆。要看风景而没有看到他,那简直就等于没有看到风景。要排除解剖摩托车时心中的佛性,就等于完全排除了佛性。
\par 然而有一个一直存在的古典问题,就是摩托车的哪一部分、沙堆中的哪一粒沙才是佛陀呢?很明显地,问这个问题是找错了方向,因为佛是无所不在的;但是同样很明显地,问这种问题也没错,因为佛是无所不在的。对于佛独立于任何分析的思想之外而存在,前人已经说得很多了——有些人说得太多了,所以我怀疑根本不需要再多说什么,但是关于佛存在于分析的思想之内并指引着它的方向,很显然,还没有人讨论过。其中有历史的因素在内,但是历史不断地在演进,在这方面进一步地研究,似乎对我们的历史宝藏并没有什么坏处,反而有些好处。
\par 一旦我们把这种分析的思想,也就是那把刀应用到生活中,总会丢掉一些东西。我们都明白这一点。最起码在艺术当中是如此。这使我想起马克?吐温的经验,马克?吐温在掌握通过密西西比河的方法之后,发现这条河已经失去了它的美丽——总会丢掉一些东西,但是在艺术当中比较不受重视的东西同时也被创造出来了。让我们不要再注意丢掉了什么,而要注意获得了什么。让我们把这种过程当作再生的方式,既不好,也不坏,事实就是如此。
\par 我们经过了一座叫马马斯的城镇,John不肯停下来休息,所以我们继续往前骑,酷热依旧当头,我们骑进了一片荒地,现在我们刚刚经过了州界,进入了Montana,路旁有标示告诉我们这一点。
\par Sylvia上下挥动手臂,我按喇叭回应她,但是当我看到了标示,却一点也不高兴,因为它给我深深的震撼,而他们却毫无感觉,他们不知道我们现在是在Phaedrus曾经住过的地方。
\par 我们通过讨论古典和浪漫的认知来介绍Phaedrus,这似乎是很奇怪的方法,但又是惟一的。如果描写他的长相,或是他生活的种种情状,似乎太过肤浅,而直接去面对他,那更是一场灾难。
\par 他是一个疯子,如果你直接面对疯子,你所了解的就是他疯了,这等于是根本不了解他。要了解他,你就必须从他的角度看事情;如果你想要从疯子的角度来看事情,那么崎岖的路是惟一一条去了解他的路。不然你自己的看法会阻挡了你的视线。所以我认为只有一条路可以通到他那里,而且我们幸好还有这一条路可以走。
\par 我一直在谈论这些分析、定义还有体系,并不是为了它们本身的缘故,而是为了解Phaedrus而做的铺路的工作。
\par 我曾经告诉Chris,Phaedrus花费了一生的时间去追寻鬼魂,这是千真万确的。他所探索的就是隐身在一切科技的背后,在所有现代科学、所有西方思想背后的鬼魂——也就是理性本身。我告诉Chris他找到了,而且当他找到的时候狠狠地把他给痛打了一顿。我们从比喻的角度来看,这么说没有错。我想要讨论的就是他的发现,这个时代或许终究会有一些人发现其中的价值。过去没有人看见Phaedrus追寻的鬼魂,但是现在我想有愈来愈多的人看见了,或者在人生低潮的时候瞥见了它,它就是所谓的理性。它的表象很可能并不连贯而且毫无意义,更使得每天最平常的举止因为和其他的一切疏离而显得有些不正常。
\par 这就是日常存在的鬼魂,认为人生最终的目的,活着,是一件不可能的事,然而毕竟活着就是人生最终的目的。所以伟人们就努力医治别人,希望人可以活得长一点,而只有疯子才会追问为何如此。一个人追求长寿,就是为了活得更久,没有别的目的,这就是Phaedrus追寻的鬼魂所说的。
\par 我们在贝克停下来,在有树阴的地方温度计显示为42.2 度,我把手套摘下来,但是油箱太热了,我的手根本不能碰它,而发动机因为过热,出现了有问题的声音,情况非常糟,后车轮已经严重磨损,我用手去摸,它几乎和油箱一样热。
\par 我说:“我们一定得慢下来。”
\par “什么?”
\par “我们不应该超过五十英里。”我说。
\par John看了看Sylvia,她也看了看他。
\par 他们已经谈过我慢下来的情形。
\par John说:“我们只想赶快到达那儿。”
\par 他们两个向一间餐厅走去。
\par 链条也十分烫手,而且很干涩,我在右边的行李袋中找出一罐润滑剂,然后启动发动机,把润滑剂喷在链条上。
\par 链条非常热,润滑剂一喷上去立刻就蒸发掉了,于是我就把一点机油涂上去,让它运转一会儿,然后再关掉发动机。
\par Chris在旁边耐心地等候,然后跟我走进了餐厅。
\par “我记得你说过,第二天情绪会很低落。”Sylvia在我们走近包间的时候跟我讲。
\par “第二天或第三天。”我说。
\par “还是第四、第五天?”
\par “都有可能。”
\par 她和John又互相看了一眼,和先前的表情一样,似乎在说他们想要赶快上路,然后在前面的小镇等我。我自己也这么希望,但是如果他们骑得太快,很可能不是在小镇等我,而是在路边。
\par Sylvia说:“我真不知道这里的人怎么能够忍受这一切。”
\par 我有一点不耐烦地说:“这里的确是很糟糕,在他们来之前就已经知道这里很糟糕了,所以他们是有备而来的。”
\par 我又说:“如果一个人老是抱怨,只会让别人更难过。他们很有活力,知道该怎样活下去。”
\par John和Sylvia没有说什么,John很快喝完了他的可乐,又去喝一大杯啤酒。
\par 我出来又检查了一下车子上的行李,才发现刚绑好的行李有一些松脱,于是就重新再绑一次。
\par Chris在阳光下指着一个温度计,我们看到它已经超过了48.8 度。
\par 还没离开小镇,我就又开始流汗了,凉快干爽的时间不超过半分钟。
\par 我们几乎被这一片迎面袭来的热浪扑倒,即使戴着墨镜,我仍然得把眼睛眯成一条缝。一路上只有炙热的沙土和白亮亮的天空,所以根本没有东西可看,到处是一片白热,就像地狱一样。
\par John在前面骑得愈来愈快,我放弃跟上他的打算,然后放慢到时速五十五英里。除非你存心自找麻烦,否则在这种天气之下,你是不会骑到八十五英里的,因为很容易就会爆胎。
\par 我想他们或许会认为我刚刚说的话有点是在责怪他们,其实我的意思并非如此。我和他们一样,在这么炎热的天气里也很难过,但是实在没有必要继续讨论下去。我整天都在想着说着Phaedrus,而他们则一直在想这样的天气真难过,我想这才是真正使他们疲惫不堪的原因,就是那些令人不快的思想。
\par 至于Phaedrus本人,也有一些事值得一提:他研究逻辑,这是古典系统中的系统,主要是讲述系统思想的法则和过程,依靠逻辑才能架构分析的知识,研究彼此之间的关系。他在分析方面的智商高达一百七十,在五万人当中只有一个。
\par 他是一个很讲求系统的人,如果我们说他的思想和行为像机器一样,那就是误解他了。它不像活塞、轮子还有齿轮一样整体地运作,彼此互相支援。我想到的反而是激光,它的能量强到足以照射到月球,然后再折返回地球。Phaedrus并没有把他的精力用在启发大众的思想上,他选定一个遥远的目标,先瞄准了然后再射出去,而启发大众的工作却留给我来做。
\par 就和他的智慧一样,他非常孤独。
\par 从各项记载看来,他没有亲密的朋友,总是一个人去旅游,即使有别人在场,他也常落单,所以别人总觉得被他排斥,因此不喜欢他。然而别人的厌恶对他来说一点也不重要。
\par 他太太和家庭受到的创痛最深,他太太说那些想要打破他的孤独的人,最后会发现他们终将面对一片空白。我的印象中他们极渴望得到亲情,但是Phaedrus从来不曾给与过。
\par 没有人真正地了解他,这就是他想要的结果。而事实上也是如此,或许他的聪明智慧造成了他的孤独,或许他因孤独而聪明智慧。这两者总是互相影响,在那种不可思议的孤独中酝酿出来了智慧。
\par 然而这样描述他仍然不够完全,因为激光的比喻会让人以为他十分冷酷,没有感情。实情并非如此,在他对于我所谓的理性的鬼魂的追寻之中,他是一个狂热的猎人。
\par 太阳已经下山半个钟头了,天上出现了些微星光,远远地望去,原本是蓝色、黑色、灰色、褐色的树和岩石颜色都加深了。这里让我想起一段往事。Phaedrus曾经待在那儿三天没有进食,他的粮食已经吃完了,但是他为了沉思、观察而不愿意离开。他离回去的路并不远,但是他不赶时间。
\par 在黄昏幽暗的天色当中他看到一条小路,然后有晃动的影子,似乎是一条狗走过来,那是一只非常大的牧羊犬,或者像爱斯基摩狗一样,他很奇怪为什么一只这样的狗在这个时候来到这里。
\par 他不喜欢狗,但是这条狗的动作还不至于使他厌恶。这条狗似乎在监视他、评说着他,Phaedrus凝视它的眼睛好长一段时间,有一阵子似乎觉得有一点熟悉,然后这条狗就不见了。
\par 很久以后他才知道这是一只狼,这件事在他脑海中徘徊了好久,我想一定如此,因为他在狼身上看到了自己的影子。
\par 我们可以从一张照片上看到当时那一刹那静止的情景,而我也可以由镜子中看到瞬间的动作,但是我想他在山上所看到的影像完全是另一种,没有实体,根本在时空中不存在。这就是为什么他会觉得有一些熟悉。现在对于我来说,这影像已经非常地鲜明了,因为昨天晚上,我又看到他了,和Phaedrus的相貌一模一样。
\par 他和山上的那匹狼一样,有一种属于动物的神气,他自顾自地走自己的路,也不计较结果,即使有的时候结果让别人大吃一惊,而我现在听到这样的事,也是同样的反应。我发现他不会经常摇摆不定,这种勇气并不是来自于任何自我牺牲的理想,而是因为他过于热切追求,所以也无所谓有什么高贵的情操。
\par 我想他之所以会这样热切追求理性,是因为他想要在理性身上泄恨,是因为他觉得自己就是由理性塑造出来的。他想要把自己从这样的形象当中解放出来,因此他要把理性给毁了。他用很奇怪的方式达到了他的目标。
\par 他这种行径听起来似乎很脱俗,但是最奇怪的还不是这个,而是我自己与他的关系,虽然早就存在,但是现在必须提出来了。
\par 通过推论一次多年以前的经历我发现了他。有一个礼拜五我去上班,那天我完成了许多工作,所以心情很愉快,下班以后就去参加一个派对。由于跟大家说话说得太多,声音太大,酒也喝得太多,于是我就到后面的房间里面躺了一会儿。
\par 当我醒过来的时候,我发现我已经睡了一个晚上,因为天已经亮了。所以我想,“天啊!我甚至连主人的名字都不知道!”这是多么令人困窘的事情。这个房间并不像我休息的那一间,但是我进来的时候,四周一片黑暗,而且我想当时我一定已喝得烂醉了,所以也没准。
\par 我站起身来,看见我身上的衣服都已经换过了,这并不是昨天晚上我穿的那一套。我走出来,立刻吓了一跳,外面并非其他的房间,而是一条长廊。
\par 我走过这条长廊,发现每一个人都在看着我,有三次一个陌生人要我停下来,问我觉得如何,我想他们在观察我喝醉的情形,就回答他我没有宿醉,这时其中一个人笑出声来,然而立刻止住了。
\par 在走廊的尽头有一个房间,我看到里面正在进行某种活动,于是我进去在旁边坐下来,希望没有人注意我,然后我就可以想出究竟这是怎么一回事。但是有一个穿白色衣服的女人朝我走来,问我是否知道她的名字,我看到她的衬衫上有一个小小的名牌就照着念,她并不知道我看见了这个,所以很惊讶地赶忙走开了。
\par 当她回来的时候,带了一个人来,他一直瞪着我看,然后在我的旁边坐下来,问我是否知道他的名字,我也照着告诉他,但是他们很惊讶我竟然知道。
\par 他说:“这是很早期的症状。”
\par 我说:“这里好像是医院。”
\par 他们点点头。
\par “我怎么会来这儿呢?”我问道,因而想到昨天晚上的那个派对。这个人什么也没有说,而那个女的低下头来,没有再解释什么。
\par 几乎花了我一个多礼拜才从周围的事情推论出,在我醒来之前发生的都是一场梦,醒来之后所发生的才是事实,我无从判断两者之间的差异,只是不断新发生的事告诉我,喝醉酒的事似乎并不存在。有一些小事,像是门上锁了,外面是我从来没看过的景色;而由监护庭来的一张条子告诉我有人疯了,他们是在说我吗?
\par 最后有人告诉我:“现在你拥有一个全新的自己。”然而这种解释等于没有解释,因为使我比以前更困惑了,我不记得以前的那个我,如果他们说,你现在是个新人了,这样似乎有意义得多。他们错以为人格是一种物品,就好像一套衣服,可以让人换穿,但是,一个人除了人格之外,还有什么呢?只有一些骨和肉罢了,或许还有一些统计数字等等,但是肯定没有人在其中,因此人只是人格穿上骨肉和一些统计数字罢了,而不是别的。
\par 但谁又是那个以前的我呢,那个他们认识,而且认为是我的前身?
\par 这是我许多年前第一次隐约觉得Phaedrus的存在,在往后的岁月里,我又知道得更多。
\par 他已经死了,他被法院的判决给毁了,从他的脑部导入交流高压电。大约连续二十八次,每次0.5 到1.5 秒,用的大约是0.8 安培的电力,就这样通过一种科学仪器,完全不着痕迹地把他给消灭了,从而也产生了我们之间的关系。
\par 我从来没见过他,永远不可能见到了。
\par 然而有一些他记忆中的碎屑突然出现了,比如说这条路,还有岩石、白热的沙地、我们周围的一切,我知道他也看过这些,他曾经在这里,否则我不会知道的。他必定来过这里,我知道,因为我看到了这些突然发生的巧合,又想起了一些奇怪的片断,这些片断的由来我也不知道,我好像有超自然的能力,像灵媒一样能够接收另外一个世界的信息,情形就是这样,我用自己的眼睛观察事情,我也用他的眼睛观察,那是他曾经拥有过的。
\par 这些眼睛!恐怖就在这里,那双我正在看的、戴了手套的手,驾驶着摩托车一路行来,曾经是他的。如果你能够了解我这种感觉,你就能了解真正的恐惧是什么——恐惧来自于你知道自己无处可逃。
\par 我们进入了一个不太深的峡谷,路边出现了我期待已久的休息站,那儿有几张椅子、一栋小屋和几株翠绿的小树,旁边有几条浇水的管子。John正在另外一个出口,准备骑车上公路。
\par 我自顾自地在小屋前停下来,Chris跳下来,我们拉起车子的脚架,发动机散出一股热气,好像着了火一样,透过热雾,旁边的事物看上去都变了形。
\par 我从眼角看到另外一辆车子骑回来了,他们两个人都看着我。
\par Sylvia说:“我们只是很……生气!”
\par 我耸了耸肩,走到水管旁边。
\par John说:“你跟我们说过的精力都跑到哪儿去了?!”
\par 我看了他一下,知道他是真的生气了。“我想你太认真了”,我说,然后就走开了。我喝了一口水,觉得很咸,好像肥皂水一样,不过还是得喝下去。
\par John走进屋里把衣服弄湿,我检查了一下油表,虽然我戴了手套,油箱的盖子还是差点烫到我的手。发动机还有不少油,后轮又磨损了一些,但是仍然可以用,而链条仍然很紧,但是有一点干涩,所以为了保险起见,我就又涂了一点油上去,而重要部位的螺钉,仍然上得很紧。
\par John身上滴着水走过来说:“这一次让你走前面,我们走后面。”
\par 我说:“我不会骑得很快。”
\par 他说:“没有关系,我们还是会到的。”
\par 于是我走前面,但是我们慢慢地骑。
\par 峡谷里的路并不直,而且出乎我们的意料,它开始向上盘旋。
\par 路开始迂回而上,一忽儿向前,一忽儿回转,很快升高了,然后又升得更高。我们行进的路线成Z 字型,每一次都有些许上升。然后出现了一些矮树丛,之后便是小树。然后是围着篱笆的草地。
\par 在头顶上出现了一小朵云,或许会下雨吧?有可能,有草地就有雨,而这些草地里还有花朵,这一切改变得多么奇怪,在地图上完全看不到。回忆也消失了,Phaedrus一定没有来过这里,但是又没有其他的路,真奇怪。路还是继续不断地向上盘旋。
\par 这个时候太阳和云之间成了一个斜角,云已经下降到我们上方的地平线,在我们四周有灌木、松树,还有阵阵的冷风,夹杂着松树的气味。草地上的花在风中摇曳,车身有一些倾斜,这个时候我们突然觉得凉爽起来。
\par 我看了看Chris,他对我微笑,于是我也就笑了一下。
\par 然后大雨下来了,地面上浮起了一阵泥土的气息,仿佛已经等了太久,而路旁的泥土被刚打下来的雨滴弄出许多的麻点。
\par 这一切都来得那么新鲜而且正是时候,这是一场新雨。我的衣服湿了,护目镜上也溅了一些水,我感到一丝寒意,但是滋味满甜美的。云从太阳底下经过,松树上和草地上的雨珠经太阳一照便闪闪发亮。
\par 我们到达山顶,空气又干燥了,但是现在已经很凉爽,所以就停下来了,脚下是一片大峡谷和河流。
\par “我想我们已经到了。”John说。
\par Sylvia和Chris走到草地上,走到松树下的花丛里,从那儿我可以看到山谷的另一段,迂回于我们之下,那么遥远。
\par 现在我想我是一个开拓之人,正望着应许之地。
\section*{第二部分}
\subsection*{8}
\par 现在大约是早上十点钟,我坐在车子旁边一块冰凉而有树阴遮阳的石头上。这里是Montana迈尔斯城的一间饭店后面。Sylvia带着Chris到洗衣店里去替我们一行人洗衣服;John出去找一种鸭嘴兽的雕刻,好放在头盔上。他记得昨天我们刚到城里的时候,在一间修理店看到过一只;而我则要去调整一下发动机。
\par 现在我们觉得很舒服,昨天来到这里的时候已经是下午了,于是就好好地睡了一觉。停下来是对的,我们真笨,竟然不知道自己究竟有多累,John甚至累到订房间的时候都不记得我的名字。
\par 前台小姐问外面那些很帅气的摩托车是否是我们的,我们两个不禁大笑了起来,她感到很奇怪,不知自己说错了什么,其实只是因为我们实在是太累了,所以想借着大笑提提神。
\par 于是我们去洗了一个痛快的澡,在浴室的大理石地面上有一个非常精致的旧浴缸,它上了釉,并且雕成狮子的形状,洗在身上的水是这样滑润,好像所擦的肥皂一直没有洗净。后来我们又在街道上散步,像是一家人一样。
\par 我已经修理过这辆车不知道多少次了,以至于每次修理的时候几乎变成一种仪式,不再需要用多少脑筋,只需要检查一下就知道哪里不对劲。发动机出现了一些杂音,好像是梃杆松了。但是也可能是更严重的问题。所以我现在就要处理,看看是否能够解决这个问题。
\par 要调整梃杆必须得等发动机冷却下来,这就意味着如果在晚上停下来,你得到第二天早上才能修理它。这也就是为什么我要坐在Montana迈尔斯城的这间饭店后面树阴下的石头上的缘故。现在树阴下面已经十分凉爽了,大约还有一个钟头左右太阳就会落到树后边,这时正适合修理车子。有一件事情很重要,就是不可以在大太阳底下直接修理车子,或者在你累了一整天下来脑筋不清楚的时候修理,因为即使你已经修理过千百遍,你也应该在修理的时候保持机警的头脑,找出其中的问题。
\par 并不是每一个人都了解修理车子是一种多么理智的过程,他们认为这只需要熟练的技术,或者对机械的偏好。他们这么说也对,但是熟练的技术往往也是一连串推理的过程,而大部分的问题往往是因像以前的广播员所称的“在两耳之间短路了”后所产生的。所谓在两耳之间的短路也就是无法正常思考。摩托车的运作完全依照推理的过程,研究维修摩托车的艺术,就是研究理性艺术的缩影。那天我说过,Phaedrus追求的就是理性,因而才导致他的疯狂,但是在想要深入了解之前,最重要的是先要有理性的例子,这样才不会迷失在没有人能了解的抽象之中。要想谈论理性,非常容易让人迷惑,除非你能够举出融合了理性的例子。
\par 现在我们来到古典和浪漫的分界,在这边我们看到车子的外观,这是一种重要的观察方式;然而在另外一边,我们就好像修理师傅一样,看到它的基本形式,这也是一种重要的观察方式。比如说:这些工具的外形就有某一种浪漫的美在其中,然而它的功用却是全然的古典,因为它的目的就是要改变车子的基本形式。
\par 第一个火花塞的内缘瓷已经非常脏了。从古典和浪漫的角度来说,这都是很糟糕的现象,因为这表示汽缸里的汽油太多,空气不足,汽油里的碳分子没有足够的氧分子和它结合,因而只能堆积在火花塞上。昨天进城的时候,发动机的运转已经变慢了,就表示有这个问题了。
\par 为了看看是否只有一个汽缸有积碳,我又检查另外一个汽缸,两个都一样,于是我就拿出一把小刀,把刀子后面暗沟里隐藏着的一根小棒子拿出来,然后把它的一端削薄了,用来消除积碳。
\par 一边做着一边想究竟是什么原因,不可能是连杆或是阀门造成的。主喷嘴的口径太大,在高速的时候总会造成这种积碳的现象,然而以前也是同样的喷嘴,为什么火花塞却干净得多呢?这真是一件奇怪的事,你总会碰到这种现象,如果你想要把它们一次解决,你永远没有办法修好机器。由于一时找不到答案,我只好让问题悬着。
\par 第一个梃杆没有问题,不需要任何调整,所以我就去看第二个梃杆,太阳还有许久才会彻底落山……在我修理的时候,我总觉得像在教堂里,测量仪就好像一尊神像,而我正在进行一场神圣的仪式。它是一种所谓“测量精确的仪器”,从古典的角度来看,这么说意义深长。
\par 就摩托车而言,保持这种精准并不是为了追求浪漫或是完美,只是因为车子内部的热能和爆炸性的压力,只有这种精密仪器才能控制。每一个爆炸发生后,就会推动连杆和曲轴,它的压力达到每平方英寸好几吨。如果由连杆到曲轴的动作很精确,燃烧爆炸的力量就会传送得很均匀,机件也就承受得起这样的爆炸,但是如果其中有千分之一英寸的误差,那么就会传送得很突兀,像榔头的捶击一样,而连杆、轴承和曲轴里面都会受损,因而就会产生杂音,这杂音刚开始很像梃杆松掉了——这也就是为什么现在我要检查一下的原因,如果是连杆松动,而我却又硬要骑上山,那么声音就会愈来愈大,最后连连杆都会断裂,而撞击到运转的曲轴上,把整个发动机都给毁了。有的时候断裂的轴杆会打穿曲轴箱,让油漏出来,这个时候你就只能走着上山了。
\par 而要避免千分之一英寸的误差只有靠高度精密的仪器测量,那也就是古典美的所在——不是你眼睛能看见的,而是它们所代表的意义——也就是它们能够控制基本形式的能力。
\par 第二个梃杆是好的,我又检查发动机的另外一边,然后看看另外一个汽缸。
\par 精确的仪器是为了表达一种理念而设计的,如果你想要在空间上达到完美的境界是不可能的。因为摩托车没有任何一部分能够达到完美,但是如果你很接近完美,就会发生令你惊讶的事,因为它可以在极限之内,奇妙地飞驰过乡村田野。所以最基本的就是要了解这种理念。John看到摩托车的时候,只看到各种不同的结构,于是就厌恶它,然后拒绝进一步的接触。但是在我的眼睛里,我却看到设计者的理念。John认为我接触的是各种零件,实际我接触的是各种观念。
\par 昨天我曾经谈到过这些观念,我说一辆摩托车可以根据它的组件和功能分成两大部分,当我这么说的时候,我就是在列下面的表:摩托车
\par 组件功能
\par 然后我提到组件又可以细分为动力产生系统和动力传动系统,这个表就变成这样:摩托车
\par 组件功能
\par 动力产生系统动力传动系统这样你就会明白,我每划分一次,就会产生更多的枝节,最后变成一座巨大的金字塔。然后你看到我划分得愈来愈细,就好像在建立一种结构。
\par 这种观念的结构称之为体系,而自古即为所有西方知识的基本结构。王权、帝国、教会、军队,所有这一切都曾经成为一种体系。现代的企业也是一种结构。参考资料的内容、机械的组合、电脑的软件、所有科学和科技的知识都是运用这种结构,所以像生物这一类的知识就产生了门纲目属种的体系。
\par 比如摩托车下分为组件和功能,组件又下分为动力产生系统和动力传动系统等等,还有许多其他的结构也是这样组成的。比如说:A 在最上面,A 分出B,B分出C,C 分出D。我们要介绍摩托车的功能也可以用这种方式。这些结构彼此互相牵动,这里面十分复杂,一个人往往穷毕生之力也无法了解其中的哪怕仅仅只是一小部分。所有互相牵动的结构整体地被称为系统。摩托车也是一种系统,一种真正的系统。
\par 如果我们认为政府或是机构也是一种系统——这样说是正确的——因为这些组织的结构就如同摩托车一样,即使他们已经丧失了其他的意义和目标,也仍然维持这样的结构。人们从早上八点到下午五点,到工厂做一些完全没有意义的事,也不去问为什么,因为这就是整个结构的要求。没有任何流氓或是坏蛋要他们这样。整个结构就是如此,它所要求的就是这样,没有人愿意因为它没有意义就承担改革整个组织结构的沉重工作。
\par 但是如果仅仅因为它们是系统,就要拆毁一座工厂或是改革政府,或是不去修理摩托车,那只是问责它的结果而非它的原因。如果只触及到问题的结果,而不知道原因在何处,是不可能有任何改变的。真正的系统、真实的系统,就是我们当前的系统观,也就是理性自身。
\par 如果把整个工厂拆毁了,而架构它的理性仍然存在,那么靠着这个理性很容易就可以建造另一座工厂。如果革命能够摧毁一个政府,但是政府背后的理性仍然完整地保存着,那么很快地又可以再建立同样的政府。我们谈论了这么多有关系统的事,然而对系统了解得仍然不够。
\par 这就是所谓的摩托车,它是由一组钢铁制的零件所组成的观念体系,其中任何一部分、任何一种形状都是由人所设计出来的……第三个梃杆也没有问题。
\par 还剩下一个,最好也没有问题……我注意到,从来没有接触过机器的人,对这一点可能不甚了解——那就是摩托车基本上是精神的产物,他们把金属制成各种形状——管子、杆子、桁梁、工具、组件——把这一切都组合起来,但不能违背它运作的理论,然后让它们以实体来运作。然而从事机械铸造、打铁或是焊接的人则不认为钢有任何形状,如果你有很好的技巧,钢就能变化出任何形状,如果你技巧不够的话,就做不出来了。
\par 如果你想做成梃杆,就必须有这种技巧,而它的形状是你所设计的。这一点很重要。钢铁?钢铁也是人所设计出来的,因为在自然界之中并没有钢铁的存在,在远古的铜器时代,就有人能告诉你这个。自然界所有的,只是可以做钢铁的原料。但是什么又是原料呢?同样的这也是人所想出来的……鬼魂!
\par 这正是Phaedrus所说的,这一切都存在于人的心中,如果不举出像发动机这样的例子来,听起来就好像是疯言疯语,一旦举出实际的例子,就不会觉得我的想法很古怪了。这样一来,你就会明白,他也说过一些重要的事情。
\par 第四个梃杆太松了,这正是我希望看到的,于是我把它调整好,并且运转了一下,它仍然固定得好好的。梃杆的杂音不见了,但是这并不意味着什么,因为汽油还没有热起来,于是我让它空转了一会儿。我把工具收好,又骑上车去找修理店。在街上有一位骑士告诉我们在哪儿可以买到链条扣和脚踏板的橡皮。Chris的双脚一定很不安分,否则脚踏板的橡皮不会这样容易被磨损。
\par 又过了好几个路口,仍然没有听到梃杆的杂音,这样就对了,我想毛病总算修好了,不过除非我们骑了三十英里以上,否则我不会下任何判断。但是在此之前,此时此刻,头顶上正是明亮的太阳,空气凉爽宜人,我的头脑也很清醒,眼前还有整整一天,我们已经快到山区了,这一天值得好好享用。你之所以会有这种感觉,是因为你爬得愈高,空气就愈稀薄了。
\par 高度的改变!这就是为什么发动机会积碳的原因了,当然,这一定就是原因。现在我们已经在二百五十英尺的高度,我最好切换到标准喷嘴,只要花几分钟的时间,就可以将怠速调快,这样就不容易熄火,而且可以爬得更高。
\par 在树阴底下,我们找到比尔的摩托车店,但是比尔并不在店里。有一位路人告诉我们说:“他很可能去钓鱼了。”
\par 可是大门仍然敞开着,我们的确是在西部了,在芝加哥和纽约不可能有人这样敞开店门而人不在店里。
\par 看见他的店,我想他肯定是一位毕业于“照相机般的大脑”学校的技师,所有的东西都四散放置,扳手、螺丝刀、旧零件、旧的摩托车、新零件、新的摩托车、目录、管子,乱放的程度使你几乎看不见工作台在哪里。我没有过目不忘的能力,所以没有办法在这样的环境之下工作。而比尔在这么杂乱的情况之下,却连想都不必想就可以顺手拿起他所需要的工具。我也见过这样的师傅,你在旁边看了会觉得不可思议,但是他们却一样能把工作做好,有的时候甚至很快。如果你曾经稍微移动过他的工具,那么他要花上好几天才能找得到。
\par 比尔笑着走进来,笑意里似乎蕴含着什么。是的,他替我拿来了一些喷嘴,但是我必须等一会儿。他得先在后院专卖哈雷零件的部门把东西卖给别人,我跟他一起走到后面看他卖除了骨架以外的整套哈雷旧零件——因为顾客已经有了。他总共才收一百二十五块美金,相当便宜的价格。
\par 回到前面来,我说:“他要把这些零件组合起来,一定对摩托车有相当的了解。”
\par 比尔笑着说:“这也是最好的学习方式。”
\par 他卖喷嘴和脚踏板的橡皮但是不卖连接扣。我把橡皮和喷嘴装好之后,慢慢地骑回饭店。
\par 回到饭店之后,Sylvia、John和Chris正带着他们的东西走下楼,看他们脸上的表情,我知道他们的心情也不错。
\par 我们来到大街上找了一间餐馆吃牛排。
\par John说:“这个城市实在不错,真的相当不错,我很惊讶竟然会有这样的城市存在。一早我四处闲逛,他们有专门面向牧人的酒吧,卖高筒靴,还卖像银元一样的皮带扣,许许多多有意思的东西……这些都是货真价实的,不只是商会贩卖的商品……在路口有一间酒吧,今天早上他们跟我说话那口气,就好像我一直都住在这里。”
\par 我们叫了不少的啤酒。我从墙上的马蹄标志知道我们正在奥林匹亚的啤酒区,所以才会点啤酒。
\par John继续说:“他们一定认为我的牧场正在放假。有一个老人告诉我,他不准备留给他儿子任何东西,我很喜欢听他这样讲。他准备把牧场留给女儿。因为该死的儿子把每一分钱都花掉了。”John大声地笑了起来,“于是他又觉得自己不应该养他们等等,我以为这种情形早在三十年前就已经没有了,但是在这里仍然存在。”
\par 女服务生端上了牛排,我们立刻就用刀切了起来,修理摩托车的工作让我的胃口奇佳。
\par John又说:“还有一些事情会引起你的兴趣。他们在酒吧里还谈到波斯曼,就是我们要去的地方。他们说Montana的州长有一张波斯曼学院激进教授的黑名单,他预备要解雇他们,结果他却在一次空难当中身亡了。”
\par 我回答:“那是许久以前的事了。”
\par 这些牛排吃起来滋味真不错。
\par “我不知道这个州里有这么多的激进分子。”
\par 我说:“在这里有各种人,但是那些只属于右翼分子。”
\par John倒了一点盐,又说:“有一家华盛顿的报纸的专栏作家来到这里,曾在昨天的专栏里写到Montana,所以他们才谈论这件事。校长也证实了这件事。”
\par “他们把名单印出来了吗?”
\par “我不知道。你认识他们吗?”
\par 我说:“名单上如果有五十个人,那么我一定是其中的一个。”
\par 他们有点惊讶地看着我,事实上我知道得并不多。当然,其实是有“他”的名字。我觉得这种说法有点不正确,于是我又解释在Montana加拉廷县,所谓的激进和别的地方意义不同。
\par 我告诉他们说:“这所院校连美国总统夫人都敢唾弃,因为她有太多令人非议的成分。”
\par “是哪一位?”
\par “伊莲娜?罗斯福。”
\par John笑着说:“天啊!他们有没有搞错。”
\par 他们还想再多听一点,但是没有什么可说的了。然后我想起了一件事:“在这种情况下,一个真正的激进分子,其实有非常坚定的立场,几乎没有任何敌手,即使有也不会受制于他,因为他的对手已经让自己显得很愚蠢,所以不论他说什么,敌手只会反衬出他的优秀。”
\par 出城的时候经过了一座公园,我昨天晚上就注意到它,它让我突然又想起一些事。那些影像在我抬头看那些树时突然浮现,在去波斯曼的路上,Phaedrus曾经在公园的椅子上睡过一晚。这就是为什么昨天我没认出这个林子,因为他是晚上去波斯曼时经过的这里。
\subsection*{9}
\par 现在我们沿着Montana的黄石谷往前行,一路上一会儿出现西部才有的山艾树,一会儿又出现中西部才有的玉米田,然后反反覆覆地交替出现,这要看是否有河水灌溉。有的时候我们也会经过没有河流的岩石区,但是通常我们都是沿着河岸前行。这时我们看到路旁有块牌子,写了些关于刘易斯和克拉克的事,他们中的某人曾经在一次从西北走廊\footnote{Northwest Passage,沿着北美大陆北极海岸,从大西洋到太平洋的航线}开始的远足中走过这条路。
\par 听起来挺不错的,正符合我们这一趟Chautauqua的旅程,因为我们的旅程也正如一条西北走廊。之后,我们又经过了不少的田野和沙漠,日子也就这样过去了。
\par 我现在想要追寻Phaedrus曾经追寻过的鬼魂——理性,基本形式枯燥、复杂、古典的鬼魂。
\par 今天早上我谈过思想的体系,现在我想谈谈如何在这些体系当中找到自己的路——那就是逻辑。
\par 在这里要提到逻辑的两种方法,归纳法和演绎法。归纳法是从观察摩托车开始,然后得到普遍性的结论。比如说,如果摩托车在路上碰到坑洞,发动机就熄火了;然后又碰到了一次,发动机又熄了;然后再碰到一次,发动机仍然熄了;之后,行在平坦的路上,就没有熄火的情形,然后再碰到一次,发动机又熄火了。那么这个人就可以合理地推断,发动机熄火是坑洞造成的,这就是所谓的归纳法,由个别的经验归纳出普遍的原则。
\par 演绎法正好相反,它是从一般的原则推论出特定的结果。比如说,我们知道摩托车有一定的结构、体系,修理人员知道喇叭是受电池的控制,所以一旦电池用完了,喇叭自然也就不会响了,这就是演绎法。
\par 要解决一般思维无法解决的难题,就要通过你的观察和手册当中所提供的结构,不断交替运用归纳法和演绎法,如此才能找到解决之道。这种交织混杂的正确程序,如果正统化,就是所谓的科学方法。
\par 事实上,我没有看过任何一个摩托车的问题会使用到全部的科学方法。一般需要修理的问题并没有这么困难。当我一想到这些科学方法,心里就会出现一个影像,那就是一座巨大的推土机——它的行动缓慢,它的工作枯燥乏味,走起来声音轰隆直响,而且动作十分笨拙,但是它所做的没有人能比。它需要的技巧很可能是非正规修理的两倍、五倍甚至十二倍,但是你知道最终必能得到成功。没有任何摩托车的问题能把它难倒,一旦你遇到真正的难题,试过了所有的办法,绞尽了脑汁仍然没有任何进展,你就会知道,这一回你真的和老天爷较上劲了。“好吧!老天爷,我所能做的就是这些了。”于是你只好祭出正统的科学方法。
\par 你先拿出一个笔记本,把所有的状况都写下来,这样你就知道情况如何,问题要怎么解决。在科学和电子技术的领域当中需要这样做。不然的话,问题会复杂到让你摸不着头脑,然后忘记该如何解决,最后只得放弃。在维修摩托车的时候,问题并没有那么复杂,但是一旦有混淆的状况,最好的方法就是把它写下来,往往就在你写下来的时候,解决的方法就浮现出来了。
\par 要把问题正确地写下来,起码要兼顾到六方面:1.问题是什么。
\par 2.假设问题的原因。
\par 3.证实每个问题的假设。
\par 4.预测实验的结果。
\par 5.观察实验的结果。
\par 6.由实验得出结论。
\par 这和许多大学,甚至高中的实验作业所提到的方法并没有不同,我们不是仅仅把它当作作业而已,我们最主要的就是要求准确地思考,否则的话,很容易就会失败。
\par 科学方法最主要的目的就是让你能够准确地知道事情的真相,而不会误入歧途。每一个维修人员、科学家或是工程师都曾经因为没有准确地思考而大伤脑筋。这就是为什么大部分科学和机械方面的研究总是显得非常沉闷而小心谨慎,如果你很草率或者面对科学材料的时候怀有浪漫的想法,那么你很快就会被它蒙蔽。即使你不给它这样的机会,仍然有可能会发生。所以在研究科学的时候,一个人必须非常地谨慎,而且严守逻辑的法则。不要在逻辑上面摔跤,否则整个科学结构很快就会垮下来。只要你的推论稍有差错,你就会陷入无底的深渊当中。
\par 在科学式的思考当中,第一步就是要把问题写下来,其中主要的技巧就是只有你确实知道的东西才写下来,写的方式最好如下:问题:你的摩托车为什么发动不了?这么问听起来似乎很呆板,但是却是正确的。它要比这样写好:电路系统有什么问题?因为你尚不清楚真正的问题是否出现在电路系统,所以你应该先说摩托车出了什么问题,然后再进行第二个步骤:假设一:问题出在电路系统,把你所能想出的假设都写下来,以后再运用实验测试出哪些是正确的,哪些又是错误的。
\par 一开始就小心谨慎地记录下来,就能节省你不少时间,也不至于完全走迷了路。所以科学问题从表面上看来往往非常枯燥,为的就是避免将来可能产生的错误。
\par 第三个步骤是实验,在浪漫的人眼中往往以为实验就等于科学,因为这是眼睛所看到的。他们看到不少的试管和奇怪的设备。研究人员走来走去,不断有新的发现。他们看不到实验原本只是庞大体系的一部分,因而他们把实验和展示混为一谈。一个人操作着价值五万美金的福兰克斯坦仪器进行科学演示,如果他事先就知道了结果,那么整件事就毫无科学可言。然而修理摩托车的人如果为了检查电池是否仍然有电而按喇叭,这却是一种真正的科学实验,因为他是用实际的行动去证实他的假设。研究电视的科学家如果很悲哀地说:“这个实验失败了,我们没有达到预期的结果。”这其实是报道人员的错误,因为一个实验并不会因为没有达到预期的结果就被称为失败了,只有它的结果无法测出假设的真假时才会被称为失败了。
\par 所以实验当中使用到的技巧只是证明假设而已,既不可以多使用一点也不可以少使用一点。如果喇叭响了修理人员就认为整个电路系统都没有问题,那么他的问题可就大了,因为他的推论不合理,喇叭会响只表示电池没有问题。
\par 为了要设计适当的实验,他必须仔细推想事物之间的直接关系。这个可以通过摩托车的结构看出来。喇叭并不会使摩托车前进,电池也不会。除非使用非常间接的方法。电路系统直接点火的部位就在火花塞,如果你不检查这个部分的电路系统,你就永远不知道是否是因为它才出了问题。
\par 为了能适当地做检查,修理人员将火花塞拔起,放到和发动机相反方向的位置上去,于是火花塞的底部就布满了电流,随后修理人员牵动内燃机的杠杆,火花塞的横沟中就会爆出一簇蓝火。他将会做出以下的两点结论:(A)电路有问题。(B)他的实验很差劲。如果他很有经验,就会多试几次。检查一下触点,想尽办法使火花塞点燃,如果无法点燃,他才会认为电路系统出了问题,实验就到此结束。这样他就证明自己的假设是正确的了。
\par 最后一部分就是做结论。做结论的时候最重要的就是把实验的结果写下来,既不可多写也不可少写。实验并没有证明他修好电路系统的时候摩托车必然能发动,因为还有其他的部位可能出了问题。他所知道的就是已经把电路系统修好,摩托车可能发动得了。所以他的问题是:电路系统出了什么问题呢?
\par 于是他又写下假设,然后进行实验。
\par 所以问题要问对,也要选择对的实验,然后才能得到正确的结论。修理人员就借着这个方法,在摩托车的整个结构当中来回穿梭,直到他找出真正的原因,一旦把机器的问题解决了,摩托车才能够继续行驶。一名没有受过训练的旁观者只看到修理人员所付出的劳力,就以为他最主要的工作在于劳力。事实上,这正是他最轻松也是工作上最小的一部分,他最重要的工作就在于仔细观察和精确思考,这就是为什么技术人员往往显得沉默寡言,甚至在做实验的时候有些畏缩。他们不喜欢在做实验的时候讲话,那样就无法专心地思考问题了。他们借着实验推论出问题的结构,然后与心里正常的运作结构相比较,所以他们看到的是基本形式。
\par 一辆后面连着拖车的汽车驶进了我们的车道,打算超我们的车,然而又无法回到他的车道。我一直闪头灯,想确定他能看到我们。他虽然看到了,但还是无法转回去。路肩非常窄,而且高低不平,如果我们开上去一定会翻下去。
\par 我一边煞车、按喇叭,一边打灯号,天啊,他紧张地朝我们的侧面驶来!我只好紧紧地贴在路边。他来了!结果在最后一刻他驶回自己的车道,和我们之间只相距几英寸而已。
\par 我们前面有一个纸箱子掉到了地上,我们接近之前盯着看了好一阵子,很明显地是从别人的卡车上掉下来的。
\par 我们从旁边闪开了。如果我们开着汽车一定会撞个满怀,或是滚到水沟里去了。
\par 我们来到爱荷华州中部的一座小镇,四周种的玉米已经长得很高了,而且闻到很浓的肥料味。我们从停车的地方来到一家又大又高的老餐厅。为了配啤酒,我叫了他们卖的所有点心,我们这才共进一顿逾时已久的午餐。我们吃的有:花生酱、爆米花、扭花脆饼、洋芋片、小鱼干、有小骨刺的熏鱼……啤酒花生、火腿腊肠面包、炸猪肉皮,以及几块芝麻饼干(里面还搀了一些我分不出味道的佐料)。
\par Sylvia说:“我还是觉得很虚弱。”
\par 或许她觉得我们的摩托车就好像那个纸箱子一样,在高速公路上一直不断地翻滚着。
\subsection*{10}
\par 出来以后,我们仍然在河谷中继续前进,头上的天空仍然因为两壁岩石的夹峙而显得狭窄,可是要比今天早上离我们近多了。我们越来越接近河流的源头,而峡谷也愈来愈窄。
\par 我们正准备探讨Phaedrus是如何离开理性思想的主流,而去追寻理性的鬼魂。
\par 他曾经反覆地对自己讲一段话,是这样的:在科学的殿堂里有许多深宅大院……有各种人住在其中,而他们住在这儿的动机也是形形色色,五花八门。
\par 有些人倾心于科学是因为有优越的智力,科学成了他们独有的活动,在其中他们得到了生动的经验,也满足了他们的野心。有一些人则完全是为了实用的目的,而将自己思考的产物献在祭坛上。如果上帝派来的天使将上面两种人从殿里驱逐出去,那么殿里很显然会空旷许多,但是里面仍然会住着一批古今人物……如果殿里只住着前述两种人,那么它就只不过是一座空木屋,只有四处攀爬的蔓草…… 那些获得天使青睐的人……有些古怪、沉默和孤独,除了同是不受欢迎的人之外,彼此之间少有相似之处。
\par 是什么把他们带进殿堂里的……答案不一而足……逃避平凡生活的芜杂和无可救药的厌倦;逃离自己欲望的束缚。
\par 一个脾气好的人想要逃离喧闹、令人紧张的环境,而来到寂静的高山,在这里你极目远眺,透过静谧清新的空气,愉快地描摹永恒宁静的山色。
\par 这段话是年轻的科学家爱因斯坦在1918 年的演讲。
\par Phaedrus在十五岁的时候就已经读完大一的科学课程,他主要研究的是生物化学,而他想专攻生物和非生物之间的界面,现在被称为分子生物学。他并未把这个当作自己进取的手段,当时他还很年轻,还有一种高贵的理想。
\par 一个人会做这样的工作,必然有接近教徒和爱人的奉献情操,他每天的努力不是靠刻意的筹划而是来自于内心的动力。
\par 如果Phaedrus研究科学为的是自己的野心,或是实用的目的,那么他就永远都不会去研究科学的假设是否是一种实体。然而他的确是跨入了这个领域,但是却对答案不满意。
\par 在所有的科学方法里面最神秘的就是假设的形成。没有人知道它们的来处。
\par 一个人坐在那儿沉思,突然之间——一闪而过——他顿悟了。一直到经过实验,才能够证明假设的真假。然而实验并不是它的源头,它的源头在别的地方。
\par 爱因斯坦曾经说过:人类用最适合自己的方式,描绘了一幅最简洁、最容易了解的世界图像。
\par 然后试着用经验取代某种层次的世界,然后征服它……他创造了这个宇宙和他感情生活的支柱,这样才能由中找到安宁,而这安宁是无法从个人狭窄的经验当中获得的……最崇高的工作……就是要建立这些宇宙基本的法则,这些法则经过演绎就能创造出现今的世界。而要通往这些法则没有合乎逻辑的路;只有靠着直觉和对经验的体谅才能进入其中……
\par 直觉?体谅?用来形容科学的源头是很奇怪的字眼。
\par 一位没有爱因斯坦那么重要的科学家认为:“科学知识来自于自然,而自然也提供了假设。”但是爱因斯坦知道,自然并没有提供假设,自然只提供了实验的材料。
\par 一位功力较差的科学家可能会认为:那么是人想出来的假设。但是爱因斯坦仍然不认为是如此。他说:“任何真正进入其中的人都不会否认,事实上惟独现象界决定了理论的系统,虽然在现象和理论之间并没有一条合乎理论的桥。”
\par Phaedrus开始对假设的本身就是一种实体非常感兴趣,这是他实验的结论。
\par 在工作中他注意到,一般认为假设可以说是科学工作中最难的一部分,但是他却认为是最简单的。很正规地把一切都精确地记下来就为假设作了提示。首先在他实验假设是否正确的时候,其他的假设又不断地涌现出来;以后在进行其他的实验时,又会涌现更多的假设。在他继续研究下去的时候,仍然会涌现出更多的假设,直到最后他才非常痛苦地发现,在他作了这么多研究之后,不论是否定或是肯定原先的假设,假设并没有减少,反而不断在增加。
\par 一开始的时候他觉得很有趣,所以他就模仿帕金森定理写了另外一个定理:能够解释任何既有现象的理性假设有无穷个。于是在他的研究工作似乎到了尽头时,他知道,如果他坐下来好好地思考一番,那么另外一个假设就会出现。屡试不爽。就在他写下这条定理之后几个月,他开始对它的幽默和好处怀疑了起来。
\par 如果这条定理属实,那么它在科学的思维上就不只是一个小瑕疵了,这条定理完全摧毁一切,因为它否认所有科学方法的效用。
\par 如果科学方法的目的就是要从一大堆的假设当中选出正确的,然而假设出现的速度远远超过实验所能处理的速度,那么很明显地就来不及证明所有的假设。如果不能够证明所有的假设,那么任何实验的结果都变得很不可靠。这样一来,整个科学的方法就缺乏建立实证知识的目标。
\par 关于这一点爱因斯坦认为:“根据进化所显示的,在历史上任何一刻,所有可想见的存在,总有一个会证明它比其他的一切要优越。”这个答案在Phaedrus看来脆弱无比,然而“在任何一刻”倒给他深深的震撼。难道爱因斯坦认为真理是一种时间的功能?这种论点会把所有科学的最基本假设都毁掉。
\par 但是我们由整个科学的历史来看,你会发现过去的事实不断被新的解释取代,每一项研究的时效也长短不一,完全没有规律,有些科学真理似乎能够持续几个世纪,有些甚至不到一年,科学真理不像教义一样能永远存在,它像所有的一切一样可以被研究。
\par 研究过科学真理之后,他对它们出现一瞬就消失的情况很懊恼,因为科学真理存留的时间和他所付出的努力正好相反。所以在20 世纪,科学研究成果的寿命似乎比19 世纪要短得多,就是因为科学研究的规模现在大多了。如果下一个世纪科学研究的速度是现在的十倍,那么任何科学研究成果的寿命,很可能只有现在的十分之一。是什么缩短了它的寿命?最主要的就是假设的增加,假设愈多,研究成果的寿命就愈短。近几十年来假设大量增加的原因似乎来自于科学方法的本身。你看得愈多,知道得就愈多。你不是从一大堆假设当中筛选出一项真理,你是不断地提供大量的假设。这也就是说,你想要借着科学方法接近真理,实际上你根本没有任何进展,甚至离它愈来愈远,这是你所运用的科学方法造成的。
\par Phaedrus所看到的只是个人之见,但是却反映出科学最真实的特性。许多年来它都被人忽视,人们期望从科学研究当中得到的结果和实际上所得到的结果,在这里正好互相冲突。然而似乎没有多少人正视这个问题。运用科学方法的目的,就是要从许多假设当中找出正确的一个,这就是科学的目的。然而我们从科学的历史来看,事实恰恰相反。
\par 各种资料、史料、理论和假设不断大量地增加,科学把人从惟一绝对的真理,引向多元、摇摆不定、相对的世界,是造成社会混乱、思想价值混淆的主要元凶。而这一切现象原本是科学要消灭的。
\par 在许多年前,Phaedrus在实验室中已经觉察到的结果,现在在这个科学世界中我们随处可见。科学反而制造出反科学的混乱。
\par 让我们再回过头来看为什么研究这个人这么重要,以及我们前面提过的古典和浪漫的差异,以及两者之间的冲突。
\par 心存浪漫的人认为科学和科技使得人的心灵更加混乱,而Phaedrus和他们不同,他受过严密的科学训练,他所能做的不只是愁眉苦脸地搓着手或者逃避,或是站在一边诅咒,而提不出任何解决方法。
\par 我曾经提过,他最后的确提出一些解决的方法,然而由于问题非常地深奥而且复杂,没有人真正了解他解决这个问题的重要性,所以不理解甚至误解他所说的。
\par 他认为引起我们目前社会种种危机的原因是理性天生的一种缺憾。除非这种缺憾能得到弥补,否则危机会一直存在。我们目前所谓的理性模式并没有把社会带向更美好的世界,反而离它愈来愈远。自从文艺复兴以来,这些模式就一直存在。只要人们主要的需求还在于衣食住行,这些模式就会存在下去,而且还会继续运作。但是对现在大部分的人来说,这些基本的需要不再是主要的问题,因而从古代流传下来的理性结构已经不符合所需,从而显露出它真正的面目——在情感上是空虚的,在美学上没有任何表现,而在灵性上更是一片空白。这就是它的现状,而且它还会持续很长的一段时间。
\par 对于这种持续扩大的社会危机,没有人了解究竟有多么严重,更不要说有任何解决之道了。我看到像John和Sylvia这样的人,在整个文明的理性结构下,活得很盲目而且很疏离。他们想要从这个结构之外寻找答案,但是却找不到持久而令人满意的答案。于是我就想到Phaedrus和他在实验室里独自想出来的解决方法——虽然关心的是同样的危机,但是却从不同的角度出发,而且是朝着相反的方向——我在这里所做的就是想要把它结合起来。问题非常庞杂——这就是为什么我有时候会有些失去方向。
\par Phaedrus从来没有遇到一个人能够真正关心这个困扰他的问题,他们似乎都这样说:“我们知道科学方法很有效,为什么要这样问呢?”
\par Phaedrus不理解这种态度,也不知道该怎么办。由于他研究科学并不是为了个人或是实用的目的,所以这使他完全停顿了下来。这就如同他在观赏爱因斯坦曾经描述过的那座澄静的山,突然在山与山之间裂开了一道沟,里面什么也没有。然后你得慢慢地、十分困难地解释它的由来。起初这些山岭看起来好像会永远存在,其实却可能变成别的东西……很可能只是他自己的幻想,所以他停下来了。
\par 因此,在十五岁的时候就已经读完了大一课程的Phaedrus,在十七岁的时候,却因为不及格而被退学了。他们认为他很不成熟,而且上课不专心。
\par 别人都无能为力,既没有办法避免它发生,也没有办法帮助他改变,除非学校修改校规,否则他一定得退学。
\par 在这种情况之下,Phaedrus觉得很震惊,于是开始了一连串心灵上的流浪和探索,最后他仍然回到我们现在所沿循的这条路,明天我会试着开始走这条路。
\par 在劳雷尔我们终于看到山了,于是我们就留在那儿过夜。晚风徐徐吹来,颇为凉爽,因为它是从山上的积雪那里吹下来的。虽然太阳在一个钟头之前就已经西沉了,天空仍然残留着一线光亮。
\par Sylvia、John、我以及Chris在逐渐沉重的暮霭当中,走在那条长长的大街上,我们可以感觉到,虽然我们在谈论其他的事情,山依然存在。我很高兴再来到这里,但也有一点哀伤。有的时候到达目的地还不如在旅途中。
\subsection*{11}
\par 我醒来的时候在想,是否是因为回忆或是空气里某些东西的关系,我才知道自己已经靠近山了。我们住在饭店里一间美丽的老木头房间里,太阳透过百叶窗照射在黑漆漆的木头上,虽然有百叶窗遮着,我仍然可以感觉出,我们已经离山不远了。因为在房里可以嗅到山的气息,那是一种很清爽、有雾气而且带着芳香的空气。
\par 我深吸了一口气,接着又吸进了另外一口,然后又一口,一直到我吸足了,我跳下床,拉起百叶窗,让所有的阳光——那些灿烂、清凉、明亮、耀眼的阳光都照进来。
\par 我有一种冲动,想去把Chris拽起来,要他起来看看这种景象。但是或许是由于我很尊重他,我让他又继续睡了一会儿。我拿起了刮胡刀和香皂,走到长廊尽头一间同样是木板搭建的盥洗室。一路走来,地板嘎嘎作响。浴室里的水非常热,几乎不适合刮胡子,但是混了冷水之后就好多了。
\par 透过镜子上面的窗户,我看到后面有一个天井,于是我梳洗好之后就走出去站在那儿。天井和饭店四周种的树顶端一般高。那些树和我一样在迎接早晨清新的空气。树枝和叶子轻轻地摇摆着,似乎也在期盼这一刻的来临。
\par Chris很快就起来了,Sylvia从房里出来说她和John已经吃过早餐了,John到外面去散步了,但是她会陪我们去吃早餐。
\par 今天早上我们爱上了周遭的一切,去餐厅的一路上也都谈着美好的事物,连早餐的蛋、煎饼和咖啡也好像从天而降的一般。Sylvia和Chris亲密地谈着他的学校、朋友和个人的事,而我在一旁静静地听着,然后透过餐厅前面宽大的玻璃窗,看看外面路上发生的事。此刻所看到的,和在南Dakota那个孤寂的夜晚所看到的是多么不同!在这些建筑之外,就是绵延不断的山脉和雪地。
\par Sylvia说John已经在城里向别人打听过,有另外一条路可以去波斯曼,从南边走黄石公园。
\par “南边?”我说,“你的意思是说,雷德洛奇?”
\par “我想是吧!”
\par 于是我想起来,那儿的六月依然是一片皑皑的白雪,“那条路的高度远在雪线之上。”
\par Sylvia问:“有那么糟吗?”
\par “一定会很冷的,”在我心里出现我们骑着摩托车经过雪地的情形,“但是一定非常壮观。”
\par 我和John碰头,然后就把事情决定了。很快地,我们经过了一条立体交叉铁道的地下通道,然后上了一条弯曲的柏油路,直往前面的山前进。这是Phaedrus一直非常熟悉的路,在这里我到处可以看见他的影子,而前面横亘的是黑色的阿布萨罗卡岭。
\par 沿着一条溪水往源头前进,溪水在一个钟头之前可能还是白雪。绿色的田野和岩石之间是溪水和小路,它们不断地往上攀升。在这样的阳光之下,周遭一切的颜色都显得非常浓重,黑色的影子、耀眼的阳光、湛蓝的天空,太阳照过来的时候,非常刺眼而且酷热,一旦来到树阴底下又突然变冷了。
\par 晚上我们和一辆蓝色的保时捷车子比赛,超过他们的时候我们吹口哨,被超过的时候,也吹口哨,于是就这样竞争了好几次,而四周是白杨、青翠的草地还有树丛。这一切都值得记忆。
\par Phaedrus也是从这一条路到高山上去的,然后又从这条路走了四五天下来,运一些东西上去,然后再继续向前行,他几乎从生理上产生了对这座山的需要。他抽象的思路已经变得这样绵长,必须要在一个非常安静的地方,才能够保持思路的清晰。稍有分心或是有其他的思想或是有责任在身,都很可能破坏思想的进展。在他发疯之前,他和别人的思考方式也非常不同。在他的思想之中,所有的一切都在不断迅速地改变,而社会的价值标准和理论也都消失了,只剩下自我的精神在鼓舞着他不断前进。早期的失败使他觉得不需要按照一般的社会标准去思考,他的思想早已很少有人能明白。他认为像学校、教会、政府和政治组织这种机构,都是想用某种特定的目标而非真理来引导别人的思考,以使他们的机构能够继续存活下去,以控制别人来继续为这些机构服务。因而,他认为早年的失败,其实对他来说是一种福气,在偶然之间就使自己从为他所设下的陷阱中逃了出来。在他的下半辈子,他对于这些机构所谓的真理警戒心变得非常高。当然一开始,他并没有这样想,只是后来逐渐演变成这样。
\par Phaedrus原本追寻的真理是侧面的真理,而不是科学正面的真理。想要研究这些正面的真理,必须受过相当的训练,但是如果是从侧面去了解真理,就要从你的眼角去观察。在实验室里,一旦你的研究开始混乱,所有的一切都不对劲,而且你掌握不住重心,甚至被意料之外的结果困住,这个时候你便觉得没有任何的进展,只能开始从侧面的角度去思考。
\par 从外表来看,他似乎是在飘浮,事实上也是如此。你想要从侧面了解真理的时候,你只有飘浮,而没有办法从任何已知的方法和过程当中去了解真相,因为正是这些方法和过程从一开始就把他局限住了。所以他只有任凭自己四处飘荡,他所能做的也就只有这些了。
\par 他飘到了军队里,军队把他送往韩国。关于这里,他有着很美妙的一段记忆。那是一面墙的画面,他站在船头,透过海港里层层的浓雾,看到那面墙闪烁着光芒,仿佛是天国的门。他一定很珍惜这些片断的回忆,因而反复思考了许久。虽然它和其他的事物并不相配,但是令他印象十分深刻,深刻到我自己也回忆了许多次,它似乎象征了某些非常重要的事,可以算是一个关键点。
\par 他在韩国时所写的信函和他早期的完全不同,就表示这也是个关键。信散发出浓烈的情感,他把观察到的一切事物都巨细靡遗地写了下来:菜市场、玻璃门会滑动的商家、石板瓦的屋顶、马路、用稻草铺的小屋,还有他所看到的其他一切。有的时候充满了狂野的热情,有的时候十分沮丧,有的时候又十分地愤怒,有的时候甚至有些幽默。他就像有些人或者是动物,从他自己也不知道的囚笼中找到了出口,然后在田野间四处游荡,狼吞虎咽着所看到的一切。
\par 后来他和一些韩国工人做朋友,这些人会说一些英语,但是想学更多去当翻译。下班之后,他就和这些人在一起待着,周末的时候他们带着他去游玩,或是穿过山野回家去看看他们的朋友和亲人,然后告诉他韩国文化中的生活方式和思考方式。
\par 他坐在美丽的山脚下,眺望着远远的黄海,山脚下的梯田里种的稻米已经成熟了,黄澄澄的。他的一位朋友和他一起看海,看见离岸边很远的地方有一些小岛。他们吃过午餐之后,在一起聊了一会儿,所谈的内容是象形文字和世界的关系。他认为,宇宙间的一切事物竟然都能够用他们那二十六个字母来描述,真是很不可思议的事。那位朋友点点头微笑着,然后吃他们自己带来的罐头食物,然后很高兴地说“不”。
\par 他往往被他们点头表示拒绝搞晕了,这就是这一段回忆的终点,但是就像刚才那一面墙一样,他曾经回忆过许多次了。
\par 最后一个值得他回忆的片断是军舰上的一个房间,当时他正在回家的路上,这个房间还没有人住过,他一个人躺在床铺上。床铺是帆布做的,然后缝在一个钢架上,就好像马戏团的跳床一样。
\par 总共有五层,整个房间都是床铺。
\par 这个房间在船的最前面,当船起伏的时候,床的帆布也跟着起伏不定,随之而来的是,他觉得胃里的东西在翻搅。
\par 他沉思着,四周的钢板突然发出一阵沉重的巨响,这时他才发现整个房间都在随着海浪忽上忽下。他以为是因为这些起伏他才无法专心阅读手中的书,后来才知道是书太艰深了。这是一本有关东方哲学的书,是他读过的最难的一本,他很高兴能够独自一个人在空旷的船舱里读这本书,否则他永远不可能读进去。
\par 这本书提到,西方人认为,理论上人的存在通常可以分成好几个部分(这就等于Phaedrus过去在实验室当中的经历),然而东方人的看法比较偏重于美感的成分(这相当于Phaedrus在韩国的经验),而这两者似乎不曾碰过面。书中提到的理论和美感与Phaedrus后来称为古典和浪漫的用法相当,并且似乎暗暗影响了Phaedrus日后使用那两个词。两者主要的差异在于,古典的事实主要是理论的,但是也有它自己的美学,而浪漫的事实主要是美感的,但是也有它自己的美学。
\par 理论的和美感的是在一个世界之中的分歧现象,而古典的和浪漫的则分属两个不同的世界。
\par Phaedrus不明白这些道理,回到西雅图之后,他从军队里退伍,坐在饭店的房间里整整两个礼拜,啃着硕大的华盛顿苹果不断地思考,然后再吃些苹果继续思考。最后思考的结果是他想回到学校里去读哲学,他飘荡不定的时期结束了,他现在很积极地追寻着某个目标。
\par 我们将要接近雷德洛奇了。一阵冷风带着松香不断向我们吹来,吹得我几乎发抖。
\par 在雷德洛奇,马路几乎延伸到山脚下,而山庞大的身影几乎遮住了街道两旁的屋顶。我们停好了车子,把沉重的行李卸下来,脱掉厚实的衣服。我经过一间滑雪店走进餐厅,餐厅的墙上挂着一张很大的照片,上面是我们将要走的山路。山路一直盘旋向上,直到超过世界上最高的一段人工路。我有些担心,我知道这种不安没有来由,所以就想借着和别人谈路况来把它忘掉。对摩托车来说,不可能坠落山谷,不会有任何危险,你只会记起有些地方可以停下来,丢下一块石头,石头会一直下落几千英尺才抵达谷底。你很自然地就会把那块石头和摩托车以及骑士联想到一起。
\par 喝完咖啡之后,我们又穿上厚重的衣服,然后再把行李安置好,然后很快地就开始沿着向上爬升的弯道往上骑。
\par 柏油路比印象中的要宽而且也安全许多。坐在摩托车上你会拥有最大限度的空间。John和Sylvia顺着U 型的山路向前骑去,等到骑回来的时候,已经在我们的上方面对着我们,和我们微笑打招呼。不一会儿,我们也到了他们的位置,然后看到了他的后背。之后又到了另外一个转弯,我们又再度相逢,大家都哈哈大笑起来。如果想事先想像这种情景并不容易,但是如果你去做,就会变得很容易了。
\par 我曾提到过Phaedrus的飘荡时期,最后他开始接受哲学思想的训练。他认为哲学是所有知识里面最高级的,所有的哲学家都这么认为,所以它几乎已经变成了一种陈词滥调。但是对他而言却是一种启示,他才发现他曾经一度认为的世界上惟一的知识——科学,其实只是哲学的一支,哲学比科学宽广许多,甚至更基本。他所问的有关无限假设的问题科学家并不感兴趣,因为这不是科学问题。科学没有办法在研究科学方法的时候,不落入会摧毁它所有答案的陷阱。
\par 所以他问的问题比科学的层次还要高。
\par 于是,Phaedrus在哲学当中发现了引领他走向科学那个问题的自然延伸。这一切究竟意味着什么呢?这一切的目的又是什么?
\par 我们在路边停下来,拍了一些照片作纪念,然后从小路走到悬崖边。在我们这条路的正下方有一辆摩托车,车子小得几乎都快看不见了。我们把自己裹得更紧,以抵挡迎面而来的寒风,然后继续向上骑。
\par 阔叶林早已消失了,只剩下一些小松树,它们枝干扭曲,形状怪异。
\par 不久这样的松树林也完全消失了,我们置身在一片高山的草原上,四下完全没有一棵树,只有一些粉红色、蓝色和白色的小花,哇!到处都是野花,只有这些小野花、野草、苔藓和地衣才能在这里继续生存下去,我们已经到达雪线之上了。
\par 我回过头去最后看了一眼峡谷,就好像看海底一样。有些人一辈子都生活在山底下,从来不知道有这么高的地方存在。
\par 路转到向阳的地方,我们离开了峡谷,进入了雪区。
\par 发动机因为缺氧而逆燃,这表示摩托车随时会熄火,但是幸亏一直没有发生。现在我们两边都是雪墙,看起来就像早春融解过后留下来的,到处都有淙淙流水四处奔窜,弄得地上如同一摊烂泥,要不就流入才长了一个礼拜的草里,或是流入小野花里。这些小小的、粉红色、蓝色、黄色和白色的野花,在黑色的阴影之中,闪烁着太阳一样的光芒。
\par 到处都是这样的风景。一束小小的、彩色的光向我射来,而它的背景却是一片沉郁的绿色和黑色。现在天空涌起一团乌云,在它的阴影中十分寒冷,而有阳光照到的地方就不一样了。我的手臂、腿和夹克在有阳光照射的地方很热,照不到就非常凉。
\par 现在雪变厚了,我们从地上的深沟知道除雪机曾经开到这里,雪堆几乎有四英尺高,然后是六英尺,然后是十二英尺,我们在两边的雪墙之间前进,几乎是走在一条用雪堆成的隧道里。隧道的上方,天空一片阴暗,等到我们钻出来的时候,才发现已经到山顶了。
\par 在山的那一边是另外一个城镇,我们的脚下是高山湖、松树,还有雪地。
\par 在它们之上和之外,我们所看到的是更远的山脉,覆盖着终年的积雪。这就是高山区的景象。我们停下来,把车停在一个转弯处,那儿有一些观光客在照相。
\par 我们四下看了看风景,看了看对方。John从他的背包里拿出相机,而我则把工具箱拿出来,在椅垫上打开,拿出螺丝刀,发动车子,然后调整汽化器,一直到怠速的声音从非常缓慢的速度逐渐加快。我实在很惊讶,这一路上它不断地出现许多次逆燃的声音,还劈啪作响,每一次我都以为它会熄火,但是一直都没有发生。我没有去调整它,想知道在一万一千英尺的高度它会如何,而现在我也没有多做修理,因为我们将往黄石公园前进,高度多少都会下降些。
\par 当我们到达高度比较低的地区时,这些声音就逐渐消失了,我们周围又是一片森林,我们在岩石、湖泊和树林之间前进,不时来一个很美妙的转弯。
\par 我现在想要谈谈思想上的高山区,最起码对我而言,和到这里的感觉很接近,所以称它为心灵的高山地带。
\par 如果人类所有已知的知识是一个巨大的体系,那么心灵的高山地带就出现在这个体系的最高处,它是所有思想当中最抽象也是最普遍的。
\par 很少有人到此一游,因为你不能从这一趟旅程当中,获得任何实质上的利益。但是就像我们周遭的这一片高山区,它有它自己庄严的美感,所以对某些人来说,即使费尽九牛二虎之力到此一游也是值得的。
\par 来到心灵的高山地带,一个人必须习惯不稳定的稀薄空气,还有大量的问题以及各种假设的答案。这种情形会不断地扩大,一直到这个人几乎无法控制,因而迟疑是否要接近它,因为他害怕很可能会在其中迷失,而且永远找不到出路。
\par 而真理究竟是什么?你怎样知道自己拥有它?我们究竟如何能有真实的认知呢?是由一个我或者是灵魂去认知的吗?或者这个灵魂仅仅等于另外一种感官?现实基本上是在不断地改变吗?或者是永远不变……当你说这个东西就表示这个东西的时候,这又是怎样的意思呢?
\par 自从开天辟地以来,在这座高山上,已经有许多前人所走过的路径,但是都被世人遗忘了。虽然他们都声称自己的答案是永存的,而且放之四海皆准,然而文化上的差异,使我们对于同样的问题有着截然不同的答案。这些答案在他们自己的体系之内可说是正确的。但即使在同样的文化之内,旧的思想仍然会被新的思想取代。
\par 所以有人认为,人类并没有任何进步,因为在文明交替的时候,大量的人口在战争中死亡,大地和海洋被大量的碎屑污染。人们的自尊被剥夺了,他们被迫接受奴隶的生活,这种生存方式比史前时代的渔猎和农牧时期不见得进步多少。这种看法较为浪漫,但是并不能成立。因为原始部落给与个人的自由远较现代人为少。古代人为道德而战的情况也远少于现代人。现代科技虽然制造了不少废物,但是它仍然有办法处置这些废物而不至于造成生态的污染。在不文明的时代,学校里的教科书常常会省略生活中的痛苦、疾病、饥荒、维持生存所需的劳苦。所以我们可以很冷静地说,以前为了生存需要承受不少痛苦,现代人与之相比,可以说有进步,而产生这种进步的力量,最主要的来自于理性。
\par 我们可以看到,很多个世纪以来,在假设、实验、结论各方面不断出现新的材料,同时也建立起它的思想体系,因而消除了那些对古人的生存不利的因素。从某个角度来看,浪漫的人对于理性的诅咒,主要是因为理性把人类从原始的状态当中提升起来,它是这样有力而又主宰了一切,因而排除了其他所有的一切,完全控制了人自己,这就是抱怨的来由。
\par Phaedrus开始在这高山地区流浪的时候,没有任何的目标,只要有路就去探索。有时候他反省起来,觉得的确是有些进步,然而展望前程,却没有人告诉他该走哪一条路。
\par 在堆积成山的现实和知识的问题当中,曾经出现过几位伟大的人物,比如苏格拉底、亚里士多德、Newton和爱因斯坦,虽然每一个人都知道他们,但是还有更多伟大的人不为人所知。有许多他从来没有听过的名字,他对这些人的思想和整个思考的方式非常着迷,他小心谨慎地跟随他们的脚步,一直到他们逐渐丧失活力才放弃。这个时候他写的东西只能算符合学院的标准,这并不表示他没有工作或是思考,而是因为他殚精竭虑地思考。在这思想的高原地带,你想得愈用力,走得就愈慢。他以科学的方法来阅读,不只读字面的意思,而且把每个句子都拿来实验,同时记下问题,以待日后解决。我的运气不错,我有他大量的笔记。
\par 最让人震惊的是,在许多年之后他所提出的言论,在当年早已说过了。所以,看到他完全没有意识到自己当时言论的重要性,实在是一件很可惜的事。
\par 就好像你看到一个人手上拿着一片片的拼图,你知道拼凑的方法,你想要告诉他,“看,这块放在这儿,那块放在那儿。”
\par 但是你不能说,你只能看着他胡乱地拼凑。当他拼错的时候,你不禁会咬牙切齿,直到他拼对的时候,才松了一口气。
\par 有的时候他自己会很沮丧,于是你想要告诉他:“不要担心,继续拼下去。”
\par 但是他实在不是一个好学生,一定是老师同情他,他才能够通过所有的课程。他对于所研究的每一位哲学家都有成见,往往把自己的看法强加在他所研究的材料上,这是非常不公平的。他十分偏心,他想要每一位哲学家都按着他的方式走,一旦结果不符他的期望,他就会非常愤怒。
\par 我想起来,有一天,早上三四点钟的时候,他坐在房间里看康德最著名的一本书,《纯粹理性批判》。他就像下棋的人在研究对手下的第一步棋一样,他要用自己的方法检验其中的每一个句子,找出里面的矛盾和前后不连贯的地方。
\par 和20 世纪的中西部美国人比起来,Phaedrus可以算是一个古怪的人,但是在他研究康德的时候,就不会有这种感觉了。他很尊敬这位18 世纪德国的哲学家,并不是同意他的看法,而是欣赏他有条理的思想。康德总是用非常合乎逻辑规律而又仔细的态度去研究那一片有关心灵的高山地带。对于现代的爬山者来说,他的思想可算是最高峰。现在我想把康德的形象放大,同时谈一点他的思想方式以及Phaedrus对他的评价,以便呈现高山地带的心灵风貌,同时也为了解Phaedrus的思想而铺路。
\par 在这种心灵的高原之中,Phaedrus开始解决古典和浪漫之间的全部问题。除非一个人了解这个高原和其周遭的关系,否则他在这儿所说的一切,很容易就被低估或者是误解。
\par 想要了解康德的人必须要知道英国哲学家大卫?休谟。他认为,一个人如果能够遵循经验中最严格的归纳和演绎的思维,就能够认识世界真正的本质,而得到某种结论。他的论点来源于下面这个问题的答案。假如婴儿生下来的时候没有所有的感觉器官,他看不见、听不到,没有触觉、嗅觉和味觉,他完全无法接收外界任何感官上的信息。如果我们通过静脉注射供给这个小孩营养,十八年后他的大脑里会有任何思想吗?如果有,这些思想是从哪里来的呢?他又是怎样得到的呢?
\par 休谟认为这个孩子不会有任何思想,他这种看法我们认为属于经验主义,也就是他相信所有的认知来自于人的感官。所以他们信奉科学的实验方法。今日大部分的常识都属于经验主义的范畴,因此绝大部分的人都会同意休谟的看法。然而在另一种文化和时代之中,很可能有不少人会有不同的看法。
\par 经验主义的第一个问题和本体的性质有关。如果我们所有的知识都来自于感官,那么给与这些感官知识的本体又是什么呢?如果排除掉感官得来的知识而想要了解这个本体究竟是什么,你将会一无所获。
\par 由于所有的知识都是来自于感官的印象,而且本体又没有感官的印象存在,所以就很自然地推论:本体并不存在,它只是我们想像出来的,完全出自于我们的内心。所以如果我们认为自己所观察到的事物来自于某个本体,就像孩童认为地球是平的,平行的两条线永远不会交叉一样,不过是一种根基薄弱的常识。
\par 经验主义的第二个问题是,如果一个人假设我们所有的知识都来自于感官,那么哪一个感官接收了因果关系的知识?
\par 休谟的答案是没有任何感官接收得到,在我们的感官世界中没有所谓的因果关系,就像本体一样,它只是当许多事件不断重复发生时,我们所想像出来的法则而已。在我们生存的世界当中,并没有真实的存在。一个人如果接受所有的知识都是来自于感官的前提,休谟认为:那么他必然很合理地认为自然和所谓自然的法则只不过是我们想像的产物。
\par 如果休谟只是推论说整个世界出自于人的想像,那么我们大可以不接受,但是他的理论结构却异常紧密。
\par 我们必须拒绝休谟的结论,但是很不巧的是,除非你同时拒绝经验论的理性本身,然后退回到中古世纪的经验理性,否则没有办法拒绝休谟的理论。康德不愿意这样做,所以康德说是休谟“唤醒了我的沉睡”,因而促使他写出最伟大的哲学作品之一— —《纯粹理性批判》,这一本书往往可以作为大学四年学习的课程。
\par 康德企图使科学的经验主义逃离被自身逻辑吞噬的命运。一开始他沿着休谟已经为他铺好的路前行,他认为:“毫无疑问地,我们所有的知识开始于经验。”但是他很快就否认所有的知识完全来自于感官,他说:“虽然所有的知识开始于经验,但是知识的累积并不是全出自于经验。”
\par 一开始他的言论似乎是在鸡蛋里挑骨头,其实并不是这样。康德绕过休谟的理论所导致的唯我主义的深渊,走出了一条完全不同的、属于自己的路。
\par 康德认为,事实上有许多知识并不是来自于感官。
\par 其中一个例子就是时间。你看不到时间,也听不到、闻不到、尝不到或者是接触不到时间,所以它并不存在于感官的世界当中。康德称时间为一种直觉,当人心接收外界的讯息时,时间必然已经存在于心中。
\par 空间也是一样。除非我们能赋与所接收的讯息以时间和空间,否则这整个世界将无法让人理解,而只是一大堆混杂的颜色、图形、噪音、气味、痛苦和味道,没有任何意义。我们之所以能通过某种特定的方式认知世界,就是因为我们应用了这样的直觉,比如空间和时间,而且这些并不是来自于我们的想像,虽然有某些纯粹的哲学理想家就这么认为。这种直觉早已存在于人性之中,所以它并不是由外界所引起的,或是由外界赋与它生命。当我们接收外界的讯息时,它提供一种审查的作用。比如说,当我们闭眼睛的时候,我们的感官告知我们世界消失了;但是我们的心灵知道,这个世界仍然存在,所以不会认同感官的讯息。所以我们认为的现实,其实是由这种直觉的观念与感官不断接收到的各种讯息相融合而成。
\par 现在让我们暂且打住,把康德的观念运用到摩托车上,看看我们和它之间的关系如何。
\par 事实上休谟认为我对于这辆摩托车的了解完全来自于我的感官系统——情形一定是这样,没有别的方法。如果我说它是由金属和其他物质造成的,他就会问,“什么是金属?”如果我说金属摸起来很坚硬、光滑而且冰冷,如果用一个更坚硬的材料来撞击它,并不会断裂,休谟就会认为这些都是眼睛、耳朵和手所感受到的,并没有实体存在。除了这些感觉之外,金属究竟是什么?当然这时候,我无言以对。
\par 但是如果没有实体,我们又怎么解释接收到的讯息呢?如果我看向左下方,能看到车把手、前轮、装地图的位置还有油箱,我从感官得到一种印象;如果我往右下方看,又看到另外一种稍有不同的情形。这两种印象不一样,平面的角度和金属的曲线也不一样,太阳照射的角度也不一样。如果没有实体,那么我无法证明这两种印象得自于同一辆摩托车。
\par 现在我们来到一个知识上真正的死胡同,你的理性原本要让事情更容易理解,但是事实上却正好相反。如果理性已经摧毁了自己的目的,那么它本身的结构势必要有所改变。
\par 这个时候康德的说法救我们脱离了险境。他说,不能由感官认知摩托车并不能证明摩托车就不存在。在我们心中有一种直觉能认知摩托车。它在时间和空间上有一种连续性,所以当一个人转头的时候,摩托车的形象也跟着改变,所以它和我们在感官上所接收到的讯息并不冲突。
\par 所以,我们前面提到的那个躺在床上十八年毫无知觉的病人,如果有一天突然让他感知到摩托车的存在,然后再去除掉他的感官知觉,那么我想在他的心中就会有休谟式的摩托车印象,也就是不具有因果观念的摩托车。但是就如同康德所说的,我们并不是那个人,在我们心中有一种直觉的摩托车形象,我们不需要怀疑它,我们能随时证实它的存在。
\par 由于多年来感官累积的资料,我们已经在心目中建立起这样一种直觉的摩托车形象。一旦有新的讯息进来,这个形象就会不断改变。就拿我所骑的这部车子来说,由于路况的关系,它的变化就非常地迅速而且短暂。这一路上,我一直都在注意而且不断修正,一旦所得的资料没有价值,我就会把它忘掉,因为还有更多新的讯息要进来。直觉中其他的变化则比较缓慢(比如说,油箱的油逐渐减少,轮胎的橡胶逐渐磨损,螺丝钉逐渐松脱)。摩托车其他方面的变化非常缓慢,因而看起来几乎像永远都会存在一样——比如说,油漆、轮子的轴承、控制的线缆——而这些其实也一直在改变,如果我们从长期的角度来看,车子由于承受路面的震动、温度的改变,以及内部零件的耗损,造成它整个骨架也会改变。
\par 它只是一部机器,一部通过直觉所了解到的摩托车,如果你停下来仔细地想一想,就会发现它才是主体。你的感官所得到的讯息只能证实它的存在,但是这些讯息并不等于它。我通过直觉所了解到的摩托车,就像我存在银行里面的钱。如果我到银行要求看我的钱,他们一定会很奇怪地看着我。因为我的钱并没放在他们的抽屉里,他们没法拿出来给我看,我的钱其实只是电脑存档里面的一个数字。但是这样就够了,因为我相信如果我需要钱的时候,银行会通过他们的系统让我取到钱。同样的,即使我的感官并没有看到真正的钱,但是我仍然有能力感受到我的钱在那儿,随时可以取用。康德的《纯粹理性批判》就是探讨我们如何得到这种直觉的知识,以及如何运用它。
\par 康德认为这种直觉的思想和感官的认知是分开的,它能够认知“哥白尼的革命”。他提到,哥白尼认为地球绕着太阳公转。这种革命性的认识似乎没有改变任何自然现象,但是却改变了人类所有的观念。照康德的说法,就是客观的世界完全没有改变,但是我们主观的认知却彻底改变了。他所带来的震撼无与伦比,他使人迈入了现代而脱离了中古时期。
\par 哥白尼所做的就是,打破了人们心中原先对世界的认知——以为地球是平的,而且在天地之中是不动的。他提出另外一种世界观,认为地球是圆的,而且绕着太阳运行,而且他证明了这两种认知是符合现存世界的。
\par 康德认为,他在形而上学上也做了同样的事,如果你假定我们脑海中的直觉观念与我们所看到的是两回事,同时能过滤我们所看到的,这就表示你和古代亚里士多德学派的观念一样,认为研究科学的人只是被动的观察者,是一块空白的平板,这样就真正误解了这个观念。康德和他数以百万计的跟随者都认为,经过这样的革命,你对于我们如何认知有了更令人满意的理解。
\par 我想要更深入地探讨这个例子,部分原因是要从更近的角度观察心灵的高原,但是更重要的是要为Phaedrus往后所做的铺路,他也带来了一场哥白尼式的革命,因而解决了古典和浪漫之间的争端。对我而言,使我对这个世界有了更满意的理解。
\par 一开始Phaedrus对康德的哲学感到非常震惊,但是逐渐他觉得它变得很迟滞,他不知道为什么。他思考过之后认为很可能与他在东方的经验有关。他以为自己已经从知识的监狱里逃了出来,但现在仿佛又到了另一座监狱。他读了康德的美学之后很失望,甚至有些愤怒,因为康德思想中所谓的美感对他而言非常丑陋。这种丑陋非常深入而且非常广泛,以至于他无从加以攻击或者避开它——似乎它早已存在于康德的整个世界之中。它不是18 世纪的或者科技的丑陋,他所读过的所有哲学作品都让他有这种感觉。他在大学里也嗅到同样的气息,在教室里、在书本里,甚至在他的身上都有这种气息。但是他不知道原因,也不知道是如何产生的。因为那是属于理性自身的丑陋,你无法摆脱。
\subsection*{12}
\par 在库克城John和Sylvia十分快活,比过去几年我所看过的他们都要快活。
\par 我们开怀地大口嚼着刚买来的热乎乎的牛肉三明治,很高兴地听他们讲述在高山地区的丰富收获。不过我并不想多说什么,只是吃着自己的三明治。
\par 从窗子看出去,在马路的另外一边是高大的松树林,许多在它们脚下经过的车子都开到公园里去了。现在我们早已离开雪线,天气变得暖和许多,但是经常还是会出现低云,很可能会下雨。
\par 我想如果我只是一个小说家,而不是Chautauqua的主讲者,我很可能会去描写John、Sylvia以及Chris的个性。借着他们不同的举动可以反映出禅的内在意义,甚至包括艺术或者是摩托车的维修、保养。那会是一部相当不错的小说,但是为了某种原因,我不想这样做。他们是我的朋友,并不是书中的人物,就像Sylvia有一次说的,“我不想被当成物体”,所以我知道的许多事情都没有写出来。这样并没有什么不好,因为他们和Chautauqua没有多大关系,这样才是对待朋友之道。同时我想你能够了解,在我前面的叙述当中,为什么我总是对他们持保留的态度并且维持相当的距离。他们曾经一直问我究竟在想什么,想要我进一步的解释,但是,如果我据实以告前一章的论点,对Chautauqua并没有真正的助益。他们只会很惊讶,而且奇怪我究竟出了什么问题。我对于我们当时的思考方式和说话方式十分感兴趣,所以不想按时吃午餐,因而在态度上有些冷淡,这可能就是问题所在了。
\par 这也是我们这个时代的问题。如今,由于人类知识的范围太过复杂,结果每一个人都变成专家,然而却造成了彼此之间的疏离感。如果有人想在各种学问之间自由地游荡,势必会和周围的人疏远。同样,必须在“此时此刻”吃午饭也是一种专门之学。
\par Chris似乎比其他的人更理解我的冷淡,或许是因为他习惯了。同时他和我的关系这么密切,所以他更关心我的态度。有的时候,我会在他脸上看到一丝忧虑,最起码是有些不安,在怀疑我为什么会这样,然后发现我生气了。如果我不看他的表现可能就不知道了。而其他的时候,他都到处跑跑跳跳。我想,为什么会这样?我发现当时我的心情其实不错。现在我看到他有点紧张。他在回答John的问题,而这问题很明显是冲着我来的。它们和明天我们要见的朋友有关,那就是狄威斯先生。
\par 我不太确定John问了什么,只补充说:“他是一位画家,在那儿的学校教艺术,他的画应该是属于抽象派的印象主义。”
\par 他们问我是怎样认识他的,我回答的是我不记得了,印象有些模糊了。我只记得一些片断,因为他们夫妇是Phaedrus朋友的朋友,所以我们就这样认识了。
\par 他们很奇怪,像我这样一个机械手册的作者竟然会认识一位抽象派的画家。我只好说我自己也不知道。我不断地回想过去,却找不到任何答案。
\par 他们的个性截然不同,这个时期Phaedrus的相片看上去表情冷淡激进——班上的同学半开玩笑地称之为破坏分子的表情——而狄威斯同时期的照片表情则非常呆板,几乎可以说是毫无表情,除了有一点点疑问的眼神。
\par 我曾经看过一部有关第一次世界大战的间谍片,这个间谍用一面可以透视的镜子仔细研究一位德国军官俘虏(两个人长得一模一样),观察了几个月之后,他可以模仿他所有的手势和说话的腔调,然后打算假冒这名脱逃的军官,潜返德国军队司令部。我还记得他第一次面对这名军官的老朋友时,他十分紧张和兴奋,想看看他们是否会看穿他的伪装。现在我对狄威斯也有同样的感觉。
\par 他会很自然地认为我是他的老朋友。
\par 屋外起了一阵薄雾,把摩托车弄湿了。我从袋子中拿出面罩,套在头盔上。
\par 我们很快就会进入黄石公园了。
\par 前面的路上有薄雾,好像云层降到山谷里了,其实并不能算是山谷,只能说是山里面的通道。
\par 我不知道狄威斯对Phaedrus的认识有多深,也不知道他希望知道Phaedrus哪些情形。我以前也有过这样的经验,所以尚能处理这种尴尬的时刻。每一次这种机会,都让我对Phaedrus有更深的认识,这些年来已经提供给我不少的资料,我都已经写出来了。
\par 在我的记忆中,Phaedrus相当敬重狄威斯,因为他不了解他。对Phaedrus来说,不了解就会产生极大的兴趣,而狄威斯的态度往往更让他迷惑。比如说,Phaedrus说一些他觉得很有趣的事,但是狄威斯会很困惑地看着他,或是用很严肃的态度面对他。有的时候Phaedrus谈起非常严肃的事或是他深深关心的事,但是狄威斯却反而会大笑一场,仿佛他听了一个很精彩的笑话。
\par 比如说,我记得他家的餐桌夹层脱落了,Phaedrus就用胶水把它粘起来,然后在松脱的地方用一大团线把餐桌脚一圈一圈地捆起来。
\par 狄威斯看到这种绑法,不知道究竟是怎么一回事。
\par Phaedrus说:“这是我最新的雕塑,你不认为很有创意吗?”
\par 狄威斯很惊讶地看着他,研究了许久才说:“你从哪儿学来的?”
\par Phaedrus以为他在开玩笑,但是看他的表情又很严肃。
\par 又有一次Phaedrus为某些不及格的学生很难过,在回家途中和狄威斯一道经过树下,他谈起这件事,狄威斯觉得很奇怪,他为什么会有这样的反应。
\par Phaedrus说:“我自己也觉得很奇怪。”
\par 然后很困惑地说:“我想或许每一个老师都可能会给最像他的人最高的分数。如果你很看重字迹,那么字写得好的学生就会得到高分;如果你的字写得很大,你就会喜欢那些写大字的学生。”
\par “当然是这样,有什么不对呢?”
\par 狄威斯说。
\par “奇怪就在这里,”Phaedrus说。“因为我觉得自己最喜欢的学生,也就是最认同的,竟然是成绩不及格的学生。”
\par 这个时候狄威斯不禁大笑起来,Phaedrus非常恼怒,他认为这是一种科学现象,可能是增加更多新知识的线索,然而狄威斯只是一个劲儿地大笑。
\par 开始他以为狄威斯只是觉得他不经意地贬低自己很好笑,但是情形并非如此,因为狄威斯并不是一个喜欢损人的人。后来他认为他的笑中含有极为深刻的道理。最好的学生往往都不及格。每一位好老师都知道这一点。狄威斯的笑解除了Phaedrus谈话中所隐含的紧张,因为他对这件事实在太严肃了。
\par 所以狄威斯这种谜样的反应,让Phaedrus认为狄威斯对事情隐藏着极多的了解。他总是把某些东西隐藏起来,不让Phaedrus知道,而Phaedrus猜不透究竟是什么事。
\par 我很清楚地记得,有一次他发现狄威斯似乎对他也有同样迷惑的看法。
\par 狄威斯工作室里的电灯开关坏了,他问Phaedrus是否知道原因,这时他有些不好意思,就像赞助艺术的人在和画家谈话,赞助人有些羞怯地想要掩饰自己知道得太少,但是脸上却又带着笑容希望能学得更多。他不像John夫妇一样仇视科技,他从来不觉得科技对他有任何特殊的威胁。其实狄威斯是支持科技的。
\par 他虽然不了解科技,但是他知道自己喜欢什么,而且总是以学习为乐。
\par 他以为问题是出在灯泡附近的电线,因为只要按紧开关,灯就灭了。如果问题是出在开关,那么灯泡出问题之前会延迟一阵子。Phaedrus不想和他争辩这个,只到对街的五金行买了一个开关,在几分钟之内就把它装好。当然电灯立刻就亮了,让狄威斯带着一脸的困惑、沮丧。“你怎么知道问题出在开关?”他问。
\par “因为我轻摇开关的时候灯光时断时续。”
\par “难道不可能是电线引起的吗?”
\par “不可能。”
\par Phaedrus自信的态度激怒了狄威斯,于是他开始争辩,“你怎么知道的?”他说。
\par “情况很明显啊!”
\par “那么为什么我没看见呢?”
\par “你一定得有经验才行。”
\par “那么情况就不是很明显了,对不对?”
\par 狄威斯总是从一个非常奇怪的角度和人争辩,因而往往让人无法回答他。
\par 就是这种角度让Phaedrus认为狄威斯对他隐瞒了些什么。一直到他待在波斯曼的最后一天,他才以他分析的和有条理的方式,看见狄威斯真正的态度。
\par 在公园的入口处,我们停下来付钱给一位戴着帽子的男士,他交给我们一张当天适用的通行证。我看见前面有一位上了年纪的观光客在给我们拍照,于是我就朝着他笑了一下。他穿着一条短裤,露出两截没有血色的小腿,穿着长统袜和皮鞋。他太太在一旁看,小腿也和他一样。走的时候我向他们挥挥手,他们也向我们挥手道再见,这是值得拍下来,留作以后纪念的一刻。
\par Phaedrus很讨厌这个公园,不知道为什么——可能因为不是他发现的,或者也不是这个原因,或许是别的原因。公园里导游解说的态度激怒了他,不过布朗克斯动物园里的观光客让他更厌恶,这里的一切和高山区是如此不同。这里好像一座巨型的博物馆,里面的展览都经过小心谨慎的修饰,让人产生真实的幻觉,然后又用铁链围起来,让孩童不至于破坏它。来到公园里的人都变得比较有礼貌,甚至可以说有些虚伪,是公园里面的气氛使他们变成这样。所以虽然他的住处离这儿不到一百英里,但他只来过一两次。
\par 噢,顺序错乱了,中间丢了大约十年。他并不是从康德就直接跳到Montana的波斯曼。在这十年间,他住在印度,在印度大学研究东方哲学。
\par 就我所知,他并没有在那里学到任何奥秘,除了不断地学习什么事都没有发生。他听哲学家的演讲,拜访虔诚的人士,一面吸收一面思考,情形就是这样。你可以从他的信件当中发现,他原先通过观察事物所归纳出的原则此刻出现了极大的混乱、矛盾、分歧,他去印度的时候,是一个经验主义的科学家,离开印度的时候仍然如此,并没有比他刚来的时候更有智慧。他接触了许多学问,这些学问都潜藏起来,一直到日后才逐渐发挥出来。
\par 有一些学问应该略加叙述,因为日后将会变得非常重要。他发现印度教、佛教和道教教义上的不同,和基督教、回教、犹太教在教义上的不同不一样。
\par 东方不曾出现圣战,因为他们口说的真实永远不是真实本身。
\par 在所有东方的宗教当中,最看重的教义就是“你之所是”与“你之所视”是不可分的。如果能够充分了解这一点,就可以说是顿悟了。
\par 逻辑就是把主客观分开,所以逻辑不是最高的智慧,想要消除这种因划分主客观所产生的幻觉,最好的方法就是减少生理、精神和情感上的活动。为了达到这个目的,有许多修炼的方法,其中最重要的一种方法,就是所谓的“禅”了。Phaedrus从来没有打坐的经验,因为他不认为这有任何意义。他在印度时,一直坚持依靠逻辑,因为他找不到任何真实的理由抛弃这种信仰。我想他这么做是值得信任的。
\par 但是,有一天在教室里,哲学教授很愉快地解说世界的幻象,这似乎是第五十次了。Phaedrus举起手来,冷冷地问他是否相信落在广岛和长崎的原子弹是一场幻觉。教授笑了笑说是的。于是他的游学就到此终止。
\par 就印度哲学的传统来说,这个回答很可能是正确的。但是对Phaedrus以及任何经常阅读报纸的人,还有关心人类大量地被摧毁的人来说,这个回答实在令人无法接受。于是他就离开了教室,离开了印度,放弃继续研究下去。
\par 他回到美国的中西部念了一个实用的新闻学位,结了婚,先后住在内华达州和新墨西哥州,做一些奇怪的工作,比如记者、科学作家以及工业广告的撰稿人。他有两个孩子,买了一个农场、一匹马、两辆车,然后逐渐地步入中年,身体开始发胖。他对理性的追求似乎已放弃了,这点非常重要,一定要了解,他放弃了。
\par 由于他放弃了,所以生活对他来说很容易打发。他工作得很勤奋,也很好相处,从当时他所写的短篇小说来看,我们偶尔会发现他内心的空虚,他的日子过得非常平淡。
\par 至于究竟是什么再次激起他的追寻并不确定。连他太太也不清楚。我猜很可能是他内在的挫折感,以及希望再度继续追寻下去的意愿。他变得成熟多了,似乎在他放弃内心的目标之后,成熟得更快。
\par 我们在加德纳走出了黄石公园。当地的雨下得似乎不多,因为在星光下,山边只有青草和鼠尾草,我们决定在这里过夜。
\par 这座城边的河岸非常高,上面架了一座桥,溪水从平滑而干净的小圆石上流过。过了桥之后,前面汽车旅馆的灯已经亮起来了。从窗户里透出人世的灯光,我看见每一间小屋前都细心地种了许多花。于是我便慢慢地走以免踩到它们。
\par 我还注意到关于小屋的一些其他现象,就指给Chris看。窗户有两层,而且是上下拉动的。关门的时候,听声音就知道很密合,没有任何松脱的迹象。
\par 所有的建筑都结构严密,虽然称不上艺术,但是做工很细。因而让我觉得这完全是手工制品。
\par 当我们从餐厅走回汽车旅馆的时候,有一对上了年纪的老夫妇坐在外面的一个小花园里,享受着夜晚习习吹来的凉风。老先生承认这些小木屋都是他盖的,而且他很高兴有人注意到这件事。
\par 他的太太请我们都坐下。
\par 我们不赶时间,于是慢慢聊着。这是公园最早的出入口,早在有摩托车之前就有了。他们也告诉我们这些年来的种种变化,从而让我们对周遭的一切有了更深刻的认识。同时也让这座小城因为这对夫妻以及过去的岁月而蒙上了美丽的色彩。John挽着Sylvia的手臂,我听到河水淙淙流过的声音,还嗅到晚风中阵阵的香气。那位老妇人对这香气很熟悉,她说这是金银花的香气。我们沉默了一会儿,我觉得有些倦意,在我们决定回房的时候,Chris几乎已经睡着了。
\subsection*{13}
\par John和Sylvia早餐吃煎饼喝咖啡,他们似乎沉醉在昨天晚上的气氛当中,但是我发现食物似乎很难下咽。
\par 今天我们会到那所学校。在这里发生过许多事,而我已经开始觉得有些紧张了。
\par 我记得曾经读过一个考古学家在中东进行挖掘的故事,了解到他第一次打开封了好几千年的坟墓的感觉。现在我觉得自己有一点像考古学家。
\par 通往李文斯敦的峡谷里有鼠尾草,它就像由这里一直长到墨西哥一样。
\par 今天早上的阳光和昨天的一样,甚至更温暖更柔和。现在我们的高度比较低了。
\par 一切都很正常。
\par 只是这种考古的情绪让我觉得周围的宁静里似乎掩藏了什么。这是一个鬼魂经常出没的地方。
\par 我实在不想去那里,我巴不得赶快转身往回走。
\par 我想大概是紧张在作怪吧!
\par 这和我记忆的片断颇为吻合。不知道有多少个早晨,在Phaedrus要去教室之前,他紧张得几乎把所有的东西都吐出来。他很讨厌站在学生面前讲话,因为这完全违拗了他孤独的生活方式。他所感受到的就是站在别人面前的恐惧,在学生面前,他所有的举动都显得十分紧张。学生曾经告诉他太太,那好像空气中的电流,当他一走进教室,所有的眼睛都会盯着他看,一直跟着他走到教室前面。于是原本高谈阔论的学生,突然之间都变成窃窃私语。在上课之前,保持了好一阵子这种情形,整堂课所有的眼光都没有从他身上离开一下。
\par 于是Phaedrus成了颇受争议的人物。
\par 大部分的学生都像避开黑死病一样地避开他,因为他们已经听到了太多有关他的故事。
\par 这个学校可以称得上是师范学院,在这里你不断地上课、上课、上课,完全没有研究的时间,也没有思考的时间,更没有参加校外活动的时间。只是不断地上课、上课、上课,一直上到你的心灵枯竭,创造力也消失了。而你成了一部机器,不断地对那些如潮水般涌来的天真学生重复同样枯燥乏味的教材。他们不了解为什么你变得这样乏味,因而对你失去了尊敬。大家也受了你的传染。
\par 你不断上课、上课、上课的原因是,这是经营一所学校最经济的方法,让外界的人误以为学生得到了完整的教育。
\par 然而Phaedrus给这所学校的称呼并没有多大的意义。事实上,和它真正的特质比起来有些荒谬。但是这个名字对他却有莫大的意义。他牢牢地记着,要在他离开之前,把这些观念深深地植入学生的心中。这个名字就是“理性教会”,如果人们了解了这个名字的意思,就不会觉得他很神秘了。
\par 在这个时候Montana的极右派发动了一次暴动,就好像在得克萨斯的达拉斯发生的一样,正好在肯尼迪总统被刺杀之前。在米苏拉的Montana立大学有一位全国知名的教授被禁止在校园里演说,因为他很可能制造纷乱。学校当局告诉教授,他所有公开发表的言论都必须经过学校公关组的审核。
\par 学校的标准被破坏了,国会曾经立法,禁止学校拒收二十一岁以上的学生,不论他是否有高中文凭。现在他们又通过了一项法律,如有学生不及格,就要罚学校八千美金。也就是说要让所有的学生都通过。
\par 刚当选的州长为了个人和政治上的理由想解聘校长,因为校长不但对他个人有敌意,同时也是民主党人,而州长并不只是个一般的共和党员,就是这位州长提供了几天前我们听到的那个五十人的黑名单。
\par 由于这种互相报复的状况,学校的经费被削减了,而校长也跟着把英语系的预算削减了一大半。当时Phaedrus正任教于英语系,而他们班上的同学大力主张学术自由。
\par Phaedrus本人放弃继续抗争,转而求助于西北评鉴委员会,希望他们能够设法制止校长的这种不法举动。除了私下联络该委员会之外,他还公开呼吁调查整个学校的情形。
\par 这个时候,Phaedrus班上有一些学生不怀好意地问他,这样做是不是不让他们得到教育的机会。
\par Phaedrus说他并无此意。
\par 然后,有一位很明显地是属于州长那派的学生,愤怒地抗议说,州议会应该出面防止学校丧失它的资格。
\par Phaedrus问他要如何进行。
\par 那位学生说他们可以通知警方来处理。
\par Phaedrus思考了一阵子,然后他发现这个学生对于学校资格评鉴的误解有多么深。
\par 为了第二天的讲座,当天晚上他为自己的行为写了一篇辩护,这就是理性教会的讲词,和他平常潦草的讲稿比起来,算是长了些,而且叙述得很详尽。
\par 文章一开始就先提到报纸上的一篇文章,说到乡间有一座教堂在入口处挂了一幅电动的啤酒招牌,因为教堂已经卖给人开酒吧。你可以想像得到,这个时候有人笑了起来。这所大学素以举行饮酒派对而闻名,因此两者的形象有些隐隐相合。报上说,有一些人向教会当局抱怨此事。这是一间天主教教堂,奉命处理这些抱怨的神父对整件事情颇为不耐。对他来说,这些人对于教会的本质无知到了令人咋舌的地步。难道他们认为那些砖墙和彩色玻璃就代表教会了吗?还是屋顶的形状代表教会呢?这种虚伪的虔诚正是教会大力反对的物质主义。这幢建筑本身并非圣地,既然移作他用就算结束了作为一间教堂的任务。
\par 所以电动啤酒招牌是挂在一间酒吧前,而不是教堂前。因此没有办法察觉这种差异的人,只是表现出了他们自己的无知罢了。
\par Phaedrus认为学校就存在这种混淆不清的状况。这就是为什么丧失资格会令人难以理解了。真正的大学本质上并不是物质的,也不是警察所能保护的一些建筑。他解释说,一所大学如果丧失了它的资格,没有人能封锁学校,不会有法律的制裁,也不需要罚款,更不会判入狱。学校不会停课,一切还是照常进行,学生就像学校还没有丧失资格的时候一样接受教育,所发生的只是撤销了对这所学校的承认而已,这和被逐出教会颇为类似。真正的大学并不听命于任何民意机关,也不是由任何建筑物所构成的,只要它自己宣布这个地方已不再是圣所,真正的大学就已经消失,所遗留下来的只是一些砖墙、藏书和种种物质的结构罢了。
\par 对于所有的学生来说,这一定是个奇怪的观念,我可以想像得到他已经期待很久了,希望将这个观念灌输给学生,因此等待他们提出问题。你认为什么才是真正的大学?
\par 为了回答这个问题,Phaedrus所做的笔记是这样写的:真正的大学并没有特定的地点,也没有校产;既不支付薪水,也不接受物质的报酬。真正的大学是心灵的世界,是多少世纪以来流传给我们的理性思想,它不存在于任何特定的建筑物之内。这种心灵的世界,许多世纪以来都是通过一群所谓的教授所传递的,而教授这个头衔并不属于真正大学的一部分,大学的本质在于流传下来的理性的自身。
\par 除了这种心灵的世界之外,不巧也有一种合法的机构有同样的名称,但是却完全是两码子事。它是非营利性的组织,隶属于州政府,同时坐落在特定的地方,它不但拥有校产,还能发薪水,收学费,还要受法律的约束。
\par 然而这种大学,也就是合法的组织,却没有办法真正提供任何教导,它不但无法激发新知识的产生,也无法衡量学问的价值。它根本就不是真正的大学,它只像教会外表的建筑一样,坐落在某个特定的地点,提供真正的教会各种有利于生存的环境。
\par Phaedrus认为,凡是没有办法觉察这种差异的人,就会误以为掌握了教会的建筑就等于掌握了教会。他们认为,学校的教授既然领了薪水,一旦得到上面的指示,就应该抛弃自己的见解,毫无异议地接受学校的指挥,就像受雇于一般公司行号,处处要为老板说话一样。
\par 他们看到的是虚假的大学,而没看到真正的大学。
\par 我第一次读到这样的言论时就注意到Phaedrus采用的分析的手法。他避免把大学分成不同的科系,然后进行分析;同时他也不像传统的划分法一样,把学校分成学生、教授和行政部门。不论你用哪一种分类法,你所得到的都只不过是一堆乏味的资料,对你并没有什么帮助,而且也跳脱不出传统的范围。但是Phaedrus却从教会和地点去谈,因此得到前所未见的真相。以此为基础,他对大学生活中一些使人迷惑却又不正常的现象做了一番解释。
\par 解释之后,Phaedrus又回到教会这个主题上。出钱兴建教会的人可能会认为,他们这样做是为了全体着想。教会有好的布道才可以让信徒面对未来的这个礼拜。主日学校\footnote{Sunday School,基督教教会为了向儿童灌输宗教思想,在星期天开办的儿童班}能够帮助小孩健康成长,布道和主管主日学校的牧师了解了这些目标,一般都会尽力配合。但是同时他也知道,他最主要的目标并不是为信徒服务。他最主要的目标就是服事上帝,一旦执事反对传道人的布道,而且威胁要削减传道人的开销时,就会产生冲突了。
\par 面对这种状况,一位真正的牧师必须表现出他没有听到这些威胁,因为他最主要的目标并不是为了服事信徒,而是上帝。
\par Phaedrus认为,理性教会追求的最主要目标,就是苏格拉底一向认为的真理。
\par 只不过它不断以不同的面貌出现在历史中,所有的一切都隶属于它。平常,这个目标和提高市民的水准不相冲突,但是在某一种情况之下就会出现对立,就像出现在苏格拉底身上的情形一样。每当曾经贡献了大量时间金钱的执事人员和立法者,和教授的言论以及公开的看法有出入时,他们就会借着行政力量,威胁要削减预算,强迫教授听命于他们。
\par 真正的神职人员在这时就应当表现出他们没有听到这些威胁,因为他们的目标并不是把服务大众放在第一,他们最主要的是要服事真理。
\par 这个就是他所谓的理性教会。毫无疑问这是他长久以来发自内心的感想。
\par 我们发现,他并没有因为他所引起的轩然大波而受到责难。他之所以能够避开周遭的指责,一方面是因为他们不愿意去支持学校的敌人;另外一方面,他们也只能暗自嫉妒自己不能拥有像他这样的动力:勇于说出真理的使命感。
\par 从他的讲词当中,我们便几乎可以了解他为什么会这样做,但是只有一点没有解释——他那狂热的态度。一个人可以信仰真理,也可以通过理性去追寻真理,或者和当局对抗,但是为什么会像他这样夜以继日地燃烧自己?
\par 心理的解释我认为并不够,怯场无法支持那种经年累月的努力。另外一种说法似乎也不正确,那就是他想弥补早年的失败。因为他并不认为被学校驱逐是一种失败,只是一团谜而已。最终我发现,他对实验室里的科学理性缺乏信心,而对理性教会又狂热地信任,正是这二者的差异为我提供了一个合理的解释。有一天我思考着这种差异,突然,我明白了,这两者原来互为因果而非对立。由于他对理性这样缺乏信心,因而才会有这么狂热的研究态度。
\par 如果你对事情有完全的信心,就不太可能产生狂热的态度。就拿太阳来说吧,没有人会为了它明天会升起而兴奋不已,因为这是必然的现象。如果有人对政治或是宗教狂热,那是因为他对这些目标或是教义没有完全的信心。
\par 他曾以耶稣会的奉献精神为例阐述自己的观点。我们从历史中可以看见,他们的热忱并非来自于天主教会,而是因为面对新教时天主教会显出了自己的弱点。所以,Phaedrus就是因为对理性缺乏信心,才成为了狂热的研究者。这种说法比较合理,同时也让其他许多事件更有说服力。
\par 很可能这就是为什么Phaedrus对教室里坐在后排表现差劲的学生有着深切的认同感。他们脸上的轻蔑神情,就和他对整个理性知识的教育所有的态度一样,两者的差异在于,他们是因为不了解所以轻视,而他则是因为了解所以轻视。他们因为不了解,所以没有解决的办法,于是必然失败,而余生将永远记得这场痛苦的经验。而他从另外一个角度产生了狂热的使命感,觉得自己必须贡献力量做点什么,这就是他为什么会十分严谨地草拟理性教会的讲词。他告诉学生,你必须对理性有信心,因为除此之外,没有什么值得信奉,但是这种信仰连他自己都没有。
\par 我们要记得,当时是20 世纪50 年代,而不是70 年代。披头士和嬉皮士振振有词地对整个体制和一丝不苟的理性主义大加攻击,几乎没有人知道整个问题牵涉得有多么深广。然而Phaedrus义无反顾地替理性教会进行了辩护。当然在Montana的波斯曼,没有任何人有理由去反对,他仿佛再一次向每一个人保证,明天太阳依旧会升起。而事实上,没有人会担心这一点,他们所怀疑的是他这个人。
\par 但是现在,这十几年来是本世纪最混乱的年代,理性被强烈批判的程度,远远超过50 年代的人所能想像的。我想,在这一次以他的发现为根基的Chautauqua之中,我们多少能进一步了解他的思想……整个问题的解决……如果他说的是真的……但是他的言论有大部分已经失散了,究竟有多少,我们无从得知。
\par 或许这就是为什么我觉得自己好像考古学家一样,同时心里有些焦虑,因为我只有这些片断的回忆,还有别人告诉我的事件。在我们愈快挖开的时候,我愈会这样想,有些坟墓最好还是不要挖开吧!
\par 我突然想起坐在我后面的Chris,我不知道他究竟知道多少,又记得多少。
\par 我们来到一个交叉路口,通往公园的路在这儿和东西高速公路干线相会。
\par 我们停下来,然后再骑上去。从这里开始,我们骑过的路一直都很低平。一直到波斯曼。现在又逐渐变成朝西而行的上坡路。突然间,我有些好奇,不知前面会是怎样的地方。
\subsection*{14}
\par 我们现在骑到了一片绿意盎然的小平原。向南望去,山顶的松树林上仍然留着去年的残雪。周围其他的山都很低矮,而且都有相当一段距离,不过边缘都很清晰、陡峭。这里就像明信片中的风景,同我隐隐约约的记忆片断很是相似,但是并不完全吻合。这条州际公路当时一定还没有建好。
\par 这时候,我又想起了那句话:到达目的地还不如在旅途中。我们已经旅行了好一阵子,现在即将到达目的地。当我将要完成这种短暂的目标,接着而来的会是空虚的感觉。我必须调整自己,以适应下一个目标。一两天之内John和Sylvia就要离开,Chris和我必须决定接下来做些什么。所有的一切必须重新计划。我对城里的大街有一点模糊的印象,但是我现在感觉自己就像观光客一样。我看到招牌的时候就有这种感觉。
\par 这里其实并不小,人流动的速度很快,因而彼此之间都不怎么认识。这儿的人口大约在一万五到三万左右,算不上是一个镇也算不上是一个城——其实什么都不算。
\par 我们在一间有黄色玻璃窗的餐厅里用过午餐。但是我一点也记不得这间餐厅,似乎Phaedrus离开这里之后它才盖好,从大街上看来印象也一样十分模糊。
\par 我找到一本电话簿,想找罗伯特?狄威斯的电话号码,但是没有找到。我拨给接线生,她也没有办法查出这个号码。
\par 我几乎无法相信,难道他们是他想像出来的吗?接线生的回答让我惊讶了一会儿,但是我想起他们给我的回函,在信里我曾告诉他们我很快就会来拜访他们,所以就安心了。爱幻想的人是不会用写信的方式的。
\par John建议我打电话到艺术系或是其他的朋友那儿,我抽了一会儿烟然后又喝了一杯咖啡。等心情放松之后再拨电话。我终于打听到地址。其实不是电话这项科技使人提心吊胆,而是通过用电话所产生的人际关系,比如像拨电话的人和接线生之间才会发生这种情况。
\par 由城里面到山里必须经过溪谷,一共不到十英里的路。一路行来,烟尘满布,溪谷里长满了高高的绿色紫花苜蓿,等待牧民收割,草很密实,看起来似乎很难通过。田野缓缓地向四面铺展开来,到山脚下的时候慢慢地升起,然后眼前突然出现一片绿意深浓的松树林。狄威斯夫妇就住在那儿。住在淡绿色和墨绿色的交接处。我从风里嗅出刚刚收割的青草气息,还有家畜的气味。走了不久又变成松树林的味道,然后又恢复暖洋洋的气息。放眼望去,是一片阳光和草地,还有逼在眼前的山色。
\par 正当我们接近松树林的时候,路上出现厚厚的一层沙石。于是我换到低速挡,每小时十英里,然后两只脚离开踏板,让车子自然滑行。接着我们转过了一个弯,突然进入松树林里。眼前是一个非常深的V 形峡谷,路边有一座灰色的大房子,房子的一边紧挨着一座巨大的铁制抽象雕塑,雕塑下坐着狄威斯,他手上还拿着一罐啤酒,正在向我们招手。这种情形简直就像旧照片里的情景一样。
\par 我正忙着向上骑,不能松手,所以就踢踢腿。狄威斯朝着我们微笑。
\par 他说:“你找到了。”然后一脸的轻松,眼中带着笑意。
\par 我说:“好久不见了。”我也觉得很高兴,虽然突然看见他并和他说话有一点奇怪。
\par 我们下车来,他和宾客站在上面的门廊里,地板尚未完工。狄威斯朝下望,离我们只有几英尺的距离,但是峡谷的坡度非常陡峭,在屋子远远的另外一边,门廊离地面就有十五英尺以上,而到下面的河水也有五十英尺远。在树木和草丛的深处,有一匹马隐约藏身其中,悠闲自在地吃草,头也不抬一下。现在我们得把头抬起来才能看到天空。在我们四周就是刚才一路上看到的墨绿色的森林。
\par “这里真美!”Sylvia说。
\par 狄威斯看着她笑了笑说:“谢谢你的夸奖,很高兴你喜欢这里。”他的声音显示十分自在。我知道说话的就是狄威斯本人,但他如今是一个全新的人物,因为他一直不断地在追求进步,所以我得重新认识他才行。
\par 我们踏上门廊的地板,在木板与木板之间有很大的空隙,像栅栏一样。由上面可以看到地板下的地面。狄威斯一面微笑着说:“我实在不知道该怎么介绍。”一面把他的朋友介绍给我,但是我左耳进右耳出,永远记不住别人的名字。
\par 他的朋友在学校里教艺术,戴了一副牛角边的眼镜,他太太有点腼腆地笑着,我想他们一定是新来的。
\par 我们谈了一会儿,狄威斯主要是向他们介绍我是谁,然后珍妮?狄威斯从门廊的转角处捧了一盘啤酒过来,她也是一位画家,又很善解人意。她走过来的时候,已经有人先从盘子里拿起一罐啤酒,代替握手寒暄。这时她说:“邻居正好送来一堆鳟鱼当晚餐。真是棒极了。”
\par 我实在很想挤出一些适当的话来回答,但是却只能点点头。
\par 我们坐下来了,我坐在阳光里,很难看清楚门廊遮阴的那一端。
\par 狄威斯看着我,想要把话题转到我的外表上。当然我的外表对他来说已经有了相当大的改变。但是不巧有人打岔,所以他转而和John聊起这一趟旅行。
\par John告诉他这一次真是棒极了,正是他们夫妇长久以来想要做的事。
\par Sylvia补充道:“就是想出来到空旷的地方走走。”
\par 狄威斯说:“Montana空旷得很。”
\par 他和John还有那位艺术家朋友很热络地谈起Montana和Minnesota之间的差异。
\par 在我们的下方有一匹马,正安静地吃草。再过去有一条湍急的小溪。他们又开始谈狄威斯在峡谷里的家园,狄威斯已经住了多久了,还有教艺术的工作是怎样的等等。John实在很有本领闲聊,而这正是我最不擅长的。所以我只是静静地听着。
\par 过了一会儿,太阳太大了,于是我就把毛衣脱掉,把衬衫解开。为了不用再眯着眼睛,我就拿出太阳镜戴上,这样子就好多了。可是又觉得太暗了,我完全看不清别人的脸。让我感觉自己仿佛与周遭的事物都隔绝了,只看得见太阳还有向阳的山坡。我想到应该把行李卸下来,但还是决定先不要提这件事。
\par 他们知道我们要住下,姑且让事情自然地发生。首先让我们轻松一下,然后再把行李卸下来,这有什么好着急的呢?
\par 不知道多久之后,我听到John说:“哪儿来的电影明星?”我知道他指的是我还有我戴的太阳眼镜。我从眼镜上方看到狄威斯和John还有他的朋友正对着我微笑,他们一定希望我也能一起聊聊旅途上的事情。
\par John说:“他们想知道万一路上机件方面出了问题该怎么办。”
\par 于是我告诉他们那一次暴风雨中Chris和我被困住,连发动机也坏掉了的事。这的确是个不错的题材,但是我知道讲的时候有一点失去重心。最后提到抛锚的原因是因为没有油的时候,必然会引起他们的一阵嗟叹。
\par Chris说:“我都告诉他要去检查油箱了。”
\par 狄威斯和珍妮谈到Chris的身材,他变得有点害羞,脸也红了起来。他们也问起Chris的妈妈和兄弟,我们尽可能地回答他们的问题。
\par 最后我觉得太阳太大了,就把椅子搬到阴凉的地方,突然间我不禁打了一个寒颤,于是就把扣子扣上。珍妮注意到这一点,说:“等到太阳下山,那才真的冷呢。”
\par 虽然太阳和山脊之间的距离很近了,但目前仍然是下午。不出半个钟头太阳就不会直射了。John问他们,冬天的时候山上的生活如何。他们谈了一会儿,也谈到雪鞋使用的情形,我只能一直静静地坐着。
\par Sylvia、珍妮还有教艺术的朋友的太太在一边聊房屋的情形,不一会儿珍妮就邀他们进去看看。
\par 接着我又想起他们说Chris长得真快,突然间考古挖坟墓的感觉又回来了,我听到他们在谈Chris住在这儿时的情形,似乎他们从来不觉得Chris离开过。
\par 我们仿佛完全活在不同的时空中。
\par 他们又谈起艺术、音乐还有戏剧方面的现状,我很惊讶,John在这方面很能聊,我对这方面并不是非常感兴趣。
\par 他很可能知道这一点,所以从来不曾和我聊过这些。正和摩托车维修的情形相反。我在想,是否现在的我就和谈起连杆和活塞时的他一样目光呆滞。
\par 但事实上,他和狄威斯真正相通的话题是Chris和我,但是自从他们提到我好像电影明星一样坐在那儿,就产生了一种很可笑的现象。因为John对我不经意的揶揄,让狄威斯有些扫兴,因而从他的声音听得出来对我更加敬重。如此一来更让John加倍地揶揄我。他们两个人都意识到了这一点,所以自然想把话题从我身上转开,但是不一会又回到同样的话题上,就这样转来转去,不时地谈一些令他们愉快的话题。
\par John说:“坐在这里的家伙告诉我们,来到这里的时候会很失望。但是到现在为止,我们一点都不觉得。”
\par 我笑了起来,我并不想改变他的看法,狄威斯也笑了起来。但是John转过身来对我说:“喂!你真的是头脑有问题,竟然要离开这种地方。我不管学校里的情形怎样,这个决定很荒谬。”
\par 我看到狄威斯看向他,十分吃惊,然后便生起气来。狄威斯看了看我,我挥挥手叫他不要计较。我们之间有一些僵持不下的气氛,但是我不知道该怎样处理。我淡淡地说了一句:“这里的风景的确很好。”
\par 狄威斯有些防卫地说道:“如果你在这儿多待一会儿,你就会看到它的另外一面。”他的朋友同意地点点头。
\par 刚才僵持不下的气氛带来一阵沉默。这是很难化解的。John并不是要故意伤人,他比别人都心软。只是John和狄威斯所认识的我截然不同,现在的我只是一个普通的中产阶级的中年人而已,心里所记挂的只是Chris,除此之外,没有任何特殊之处。
\par John不知道而狄威斯知道的是,过去曾经有过这样的一个人住在这里,他的内心燃烧着一股熊熊的创作欲望,他有一种前所未闻的思想,但是后来发生了一些无法解释的不幸,狄威斯既不知道情形,也不知道原因,就连我也不知道。至于气氛僵持不下的原因,是狄威斯觉得那个人又回来了,而我无从向他解释。
\par 刹那之间,山脊上的太阳透过树林散出光晕,落在所有人的身上,当然也落在我身上。
\par “他看到的太多了。”我说,心里仍然在想刚才僵持的气氛,但是狄威斯不解地看着我,John则毫无表情。当我发现说得不妥时已经太晚了。远处有一只鸟在悲鸣。
\par 突然间太阳落下山头,整个峡谷变得一片漆黑。
\par 我认为刚才那样说是多余的,我根本不需要这样。你离开医院的时候,就明白自己不需要这样说。
\par 这个时候珍妮和Sylvia出来了,建议我们把行李卸下来。我们站起身来,珍妮带我们到房间去。我看到床上有厚厚的棉被,再也不用怕今天晚上的寒冷。
\par 好美的房间。
\par 搬了三趟,终于把东西卸完,然后我就走到Chris的房间,看看他需不需要帮忙,但是他兴高采烈,而且也长大了,不需要任何协助。
\par 我看着他:“你喜欢这里吗?”
\par 他说:“不错啊!它一点都不像你昨天晚上说的那样。”
\par “什么时候?”
\par “就是在我们睡前,在那个小屋里。”
\par 我不知道他指的是什么。
\par 他又说:“你说这里很寂寞。”
\par “我为什么会这样说呢?”
\par “我不知道。”我的问题使他无法回答,所以就不再继续问下去,他一定是在做梦。
\par 当我们下来到客厅的时候,我能够闻到厨房里煎鳟鱼的香味。在房子的一角,狄威斯正点燃报纸,预备把炉火生起来,我们看了他一会儿。
\par 他说:“我们整个夏天都要用火炉。”
\par 我问:“有这么冷吗?”
\par Chris说他也觉得很冷,我叫他回去拿我们的毛衣。
\par 狄威斯说:“这里的晚风是从峡谷上方吹下来的,那儿才真正的冷呢。”
\par 炉子里的火忽明忽灭,我想一定是风大,我从落地窗望出去,看到林子里的树正在剧烈地摇晃着。
\par “没错,”狄威斯说,“你知道上面有多冷,你过去一直喜欢待在那儿。”
\par 我说:“这又让我想起许多事。”
\par 我想起来有一个晚上,我在山顶上生起营火,火苗比现在的这个要小,四周用岩石围起来挡风,因为四面都没有树木。火旁边是烧饭的用具,还有背包帮忙挡风,饭锅里是融化的雪水,搜集这些雪水要趁早,因为在雪线以上,太阳一下山,雪就不再融化了。
\par 狄威斯说:“你变了好多。”他观察着我,由他的表情我知道,他正在迟疑是否可以继续谈这个话题。他知道我无意再谈下去,就说:“我想我们都变了。”
\par 我回答他:“我和以前完全不同了,”
\par 我这样说可能让他心安不少,接着我又说:“后来我又发生了不少事,现在我觉得我得开始慢慢地解决一些事,最起码我是这样想,这就是我来这儿的一个原因。”
\par 他看着我,希望我能多说一些,但是他那位艺术家朋友和他太太加入了我们的谈话,于是我们就停下来了。
\par 他的朋友说:“听风声今天晚上好像有暴风雨。”
\par 狄威斯说:“我想不会。”
\par Chris拿了毛衣回来,而且问我们在峡谷里是否有鬼。
\par 狄威斯逗他说:“没有鬼,但是有野狼。”
\par Chris听了又问,“ 它们做些什么?”
\par 狄威斯说,“它们会破坏牧场。”他皱皱眉头。“它们会吃小牛和小羊。”
\par “它们会追人吗?”
\par “没听过,”狄威斯说,发现Chris很失望,又补了一句,“但是它们也可能会喔。”
\par 吃晚饭的时候,我们喝法国白兰地葡萄酒配鳟鱼,我们随意坐在客厅的椅子和沙发里。客厅里有整面玻璃墙可以俯视峡谷,然而因为现在外面一片漆黑,落地窗只映着火炉里熊熊的火光,正好和我们因为喝酒吃鱼燃起的兴奋情绪交相辉映。我们说得不多,只低声地称赞晚餐的美味。
\par Sylvia低声要John注意房间里那些大花瓶。
\par John说:“我已经注意到了,棒极了!”
\par “它们是彼得?福克斯的作品。”Sylvia接着说。
\par “是吗?”
\par “他是狄威斯先生的学生。”
\par “天啊!我那会儿差点踢倒了一只。”
\par 狄威斯在一旁笑着。
\par 后来John又喃喃自语了好几次,并且抬起头来看一看,然后说:“这样做不错……正好能烘托整个气氛……我们可以再回去住八年。”
\par Sylvia幽幽地说:“现在不要谈那件事。”
\par John看了我一会儿:“我想能够提供这样一个夜晚的人,他的朋友一定不坏。”他缓缓地点点头,“我想收回所有对你的看法。”
\par 我问他:“所有的吗?”
\par “反正有一些。”
\par 狄威斯和他的朋友笑了起来,刚才的僵持气氛有一些消散了。
\par 吃完晚饭以后,杰克和维拉夫妇来了。在我回忆的片断中,杰克是一个好人,在学校里教英语,而且自己也写作。
\par 接着又来了一位朋友,是位雕刻家,他从Montana的北部来,以养羊为生,我从狄威斯介绍的方式中知道,我可能没有见过他。
\par 狄威斯说他正想说服这位雕刻家到学校教课,我说:“那么我要先说服他不要去。”于是我就在他旁边坐下来,但是谈话一直无法展开,因为对方一直很严肃,而且说话很谨慎。很明显,是因为我并不是一个艺术家。他表现得好像我是个侦探,想要从他身上挖出什么。一直到他知道我会焊接才对我放心不少。
\par 修理摩托车倒是个不错的话题,他说他和我一样,有的时候也自己焊接。因为一旦你掌握了技术,焊接会让你非常有成就感,而且你能掌握金属的形状,所以你就有信心做任何事情。他拿出一些照片,是他焊接的作品。是由一些表面非常光滑的金属所焊接成的鸟和动物,造型非常独特。
\par 后来我过去和杰克及维拉聊天,杰克正准备到爱达荷州的波斯大学英语系当主任。他对这儿英语系的态度十分谨慎,而且有些消极,当然学校方面也是如此,否则他不会离开。我现在似乎想起他是一名小说家,在英语系任教。他不是一般以研究为主的学者。在英语系里一直都存在这种不和谐的现象,这种现象是激发Phaedrus的狂野思想的部分原因,而杰克很支持Phaedrus的看法,虽然他不完全了解Phaedrus在说些什么,但是他认为小说家比语言学家容易接纳Phaedrus的思想。这种分裂的历史很悠久,就像艺术和艺术史之间的分歧一样。一个是创作者,而另外一个则是研究创作的过程,而两者之间从来没有和平相处过。
\par 狄威斯拿出一些户外烤肉架的使用说明书,他希望我能从专业科技作者的角度加以评估。他已经花了整个下午想把烤肉架组合起来,但是他觉得说明书写得一塌糊涂。
\par 对于我来说,它们就像一般的说明手册一样,所以一时间不知道哪里出了问题。我不想明说,所以就尽可能地找毛病,其实你无法判断一份说明书是否正确,除非你能把实物拿来操作一番。
\par 然后我发现有一部分设计得非常不妥,你必须把手册翻来覆去才能对照上下文和图片。我针对这一点严厉地批评,而狄威斯在一旁附和,Chris则把手册拿去看是怎么一回事。
\par 我严厉批评这种翻阅方式可能造成的误解,我说这不是狄威斯的问题,而是手册编写得不够顺畅,才使他毫无头绪。因为这种支离破碎的语法,对工程和技术人员来说十分熟悉,但是狄威斯却无法吸收。科学所要处理的是一些零零散散的东西,其中可能存在相关性;而狄威斯所能接受的则是一连串原本就相关的事物。他希望我批评的是其中缺乏艺术性的连贯,这一向是工程人员最忽视的东西。它和其他与科技相关的事物一样,经常出现在古典和浪漫的对立中。
\par 但是Chris把说明书拿去折了一下,竟然让图、文同时呈现。我没有想到这一点。我就好像卡通影片里的人物,冲出了悬崖,一时还没有落下去。因为他尚未发现自己的困境。我点点头,大家沉默不语,然后我才发现自己忽略的地方,于是我拍拍Chris的头,大家笑了起来。笑声一直传到谷底,当笑声停歇的时候,我说:“反正……”于是大家又笑了起来。
\par 后来我说:“我想说的是,我家里有一份说明书,它为科技方面写作的水准提升开拓了一个伟大的领域。手册一开头就写,组合日制自行车需要心平气和。”
\par 这引来更多的笑声,但是Sylvia、珍妮和雕刻家都很同意我的说法。
\par “那本说明书倒不错。”这位雕刻家说,珍妮点头表示同意。
\par “这就是我保留下来的原因,”我说,“起初我笑了,因为我想起我组合过的那些自行车,当然,日本的制造过程十分草率,但是这句话其实隐藏了许多智慧。”
\par John和我会心地一笑,他说:“教授要开讲了。”
\par “事实上,要心平气和并不简单。”
\par 我进一步解释说,“那是整个事情的灵魂,保养的良好与否就取决于你是否有这种态度。我们所谓机器运转是否正常正是心平气和的具体表现。最后考验的往往是你的定力。如果你把持不住,在你维修机器的时候,很可能就会把你个人的问题导入机器之中。”
\par 他们只是看着我,思考我的看法。
\par “这是一种新观念,”我说,“但是它的来源却很传统。客观的物质,比如说,自行车或是烤肉架,本身无所谓对错,分子仍然是分子。机器没有感受力,除了人施加给它们的东西。要想测验机器的好坏,就看它给你的感受,没有别的测验方法。如果机器发出的声音很顺畅,就表示没有问题。如果声音不对,那就表示有问题,除非你或是机器任一方有改变。所以检验机器也是对你的一种检验。没有别的检验。”
\par 狄威斯问我:“如果机器出了问题,而我觉得很平静,又该怎么办呢?”
\par 大家笑了起来。
\par 我回答,“这是自相矛盾的事。如果你真的不关心,你就不会发现它出问题了。所以发现它出问题就表示你关心它。”
\par 接着我又说:“比较常出现的情形是,即使它已经恢复正常了,你仍然忐忑不安。我想这才是实际的状况。在这种情况之下,如果你仍然担心,就表示还有问题,因为你没有彻底检查过它。
\par 在工厂里,任何一台机器没有彻底被检查过,就不能上线运转,即使它的运转情形良好。你对烤肉架的忧虑也是一样。
\par 你还没有完成让你心安的种种检查步骤,因为你总觉得这些说明书太复杂了,你很可能无法完全了解。”
\par 狄威斯问:“要怎样做才会心安呢?”
\par “那要做比我现在所说的更多的研究,这件事有很深奥的道理。每一份说明书说明的对象都是特定的机种。但是我所说的方法并没有这么狭窄。说明书真正让人气愤的是,它们限定你只使用一种方法组合,也就是工厂设定的方法,这种前提抹煞了所有的创意。其实组合烤肉架有千百种方式,但是他们不让你了解整个状况,因而只要你出一点错,就拼凑不成了。于是你很快就失去了兴趣。不只这样,他们告诉你的方法,可能不是最好的。”
\par John说:“但是它们是从工厂来的。”
\par 我说:“我也来自工厂,我知道这些说明书是怎样写成的。你只要带着一个录音机走到生产线上,主管会找一个他最不需要的人陪你,而他正好逮到打发时间的最好方法。于是不管这个人说什么,就成了这份说明书上的指示。下一个人很可能告诉你完全不同的内容,或是更好的方法,但是他太忙了。”
\par 他们都很吃惊。
\par “我早该知道的。”狄威斯说。
\par 我说:“情况就是这样,没有作者抵制这种做法,因为科技原先就假定只有一种正确的方法。然而情况完全不是这样。所以一旦你有这样的假设,当然说明书只限定说明烤肉架。一旦你可以选择千百种组合的方法,就要同时考虑到你和机器之间的关系,还有你与外界的关系。这样一来,整个工作就需要把你的心灵状态和机器结合在一起,这就是为什么你需要内心的平静。”
\par 我接着说:“其实这种想法并不奇怪,有时候你只要把新手或蹩脚的人和高手做比较,你就会发现其中的差异。
\par 老手根本就不会照着指示去做,他边做边取舍,因此必须全神贯注于手上的工作,即使他没有刻意这样做,他的动作和机器之间也自然地有一种和谐的感觉。他不需要遵照任何书面的指示,因为手中机器给他的感觉决定他的思路和动作,同时也会影响他手中的工作。所以机器和他的思想同时不断地改变,一直到把事情做好了,他的内心才真正地安宁下来。”
\par 狄威斯说:“听起来好像艺术一样。”
\par 我说:“的确就是艺术,把艺术和科学分离是完全违反自然的,两者分离太久了。你必须像考古学家一样,不断追溯到两者最初分离之处。其实组合烤肉架是雕刻艺术早已失传的一支,多少世纪以来,由于知识错误的分野,造成两者的分隔,因而如今一旦把它们连起来,就会显得有些荒谬。”
\par 他们不知道我是不是在开玩笑。
\par 狄威斯问我:“你的意思是,当我在组合烤肉架的时候,实际上我是在雕刻它?”
\par “没错,就是这样。”
\par 他想了一想,脸上的笑意愈来愈深。
\par “我真希望能明白这个道理。”他说完就笑了起来。
\par Chris说他不明白我在说什么。
\par 杰克说:“Chris,没关系,我们也不明白。”他这么说引来更多的笑声。
\par 雕刻家朋友说:“我想,我还是研究一般的雕刻就可以了。”
\par 狄威斯说:“我想我只要研究绘画。”
\par John说:“我想我只要研究打鼓。”
\par 说完大家又笑了起来。
\par 大伙对我冗长的演说似乎不以为意。一旦你脑海中只想到Chautauqua的事,就很难让自己不在不知情的人面前大放厥词一番。
\par 于是大家各自散开聊天,剩下的时间,我一直在和杰克、维拉谈英语系的发展情形。
\par 聚会结束之后,John夫妇和Chris都回房睡觉。狄威斯却和我很认真地讨论刚才我发表的见解。“你刚刚提到的有关烤肉架说明书的事很有意思。”
\par 珍妮也很认真地说:“你似乎已经思考了好长的一段时间。”
\par “我已经在这些事物背后的观念上花了整整二十年。”我说。
\par 外面的风吹得很强劲,炉火不断地爆出火星,冲上烟囱,愈烧愈旺。
\par 我几乎在告诉自己:“你如果一直向前看,或者只看到目前的状况,对你并没有任何意义。一旦你回顾以往,就会看到一种模式隐隐出现。如果你由这个模式出发,那么很可能会迸发出一些东西。刚才有关科学和艺术的见解,只是我生活的一部分,它代表了一种超越,我想那是别的许多人也想要超越的。”
\par “是什么呢?”
\par “并不只是艺术和科学,而是想超越理性和感情的对立。科学的问题在于它并没有和人的心灵连在一起,所以在盲目之中表露出它丑陋的一面,因而必然引起人们的厌恶。然而过去人们并没有注意到这一点,因为大家最关心的是衣食住行的问题,而科学正好能满足人们这方面的需要。
\par “但是现在有更多人相信,也注意到科学所产生的丑陋现象,因而怀疑我们是否需要牺牲灵性和美感上的需要,以满足物质方面的欲望。这点最近已经引起全国人民的注意,大家开始反对工业所带来的污染,反对一切科技化等等。”
\par 狄威斯和珍妮早已了解了这一点,所以不需要我做任何解释,于是我又继续说:“然而我又相信,这种情形主要是因为现存的思想无法解决目前的问题,因为理性的方法不可能解决理性自己所产生的问题。有些人解决的方法较偏重个人的方式,就是直接抛弃一丝不苟的理性,然后跟着感觉走,就像John和Sylvia一样,而且有数百万的人都和他们一样。但是这似乎也不是解决问题的方法,所以我想说的是,解决的方法不是抛弃或否定理性,而是拓展理性的内涵,使它能够找到解决的方法。”
\par 珍妮说:“我想我不明白你的意思。”
\par “这真是一场独力作战的苦仗。就好像Newton当年尝试解答‘瞬时变化速率’
\par 的问题时所面临的困扰。在他那个时代里,无人能想像得出物体如何在瞬时间发生变化。虽然,那时的人们已颇能处理在数学上接近零重量的物质,例如对于地点或时间上的变化,大家都认为那是合理的,实际上并没有什么差别。所以,Newton曾说道:‘我们首先假定物体会做瞬时变化,然后看看我们如何确定它在各种应用之中的体现。’微积分即是根据这个假定所发展出来的数学原理,至今仍为工程师所广泛运用。Newton据此发明了一种新的理性思考模式。他将理性扩展至物体极细微的变化上,而我认为我们也应该将理性拓展至科技丛生的丑陋面上去。
\par 困难在于,一定要从根本做起,而不是光在枝枝节节的地方上扩展理性,从而徒劳无功。
\par “我们活在一个价值混淆的时代,我想造成这种现象的最主要原因就是,过去的古老观念已经无法应付新的状况。曾经有人这么说,真正的学习来自于四处游荡。你必须先停止拓展原先的知识,四处游荡一阵子,直到碰到一些事,能够让你拓展原先知识的根基,才会继续前进。每一个人都很熟悉这种经验,我想一旦整个文化的根基需要拓展的时候,就会出现相同的状况。
\par “当你回顾过去三千年来的历史,就会后知后觉地以为发现了事情的因果关系,但是一旦你去查考当代的史料,就会发现这些原因在当时往往并不明显。每到拓展根基的时候,世事就会变得像现在一样混淆不清,而且目标不明。
\par 哥伦布发现新大陆之后,在社会上引起了价值观的混乱,因而出现了文艺复兴。
\par 他的发现极大地震撼了当时人类的思想。从各种纪录都可以发现这种价值混淆的现象。圣经的新旧约并不认为地球是圆的,而且也没有预言到这一点。但是人们又无法否定这个事实,他们所能采取的行动,就是抛弃中古世纪的价值观,接受理性的新世界。
\par “于是哥伦布成了学校教科书的主角,我们很难想像他原来也是一般人。
\par 如果你暂时停止思考他发现新大陆所带来的影响,进入他当时的世界,那么你可能会发现我们目前登陆月球和他当时的壮举相比,简直就是小巫见大巫。登陆月球并没有在思想的基础上产生变革,我们知道现有的思考模式就足以解决这个问题,它只是哥伦布发现新大陆的一个分支。要想出现一个真正前无古人的像哥伦布一样的发现,必须有全新的方向。”
\par “比如说?”
\par “比如说,进入超越理性的领域。
\par 我认为目前的理性就好像中古世纪认为地球是平的一样,如果你走到尽头,很可能就会掉到深渊里变成疯子。而人们对这一点非常恐惧。我认为这种对疯狂的恐惧就好像中古世纪的人恐惧掉到世界尽头之外,或者就像恐惧异教徒一样,两者之间非常相似。
\par “而目前的状况是,每一年我们都发现,传统的理性愈来愈无法处理现有的经验,因而造成目前世界上价值十分混乱的现象,结果愈来愈多的人开始研究超越理性的世界。比如说:占星术、神秘主义、吸食毒品等等,因为他们觉得传统而又严谨的理性无法处理现实之中的经验。”
\par “我不太了解你所谓严谨的理性。”
\par “就是分析、辩证法的理性。在大学里,理性被视为了解世界的全部基础,你从来不曾真正地了解它。但一谈到抽象艺术,理性就完全派不上用场了,这就是我所谓根本的经验。有一些人很可能会诅咒抽象艺术,因为它毫无道理可言。但是错不在于艺术本身,而是所谓的道理——它来自于严谨的理性,无法掌握艺术的现象。大家一直想要从理性当中,找到能够涵盖抽象艺术的理论,但是答案并不在理性的枝节当中,而在根本。”
\par 这时候从山上吹下来的寒风愈来愈强劲,我说:“严谨的理性来自于古代的希腊人。他们听到风的声音就能够预测未来,听起来有些不可思议,但是为什么支持理性的人,却去做超乎理性的事呢?”
\par 狄威斯眯着眼问我:“他们怎么能听到风的声音就预测未来呢?”
\par “我也不知道,很可能就像画家盯着画布看就能预测自己的未来一样。我们整个的知识体系就源自于他们研究的结果。然而我们尚未了解产生这些结果的方法是什么。”
\par 我想了一下,然后说:“我以前在这儿的时候,曾经谈过理性教会吗?”
\par “你谈过很多。”
\par “我提过一个叫做Phaedrus的人吗?”
\par “没有。”
\par “他是谁?”珍妮问我。
\par “他是古希腊的…… 一位修辞学者……主要的工作是研究写作,他生活在理性被发现的时代。”
\par “你从来没有提到过。”
\par “那是我后来才想到的。这些古希腊的修辞学者是西方世界当中的第一批学者。Plato在他所有的作品当中猛烈地诋毁他们。对他们的了解,我们几乎完全来自于Plato的作品。因而在历史上,他们并没有受到肯定。而我提到的理性教会,就是建在他们的坟墓之上,如果你掘得够深,就很可能会碰到他们的灵魂。”
\par 我看了看手表,已经两点多了,我说:“这是一个很长的故事。”
\par 珍妮说:“你应该把这些都写下来。”
\par 我同意地点点头:“我正在构思一连串的演说论集——一种Chautauqua。在我们结束旅程之后,我打算把它写出来……这就是为什么在这些方面我研究得十分彻底,整个问题非常庞大而又艰难,就好像想要徒步行过这些山脉。
\par “但问题是,论文总像上帝在谈永恒一般严肃。然而情况不是这样的,人们应该了解,这只不过是一个人在特定的时空和环境背景下发表他的看法,情况仅止于此,但是你无法在论文当中使人明白这一点。”
\par 珍妮说:“反正你应该写下来,但是不要想做得很完美。”
\par 我说:“我想是的。”
\par 狄威斯问我,“ 这和你研究良质\footnote{Quality,原指品质、特性、高级、素养等,但作者赋予其新的涵义,此涵义非三言两语所能涵盖,必须由读者根据上下文而加以揣摩,因为任何定义终将破坏其不可说的特性}有关吗?”
\par “这是它直接的结果。”我说。
\par 我想起一些事情,然后看了看狄威斯。“你不是劝我放弃它吗?”
\par “我是说像你这种研究没有人成功过。”
\par “你认为可能成功吗?”
\par “我不知道,谁晓得呢?”由他的表情我知道他真的很关心这件事,“不过现在有不少人比较注意你的言论,特别是孩子们。他们真的在听……不只是听到而已,而是心悦诚服地接受,这完全是两码子事。”
\par 从积雪的山顶吹下来阵阵狂风,风声已经在整个房子里回荡了好久,声音愈来愈大,仿佛要把整座房子给扫倒,连我们一起吹向远方,恢复这座峡谷的本来面貌。但是房子仍然直挺挺地站立着,于是风逐渐退却,仿佛被打败了一般。然后它又回来了,在远处先吹起一阵小风,然后到了我们这里,突然变成一道狂风。
\par 我说:“我一直在听风的声音。”
\par 接着我又说:“我想John夫妇回去之后,Chris和我应该爬到山顶上。我想是让他好好看看那个地方的时候了。”
\par 狄威斯说:“你可以从这里开始往上爬,然后由峡谷的背面再爬上去。这条路不太好走,你可能没法儿骑到七十五英里以上。”
\par “那么我们就从这儿开始爬起。”我这样说。
\par 上楼之后,我很高兴地看到床上铺着厚厚的棉被,现在寒气逼人,很需要这样的棉被。我赶快脱掉衣服,钻进棉被里,在温暖的被窝中,我又想了好一阵子山顶的雪和风,还有哥伦布。
\subsection*{15}
\par 接下来的两天,John、Sylvia、Chris和我四处闲逛着,偶尔聊聊天。后来我们骑马去了一座古老的矿城,然后再回来。接下来John和Sylvia要告别了,这是我们最后一次一起由峡谷骑到波斯曼了。Sylvia在前面已经回头三次,很显然是要看看我们是否无恙。过去两天来,她都不多话。昨天我看见她的眼神显得很忧虑,又有些害怕,她太担心Chris和我了。
\par 在波斯曼的酒吧里,我们喝完最后一杯啤酒,然后我和John讨论骑回去的路线,又说了一些例行的话。比如说,这一路上相处在一起的时间有多么好,我们很快就会再见。突然我觉得这样说让人很伤感。因为反倒像普通的朋友一样。
\par 到街上的时候,Sylvia转过身来,面对着我和Chris停下来,说:“你们不会有事的,不要担心。”
\par 我说:“当然。”
\par 她的眼里又再度出现恐惧的神色。
\par John已经发动了摩托车等她上路,我说:“我相信你说的。”
\par 她转过身骑上去,John看着路上的车流,准备找机会骑进去,我说:“再见了!”
\par 她又看了我们一眼,这次脸上没有特殊的表情,John找到机会就骑进车流里面去了,然后Sylvia朝着我们挥挥手,就好像电影中的情节一样。Chris和我也向她挥手再见。他们的摩托车很快就消失在州际公路上拥挤的车流里,然后我又看了好一阵子。
\par 我看了看Chris,然后Chris也看了看我,他没有说什么。
\par 我们坐在公园里的博爱座上,接着吃早餐,然后就到修理店里去换轮胎和车链,由于车链必须要额外加工,所以我们等待的时候就出去逛。在大街上,我们在教堂前的草地上坐下来,Chris躺在草地上,用夹克盖着眼睛。
\par 我问他:“你累了吗?”
\par “没有。”
\par 从这儿到北边山脉的山脚下,天气非常热,有一只翅膀透明的小甲虫,因为受了热气的影响,停在Chris脚旁的一根草上,我看着它挥动翅膀,愈飞愈慢。我也躺下来想小睡一会儿,但是又睡不着,反而有点不安,于是就站了起来。
\par 我说:“我们起来走一走。”
\par “去哪里呢?”
\par “到学校去。”
\par “好吧。”
\par 我们走在树阴底下,一路上的人行道非常干净,两旁的房子也很清爽。走在街道上又让我想起过去的许多事。Phaedrus也常在这些街道上行走。在逍遥自在的气氛中准备他的讲稿,把这些街道当作他的学校。
\par Phaedrus准备到这儿来教的是修辞学和写作。而他教过的是一些高级的技术性写作,以及大一英语。
\par 我问Chris:“你记得这条街吗?”
\par 他四下望了望,然后说:“我以前常常坐在车子里出来找你,”他指着对街。
\par “我记得那个房子的屋顶很有趣……谁要是先发现你,就可以得半分钱,然后我们就会停下来让你坐在后座,你都不和我们讲话。”
\par “那个时候我正在沉思。”
\par “妈妈也这么说。”
\par Phaedrus当时确实思考得很辛苦,教书的压力已经够沉重了,然而对他更不利的是,以他精确的分析能力,他知道他所要教的题材,毫无疑问的是整个理性教会最无法分析、最不精确的一部分。
\par 这就是为什么他会思考得这样辛苦。对一个受过方法和实验训练的人,修辞学简直就是无可救药,其中毫无逻辑可言。
\par 在大一修辞学的课堂上,只需要读一小段论文或是短篇故事,然后讨论作者为了产生某种效果所运用的技巧,然后要学生也运用同样的技巧模仿着写论文和短篇故事,看看他们是否做得到。
\par Phaedrus不断试着这样做,但是还是无法让学生真正学到什么。写出来的东西和原作往往相去甚远,甚至他们的写作能力变得更糟,因为在这些规则之中,总是充满了各种例外、矛盾、混淆不清以及限定好的状况,以至于使他希望一开始就不曾谈过这些规则。
\par 有一个学生总是喜欢问,在某一种特定的情况下如何运用这些规则。Phaedrus这时候就必须做选择,是编造一套如何运用的解释,还是坦白地告诉对方他真正的想法。而他真正的想法是,这些规则是作品写好之后才找出来的,作者不是依照这些原则来写作。他最后终于承认,这些学生想要模仿的作家,原先根本就没有所谓的原则,只是把他们认为对的东西写下来,然后再回头看看是否有问题,如果修辞不妥,可以再修正。
\par 有一些学生的作品由于事先经过周密的思考,注意是否符合修辞学,因此读起来很乏味,仿佛其中的确有点蜜汁,但却无法汹涌而出。但是你又如何教学生那些无法事先周密策划的东西呢?这似乎是不可能达到的要求。于是他就拿起教科书随兴评论,希望学生能够由此得到一些东西,但是情形并不令人满意。
\par 它就在前面了。这个时候,我的胃又开始紧张起来。
\par “你记得那栋建筑吗?”
\par “那是你过去教书的地方……为什么我们要来这里呢?”
\par “我也不知道,我只是想看看它。”
\par 周围似乎并没有多少人,显然现在正在放暑假,不会有多少人。建筑物的屋顶呈人字形,墙壁是深褐色的砖墙,这是一座很优美的建筑,有仅仅属于这里的风格。通往大门的阶梯是石头铺成的。不知道有多少人走过,每一个石阶都凹进去一个浅浅的窝。
\par “我们为什么要进去呢?”
\par “嘘,现在不要说话。”
\par 我打开沉重的大门走进去,里面有很多老旧的楼梯,走在脚下还会嘎吱作响,而且透出几百年来打扫和上蜡的气味。走到一半,我停下来听一听,没有任何声音。
\par Chris小声地问:“我们为什么要来这里?”
\par 我只是摇摇头,我听到门外好像有车子经过的声音。
\par Chris又低声说:“我不喜欢这里,这里好恐怖。”
\par 我说:“那么你就到外面去吧!”
\par “你也跟我一起来。”
\par “等一下。”
\par “不要,就现在。”他看着我,看我没有要跟他去的迹象。他有点害怕,我几乎想要改变心意。但是突然之间,他转身跑下楼,然后跑出去,我来不及追上他。
\par 外面传来沉重的关门声,现在我在这里单独一个人,我仔细听,有一些声音……是谁呢……是他吗……我听了好一阵子……
\par 当我走到门廊的时候,地板嘎吱作响,我想Phaedrus真的来了。在这个地方他才是真实的人,而我是鬼魂。在某一间教室的门把上,我看见他的手停留了一阵子,然后慢慢地转开门把,推门进去。
\par 教室里面还是和以前Phaedrus在的时候一样。现在他来了,他看到所有我看到的东西,这一切激起了我鲜活的回忆。
\par 墨绿色的黑板两边都已经剥落了,需要整修,情形就像以前一样。黑板槽里的粉笔永远都不是完整的一支,只是一小段一小段的。在黑板的另外一边是一排窗户,Phaedrus可以透过窗户看到户外的山色。在学生写作的时候,他就沉思其中。他坐在暖气旁边,手上拿着一支粉笔,两眼望着窗外的山景,学生常常打断他:“我们必须……”然后他只好转过身来回答学生的问题,在这里学生曾经很安静地听他讲课,而他也会倾其所有地教授学生。这里不是一间教室,而是一千间教室。每天都有不同的风、雨、雪,还有山上的云,班级不同,学生不同,教室就有不同的气氛,不曾有相同的两个钟头,所以对他来说,接下来会发生什么事情,总是一个谜。
\par 我对时间的感觉几乎丧失了,然而我听到大厅里的脚步声愈来愈大,然后我听到它停在这间教室的门口,门把慢慢地转动,门打开来了,有一名女子向里面观望。
\par 从表情上看仿佛她是在这儿逮到了什么人。看上去她已经超过了二十五岁,长得并不很美。她说:“我想……”她的脸上有点不解的表情。
\par 她走进房间向我走来,希望看得更仔细点。于是她脸上急切的表情消失了,慢慢地转成怀疑的眼光,然后她大吃一惊。
\par “我的天,是你吗?”她说。
\par 我完全不记得她。
\par 她说出我的名字,然后我就点点头说:“没错,是我。”
\par “你回来了!”
\par 我摇摇头说:“只是回来几分钟而已。”
\par 她一直看到连自己都觉得不好意思才问我:“我能坐下来一会儿吗?”她这样羞怯地问话,表示她很可能过去是Phaedrus的学生。
\par 她在前面的椅子上坐下来,没有戴戒指的手有一点儿颤抖。我真的是个鬼魂啰!
\par 这个时候,她反而变得不好意思起来,“你想要待多久?不是,我问的是你……”
\par 我接着说:“我准备在狄威斯家住几天,然后继续向西走,在城里还有一点时间,所以想过来看看。”
\par 她说:“哦!我很高兴你回来,学校变了……我们都变了……自从你离开之后,变了好多……”接下来又是一阵子令人不安的沉默。
\par “我们听说你住院了……”
\par 我说:“没错。”
\par 接下来是一段令人更不安的沉默。
\par 她没有继续追问下去,这表示她很可能知道原因,她又犹豫了一阵子,想要找话讲,然而这样子令人很不好受。
\par “你现在在哪儿教书呢?”最后她又问道。
\par “我不再教书了,”我说,“我已经不再教了。”
\par 她不相信地望着我:“你不教了?”
\par 她皱了皱眉,又看了看我,仿佛要确定她说话的对象的确是那个人,“你不可以这样。”
\par “可以的。”
\par 她摇摇头,十分不解地说:“你不是他。”
\par “是他。”
\par “为什么?”
\par “对我来说,这些都已经结束了。
\par 我现在在做别的事情。”
\par 我一直在想,她究竟是谁?而她的表情看起来十分羞涩,“但是那……”她想继续说下去。“你已经完全……”但是这句话仍然未说完。
\par 她想要说的是“疯了”,但是她两次都不让自己脱口说出。她了解了一些事,咬了咬嘴唇,然后有些伤感的样子。我一直想说些什么,但是不知道从哪里说起。我真想告诉她我不认识她。但是她站起来说:“我应该走了。”我想她一定知道我不认识她。
\par 她走到门口,飞速地用僵硬的口吻跟我道再见。等到门一关起来,她走得更快了,几乎是小跑着走出了大厅。
\par 外面的大门关上了,教室里一片沉寂。除了她走后所留下的精神涡流,教室里只剩下一股悲伤的气氛。而原先我所要来看的东西已经消失了。
\par 我想这样也好,我很高兴回到这里来,但是我想我不会再想要回到这里了。
\par 我宁可去修理摩托车,还有人在那里等我。
\par 出门的时候,我很勉强地打开门。
\par 突然,我在墙上看到一样东西,让我不禁打了一个寒颤。
\par 那是一幅画,我原先忘了有这幅画,但是我现在知道,是Phaedrus买的挂在这里。突然间我发现它不是原画,而是他从纽约邮购的一幅复制品。狄威斯看到它的时候皱皱眉,因为这只是一幅印刷品,并不是原作,当时他并不明白这种感觉。而这幅题名为《少数人的教会》的印刷品,内容和名字似乎毫不相关,它用半抽象式的线条画哥特式的教堂还有草原。色彩、层次似乎都能反映出当时他所谓理性教会的心态,这就是他挂在这儿的原因。现在这一切都回来了。
\par 这里是他的办公室,这是个发现,这就是我在寻找的房间。
\par 由于刚才那幅画的震动,我一走进房间,过去的回忆突然间全都涌上心头。
\par 照到画上的光线是透过旁边墙壁上狭长的窗户射进来的,当时Phaedrus正从这个窗子往外看,越过河谷,看着麦迪逊山脉,也看着暴风雨袭来,看着眼前的这个山谷,就在这个窗户旁边……整件事都回来了,当时就是在这里发狂的,就是这个地点!
\par 而那个门通向莎拉的办公室,莎拉!
\par 我想起来了,她手上拿着浇花的水壶,快步地从走廊走到她的办公室,然后说,“我希望你把所谓的良质教给学生。”这位女士即将退休,正要去浇她的花草,就是这一刻引发了后来的一切。它就是晶种。
\par 晶种。我又回想起许多更清楚的画面。实验室、生物化学。当时Phaedrus正在研究一种极度饱和的溶液,这时有一些类似的事情发生了。
\par 极度饱和的溶液就是溶质超过了它的饱和点,在这种情况之下,不会有任何物质再溶解,只要溶液的温度增加,溶点就会升高。如果你在高温下溶解物质,然后冷却溶液,这些物质往往不会结晶,因为分子不知道如何开始,它们需要一些物质去引动结晶的过程,而晶种或是一小粒灰尘,或者是在烧杯的外面轻敲和刮动,都可能促使结晶开始。
\par Phaedrus想走到水龙头那儿去冷却溶液,但是永远都没有走过去。在他走动的时候,眼前的溶液突然开始结晶。然后刹那间,结晶充满了整个容器,他清楚地看见,结晶之前还是清澈的液体,而现在却是一团固体。他可以把容器倒过来,什么都不会流出来。
\par 然而就在那一句“我希望你把所谓的良质教给学生”之后那几个月,你几乎可以看得见它成长的速度,它引发出一套庞大、精密而且复杂的思想体系,仿佛是用魔术变出来的。
\par 我不知道她说这句话的时候,Phaedrus是怎样回答的,很可能是什么都没有说。她每天要从他的背后走到自己的办公室许多次,有的时候她会停下来说一两句很抱歉打扰他的话;有的时候又会提到一些片断的消息。而他已经习惯了这种方式。我知道她又来过一次,问道:“这个学期你真的要教良质?”他点点头,然后坐在自己的椅子上看了她一眼,然后说:“当然。”于是她又走开了。这个时候他正在准备讲稿,心情正处于极度的沮丧之中。
\par Phaedrus沮丧的原因是,那本教科书是所有修辞学的教材里面理性最重的一本。他曾经去找这本书的作者,他们是系里的同事,他就书上的问题向他们请教和讨论,也耐心地听他们的回答。然而他对他们的解说并不满意。
\par 这本教科书的前提是,如果要在大学里面教修辞学,就必须把它当作是理性的一支,而不是神秘的艺术。因此要了解修辞学,它强调的是要掌握沟通的理性基础,必须介绍基本的逻辑学,以及基本的刺激和反应的理论。接着就要谈一些如何撰写一篇论文的方法。
\par 第一年开始教的时候,Phaedrus对这种结构尚算满意,然而他总觉得有哪里不对劲,毛病并不在于把理性运用在修辞上,问题在于他梦中的鬼魂——理性本身。他发现理性和困扰他许多年的问题如出一辙,然而对于这个问题他并没有解决的方法。他只是觉得,没有任何一位作家是依照这种严谨、有条理、客观而又讲究方法的步骤在写作。而这却是理性所要求的。几天之后,莎拉从后面快步地走过时又停下来说:“我很高兴你这学期要教良质,这个时代很少有人会做这样的事了。”
\par Phaedrus说:“我就是这样的人,我一定要让学生彻底了解它的意义。”
\par “很好。”她说,然后又走开了。
\par Phaedrus又回到自己的笔记上,但是不一会儿他就想起莎拉刚才奇怪的言论,她究竟在说什么?良质。当然他教的是良质。谁不是呢?于是他又继续写他自己的笔记。
\par 另外一件让Phaedrus沮丧的事是僵化的文法。这一部分早该作废,但是仍然存在。你必须要有正确的拼写、正确的标点以及正确的用词。有数以百计的各种规则为那些喜欢零零碎碎的人而设立。没有人在写作的时候还会记得那些琐碎的事,这些就好像餐桌上的繁文缛节一样,不是从真正的礼貌和人性出发,而是为了满足自己像绅士和淑女一样表现的欲望。绅士淑女般良好的餐桌礼仪以及说话、写作的合乎文法,被认为是挤进上流社会的晋身阶。
\par 然而在Montana这一套根本不管用。
\par 系上对于这方面的要求很低,他只得和其他教授一样,只要求学生把修辞学当作是一门必修课。
\par 不一会儿Phaedrus又想起所谓的良质,对这个问题他有一点儿坐立难安,甚至挑起他的怒气。他想了一会儿,然后再想下去,接着望向窗外,又回头再想一阵子。良质?
\par 四个钟头之后,Phaedrus仍然呆呆地坐在那儿,而窗外早已暗下来了。这时电话铃响了,那是他太太打来的,想知道发生了什么事情。他告诉她很快就会回去。然而不一会儿他又忘记了,连其他的一切都忘了。一直到凌晨三点钟他才很疲倦地承认他实在不知道良质是什么意思,然后拿起公文包回家去了。
\par 大部分的人在这个时候已经放弃研究什么是良质或者让问题悬在那儿,因为他们实在想不出来,况且还有别的事要做。但是Phaedrus对自己无法教授学生自己所信仰的东西感到十分气馁。他实在不知道自己该怎么做。第二天一早起来的时候,他仍然没有答案,由于只睡了三个钟头,所以十分疲惫。他知道自己今天无法上课,而且笔记还没有写完,所以他在黑板上写道:“请写出三百五十字的短文,回答这个问题:在思想和言论上良质是何意义?”于是他坐在暖气旁边,在学生奋笔疾书的时候,他也在想这个问题。
\par 这堂课结束的时候,似乎没有人写得出来,所以Phaedrus就让学生带回去写。
\par 下一堂课是在两天之后,他还有时间再进一步想这个问题。在这段期间他碰到课堂上的学生,向他们点头的时候,看到他们脸上有愤怒和害怕的表情。他想他们一定和他碰到了一样的问题。
\par 良质……你知道它是什么,然而你又不知道它是什么。这是自相矛盾的。如果有一些事情比其他的要好,那就是说它们的等级比较高。但是一旦你想解说良质,而不提拥有这种特质的东西,那么就完全无法解释清楚了。因为所说的根本就没有内容,但是如果你无法说出良质究竟是什么,你又如何知道它是什么呢?或者你怎样才知道它存在呢?如果不知道它究竟是什么,那么从实用的角度来说,它根本就不存在,而实际上它的确存在。那么等级的根基又在哪里呢?为什么有些人愿意花更多钱去买这些东西,而把另外一些东西丢到垃圾桶里呢?很明显地,有些东西的确比其他的东西要好,但是什么又是比较好呢……
\par 你的思想一直在打转,找不到出路。究竟良质是什么呢?它是什么呢?
\section*{第三部分}
\subsection*{16}
\par Chris和我睡了一晚好觉,第二天一早,小心地把行李装上了车。现在我们已经上路一个钟头了,准备去爬山。
\par 谷底种的大部分是松树,还有一些白杨,以及阔叶灌木。就在我们两旁,高耸的岩壁陡直而上,偶尔眼前会出现一片阳光和草地,绿草沿着峡谷的溪岸生长,但是很快又被松树的阴影遮蔽。一路上地面都铺着一层松软的松针,四周一片宁静。
\par 在许多禅学的书以及世界各大宗教的记载当中,我们都会发现这样的山岭、朝山的旅人,以及发生在他们身上的种种故事。而实体的山往往能象征人们灵性长进的路。就好像那些在我们身后的山谷里的人们,大部分人望着灵性的高峰,但是一生从来不曾攀上过,只是听听别人的经验就已经很满足,而自己不愿意花费任何心血。有一些人则是靠着有经验的向导,他们知道最安全的路,因而能够很顺利地到达他们的目的地。
\par 但是还有另外一批人,不但没有经验,而且不太相信别人的经验,想要走出自己的路。其中很少有人能成功,但是总有一些靠着自己的意志、运气还有上天的恩典而做到了。
\par 那些成功的人要比别人更明白,其实登山并没有惟一或是固定的路线,有多少这样的人物就有多少条路。
\par 现在我想谈谈Phaedrus探讨良质的意义,在他来说,这就好比开拓出一条到达灵性高峰的路径。我会尽量解说清楚,它有两个重点。
\par 第一点,他不想建立一种僵化而系统的定义,所以良质的这一面是快乐的、充满成就的和富有创意的。他在我们身后山谷里的学校教书的时候,大部分的时光都是如此。
\par 第二点则是因为一般人批评他对于自己所探讨的内容缺乏定义,于是他提出对于良质的系统而刻板的定义,从而建立起庞大的思想体系。他绞尽脑汁地建立起有关生存的系统解释之后,让我们对它的了解远远超过了从前。
\par 如果这真是一条通往山顶的新路,当然是需要的。因为三百年来旧的路都是一些捷径,而且因为自然的侵蚀改变了路径,而科学的研究也改变了山的形状。早期登山者开辟出来的路似乎能让所有的人都上山,但是今天在西方世界,这些路面对着社会不断的变动,都因为教条的僵化而封闭了。如果你怀疑耶稣或摩西所传讲的讯息,必然会招致大部分人的厌弃。然而如果耶稣或是摩西生在今日,不为人认出他们的身份,仍然传讲当初的讯息,他们的想法一定会受到质疑。这并不是因为耶稣或是摩西所说的不是真的,或者现代的社会出了问题,只是他们表达的方式已经和这个社会脱节,因而一般人无从理解。在这个太空时代,天堂在上的意义已经逐渐消失。哪里才是上方呢?然而,虽然因为语言上的僵化,这些老旧的路即将丧失它们的日常意义,甚至算是封闭了,但是这并不表示山已经消失了,它仍然在那儿,只要人有意识它就存在。
\par 然而Phaedrus提出的第二点形而上学的看法对他是一个打击。在他接受电击之前,他已经丧失了所有的一切:金钱、财产、孩子。法院甚至下令剥夺他的公民权。他所剩下的只是对良质的梦想,一张通往山顶的地图。为了这张地图,他牺牲了一切。然而他被电击之后,连这个也丧失了。
\par 我想我永远不可能知道当时他脑海里在想什么,而且也不会有其他人知道。
\par 他留下来的只是一些断简残篇,还有四散的笔记。虽然可以拼凑成篇,但是仍然有许多无从解释的地方。
\par 第一次发现这些资料时,我觉得自己好像是雅典郊外的农夫,很偶然地挖出了许多石头,上面有些奇怪的记号。
\par 我知道过去有一整套完整的计划出现过,但是它远远超乎我的了解。一开始,我刻意避开这些资料,不想去深入研究。
\par 因为我知道,这些石头会引起某些麻烦,我应该避开。但是那个时候,我已经知道它们是一套庞大思想体系中的一小部分,而我也有无法言明的好奇。后来等我更有信心能对他的影响产生免疫力,我对这些资料就更感兴趣。于是我开始把许多片断按着它们显现给我的次序记下来,而不是按一定的系统。其中有许多是朋友提供的,已经有几千条了。但是只有一小部分适合这一次的Chautauqua。
\par 所以这一次的Chautauqua主要的根基就在于这些资料。
\par 要想完全了解他的思想要走一条漫长的路。尤其是我得从这些断简残篇中归纳出当初的整个结构。我一定会犯错,而且原先的记载也不连贯,为此我希望得到读者的原谅。在许多情况之下,这些片断都模糊不清,可能会产生许多不同的结论。如果有问题出现,就表示是我重建的结构出了问题,而不是他原先的思想。以后我可能会建立更好的结构。
\par 我听到一阵翅膀拍打的声音,有一只山竹鸡消失在树林里。
\par Chris说:“你看到了吗?”
\par “噢!”我回他,“看到了。”
\par “那是什么?”
\par “山竹鸡。”
\par “你怎么知道?”
\par “它们飞起来的时候,就是那样拍打翅膀的,”我说,“我不太肯定但似乎没错,而且它们很接近地面。”
\par Chris说了一声“哦”。然后我们继续爬山,阳光透过松树林照下来,就有如教堂里那圣洁的光芒。
\par 今天我想先提出Phaedrus探索的良质的第一部分,属于非形而上学的一面。
\par 这一面会让人颇为愉快。开始旅行总是令人愉快的,即使你知道结束时的情形不见得会这样。我想借着他上课的笔记指出,在他教修辞学的时候,良质对他来说是一个活生生的观念。
\par 他这个人创意很多,所以他对脑袋中一无所有的学生十分头痛。刚开始他以为是学生懒惰,后来才发现情形并不是这样。他们就是怎么样也想不出可以表达的东西。
\par 其中有一个女孩子,脸上戴了一副很厚的眼镜,想要写一篇有关美国的五百字短文,他一听到这样的题材就知道会有问题,所以就建议她把题材缩小,只谈波斯曼。
\par 要交稿的时候,她交不出来,于是十分难过,她已经试过一切方法,就是想不出要写些什么。
\par Phaedrus和她以前的老师谈起这事,他们的说法也跟他的印象一样。她很认真,也很努力,受过良好的训练,但却是个非常乏味的人。从她身上找不出一丝创意。她厚厚的镜片底下,无神的双眼好像做苦工的人一样。她没有骗他,她真的想不出任何东西来,因而对于她自己的无能十分难过。
\par 这一点令他大吃一惊,现在换成他说不出话来。两个人沉默了一阵子,他突然提出一个奇怪的想法,那么就写波斯曼的大街吧!
\par 她很认真地点点头就出去了。等到下一堂课的时候,她变得更沮丧,甚至流下泪来。很显然,她已经好长一段时间都非常沮丧了,她仍然想不出有什么可写,她不明白为什么会这样,如果她想不出波斯曼有何可写之处,她应该想得出来大街上有何可写。
\par 当时Phaedrus颇为震怒。他说:“你根本没有去观察。”这时他突然想起自己因为意见太多而被学校解雇的事。每件事都有无穷的假设,你观察得愈多你看到得就愈多。她还没有开始观察,然而她并不明白这一点。
\par 他很生气地说:“那么就把主题缩小到波斯曼大街上一栋建筑物的正面墙壁。就拿歌剧院为例,从左手边上面的砖块开始写。”
\par 她厚厚的镜片底下,眼睛睁得好大。
\par 下一堂课她不解地交给他五千字的文章,“我坐在对街的汉堡摊旁,”她写道,“开始写第一块砖然后是第二块砖。
\par 在写第三块砖的时候,突然间,我再也停不下来了。别人以为我疯了,不时嘲笑我。但是这就是我所写的,我自己也不明白为何会这样写。”
\par Phaedrus也不明白。但是他在散步的时候仔细想了一阵子,终于得到了结论。
\par 很明显,她就像自己第一天教书的时候,思想一时阻塞反应不过来。她之所以会卡住是因为她只想重复听过的事,就像他第一天,只想重复早已决定要说的内容。她之所以写不出有关波斯曼的事,是因为她想不出波斯曼有什么值得重复写下来的地方。很奇怪,她竟然不知道自己可以从不同的角度观察,而不要在乎别人说过什么。而把题材缩减到一块砖就突破了她的瓶颈。因为很明显地,她必须直接地、不受任何阻碍地观察这块砖。
\par 他又进一步实验。在课堂上,他要所有的人花一个钟头描写他大拇指的背面。一开始大家觉得很滑稽,但是每一个人都照着做了,而没有任何人抱怨不知从何下笔。
\par 在另外一班他把题材改为钱币,每一个学生整个钟头都在奋笔疾书。而在另外一班也是同样的情形。有的人会问:“需要写两面吗?”一旦他们能自己直接观察,就会明白有无穷的题材值得写,这是一种培养信心的训练,虽然他们所写的看似微不足道,但是终究是自己的作品,而不是模仿别人之作。做过这种练习的班级,学生所写出来的文章都流畅得多而且有意思多了。
\par Phaedrus经过实验得出结论,模仿是一种真正的罪恶。在他开始教修辞学之前必须先清除这种习惯。模仿似乎是一种外界的压迫,小孩子从来不会这样,似乎是后来附加上去的,也很可能是学校教育的结果。
\par 这种见解听起来似乎很正确,他越想越觉得错不了。学校教你去模仿,如果你不模仿,老师就给你很差的分数。
\par 而在大学里,情况就复杂多了,你必须要让老师觉得,虽然你实际是在模仿,但是表面上并没有模仿。你就是吸收老师指示的重点,然后再走自己的路。这样你就能得到高分。而原创的学生则可能从最高分到最低分都有,整个学校的价值评估都反对创意。
\par 他曾经和住在隔壁的心理学教授讨论这个问题。他是一位非常有想像力的老师,他说:“没错,你只有把整个教育的学位和评分制度取消,才能得到真正的教育。”
\par Phaedrus想过这一点。几个礼拜之后,有一位非常聪慧的学生想不出学期报告的题目。由于她仍然在思索当中,所以就把这个题目给她当学期报告。一开始她并不喜欢这个题目,但是还是勉强接下来了。
\par 在一个礼拜内,她和每一个人都谈论这个题目。两个礼拜之后,她交出了一篇非常精彩的报告。当她向同学讲解的时候,因为大家并没有花两个礼拜的时间思索过,所以对于取消分数和学位的看法极力反对。然而这一点并没有使她放慢脚步,她陈述时的声调听起来好像古代热情宣扬教义的传教士,她恳请其他的学生听她的讲述,了解她所说的才是正确的。她说:“我所说的这些,不是为了他,”然后看了Phaedrus一眼,“而是为了你们。”
\par 她恳求的声调、宗教的热忱,深深地打动了他。而他也知道,在大学入学考试的时候,她的成绩十分优秀,所以算是班上前几名的学生。在下学期教如何写具有说服力的文章时,他又以这个题目作为示范。他先自己写了一篇文章,然后在学生面前反复修正。
\par Phaedrus拿这篇文章当作范例,避免去谈作文的种种规则,这些规则连他自己都十分怀疑它们的作用。他认为直接把自己的文章拿给学生看,可能其中有不少错误的地方,也有不通之处,还有需要删减的地方;但是惟有这样,才能让学生多明白一点儿真正的写作是怎么一回事,而不必浪费时间慢慢去挑出学生的错误,或者拿大师的作品来模仿。
\par 于是他就进一步研究取消整个分数和学位体系的可能,为了让学生有真正的参与感,他决定这个学期不给学生打任何分数。
\par 现在可以看到山顶上的积雪了,从山脚看上去,似乎需要几天的时间才能爬到山顶。山顶的岩石似乎非常陡峭,很难直接爬上去,尤其我们身上的行李又十分沉重,而且Chris还太小,不能用登山绳和爬岩钉的方式上去。我们必须先穿越这一片山脊上的树林,进入另外一个峡谷,然后走到尽头,再转回头爬上山脊。三天之内要爬上去可能赶了些,四天就会很轻松。但如果我们九号以前没有回去,狄威斯很可能就会开始找我们。
\par 我们停下来休息一会儿,靠着一棵树坐下,这样才不会因为重心不稳而向后摔倒。过了一会儿,我把手伸到背后,从背包里拿出一把弯刀给Chris。
\par “你看到那里有两棵白杨树吗?直直的两棵,在边上?”我指着它们说,“把它们从离地面一英尺的地方砍下来。”
\par “为什么?”
\par “爬山的时候我们可能需要它们,也可以做帐篷的柱子。”
\par Chris拿了弯刀正要去砍树,又转了回来,他说,“你去砍吧!”
\par 于是我拿起弯刀,走过去把树砍下来,只要一刀就可以把树砍得十分整齐,只差把树皮扯掉。走到岩石地区需要拐杖才能保持平衡,而上面的松树并不适用,这是仅有的白杨树。不过我有点担心,Chris不愿意帮忙,在山上这不是个好现象。
\par 休息了一会儿,我们继续前行,过了好一会儿才习惯身上的重量。现在我们对所有重量都会有消极的反应,继续前行下去就会逐渐习惯了……
\par Phaedrus想要废除分数和学位的论点,使学生们十分困惑,开始反抗。有一些学生认为他想摧毁整个学校制度。
\par 有一个学生开门见山地说:“你当然不可能废除分数和学位制度,毕竟这是我们来这里的目的。”
\par 她说的没错。虽然每一个人都不喜欢暴露自己的真实想法,但是如果说大部分的学生来学校受教育不是为了学位和分数,实在有点虚伪。当然,确实有些学生只是单纯为了受教育而来的学校,但是学校里机械化的教学方式很快就使他们放弃了自己的理想。
\par Phaedrus在范文当中认为,取消分数和学位的制度可以消除这种虚伪的现象。不过他不以整体为研究的对象,他单单举出一位想像力十分丰富的学生作为代表——他来学校就是为了分数,而非真正的知识。
\par 根据范文中的假设,这样的学生上学之后就开始准备交报告,很可能出于惯性,第二个、第三个报告一直做下去,然后这门课的新鲜感逐渐消失,由于求学并不是他生活中惟一的目标,还有其他的任务和需求给他压力,他很可能就无法再交报告了。
\par 由于没有评分和学位的制度,他很可能不会受到处罚,而老师在接下来的授课中则假设他已经交了作业,课程仍然循序渐进,他可能就会开始觉得困难。
\par 接下来,这困难就减弱了他对这门课的兴趣。这种恶性循环之下,他很可能会根本交不出报告,然后他又不会受到任何惩罚。
\par 在他愈来愈跟不上学校的进度时,他也可能愈来愈无法集中精神,最后他发现自己什么也没有学到,却要不断面对外界的各种压力,于是他只好停止上学,同时对自己的这种行为很惭愧。这个时候,学校仍然没有给他任何惩罚。
\par 但是会发生什么事呢?由于别人对他的评价不高,这个学生很可能就把自己给毁掉。这样最好,这正是应该有的现象,因为他最初就不是为了求取真正的知识而来,因而在班上也无所作为,这样就省下不少时间、金钱还有精力。
\par 在他心目中也不会认为自己曾经失败,从而影响了他的后半生。
\par 学生最大的问题就是,因为多年来胡萝卜和鞭子的教育方式,造成了他思考上的惰性。就好像一头驴子:“如果你不打我,我就不工作。”如果没有人鞭打它,它就不会工作。而训练它去拉的文明的车子,很可能就会因此而走慢了一点。
\par 然而如果你认为人类文明的前进是靠着驴子来拉的话,那真悲哀。这是一般人的看法,却不是教会的态度。
\par 教会的态度是:文明、制度或是社会,不论你如何称呼它,最好是由有自由意志的人而非驴子来维系。废除分数和学位的目的,并不是要去处罚驴子或者是抛弃它们,而是给这些驴子适当的环境,让它变成自由的人。
\par 这位像驴子一样的、假设出来的学生会继续游荡一阵子,他可能得到另外一个像他抛弃的教育一样珍贵的学习机会,他不再浪费时间和金钱去做一头高级的驴子。他可能找到一份工作,安然地做一头低级的驴子,也可能做一名技工。然而事实上他真正的地位会提高,因为这样才可能有所贡献而带来改变。
\par 也可能他终身就做这份工作,也可能他就此找到自己生活的层面,然而并不满足于此。
\par 短则六个月,长则五年,很可能会产生变化,他对自己每天机械化的工作愈来愈不感兴趣,过去被学校的理论和分数所压抑的创造本能,现在很可能因为工作的无聊而被唤醒了。他花了数千个钟头去解决机械方面的问题,因而对机械设计愈来愈有兴趣。他可能想要自己设计机器,因为他相信自己会做得更好,于是尝试改造一些发动机。成功之后,就想要更大的成功。然而这个时候,他可能会遇到瓶颈,因为他没有理论基础。这个时候,他就会发现以前自己丝毫不感兴趣并觉得一无是处的理论,现在变得有一些值得敬重之处。
\par 于是他就会回到没有分数也没有学位的学校里,这时他变了,不再为分数而来,而是为了追求真正的知识。他不需要别人强迫他去学习,他的动力来自于内在。这个时候,他就是一个自由的人,他不需要许多训练的督促。事实上,如果老师上课的态度松懈,他很可能会唐突地问许多问题去鞭策老师,于是他就会常常来上课,即使花钱也在所不惜。
\par 一旦转变成这种学习动机,就会产生强大的爆发力,在没有分数和学位的教育机构里,学生找到了自己,他不必浪费时间在机械化的理论上,研究物理和数学是发自内在的兴趣,因为他知道这是自己的需要。而冶金和电子工程也会得到他的青睐。他对这些抽象的学问熟悉后,就去研究其他的理论,虽然和机械不直接相关,但是也会成为他学问的一部分。这种学习方法和今日大学教育强调的模仿不同,虽然你得到了分数和学位,让人以为你有很高深的知识,然而事实上,只有你自己知道内在空空如也。这就是Phaedrus提出的范例,也是他不受欢迎的论点。他整个学期不断地删改,反复地研究。学生交来的报告,他只给评语,没有任何分数,然而在另外一本小册子里,却记下学生的分数。
\par 就像我以前说过的,一开始几乎每一个人都有一些茫然不解,大部分的学生以为他们碰到了一个理想主义者,认为取消分数可能会让学生快乐一点,因此更努力地研究学问。
\par 事实上,没有分数每一个人都会很茫然。上学期得到甲等的学生一开始非常愤怒而且轻视这种做法。然而由于他们本身具有自我训练的素养,所以仍然会做作业,至于得到乙等的学生以及丙上的学生,就会漏掉部分的报告,即使交来也很松散,而许多丙下和丁的学生甚至不来上课。这时别的老师就会来问他,面对这种消极的反应该怎么办。
\par 他说:“慢慢等下去就知道了。”
\par 刚开始,他对学生放松的态度令他们颇为不解,继而他们就怀疑起来。有些学生开始暗暗地问一些讽刺性的问题,然而他都用很温和的口吻回答他们,上课仍然照常进行,只是老师不再给任何分数。
\par 然后希望出现了。在第三四个礼拜的时候,甲等学生开始有些紧张,于是交来非常精彩的报告,下课之后也围着他问问题,希望得知他们究竟做得如何。
\par 乙等和丙上的学生开始注意这个现象,于是也交了一些符合他们程度的报告。
\par 至于丙下和丁甚至戊的学生也开始来上课,看看究竟发生了什么事。
\par 学期终了之后,甚至出现另一种更令人振奋的现象,甲等学生不再紧张,而变得积极地参与课堂上的活动,态度也十分友善,这在原先注重分数的班级是少有的现象。这个时候,乙等和丙等的学生开始紧张了,由交来的报告就可以看得出来,他们花了不少心血。至于丁等和戊等的学生也都交出令人满意的作业。
\par 一般在学期的最后几个礼拜,大家都知道了自己的分数然后就心不在焉地斜坐着上课。然而Phaedrus却让学生仍愿意积极参与课堂上的活动,因而引起了其他老师的注意。乙等和丙等学生开始参加甲等学生自由自在的讨论,让整个课堂像在开一个很成功的聚会,只有丁等和戊等的学生呆呆地坐在位子上,显得十分焦虑。
\par 后来有两个学生告诉他产生这种轻松友好的气氛的原因,“有很多人下了课就动脑筋,想要打破这种做法。每一个人都相信最好的方法就是假定你可能会被留级,然后尽量做好,这样你就会觉得很轻松,否则你可能会发疯。”
\par 另外一些学生补充说:“一旦你习惯了,其实也不坏,你会对老师教的更感兴趣。”但是他们重复一点:“要习惯并不容易。”
\par 在学期末的时候老师要求他们写一篇评估这种做法的文章。这个时候没有人知道他们的分数如何,百分之五十四的人反对这种做法,百分之三十七的人赞成,百分之九的人保持中立。
\par 若是按一个人一票算,这种做法并不受欢迎,大部分的学生仍然想要分数,在得到调查结果之后,Phaedrus根据他小册子里的分数加以分析,他发现一个现象,甲等的学生赞成与反对的比例是二比一,而乙等和丙等的学生则是一半一半,至于丁等和戊等的学生则一致反对。
\par 这种结果让他证实了一种暗暗觉得不妙的现象:愈聪明愈认真的学生愈不需要分数,很可能是因为他们对学问的本身比较感兴趣。而愈懒惰愈愚笨的学生则愈需要分数,因为可以让他们知道自己是否及格了。
\par 正如狄威斯说的,从这里往正南方走有七十五英里长的森林和积雪,了无人迹,也无路可走,东西向的道路倒是很多。我的安排是,如果第二天路上的情况不妙,我们可以走最近的一条路及时脱身。Chris并不知道这一点,所以有点伤到了他青年会式的冒险精神。然而一旦进入深山之后,他这种冒险的精神就逐渐消失了,因为有不少实际的危险出现,或是走岔了一步,或是扭到了脚踝,或是发现自己和文明的距离有多么遥远。
\par 这么高的地方,很显然少有人来到这里,又走了一个钟头之后,我们发现人迹几乎已经消失了。
\par Phaedrus认为不评分是一个不错的做法,但是他并没有从严谨的角度评估它的价值。在真正的实验当中,你会提出各种原因,保持其他,只改变其中一项,看看它的改变会产生什么效果。然而在教室里你不可能这样做,学生的知识、学习的态度、老师的态度都可能受各种无法控制的因素和不可知的力量影响。
\par 观察者也是原因之一,如果不改变自身,他就不可能对效果做客观的判断。所以他并不想做任何严谨的推论,他只想按照自己的喜好进行。
\par 当他做这个实验的时候,会产生一种不良的现象。如果老师很差劲,很可能一整个学期都没有教学生任何东西,而是根据一些不相关的测验计分。然后让人以为有些人学得好,有些人学得不好。但是一旦取消了分数,学生每天就被迫去思考到底学到了什么,老师教了什么,目标是什么,作业如何达到目标等等。因此,取消分数之后,就产生了一个非常令人恐惧而又庞大的真空地带。
\par 然而Phaedrus想要怎么做呢?这个问题变得愈来愈重要。他开始做了之后,发现原先认为对的答案似乎愈来愈走样。本来他希望学生自己决定什么是好文章,而不要一直问他。因为取消分数的真正目的,就是要他们深切地自我反省,由他们自身找到对的答案。
\par 然而现在这样并没有多大意义,如果他们已知道好坏之分,他们就没有必要来修这门课。他们之所以来学,就是假定他们无法分辨好坏。而他身为老师,就有必要告诉他们好坏的差异在哪里。
\par 所以发掘个人的创造力,以及训练学生在课堂上的表达力,基本上和学校的整个思想模式是互相抵触的。
\par 对许多学生来说,分数取消无异于一场恶梦。他们要去做一些事,这是为自己的失败受的处罚,但是没有人告诉他们该做什么。他们一再反省也不明白,看看Phaedrus也没有答案,只好无助地坐在那里,不知道该做些什么。那种气氛甚至让一位女孩子精神崩溃。你不能取消分数,这会让学生变得毫无目标,你必须让学生有一个努力的目标。然而他并没有这样做。
\par 他不能这样做,因为一旦他告诉他们怎么做之后,就可能落入权威、教条式的教法。然而你又如何把每一个独立个体的内在神秘的目标写在黑板上呢?
\par 第二个学期,他放弃了这种做法,恢复打分数。然而他觉得很沮丧也很苦恼,因为他觉得自己那样做是对的,而结果却完全不是那么回事儿。在班上的确产生了主动追求学问的热情,但这不是他的指导所产生的。他准备辞职了。
\par 把心怀怨恨的学生教成一个模子里出来的,这不是他想要做的。
\par 他听说俄勒冈州的瑞德大学一直到毕业都不曾打过分数。暑假的时候,他到那儿去了一趟。听说教授也分成两派,但没有人真正喜欢这种做法。在整个剩余的假期当中,他变得非常沮丧懒散。
\par 他和太太在山里露营了许久,她问他为什么一直都这么沉默,他也说不出原因,他只是停下来等待,等待那颗思想上尚未出现的晶种,能够突然地把一切都具体化。
\subsection*{17}
\par Chris心情似乎很不好,有一阵子他远远地走在我前头,现在坐在树下休息,看也不看我一眼,所以我知道有问题了。
\par 我在他旁边坐下来,他的表情很冷淡,脸也涨红了。我知道他已经筋疲力尽,于是就静静地坐着,听风吹过松树林的声音。
\par 我知道最后他仍然会站起来继续向上爬,但是他自己不知道,所以害怕不能再继续爬了。我记得Phaedrus写过有关这些山的事,所以就告诉Chris。
\par “许多年以前,你妈妈和我在离这儿不远的湖边露营,旁边有一个沼泽。”
\par 他没有抬起眼来看我,但是他在听。
\par “大约在天亮的时候,我们听到落石的声音,以为是山中的动物。除此之外,通常是不会有这种声音的。然后我听到有东西掉进沼泽里,这时候我们都醒过来了,我从睡袋里慢慢地爬出来,从夹克里拿出手枪,蹲在一棵树旁。”
\par 这个时候,Chris忘记了自己的问题。
\par “这时又传来一块落石的声音,我以为有人骑马经过此地,但是不会在这个时候啊!声音又来了。接着是轰隆轰隆的声音。这不是骑马。轰隆的声音愈来愈大,在晨曦的微光中,我看到一只身形非常大的鹿,它的角有一个人那样宽,长得又高壮,可说是山上仅次于灰熊的危险动物。也有人认为它才是最可怕的。”
\par Chris睁大了眼睛。
\par “又是一阵响声,我扣上了扳机,心想这把38.8 手枪可能对付不了这只鹿。但是它没有看到我,然后又是一声巨响。我们不能挡住它的路,但是你妈妈的睡袋正好在它要经过的路上。然后又是一声响,它已经跳到十码之遥,于是我站起来瞄准目标,它又向前跳了好一阵子,然后停下来,离我们只有三码,然后看着我……我用准星瞄准它的两眼之间……我们都一动也不动。”
\par 我的手伸到背包里,拿出一些奶酪来。
\par Chris问我:“然后呢?”
\par “让我先切点奶酪。”
\par 我拿出小刀,把奶酪用纸包起来,以免手沾到,然后切下一片来给他。
\par Chris接过去又问,“然后呢?”
\par 我一直到他吃了第一口,才又继续说下去,“那只公鹿大约看了我五秒钟,然后看了看你妈妈,然后又看了看我,然后看了看我手中的枪,就微笑着慢慢走开了。”
\par Chris说,“哦!”他有点失望。
\par “通常它们碰到这样的状况都会攻击,但是它觉得这么好的早上,又碰到我们,为什么要惹麻烦呢?这就是它为什么会笑。”
\par “它们会笑吗?”
\par “不会,但是看起来好像在笑。”
\par 我放下奶酪,然后说:“后来我们爬山的时候,要找圆的石头当踏板,我正要踏一块棕色的大岩石,突然之间,它跳了起来,跑到树林里去了,原来就是刚才的那只公鹿,我想它对我们一定很没办法。”
\par 我帮Chris站起来,说:“你走得太快了,现在山路已经很陡峭,我们必须慢慢地走。如果你走得太快就会喘气,喘气太严重就会头昏,精神也会变得很差,然后你就会以为自己没有办法再爬下去了。所以,还是慢慢地走一阵子。”
\par 他说:“那么我跟在你后面。”
\par “好啊!”
\par 我们离开原先沿着走的小溪,顺着峡谷旁边坡度最小的路走。
\par 爬山必须尽可能地少费力,不要存有任何妄想,而要以自身的状况决定速度。如果你已经觉得很不耐烦,那就加快速度,如果有点气喘就慢下来,要在这两者之间保持平衡。当你的思想不再集中在眼前的行动上,每爬一步不是为了爬上山顶,你会发现,这里有一片锯齿状的叶子;这块岩石有点松动;从这里山顶上的雪不太容易看见,即使愈来愈接近山顶。这些都是你应该注意的事。
\par 如果你只是为了爬到山顶,这种目标是很肤浅的,维持山的活力是靠这些周遭的环境,而不单单只是山顶而已。
\par 但是当然,没有山顶,就不会有山的周围,是山顶界定了周围。于是我们又继续向上爬…… 我们还有好长一段路……所以不必急躁……只要一步接着一步慢慢地爬,偶尔来一段Chautauqua点缀……
\par 精神活动远比看电视有趣多了。大部分人只看电视,这真是很丢脸的事。他们可能认为听到的一点也不重要,但是情形完全不是这样。
\par Phaedrus的记载中有很大一段,写一次他要班上的学生写一篇《思想和陈述中的良质》。学生们的情绪逐渐不安起来,几乎每一个人都像他过去一样,对这个问题又懊恼又愤怒。
\par 他们说:“我们怎么可能知道良质是什么呢?应该是你来告诉我们。”
\par 然后他告诉他们,他也不知道,而且很想知道答案。他提出这个问题,就是希望有人能够找到答案。
\par 他这样说就更加点燃了大家愤怒的情绪,教室里掀起了一阵骚动,有一位老师甚至探头进来看究竟发生了什么事。
\par Phaedrus说:“没事,我们正好在某一个问题上有一点冲突,一时情况很难恢复正常。”有一些学生对这种现象还很好奇,而吵闹声逐渐平息下来了。
\par 有一位学生说:“我坐着想了一整晚。”
\par 有一位坐在窗户旁边的女孩子说:“我要哭了,我快要疯了。”
\par 第三位同学说:“你应该事先提醒我们。”
\par “我应该怎样提醒你们呢?我自己也不知道你们会有怎样的反应。”有一位十分不解的学生看着他,终于有一点明白——他真的不是在玩弄他们,他真的是想知道答案。
\par 他真是一个奇怪的人。
\par 然后有一个人问:“你的想法呢?”
\par 他回答:“我不知道。”
\par “但是你究竟怎么想呢?”
\par 他沉默了好一阵子:“我知道有所谓的良质存在,但是一旦你想要去界定它,情况就会变得很混乱,因而无法做到这一点。”
\par 大家都十分同意。
\par “为什么会有这种现象,我不知道。
\par 我想或许我能够从你们的报告中得到一点概念,我真的不知道。”
\par 这一次轮到同学们沉默了。
\par 在当天接下来的其他课堂上,也出现了同样的情况。但是每一班都多少有一些学生会自动地做出一些善意的回应。
\par 过了几天,他自己想出一个定义,于是把它写在黑板上让学生抄下来,定义是这样的:“良质是一种思想和陈述的特质,我们不能经由思考的方式了解它,因为要给它定义是一种僵硬而正式的思考过程,良质是无法被界定的。”
\par 这个定义其实就是拒绝给它定义,并没有引起学生的评论,因为这些学生没有受过正式的训练,不知道他写下来的句子其实是完全不合理的。如果你不能为某件事下定义,你就没有办法用理性的方法研究它的存在。于是你也无法告诉别人它究竟是什么。因而事实上在无法界定和愚蠢之间就没有差别了。当我说我无法界定良质,我其实就是说,我在研究良质这件事上很愚蠢。
\par 幸而学生们不知道这一点,如果当时他们对这一点有意见,他很可能就无法回答他们了。然而在黑板上的定义下面,他又写着:“但是即使良质无法界定,你仍然知道它是什么。”这个时候又引起学生一阵骚动。
\par “哦!我们不知道。”
\par “你们知道的。”
\par “哦!我们不知道。”
\par “你们知道的!”他已经准备了一些资料要拿给他们看。
\par 于是他选出学生的两篇文章做例子。第一篇写得十分凌乱,有一些很有趣的想法,然而却无法构成完整的文章。
\par 第二篇写得非常好,但是他刻意隐瞒自己为什么写得这么好。Phaedrus把两篇都读给大家听,然后要大家举手表决,谁认为第一篇比较好,有两个人举手;他又问有多少人认为第二篇比较好,有二十八位同学举手。
\par 他说:“你们有二十八位同学举手认为第二篇比较好,这种价值判断就是我所谓的良质。所以你们知道良质是什么。”
\par 大家沉默了许久,在重新思考他的话。为了更进一步地强调他的看法,他在班上读四名学生的报告,然后要每一位学生按着他们的标准把优劣写在纸上,他自己也写下来。然后在黑板上统计出全班的意见,同时把他的评价也写上去,两者之间十分接近。一开始班上对这种练习很感兴趣,但是过一阵子就觉得很没意思了。他所谓的良质非常明显,他们早已经知道究竟是怎么回事了,所以没有兴趣再继续听下去。现在他们的问题变成:“好吧!既然我们知道良质是什么,我们怎么样得到它呢?”
\par 现在必须研究修辞学了,而他们也不再反对其中种种的规条。这些规条不是目的,只是一些写作技巧,以制造某些效果。他把良质的各个层面列出来,比如说:统一、生动、权威、简洁、敏锐、清晰、强调、流畅、悬疑、出色、准确、比例适当、有深度等等。由于这些抽象名词都很难定义,所以他就利用刚才的比较手法介绍给学生们。比如说文章的统一,也就是故事如何前后连贯,可以借着撰写大纲改进自己的技巧。而要提高文章的权威性,则可以增加注释,因为注释能够提供更多权威性的参考。
\par 在大一的课程里面都会提到的大纲和注释,但现在却被作为提高良质的方法。
\par 如果学生交来的报告中列出的注释毫不起眼或是大纲松散,就表示他只是敷衍了事,没有达到报告应有的良质,所以毫无价值可言。
\par 然而要回答学生的问题,“我怎样才能得到良质?”这几乎使他想要辞职。
\par 他认为:“这和你要如何得到它完全无关。它就是这样好的东西。”有一位不满意的学生在课堂上问:“但是我们要怎样才知道什么是好呢?”就在他问出口之前,他明白已经有答案了。别的学生经常告诉他:“你已经看到了。”如果他说:“我没有。”他们就会说:“你看到了,他已经证明了这一点。”学生已经完全可以自己评断良质了——就是这样,他教会了他们写作。
\par 现在,Phaedrus被学校的体制所逼,不得不说出他想得到什么,但他认为这样强迫学生接受他的看法,会摧毁了他们的创造力。完全顺从他看法的学生注定要丧失创造力,或者写不出自己认为真正够水准的文章。
\par 于是他修正一项基本规则,就是任何要教授的内容必须先界定,他已经找到一条出路。一篇优秀的作品不需要任何规则,不需要任何理论,然而他指向了某种东西,非常真实,他们无法否认它的存在。因为取消分数所造成的真空突然之间被良质的正面效应所充满,两者完全结合在了一起。学生十分惊奇地到他办公室来告诉他:“我过去真的很恨英语,现在我在上面所花的时间比其他的都要多。”并不是只有一两位学生来告诉他,而是许多学生都有同样的反应。
\par 这个良质的观念非常棒,它发挥作用了。
\par 最后大家终于明白,凡是有创意的人都有那个神秘而属于个人的内在目标。
\par 我转过身来看看Chris在做什么,他脸上的表情显得很疲惫。
\par 我问他:“你觉得怎么样了?”
\par 他说:“还好。”但是他的口气有些冲。
\par “我们可以随时停下来扎营。”我说。
\par 他瞪了我一眼,我也不再想说什么了。一会儿他在我旁边慢慢努力地向上爬,于是我们又继续前行。
\par Phaedrus之所以能够将良质的观念拓展到目前这个地步,是因为他刻意专注于班上同学的反应而忽视其他的一切。
\par 克伦威尔曾经说:“一个没有目标的人才能爬到最高。”这话颇为适合这种状况。
\par 他不知道自己要往哪里去,他知道的只是这么做有效。
\par 然而就在他已经知道这种做法是非理性的之后,他在想为什么它很有效。
\par 为什么所有理性的方法都一无可取的时候,这种非理性的方法反而有效呢?他有一种直觉,很快就找到了结论,他偶然发现的道理非比寻常。至于这个道理究竟有多么深奥,他并不晓得。
\par 这是我前面曾经提过的结晶的开始,别人这时都很惊讶为什么他对良质这么感兴趣。他们只看到这个词和它的一般定义,他们并没有看到他过去研究所费的心血。
\par 如果有人问,“什么是良质呢?”这只不过是另外一个问题。但是如果由他来问,因为他有过去的经验,这个问题就会像向四面八方散开来的波浪,并不是呈金字塔的结构,而像一个同心圆,在中间激起波纹的是良质。当这些思想的波浪向四面八方散开的时候,我确信他衷心期望这些波浪能够到达某些思想的彼岸,这样他就能与这些思想结构连接在一起。但是如果真的有任何彼岸存在,那么一直到最终,他也未能到达彼岸。对他来说,只有不断向四面八方结晶的波浪。我现在就是要追寻这些结晶的波浪,也就是他研究良质的第二个层面。
\par Chris在我前面爬,从他的动作看得出来,他已经十分疲倦了,而且火气也大,又不时踢到东西,或让树枝刮到身体却不拨开。
\par 看到他这样我很难过,这要归咎于我们出发前,他曾经参加了两个礼拜的青年会夏令营。他给我讲过他们的野外活动,借着游泳、结绳……来训练他们的男子汉气概,他曾经提过十几种,但是我都忘了。
\par 因为有一定的目标,所以夏令营里的同学在参与这些活动的时候,都非常合作而且非常热忱,但是这种动机却会有不良的结果。任何想要以己为荣的目标,结局都非常悲惨。现在我们就开始付出代价了。如果你想通过爬上山顶来证明你有多么伟大,你几乎不可能成功。
\par 即使你做到了,那也是一种很虚幻的胜利。为了维持这种成功的形象,你必须在其他方面一再地证明自己,而内心则常常恐惧别人可能会发现这种形象是虚幻的,所以这么做是错的。
\par Phaedrus曾经从印度写过一封信,提到和一位圣者以及他的信徒去爬喜马拉雅山,它是恒河的源头,也是印度教三大神明之一湿婆\footnote{Shiva,为印度教三大神明之一,象征毁灭之后的再生。另外两大神明为大梵天(Brahma)及毗湿奴(Vishnu)}的住所。
\par 他一直都没有爬到山顶,到了第三天他就放弃了,因为他已经筋疲力尽,于是大家留下他继续往前行。他知道自己仍然有些体力,但这些体力不够。他也有动力,但是也不够。他并不认为自己很孤傲,但是他想通过这一趟朝圣来拓展自己的生活经验,以进一步地了解自己。他把山和朝圣当作自己的目标,把自己视为不变的实体,而不是这趟朝圣或是高山,因而还没有准备好。他想其他的朝圣者之所以能够到达山顶,是因为充分领受到山的神圣,以至于每一步都是一种奉献的表示,是对这种神圣的心悦诚服。山神圣的一面融入了他们的心灵,因而使他们的耐力远远超过了体力所能负荷的。
\par 对没有辨识力的人来说,自我的爬山和无私的爬山看上去可能都一样,都是一步一步地向上爬;呼吸的速度也一样;疲累的时候都会停下来;休息够了又会继续向前行。但是事实上两者多么不同啊!自我的爬山者就像一支失调的乐器,他的步伐不是太快就是太慢,他也可能失去欣赏树梢上的美丽阳光的机会。在他步履蹒跚的时候却不休息,仍然继续前进。有的时候,刚刚才观察过前面的情况,他又会再看一遍。所以他对周围环境的反应不是太快就是太慢。
\par 他谈论的话题永远是别的事和别的地方。他的人虽然在这里,但是他的心却不在这里。因为他拒绝活在此时此地,他想要赶快爬到山顶,但是一旦爬上去之后仍然不快乐,因为山顶立刻就变成“此地”。他追寻的,他想要的都已经围绕在他的四周,但是他并不要这一切,因为这些就在他旁边。于是在体力和精神上,他所跨出的每一步都很吃力,因为他总认为自己的目标在远方。
\par Chris现在似乎就有这个问题。
\subsection*{18}
\par 在哲学上曾经专门讨论良质的定义,就是所谓的美学,它提出来的问题就是何谓美感。这个问题要追溯到古代。
\par 但是以前Phaedrus在哲学系念书的时候曾经极力避免接触这门学问。他故意让自己这门课不及格,而且写的报告让老师异常震怒。他憎恨这门学问,几乎无一处不批评。
\par 并不是某一位美学家激起了他这种反应,而是这门学问。因为他们把良质归纳于某些学问之下,把良质的地位降低,而加以侮蔑。我想这是他生气的原因。
\par 他在一篇报告中写道:“这些美学家认为他们的研究好像一支薄荷的棒棒糖,他们光明正大地用肥厚的嘴唇去舔舐,或是可以大肆地狼吞虎咽一番。通过他们精密的批判,小心谨慎地把良质切成一块块,用刀叉慢慢地送进嘴里,这让我十分恶心。他们所舔的正是早就被他们扼杀而且已经腐烂的东西。”
\par 在结晶的过程当中,他首先看到,如果不去界定良质,那么整个美学也就完全不存在了。就像一个被剥夺公民权的人一样……如果拒绝界定良质,那么它就脱离了分析的过程。如果你无法界定良质,那么你就无法让它隶属于任何知识的领域,美学家也就无话可说了,而界定良质的整个世界也就消失了。
\par 这种想法让他非常震惊,就好像发现了治疗癌症的方法。不再需要解释艺术是什么,学校不再培养冷静的批评家去分析哪一位作曲家是成功的,哪一位是失败的。所有这些自命学问广博的人都必须闭嘴。这不仅只是一种很有趣的念头,更是一种梦想。
\par 我想没有人一开始就知道他准备做什么。他们没有看见他的目标与他们的习惯完全不同。他不但不支持理性的分析,反而否定它。他借用理性的方法来攻击它自己,反而去支持这种非理性的观念,也就是无法界定的良质。
\par 他这样写道:1.每一位作文老师都知道良质是什么(如果有人不知道,他就该小心谨慎地隐藏这一点,因为这只会证明他自己的无能)。
\par (2)如果有老师认为写作的良质能够先界定也应该先界定清楚,那么在他教之前就先界定吧!
\par (3)那些认为写作的良质的确存在但是无法界定,而却值得教学生明白这一点的人,就能从下面的方法中得到益处。
\par 我们不去界定它,而只教给学生纯粹的良质。
\par 于是他又继续提出曾在课堂上做的实验。
\par 我相信他的确希望有人能向他挑战,试着替他界定良质,但是没有人这样做。
\par 他维护自己自由发表意见的权益,这点被大家看重。高年级的同学似乎十分赞同他独立的见解,而像教徒一样地支持他。但是这和强调学术自由不同,他们并不认为老师可以不负责地向学生们胡说八道。这种宗教的态度只是要向理性负责,而不是向政治的偶像膜拜。
\par 他侮辱别人的事实和他言论的真假无关,因此他的理论不会被击垮。但是他们想要打击他的是,他并没有说出一番道理来。他可以随心所欲地去做,前提是得用理性的方法去证实他的理论。
\par 但是你如何用理性去界定拒绝被界定的事物呢?定义就是理性的基础。有理性就有定义,他可以利用辩证法的战术和无能与否的侮辱暂时压制住别人的攻击,但是迟早他得提出一些更实在的理论,引导结晶继续进行,超越传统修辞学的范畴,而进入哲学的领域。
\par Chris回头看了我一眼,神情显得十分痛苦。不会很久了。在我们动身之前就有迹象会发生这种事。狄威斯告诉邻居,我对爬山很有经验,那时候Chris就闪过一丝崇拜的神情,他认为那是很伟大的事。很快地他就会支撑不住了,那么我们就可以休息了。
\par 噢!他倒下来了,他爬不起来了。
\par 不像突然摔倒,而是结结实实地倒下来了。我从他的眼睛里看到了受伤而且愤怒的表情,他想要责怪我,但是我不给他机会。于是我在他旁边坐下来,看着他几乎快崩溃的样子。
\par 我说:“那么我们是在这儿停下来,还是要继续向前走?或者我们也可以往回走,你想要怎么办呢?”
\par 他说:“我不管,我不要……”
\par “你不要什么?”
\par “我不管。”他很生气地说。
\par “既然你不管,那我们就要继续走下去。”我说。
\par 他说:“我不喜欢爬山,一点意思也没有,我以为会很好玩。”
\par 这时我也有些生气,就说:“你说的或许对,但是你不应该把它说出来。”
\par 他站起来的时候,我看到他的眼睛里闪过一丝恐慌。
\par 我们继续向前走。
\par 峡谷一边的天空已经暗下来了,而在我们周围的松树林里,风非常凉爽,但是,似乎有些不祥的兆头。
\par 至少凉爽的风让我们爬起来比较舒服……
\par 我正要谈到因为Phaedrus拒绝替良质下定义,从而在修辞学之外产生的结晶过程。他必须回答这个问题。如果你不去界定它,你又如何肯定它存在呢?
\par 他的答案,在哲学上可称之为实在论。他说:“要证明一个东西的存在,可以把它从环境中抽离出来,如果原先的环境无法正常运作,那么它就存在。如果我们能证明没有良质的世界运作不正常,那么我们就能证明良质是存在的。
\par 不论有没有给它定义。”于是他接着把良质从我们所知道的这个世界中抽离出来。
\par 第一个受伤的就是艺术。如果艺术无所谓好坏之分,那么艺术也就不存在了。因为墙上挂不挂画也无所谓好坏,那就没有必要去挂了。接下来交响乐也是同样的情形。如果刮到唱片的声音或者是演奏者的哼唱声和演奏的音乐一样好的话,那就没有演奏交响乐的必要了。
\par 诗也会消失。因为它通常没什么意义,也没有实用的价值。很有意思的是喜剧也会消失。没有人了解何谓笑话,因为幽不幽默的界线,就取决于是否有纯粹的良质。
\par 接下来消失的是运动。足球、棒球、各种游戏都会消失,因为分数已经丧失了意义,只是空洞的统计,就好像是石头堆一样。还有谁会来参加呢?
\par 接下来他把良质从市场抽离,他预测市场也会发生改变,因为气味的等级变得毫无意义。市场上只会卖最基本的食品,像稻米、玉米粉、黄豆还有面粉;或者一些没有分级的肉和牛奶,只是为了哺育瘦弱的婴儿;还有维生素、矿物质的补充品,以避免营养不良,而烈酒、茶、咖啡和烟草也都会消失。电影、舞蹈、戏剧以及宴会也是一样。所有的人都会改搭大众交通工具,然后穿着像美国大兵一样的鞋子。
\par 有许多人将会失业,但是这可能是短暂的现象,因为我们以后会在基本而缺乏良质的事物中找到工作。应用科学和科技都会急剧地改变,但是纯粹的科学、数学、哲学,特别是逻辑仍然不会变动。
\par Phaedrus觉得继续推演下去非常有意思。纯粹的知识最不受影响。如果抽离了良质,只有理性仍然不变。这是很奇怪的一点,为什么会这样呢?
\par 他自己也不知道。但是他确实知道的是,如果现存的世界没有了良质,就会发现良质原来这样重要。这个世界缺少它仍然能运作,但是生命变得非常呆滞,几乎不值得活下去。事实上的确是不值得活下去的。“值得”就是一种良质的字眼,因为生命不再有价值或是目标。
\par 他重新检视自己的思考过程,认为他证明了自己的看法。一旦这个世界被抽离了良质就不能正常地运作,所以良质是存在的,不论它是否有定义。经他这样抽离之后,他突然想起有一种社会就是这种现象,像古代的斯巴达人,赫胥黎的《美丽新世界》和奥威尔的《一九八四》。他又想起,在自己的生活中有一些人就仿佛属于这种缺乏良质的世界。有一些朋友想说服他戒烟,要他说出抽烟的理由,结果他说不出来,于是他们就表现出很优越的样子,仿佛他做了很丢脸的事,因为这些人对所有的事情都要求理由、计划和解决的方法。他们曾经和他是一类人,但现在他们是他攻击的对象。他想了很久,想找出一个能够形容他们的总称。
\par 他们的世界以知识为主,但是不仅如此,他们假设这个世界的运行要倚靠法则——理性——人类的进步就在于发现这些法则,之后为了满足自我的欲望,而应用这些法则。这就是他们的世界观。
\par 他思考了一会儿这种世界观,接着又想出更多的细节,然后又反反覆覆地想了一阵子,最后回到原点。
\par 朴质\footnote{Squareness,原意是指方正拘谨而且一丝不苟、不要花哨的个性。作者用以表示因这样的态度所带来的平淡无奇的生活形态}。就是这个意思,朴质。一旦你把良质抽出来,你就得到朴质了。缺乏良质就是朴质的精髓。
\par 他想起曾和一些朋友一起旅行,横跨美国大陆,他们是些黑人艺术家,曾经一直抱怨他所描述的这种朴质现象太乏味。他们就是用这个字眼。早在传播媒体使用这个词之前,他们就认为那些所谓的知识枯燥乏味,一点都不想跟它们有任何关联。在他们的聊天中有一种非常有趣的现象,那就是他们认为他就是乏味的化身。他愈想弄明白他们说的是什么,他们就说得愈混乱。现在提到良质,他似乎跟他们一样说得模糊不清。
\par 良质。那正是他们一直谈论的。“嗨!
\par 朋友,是不是请你弄明白一点儿,”他记得有一个人这样说,“请你不要再问那些听不懂的问题。如果你一直问那是什么,就永远没有时间去了解了。”他们黑人所谓的精髓和良质是不是一样的呢?
\par 结晶继续进行下去。他同时看到两个世界。在知识这一边,也就是朴质这一边,他看见良质是一个分裂的字眼,也就是每一位有学问的分析家所热切寻求的。拿起你分析的刀子,把它放在良质这个字眼上,轻轻地敲它,不需要费多大的劲,整个世界就会一分为二——嬉皮式的和严谨的,古典的和浪漫的,科技的和人性的——分得十分清楚,不会乱成一团,也不会有任何遗漏。不只切割得很有技巧,而且运气很好。有时候最优秀的分析家,经此一敲,什么也得不到,只得到一堆垃圾。然而我们这里所提到的良质,就像是我们宇宙观里一条不合逻辑的线,如果你轻敲剖析它的刀子,整个世界就会裂开,切割口利落之至。他真希望康德仍然活着,他会欣赏这种做法。他将发现那是一把超级的钻石刀——而不要给良质任何定义就是关键之处。
\par Phaedrus写道,他意识到自己似乎有些反智的倾向:“在从学术的角度严格定义朴质之前,你可以很简洁地将它定义为:无法察觉良质的存在……我们已经证实,良质虽然没有定义,但是的确存在。
\par 我们可以从教室里的实验中知道它的存在;也可以通过把它抽离现存的世界,在世界无法正常运作时发现它的地位。
\par 抽离良质则只剩下所谓的朴质,朴质往往阻碍我们与良质的接触。”
\par 于是他攻击的矛头转过来指向分析,躺在床上的病人不再是良质而是分析的本身。良质很健康而且毫无问题,然而分析似乎出了毛病,因为它阻挡了人们认识良质。
\par 我向后看,发现Chris落后了好长一段距离,我大声喊道:“加油!”
\par 他没有回答我。
\par “加油啊!”我又叫他。
\par 然后我看他跌坐在草地上,我放下行李,走到他那儿,山坡非常陡峭,我必须先踏稳一步,才可以再踏下一步。
\par 当我走到他那儿的时候他正在哭。
\par “我的脚踝受伤了。”他说着,也不抬头看我。
\par 当登山者想要刻意保护自己的形象时,通常都会撒谎。但是这种情形很惹人讨厌,我竟然让这种事发生真是可耻。
\par 这个时候受他的眼泪还有挫折的影响,我也很生气。我静静地和他对坐了一会儿,然后拿起其他的背包,说:“我先把这些行李背到上面去,然后你在旁边帮我看着它们,不要丢了,然后我会再把我的行李搬上去,然后再回来拿你的,这样你就能休息个够。我们会慢一点儿到山顶,但是还是会到达的。”
\par 但是我说得太快了,所以他听得出来我口气上的厌恶。他也很不高兴,但是什么也没有说,因为他害怕还要再背行李,于是紧皱着眉。在我背行李的时候,他故意不看我。为了平息胸中的气恼,我就告诉自己,这些工作对我来说不算什么。
\par 接下来的一个钟头,我们前进的速度很慢,我把行李向上搬,然后把它们放在一条小溪旁边。我叫Chris拿容器去舀水,他回来之后问:“我们为什么要在这儿停下来呢?我们继续走。”
\par “接下来可能会有好一阵子再也看不到任何小溪,Chris,我很累了。”
\par “你为什么会这么累呢?”
\par 他是不是想把我激怒呢?如果是,那他就做到了。
\par “Chris,我累了,因为所有的行李都是我在背,如果你要赶时间,那么你就拿你自己的行李往上爬,我会跟上来的。”
\par 他有些恐惧地看着我,然后坐下来,几乎要哭了,说:“我讨厌这一切,我真后悔跑来这里,我们为什么来这里呢?”
\par 他大声地哭了出来。
\par 我回答他说:“你也让我很后悔,你最好吃点儿东西当午餐。”
\par “我不要吃,我的胃在痛。”
\par “随便你。”
\par 他走到一边儿,摘下一根草放进嘴里,然后把脸埋在手心里。我独自吃过了午餐,然后又休息了一会儿。
\par 当我醒过来的时候他仍然在哭,我们两个都不知道路该怎么走,只是必须面对眼前的情况,而我不知道眼前究竟发生了什么事。
\par 最后我说,“Chris。”
\par 他没有回答我。
\par 我又叫他,“Chris。”
\par 他仍然没有回答,最后他火气很大地说,“什么事?”
\par “Chris,我要说的是,你不必向我证明任何事。你知道吗?”
\par 他的脸上闪过一阵恐惧的神情,很生气地把头转开。
\par “你不明白我的意思是不是?”我说。
\par 他仍然没有转过头来,也不回答我,风在松林里低喃。
\par 我真的不晓得究竟是怎么一回事,不只是青年会的教导让他这么难过,还有一些其他的事,让他仿佛面临世界末日,每当他想做什么却又做不成的时候,总是会大发脾气或是大哭一场。
\par 我坐在草地上休息,我不想再继续问下去,因为他似乎不会再回答我,我们静静地等着。
\par 后来我听到他在背包里找东西,我转过身来看见他正看着我,他问我,“奶酪呢?”他的口气好像仍然在生气。
\par 但是我不打算松口,我说,“你自己找吧!我不必服侍你。”
\par 他搜寻了一会儿,找到了一些奶酪和饼干,我给他一把小刀子去切奶酪,我跟他说:“我想我准备这样做,Chris,就是把所有重的东西都放在我的背包里,轻的东西放在你的背包里,这样我就不必来来回回地走了。”
\par 他也同意这么做,心情好些了,似乎替他解决了什么。
\par 我的背包现在大概有40 到45 磅重,我们爬了一阵子,这时呼吸可以调节到每踏一步呼吸一次。
\par 崎岖的地方就要每踏一步呼吸两次,有些地方几乎需要直直地爬上去,这时就要依靠树枝和树根。我觉得自己没有绕道走有些失算。现在用白杨树做成的拐杖十分称手,Chris也对使用这根棍子很感兴趣。行李很沉重,但有拐杖就不会跌倒了。你先踏出一步,然后利用拐杖去踏下一步,将身子靠上去,向上一爬,呼吸三下,然后再踏下一步,再把拐杖向上一戳,然后再靠上去……
\par 我不知道今天是不是还要继续Chautauqua。下午的时候,我的思考已经有些模糊不清了,或许我还可以再谈一点,然后今天就到此为止。
\par 在我们开始这趟旅程之前,我提到John和Sylvia对于科技给人的窒息避之惟恐不及。事实上,有很多人都像他们一样。我提过有些从事这方面工作的人,也有同样的反应。产生这种现象最主要的原因就是他们只看事情的表面,而我则是看事情的内部。我称John的观点为浪漫的,而我的则是古典的。以六十年代的背景来说,他是嬉皮式的,而我的则是朴质的。然后我们了解了朴质的世界如何运行,我们讨论过它的资料分类、系统、因果关系还有分析等等。然后我提到从我们周遭的世界取来一把沙,以及如何把这一把沙分类。古典的认知步骤就是了解这些沙子的性质,以及分类的方法和彼此之间的关系。
\par Phaedrus拒绝给良质下定义,就是想要在古典和浪漫的世界之间找到一个平衡点。而良质似乎就是关键所在。两个世界都用到这个词,两个世界都知道它究竟是什么。浪漫的人因为它的本质而欣赏它,而古典的人则是企图用它作为知识体系的根基。由于没有下任何定义,古典的人被迫要从浪漫的角度去看它,而不会因思想的结构而失真。
\par 我想要连结古典和浪漫这两个世界,但是Phaedrus的目的不同。他对于融合这两者并不感兴趣,他追求的是他自己的灵魂。因而他要探索良质更宽广的含意,这最终导致了他的崩溃。而我和他不同的是,我无意往哪方面走。他只是经过这块地区,把它开发出来,而我想要留下来,看看我是否能培养出一些生命。
\par 我认为,如果一个词能够把世界分成两半,那么它也势必能再将其合而为一。真正了解了良质,不单单能满足体系的需要,甚至能超越它。真正了解良质之后就能掌握这个体系,将它驯服,然后能为个人的目标派上用场,让人拥有完全的自由,从而实现他内在的目标。
\par 我们停下来向下望,Chris的精神显然好多了。但是我害怕他的自我又在作怪。
\par “你看我们已经爬多远了。”他说。
\par “我们还有许多路要走。”
\par 后来,Chris向山谷大叫,想要听自己的回声,然后把石头丢下去,看看会落到哪里。他开始有点骄傲起来,于是我加快呼吸的速度,大约有以前的一倍半。这样就让他的气焰稍稍降下来,于是我们又继续爬上去。
\par 大约在下午三点的时候,我的步伐变得沉重起来,是到该停下来的时候了,而且我的精神不太好,在目前这种状况下继续爬下去,很容易扭到脚,然后第二天就会很惨。
\par 我们来到一处平坦的地方,有一个大圆丘拱起来,我告诉Chris说今天就到这里为止。他似乎很高兴,或许他认为自己有所进展。
\par 我想要睡个午觉,但是看天上的积云仿佛要下雨了。由于云层很浓,我们看不到谷底,只看得到另外一座山峰的山脊。
\par 我把背包打开,然后拿出帐篷,还有军队用的斗篷。我拿了一条绳子,把它绑在两棵树之间,然后再把帐篷挂上去,我用一把弯刀砍了一些灌木当棍子,然后在周围挖了一条小沟,让雨水可以流下去。当雨落下来的时候,我们已经把所有的东西都搬进帐篷里了。
\par Chris看到雨势这么大,反而十分兴奋。我们躺在睡袋上,看雨白花花地落下来,听它叮叮咚咚地敲打着帐篷顶。
\par 由于森林里弥漫着一股浓雾,我们两个都变得不爱说话,只是静静地看着雨打在灌木丛的叶片上。打雷的时候,我们不禁吓了一跳,但是心里还是很高兴,周围全都被雨淋湿了,而我们却不受影响。
\par 过了一会儿,我把手伸到背包里去找梭罗平装本的书。在昏暗之中,有点费力地念给Chris听。我想我说过,我也念其他的书给他听,都是他不懂的书。
\par 情形都是这样的,我先念一个句子,然后他提出许多相关的问题,一直到他对我的回答满意为止,再念下一个句子。
\par 我这样念了梭罗的书好一会儿,大约半个钟头之后,我有点失望,因为梭罗并没有来到我们当中。Chris跟我都有一些不安,句子的结构有些跟不上潮流,最起码这是我的感觉。这本书读起来有些消沉,我从来不认为梭罗是这样的。但实际情形就是这样。他谈的是另外一个时空底下的事情,只是提出科技的恶果,而不是解决的办法。所以他并不是在对我们说话。于是我很不情愿地放下这本书,我们两个都沉默下来,各自思索。只剩下Chris和我,还有一片树林和雨水。没有任何一本书能指引我们的路了。
\par 我们摆在帐篷旁边的小盘子里已经贮满了水,于是我们把它倒进一个大锅里,然后加了一点浓缩的鸡汤,在一个小火炉上煮起来。每当爬得筋疲力尽的时候,吃任何食物、喝任何汤都会觉得非常美味。
\par Chris说:“和John夫妇比较起来,我更喜欢和你一起露营。”
\par “大家的情况不同。”我说。
\par 肉汤煮好之后,我又拿出一罐猪肉和青豆子倒进锅里。需要好久才会煮熟,但是我们并不赶时间。
\par Chris说:“闻起来真香。”
\par 雨已经停了,只有雨滴偶尔打在帐篷上。
\par “我想明天会是晴天。”我说。
\par 我们把这一锅猪肉和青豆子传来传去,两个人从不同的方向吃。
\par “爸爸!你这一阵子都在想什么?
\par 你总是一直在思考。”
\par “嗯!各种事情。”
\par “什么样的事呢?”
\par “像是下雨后会有什么问题,还有别的事。”
\par “什么样的事?”
\par “就像你长大了会是什么样?”
\par 他很感兴趣,“那会是什么样呢?”
\par 我看到他的眼神中有一些自大的神情,所以我的回答自然就是这样:“我不知道!其实那也正是我正在想的。”
\par “你认为我们明天会爬到山顶吗?”
\par “会啊!我们离山顶不远了。”
\par “早上吗?”
\par “我想是吧!”
\par 不一会儿他睡着了。从山脊上吹来一阵潮湿的晚风,吹得松树林响起一阵仿佛叹息般的声音。而松树也缓缓地随着风摇动,一会儿直起身来,一会儿又被风吹弯了。它们受到这些外力的影响,变得无法稳定下来。帐篷被风吹得有点晃动,我起身把钉子钉好,然后在圆丘四周潮湿的草地上走了一阵子,就爬进帐篷,静静地等着睡意来到。
\subsection*{19}
\par 阳光透过松树林洒到我的脸上,让我慢慢地知道身在何处。它也驱走了我的睡意。刚才我做了一个梦,梦见我在一间有着白色墙壁的房间里,看着一扇玻璃门,在门外的是Chris和他弟弟还有母亲。Chris向我挥手,他弟弟在旁边笑,而他母亲却在一旁流泪,然后我看到Chris脸上的笑容很僵硬,事实上,相当恐惧。
\par 我向门靠近,他开朗些了,他示意我把门打开,我想打开它,但是打不开。
\par 他脸上又出现惊恐的表情,但是我转身走开了。
\par 我以前常常做这个梦,它的意思很明白,而且和我昨天晚上提到的事情颇为契合。他一直想和我亲近,但是又怕永远没有这个机会。情况愈来愈清楚。
\par 帐篷外,地上的松针被太阳晒得冒起了腾腾的蒸气,空气有些潮湿而且十分清凉。Chris仍然睡得很熟,于是我小心翼翼地爬出了帐篷,站起身来,伸展四肢。
\par 我的腿和背很僵硬,但是并不痛,于是我就做了几分钟柔软体操,把全身放松,然后快步地从圆丘跑到树林里,这样才觉得好多了。
\par 今天早上松树林的气味十分湿重,我蹲下来,在晨曦当中瞭望下面的峡谷。
\par 后来我回到帐篷这边,听到里面有声音,知道Chris醒过来了。我探头进去看他,他正静静地躺着。他一向醒来很慢,在他开口之前几乎需要五分钟的缓冲时间,这时他正眯着眼睛看着太阳。
\par “早啊!”我说。
\par 他没有回答我,从松树上落下几滴雨水来。
\par “你睡得好吗?”
\par “不好。”
\par “那可不妙了。”
\par 他问我,“你怎么会这么早就起来呢?”
\par “不早了。”
\par “什么时候了?”
\par “九点了。”我说。
\par “我敢打赌,我们一直到凌晨三点才睡的。”
\par 三点钟吗?如果他一直到凌晨三点还醒着,那么今天他就要尝到苦头了。
\par 我说,“但是我先睡了。”
\par 他很奇怪地看着我说,“是你害我睡不着。”
\par “我?”
\par “你一直在说话。”
\par “你是指我说的梦话?”
\par “不是,你提到山的事!”
\par 这就奇怪了,“Chris,我一点儿也不知道这些事。”
\par “你昨天整晚都在说,你说,在山顶我们可以看到一切,你说你会在那儿和我相会。”
\par 我想他在做梦,“我现在和你在一起,怎么可能和你在那儿相会呢?”
\par “我不知道,这是你说的,”他看起来十分不舒服,“你听起来好像是喝醉了。”
\par 他还没有完全醒过来,我最好让他自己慢慢起来,但是现在我很渴。这时才记起我没有把水壶带上来,以为路上能找到水喝,真是笨透了。一直到我们爬过山脊才有早餐可吃。下到另外一边才会有一条小溪。于是我说,“我们赶快把行李收拾好上路吧!这样才能找到水做早餐。”天气已经渐渐地热起来了,下午可能会更热。
\par 帐篷很容易就折起来了,我很高兴东西都吹干了,半个钟头之内就收拾好了。除了倒下的小草之外,附近的地上就像没有人来过。
\par 我们仍然有好长的路要走,但是感觉上比昨天早上容易爬多了。我们逐渐接近圆圆的山顶,而山坡也不像昨天那样陡峭。四周的松树林似乎从来没有人砍伐过,地面上已完全看不到阳光,所以也没有任何灌木生长,还有一整片走起来颇有弹性的松针,很适合走路……
\par 现在又该讨论Chautauqua了,要继续结晶的第二道程序,也就是形而上学的部分。
\par 波斯曼的英语系教授在听到Phaedrus的想法之后,提出这样的问题:“没有被界定的良质是否存在于我们观察到的事物之中?或者它只主观地存在于当事者的心中?”这是一个很简单而又十分正常的问题,不需要急着回答。
\par 哈!不需要急着回答,其实它是一个钓饵,是致命的一击——是让你一旦被击倒之后就再难爬起来的问题。
\par 如果良质是一种客观的存在,那你就必须解释为什么科学仪器无法侦测到它的存在;或者你必须提出能够侦测到它存在的科学仪器。如果仪器无法侦测出来,那么很简单,你这种良质的观念完全是在胡说八道。
\par 从另外一方面来说,如果良质是主观的感受,完全存在于当事者心中,那么你所谓的良质只不过是自封的美名。
\par 这位教授提出来的问题其实是一个古老的问题,就是让你落入两难的境地。
\par 两难在希腊文里,原意是指一只凶猛的、正准备攻击人的野牛头上的两只角。
\par 如果他认为良质是客观的存在,那么他就被野牛的一只角刺住了;如果他认为良质是主观的,那么他又被另外一只角刺中了。所以不论他如何回答,他都会被牛角刺住。
\par 他从一些教授的眼中看到善意的微笑。
\par 然而Phaedrus受过逻辑训练,他知道两难的问题并不是只有两种而是有三种严谨的方法足以辩驳。同时他也知道许多并不严谨的反击方法。所以他笑着面对他们。他可以针对左角,反驳所谓的客观暗指的是用科学测量的方法;或者他也可以针对右角,反驳主观暗指的是你喜欢的一切。或者他也可以选择两角之间,否定主观和客观是惟一的选择。
\par 当然他会从三个角度分别进行。
\par 除了这三个符合逻辑的反驳方法之外,同时也有一些非逻辑性的反驳方法。
\par Phaedrus身为修辞学家当然很明白这一点。
\par 你可以把一把沙子丢进公牛的眼里。他已经这样做了,同时还说,对良质的无知就是无能。根据逻辑的推论,发言者的能力和他言论的真假无关,所以无能只是那把沙子而已。天底下最笨的人可以说太阳会照耀,但是这并不表示他会让太阳西沉。而苏格拉底若是活着,会给Phaedrus这样的难题:“没错,我能接受你认为我对于良质无知的假设。
\par 那么现在请你告诉一位无能的老人,良质究竟是什么?否则,我该如何改进呢?”或许Phaedrus会思考几分钟,然后他不得不承认自己也不知道良质究竟是什么。所以以他的标准来说自己也是无能的。
\par 你也可以用唱歌的方式把公牛哄睡。Phaedrus可以告诉质问的人,他对这种两难的问题无法回答,因为远超过他的能力。但是他无法回答并不能证明就没有答案。这些经验更丰富的人不是要帮助他找到答案吗?然而现在用这种方法太迟了。他们只要这样回答:“不行,我们太朴质。除非你能找到答案,否则就按照既定的课程上课,这样下学期我们就不会让你的学生不及格了。”
\par 而第三种解决两难问题的方法,我认为它是最好的,就是根本拒绝回答这个问题。Phaedrus可以这样说:“想划分良质是主观还是客观,就是要去界定它。
\par 我已经说过它是无法被界定的。”然后就不必去解决这个问题,我相信狄威斯肯定这样劝过他。
\par 为什么他没有接受这种建议,而选择用逻辑和辩证的方法回答我不知道,但是我可以推测出来。我想他认为整个理性教会属于逻辑的范畴,如果他拒绝接受从逻辑的角度去讨论这个问题,无异于自绝于任何学术的讨论之外。哲学上的神秘主义认为真理是无法界定的,自有历史以来就存在,只能通过非理性的方式了解。这就是禅的根基,但是这并不属于学校研究的范围。而学校这座理性教会主要就是研究那些能被界定的事物,所以一个人如果想研究神秘的主义,他就应该去修道院而不是去大学,大学要研究的是能够形之于文字的事物。
\par 我想另外一个他接受这个问题的原因是他的骄傲。他知道自己在逻辑和辩证方面功力深厚,他把这个两难的问题当作是一种挑战。然而这种骄傲自大的心态引发了他所有的问题。
\par 在前方两百码远的地方,我看到有一只鹿在动,鹿在我们上方的松树林里,我想要指给Chris看,但是一瞬间它就不见了。
\par Phaedrus的第一个像牛角一样的难题是:如果良质的确是客观的存在,为什么科学仪器总是无法探测出来呢?
\par 这只牛角非常卑鄙,一开始他就知道它的杀伤力有多么强。如果起初他就假定自己是超级的科学家,能够看出其他的科学家看不到的客观事实,那么无异于是想证明自己是疯子或是笨蛋,甚至兼而有之。因为在现今的世界里,和科学相抵触的思想是无法站住脚的。
\par 他记得洛克曾经说过,不论是否属于科学范畴,你只能了解一个事物的良质而非其他。这个无法驳倒的真理似乎认为,科学家之所以无法侦测出良质,是因为良质就是他们所侦测出来的全部。客观的事物就是一种理性的产物,是从许多性质当中推演出来的。如果这个答案成立,自然就破解了这个难题。
\par 这使他兴奋了好一阵子。
\par 但是这个答案最终证实并不成立。
\par 他和学生在教室里观察到的良质和在实验室里观察到的颜色、温度、硬度的性质是不同的。那些物理性质都可以借用仪器测量,而他的良质——卓越、价值、善——却不属于物理范畴,所以无法测量。他被良质这个字眼的模糊特性困住了。他奇怪为什么会有这种现象,于是记下来,要研究这个词的历史根源,然后把它暂时搁置。牛角的难题仍然存在。
\par 于是他转而注意另外一个有可能反驳的难题。所谓的良质只是你所喜好的事物吗?这么说使他十分愤怒。历史上的伟大艺术家如拉斐尔、贝多芬、米开朗琪罗,他们只是把人们喜好的事物表达出来。他们人生最重要的目标只是用深刻的方法引导人们的感觉。是不是就是这样?这么说让他愤怒。然而更让他生气的是,他没有办法立刻推翻这种看法。所以他小心谨慎地研究这句话,就像他在反击之前,一定会仔细反覆地思考。
\par 然后他找到症结了。他拿出刀来,把使人愤怒的那个词挑了出来,那就是“只是”这字眼。为什么良质只是你所喜好的事物呢?为什么“你所喜好的”是“只是”呢?在这种情况之下,“只是”究竟是什么意思呢?经过这样反覆的思考之后,他认为,“只是”在这种状况之下并没有任何意义。“只是”是一种轻蔑的口吻,对这个句子的分量毫无贡献。
\par 如果把这个词拿掉,整句话就变成良质就是你所喜好的。它的意义完全改变了,变成不具杀伤力的事实。
\par 他在想为什么这句话一开始就强烈地激怒了他,听起来似乎非常自然,为什么他花了那么多的时间才知道它真正的意思。这句话实际是在说:“你的喜好是不好的,最起码是不重要的。”在这句自以为是的假设之下暗示的是,让你快乐的事是不好的,最起码是不重要的。
\par 这正是他全力加以反击的朴质之精髓。
\par 大人训练小孩子不可以做他们喜欢的事,但是……但是什么呢?当然!要去做别人喜欢的事。而别人是指谁呢?父母、老师、督学、警察、法官、上司、国王、独裁者,这些都是在上的权威。一旦你被训练得轻视自己的喜好,那么当然你就会对别人更加顺服——变成好奴隶。
\par 一旦你学会不做自己喜欢的事,那么你就会为整个体系所接受。
\par 但是假设你去做你喜欢的事呢?难道这就表示你会跑出去把英雄给射杀了?去抢劫银行?或是强暴老妇人吗?
\par 劝你不要做自己喜欢的事,等于这个人在作一种大胆的假设,他似乎不了解,别人考虑过抢银行的后果之后,很可能就不喜欢去抢银行了。他不明白银行存在的首要理由就是因为它是人们所喜好的,因为银行能够提供融资贷款。于是Phaedrus开始思考,为什么社会很自然地反对你做自己所喜好的事。
\par 结果他有许多意外的发现。当别人说不要做你喜欢的事,并不只是表示要顺从权威,还有其他的含意。
\par 其他的含意代表的是深厚的古典科学的信念:为什么你所喜好的是不重要的?因为它来自于非理性的情感。他研究这个论点好长一段时间,然后把它切割成两部分,他称之为科学的物质主义和古典的形式主义。他说这两者往往在同一个人身上出现,但是理论上却是分开的。
\par 科学的物质主义出现在对科学感兴趣的一般人身上的次数,远比出现在科学家身上的为多。他们认为,能由科学仪器测量的物质和能量才是真实的,其他的都不真实,或者最起码不重要。你所喜欢的事是无法用科学仪器衡量的,因此就不真实。你喜欢的可能是一个事实,也可能是一种幻觉,感觉无法分辨这两者。科学方法的主要目的就是要分辨真假,然后消除主观、不实、想像的因素,进而得到事实、客观而且真实的一面。当他说良质是主观的,也就是说良质是想像出来的。因而从严格考量事实的角度来讲,应该摒弃。
\par 另外一面则是古典的形式主义,也就是认为无法通过理智了解的事就不存在。良质在这种情况之下是不重要的,因为这是一种不能被理智分析的情感认知。
\par 通过以上两种看法,Phaedrus认为,第一种科学的物质主义很容易推翻。他由早年的教育知道,这是一种天真的科学理念,于是他用归谬法找出它的矛盾之处。这种方法的基础在于,如果前提是荒谬的,那么结论也是荒谬的。首先让我们研究,凡是无法测知能量的就不存在或不重要,这种说法是否正确。
\par Phaedrus以数字零为例,零原是印度数字,在中世纪的时候由印度传到西方世界,所以古希腊罗马人不知道有零的存在。这是怎么回事儿呢?他不禁怀疑是否自然界将零隐藏得这么好,以至于数以百万计的希腊罗马人都没有发现它的存在。一般人很可能认为零原本就在那儿,所有的人都可以看到。他揭示出,认为零具有极大的能量是荒谬的。然后他指出,这是否就表示零是不科学的呢?如果是不科学的,那是否就表示现在完全根据零和一运算的电脑,就应该改成只用一来运作的呢?很快地,我们就发现了其中的矛盾。
\par 于是他又提到其他的科学观念,一个一个地揭示它们都无法脱离主观的考量而存在。他以重力法则结束,也就是在我们旅行的第一天晚上,我给John、Sylvia以及Chris举的例子,如果主观被视为不重要的,那么整个的科学体系也会随之瓦解。
\par 这种对于科学的物质主义的攻击,似乎将他归入了哲学理想主义的阵营——贝克莱、休谟、康德、费希特、谢林、黑格尔、布拉德利、博桑基特——全都是些伟大的人物。但是我们很难用普通的言语证明这在他对良质的辩护上是有害还是有益。唯心论的说法虽然可能在逻辑学上比较合理,但是在修辞学上却不然。对大一作文来说,这个主题实在太枯燥,而且十分困难,他们确实无法理解。
\par 从这个角度来看,主观的难题和客观的难题几乎都一样缺乏新意,古典的形式主义甚至更糟。这些论点都必须将整个理性的背景纳入考量,而不应该单单因感情的冲动而立刻做出反应。
\par 大人教小孩:“不要把所有的零用钱都拿去买泡泡糖(孩子情感的冲动),因为要留做以后之用(理性的背景)。”大人明白,“这间造纸厂即使有最好的防治污染系统,依然会有恶臭(情感的反应),但是如果没有它,整座城就会瓦解(理智的背景)。”根据我们古老的二分法,上面所说的就是,在你作决定的时候不要因为表面上浪漫的诉求,而不去思考它古典而且根本的理由。这一点他算是勉强同意的。
\par 而古典的形式主义者之所以反对“良质只是你所喜好的事物”,是因为他们认为,他所提倡的主观而无法定义的良质只是表面浪漫的诉求。在教室里,针对文章的投票可以立刻决定这篇文章是否得到认可,但是这是否就是良质呢?是否良质就是你所看到的或是比这个更微妙呢,所以你很可能无法立刻发现它,而是在好长一段时间之后才明白。
\par 他愈检查这个论证它愈显得难以应付。这看来好像会是他整个论文所论述的。
\par 是什么使它出现了坏兆头?那似乎是课堂中经常提起的问题,而他总是必须多少有点诡辩式地回答它。这就是问题所在。如果每个人都知道良质是什么,为什么对它会有这么不一致的意见?
\par 智者的答案总是那样,虽然纯粹的良质对每个人都一样,但是体现良质的本原却是人人各异的。只要他不对良质加以定义,也就无法就此争论,但是他自己知道,而他也知道学生会晓得,其中有种错误的意味。它并没有真正解答问题。
\par 现在有另外一个解释:人们对良质意见不同是因为有些人只是用他们当下的情绪,而其他人则是应用他们整体的知识。他知道在一群英语教师的受欢迎度评比中,能支持他们权威的后一论证会取得压倒性的拥护。
\par 但是这个论证完全是毁灭性的。曾经只是一个单独的统一的良质,现在则似乎变成了“两个”,浪漫的一个,只是看,是学生所拥有的;而古典的那一个,全体的了解,是老师所拥有的。一个基础的和一个平直四方的。平直四方者并非良质之阙如,它是古典的良质。基础者亦并非良质之出世,它只是浪漫的良质。他所发现的基础者与平直四方者之间的裂缝仍在那里,可是良质似乎并不完全落在裂缝的任何一边,如他先前所假设的一般。相反地,良质本身裂成两种,裂缝两边各有一种。他的简单的、整齐的、美丽的、未加定义的良质正开始复杂起来。
\par 他不喜欢这种进行方式。裂缝这一术语本来打算用于综合古典及浪漫地看待事物的方式,但其自身已经断裂成两部分,不能再综合任何事物。它已经被分析的捣碎机所虏获了。主观性和客观性的刀刃已将良质一分为二,而且凭借一个实际概念消灭了它。如果他想挽救,就不能让那刀刃靠近它。
\par 而事实上,他所谓的良质并不是古典的良质或是浪漫的良质。它超越两者之上,既不属于主观,也不属于客观,它超出了这两个范畴之外。事实上,整个主客观以及唯心、唯物与良质之间的关系是不平衡的。因为唯心、唯物的争论已经出现了几百年,它们只是用这个争论把良质拖下水,他如何能够断定良质究竟是唯心还是唯物呢?从一开始唯心、唯物就没有很清楚的分野。
\par 如此一来,他摆脱了左角。良质不是客观的,它不存在于物质的世界。
\par 然后他也避开了右角,良质也不是主观的,它不单单存在于人心之中。
\par 最后,Phaedrus进入了西方思想史上从未有过的境地。那就是主客观这两只角之间的区域。他认为良质既不属于人心的一部分,也不属于物质。它将独立于这两者之外。
\par 有人在Montana立大学的大厅里听到他在楼梯上和走廊里轻声地哼着:“圣哉,圣哉,圣哉……三位一体的存在。”
\par 我隐隐约约地想起,很可能记错了,也可能是我自己想像出来的,那就是他让整个思想的结构维持了好几个礼拜,而不再进一步地探讨。
\par Chris大叫:“我们什么时候才会爬到山顶?”
\par 我回答:“可能还有好长的一段路。”
\par “我们会看到很多东西吗?”
\par “我想会吧!看看树之间的蓝天。
\par 只要我们看不到天,就还有好长一段路。
\par 爬到山顶的时候,很自然地就会看到蓝天。”
\par 昨天晚上的雨把地上的松针都浸湿了,踏上去十分舒服。有时候山坡上会有一层这样的松针,如果是干的就会很滑,你必须踏稳每一步,否则就容易滑倒。
\par 我跟Chris说:“这儿什么灌木丛都没有,不是很棒吗?”
\par “为什么没有呢?”他问我。
\par “我想这里从来没有人砍伐过,一旦几百年下来都是这样,大树就会遏止所有灌木的生长。”
\par Chris说:“这里就像是公园一样,你向四处望都是一片空旷。”他的情绪似乎比昨天好很多。我想接下来的一段路,他会走得很好。森林里面的寂静会让每一个人都有所进步。
\par 根据Phaedrus的见解,这个世界是由三种事物所组成的,就是心、物和良质。
\par 一开始他并没有因为没能在它们之间建立任何关联而苦恼。假如心与物之间的战争已经持续了好几百年尚且没有得到解决,为什么他的发现要在短短的几个礼拜之中骤下结论呢?所以他暂时把它搁在一边,放在心灵的架子上。在那儿有许多他一时找不到答案的问题,他知道这三者之间的关系迟早会建立起来。
\par 但是现在不用着急,他只想好好地放松一下,因为刚刚避开了这两难的处境。
\par 然而继续研究下去,他发现,虽然现在暂时没有任何理论能推翻这种说法,但是这种三位一体的状况仍很特殊。
\par 一般哲学家研究的可能是一元论,比如说像上帝,他是这整个世界惟一的解释。
\par 或者研究的是二元论,将万事万物分成心与物。也可能研究的是多元论,把它的源头归于无限多的来源。但是三是一个很奇怪的数目,你立刻就会想知道,为什么会是三呢?它们之间的关系如何呢?Phaedrus一旦休息够了,也对这种关系十分好奇。
\par 他强调,虽然你可以把良质与物体连在一起,但是良质的感觉仍然可能单独出现。这导致了一开始他认为良质是全然主观的看法,但是主观的感受并不是他所谓的良质,良质反而会减低主观性,良质使你能跳出自己,让你意识到周围的世界。良质和主观是对立的。
\par 我不知道他得到这个结论时,思考过多少事物,但是最后,他认为良质不会单独与主观或客观发生关系,而是只在这两者产生关系的时候才会出现,也就是说在主观和客观交会的一刹那。
\par 听起来很顺耳。
\par 良质并不是一种物体,它是一种事件。
\par 更顺耳了。
\par 它是主观意识到客观的存在时所发生的事件。
\par 因为没有客观就无所谓主观。因为客观会让主观意识到自己的存在——所以良质就是同时意识到主客观存在时所发生的事件。
\par 他的看法愈来愈精辟。
\par 现在他知道就快到了。
\par 这表示良质不仅仅是主体和客体相遇所产生的结果,它们是由良质这事件所产生的,良质是主体和客体的因,过去大家误以为主体和客体才是因。
\par 他写道:“良质像一个太阳,它并不是绕着我们的主体和客体运转。它不是被动地照亮它们。它也没有隶属于它们。
\par 主体和客体是由它所创造的,它们才是隶属于它的。”
\par 当他写下这段话的时候,他知道,经过这么多年来的努力,他终于到达了思想上的一个高峰。
\par Chris大叫:“天空。”
\par 就在我们上方,在树干之间有一道窄窄的蓝天。
\par 我们走得更快了,而那一道蓝天变得愈来愈宽阔,然后树越来越稀疏,我们看见空旷的山顶。在离山顶还有五十码的时候,我说:“让我们跑过去吧!”
\par 于是我们把剩余的精力一股脑地全部散放出来。
\par 我奋力地跑,但是Chris很快就赶上我,然后超过我,而且还一直不停地笑。背着这么沉重的行李,高度又这么高,在这种情况下,我们并不想创造任何纪录,只是尽可能地发挥出自己的能力。
\par Chris先到达山顶,而我正从树林中冲出来,他举起手臂大声喊着:“我赢了!”
\par 十分自我的人。
\par 我喘得很厉害,跑到的时候几乎不能说话。我们把背包卸下来,然后靠着几块石头坐下来。地表已经被太阳晒干了,但是下面还有昨天晚上下雨所造成的泥浆。在我们下方,离这片森林几英里远的地方是加勒廷河谷。在河谷的一角则是波斯曼。有一只蚱蜢从石头上跳起来,然后飞到离我们有好远一段距离的树顶上。
\par Chris说:“我们成功了。”他非常高兴。我还是喘得很厉害,无法回话。
\par 于是我脱下了靴子和袜子,流汗太多,它们已经湿了。然后把它们放在一块石头上晒干。我静静地看着它们冒起了一阵烟,径自想着自己的事。
\subsection*{20}
\par 很显然我睡着了,太阳正在头顶上,现在只差几分钟就到十二点了。我看了看石头的另一边,Chris也睡着了。就在他上方,树林已经不见了,只剩下岩石和一堆堆的积雪。我们可以从山脊背面直接爬上去,但是接近山顶的时候会很危险。我看了好一会儿山顶,Chris说我昨天晚上告诉了他什么来着?我们会在山顶见面……不是……我们会在山顶相会。
\par 我一直和他在一起,怎么会和他在山顶相会呢?这一点真的很奇怪。他说前天晚上我还告诉他一些别的事,这里很寂寞。这和我想的完全不同,我一点也不觉得这里很寂寞。
\par 落石的声音让我注意到山的另外一面,那儿没有任何东西在动,纯粹的寂静。
\par 没有关系,你经常会听到小落石的声音。
\par 有的时候落石并不小。在雪崩之前就会先落石,如果你在雪崩之上,或是在它们旁边,那么看雪崩是一件很有意思的事。但是如果它们在你上面,那么赶快逃命吧!你必须注意什么时候会雪崩。
\par 人睡着的时候往往会说些奇怪的事,但是为什么我告诉他会在山顶相会呢?为什么他认为我醒着呢?这实在很奇怪,让我觉得很不安。你得先有感觉才会去思索原因。
\par 我听到Chris在动,转头看到他正向四周观望。
\par 他问我:“我们在哪里?”
\par “在山脊的顶端。”
\par “噢!”他说着笑了笑。
\par 我拿出奶酪和饼干,然后小心地把奶酪切成薄片。周围的宁静可以让你把事情做得很漂亮。
\par 他说:“让我们在这儿盖座小屋吧!”
\par “哦!”我说,“每天都要爬上来吗?”
\par 他开玩笑地说,“当然啦!爬上来并不难!”
\par 昨天在他的记忆中已经很遥远了,我拿了一些奶酪和饼干给他。
\par 他问我:“你一直都在想什么呢?”
\par 我回答他:“各种事情。”
\par “什么事情?”
\par “大部分的事情对你来说没有任何意义。”
\par “比如说?”
\par “比如说,为什么我会告诉你我们会在山顶上相会呢?”
\par 他说了一声“喔”就低下头去。
\par “你说我好像喝醉了。”我告诉他。
\par 他说:“不是喝醉了。”他的头依然垂着。他这种表情让我再次怀疑他是否在说实话。
\par “那么又是怎样了呢?”
\par 他没有回答我。
\par “Chris,到底是什么样的情形?”
\par “就是不一样嘛!”
\par “怎么个不一样呢?”
\par “这个我不知道!”他抬起头来看着我,眼睛里闪过一丝恐惧。“就像你很久以前那个样子。”他说完又低下头去。
\par “什么时候?”
\par “我们住在这里的时候。”
\par 我不动声色,不让他知道我心里的变化。然后我小心地站起来,走到石头边,很技巧地把袜子翻过来。它们早已经干了。当我把袜子拿回来的时候,他仍然望着我,我很平淡地说:“我不知道我说话的声音和以前不一样。”
\par 他没有回答。
\par 我把袜子穿上,然后套上靴子。
\par Chris说:“我渴了。”
\par “下面不远就可以找到水喝,”我说着站了起来,看了一会儿山顶上的雪,然后说,“你准备好下山了吗?”他点点头,于是我们把背包背起来。
\par 我们沿着山脊朝着一条溪谷的源头走去,这时我们听到一阵落石的声音,比刚才的声音大多了。我抬起头,想看看究竟是从哪里落下来的,但是毫无所获。
\par Chris问我:“怎么回事儿?”
\par “落石。”
\par 我们两个人停下来听了一阵子,Chris问我:“那里有人吗?”
\par “没有,我想只是融雪把石头松动了。在初夏的时候像这么热,通常会有一些小落石,有的时候也会落大的石块,这是山老化的过程。”
\par “我不知道山也会疲劳。”
\par “不是疲劳而是老化。它们会变得愈来愈浑圆。这里的山才刚刚开始老化。”
\par 现在除了山顶,我们的四周都覆盖着一层墨绿色的森林,远远地望去,森林好像柔软的天鹅绒一般。
\par 我告诉他,“你看这些山,现在看起来这样宁静,仿佛会永远在那儿。但是它们其实一直在改变,而这些变化往往并不安宁。就在我们脚底下,有一股力量可以把这整座山撕开。”
\par “发生过吗?”
\par “发生过什么?”
\par “把整座山撕开。”
\par 我说,“发生过。”然后我记起来,“就离这里不远,有十九个人被百万吨的岩石给活埋了。每个人都很惊讶只有十九个人。”
\par “是怎么回事?”
\par “他们是从东部来的观光客,到这里露营,晚上的时候地突然裂开了。第二天早上救助人员赶到的时候只能摇摇头,甚至连挖都没有挖,因为他们需要从几百英尺的岩石底下把尸体挖出来,然后还要再埋葬一次。所以干脆让他们留在那儿,现在还在那里。”
\par “他们怎么知道是十九个人呢?”
\par “根据城里的亲戚邻居通报失踪的人口知道的。”
\par Chris看着我们眼前的山顶,“事前难道没有任何警告?”
\par “我不知道。”
\par “你认为会有警告吗?”
\par “可能有。”
\par 我们走到溪谷的源头,发现可以沿着这条溪谷下去,然后就能找到水喝。
\par 从上面又落下一些石头来,突然间,我有些害怕起来。
\par 我叫着:“Chris!”
\par “什么事?”
\par “你知道我在想什么吗?”
\par “不知道,你在想什么?”
\par “我想我们现在最聪明的做法是不要去爬那个山顶,以后夏天再来爬。”
\par 他沉默下来,然后说,“为什么呢?”
\par “我预感情形不妙。”
\par 他好一阵儿不说话,最后他说,“会发生什么呢?”
\par “我想很可能会碰到风雪,或是摔下去什么的,那么我们就真的有麻烦了。”
\par 他又好久不说话,我抬起头来,看见他脸上失望的表情。我想他知道我有一些话没有说出来。“你先想一想,”我说,“然后等我们到达水源边吃午餐的时候,我们再作决定。”
\par 我们继续往下走。“好吗?”我说。
\par 他终于很不情愿地说,“好吧!”
\par 现在下山容易多了。但是我知道很快就会变得更陡,这里仍然是一片蓝天和烈日,不久我们又会进入森林里。
\par 我不知道该怎么解释,晚上说的那些古怪的话,给我的感觉很不好,对我们两个人都一样。似乎到了晚上,这一切旅程、露营、Chautauqua和让人怀念的老地方对我都有不好的影响。所以我想尽快离开这里。
\par 过了一会儿,这里的高度变得有点儿令人毛骨悚然,我想要赶快下去,下到底下去。
\par 一直到海里去,这样就对了。在那儿,海浪慢慢地翻滚着,你可以听到海涛的怒吼,而你不会再落到任何地方,因为你已经在底部了。
\par 现在我们又进入林子里,山顶已经被树枝挡住了,这样我很高兴。
\par 我想在这一次Chautauqua之中,我们已经走得像Phaedrus一样地远,现在我想离开他的路。我已经忠实地把他的思想记下来了,现在我想发展自己的想法,就是他忽略的一面。这一次Chautauqua的主题是“禅与摩托车的维修艺术”,而不是“禅与爬山的艺术”。在山顶没有摩托车,也很少有禅。禅是山的精神,而不是山顶。
\par 你在山顶发现的禅,就是你把它带上去的。让我们离开这里吧!
\par “下山的感觉真好,不是吗?”我跟Chris说。
\par 他没有回答我。
\par 我想很可能我们之间有一些摩擦。
\par 你千辛万苦爬到山顶,得到的却只是一块大石板,上面有一堆戒条。
\par 对他来说,情形就是这样。
\par 以为他自己是他妈的弥塞亚。
\par 我可不是。我们花的时间太多,但是收获太少。让我们离开吧!离开吧……
\par 于是我两步并作一步地跳下山去,直到听见Chris叫我,“慢一点儿… … ”然后我才发现他已经在我身后几百码的林子里。
\par 于是我慢下来,但是过了一会儿,我发现他故意在后面慢慢地走,当然,他很失望。
\par 我想,在Chautauqua里我应该做的是,把Phaedrus所走的方向简明地指出来,而不要加以任何的价值判断,然后再发展我自己的思想。相信我,如果我们不从二元化的角度去看事情,而是从良质、心和物三位一体的角度,那么摩托车维修的艺术以及其他的艺术都会产生前所未有的意义。John夫妇所逃避的科技怪物就不再面目狰狞,而成了很有意思的东西了。要把这一点解释清楚,是一项漫长而有意思的工作。
\par 但是首先我要提出的是:或许他也会赞同我走这样的方向,也就是深入每天的日常生活之中。我认为如果能够改善人们每天的生活,那么哲学就是好的,否则宁可忘掉它。但是他并没有朝这个方向走。
\par 他曾经推演良质与心、物之间的关系,而确认良质是心、物的根源。如果没有经过仔细的解说,这种发现就会像哥白尼的发现一样,完全扭转了别人对这三者之间关系的认定,听起来似乎有些神秘,但是他并不希望如此。他的意思只是,在认知一项物体之前,必然有一种非理智的意识,他称之为良质的意识。在你看到一棵树之后,你才意识到你看到了一棵树。在你看到的那一刹那以及意识到的那一刹那之间,有一小段时间。我们常认为这一段时间不重要,但是并没有证据显示这一段时间不重要——情形完全不是如此。
\par “过去”只存在于我们的记忆之中,“未来”则存在于我们的计划之中,而只有“现在”才是惟一的真实。你理智上所意识到的那棵树,由于这一小段的时间的关系,便属于过去,因而对你来说并不真实。任何经由思想所意识到的总是存在于过去,因而都不真实。所以真实总是存在于你所看到的那一刹那,且在你还没有意识到之前。除此之外,没有别的真实。这种在意识之前的真实,就是Phaedrus所谓的良质。由于所有经由思想所认知的事物必须来自于这一段思考前的真实,所以良质是因,而果才是所有的主体以及客体。
\par 他认为知识分子最难了解这种良质,因为他们反应过快,立刻将一切化成思考的形式。而最容易看见良质的是儿童以及未受过教育的人,还有丧失文化的人。他们很少受到文化的影响,因而较少接受正规的训练,没有让文化渗透他们的心灵。他认为,这就是为什么朴质是一种独特的知性上的疾病。他发现由于学校教育在他身上的失败,使他很偶然地拥有了对这种疾病的免疫力。
\par 最起码在某种程度来说,不具有这种习惯。之后他很自然地就不认同知性,也同情那些反知性的教条。
\par 那些拥有知性成见的人,通常认为良质这个知性前的真实毫不重要,只不过是客观事实和主观意识之间一段短暂的时间。由于他们早已认定它不重要,所以不会去研究它和知性的观念有何不同。
\par 他认为的确不同。一旦你听过良质的声音,看到那面韩国的墙,以及非知识实体最纯粹的形式,你就会想要把所有那些文字玩意儿忘掉,因为你已发现一片新天地。
\par 现在他有这一套三位一体的新理论支持,就阻止了浪漫与古典之间的分裂。
\par 这种分裂差一点把他给毁了。它们再也不能把良质肢解,而他可以轻松地坐在那儿,把“它们”肢解。浪漫的良质总是与视觉的印象相结合,而理智的良质总是需要一段时间的考量。浪漫的良质是指此时此地的事情,而古典的良质则超越此刻,必须考虑现在与过去和未来的关系。比如说,从浪漫的观点来看,如果摩托车此刻仍然正常行驶,为什么要替它操心呢?如果你从古典的角度去看,现在只不过是过去与未来之间的一瞬间,忽略过去和未来对现在的影响,就不是好的良质了。摩托车现在可能正常行驶,但是最近什么时候检查过油表呢?从浪漫者的观点来看,这样想有些大惊小怪,但是对古典的人来说却是常识。
\par 现在我们有两种不同的良质,但是它们不再把良质分裂,它们只是存在于不同时间的两种良质。前面所提到形而上学的等级我们可用图表示如下:真实
\par 客观(物质) 主观(精神)
\par 浪漫(感情) Phaedrus提倡的良质古典(理智) Phaedrus应该提倡的良质
\par 而现在他所建立的形而上学体系则是这样的:良质(真实)
\par 古典的良质(知性的真实) 客观的真实(物)
\par 主观的真实(心) 浪漫的良质(知性前的真实)
\par 他所提出来的良质不只是真实的一部分而已,它是真实的全部。
\par 于是他根据三位一体的概念回答这样的问题。为什么每一个人所看到的良质都不同?以前这样的问题他都要花极大的篇幅回答,现在他这样回答:“良质无形、无状,也无法形容,看得到形状和形式就是由理性去认知。良质是超越形状和形式之上的,我们给良质的名字、形状和形式只有部分基于良质自身,另一部分则是基于我们由经验中得出的印象。我们经常在良质中寻找与我们过去经验相似的东西。如果我们找不到就无法行动。我们也是根据这些东西建立起语言和整个的文化体系。”
\par 他认为每个人看到的良质都不同的原因是,每一个人都有不同的背景。他以语言为例,我们听印度的语言,因为没有和他们相似的背景,所以无法分辨那些音节的差异。同理,大部分说印度系语言的人,也不能分辨英语中的某些词语有何不同。所以他认为对印度的村民来说,看见鬼魂是很正常的现象,而要他们明白重力定理却十分困难。
\par 他认为这就说明了,为什么大一新生衡量文章的等级有相同的标准,因为他们都有相似的背景、相似的知识。但是如果班上有一群外国学生,或者研读很难了解的中古诗文,那么学生在评断良质的等级时就可能出现极大的差异。
\par 所以,在某种程度上,是学生对于良质的选择限制了他。人们对于良质有不同的看法,并不是因为良质本身有差异,而是每一个人的经验背景不同。所以他推测,如果两个人有完全相同的背景,那么他们眼中的良质也会完全相同。
\par 然而我们无法进行这种试验,所以这仍然只是一种推测。
\par 他是这样回答同事的:“要想从哲学方面解释良质,是一件既对又错的事,因为这是一种哲学的解释。哲学解释的过程就是分析,把一样东西细分成主语、述语。我的意思是(以及每一个人的意思都是如此),良质这字眼不能分解成主语和述语,这不是因为良质是神秘的,而是因为良质是非常简单、迅捷而直接的感应。
\par “要让我们这种背景的人了解纯粹的良质,用最简洁的语言形容就是,‘良质是有机体对环境的反应(他用这种语言,因为质问他的人习惯用刺激和反应的行为理论衡量事物)’。如果你把一只阿米巴原虫放在一盘水里,然后在附近滴一滴硫酸,我想它会避开它,即使它不知道硫酸是什么。要是它能开口说话,它也会说:‘这个环境很恶劣。’如果它有神经系统,就会用非常复杂的方式去克服恶劣的环境,于是它会从过去的经验当中寻求解释,以认定目前这种环境是不适合它的,因而对环境算是了解了。
\par “然而在我们高等复杂的生物体对环境的反应上,我们发明了许多事物,包括天、地、树、石头、海洋、神明、音乐、艺术、语言、哲学、工程、文化和科学,我们称这些为模拟实体。而它们的确是真实的,所以我们就说服孩子相信它们是真实的。然后把不接受这套理论的人丢进疯人院里,但是让我们发明模拟实体的是良质。良质就是环境给我们不断的刺激,让我们创造所居住的世界,包括所有的一切,以及其中的一点一滴。
\par “如果我们想用我们所创造的世界去涵盖、去刺激我们创造的源头,这是不可能的。这就是为什么良质无法被界定,如果我们去界定,我们所界定的就无法涵盖整体的良质。”
\par 这一段话是我记忆当中最深刻的一段,很可能是因为它最重要。他写的时候有些震惊,想要把“包括所有的一切,以及其中的一点一滴”涂掉,我想他认为这是一种疯狂的念头,但是他找不到任何理由剔除这些字眼。现在害怕已经为时太晚了,于是他只好忽略这一警告,依然把这几个字保留下来。
\par 他放下铅笔……他觉得仿佛有东西释放出来了,原先被禁锢得太痛苦,然而发现时已为时太晚。
\par 他开始明白,他已经偏离了最初的见解,他不再讨论形而上学的三位一体,而是绝对的一元论。良质是一切的源头和本质。
\par 这时他的心中充满了全新的哲学思潮。黑格尔在他的唯心论当中也提过这一点,他认为人心超越在主客观之外。
\par 但是黑格尔认为人心是一切的源头,而把浪漫的经验剔除掉,所以他的世界是完全古典、理智和有秩序的。
\par 良质却不是这样。
\par Phaedrus记得黑格尔曾经被视作东西方哲学的桥梁,东方的吠陀、道家、佛家的思想都被视为绝对的一元论,与黑格尔的哲学相似。Phaedrus当时怀疑神秘主义和一元论是否可以互相转换,因为神秘主义没有规则可循,而一元论也是如此。而他所提出的良质是形而上学的实体,并不是神秘主义。或者也是神秘的——有什么差异呢?
\par 他给自己的回答是,两者之间的差异在于定义。形而上学的实体都有定义而神秘的思想却没有,因此良质是神秘的。不!实际上两者都有。虽然他一直从形而上学的角度去思考,把它当作形而上学的问题,但是他一直拒绝去界定它,因而使它带有神秘的特质。正是由于无法界定,反而使它脱离了形而上学的局限。
\par 然后Phaedrus心血来潮地走到书架旁,拿起一本蓝色精装的小书,许多年前,他把这本书整本印下来,然后自己装订成册,因为他在书店里根本买不到。
\par 这是两千四百年前的老子的《道德经》,他开始念其中已经念过许多遍的经文,但是这一次他想从中找到某一种替代品,于是他一边念一边解释,他这样念:道可道,非常道。名可名,非常名。
\par 无名天地之始,有名万物之母。故常无欲以观其妙,常有欲以观其徼。此两者同出而异名。同谓之玄,玄之又玄,众妙之门。
\par 道冲而用之或不盈,渊兮似万物之宗。挫其锐,解其纷,和其光,同其尘,湛兮似或存。
\par 吾不知谁之子,象帝之先。
\par ……根绵绵若存,用之不勤。……
\par 视之不见名曰夷,听之不闻名曰希,搏之不得名曰微,此三者不可致诘,故混而为一。其上不皎,其下不昧。绳绳不可名,复归于无物。是谓无状之状,无物之象,是谓惚恍。迎之不见其首,随之不见其后。执古之道,以御今之有,能知古始,是谓道纪。
\par (The quality that can be defined is not the Absolute Quality.The names that can be given it are not Absolute names.It is the origin of heaven and earth. When named it is the mother ofall things…Quality (romantic Quality) andits manifestations (classic Quality) are in their nature the same.It is given different names (subjects and objects) when it becomes classically manifest. Romantic quality and classic quality together may be called the “mystic.” Reaching from mystery into deeper mystery, it is the gate to the secret of all life. Quality is all pervading. And its use is inexhaustible!
\par Fathomless! Like the fountainhead of all things… Yet crystal clear like water it seems to remain. I do not know whose son it is. An image of what existed before God…. Continuously, continuously it seems to remain. Draw upon it andit serves you with ease… Looked at but cannot be seen… listened to but cannot be heard… grasped at but cannot be touched… these three elude all our inquiries and hence blend and become one. Not by its rising is there light, Not by its sinking is there darkness unceasing, continuous. It cannot be defined and reverts again into there alm of nothingness. That is why it is called the form of the formless. The image of nothingness. That is why it is called elusive. Meet it and you do not see its face. Follow it and you do not see its back. He who holds fast to the quality of old is able to know the primeval beginnings, which are the continuity of quality.)\footnote{老子《道德经》英语部分经作者融入个人见解,较中文原文更易明白,是故需列英语部分辅助说明}
\par Phaedrus一句一句地念,一行一行地念,发现它们正符合他的意思,只不过他表达得很僵化,而《道德经》中却说得非常清楚而准确,这就是他一直想说的,只是此刻却从不同的背景,用不同的语言说出来。他从另一座山谷看到这一座山谷的景象,他所说的不是陌生人所讲的故事,他本身也是山谷的一部分。
\par 他看到了一切。
\par 他已经破解了密码。
\par 他一行一行地继续读下去,一页一页地读下去,其中没有任何不合之处。
\par 他所提倡的良质就是这里所谓的道,是所有宗教的原创力,不管是东方或是欧美,不管是过去还是现在,是一切的知识,是所有的一切。
\par 然后他睁开心眼,看见他自己的形象,发现了自己身在何处,他看到了什么……我不知道究竟发生了什么事……Phaedrus先前感受到的滑动,内心的崩解,突然之间冲力越来越大,就像山顶滑落的岩石一样。在他能阻挡之前,突然累积起来的意识开始逐渐膨胀,一直膨胀到失去了控制,于是雪球愈滚愈大,远远超过它原先的体积,直往山下滚去,一直到山上不曾留下一物。
\par 一切都消失了。
\subsection*{21}
\par Chris说,“你不太勇敢,对不对?”
\par “对。”我拿起一条意大利腊肠,用牙齿撕去外皮,“但是你对我的智慧十分佩服。”
\par 现在我们已经离山顶有好一段距离了,松树林和灌木丛比峡谷另外一边高多了也密多了。很显然,这儿下的雨比较多,我拿起水壶,喝了一大口,是Chris从溪边取回来的水。然后我看了看他,从他的表情可以看出,他已经接受了下山的提议,所以不必再和他争论了。
\par 我们中午只吃了一些糖果,然后喝了一些水把它们冲下去,然后躺在地上休息了一阵子。山上的泉水是世界上最美味的饮料。
\par 过了一会儿,Chris说:“我现在可以背重一点儿的东西了。”
\par “你确定吗?”
\par “当然。”他有一点骄傲地说。
\par 我很感激地把一些较重的行李交给他。于是我们背上背包站了起来。我可以感觉到重量变轻了。他心情好的时候总是很体贴的。
\par 从这儿开始坡度降低了。很明显,这儿的树林有人砍伐过,有许多灌木比人还高,所以走起来特别慢,下一次我们可得绕道而行。
\par 现在我想在Chautauqua中摆脱抽象的解释,进入每天实际的世界中,而我无法确定该如何进行。
\par 拓荒者有一种很少被人提起的特性,就是很容易制造大量的混乱。他们只看见自己高远的目标,而完全忽略自己所制造的各种混乱。别人必须跟在他们屁股后面收拾。这种工作并不光彩,也很乏味。在你做之前,很可能会沮丧一阵子,然后一旦你的情绪习惯低调之后,感觉就不会那么恶劣了。
\par 能够在山顶上明白良质和佛在形而上学上的关系是一件很激动人心的事,而且它也非常重要。如果这就是Chautauqua所要说的,那么我应该结束了。然而还有更重要的,就是这个发现与我们眼前无穷无尽的、单调乏味的工作和岁月之间的关系。
\par Sylvia明白自己第一天究竟在说什么,当时她看到大家都朝着一个方向走。
\par 她称之为什么呢?“送葬的行列”。现在我的工作就是要从更宽广的角度进入这一列人群当中。
\par 首先我要澄清一点,我并不知道Phaedrus认为良质就是道是否是真的。我也不知道任何证明的方法。因为他所做的一切,只不过是把他对某一个神秘实体的了解与另外一个互相比较。他当然认为它们是相同的,但是他也可能不完全了解良质究竟是什么,或者他更可能不十分了解道。他当然并不是智者,在那本书里有许多给智者的建言,他本应该更注意。
\par 因此我认为,他所有形而上学的爬山,对于我们对良质的了解以及对道的认识完全没有任何帮助。
\par 听起来似乎是在排斥他的思想,但事实上并非如此。我想他也会同意我这样说,因为任何形容良质的方法都是一种界定,因而必然会有它不足之处。我想他也会说,他对良质的说明,甚至比不说明还要糟。因为说明就很可能会阻碍别人对良质的了解。
\par 所以他做的对良质或道并没有什么帮助,而受益的则是理性。因为他提出了一种方法,扩展了理性的范围,涵盖非理性的层面。我想二十世纪的混乱和不连贯的精神之所以产生,就是因为极度缺乏对这些非理性的认同,我现在就想尽可能有条理地探讨这一点。
\par 现在我们脚下所踩的烂泥变得十分滑溜,几乎很难站得住,我们必须抓住两旁的树枝和灌木丛,才能够稳住身体。
\par 我先踏出一步,然后探索下一步要踏在哪里,然后再踏下一步。
\par 很快我发现灌木丛变得十分浓密,必须要砍断一部分才能通过。我坐下来,让Chris从我的背包里拿出弯刀,他把弯刀交给我,然后我就开始一路砍下去。
\par 我一头钻进灌木丛里,我们走得很慢,每走一步就必须砍断两、三枝,这种情形持续了好长一段时间。
\par 首先,Phaedrus说过“良质就是佛”,这种断言如果是真的,就能替人类现在分裂的三种经验找到融合的理性基础,这三方面是宗教、艺术和科学。如果能够证明这三者的中心思想就是良质,而良质只有一种没有许多种,那么这三种分裂的经验就有了互相交流的基础。
\par 良质和艺术的关系可以通过Phaedrus在修辞艺术中了解良质得到充分的说明,我想在这里不需要进一步分析。艺术是一种高级良质的努力,需要说的就是这些了。如果还要更深入的解释,那就是:从人的作品中可以看到,艺术就是神。所以,我们可以由Phaedrus所建立的认知了解到,这两个大大不同的境界其实就是同一个。
\par 然而在宗教的层次,神与良质的理性关系,我想需要更完整地进行建构,我想以后再讨论这一点。现在我们起码可以知道良质与佛在古英语中的字根其实是同一个字(Good and god)。
\par 现在我想要讨论的是科学的层面,因为这是最急需建构的一面。认为科学和科技不受价值约束的看法应该终止了。明天我想从这一点开始讨论。
\par 下午剩下的时间里,我们一路跨过倒下的枯树,然后在陡峭的山坡上走着Z字形的路下山。
\par 我们走到一处悬崖,然后沿着它的峭壁找出一条路下去。就在这条小路旁边,岩石裂了一条大缝,里面有一条小溪,从里面长出不少灌木和大树,然后我们又听到远处有更大的溪水声。
\par 我们利用绳子涉过小溪,然后在路上看到其他的路人,就请他们载我们回去。
\par 到波斯曼的时候天已经黑了,为了不打扰狄威斯夫妇,不让他们来接我们,我们就在城里的旅馆住下了。在大厅里有一些观光客瞪着我们。我穿着旧军服,走路的姿势又很僵硬,两天没有刮胡子,又戴了一顶黑色的鸭舌帽,我想看起来一定好像是以前的古巴革命军人在演习。
\par 到了房间里,我们非常疲劳,把所有东西都丢在地上,然后把靴子里的溪石倒在垃圾筒里,再把靴子放在窗户旁,让风慢慢把它吹干。我们什么话也没说,就瘫在床上了。
\subsection*{22}
\par 第二天早晨,我们感觉清新无比地离开旅馆,跟狄威斯夫妇道过再见,离开波斯曼到北边开阔的公路上去。狄威斯夫妇希望我们留下来,但是我想往西走,以继续我的思索,这个特别的向往统治了我的心。今天我想要谈一位Phaedrus从未听过的人物,而我为了准备这次的Chautauqua已经广泛阅读过了他的作品。
\par 不像Phaedrus,这个人在三十五岁时已是国际名人,五十八岁时成为了一个活着的传奇,贝特朗·罗素描述其为“大家公认的、那个时代最著名的科学家”。他是集天文学家、物理学家、数学家、哲学家于一身的人。他的名字叫朱利斯·亨利·彭加列。
\par 对我而言它总是这么不可思议,而且我猜至今犹然。我认为Phaedrus游历过的应该是一条以前从来不曾有人游历过的思想路线。但是,一定有某个人,在某个地方,已经思考过所有这些。而Phaedrus是这么一个可怜的学者,可能他只是在复制一些有名的哲学系统而不肯费劲去检视。
\par 因此,我花了超过一年的时间阅读这冗长而且有时非常琐碎的哲学史,去探寻他所复制的观念。然而读哲学史是一件迷人的事,虽然至今我仍然不知它是由什么所组成的。有两个假定彼此对立的哲学系统似乎都跟Phaedrus所思考的很接近,只有很少的差异。日复一日,我认为我已经发现他在复制谁,可是每次总会显露出些微不同,我便发现他采取了迥异的方向。以黑格尔为例,我早先曾提到他,他拒绝印度哲学系统,认为其一点都不是哲学。但是Phaedrus似乎吸收了印度哲学,或者被它所同化。这里毫无矛盾冲突感。
\par 终于我提到了彭加列。又一次地,我们发现一点复制,但是实际是另一种现象。Phaedrus沿着一条长而弯曲的路径,得出了最高的抽象结论,他似乎要往回走了,但是却突然停住了。而彭加列从最基本的科学真理开始,达到同样的抽象层级后停下来。两人的路径刚好停在彼此的尽头!在他们之间有完美的延续性。当你站在非理性的阴影下,另一个所思所谈如你所为的心灵的出现,几乎可算上天的赐福了。就如鲁宾逊?克鲁索在沙滩上发现的足迹。
\par 彭加列从1854年活到1912年,是巴黎大学的教授。他的胡子与夹鼻眼镜使人回想起Henri Toulouse-Lautrec\footnote{Henri Toulouse-Lautrec(1864-1901),法国艺术家,在他的绘画、石版画和海报中描绘了蒙马尔特区的音乐厅及咖啡馆,包括《红磨坊的贪食者》(1892)},他当时住在巴黎,只比彭加列年轻十岁。
\par 彭加列活着的时候,精准科学的基础中已产生了一种令人担忧的深刻危机。多年以来,科学真理已经不容许怀疑的存在;科学的逻辑是不会错误的,如果科学家有时出错了,一定是因为他们弄错了它的规则。所有伟大的问题都已经被回答过了。现在科学的使命只是去精炼这些答案,使其更精确。的确仍有未经解释的现象存在,如放射能、经过“以太”\footnote{ether,一种在以前被假定为电磁波的传播媒质,具有绝对连续性、高度弹性的极其稀薄的媒体}的媒介光,以及磁铁与电力之间的特别关系;但是,如果过去的倾向是种指示,这些就已逐渐衰落。很难去猜测在数十年之内是否不会再有绝对空间、绝对时间、绝对实体甚或绝对磁力;那古典物理学——时代的科学岩石——可能会变成“可适用的”;而最清醒的、最受尊敬的天文学家会告诉人类,如果有一个威力足够的望远镜,看得够远的话,我们所看见的会是自己的后脑勺。
\par 相对论能动摇一切的基础,而它的基本原则只为非常少的一些人所了解,彭加列,那个时代最杰出的数学家,是其中一个。
\par 在他的《科学原理》一书中,彭加列解释说,科学原理危机的历史已经非常久远了。他说,人们徒劳地找了它很久,想去推演一个著名的公理——欧几里得的第五假设,而这个探求正是危机的开端。欧几里得有关平行的假设,描述了经过一个定点有且只有一条已知直线的平行线,这是我们通常会在初中几何里学到的。它是建构整个几何学建筑的基础。
\par 所有其他的公理似乎都如此明显,以至于不可加以怀疑,但这个并非如此。
\par 然而你必须否定数学的很大一部分现存内容才能摆脱它,似乎没有人能够将它还原为任何更基本的事物。彭加列说,我们真的无法想像,有多少努力被浪费在了那个荒诞不经的希望上。
\par 终于,在十九世纪的前四分之一期,而且几乎是同时,一个匈牙利人及一个俄国人——波耳雅以及洛巴契夫斯基——无可辩驳地证明了欧几里得的第五假设是不可能的。他们通过推理来证明,如果能够以任何方式把欧几里得的假设还原到某个更确定的公理,我们就需要注意另外一个效果了:逆转欧几里得的假设,则会在几何学中产生逻辑性的矛盾。所以他们逆转了欧几里得的假设。
\par 洛巴契夫斯基假设,先是通过一个定点可画出已知直线的两条平行线。他保留欧几里得的其他一切假设。从这些假设中他演绎出一系列公理,而其中没能发现任何矛盾。从而他建构了一个新的几何学,它的逻辑没有任何错误,丝毫不劣于欧几里得的几何学。
\par 因此,由于他没能发现任何矛盾,他证明了第五假设不可能还原至更简单的公理。
\par 并非这个证明令人惊慌。但是当它的理性副产品迅速覆盖了它以及数学领域中的每样事物。数学,这科学确定性的基础突然不再确定了。
\par 现在,不可动摇的真理在我们眼中有两种互相矛盾的形象,对各种年纪的人而言,它们都是真的,不论他们喜好如何。
\par 正是这一深远危机的基本原则动摇了这个镀金年代的科学家的自大。我们怎么知道这些几何学公理中的哪一个是正确的?如果没有任何基本原则可去分辨,那么你就有了一整个承认逻辑性矛盾的数学。但是一个承认内在逻辑性矛盾的数学根本就不是数学。非欧几里得几何学的最终效果不过只是魔术师莫名其妙的咒语,其中的信念全然由信仰所维持着!
\par 当然,一旦门被打开,人类便不能再抱此期待,不可动摇的科学真理的矛盾系统不可能仅仅被限制为两个。一位叫黎曼\footnote{德国数学家, 非欧几里德几何学的创始人}的德国人建立了另一个不可动摇的几何系统,他不只打击了欧几里得的假设,而且也波及第一公理,即两点之间有且只有一条直线可以通过。他的几何学并无内在矛盾,只是与洛巴契夫斯基几何学及欧几里得几何学不一致。
\par 根据相对论,黎曼几何学最好地描述了我们所生活的世界。
\par 在三叉镇,路拐进一条狭窄深长的白锡岩峡谷,路边有一些刘易斯与克拉克走过的洞穴。在布特的东面,我们爬上一条很长的阶梯,经过了大陆分水岭,然后下到一个溪谷里。过了一会儿,我们经过了阿耶孔达精炼厂的一排排厂房,绕进了阿耶孔达城,找到了一间有牛排和咖啡的好餐馆。吃过饭,我们再次出发,爬上长坡,来到松树围绕的湖畔,一些渔夫正在推小舟入水。然后,路又一次经过松树林蜿蜒而下,阳光照过来,我知道早晨即将结束。
\par 我们经过菲利普堡来到了一片山谷中的草地。前方的风变得更加暴烈,所以我减速到了五十五英里。然后我们经过了麦斯威勒。到达会堂的时候我已经疲惫不堪,一心只想着休息。
\par 我们在路边发现了一片教堂墓地,于是停下来休息。风吹得更烈也更寒冷了,但是阳光还算温暖。我们把夹克和安全帽放在草地上,在教堂的下风处休息。这里寂寞而空旷,但是非常美丽。
\par 当远方有座高山或者哪怕只是山丘,你就拥有了空间。Chris把他的头埋在夹克中试着睡去。
\par 没有了John夫妇,每一件事物都不同了——如此寂寞。如果你不介意的话,我想现在就来谈谈Chautauqua,直至寂寞消失。
\par 彭加列认为,要解决数学中真理是什么的问题,我们应该先问问我们自己几何公理的本质是什么。它们是像康德所说的先验综合判断吗?也就是说它们是否作为人的意识的固定部分,独立于经验而非由经验创造?彭加列认为不是这样。如果是这样,它们会以强大的力量强加于我们身上,从而使我们无法察觉相反的命题,或者我们会以它为基础建立一个理论组织。不会有非欧几里得几何学。
\par 我们应该就此下结论,说几何学公理是实验性的真理吗?彭加列还是认为并非如此。如果它们是,那么当新的实验资料进来时,它们会倾向于持续的变化和修正。这似乎跟几何学自身的整个本质相反。
\par 彭加列下结论道,几何学的公理是“传统”,我们在所有可能的传统中所做出的选择是由实验事实所指导的,但是它仍保有自由之身,并被避免所有矛盾的必要性所限制。因此,也就是说,即使实验规则决定了它们的被采纳只是近似的,假设仍然可以保持严密的真实。
\par 换句话说,几何学的公理不过是化装过的定义。
\par 然后,既已认同了几何学公理的本质,他转而考虑这个问题,欧几里得几何学是真的还是黎曼几何学是真的?
\par 他回答:这问题毫无意义。
\par 这好像我们这么问:是否英尺制是对的而常衡制是错的?是否笛卡儿坐标是对的而极坐标是错的?一个几何学不可能比另一个更正确;它只可能是更方便。几何学不是真实的,它只是更先进的。
\par 然后彭加列继续验证其他科学概念的传统本质,例如空间与时间。他告诉我们,测量这些实体时,没有任何一种方式会比其他方式更真实;通常被采纳者只是更方便的。
\par 我们对空间与时间的概念也只是定义,是在它们处理事实的方便性基础上所做出的选择。
\par 然而,我们对最基础的科学概念尚未彻底理解。时间和空间是什么的奥秘,可能通过这个解释变得更可理解了,可是现在维持宇宙次序的负担落在了“事实”身上。那么事实是什么呢?
\par 彭加列继续批判性地检查这些。他问,什么事实是你将要去观察的?它们有种无限性。与一只猴子坐在打字机前以产生祈祷诗篇的机会相比,未经抉择便对事实进行观察以产生科学的机会,不会更多。
\par 这点对于前提来说一样真实。哪些前提?彭加列写道:“如果一个现象承认一个完全机械的解释,它也会承认其他解释的无限性,这些解释同样说明了实验所揭示的全部特质。”这正是Phaedrus在实验室里所做的陈述;正是它引发了后来使他被退学的那个问题。
\par 彭加列说,如果科学家可以随心所欲地拥有无限的时间,只需要告诉他好好注意观察。但既然没有时间去观察每样事物,而且与其错看不如不看,对他来说,还是必须做一个选择。
\par 彭加列确定了一条规则:有一个事实的层级存在。
\par 一个事实愈普通,愈是值得珍惜。
\par 那些能够多次运用的比很难再次出现的更好。比如,生物学家若是建构了这样一个科学,它只有个体而无种族,而遗传又不能使小孩像父母,他将会十分地迷惘失落。
\par 什么事实像是能够再次出现的?简单的事实。怎么认出它们?选择那些看上去简单的。要么这朴素性是真实的,要么就是那些复杂的要素不好辨认。在第一种情况中,我们极可能再次遇见这简单的事实,它要么独自出现,要么作为一个复合事实的要素出现。在第二种情况中,它也有极大的可能再次出现,因为自然并不是随意建构了这些情况。
\par 简单的事实在哪儿?科学家一直在两个极端中寻找,无限大和无限小。比如,生物学家一直本能地倾向于认为,细胞比整只动物更有趣;而从彭加列时代起,蛋白质分子比细胞更有意思。这结果显示了这件事的智慧,因为人们发现,分属不同有机体的细胞和分子要比那些有机体本身更相像。
\par 那么如何去选择这个周而复始的有趣事实?方法正是对这一事实的选择;我们必须做出选择,要先创造一个方法去占据它;因为没有一个事实自告奋勇,所以人们想像了许多个。以有规律的事实开始是合适的,但是在一个超越所有疑问的规则被建立后,跟它相符的事实便变得枯燥乏味,因为它们不能再教给我们任何的新东西。于是例外就变得很重要。我们找寻的不是相同处而是歧异处,我们要选择最引人注意的歧异,因为它们最震撼人心,而且也最具指导意味。
\par 我们首先去找那些这个规律最可能失败的情况;通过在空间中走得更远,在时间中走得更久,我们也许会发现我们通常的规律完全被推翻,而这些伟大的推翻使我们能更清楚地看到那些也许会发生在我们周遭的小变化。但是我们应该针对的并不是对相同与歧异的再确定,而是对隐藏在明显歧异中的相似性的识别。特别的规律似乎在一开始总是不一致的,但是看得更仔细一点的话,我们将看到,大体上来说它们都很相似;质料不同的,样貌相似,各部分的次序也相似。当我们带着这种偏见去注视它们,我们会看见它们逐渐变大而且试图包囊一切事物。而正是它造就了某些事实的价值,使其能够构成一个集合,而且告知我们,这个集合是其他已知集合的忠实影像。
\par 彭加列下结论道,不!一个科学家并不随意选择他所要观察的事实。他试图将更多的经验与思考浓缩成薄薄的一册;这就是为什么这一本物理学的小书包含这么多过去的经验,以及更多的结果事前已知的可能经验。
\par 然后彭加列证明了一个事实是如何被发现的。他概略叙述了科学家是如何找到事实与理论的,但是现在,他将早期给他带来名声的数学函数,严密地融入了他个人的亲身经验。
\par 他说,有十五天之久,他努力去证明不会有这样的函数。每天他坐在工作台前,待上一两个钟头,尝试过一大串组合,但是没有达成任何结果。
\par 然后有一个晚上,很不寻常地,他喝了黑咖啡却睡不着。观念成群结队地产生。他感到它们互相撞击直到成双成对地联结,也就是所谓的有了稳定的组合。
\par 第二天早晨他只需要写出结果。结晶的波浪产生了。
\par 他描述了第二波的结晶怎么发生的,那是由已建立的数学类比所引导的,产生了他后来命名的数学函数。他离开他居住的凯恩,参加地质学的远足。旅途中的变化使他忘记了数学。在他进公共汽车之前,当他把脚放上台阶的那一刻,没有任何他先前的思考铺路,观念便跑向了他。这就是他曾用来定义这数学函数的转化,跟非欧几里得几何学的转化恒等。他说,他并未确认这观念,他只是在公共汽车上继续交谈,但是他感到一阵完美的确定感。之后,他利用闲暇验证了这个结果。
\par 后来另一个发现出现在他在海边高地散步时。它以同样简明、突然和立即的确定性的特色发生在他身上。另一个伟大发现出现在他走下一条街道之际。
\par 其他人赞颂这过程是天才的神秘工作,但是彭加列并不满意于这样一种肤浅的解释。他试着去深入探索所发生的事。
\par 他说数学不只是应用规则的问题,不仅仅是科学而已。它不只根据某些固定的规律去做最多可能的组合。如此得来的组合会是过量的、无用而且累赘的。
\par 发明者的真实工作在于选择这些组合,以便减少无用者,或者设法避免制造它们的麻烦,而指引这选择的规则是极其精致讲究的。几乎不可能精确地描述它们;它们必须被感觉而非被陈述。
\par 然后彭加列假设这个选择是由他所谓的“潜意识自我”所做出的,一个跟Phaedrus所谓的先知先觉的觉察刚好符合的实体。彭加列说,“潜意识自我”注视着一个问题的一大串解决方案,但是只有有趣的可以闯进意识领域内。数学解答是由潜意识自我所选择的,是基于“数学之美”,数字与形式的和谐,以及几何学的优雅。彭加列说:“这是一个所有数学家都知道的真实的美感,可是世俗者对此是如此无知以至于经常想笑。”但是这和谐、这美丽,是它整个的核心。
\par 彭加列清楚地说明,他不是在谈浪漫美,震撼感官的外表美。他是在谈古典美,它从部分的和谐秩序中所生,是一种可以把握的纯粹智慧;它给浪漫美以结构,没有了它生命会变得模糊无常,变成一场庸俗而无法辨别的梦幻,因为将没有任何依据去做这辨别。正是这特别的古典美的探求,这种宇宙和谐的感觉,使我们选择了更适于贡献给这个和谐的事实。并非是这些事实,而是事物之间的关系造就了宇宙的普遍和谐,这才是惟一的客观实在。
\par 保证了我们存身之世界的客观性是,这个世界是我们以及其他会思考的生物所共有的。通过跟其他人的沟通,我们接收到现成的和谐推论。我们知道这些推论并不是我们做出的,而同时因为它们的和谐,我们在它们之中辨认出了那些像我们一样的理性生物的行为。
\par 因为这些推论似乎适合我们感官的世界,我们认为我们可以推论出,这些理性的生物和我们看见过同样的事物;于是这就让我们知道,我们不是一直在做梦。正是这和谐,如果你想的话,就说正是这良质,是我们可知的惟一实在的惟一基础。
\par 与彭加列同时代的人拒绝承认事实是预先选择的,因为他们认为这样做会摧毁科学方法的有效性。他们假定“预先选择的事实”就是说真理是“任何你喜欢的东西”,而且称他的观念是传统主义。他们有力地忽视掉这一真相,他们自己的“客观性的原则”本身并非一个可观察的事实——因此按他们自己的准则,应该被暂时搁置。
\par 他们感到自己必须这样做,因为如果他们不这样,整个科学的哲学支柱就会崩溃。彭加列并未提供任何对这个迷惑之境的解答。他走得不够远,还没能进入他说要到达的形而上学的层面。他所忽略未说的是,在你“观察”它们之前,你要选择的事实是“任何你喜欢的东西”,只存在于一个二元的、主客观的形而上学系统中。当良质作为第三个形而上学的实体进入这幅图像中,事实的预先选择不再是武断的。事实的预先选择并不是基于主观的、反覆无常的“任何你喜欢的东西”,而是基于良质,即实在自身。因此,困境消失了。
\par 这宛如Phaedrus一直努力于他自己的谜题,却因为缺少时间而留下一整面还没完成。
\par 彭加列一直在他自己的谜题中工作着。他判断科学家选择事实、预设和公理是基于和谐,这也使一个谜题的潦草锯齿边缘不完整。在科学世界中留下了这印象,所有科学事实的源头不过是一个主观、反覆无常的和谐。这就如同试图解决知识论的问题,却给形而上学留下了未完成的边缘,从而使得这一知识论无法被接受。
\par 但是我们从Phaedrus的形而上学中得知,彭加列所谈论的和谐不是主观的。
\par 它是主体与客体的源头,并且存在于它们先前的关系中。它不是反覆无常的,而是抗拒反覆无常的力量;它是所有科学与数学思想的命令原则,它否定了反覆无常,没有了它就没有任何科学思想能够前进。让我流出认同的泪水的是下面这个发现,这些未完成的边界以一种和谐的完美使Phaedrus与彭加列二人所谈论的相吻合,由此产生了一个完整的思想结构,它能够使科学与艺术的不同语言合而为一。
\par 我们两旁的岩壁变得十分陡峭,形成了一条狭长的山谷,一直蜿蜒到米苏拉。风迎面吹来,吹得我昏昏欲睡。Chris拍拍我的背,指向远处一座山崖,上面漆了一个大的M 字,我点头表示看到了。今天早上我们离开波斯曼的时候,也看到了这个字。我突然记起来了,每个学校的新生都要爬上去漆一遍M。
\par 到了一座加油站,我们加满了油,看见一个人开了一辆拖车,上面载了两匹专门供骑乘用的马,我们闲聊了一会儿。大部分骑马的人都厌恶摩托车,但是这个人不一样,他问了我许多关于摩托车的问题。而Chris一直问我是否可以爬到那个M 那里去,但是从这里我就可以看到,上去的路非常陡峭而且崎岖难行。我们的摩托车只适合在高速公路上骑,再加上沉重的行李,我不想给自己惹麻烦。我们伸展了一下四肢,在附近走动了一下,有些疲惫地从米苏拉向娄娄帕斯而去。
\par 我想起不久以前,这条路还尘土飞扬,都是崎岖的小径,但是现在已经铺成一条宽阔的马路,就连转弯的地方也一样宽阔。很明显地,大家都在往北方的开立斯斐尔或者是寇尔戴洛尼而去。
\par 不过现在人稀车少,我们朝着西南方向行走,风从后面吹来,让人觉得舒服多了。路开始转进娄娄帕斯了。
\par 这里再也找不到一丝东部的气息,至少在我印象中是如此。这里下的雨都是由太平洋的气流带来的,而且所有的河流都向太平洋流去。再过两三天,我们就可以看到海了。
\par 在娄娄帕斯我们看到一家餐厅,于是就在一部老哈雷旁边停下来。它后面的行李架是自制的,里程表上显示的是三万六千公里。这是一位真正横跨大陆的骑士。
\par 进了餐厅,我们点了比萨和牛奶,吃完后马上就离开了。太阳就要下山了,天黑以后不容易找到露营的地点,而且路也很不好走。
\par 我们要离开的时候看到那个人和他太太站在摩托车旁边,于是互相打了一声招呼。他从Missouri来的,他太太脸上的表情告诉我,他们这一趟旅程过得不错。
\par 他问我:“你到米苏拉的时候,当地还是吹着这种风吗?”
\par 我点点头,“每小时大约三四十英里。”
\par 他说,“至少。”
\par 我们又谈了些露营的事,他们认为天气太冷了,他们做梦也没想到Missouri的夏天会这样寒冷,即使在山上也不应该这样。他们得去买衣物和毛毯。
\par 我说,“今天晚上应该不会太冷,现在只有五千英尺。”
\par Chris说:“我们要在路边露营。”
\par “在某一个营区吗?”
\par “不是,只要离路不远的地方就可以。”我说。
\par 他们似乎无意加入我们,所以过了一会儿我就发动车子,和他们挥手道别。
\par 一路上,树影拉得很长,大约走了五到十英里,我们看见有些伐木的专用道路,于是就骑了过去。
\par 伐木的专用道路通常都有很多沙土,所以我换到低速挡,把脚翘起来。
\par 从主路上我们看到还有其他的小路,但是我仍然骑在主路上。大约骑了一英里,看到一些压路机,这表示他们仍在这里伐木,于是我们就回过头来,骑到一条小路上。大约又骑了半英里之后,有一棵树横倒在路中央——这表示这条路已经废弃了。
\par 我跟Chris说,“露营的地方到了。”
\par 然后他下了车。现在我们在一处斜坡上,可以看到好几英里外绵延的森林。
\par Chris兴致勃勃地想要四处探险一番,但是我疲劳得只想休息,“你自己去吧!”我说。
\par “不行,你跟我一起去。”
\par “我真的很累了,Chris,明天早上我们再去探险。”
\par 我把背包打开,拿出睡袋铺在地上。
\par Chris走开了。我躺下来,把四肢伸展开,感到极度的疲倦。四周一片宁静,这真是一座美丽的森林。
\par Chris回来的时候,说他在拉肚子。
\par “哦!”我站起来,“你要换内衣吗?”
\par 他有些不好意思地说,“要。”
\par “内衣放在车子前面的背包里,换好了之后,从背包里拿一块肥皂,我们去河边,把脏衣服洗干净。”他因为很不好意思,所以乐于接受我的命令。
\par 下坡的时候,我们的脚步听起来特别沉重。Chris拿出趁我睡着时搜集来的石头给我看。在这儿我们可以闻到浓重的松树林的气息。天气转凉了,太阳也快下山了。疲劳、四周的沉寂还有西沉的太阳让我有些沮丧,但是我仍得打起精神来。
\par Chris把内衣洗好之后用手绞干,我们又回到原地。向上爬的时候,我突然觉得十分沮丧:我觉得一生都在不断地努力向上爬。
\par “爸爸!”
\par “什么事?”有一只小鸟从我们眼前的树上飞了起来。
\par “长大以后,我会变成什么样子?”
\par 小鸟飞到山那边去了,我不知道该怎么回答,最后我说:“很诚实。”
\par “我是指要做什么工作?”
\par “任何工作都可以。”
\par “为什么我问你的时候,你那么生气呢?”
\par “我没有生气……我只是想……我不知道…… 我只是太疲劳了,不想动脑筋……你要做什么都没有关系。”
\par 路愈来愈窄,然后就突然中断了。
\par 后来我发现他并没有开心起来。
\par 太阳已经下山了,天上是一片星光,我们走回原地,然后爬进睡袋,一句话也没说就睡了。
\subsection*{23}
\par 长廊的尽头有一扇玻璃门,玻璃门后面站的是Chris,他一边站的是妈妈,另一边站的是弟弟,Chris双手抵着玻璃门,看着我,一直向我挥手,我也向他挥挥手,然后向门走去。
\par 一切都很安静,好像在看一部无声电影。
\par Chris抬头看了一下母亲,然后笑了笑。她也回他一个微笑。但是我看到她按捺着自己的忧伤,她很难过,但是不希望孩子们看到。
\par 现在我看到这个玻璃门了,它是我棺材的门。
\par 这不是普通的棺材,而是一副石棺,我躺在里面,已经死了,他们在向我道最后的再见。
\par 他们这样做让我很感动,但是他们其实不必这样做的。
\par 现在Chris要我把玻璃门打开,我看见他想跟我说话,或许他希望我告诉他死亡是怎么一回事。我想要告诉他,他来看我真好,还向我挥手。我会告诉他,还不坏,只是很寂寞。
\par 我想把门打开,但是在门边有一个黑影,命令我不可以去碰它。他把手指比到唇边,我看不见他的脸,已死的人是不许说话的。
\par 但是他们希望我说,可见他们仍然需要我。难道他看不到吗?一定是出了什么差错。他难道看不见他们需要我吗?我恳求那个黑影让我和他们说话,还有许多未完成的事要告诉他们。但是我发现黑影似乎并没有听到我在说什么。
\par 我大吼一声:“Chris!”声音穿过了玻璃门,“再见!”那道黑影向我逼近,但是我听到Chris的声音,“我们将在哪里相会?”声音十分微弱而且很遥远,他听到我的声音了!然后那道黑影很生气地把门上的布帘拉起来。
\par 我想不在山上,山已经不见了。我大声地叫着,“在海底!”
\par 现在我一个人站在一座城的废墟中间,废墟包围着我,一望无际,而我得一个人慢慢地走下去。
\subsection*{24}
\par 太阳升起来了。
\par 有好一阵子我不能确定自己身在何方。
\par 我们在森林里的某一条路上。
\par 做了一场恶梦,又是那一扇玻璃门。
\par 车子在我旁边闪闪发亮,然后我看到松树林,又想起爱达荷。
\par 那扇玻璃门还有那旁边的黑影都是我的想像。
\par 我们在伐木的专用道上,对了……这是大白天……四处都是耀眼的阳光,哇!
\par 天气真好!于是我们向太平洋前进。
\par 我又想起刚才的梦,还有我说的“我们在海底相会”的话。我反覆地思考。
\par 但是松树林和太阳的魅力远远超过任何梦境,于是这些幻象都消失了,摆在眼前的是一片美景。
\par 我爬出睡袋,外面的寒气很重,于是我赶快把衣服穿上。Chris仍然在睡,我绕过他,跨过一棵倒在路旁的枯树,走到伐木专用道上。我先慢跑暖身,然后沿着马路飞快地跑着。好的,好的,好的,好的,好的,这个词和我慢跑的节奏刚好吻合。有几只飞鸟飞出树林,飞向太阳,我看着它们一直飞,一直飞,一直飞到不见了。好的,好的,好的,好的,好的。路上有不少的碎石子,好的,好的。太阳底下还有一片黄色的沙,好的,好的,好的,像这样的路有的时候可以伸展好几英里。好的,好的,好的。
\par 最后我喘不过气来了,不得不停下来。路升高了不少,我可以看到绵延好几英里的森林。
\par 好的!
\par 我仍然在喘气,我用轻快的步伐跑回来,脚下的碎石子声音小了一些,路旁的松树已经被砍走了,只剩下一些矮小的植物和灌木丛。
\par 回到露营的地点,我动作敏捷而且轻巧地把行李收拾好。现在因为十分熟悉收拾的步骤,所以不需要动脑筋就收拾好了。最后要收Chris的睡袋了。我摇了摇他,告诉他,“天气很好啊!”
\par 他四下看了看,还没有完全醒过来。
\par 他爬出睡袋,在我折睡袋的时候他把衣服穿好,然而神智还不十分清醒,不知道自己在做什么。
\par “把毛衣和夹克穿上,”我说,“这一路会很冷。”
\par 他照着我的话去做,然后爬上车。
\par 我用低速挡沿着这条路骑下去。出发前,我回头看了一眼,这里的确是个露营的好地方。
\par 今天的Chautauqua会很长,这是我在整个旅程当中最期待的一段。
\par 我用二挡,然后三挡,在这些弯道上,我不能骑得太快。阳光洒在四周美丽的森林上。
\par 截至目前为止,Chautauqua似乎有一层薄薄的迷雾尚未揭开。第一天我曾经谈到关心,然后我发现,如果大家不了解它的另外一面——良质是什么,那么我所说的关心就没有任何意义了。我想,现在重要的就是把关心和良质联结起来,指出关心和良质其实是一体的两面。
\par 如果一个人在工作的时候,能够看到良质,而且感觉到它的存在,那么他就是一个懂得关心的人。如果一个人对自己所看到的和手中所做的都细致入微地关心,那么他一定有某些良质的特性。
\par 所以,如果科技的根本问题在于,科技专家或是反科技的人都缺乏关心之情;而且,如果关心和良质是一体的两面,那么我们就可以推论出,今天在科技上出现的根本问题,就在于学科学的人和反科学的人,都缺乏在科学上洞悉良质的能力。Phaedrus狂热地研究良质这个词在理性、分析以及科技方面的解释,其实就是要替科技的根本问题找出答案。对我来说也是这样。
\par 所以我打起精神,把注意力转向古典和浪漫的对立。我认为其中隐含了整个人性与科技之间的问题。我想这也需要深入地研究良质的意义。
\par 想要从理性方面了解良质的意义,就需要了解形而上学以及它与日常生活的关系。所以接下来,我要从理性的层面研究形而上学,然后进入良质,然后再从良质回到形而上学和科学。
\par 现在我们已经由科学进入了科技之中,而我非常相信,最终我们仍将回到原先的起点。
\par 但是现在,我们先来研究一些影响深远的观念。良质就是佛,良质就是科学的实体,良质也是艺术的目标。这些观念仍然需要融入日常生活当中。而最简单的方法莫过于我一直提到的——修理摩托车。
\par 路一直在峡谷里蜿蜒前进。我们被清晨的阳光包裹着。摩托车在寒冷的空气里、在松树林里低吼。这时我们看到一个小标志,写着前面一英里左右有餐馆。
\par 我大声问Chris:“你饿了吗?”
\par Chris也大声回答我,“饿了。”
\par 第二面牌子上写着“旅店”,下面有一个指向左边的箭头,我放慢车速,转向左边。这条路不太干净,我们来到树下一些漆过的小木屋旁,把车停在树下,熄了火,走到大厅去。靴子踩在木头地板上,沉重的步伐声,十分好听。我坐在一张铺了桌布的餐桌前,点了蛋、煎饼、蜂蜜糖浆、牛奶、腊肠以及橘子汁。
\par 刚才的寒风激起了我们的食欲。
\par Chris说,“我想写一封信给妈妈。”
\par 我也这么想,于是就走到桌旁,拿了旅馆的文具,把它们递给Chris,然后把我的笔给他。早晨清新的空气让他的精神好多了。他把纸放在面前,然后紧紧地抓着笔,把心思集中在眼前的白纸上。过了好一会儿,他抬起头来问我,“今天是星期几?”
\par 我告诉他,他点点头就把它写下来。
\par 然后我看着他写,“亲爱的妈妈,”
\par 然后他又看着纸发呆。
\par 然后抬起头来问我:“我该写什么呢?”
\par 我笑了笑,我应该也让他练习一下描写钱币的某一面。有的时候我会把他当成学生,但还不至于当成修辞学的学生。
\par 这时煎饼端上来了。我叫他先把信放在一边,等一下我再帮他写。
\par 用过早餐,我抽着烟,刚才的煎饼、蛋和所有的一切让我现在舒服得一动也不想动。从窗子望出去,窗外的松树下洒了一地的阳光。
\par Chris拿出信纸来说:“帮我写吧!”
\par 我说,“好吧!”我告诉他,写不出来是一种最常碰到的情形,如果你想一下子说太多东西,往往就会这样。你要做的就是,不要强迫自己立刻写出来,因为这会使你更写不出东西。你只要先把事情一样一样地区分清楚,然后每次只写一样。如果你一面想要说什么,一面想先说什么,就太复杂了。所以要先把它们区分清楚,列出要说的事,然后再排出先后顺序。
\par 他问我,“比如说哪些事呢?”
\par “你想告诉她什么呢?”
\par “我们这一次的旅行。”
\par “旅行中的哪些事呢?”
\par 他想了一下,“我们爬的山。”
\par “好!那就把它写下来。”我说。
\par 他照着做。
\par 然后我看着他一项一项地写下来,而我在一旁喝咖啡。等我抽完了烟,他已经把要写的事情列成三张清单。
\par 我告诉他,“把这些清单留着,以后我们还会再继续写。”
\par 他说,“我不可能把这些写成一封信。”
\par 我笑了起来,他看见不禁皱起了眉。
\par 我说,“只要选出最好的事。”于是我们走出去,骑上摩托车。
\par 穿过峡谷,我觉得高度在不断向下降,耳朵里有所感觉。天气愈来愈暖和,空气也不像刚才那样稀薄了。我们和高山地区挥别,自从迈尔斯城之后,我们一直待在这样的地区里。
\par 今天我要说的就是“卡住了”。
\par 你应该记得,在我们离开迈尔斯城的时候,我提过如何在修理摩托车时运用科学方法。所谓的科学方法就是通过实验找出事物的因果关系。当时的目的是要指明古典理性的意义。
\par 现在我想提出一点,通过对良质的认知,古典的理性会有大幅度的进步,它的意义也会更加深广。在提出这一观点之前,我应该先提出传统的维修方法有哪些问题。
\par 首先第一个问题就是,在精神上和生理上都可能被卡住——就像Chris写不出信一样。以侧盖的螺丝取不下来为例:你翻遍了手册,想看看是否有任何说明能告诉你螺丝卡住了如何解决。所有的说明都只叫你把盖子取下来。这根本不是你想知道的。你也不是因为漏掉了任何步骤,才造成螺丝取不下来。
\par 如果你有经验,可能会先涂抹上渗透力强的油,然后再用撞击螺丝刀。但是如果你经验不够,就会用一般的螺丝刀,那时只要你用力一转,保证一定会破坏螺丝的沟槽。
\par 本来你一直在想盖子拿下来之后该做什么,所以过了一会儿你才发现,原来以为螺丝被卡住了只不过是小事一桩,现在问题可大了。这时所有的事都得停下来。
\par 在科学界或是科技方面这种情形最常出现。就传统的维修观点来说,这是最糟糕的一刻,所以尽可能要在事情发生之前就想到这一点。
\par 操作手册对你来说形同废物。科学的理性也是一样。因为你不需要做任何实验来找出问题的根源。问题很明白,你只需要知道如何把螺丝取下来,而科学在这个时候完全不管用。
\par 这就是意识发挥不了作用的时候了,你被卡住了。你找不到答案。机器发生了故障。就感情方面来说,你可惨了!你不但耗费了许多时间,而且最终没能解决。你不知道自己在做什么,你应该为此而感到可耻。你应该把车子交给师傅,他知道该如何修理。
\par 这个时候你又恐惧又愤怒,想用凿子把侧盖给敲下来。或者必要的话就用大榔头去打。你愈想愈生气,甚至想干脆把车子从桥上丢下去,想不到这样一颗小小的螺丝钉,竟然彻底地把你给击溃了。
\par 这个时候,你面对的正是西方思想里最大的缺憾。你需要一个解决的方法,然而传统科学不曾教导你如何自己摸索着解决。它让你清楚地知道身在何处,也能够验证你拥有的知识,但是它无法告诉你该往何处去,除非你的方向只是过去方向的延续。因此创意、原创力、发明、直觉、想像——换句话说就是“流畅”——全在它的研究范围之外。
\par 我们继续沿着山谷走,路边不时有宽阔的溪水,从陡峭的山坡流下,变成了小小的瀑布。路上的转弯不再急剧,路面也平直多了。于是我换到最高挡。
\par 不一会儿树变少了,而且变得又细又高,放眼望去是一片青草和灌木丛。
\par 这个时候穿着夹克和毛衣实在太热,所以我在路边停下,把它们脱下来。
\par Chris想沿着一条小径向上爬,我让他爬,自己则找到一处阴凉的地方坐下来休息。这个时候,我们静静地各自思索着自己的事。
\par 以前我看过一则报导,说许多年前这里发生过大火,虽然树木又长出来了,但是想要恢复原状,还得很长一段时间。
\par 后来我听到碎石子的声音,知道Chris回来了。他走得并不远。回来之后,他说,“我们走吧!”我把行李捆紧,又一次上路。刚才流的汗早已被凉风吹干了。
\par 让我们仍然来讨论那颗螺丝。要取下它的方法就是,先放下传统的科学方法,因为它根本不管用。我们先研究这种方法究竟有什么弱点。
\par 我们一直客观地研究那颗螺丝,根据传统的科学方法,客观是首要的条件。
\par 我们对螺丝的个人喜好和正确的思考无关。我们不能评价眼前所见的,而应该保持心灵一片空白,然后思考观察得来的事实。
\par 但是,当我们开始冷静地思考,就会发现这种方式很愚蠢。事实在哪里呢?我们要冷静观察些什么呢?是破损的沟槽吗?是盖紧了的侧盖吗?还是上面油漆的颜色?还是里程表?还是车把手呢?一辆摩托车有无数可以观察的事实,然而你所该观察的不会突然自己跳出来介绍自己。所以,我们真正需要观察的部位不仅是被动的,而且根本模糊不清。我们不能静静地坐着观察,我们必须把它们找出来,否则我们就得在那儿坐上好久,甚至永远都坐着。
\par 技术人员的好坏,就像数学家的好坏一样,取决于他在良质的基础上选择好坏的能力。所以他必须懂得关心。传统的科学方法从来没有提到过这种能力。过去,许多科学家在冷静观察之后忽略了这种能力的存在。我想总有人会发现,在科学研究的过程当中,接受良质的地位并不会破坏观察的结果,反而能扩展它的领域,强化它的能力,使它更接近实际的科学经验。
\par 所以,我认为“卡住了”的毛病中最基本的问题,在于传统的理性坚持要保持客观的态度。它将事实分为主客观两种,为了要得到真正的科学研究结果,就必须这样划分:“你是技术人员,它是摩托车。你和它永远都是独立的个体,你使用这种技巧,使用那种技巧,就会产生这样那样的结果。”
\par 用这种二分法来修理摩托车,听起来似乎错不了,因为我们已经很习惯它了。但是,这不是正确的态度,因为这是将人为的解释附加在事实上面,而永远不是事实的本相。一旦人们完全接受这种二分法,那么原先技术人员和摩托车之间不可分的关系,以及技术人员对工作的感情,就被摧毁了。传统的理性将世界分为主观和客观,把良质摒除在外,一旦卡住了的时候,任何主客观的事物均无法像良质一样,告诉你该往哪里去。
\par 一旦我们恢复良质的地位,就有可能让科技工作融入技术人员的关心之情。一旦卡住了的时候,良质就会显示出我们所需要的东西。
\par 现在我想像一列有120 节车厢的火车,它满载了原木和蔬菜向东行,然后再装着摩托车和其他工业产品向西行。
\par 我把这列火车称为知识,然后划分为古典知识和浪漫知识。
\par 从比喻角度来说,古典的知识,也就是理性教会所教导的知识,是指发动机还有所有的车厢,这所有的一切和里面装满的货物。如果你把火车肢解,你不会找到浪漫的知识。除非你十分小心谨慎,否则很容易就会认定火车所有的一切都在这儿了。其实并不是浪漫的知识不存在或是不重要,而是目前给火车下的定义是静态的,而且没有目的性。
\par 这正是我在南Dakota提到的两种不同存在的意义,也就是从两个完全不同的角度来看火车。
\par 浪漫的良质不是火车实体的任何一部分,它是发动机的前沿,除非你懂得真正的火车并不是完全静止的,否则浪漫的良质就只是一个没有真正意义的二维的表面。如果火车不能动,它根本就不算是火车。为了要检查这辆火车,把它划分成各个部分,我们必须要它停下来,所以我们所检查的其实并不是我们所谓的火车。这就是为什么我们会被卡住了。
\par 真正一列知识的火车并不是静止的状态,它总是要朝某个方向行进,而它的铁轨就称作良质。火车的发动机和120 节的车厢如果没有铁轨就根本动不了。而浪漫的良质,发动机的前沿,推动着火车沿着铁轨往前行进。
\par 浪漫的良质是经验的前沿,它是知识火车的前沿,推动火车沿着铁轨前进。
\par 传统的知识只是一些记忆,只是一些过去的前沿。前沿上没有主观,也没有客观,只有良质的轨迹一路在前,如果你没有衡量价值的方法,没有认知良质的方法,那么整列火车就不知该往何处去。
\par 因为你没有纯粹的理性——你只有全然的混乱。前沿就是一切行动所在。前沿包含着未来的全部可能性。前沿也包含着过去的全部历史。除此之外,我们还能到哪里去追寻过去与未来呢?
\par 过去不能回忆过去,未来不能激发未来,所以此时此地的经验就是最重要的一切了。
\par 价值,现实的前沿,不再是整个结构的一个无甚关联的分支。它是整个结构的前身,没有价值就无从选择。所以要了解有结构的真实,就要了解它的来源——价值。
\par 所以,一个人在修理摩托车的时候,对车子的了解分分秒秒都在改变,因而得到了全新认识,其中蕴含了更多的良质。修理的人不会受限于传统的做法,因为他有足够理性的基础拒绝这些思想。真实不再是静态的,它不是让你决定是要去奋战还是打退堂鼓的思想,它们是会跟着你成长的思想。所以具有良质的真实,它的本质不再是静态的,而具有爆炸性的威力,一旦你了解了这一点,就永远不会被卡住了。它虽然有形式,但是这种形式可以改变。
\par 或者用更简单明了的话来说:如果你想在盖一间工厂,或是修一辆摩托车,甚至治理一个国家的时候,不会发生被卡住的情形,那么古典的二分法,虽然必要,但是不足以满足你的需要。你必须对工作的品质有某种情感,你必须能判断什么才是好的,这一点才能促使你行动。这种感受力即使是你与生俱来的,你也仍然可以努力拓展它的范围。它不仅仅是你的直觉,也不仅仅是无法解释的技巧或是天才,它是你与良质接触之后产生的直接结果。它也是过去二分法的理性想要掩盖的一面。
\par 我这么说听起来似乎遥不可及,而且十分神秘。一旦你发现它竟是这样平凡,是你能够拥有的价值观,就会颇为惊讶。这让我想起哈里?杜鲁门提过的有关政府部门的计划:“我们会尽力去尝试……如果这些不管用……那么我们就要试试别的方案。”这里并不是引用原文,而是大致的意思。
\par 美国政府并不是静态的,如果我们不喜欢它的现状,就可以寻求某种更好的方法。所以美国政府不会受限于任何僵化的教条主义。
\par 所以关键在于“更好”——良质。
\par 或许有人会认为,美国政府的基本结构是不变的,所以无法为了产生更好的效果而改变。但是这种论点并没有切中要害。重点是总统和从最激进到最保守的每一个百姓都同意,政府为了要有更好的表现就应该改变。
\par Phaedrus认为这种不断改变的良质才是真实的,整个政府都要为之改变。虽然我们没有说出来,但是所有的人都有这种信念。
\par 所以杜鲁门所说的,其实和实验室里的任何一位科学家、工程师和技术人员对工作的实际态度,也就是不采用完全客观的方式去看待它,都是一样的。
\par 现在让我们回到那颗螺丝身上。
\par 让我们从另外一个角度衡量被卡住的情形。其实它可能不是最糟糕的而是最好的状况。毕竟禅宗曾花费了许多工夫去研究这种被卡住的情形;经由调息、打坐,让你的心灵倒空一切杂念,产生像初学者一样谦虚的态度。这样你就处在知识列车的前端,在真实的轨道上了。
\par 想一想,为了改变,我们不要害怕这一刻的来到,而应该小心地加以运用。如果真能达到这种境界,那么以后你所得到的方法,远胜过你满脑子杂念时所想出来的方法。
\par 解决的方法一开始看似不重要或是不必要,但是被卡住的那段时间让它有机会显示出真正的重要性。它之所以被认为微不足道,是因为导致你被卡住的价值观太过僵硬所造成的。
\par 但是让我们来思考这个事实,不论你被卡得多严重,这种现象终将消失。
\par 你的心灵终究会很自然地找到解决的办法,除非你非常容易被卡住。其实怕被卡住是不必要的,因为被卡住得愈久,你就愈看得清楚让你脱困的良质。
\par 所以不应逃避被卡住的情形,它是达到真正了解之前的心灵状态。要想了解良质,不论是在技术工作上或是其他方面,无私地接纳这种被卡住的现象是个关键。无师自通的技术人员就是因为常常被卡住,才比接受学院训练的人员更了解良质。因为他们懂得如何处理突发的状况。
\par 一般来说,螺丝非常便宜又不重要,所以不受重视,但是一旦你具有更强烈的良质意识,你就知道这颗小小的螺丝一点儿都不会不起眼,它甚至十分重要。
\par 现在这颗螺丝其实与整部摩托车的价值相同,因为如果你没有办法把螺丝拿下来,那么摩托车就根本发动不了。由于重新评估了螺丝,你就会愿意进一步认识它。
\par 我猜想拥有更深刻的了解就会对螺丝有新的评价。如果你把注意力集中在这上面很长一段时间,那么你可能会发现,螺丝并不只是属于某一类物体,它更有自己独特的个性。如果你再深入研究,你就会发现螺丝并不单单只是螺丝,它代表了一组功能。于是你原先被卡住的现象就会逐渐消失,同时也消除了传统理性的模式。
\par 过去你把各种事物都划分成主客观两面,你的思想就变得非常呆板,你把螺丝归入固定的类别,它比你所看到的事实还要真实,还要不可侵犯。由于你看不到任何新的构想和新的层面,所以一旦被卡住的时候,你就会束手无策。
\par 现在为了要把螺丝拿下来,你对它究竟是什么已经不感兴趣了。它的功能才是你研究的重点。于是你会提出有关功能方面的问题,由你的问题就可以知道你对良质的分辨能力。
\par 只要其中有良质,你究竟用什么方法解决它已经不重要了。你想到螺丝不但坚硬而且牢固,再加上有螺纹,你自然而然就会想到需要用压紧的方法和溶剂。这就是一种含有良质的解决方法。
\par 另外一种方法很可能要到图书馆去找一本机械用具的目录,查出哪一种螺丝刀能解决你的问题。或者你也可以打电话给了解机械的朋友。或是硬把螺丝给拔出来,甚至把它给烧了。再不就经过一番沉思之后,想出把螺丝拔出来的新方法,因而申请到专利,让你在五年之内变成百万富翁。所以解决的方法多得难以预估。一旦等你想出来之后,你就会发现方法都很简单。也只有在知道答案之后,才会觉得简单。
\par 第十三号公路沿着河流的另外一条分支而行,但是现在它转为溯流而上。
\par 一路上经过的都是老旧锯木厂集结的城镇,还有令人昏昏欲睡的景致。有的时候你从国道转进州际公路,会突然发现景象完全变了,你看到美丽的山脉,清澈的河流,有些崎岖不平但是仍然不错的柏油路,老旧的建筑,站在门廊前的老人……还有许多非常奇怪、已经被废弃的建筑——工厂。你可以看到五十年前和一百年前的科技,这一切看起来总是比新的好多了。在水泥龟裂的地方长出野草和野花,原先笔直而且方正的线条,变成了杂乱的线条,原先整片油漆好的墙壁,也出现点点的斑驳。大自然似乎自有一套非欧几里得的几何学,它把建筑上的客观线条,软化成随兴所至的曲线,更值得建筑师去研究。
\par 很快地我们离开了河岸和那些老旧而令人昏昏欲睡的建筑,爬上一座干燥而且满布绿草的高原。一路上有不少弯路,而且崎岖不平。所以我必须把速度降低到五十英里之下,地面上有许多坑坑洞洞,只要仔细一瞧还会发现更多。
\par 我们现在已经很习惯长途旅行了,过去在Dakota觉得漫长的旅程,现在感觉既轻松又惬意,骑在车上甚至比站在地上还要自在。目前我对乡野再熟悉不过了,而且我觉得自己不像是个陌生的外乡人。
\par 在爱达荷州的格兰杰维尔平原,我们从烈日底下走进了一间有冷气的餐厅,里面真是透心沁凉。等餐点的时候,我们看见一名高中生坐在柜台边,和身旁的女孩子眉来眼去。那女孩子非常美,不单单只有我注意到她,柜台后面的女孩子也很生气地看着她。她以为没有人发现她的表情。大概是某种三角恋爱吧。
\par 我们在不知不觉中,暂时闯入了别人的世界。
\par 我们又来到烈日底下,离格兰杰维尔不远,我们发现那片看起来像草原一样的干燥高原,突然之间裂成了一道巨大的峡谷。我发现要穿过沙漠地区起码要转一百个弯以上,到处是裂缝和岩石。
\par 我拍拍Chris的膝盖指给他看。转了一个弯之后,我听到他大声地喊着,“哇!”
\par 在崖边我把速度换到三挡,然后关掉节流阀。发动机有些逆火,我们继续往下骑去。
\par 摩托车到达谷底的时候,已经与高原有好几千英尺的落差,我回过头来,看到远方的车子像蚂蚁一样从上面经过。现在我们必须骑过这一片像火炉一样热的沙漠,不论前面的路要怎么走。
\subsection*{25}
\par 今天早上,我们已经谈过解决传统理性被卡住的问题的解决方法,现在我们要转向它浪漫的一面,也就是传统理性所造成的科技的丑陋。
\par 一路上不断地转弯,在沙漠里面绕行。然后来到了一座狭长的小镇,叫做白鸟,然后是一条水流丰沛的大河——赛门,它在峡谷高耸的两岸间奔流着。
\par 这里的温度非常高,白色岩壁反射的强光几乎使人睁不开眼睛。我们沿着狭窄的谷底蜿蜒前进,身旁快速的交通流量让人有些紧张,而且被高温压得有点喘不过气来。
\par John夫妇所厌恶的丑陋并不是科技与生俱来的,只是对他们来说是如此。
\par 我们很难把科技中的丑陋单独分离出来。科技只是制造物品,而制造物品本身并不丑陋。否则艺术品就不可能产生美感了,因为艺术也是制造物品。实际上科技这个词的词根就意味着艺术。在古代希腊人心中,从未把艺术和制造分开过,所以二者根本就是同一个词。
\par 在现代科技的原料当中,丑陋也不是与生俱来的——有的时候你可能会听到这样的论调。大量生产的塑胶制品本身并不坏,它们只是引起不好的联想。
\par 一个人如果终生关在监狱的石室中,他可能认为石头天生就是很丑陋的,虽然石头也是雕塑的主要材料。一个人如果终生就生活在丑陋的塑胶制品之中,从他童年时期使用的玩具,以至一生的消费品,都是塑胶制的,这样的人就可能认为塑胶品的丑陋是与生俱来的。但是现代科技真正丑陋的地方并不在材料或者形状或者这种生产方式和产品上,这些只是低品质的物品所有的特质。
\par 科技的产物并非真正丑陋。若是根据Phaedrus的形而上学,发明科技或使用科技的人也不丑陋,因为良质并不在主客观的事物当中。真正的丑陋在于发明科技的人与他们所制造的产品之间的关系。同样的状况也出现在使用科技的人和产品之间的关系上。
\par Phaedrus认为,在你意识到纯粹良质的那一刹那,甚至无所谓意识的时候,也就是与纯粹的良质相遇的那一刹那,无所谓主观,也无所谓客观。先有了纯粹的良质,接着才会意识到主体、客体。
\par 所以在遇见良质的那一刹那,主客观原是一体的。这正是佛教奥义书\footnote{Upanishads,古印度哲学吠陀经最后宇宙与个人自我之一致}中的最高精神,而这种一体感也是所有艺术的根基。而现代二分法的科技正缺乏这种一体感。创造者和拥有者对他们所创造和拥有的物体没有认同感,而使用者也一样。所以根据Phaedrus的定义来说,就是没有良质。
\par Phaedrus在韩国所看到的那面墙就是科技的产物,而它的美并不是来自于精密的策划或是科学的监造,甚或是刻意塑造的形式。它之所以让你觉得美是因为建造它的人十分投入。他们并未与手中的工作疏离。所以这就是整个解决办法的核心。
\par 要解决人类价值和科技需求之间的冲突并不需要逃避科技——这是不可能的。方法在于打破传统的二分法,进而真正了解科技的本质——并不是窃用自然,而是把自然与人的精神融合为一,创造出可以超越二者的产物。当这种产物已经出现,就像第一架横越海洋的飞机,或是人类第一次踏上月球,全人类就会对科技的超越性有全新的认识。但是同时在个人生活当中,也需要提升自己的精神层次。
\par 现在峡谷两旁的崖壁几乎完全是笔直的。有许多道路都是用炸药炸出来的,没有别的路,只能顺着河流的走向而行。
\par 我觉得河流似乎比一个钟头前窄了许多,这可能只是我的想像吧!
\par 当然,提升自己的精神层次并不一定要接触摩托车,单纯到像磨一把菜刀、缝一件衣服或是修补一张坏掉的椅子,它们背后的问题都是一样的。你做任何一件事都可以把它做得很漂亮,或是很丑陋。
\par 如果你想要有高水准的表现,就必须具备鉴赏力以及达到目标的方法,也就是同时具有对良质的古典和浪漫的认知。
\par 然而如果你想得到如何进行这类工作的指导,我们的文化只会给你古典的认知方法,也就是告诉你,磨刀的时候该如何拿刀子,或者如何使用缝纫机,或者如何混合胶水,如何擦胶水,它认为只要你照着这些步骤去做,自然会有高水准的表现。然而,它把鉴赏力给忽略了。
\par 于是就产生了现代科技非常典型的结果,为了让人容易接受它沉闷的外表,就在外面加一层包装。然而对于那些对浪漫的良质十分敏锐的人来说,这种情形更糟,因为它不只乏味到令人沮丧,同时还有虚伪的矫饰。把这两者加起来,你就可以得到现在美国科技精确的形象:流行的汽车、流行的摩托车、流行的打字机、流行的时装。流行的冰箱里装着流行的食物,摆在流行的厨房里,房子也是流行的。流行的塑胶玩具给流行的小孩。在圣诞节和过生日的时候,流行的小孩和他们流行的父母一起参加流行的聚会。你得经常跟上流行而不厌倦,所以你落入了流行的陷阱之中。有一群人从来不知道世界上有良质的存在,为了制造美感和利益,就在科技丑陋的外表上蒙了一层厚厚的浪漫的虚伪。良质并不是外加在主体和客体上的,就像圣诞树上闪亮亮的装饰品。真正的良质是主客观的源头,也是树木的生长点。
\par 为了得到良质,需要采取和二元化的科技不同的步骤,排除其中的步骤一、步骤二、步骤三……这就是我准备讨论的主题。
\par 在峡谷里面转了许多弯之后,我们在一些小树下停下来休息,树旁的青草有的已经烧掉,有的被晒干了,还有游人留下来的垃圾四处飞舞。
\par 我在树下全身放松地休息,过了一会儿,我眯起眼睛望着天空。自从骑进峡谷以来,我还不曾真正看过四周,峡谷的上方似乎离我们很远。天气凉爽,天空一片蔚蓝。
\par Chris并没有像平常一样忙着跑到河边看,他也像我一样,累得只想躺在树阴下休息。
\par 过了一会儿,他说在我们和河中间有一个旧水泵。他指给我看,然后走过去。我看他用水泵把水打出来,然后蹲下去洗脸。我走过去帮他打水泵,方便他用两手洗脸。然后我也这样做。打上来的水清凉极了。洗好脸之后,我们又回到摩托车旁,然后继续骑着上路。
\par 现在让我们来谈谈解决的方法。在前面所有的Chautauqua当中,我们都是从消极的角度去看科技制造出来的丑陋结果,而且我们也提过,像John夫妇这种对待科技的态度,是于事无益的。因为你不能单单靠着情绪活着,你还需要了解宇宙运行的方式,了解自然的法则。
\par 一旦明白之后,人就能工作得更顺畅,很少会有疾病和瘟疫,甚至连饥荒也不会出现了。从另外一方面来讲,科技因为制造了大量垃圾而遭人诅咒,现在是我们停止诅咒,提出解决方法的时候了。
\par 解决之道就在Phaedrus的论点当中。
\par 古典的认知不应该仅套上浪漫的外壳。
\par 古典和浪漫必须从根本上融合在一起。
\par 过去,我们的理性世界一直都在逃避,甚至拒绝史前时代人们浪漫而非理性的认知。之所以在苏格拉底之前就开始排拒热情,也就是情感,就是为了解除人类理性的禁锢,进而了解当时谜一样的自然法则。现在,我们则要借着融入原先我们逃避的热情,进而深入了解自然的法则。人的热情、情感以及意识中情感的层面,其实也是自然法则的一部分,而且是它的核心。
\par 目前,我们的科学陷入了盲目搜集资料的状态,因为我们对科学的创意没有理智地认知。到处充斥着流行艺术——非常贫瘠的艺术,因为我们并没有融入到艺术形式中。艺术家没有科学的知识,科学家也没有艺术的知识,两者都不重视精神的层次,这样导致的结果不但十分糟糕,简直是十分恐怖。艺术和科学的融合早就该开始了。
\par 在狄威斯家里的时候,我曾经谈到,工作的时候要保持内心的宁静,并因此被他们取笑,那是因为我表达得不够贴切。现在,我想回到这个主题上进一步讨论。
\par 保持内心的宁静在机械工作上并不是一件小事,它是工作的核心。能够使你平静的就是高级的手艺,反之,则是低级的。规格说明、测量仪器、品质监督与最后阶段的品质检查,这些都是达到内心宁静的方法。而最后真正重要的,就是要达到内心的宁静,除此之外别无他物。因为只有内心宁静,我们才能觉察到良质的存在,它超越了浪漫和古典的认知,将两者融合为一。无论进行任何工作,都必须具有良质。要想具有鉴赏力,了解如何完成高级的工作,体会和工作融为一体的感觉,就要培养内心的宁静。如此一来,良质才能出现在你的心中。
\par 我所谓的内心的宁静,和外界的环境并没有直接的关系。出家人在打坐,士兵在隆隆的炮击声中,或者是机械人员正在做万分之一英寸的校准,都可能产生内心的宁静。它涉及到一种自然的态度,让人与周围的环境完全融合在一起。这种融合有许多等级,而宁静也有许多等级,你的功夫愈深,就愈了解它的深奥和困难度。事实上,很多成就都是只从某一种角度发现了良质,发现过程中必须有这种自然的态度,否则这些成就就相对没有意义,也很难达到;而自然和忸怩是完全不同的两回事,它来自于内心的宁静。
\par 内心的宁静有三种等级,生理上的宁静虽然也有许多等级,但似乎是最容易达到的境界,印度神秘的修行者就曾经埋在地下好几天仍然活着。精神上的宁静,也就是消除个人的杂念,相对来说不太容易做到,但是仍然可以达成。
\par 至于价值方面的宁静,也就是一个人没有贪念,只是单纯地过着自己的日子,这一点似乎是最难的。
\par 有的时候,我认为这种内心的宁静和钓鱼有些类似,这就是为什么钓鱼会受大众欢迎的原因。你只要坐在那儿,让线垂在水里,一动也不动,不必刻意去想什么,或是担心什么。如此一来,就可以消除内心的紧张情绪和挫折感,是它们使你无法顺利地解决问题,造成你行动上和思想上的障碍。
\par 当然,你不一定要去钓鱼,你也可以去修摩托车,或是去喝一杯咖啡,或是到附近走一走。有的时候只要放下手中的工作,然后保持五分钟的安宁就够了。当你这么做的时候,你几乎可以感觉到自我正逐渐走向安宁。凡是背离它和良质的,表现出来的水准就不佳。但如果你能够亲近它,水准就会提升。亲近和背离的方法虽然数不胜数,但是目标却是一致的。
\par 我想,一旦介绍了这个观念,并且将其视为机械工作的核心,之后在实际的工作当中,就能够融合古典和浪漫的良质。我是指,你能从技巧高超的技术人员身上察觉到这种融合。如果你不认为他们是艺术家,那就误解了艺术的本质。他们有耐心和关怀,也专注于自己的工作,但是更让人感动的是,他们与手中的工作融合为一,因而产生了内心的宁静,能够独立处理自己的工作。在工作的时候,他的思想和工作都不断在改变,一直到作品呈现出它该有的形式,他的内心才会得到真正的安宁。
\par 在我们做自己真正想做的事时,就会有这种情况发生。只是很多时候,我们仍然会和自己的工作疏离。优秀的技术人员就不会如此。如果他对手中的工作很感兴趣,他就会沉浸在工作之中,而不会产生主客观之间的对立。然而在科学界,因为传统的二元化观点,人们就很难产生这种心态。
\par 佛教的禅宗提倡打坐,就是要使人物我两忘。而在我所提到的摩托车维修问题上,你只要专注地修理车子,就不会出现物我对立的情况。一旦真正地投入了工作之中,就可以说是在关心自己的工作,这就是关心的真正意义——对自己手中的工作产生认同感。当一个人产生这种认同感的时候,他就会看到关心的另外一面——良质。
\par 所以在维修摩托车的时候,最重要的就是要培养内心的宁静,让自己不要和工作环境疏离,在做其他的工作时也是同样的。这一点做到了,其他的一切也就会变得很自然。内心的宁静会产生正确的价值观,正确的价值观就会产生正确的思想,正确的思想就会产生正确的行动,而采取了正确行动的工作,便可使别人从中看到做事人内心的宁静。
\par 这就是韩国那面墙的意义,它反映出了人们精神上的状态。
\par 我认为,如果我们想改造世界,使它更适合人类居住,改革的方法不是从政治方面着手。因为那样,你会不可避免地涉及到主体和客体的分别,以及彼此之间的关系,或者你就需要计划各种活动。我认为这种改革是本末倒置。因为各种政治活动只不过是社会良质的产物。除非社会有正确的价值观,否则它的运作不会正常。而社会要有正确的价值观,首先个人要有正确的价值观。如果想要改造世界,就要先从一个人的心灵、头脑和手开始改造,然后由它们向外发展。有的人可以谈论如何改变人类的命运,我却只想讨论如何维修一部摩托车,我认为我必须说的这些更具有长远的价值。
\par 我们来到了瑞金斯小镇,镇上有许多汽车旅馆。车道最后离开了峡谷,沿着一条小溪前进,似乎向上伸入了一座森林中。
\par 的确如此。我们的四周很快就围绕着高大的松树,阵阵的凉风迎面袭来,眼前不远处出现了度假地的招牌。我们骑得愈来愈高,然后进入了一片凉爽的、长满了青草的松树林里。在新迈德镇我们加满了油,又买了两罐机油。直到此时,我们对眼前景象的改变仍然十分讶异。
\par 离开新迈德的时候,太阳将要西沉,傍晚沉郁的感觉渐渐袭上心头。在别的时候来到山上,总会使我的精神为之一振。但是我们已经骑得太久了,经过陶马拉克的时候,路逐渐向低处走,绿油油的草地又变成了干燥的沙地。
\par 我想今天的Chautauqua就到此为止吧。
\par 这是最长的一段Chautauqua,也是最重要的一段。明天我想要谈的是如何亲近良质以及如何远离良质,也就是可能面临的一些陷阱和问题。
\par 在离家如此遥远的这片沙地里,西沉的太阳射出最后的霞光,带给我们阵阵的伤感。我不知道Chris是否也有同样的感受。这是一种无法解释的伤怀,又一天消逝了,展现在眼前的只是逐渐沉重的暮色。
\par 霞光逐渐晦暗,仿佛失去了白天的热忱。在那一片沙丘外,在更远处的小屋里生活着许多人,他们终生居于此处,每天忙着同样的工作,而此刻,夜幕低垂之时,他们就像我们一样,觉得日子平淡无奇。如果我们早一点儿来到这里,他们很可能会好奇地问我们为什么来这里,但是到了傍晚可就不希望我们出现了,已经工作一整天了,是大家围坐一起吃晚餐的时候了。我们不想打扰他们,于是就骑过了这座以前从未来过的小镇。太阳已经下山了,我有一种强烈的寂寞感和孤独感,我的精神也随太阳西沉了。
\par 我们来到一座废弃学校的操场,在一棵白杨树下,我们把车停下来换机油。
\par Chris也有一点毛躁不安,不知道我们为什么要休息这么久,或许是他不知道的外界环境使他焦躁不安。在我换机油的时候,我把地图拿给他。换好之后,我们一起翻阅地图,决定一遇到好餐厅就吃晚餐,一遇到合适的地点,就准备露营。这样才使得他的心情好多了。
\par 在一座名为剑桥的小镇,我们吃过了晚餐。这时早已夜幕低垂,我们打开大灯,沿着一条小路骑往俄勒冈,路旁有一个牌子,写着布朗?李露营区。露营区在山谷里,四周一片漆黑,我们很难看出周围是怎样的环境。一路上崎岖不平,穿过一些灌木丛之后,我们来到露营区的停车场。这里似乎一个人也没有。
\par 我把车子熄火,卸下行李,这时我听到附近有一条小溪流过,还有小鸟嘤嘤鸣叫,除此之外,一片寂静。
\par Chris说:“我喜欢这里。”
\par 我说,“这里好静。”
\par “明天我们要去哪里呢?”
\par “进入俄勒冈州。”我拿了一个手电筒给他,在我卸行李的时候帮我照明。
\par “我去过那儿吗?”
\par “可能,我不太确定。”
\par 我把睡袋铺开,然后把它摆在野餐桌上,这种睡法让他十分兴奋。晚上的这一觉一定不会有问题了。很快地我就听到他的鼾声,他已经沉沉入睡了。
\par 我希望我知道要和他说些什么,或者问些什么。有的时候他似乎和我如此亲近,而这种亲近和我说什么问什么无关。有的时候他又似乎离我好远,站在一个有利的位置观察我,而我却没摸清状况。有的时候他又很幼稚,那个时候就和我完全无关了。
\par 有的时候,一想到这一点我就认为,所谓一个人能够进入别人的心灵,只是言语上的幻觉,只是一种说词,只是一种两个基本上独立的个体之间可能有的交流而已。两者真正的关系仍是无法得知的。想要探测别人的内心只会扭曲你的观察所得。所以我想做的就是,在某些情况之下让自己看到的不被扭曲。但是我不明白他问那些问题的方式。
\subsection*{26}
\par 我被一阵寒意冻醒了,从睡袋中望出去,天色一片灰暗,于是我又低下头,闭上眼睛继续睡。
\par 之后,我发现天色逐渐转明,但是寒意仍然逼人。我看见自己呼的气变成一股白烟。忽然我想到了天空中灰暗的云——很可能会下雨。但是仔细一看,那只不过是黎明前的暗淡。这么早就准备上路似乎太早也太冷了,于是我依然躺在睡袋里,而睡意早已消失了。
\par 透过摩托车的车辐,我看见Chris躺在餐桌上,睡袋卷成一团,他一动也没有动。
\par 摩托车的车影覆在我身上,仿佛它正准备前进,好像沉默的守卫等了一整晚,准备出发上路。
\par 这部车是银灰色的,掺杂铬色和灰色——目前污泥满布——这是从爱达荷、Montana、达科他、Minnesota州一路走过所累积下来的尘土。从下往上望,的确令人十分难忘。
\par 我想我不会把它卖掉。其实也没有什么原因。它们不像汽车一样,几年之内,车身就锈蚀了。如果经常保养、定期翻修,它们就会和你一样长命,甚至更长一点。这就是所谓的良质。我们骑了这么长的路,它都没有出过问题。
\par 清晨的阳光从山谷顶端的断崖上射下来,小溪上方出现了薄薄的晨雾,这就表示今天一定会是个温暖的好天气。
\par 我爬出睡袋,把鞋子穿上,然后把东西打包。最后我才走到餐桌旁边,打算把Chris摇醒。
\par 但是他没有反应。我四下看了一下,什么都做完了,只剩下把他叫起来,我有些犹豫。但是早晨清新的空气让我精神抖擞,于是我大声地喊他,“该起来了!”他突然坐了起来,两眼睁得大大的。
\par 他抬头看了看山顶上的晨曦,呆呆地坐在那儿,睁着眼睛看着我。
\par 很快地,我们就上路了,一路上依旧是大弯接着小弯。我们来到一座巨大的峡谷里,两旁是白色的断崖,吹在脸上的风寒气逼人。之后,有一些阳光照下来,晒透了我的夹克和毛衣,我觉得暖和些了,然而过了一会儿,我们又骑到背阴处,迎面而来的风又变得十分冰冷了。沙漠里干燥的空气无法聚热。风真大,我的双唇被吹得异常干裂。
\par 接下来,我们经过一座水坝,出了峡谷,进入半沙漠高地,这里就是俄勒冈州了。公路蜿蜒前进,除了我们,别无他人。
\par Chris大声告诉我,他又要拉肚子了。于是我们骑到一条小溪旁停下来,他很不好意思。但是我告诉他我们不急着赶路,把换洗衣服拿出来,给了他卫生纸和肥皂。告诉他上完厕所之后要彻底把手洗干净。
\par 我看着Chris从山坡上走回来,他脸上的表情看起来很愉快。
\par 今天我们的Chautauqua要谈论的是进取心。我喜欢进取心这个词,因为它十分亲切。但是同时它又很有个性,因此不免有些孤独。其实只要有人愿意跟它做朋友,它似乎都不会拒绝。这是一个苏格兰的古词,曾经有许多拓荒者用过,但是它就像“自己人”这个词一样,已经过时了。我喜欢它,是因为它真切地描绘出了一个具有良质的人,他很有进取心。
\par 希腊人称之为热忱。一个具有进取心的人,不会闲散得无事可做,在一旁忧心忡忡地焦虑。相反,他总是站在自我意识的火车头前,一发现有什么出现,必然立刻迎上前去,这就是进取心。
\par Chris回来说,“我现在觉得好多了。”
\par 我说,“那好。”于是我们把肥皂和卫生纸收好,然后把毛巾和湿衣服放在一块儿,这样才不会弄湿其他的东西,然后我们又骑上路了。
\par 一个人如果能够在安静当中,真正看见、听见、感受到真实的宇宙,而不是一些八股的思想,他必然会充满了进取心。进取心不是某种稀奇古怪的东西。
\par 这也是我喜欢这个词的原因。
\par 钓鱼钓过好长一段时间的人,身上往往会有这种特质。通常,他们会对自己花这么多时间去从事这项看似无甚收获的活动有些防卫心。因为他们不知道如何为自己辩解。然而,钓鱼回来的人通常充满了热忱,有力量去面对几个礼拜前他已经厌恶至极的事物。因此,事实上他并没有浪费时间,只是我们以世俗的眼光认为他是如此。
\par 如果你想要修理一部摩托车,那么充足的进取心是最重要的工具。如果你还没有足够的热忱,你最好收拾起工具,暂时放在一边,因为它们不会对你有任何帮助。
\par 进取心是精神的补给品,能够推动事情的进行,如果你没有它,就不可能修理摩托车。但是如果你有了它,你就会知道如何运用它,那么无论如何一定能修好这部摩托车。事情必然是这样发展的,所以在开始之前,最重要的就是要保有这样的热忱。
\par 进取心的重要性,解决了Chautauqua形式上的一个重要问题,那就是如何摆脱概念。如果Chautauqua只是深入研究维修某特定机种的细节,那么,它所提供的信息不仅可能对你无用,甚至可能会产生危险。因为适合修理某一机种的方法,很可能会毁损另外一种。要针对不同的机种进行维修,就需要不同的手册。
\par 但是还有另外一种细节,它从不在手册上出现,而且适用于所有的机型,我想在这里谈一下。这就是良质关系,也就是机器与人的关系。这种关系和机器一样地精密。在你修理机器时,经常会出现劣质的状况:关节生锈了,零件无法组合。这些意外都会消耗一个人的进取心,减少你的热忱,让你觉得十分沮丧,以至于想放弃。我称这些为进取心的陷阱。
\par 这种陷阱有成千上万个,甚至百万个以上。我不知道究竟有多少。看起来似乎我已经遇到过所有能想像得到的陷阱,但我知道并非如此,因为我每做一项工作,就会发现更多的陷阱。维修摩托车容易让人受挫,也容易让人愤怒,但这正是它让人觉得有趣的地方。
\par 我看了一下地图,知道前面不远就是贝克镇了,眼前的土地比较适合发展农业,因为降雨量比较大。
\par 我现在心里想的是一份“我所知道的进取心陷阱”的目录,我还想在学校里面开一门进取学的课。在课堂上把这些陷阱分类,建立体系,并且研究其中的关系,让未来的学子和人类从中受益。
\par 在过去的维修工作中,热忱一直被视为是与生俱来的,或者是通过良好的教育而得到的。它是一种固定不变的东西,由于只有很少人知道该如何得到它,所以我们就假定,没有进取心的人,就一定不会有远大的前程。
\par 然而在非二元论的维修中,我们并不认为进取心是固定不变的。它是会起变化的,它所蓄积的士气会增加也会减少,它是因人类意识到良质而产生的,所以,进取心的陷阱可以定义为,因无法意识到良质,从而使人丧失做事的热忱。由这个定义可以看到,它的范畴十分宽广,因此,在这里我们只能做初步的探索。
\par 据我所知,陷阱主要有两种:第一种是因外在的环境使你放弃了良质,我称之为挫折。第二种是你内在的因素引起的,我还没有一个确切的称呼,姑且称之为忧虑。我要先从外界引起的挫折说起。
\par 在你刚刚开始从事一项工作时,突然出现的挫折似乎是最让你担心的,尤其是它经常会在你认为大功告成的时候出现。经过几天几夜的努力之后,你终于完成了。但是:这是什么?连杆轴承垫圈?你怎么会把这个给漏了呢?天啊!所有的一切都必须重新来过,这个时候你似乎觉得自己已经彻底气馁了。
\par 你只好把它拆开,然后重新组合……
\par 经过大约一个月之后,你才逐渐认定这件工作已经完成了。
\par 我有两个技巧可以避免这种情况发生,尤其在拆卸一套非常复杂的组合之时,我会运用这两个技巧。
\par 在这里要先插入一点。有些人认为,如果这个部件你不了解,而且又十分复杂,就不应当自行拆卸。你应该先接受训练,或者让专家去做这项工作。我真希望有一天这样的说法会消失。因为就是所谓的专家把我的车给修坏了。我的工作也包括编写手册让IBM 的专家受训,他们很明白自己的本领并没有多么高强。第一次拆卸部件可能会有许多不利,因为你要花更多的时间和金钱去应付意外的损害。但是毫无疑问,下一回你就会远远超过专家了。虽然这个过程很辛苦,但是你对它已经有了感情,这是专家不可能拥有的。
\par 不论如何,首先我们谈谈,避开陷阱的第一个技巧就是拿出你的笔记本,写下拆卸的每一个步骤,然后记下以后重新组合时可能产生的问题。这本笔记本上面一定会沾染许多的油污,但是几次下来之后,写在上面的一两个字虽然往往看似不甚重要,但却避免了许多麻烦,节省了不少的时间。写的时候,要特别注意各个零件安装的方式和颜色,还有电线的位置。而且,如果有某个零件磨损,正好记下来,以便以后一起采购。
\par 第二个技巧就是在地上铺一张报纸,把所有的零件由左到右、由上到下排列整齐。这样一来,一旦要重新组合的时候,你可以由最后一步开始,许多小螺丝、垫圈还有扣针才不会被遗漏。
\par 即使有这样周全的准备,可能仍然会出意外,这时你就要特别注意自己的士气。一定要静下心来。如果你想加紧脚步,弥补损失的时间,可能反而会错误百出,最后又得全部重新来一次。
\par 通常是因为你缺乏某种信息,重新组合的顺序才会错乱。明确这一点很重要。常常,重新组合会变成一种试验性的技术,你把它拆解开,变化一下,然后组合起来,看变化之后是否仍能运转。
\par 即使不能,这也不是失败,你通过这一尝试得到的信息才是真正目的。
\par 如果你只是在组合的时候犯了错,你仍然可以挽回自己的士气,你要告诉自己第二次拆卸和组合的时候会比第一次快得多,因为你已经记住不少步骤,不需要再重新学习了。
\par 从贝克开始,我们一路经过了很多森林地带。眼前尽是绵延无尽的林区。
\par 然而,下山的途中,随着海拔逐渐降低,树林也逐渐稀疏,我们又来到了沙漠中。
\par 接下来我们要谈的是一些时断时续出现的问题,也就是说,你觉得有些问题需要修理了,但往往在动手之前又突然恢复正常。电子方面的短路常常是这样。摩托车发动的时候会短路,一旦停下来又没有问题了。所以你很难修理它,只能试着发动机器看看短路是否会再次出现,如果短路消失了,就不要再管它了。
\par 这种情形之所以是一种陷阱,是因为它让你误以为已经修好了。所以修理完一个机器之后,最好经过一段试用期再做已修好的结论。如果毛病一再出现,的确让人沮丧,但是总比一开始就去找专家要好,因为那是更大的陷阱。你得一直找专家,然而问题仍然得不到满意的解决。如果你自己修理,可以花时间去仔细研究,这是专家无法做到的。然后你可以随身带着自己需要的工具,一旦状况出现,立刻动手修理。
\par 一旦毛病再出现,要尽量把它与摩托车的使用情况联系在一起。以熄火为例,熄火是在车子跳动、转弯或者加速的时候出现呢?还是只在天气热的时候出现?这种推测是找出因果关系的有效途径。若是迟迟无法解决某些状况,你就需要去好好地钓一次鱼。其实,不论修理有多么枯燥,远比几次三番送去店里修理要好得多。有的时候,我很想仔细地解说我遇到过的问题和解决的方法,但是这就像鱼经只有钓鱼的人才会感兴趣一样,他们不懂为什么有人会听得打哈欠。只有他们自己才能享受其中的乐趣。
\par 除此之外,我想最容易出现的陷阱就在零件方面。刚买车的时候谁都没打算买零件。零售商不愿意积压存品;批发商又行动缓慢,而且在购买零件的旺季,又总是人手不足。
\par 零件的价格则是陷阱的另外一部分。就商业技巧来说,把原始设备的价格订得非常便宜是一种策略,不然客户会流失。而零件的价格则订得非常高,这样才能由中赚钱。而且由于你并非专业人员,零件的价格又会格外地贵。这是一种很狡猾的技巧,让许多专业人员有机会使用许多不必要的零件,从而发了大财。
\par 还有一个问题是零件可能会不符合需要。零件表往往会出错,零件的规格和种类也很复杂。因为工厂没有做好质量监督,往往会出现不耐用的零件。有时你买到的零件是由专业人员制造的,但是他们正好缺乏某些知识,因而产品出了问题。有时候,他们记错你需要的零件的编号。有时候,是你拿错了。所以往往拿回家后,才发现根本就不能用。
\par 这方面的问题可以这样改进。如果镇上不止一家经销商,那么无论如何要选择与你最配合的人。你甚至需要知道他的名字。这样的人往往自己当过技术人员,所以能提供你需要的资讯。
\par 试试看能否砍价。有的时候,你的确能砍到合算的价钱。自助店和邮购商店的零件价格往往大大低于经销商。比如说,你可以直接从链条工厂买链条,价格远低于摩托车修理店。
\par 要记得随身携带零件,以免买错了。
\par 同时要带专业的测径器备用。
\par 最后,如果你和我一样吃过零件的不少苦头,而且又有投资的能力,那么你也可以学着自己制造零件。自己制造不但不会破坏进取心,反而会激励自己,有一种特殊的感受。
\par 我们来到了长着鼠尾草的沙漠地带,发动机开始劈啪作响。我把备用油的开关打开,然后看了看地图,我们在尤尼蒂加了油,然后继续沿着两旁长满鼠尾草的炙热的柏油路往前骑。
\par 我认为以上这些就是最常出现的问题,但都是外在的环境因素造成的。现在我们要谈谈内心因素导致的陷阱。
\par 这一部分有三个陷阱。第一个陷阱会限制情感理解,称之为“价值的陷阱”;第二个则会阻碍认知理解,称之为“真理的陷阱”;第三个会阻碍精神运动行为,称之为“肌肉的陷阱”。其中价值的陷阱最严重也最危险。
\par 在价值的陷阱中,最常出现而又有害的是价值的僵化。这是指固守以前的价值观,无法从新的角度衡量事物。在维修摩托车的过程中,你必须不断评估,僵化的价值观不可能做到这一点。
\par 最典型的状况是摩托车出了问题,但是你却对它视而不见。你看着摩托车,但是却找不到原因。这就是Phaedrus提到的状况。良质、价值创造了世界的主体和客体。有价值才有它们。如果你的价值观是僵化的,你就无法接受任何新的观念。
\par 通常不成熟的判断就会导致这种状况。你认定了问题就在这里,但结果证明不是,你就傻住了。你必须找出新的线索。但是在你找出之前,你要先摒弃旧的观念。如果你一直坚持自己原来的看法,就无法找到真正的答案,即使它就在你的眼前。
\par 发现新的观点往往是一件令人兴奋的事。我们从二元论的角度称之为“发现”,因为我们假定原来没有人注意到它的存在。或者一开始人们对它的评价很低,然后由于你的发现,还有它本身潜在的价值,逐渐受到了人们的重视。
\par 而在我们周围,我们的眼睛和耳朵所接触到的事物中,其实并没有良质。
\par 事实上,它们的良质是负面的。如果它们同时出现,我们的意识可能会被这些毫无意义的资讯阻塞,从而根本无法思考和行动。所以我们要根据它们的良质加以选择,或者按照Phaedrus的说法,良质自身会过滤出我们需要了解的资讯,让我们的现在与未来相辅相成。
\par 如果你的价值观僵化了,你所能做的就是放慢脚步——不论你愿不愿意,你必然会慢下来——但是你要做的是刻意放慢脚步,然后重新检视过去你认为重要的事物是否仍然重要……只需要静静地注视着机器,这么做没有什么问题,静静地和它相处一阵子,用你注视鱼线的方式注视着它,不久你一定会看到鱼线在动。车子会用很谦虚而且微弱的声音询问你,是否对它的问题感兴趣。这是世界上到处都会发生的状况,所以要对它感兴趣。
\par 首先要明白,你发现的问题并不如你想像中的大,也不如你想像中的小。
\par 用不了多久你就会发现,这些问题往往比你原先所想的修理摩托车更有意思。
\par 一旦出现这种现象,你就不只是修理摩托车的技师,同时也是研究摩托车的科学家,这时就完全跳出了价值僵化的陷阱。
\par 我们又骑到了松树林里,从地图上看,这条路并不长。路的两边有一些广告牌,一些孩子在广告牌下面戏耍,仿佛是广告的一部分。他们一面捡松果玩,一面向我们招手,结果把松果撒了一地。
\par 我一直想再回到钓鱼这个比喻上来,我可以预见,会有人沮丧地问我:“没错,但是你想钓到怎样的事实呢?情况一定不只是你所说的这样。”
\par 而我的答案是,如果你已经知道结果,你就不是在钓鱼,而是在抓鱼。我想举一个比较特别的例子……
\par 随便举维修摩托车的各种例子都适合,但是我想到的最佳例子是南部印第安人抓猴子的故事。首先猎人把挖空的椰子用绳子绑在一根木头上。椰子里面放了一些米,通过一个小洞就能摸到。
\par 由于洞很小,猴子只能把手伸进去。而当手中握了米,就很难拉出来。所以要抓猴子,就是靠它僵化的思想。它不会衡量自由和拥有白米孰轻孰重,因而让村民有机会抓到它。但是在这种情况之下,你该给那只猴子怎样的建议呢?
\par 首先,这只猴子应该知道一个事实:如果它把手松开,它就自由了。但是它要怎样才能知道这个事实呢?那就是避免思想的僵化。不要再认为白米比自由重要。它要怎样才能明白这一点呢?它应该慢下来,在椰子旁边走走,看看它原先认为重要的是否仍然重要,不久它就会有一个新想法,它不知道自己是否仍然对白米感兴趣,这样它就有机会重新评估了。
\par 在普雷里市我们出了山区,来到一座干燥的城镇。骑在大街上,你可以从镇的这头一直望到那头的草原。我们去敲一家餐厅的门,但是没有人开门。于是我们又穿过大街去敲另外一家的门。
\par 门开了,我们进去坐下来,点了麦乳精喝。我拿出一封信的大纲交给Chris,那是他准备写给他母亲的。我很惊讶,他没有问我多少问题就开始写了。我在雅座里静静地坐着,不想打扰他。
\par 我一直觉得我和Chris之间也有某种僵化的关系。然而我就是没有办法找出问题所在。有的时候我们似乎像两条平行线,没有交集,有的时候又会相撞。
\par 通常他在家里惹的麻烦,都是学我向他发号施令一样,向别人发号施令。
\par 尤其是向他弟弟发号施令。别人当然不会接受他的指令,他不明白这一点,所以必然会引起冲突。
\par 他似乎并不关心别人是否喜欢他,他只想得到我的欢心。从哪个角度来说,这都不是好现象。所以是让他开始学习独立的时候了。虽然过程会很艰难,我要尽可能地让他容易接受,但总该有个开始,愈早开始愈好。
\par 但是现在想到这一切,我再也不这么认为了。我不知道问题究竟出在哪里,我总是想起那个梦,我没有办法忘掉它:我永远在玻璃门的另外一边,没有办法打开,他想叫我打开,但是总在打开之前,我就离开了。现在我们中间有了新的隔阂,那就是陌生。
\par 过了一会儿,Chris说他写累了。
\par 我们站起身,付了账就离开了。
\par 现在我们又上路了,可以再次开始讨论陷阱。
\par 接下来的陷阱很重要,这个陷阱来自于自我,它和价值的僵化颇有渊源,而且是造成价值僵化的原因之一。
\par 如果你自视甚高,那么你观察新事物的能力就会降低。你的自我会让你远离良质的真实。如果你把事情搞砸了,你很可能不愿意承认。如果你被蒙蔽,自以为表现得很好,你很可能会相信你确实表现得很好。所以在修理机器这方面,如果你的自我太强,往往无法把工作做好。因为你总是会被愚弄,很容易犯错,所以修理人员自大的个性对他颇为不利。如果你认识很多技术人员,我想你会同意我的观点,他们往往相当谦虚而且安静。当然,也有例外。不过即使他们起初无法保持安静和谦逊,长久工作下来,也会变成这样的个性。同时,他们还具有高度的警戒心,专注而又懂得怀疑,不会以自我为中心。
\par ……我想说的是,机器会反映出你真正的个性、感受、推理和行动,而不是反映你自我吹嘘、膨胀的那一部分。如果你的士气来自于你的自我,而非良质,那么这种虚假的形象很快就会完全崩溃,那你就会非常沮丧。
\par 如果你一时无法谦虚下来,有一个方法,就是无论如何也要装出这种态度。
\par 因为如果你刻意地假设自己表现得不够好,那么一旦事实证明的确如此,你的进取心反而会提升。你会继续这样做,一直到事实证明你的假设是错误的。
\par 焦虑是另外一个陷阱。它是自我的反面。如果你确知做什么事都做不对,那么你就会很害怕。就是这个因素往往让你迟迟不敢动手,而不是懒惰。这种过度担心的情况往往造成各种错误,于是你会去修理不需要修理的东西,去担忧假想中的困扰,然后产生各种荒谬的结论。你会因为自己的紧张而认定机器出了各种问题。一旦机器真的出现某些问题,就更验证了你起初对自己的低估,因而产生了更多的错误。如此恶性循环下去,就会不断给自己各种打击。
\par 要想打破这种恶性循环,我想你应该把自己的焦虑写下来,然后参考各种书报杂志。因为你有焦虑为动力,所以会很努力地研究。你愈研究就会愈平静。
\par 你要记得,你追求的是内心的宁静,而不仅仅只是把机器修好而已。
\par 在开始修理之前,你可以把要做的事写在纸上,然后再组织成适当的结构。
\par 你会发现,在不断的重组过程当中,会出现更多的想法。这么做不但节省不少时间,而且会让你不再慌张得出问题。
\par 为了减轻自己的焦虑不安,你可以告诉自己,没有哪一个技术人员不会犯错的。你和他们之间的差异是,他们犯错的时候,你并不在现场,但你要为犯错的结果付出代价,就是你的账单。所以如果是你自己犯了错,你最起码还有学习的机会。
\par 枯燥是我想到的下一个陷阱。这是焦虑的反面,通常和自我的问题连在一起。枯燥就表示你已经丧失了从新鲜角度看事情的能力。这样一来,你的摩托车可就危险了。在你觉得无聊的时候,就表示你的进取心很低落,在你开始做任何事之前,先好好地补充一下能量吧!
\par 当你觉得很厌倦的时候,放下手中的工作去看场表演,打开电视机或者和朋友联络一下,暂时离开那台机器。如果你不停下来,接下来很可能就会出大问题。所有的枯燥和问题累积到一定程度,就会突然爆发出来,于是你就真正地动弹不得了。
\par 而我自己医治枯燥最好的方法就是睡觉。在你觉得枯燥的时候非常容易睡着。一旦休息够了之后,就很难再觉得枯燥了。我的第二个选择就是喝咖啡。
\par 通常在我工作的时候,我会泡一壶咖啡放在旁边。如果这些都不管用,就表示我身体里面出了更严重的问题,让我无法集中注意力,而枯燥提醒你去注意这些问题——在你开始修理车子之前,先解决它们。
\par 对我而言,最枯燥的工作莫过于清洗机器。因为我总认为这是在浪费时间。
\par 好不容易洗好了,一骑上去又弄脏了。
\par John总是把车子保持得干干净净,看起来的确不错,而我的总是有些肮脏。这就是因为我思考的角度不同。我更注重机器运作得是否良好,外表的脏乱与否并不重要。
\par 要想使上油、换油、调整的工作不再那么枯燥,最好的方法就是把它们变成一种仪式。我听说有两种焊接工:生产线上的和维修的。生产线上的焊接工不喜欢复杂的事,而喜欢重复同样的动作,而维修的焊接工却很讨厌重复相同的动作。所以有人建议,在你雇焊接工之前,一定要确定他属于哪一种,因为这两者无法同时存在,我是属于后者的。
\par 这很可能就是为什么我喜欢研究问题,而最讨厌清理的工作。当然,非得做的时候我也能做。所以在清洗摩托车的时候,我就像别人上教堂一样,虽然不会有什么新发现,但还是让自己再去接触一次已经熟悉的事,有的时候这种感觉也不错。
\par 禅学里也提到有关枯燥的情形,因为它最主要的活动——打坐——就是世界上最乏味的活动。打坐的时候,你所能做的不多,身体既不能动,也不能思想,也不去关心外界事物,还有什么比这个更枯燥呢?然而打坐的核心却是禅学最重要的理念,它是什么呢?在枯燥的中心,你看不到的是什么呢?
\par 烦躁和枯燥颇为接近,但是它的原因是:低估工作所需要的时间。你不知道会有怎样的结果,因而无法依照预定的计划完成大量的工作。面对这样的挫折,首先你的反应就是烦躁。一不小心,很可能就会变成愤怒。而摒除烦躁最好的方法,就是增加工作的时间。尤其是新的工作,需要许多不熟悉的技巧,如果要赶时间,那么尽可能增加预定的时间,然后降低过高的期望。这一点需要我们的价值观有些弹性。在改变价值观的时候,通常会丧失掉一些进取心,但是这种牺牲是必需的,这总比因为烦躁而引发很多错误,最终导致进取心丧失殆尽要好得多。
\par 以上所说的就是价值方面的陷阱。
\par 当然还有许许多多的陷阱,我只是点到为止。几乎任何技术人员都可以告诉你许多他所发现的陷阱,那都是我不知道的。同样,你也会在你自己的工作上发现各种陷阱。或许学习的最好方式就是一发现陷阱就停下来,仔细研究,然后再去进行手中的工作。
\par 我们骑进了戴维尔。加油站旁边有几棵大树,我们在树下等待服务生过来,但是没有人出现。我们下了车,觉得全身僵硬。因为不急着上路,所以我们就在树阴下运动。这棵树非常高大,几乎把整个路面都遮住了。在这种沙漠地区竟然会有这么大的树,真是奇怪。
\par 服务生仍然没有出现,对街加油站的服务人员看到了我们,于是就走过来帮我们加油。他说,“我不知道John跑到哪里去了。”
\par John回来的时候向对方道了谢,然后很骄傲地说,“我们总是这样互相帮忙。”
\par 我问他这里有什么地方可以让我们休息一会儿,他说:“你们可以到我家门前的草坪上去休息。”他指了指对街的房子,屋前有几棵高大的白杨树,直径都有三四英尺粗。
\par 我们在绿草如茵的草坪上运动,路旁有一条水沟,里面有清澈的水流,用来灌溉这些草地和大树。
\par 我们在草地上睡了大约半个钟头,醒过来之后,看见John在我们旁边的绿草地上,一面坐在安乐椅上摇晃着,一面和另外一张椅子上的消防队员聊天。
\par 我静静地听着。他们聊天的节奏吸引着我,就是那种哪儿也不急着去,只是在消磨时间的调调。自从三十年代起,除了听过我祖父和曾祖父,以及叔伯和他们的父亲用这种方式谈话之外,再没有听到过这样缓慢而沉稳的聊天了。两个人一直聊着,没有任何目的,只是为了消磨时间,就像他们坐的摇椅一样。
\par John看见我醒了,就和我说了一会儿话。他说灌溉的水来自“中国人的水沟”。他说,“你不可能叫白人去挖那样的水沟。八十年前他们以为那里有黄金,所以挖了这条水沟,现在再也不可能有这样的水沟了。”他说这就是树长得这样高大的原因。
\par 我们接着又聊起我们从哪里来,要往哪里去。终于我们要离开了,John说他很高兴认识我们,希望我们休息好了。
\par 我们来到大树下,准备动身。Chris向他们挥挥手,他们也很高兴地向我们挥手说再见。
\par 沙漠里的路在峭壁和岩石之间蜿蜒回转。这里是目前为止最干燥的地区!
\par 接下来我想再谈谈真理的陷阱和肌肉的陷阱,然后就结束今天的Chautauqua。
\par 真理的陷阱和二元论有关。人类目前所有的知识,都是根据传统的二元论的逻辑和科学方法建立起来的。
\par 是或非……这或那……一或零。电脑就是根据零和一这两个数字来储存所有的知识。
\par 通常我们无法看到,除了是与非之外,还有第三种可能性,因为这不合乎思考的习惯。这第三种可能性能够拓展我们的视野,引领我们走向完全不同的方向。我找不到一个特定的形容词,所以想借用日文的“无”这个字。
\par “无”不是表示一无所有,“无”只是说没有等级,不是“一”,不是“零”,不是“是”也不是“非”。它表示在回答一个问题的时候,超越了“是”与“非”的等级,因而它所强调的就是不去问问题。
\par 如果答案不适合这个问题,就是“无”的现象。有人问禅宗的修行者,狗是否具有佛性。他的回答就是无。意思就是,回答有或是没有,都是不正确的,因为佛性超越了有或没有的问题。
\par 科学能够轻而易举地探知自然界是否有“无”的存在,只是我们忽而不察。
\par 比如说,有人认为电脑只有两种运算方式,一种是一,一种是零,这种说法十分可笑。
\par 任何一位电脑工程师都知道有另外一种运算方式,那就是把电力关掉的时候,电路系统会呈现“无”的状态。它既非一也非零,而是一种用一和零无法解释的状态。除了关掉电源之外,还有其他一些状态,也是无法用零、一解释的。
\par 习惯于二元思想的人会认为,“无”的状态是一种被掩藏的、与我们无关的现象,但是在所有科学研究中都会遇到这种现象。而自然不会欺骗人。自然的答案也永远都会与人相关。所以把自然给出的“无”的答案掩盖起来,是不诚实的行为,也是一种错误。能够认识和重新评估这种答案,才能够帮助理论更接近实验的结果。每一位实验室里的科学家都知道,一旦他实验的结果超越了单纯的是与否,就意味着他的实验设计不良。然而,他更应该这样想,这样的结果倒能避免将来再犯同样的错误。
\par 但是,一般人对失败的实验通常都会有错误的评价。找不到是或否的答案,就表明他设计的实验有问题。其实这也是一种非常重要的答案。通过这个答案,他对自然的了解会大幅地进步,这是实验最主要的目的之一。有一种非常正确的看法,就是说,由于有这些无法回答的问题,科学才会快速成长。是与否的问题只是肯定或者否定某一种假设,而无法回答问题,则表示你已经超越了你的假设,所以它能够刺激科学前进。其实这并没有任何深奥之处,只是我们的文化对这种实验结果的评价不高而已。
\par 在维修摩托车的时候,往往你提出的许多问题,都会碰到无法解决的状况,因而你就可能丧失信心。其实大可不必如此。如果你一时找不到答案,就表明你设计的问题无法替你找到你想要的答案,因而你对问题的了解必须更广泛。
\par 所以,你要做的是进一步研究你的问题,而不是摒弃这些无法回答的状况,它们和是与否的答案同样重要,甚至更重要,它们能够让你成长!
\par …… 摩托车似乎过热…… 但是我认为,这只是因为我们正骑过一个干燥酷热的地区……姑且不去深究其中的原因吧……一直到它真正的问题显现出来,我们才知道是更好还是更糟……
\par 我们在米切尔镇停下来吃点心,这座小镇坐落在干燥的沙丘之中。透过玻璃窗,我们能看到外面的景象。有一些孩子走下大卡车涌了进来,几乎占满了整个餐厅。虽然举止还算有礼,但是他们充沛的精力让场面显得颇为热闹。我们看到带领他们的女士对此有一点儿紧张不安。
\par 接下来又是干燥的沙漠和沙土地区,我们仍然向前骑去。现在太阳已经快要下山了。由于长时间骑车,我不但觉得全身酸痛,而且十分疲惫。Chris在餐厅的时候也觉得有一点儿提不起劲,我想或许他是……算了吧……
\par 关于真理的陷阱,这一次只谈谈“无”的状况我想就够了。现在我们要来谈谈精神运动方面的陷阱,这和机器本身的问题有直接的关系。
\par 这个陷阱最让人沮丧的就是工具不足。没有什么比这个更令人泄气了。所以要尽可能买好的工具。你永远不会后悔的。如果你想要节省,不要忘了看报纸上的旧货广告。好的工具一般来说不会磨损,而一把用过的好工具比差的新工具要好。仔细研究工具的目录,你可以从中学到许多东西。
\par 除了不好用的工具之外,恶劣的环境也是一种陷阱。要注意,你需要足够的光线。你会很惊讶,一点足够的光线能避免不少的问题。
\par 一些身体上的不适是不可避免的。
\par 但是如果身体十分不适,比如说,周围的环境太热或是太冷,就会使你在不经意之间降低判断力。比如说,你很冷的时候会加快动作,因而容易出错。如果你太热的话,你的耐力就会降低而容易发怒。所以你一定要尽可能地避免错误。
\par 如果通常你工作的时候姿势不良,就可以在摩托车的两边各放一把小凳子,这样会大幅度地增加你的耐力,你就不那么容易出错了。
\par 还有另外一种陷阱,就是肌肉失去感觉。这是造成真正伤害的原因。因为你无法分辨粗细。摩托车的外表虽然很粗糙,但是内部却很精密,很容易因为你动作不灵活而受损。这就是所谓“技术人员的感觉”。对知道的人来说,它很容易明白,但是对不知道的人来说,就很难形容。如果你看到没有这种感觉的人在修理车子,你一定会像那辆车子一样痛苦。
\par 这种感觉来自于对材料弹性的了解。有些材料弹性非常小,比如说陶瓷,所以给陶瓷零件加螺丝的时候,要小心不要用太大的力。而有些材料,比如说钢,就有很大的弹性,比橡胶的弹性还要大。但是除非你有极大的机械作用力,否则它的弹性不是很明显。
\par 当谈到内螺丝和外螺丝的时候,你就明白这是怎么一回事了。当你拿起一个螺帽的时候,有一个所谓手指紧度的点,也就是在螺帽刚接触到螺丝那点,还谈不上任何弹性,接下来,是螺帽与螺丝之间很平顺的结合,再接下来,是拧紧螺母的时候,就会感受到它的弹性。
\par 每一个内外螺丝在达到这三点的时候,都需要不同的力道,至于上了油的螺丝,情况又不同。不同的材料,比如说钢、铸铁、铜、铝、塑胶、陶瓷,所需要的力也不同。但是具有技术人员感觉的人就知道何时已经拧紧,应该停下来。没有这种感觉的人就会继续拧下去,因而损坏了整套零件。
\par 这种感觉暗示我们,不只要了解材料的弹性,同时也要知道它的柔软度。
\par 在摩托车内部的结构中,有些接合面的精确度高达万分之一英寸,如果你不小心把零件掉在地上,或者沾了灰尘,或是刮伤它们,或是拿榔头敲击过,它们就会丧失原先的精确度。很重要的就是要明白,表面之下的钢铁能够承受极大的撞击,但是表面却不能。处理表面极精密的零件时,具有这种感觉的人就会避免去损伤它的表面,然后尽可能地从不怕损伤的部分着手。如果必须直接从精细的表面着手,他通常会使用更软的材质。比如说,铜的榔头、塑胶的榔头、木的榔头、橡胶的榔头、铅的榔头,都适合这种状况。然后很小心谨慎地处理它的表面。这样,你永远都不会后悔。
\par 如果你习惯乱敲东西,那么尽可能多给自己一点时间,学习如何对待这些精密的零件。
\par 在这一片黄沙满布的地区,西沉的太阳让我们有一点儿忧伤的感觉。
\par 或许只是因为这是傍晚时分,是个容易让人感伤的时刻。但是今天提到这些事之后,我觉得多少切中了问题的核心。有些人或许会问,“如果我避开这些陷阱,那么是否就表示万无一失了呢?”
\par 答案当然是否定的,你仍然没有克服所有的问题。你必须要活得很正当,这样才容易避开这些陷阱,看到真正的事实。你想知道怎样画一张完美的画吗?很简单,你先让自己变得完美,然后再顺其自然地画出来,这就是所有专家的方式。画画和修理摩托车一样,都同你生活的其他方面密切相关。如果一周当中有六天你都很懒散,不去照顾你的摩托车,那么有什么方法能够使你在第七天突然变得敏锐起来呢?一切都是密切相关的。
\par 但是如果你六天当中都很懒散,而第七天尽量变得警惕起来,那么很可能下个礼拜就不会像这个礼拜这样懒散了。我指出这些陷阱的目的,最主要的就是提供一个人活得正当的秘诀。
\par 所以你要面对的真正的机器是你自己。你的内在和外在并不是分离的,它们会亲近良质或者远离良质。
\par 当我们到达普赖恩维尔城的时候,太阳已经逐渐西沉了。此刻我们在和第九十七号高速公路交叉的地方,我们准备在这里向南走。油加好之后,我疲惫地走到后面,坐在漆黄的水泥边石上,把两脚伸到碎石子里。夕阳透过树叶照到我的眼睛上,Chris走过来,在我的旁边坐下。我们什么也没有说。不过这还不是最沮丧的时候。我提到了这么多陷阱,此刻自己就掉进了一个。或许这就是疲劳吧!我们需要休息一会儿。
\par 我看了看高速公路上的车流,觉得它们有些孤寂。不只是这样,更糟的是空虚。就像加油站服务生脸上的表情一样。空虚的边石、碎石地,空虚的十字路口,空虚的目的地。
\par 汽车里的司机们也和加油站服务生一样,目不斜视地呆呆往前望着。自从第一天Sylvia提过这种情形之后,我再没有注意过他们。他们看上去好像是一排送葬的行列。
\par 有的时候会有人看我们一眼,然后又毫无表情地把头转回去想自己的事。
\par 仿佛因为怕我们发现而不好意思。我注意到这一点,是因为我们离开人群已经很久了。他们开车的方式也和我不一样。
\par 他们高速驶向城里,有特定的目的,所以就此时此地而言,他们只是短暂的路过,他们脑海中想的是将要去什么样的地方,而不是自己目前身在何处。
\par 我知道是怎么回事了!我们已经到西海岸了。我们对这里完全陌生。我差一点忘记最大的陷阱就是这个送葬的行列。每一个人都身处其中,身处这种摩登自我的生活方式中,自以为统管了整个地区。我们离开它已经太久了,几乎忘了它的存在。
\par 我们融入往南的车流里,我可以感受到其中蕴藏的危险。从后视镜里,我看到有人紧紧地跟在后面而不超车,于是我加速到七十五英里,他仍然跟在后面。于是我加速到九十五英里,终于把他给甩掉了。我很不喜欢这种方式。
\par 到本德城的时候,我们停了下来,在一间现代化的餐厅里吃晚餐。熙来攘往的人擦肩而过,连正眼也不瞧对方一下。服务虽然好,可是并不亲切。
\par 我们继续往南走,抵达了一片森林。
\par 里面的树木划分成许多可笑的小区域。
\par 很显然,这原是拓荒者的计划。在离高速公路有一段距离的地方,我们把睡袋铺开,这才发现松针压住了好几英尺厚的灰尘。我从来没有见过这种景象,所以十分小心,以免把松针给踢起来,使灰尘四处飞扬。
\par 我们把垫子拿出来,再把睡袋放上去。这样似乎就没有问题了。Chris和我聊了一会儿,谈谈目前身在何处以及要往何处走。我就着星光看了看地图,然后又拿出手电筒来看,我们今天骑了三百二十五英里,不算短的一天。Chris似乎和我一样累,我们两个真想好好地大睡一场。
\section*{第四部分}
\subsection*{27}
\par 你为什么不从黑影里走出来?你究竟长得什么样?你是不是在害怕什么?
\par 你害怕些什么呢?
\par 在黑影里的人后面是玻璃门。Chris要我把门打开,他现在大多了,但是仍然面带恳求。他想知道,“我现在该怎么办?我接下来要做什么呢?”他等待我的指示。
\par 是该行动的时候了。
\par 我仔细研究躲在黑影里的人。他不像过去那样给我不祥的感觉,我问他,“你是谁?”
\par 他没有回答。
\par “那扇门为什么不可以打开?”
\par 他仍然没有回答。对方保持沉默,这也表示他很怯懦,他害怕的竟然是我。
\par “还有比躲在暗处更糟的情况是吗?这就是你不说话的原因是吗?”
\par 他似乎意识到我要采取行动了,所以有些害怕地颤抖,想要退缩。
\par 我在一旁等待,然后向他靠近一点。
\par 讨厌、阴暗、邪恶的东西。我靠得更近,不是要看他,而是要看那扇玻璃门。我又停下来,双手抱肩,然后突然将手向前伸出去。
\par 我的手似乎是勒住了他的脖子。他越挣扎,我就勒得更紧,好像捉蟒蛇一样。我愈抓愈紧,想把他拖到明亮的地方。好了,让我们看看他的真面目吧!
\par “爸爸!爸爸!”我听到门外传来Chris的声音。
\par 没错!这是我第一次听到声音,“爸爸!爸爸!”
\par “爸爸!爸爸!”Chris抓住我的袖子,“爸爸!醒醒!爸爸!”
\par 他在我身旁哭着,“爸爸!醒醒!不要这样!”
\par “Chris,没事!”
\par “爸爸!醒醒!”
\par “我醒了。”透过薄薄的晨曦,我看出是他的脸。我们在户外的一棵树底下,旁边有一辆摩托车,我想是在俄勒冈州的某地吧。
\par “没关系!只是做了一场恶梦。”
\par 他还是一直哭,我静静地陪了他一会儿。
\par 我说:“没事的。”但是他还是不肯停下来,他害怕极了。
\par 我也是一样。
\par “你梦到什么?”
\par “我想看一个人的脸。”
\par “你一直叫着要把我给杀了。”
\par “不是,不是你。”
\par “谁呢?那是谁呢?”
\par “梦里的人。”
\par “是谁呢?”
\par “我也无法确定。”
\par Chris不哭了,但是他还是在发抖,可能是因为天气很冷。“你看到他的脸了吗?”
\par “看到了。”
\par “他长什么样子?”
\par “在我大喊的时候,我看到的是自己的脸……那只是一个恶梦。”我告诉他,他在发抖,应该回睡袋去。
\par 他回去了,说道:“天气好冷。”
\par “是啊!”从晨光中,我看到我们呼出的气变成了水蒸气。他爬进睡袋里。
\par 这时,我只看到自己吐的气。
\par 我睡不着。
\par 我梦到的人根本就不是我。
\par 是Phaedrus。
\par 他醒过来了。
\par 心灵分裂与自己对立……我……我就是在黑影当中的人,我就是那个可厌的人……
\par 我就知道,他一定会回来的……
\par 现在的问题就是要先做准备……
\par 从树下望向天空,看起来是那样的灰暗,那样的绝望。
\par 可怜的Chris。
\subsection*{28}
\par 绝望的感觉逐渐增强。
\par 就好像电影结束了,虽然你知道那不是真实的世界,但是却无法脱身。
\par 那是一个寒冷的十一月天,没有下雪,风把肮脏的空气从老旧而破损的车窗吹进来。车窗上还有不少灰尘。Chris当时六岁,坐在Phaedrus旁边,穿着毛衣,车上的空调坏掉了。从车窗向外望出去,天空一片灰暗,两旁是灰褐色的建筑,临街的墙是砖造的,上面还有玻璃窗,但是都破了,街上也到处飞舞着垃圾。
\par “我们在哪里?”Chris问。Phaedrus说:“我不知道。”他真的不知道,他茫茫然地在灰色的街道上盲目地开着。
\par Phaedrus说:“我们要去哪儿呢?”
\par “去找床。”Chris说。
\par Phaedrus问:“它们在哪儿呢?”
\par Chris说:“我不知道,或许我们一直开就会找到它们。”
\par 于是两个人一路开下去,边开边找卖床的。Phaedrus想停下来,把头放在方向盘上,好好地休息一下。他觉得路上的标识都完全一样,灰暗的建筑也没有什么差别。于是他们又继续去找卖床的,但是Phaedrus知道,他永远也找不到了。
\par Chris慢慢了解了,有一件奇怪的事发生了,开车的人没有办法掌握方向。
\par 他并不明了这一点,只是觉得很不对劲,于是喊“停下来”,于是Phaedrus停下来了。
\par 他们后面那辆汽车猛按喇叭,但是Phaedrus没有开走。然后其他的车子也开始按,于是后面的车更是拼命地按喇叭。
\par Chris紧张地说,“赶快开走!”Phaedrus很痛苦地把他的脚慢慢地放在离合器上,仿佛做梦一样,慢慢地启动了车。
\par Phaedrus问已经被吓坏的Chris,“我们住在哪儿?”
\par Chris记得一个地址,但是不知道该怎么回去。他心想,问问别人就可以找到路了。所以就说:“把车停下来。”
\par 下了车,他问别人该往哪个方向走,然后就牵着精神错乱的Phaedrus,走过无数的砖墙和破碎的玻璃,带他回家。
\par 不知经过几个小时之后他们才回到家,Chris的妈妈非常生气,因为他们竟然回来得这么晚。她不明白为什么他们找不到卖床的。Chris说,“我们已经找遍了。”但是他带着莫名的恐惧看了一眼Phaedrus。这就是Chris开始害怕的时刻。
\par 再也不会发生这种事了……
\par 我想我要做的就是朝旧金山骑去,然后让Chris搭公共汽车回家。接下来把摩托车卖掉,然后住进医院里……不过最后一点似乎不太重要……我不知道要做什么……
\par 这趟旅程终究有它的作用,最起码Chris长大了以后对我会有些美好的回忆。这样想减轻了我的焦虑。这的确是一个值得好好把握的念头,于是我就一直这样想。
\par 同时仍然要像一般旅行一样,时刻期望情况可能有所改进。不要抛弃任何东西,永远、永远都不要再抛弃任何东西。
\par 外面寒冷彻骨,就好像冬天一样。
\par 我们在哪里?竟然会这么冷!我们一定是在一个很高的地方。我从睡袋里望出去,看到摩托车上有霜。油箱上的露珠在晨光的照耀下正闪闪发亮。霜很快就会融化,滴到轮子上。这么冷的天气里躺在地上并不合适。
\par 我想起松针下有积土,于是小心谨慎地踏上去,避免把灰土激起来。我把摩托车上的东西都卸下来,然后拿出长袖的衬衣裤,然后再穿上毛衣和夹克,但是仍然觉得很冷。
\par 我离开松针垫,踏上公路,然后在公路上冲刺了一百英尺,然后再慢慢地跑,最后停下来。这时感觉好多了。四周寂静无声,路面上也结了一片一片的霜。经过晨曦的照射,有一些已经融化成水。然而树上的霜却仍然十分洁白,没有受到任何搅扰。我在公路上慢慢地走着,不想去打扰阳光,这正是早秋的感觉。
\par Chris仍然在睡,除非等到太阳暖和起来,否则不能出发。正好趁这个时候调整一下摩托车。我把空气过滤器的侧盖打开,从下面拿出一包工具,我的手因为寒冷而变得很僵硬,手背有些皱纹,当然这些皱纹不是因为寒冷的缘故,人到了四十岁,难免会如此。我把工具包放在车座上,然后打开它……它们都完好地躺在那儿……我好像又看到了老朋友。
\par 我听到Chris的声音,于是从车座上望过去,他只是翻了一个身,并没有起来。过了一会儿,太阳愈来愈温暖,我的手也不像刚才那样僵硬了。
\par 今天我准备谈一些有关修理摩托车的事情。修理车子的时候你会学到很多东西。它们不但丰富了你的修理经验,同时也带给你美感。但是现在谈这些似乎有些琐碎,当然,我不应该这样说。
\par 现在我想转到另外一个方向,把他的故事说完。其实我永远不可能说完它,因为我觉得没有这个必要。但是我想,在剩下的时间里应该这样做。
\par 工具冷得有些伤手了,但是伤手才好,这才让我有真实的感受,不像是在幻想,而是真真实实地握在我手中。
\par ……当你在路上前进的时候,发现了一条呈三十度角的岔路,过了不久又发现一条呈四十五度角的岔路,后来又有一条呈九十度角的岔路,这时你就会发现,所有的路都通往某一点,因为很多人都认为值得朝这个方向走。于是你也开始好奇,怀疑自己是否也应该走同样的路。
\par 在Phaedrus探索良质的路上,他不断看到各种岔路,它们都通往相同的一点。
\par 他知道大家走的就是古代希腊人的路。
\par 但是现在,他开始怀疑自己是否忽略了什么。
\par 他曾经问过莎拉,就是那位常常在他面前走来走去,手里拿着浇花水壶,同时也灌输良质观念给他的同事,她曾经在英国文学的领域里研究过良质。
\par 她说:“天晓得!我不知道,我不是英国文学专家,我是研究古典文学的,我的研究范围是希腊人的著作。”
\par 他曾经问过她,“良质是希腊人思想的一部分吗?”
\par 她说:“良质渗入希腊人所有的思想。”他曾经深思过这一点,有的时候从莎拉的口吻当中,他觉得自己仿佛探测到了一种神秘的语气,就好像希腊的神谕一样。其中隐藏着特殊的意义,但是他一直无法明白是什么。
\par 古代的希腊人。很奇怪,良质渗入了他们所有的思想,但是在今天,要提出良质这个概念似乎都是相当困难的事。其中发生了哪些改变呢?
\par 通往古希腊的第二条路,就是由这个问题——“什么是良质呢?”——指引的。这个问题曾经产生了整套的系统哲学。Phaedrus以为自己早已脱离了这个范畴,但是良质又把它带回来了。
\par 系统哲学是属于希腊人的,古希腊人先发明的,所以永远有他们的印记在上面。怀特海曾经说过:“所有的哲学都只不过是Plato的注脚。”这句话可以证明这一点,所以有关良质的问题,必须回溯到那个时代。
\par 第三条路则是在Phaedrus决定从波斯曼回去拿博士学位的时候出现的。他必须到大学去教书,他想继续研究良质的意义。但是该去哪儿研究呢?哪里才有这样的训练呢?
\par 很明显,除了哲学的范畴之外,没有任何科系在研究良质。而由他过去的经验得知,就算继续研究哲学,也不可能阐释英语作文里一个神秘的词语。
\par Phaedrus越来越清楚,没有任何学科适合让他研究良质,因为它根本就在学校的训练之外,同时也不在理性教会的管理之下。它值得大学颁与博士学位,然而研究它的博士却拒绝界定他所研究的事物。
\par 他花了好长一段时间看各大学的目录,芝加哥大学开设了一门独立于各科系的研究课程,叫做“观念分析和方法研究”。其中的委员包括英语教授、哲学教授、中文教授,而主席是研究古希腊的专家。这门课燃起了他的希望。
\par 现在摩托车已经调整妥当,只剩下换机油。于是我把Chris叫醒,整理好行李就上路了。Chris还有一丝睡意,但是路上的冷风使他清醒了过来。
\par 路一直向上延伸,两旁都是松树林。
\par 早晨路上没有多少车,松树林里的岩石黝黑,好像是火山岩。我在想,我们是否睡在火山灰上面?有所谓的火山灰吗?
\par Chris说他很饿,我也一样。
\par 我们在拉派恩城的一家餐厅停下来,我叫Chris帮我点火腿和蛋,自己到外面换机油。
\par 我在餐厅旁边的加油站买了一罐机油,然后来到餐厅后面的石子路上,把机油塞子打开,让机油流出来,然后换了一个塞子,把新的机油加进去。换好之后,我看见测油的铁杆在阳光照耀下好像清水一样干净,哈!
\par 我把工具收拾好,走进餐厅,看见Chris和我的早餐已经在餐桌上了,我赶紧到洗手间洗手。
\par 他说:“我好饿啊!”
\par 我说:“昨天晚上很冷,为了御寒,我们耗去了不少的能量。”
\par 蛋很好吃,火腿也一样。Chris提起昨天晚上做的恶梦多么可怕。他看起来好像是要问我问题,但是没有问,只是看着窗外的松树发呆。过了一会儿,他转头向我。
\par “爸爸?”
\par “什么事?”
\par “我们为什么要这样做呢?”
\par “做什么?”
\par “一直骑摩托车。”
\par “只是来看看乡野的风景。度假啊!”
\par 这个回答似乎不能令他满意,可是他也说不上有什么不对。
\par 突然他觉得有点失望,我没有和他说真话,这就是症结所在。
\par 他说:“我们只是一直骑下去。”
\par “当然,不然你要做什么?”
\par 他没有回答。
\par 我也没有继续说下去。
\par 上路之后,我想起一个回答,那就是我们正在做我认为最有良质的事。但是这个答案跟刚才我说的一样不能令他满意。我不知道还能说什么。但是如果他真的不满意的话,我必须得给他个答案,最迟在我们说再见之前,我们一定要好好地谈一谈。如果不让他明白过去发生的事,对他弊多于利。他会听到我谈Phaedrus,虽然有很多东西他可能永远也不会明白——尤其是结局。
\par Phaedrus到芝加哥大学的时候,整个思想体系已经和你我了解的完全不同。
\par 即使我记得所有的资料,也很难完全讲清楚。我知道,代理主席根据Phaedrus过去的教学经验以及言论的深度,接受了他的申请。当时他发表的言论已经没有记录了。之后的几个礼拜他等着主席回来,希望能够拿到奖学金。主席回来之后,两人见过面,但是对方只问了一个问题,Phaedrus没有回答。
\par 主席问他:“你实质研究的范围是什么?”
\par Phaedrus说:“英语作文。”
\par 主席大声说:“那属于方法学的范畴。”这就是会谈的重点。然后两个人又断断续续地聊了一会儿,Phaedrus有些迟疑有些结巴地说了一些很抱歉的话,然后就回到山里来。过去他离开学校就是因为他这种个性,一旦他回答不出一个问题,他就再也没有别的思路了。而课堂上没有他,课依然会进行下去。这一次,他花了整个夏天思考,为什么他的研究范畴应该属于实质,或者为什么应该属于方法。整个夏天他就只做这一件事。
\par 在雪线附近的森林里,他吃奶酪,睡树枝堆成的卧铺。渴的时候就喝山里的泉水,同时思考良质、实质以及方法的问题。
\par 有实质的东西是不会改变的,而方法则没有所谓的永久。实质和原子的主体有关,方法则和原子的功能有关。在科技的写作上,也有主体和功能的差异。
\par 如果想把很复杂的组合描述清楚,就必须把它的主要部分和零件与操作的方法分开。如果你把实质和方法混淆了,那么读者就不可能了解你说的是什么了。
\par 然而要把这种划分的方法应用在英语作文当中,似乎并不实际。因为所有的学院训练都包含这两种层面。而良质似乎与这两者都无关。良质没有实质也不是一种方法,它超越这两个范畴。如果一个人盖房子的时候会用到铅垂线和水平仪,那是因为垂直的墙壁比弯曲的品质要好,不容易塌毁;所以良质不是方法,而是方法所追求的目标。
\par 实质其实和客观是相当的,而这正是非二元化思想的良质所排斥的。如果所有的事都分成实质和方法,也就是主体和客体,那么良质就不存在了。所以他的理论不属于实质的范畴。一旦接受主体和客体之分,也就否定了良质的存在。如果要让良质有生存的空间,就必须取消二元化的分法,因而必然会和委员产生争执,这是他不愿意做的。但是他很愤怒,他们用第一个问题就摧毁了他整个的思想。实质的范畴?他们想要把他捆在怎样的普克斯汀床\footnote{Procrustean bed,古希腊的强盗把抓到的人施以酷刑,将人置于床上切掉多余的部分}上?他很怀疑。
\par 于是他决定进一步研究委员会的背景,他到图书馆去查资料。他觉得这个委员会的思考方式很奇怪。他看不出这种思考方式和他的思想有交会的可能。
\par 对于委员会目的的解释,他特别困惑。虽然委员会的说明用的都是非常平常的字,但是却用非常难以理解的方式组合。所以解释显得比问题本身更复杂。
\par 这和他原来的期望颇有出入。
\par 于是他尽量去研究主席的著作。他发现主席所用的词句也深奥难懂。艰涩的文体与他所见到的主席本人大相径庭。在和主席短短的会面当中,他认为主席心思敏捷,而且个性利落。然而他所看到的文体,却是高深莫测,就好像百科全书里用的词句。主语和述语往往一前一后隔得很远,甚至常常在句子中间的括号里面再加入一些无法解释的括号,因而使读者很难了解整个句子。
\par 最让人惊奇的是,其中有许多抽象观念似乎有特殊的意义,但是却没有进一步说明,读者只能猜测。这样的例子有许多,让Phaedrus不得不承认,他不可能了解眼前的文章,更不要说和它唱反调了。
\par 一开始Phaedrus认为,它之所以深奥难懂,是因为它超过了他的理解能力。
\par 你必须具有某些基本的学识,才能进入其中。然而他发现,其中有某些文章是写给根本就不具有这种学识的人看的,因此这种推测不能成立。
\par 他的第二个推测是主席是搞技术的出身。他的意思是,作者过于投入自己的研究范围,以至于丧失了与外界沟通的能力。如果情形真是如此,这个委员会为什么会给这门课这种非技术性的名称——“观念分析和方法的研究”?主席并没有技术人员常有的个性。所以这个推测也站不住。
\par Phaedrus干脆放弃研究主席的文章,转而研究委员会的背景,希望能对他的遭遇有所解释。结果证实这个方向是正确的,他开始看到问题是什么了。
\par 主席的文章充满了像迷宫一样的文句,让别人几乎完全无法了解他究竟在说些什么。这种情况就像你进入了一个房间,里面的人刚刚结束一场激烈的辩论,每一个人都静下来,没有人说话。
\par 我记得一个片断,Phaedrus站在芝加哥大学的长廊里,和委员会主席的助理交谈,他用侦探似的口吻说道:“在提到委员会的时候,你们漏了一个很重要的名字。”
\par 主席助理说:“谁?”
\par Phaedrus用很权威的语气说:“亚里士多德……”
\par 对方震惊了一下,然后像被抓到但没有犯罪的被告一样,大声笑,一直笑了许久。
\par 他说:“哦!我明白了,你不知道……
\par 任何有关……”然后他自己又想了一下,决定什么都不必再说了。
\par 我们来到通往火山口湖的岔口,然后沿着一条干净的公路进入国家公园——里面清理得非常整洁。虽然完全在意料之中,但是并不表示它具有良质。然后我们转到通往博物馆的路上。这里仍是白人到来之前的景象——到处流淌着美丽的熔岩,参天的树林里,看不到任何啤酒罐。白人来了之后,一切看起来都虚假多了。或许国家公园管理处应该在熔岩的中央堆起一些啤酒罐,那样可能就显得有生气多了。没有啤酒罐仿佛缺少了什么。
\par 我们在湖边停下来,然后下车舒展四肢,一面忙着和拿着照相机的观光客聊天,一面大叫:“不要靠得太近!”来露营的人车牌都不一样。站在火山口湖边,只觉得它就像在相片里一样。我发现其他观光客的脸上也是一片木然。我不是厌恶这一切,只是觉得这里的景致缺乏真实感,这个湖的良质被它所刻意强调的特色掩盖住了——如果你强调某一样东西具有良质,那么良质很可能就消失不见了——良质不是刻意强调的,而是你从眼角瞄到的事物。于是我望着下面的湖面,身后照来寒意逼人的阳光,吹来凝重的风,有一种很奇怪的感觉。
\par Chris问我:“我们为什么要来这里?”
\par “来看湖。”
\par 他不喜欢这里,他觉得它太做作,于是皱着眉头,想要找出个准确的词来形容,最终他说:“我讨厌这里。”
\par 一位女观光客很惊讶地看着他,然后面露厌恶的表情。
\par 我问他:“那么我们能做什么呢,Chris?我们必须一直走下去,一直到我们找出问题究竟出在哪里,或者找出为什么我们不知道问题出在哪里。你明白吗?”
\par 他没有回答。那位女士假装不听我们的谈话,但是她站在那儿一动不动,可以看出她仍然在听。我们走回摩托车旁,我想说什么,但是又想不出来。我看到他在流眼泪,但是他把头转开,不希望让我看到。
\par 于是我们离开公园朝南而行。
\par 我刚才说到委员会的主席助理当时大吃一惊,他之所以这么惊讶,是因为他发现,Phaedrus不知道自己正身处本世纪最著名的学术纷争之中。一位加州大学的校长将其形容为史上为改变一座大学的课程所做的最后努力。
\par Phaedrus的阅读挖掘出一部简明的历史,也就是在三十年代早期发生的、对实验性教育的著名反叛。“观念分析与方法研究委员会”就是那个反叛留下的痕迹。那个反叛的领导者是罗伯特?梅纳德?哈钦斯,他当时已是芝加哥大学的校长;还有莫蒂默?阿德勒,他的作品主要研究证据规律的心理学背景,有点类似于哈钦斯在耶鲁所完成的工作;斯科特?
\par 布坎南,一位哲学家与数学家;其中对Phaedrus来说最重要的是委员会的现任主席,时为哥伦比亚大学的斯宾诺莎主义\footnote{一种一元论哲学主义,认为所有实体组合在一起形成一种物质,通常称为上帝或自然,而人的头脑及身体都是这种物质的象征}者与中世纪研究者。
\par 阿德勒将对证据的研究与对西方经典著作的阅读相结合,产生了深具说服力的信念,即人类的智慧在近代进步得相当少。他持续地追溯圣?托马斯?阿奎那的观点。阿奎那吸收Plato和亚里士多德的观点,使其成为自己对希腊哲学及基督教信仰的中世纪综合的一部分。
\par 阿奎那的作品及他所诠释的希腊人的作品,对阿德勒而言是西方智慧遗产的精华,因此也成为好书的准绳。
\par 在由中世纪经院学者所诠释的亚里士多德传统中,人被视为理性的动物,能够找寻并界定优良生活,而且最终可以实践它。当这个有关人的本质的“第一原则”被芝加哥大学校长所接纳,不可避免地,它会在教育界产生回响。芝加哥大学有名的伟大典籍计划,按照亚里士多德的脉络重新组织的大学结构,还有“学院”的建立,以及学生从十五岁开始阅读经典著作,都是某些回响的表现。
\par 哈钦斯拒绝这样的观念,即实验的科学教育能够自动产生一个“优良的”教育。科学是“价值中立的”。科学抓不住良质,无法将其作为探究的对象,这使得让科学提供价值等级成为不可能的事。
\par 阿德勒和哈钦斯基本上是关心生活的“应为”、价值、良质以及良质在理论层面的哲学基础。因此很明显地,他们曾跟Phaedrus走过同样的方向,可是都或多或少地以亚里士多德为尽头并停在那里。
\par 那里就有了抵触。
\par 有些人想承认哈钦斯对良质的界定,却不愿意将最终的权力交给亚里士多德传统去定义价值。他们坚持,价值不能被固定,而一个有效的现代哲学,不需要考虑古代与中世纪典籍表达过的观念。对大多数人而言,这整件事情只是就一些含糊概念而发出的崭新而自负的胡话。
\par Phaedrus并不知道是什么造成了这个抵触。但是它似乎的确与他想研究的领域相关。他也认为没有任何价值可以被固定,但是说价值应该被忽视,或者价值并不以实体存在是毫无道理的。他也对将亚里士多德传统作为价值的定义者存在敌意,但是他并不认为应该不考虑这个传统。这种种问题使他陷入困境,而他想知道得更多。
\par 在创造这样一种狂热的四个人之中,委员会的现任主席是惟一在世的。
\par 大概是由于阶级的衰微或其他什么原因,他在Phaedrus所提到的人中并未获得和蔼可亲的评价。他的和蔼可亲不为任何人所承认,而且很锐利地被两个人所驳斥。其中一个是大学主要科系的系主任,形容他是“可怕的人”,而另一个人则持有芝加哥大学哲学学位,说这位主席以他自身的副本为事业标准。这两个人没有一位是生性爱报复的,而Phaedrus觉得他们所说的是真实的。这点后来更被他在系办公室的一个发现所证实。他想跟委员会的两位研究生谈谈,以对委员会了解更多,而他却被告知,委员会有史以来只颁发了两个博士学位。很显然地,要想让良质的现实走到台面上,他必须战斗并征服委员会主席,但由于他的亚里士多德观点,这战斗甚至不可能开始,而他的脾气也极不能容忍反对意见。所有这些加起来,就构成了一幅非常沉郁的画面。
\par 于是他坐下来,提笔写信给芝加哥大学“观念分析与方法研究委员会”主席。这是一封只可被描述为想被开除的挑衅信,作者拒绝安静地闪躲到后门,取而代之的是,他创造出有分量的一景使对立者不得不把他轰出前门,从而给与这挑衅前所未有的重量。稍后,他提起精神到街上去,在确定门已完全关上之后,他使劲地捶着它,让自己疼痛,并说着:“好吧,我试过了!”通过这个方式来减轻良心的负荷。
\par 后来Phaedrus决定写一封信给主席,说明他现在实际研究的范围是哲学而不是英语作文。接下来他说,把研究分成实质和方法两个范畴,是源自于亚里士多德二分形式和实质的方法,对拒绝二元论的人而言没有多大用处,因为他们认为这两者其实是一体的两面。
\par 他说,他并不是很确定,但是支持良质就是反亚里士多德的理论。如果这个论点是真的,他必须要找到适当的地方发表。任何一所伟大的学校必然能接纳对它基本理论的挑战,否则就是二流的学校。Phaedrus宣称他的研究正是芝加哥大学所期待的。
\par 他承认这种说法有些夸大,但他自己无法完全公正地判断自己的论点,有谁能毫无成见地评价自己的论点呢?如果有人能提出一个突破东西方哲学的理论,同时还能结合宗教的神秘主义和科学的实证主义,那就具有历史性的重要意义了,这样的理论会把这所大学推到前所未有的地位。然而在芝加哥大学,没有人真正接受这种理论,除非他把某个人给赶出去——那就是亚里士多德。
\par 接着他说得更夸张,充满了幻想。
\par 你会察觉,他已经不再重视他的言论对别人的影响力。因为他深陷在良质的形而上学之内,无法看清外界的事物。由于没有人了解他的内在世界,因此他完了。
\par 我想当时他一定觉得自己说的是真的,所以他的态度或是表达方法是否恰当并没有多大关系,因为他认为自己说的太重要了,没有时间做修饰的工作。
\par 如果芝加哥大学对他说的美学比对理性更感兴趣,那么它就丧失了建校的原始目的。
\par 情形就是这样,他真的这样相信,这不是另外一个需要现有理性方法考验的新思想,而是对现有理性思想的修正。
\par 一般来讲,如果你在研究中要发表新思想,你必须保持客观和冷静的态度。但是研究良质则推翻了这种假设。这种态度只适合用在二元化的学科上,因为只有通过客观才能产生精辟的二元论,但是具有创意的良质则不然。
\par 他深信已经解决了宇宙间一个巨大的谜团。用一个字眼——良质——快刀斩乱麻地解决了二元论思想的难题。他不愿意再让任何人把良质分成两半。如此一来,他就无法明白为什么别人认为他的言论难以置信。就算他明白,他也不会在意。他的说法是很夸大,但假如是真的呢?如果他错了,没人会在乎。
\par 但是假如他对了呢?如果为了取悦老师,而把自己对的成果抛弃,那才是最恐怖的做法!
\par 所以他不在乎别人的看法,而是自己一味狂热地投入研究。那些日子里,他活在孤独的宇宙中,没有人了解他。
\par 愈多人表示无法了解他,或是厌恶他的理论,他就变得愈狂热,愈不受欢迎。
\par 他这封信意外地得到了回响。委员会接纳了他的申请。但由于他实际研究的范围是哲学,所以他应该申请哲学系,而不是这个委员会。
\par 于是Phaedrus按要求申请,然后他和家人收拾好行李,向朋友道别,准备出发。正当他把门锁上的时候,邮差送来了一封信,是芝加哥大学寄来的。信上说,他没有得到入学许可。
\par 很明显,委员会主席在其中作梗。
\par Phaedrus向邻居借了纸笔写信给主席,声明他既然已经被委员会接受,就应该去报到,这是合法的。但是Phaedrus的言词有些火药味。从这位主席千方百计地想把他排斥于哲学系之外,可以看出他很可能没办法真正地摒弃他。即使他收到这封突如其来的信也无法再有任何举动。这让Phaedrus增加了不少信心。
\par 他们不准备暗地里做手脚,他们要不把他从前门给轰出去,要不就接纳他。或许他们连这个也做不到。这样倒好,他希望自己的论文不要欠任何人情。
\par 我们沿着克拉马斯湖的东岸而行,那是一条三线道的公路,颇有二十年代的风味。那个时代建造的公路都是三线道。我们在路旁的餐厅吃午餐,这间餐厅也是二十年代的情调。早已需要油漆的木头窗框,窗户上闪着啤酒招牌的霓虹灯,屋前的草坪上铺着小石子。
\par 洗手间里的马桶早已龟裂,洗手台上也布满了油垢。吧台后面的老板也有二十年代的长相,十分单纯,一点儿也不冷漠,挺着腰杆子。这里仿佛是他的城堡,我们就像他的宾客,如果我们不喜欢他的汉堡,最好闭上嘴。
\par 汉堡端上来,里面夹着大片生洋葱,吃起来非常美味。用餐的时候,我从地图上发现我们很早就转错了弯,因而可能提早骑到海边。现在的天气十分炎热,紧接着西部沙漠的酷热,西海岸粘湿的空气让人的情绪颇为低落。希望尽快到达海岸边,那儿要凉快多了。
\par 我在克拉马斯湖的旁边想这些事。
\par 湿热的空气,还有二十年代的恐慌……这正是当年夏天芝加哥的感觉。
\par Phaedrus和他的家人抵达芝加哥之后,就在学校附近住了下来,由于他没有奖学金,所以必须到伊利诺伊大学专任修辞学老师,这座大学坐落在海军码头,突出于海面上,不时会飘来恶臭,温度也很高。
\par 这里的学生和Montana的不一样。
\par 优秀的高中生都去了钱皮恩和厄巴纳校区,他所教的学生都属于丙等。由他们交上来的报告,你分辨不出好坏。在其他情况之下,Phaedrus还可能想些别的办法来提升他们的水平,但是由于这份工作关系生计,所以他不愿意出任何意外,于是他把主要的精力放在另一所学校。
\par 他来到芝加哥大学的注册处,把他的名字告诉正在负责注册的哲学教授。
\par 他注意到,教授听到他的名字后,表情变得不一样。教授表示,委员会主席已经让他去上“理念和方法”的课,由主席亲自教授。他给他课程表,Phaedrus发现上课的时间和他在伊利诺伊大学的课有冲突,所以就选择了另外一门课,主讲的人不是主席,而是替他办注册的哲学教授。这位教授对他的选择有些惊讶。
\par Phaedrus回到伊利诺伊大学教课,然后准备上哲学课时该读的书。现在对他来说非常重要的,就是拿出前所未有的研究精神,去研读一般古典的希腊书籍,其中最重要的一种是亚里士多德的书。
\par 在芝加哥大学成千上万的学生当中,读过古典著作的人,很难找出比他更用功的。学校有引导学生接近古典经籍的计划,但是却与现代思想背道而驰。
\par 因为现代人认为,这些古典书籍对二十世纪没有多大助益。所以大部分选择这些课的学生必须刻意表现出顺从的态度,假装这些古典书籍对他们颇有意义。
\par 但是现在Phaedrus不想这么虚伪,所以就不接受这种做法。他十分清楚,自己来到这里,就是要激烈地反对这些思想,然后用各种方法攻击它们。攻击并不是因为它们与二十世纪无关,反而是因为关系太密切了,读得愈多就愈相信它们。
\par 没有人知道,不知不觉地接受这些思想,对世界会有多么大的影响。
\par 在克拉马斯湖的南边,我们穿过一些似乎是郊区的地方,然后朝西海岸前进。这条路通往森林,而林里的树与前面沙漠里的树完全不同。高大的枞树耸立在路的两旁,我们骑着摩托车,抬起头,看见树干笔直向上,大约有好几百英尺高。Chris想停下来到树林里走一走。于是我们就停下来了。
\par 他走到林子里去了,我小心谨慎地靠着一棵树,然后向上望,想要回忆起……
\par 他学到了些什么都已经没有记载了,但是通过后来发生的事,我知道他吸收了大量的知识,他用照相一般的能力做到了这一点。要了解他如何诅咒古希腊的思想,就有必要稍微了解一下神话先于理性的论点。这是研究希腊的学者很熟悉的理论,而它本身也有十分值得研究的魅力。
\par 理性是指建立我们对世界的了解的方式,而神话则是指史前人类的世界观。
\par 神话不仅包括希腊神话,同时也包括旧约、吠陀经,还有各种文化的早期传说,它们对我们现在的知识都有贡献。神话先于理性的论点认为,现代的理性都是由这些传说而来。我们今日的知识和这些传说的关系,就像大树和它原先还是小灌木时候的关系一样。我们只要研究简单的灌木结构,就能获得对大树的了解。因为它们属于同一种类,只是大小有些差异罢了。
\par 因此在包括希腊文化的各种文化当中,你一定会发现强烈的主客观之别。
\par 因为根据希腊文化,宇宙可以分为主体和客体。而不是像中国文化,主客之间的关系在文字上并没有僵化的界定。而在犹太人的文化当中,旧约所谓的“道”,本身就很神圣,人们愿意为之牺牲。所以在这种文化中,法庭可以要求证人“说实话,所有的实话,除此之外,还是实话。所以请上帝帮助我”,因而能期望证人诚实。但是一旦把这样的法庭搬到印度,就像英国人过去所做的,并不能消除伪证。因为印度神话的观点不同,人们对于文字的神圣有不同的感受。同样的问题也在印度其他文化背景的种族当中出现。所以我们能找出无数的例子,证明不同的神话就有不同的行为模式。
\par 而神话先于理性的论点认为,每一个孩子出生的时候,都像山顶洞人一样无知。而这个世界之所以不再回到山顶洞人的时代,是因为每一代都有属于他们自己的神话。虽然神话已经被理性取代,但是理性仍然是一种神话。整个庞大的常识体系把我们的心连在一起,就像细胞把我们的身体连在一起一样。如果认为一个人和社会并不这样相连,而且可以随意接受或是拒绝神话,那就不了解神话的意义了。
\par Phaedrus认为,只有一种人能接受或是拒绝环境中的神话,这种人就是所谓的疯子。所以摆脱神话的人就会发疯。
\par 天啊!我明白了。我以前不知道是这样。
\par 他知道!他一定知道会发生什么事。
\par 真相开始显露出来了。
\par 这些片断就好像拼图一样,你把它们拼成几块大的图形,但是不论你多么努力还是无法拼成完整的图形。突然间有一块能把所有的都凑在一起,神话与疯狂之间的关系就是那块拼图。我怀疑以前是否有人说过:疯狂就是围绕在神话外围未知的领域。而他知道!他知道他研究的良质就在神话之外。
\par 是良质酝酿了神话的诞生。那就对了。那就是他所谓的“良质是持续不断的刺激,让我们创造出目前的世界,所有的世界,世界上的每一样事物”。宗教不是由人发明的,人是由宗教发明的。
\par 而人也创造对良质的反应。由这些反应当中人进一步了解了自己。你知道某些事后,良质就会给你刺激,你就会想把良质所给你的刺激界定下来,但是你必须根据自己所知的界定。所以你的定义是由你的知识组成的。情形必然是如此,不可能有其他状况。于是神话就这样展开了。根据已知的类推。神话就是不断地累积这种类推。它搜集了人类所有的知识——它是一辆装满了意识的列车,而良质则是引导这辆列车的铁轨,在这辆列车之外是疯狂的领域。他知道要了解良质就必须离开神话。这就是为什么他觉得会出意外。他知道有事情要发生了。
\par 我看到Chris从树林里回来,看起来十分轻松愉快。他拿了一块树皮给我看,问我是否可以留作纪念。我不喜欢留这些东西在车上,因为回到家的时候就会丢掉。但是这一次我答应他了。
\par 过了几分钟之后,我们顺着这条路骑到了山顶,然后又笔直地往山谷落下。
\par 一路风景十分优美。我觉得这个山谷和美国其他的山谷完全不同。往南边一点就是所有葡萄美酒的产地。山坡像波浪一样起伏,呈现出优美的曲线,而路也是蜿蜒曲折。我们的身体和车子缓缓地顺着山路向下走,同时向路边倾斜过去,几乎可以碰到树叶和树枝。高山地区的岩石和枞树远远落在身后,在我们周围是平缓的山坡和葡萄树,还有许多紫色和红色的花朵。从山谷冒出了浓郁的雾气,那是森林的气息和花香融合在了一起。在遥远的那一端,则是看不到但可以微微嗅得到的海洋气息……
\par ……我如此地深爱着这一切,却怎么会疯了呢……
\par ……我不相信!
\par 是神话。神话就是疯狂。这是他所相信的。神话认为这个世界的组成成分是真实的,但是这个世界的良质是虚假的。这就是疯狂。
\par 他相信在亚里士多德以及古希腊哲人之中,他找到了最初塑造这种神话的人,是他们让我们把这种疯狂视为真实。
\par 是它了,就是它了,它把这一切都联结起来了。此时终于如释重负。要得到这种结论非常困难,得到之后就有一种筋疲力尽的感觉。有时我觉得是自己得到的结论,有时又不确定。有时我知道不是靠自己的力量,但是,神话和疯狂,还有二者之间的关系,还有我所确信的这一切,都是从他而来。
\par 我们经过一片丘陵地,来到了梅德福城,这里有一条高速公路通往格兰茨帕斯城。这时已经夜幕低垂,迎面吹来的风非常强劲,我们向上骑的时候很吃力,甚至需要把节流阀完全打开。到达格兰茨帕斯城的时候,我们听到一声巨响,于是赶紧停下车来,发现链条护罩绞进了链条里。现在护罩完全变形了,情况不太严重,但是需要费好一番功夫才能换好。其实几天之内就要把车给卖了,换它可能很傻。
\par 格兰茨帕斯似乎足够大,明早应该能找到摩托车修理店。当我们抵达的时候,我只想找个汽车旅馆休息。
\par 从Montana的波斯曼出发后,我们还没有睡过床。
\par 于是我们找到一间汽车旅馆,有彩色电视、温水游泳池,还有第二天早上可以用的咖啡壶、香皂、白毛巾以及铺了瓷砖的浴室和干净的床。
\par 我们在床上躺下来,Chris在床上跳了一阵子。我记得小时候在床上跳可以缓解不少压力。
\par 不管怎样,明天这些都会有结果,或许,但不是现在。Chris跑下去游泳,而我静静地躺在干净的床铺上暂时把一切抛开。
\subsection*{29}
\par 自从离开波斯曼以来,我们把各样东西不断地从鞍囊和背袋里拿进拿出,损坏了不少。在晨曦中,我把它们全都摊在地板上,看上去一团乱。塑胶袋里面含油的东西破掉了,油漏出来,浸到一卷卫生纸上。衣服也被压得都是皱纹,好像很难再平整。防晒油的软管也破了,在弯刀鞘上留下一堆白色的乳液,到处都是香气。燃油膏管也破了,弄得一团糟。我在随身的笔记上写下:要为受压易破的东西买专用盒,洗衣服,买指甲刀、防晒油、燃油膏、链条护罩、卫生纸。在结账前要做完这些事,看看还真不少。于是我把Chris摇醒,叫他起床,我们要去洗衣服。
\par 到了洗衣店,我教Chris如何操作甩干机,如何启动洗衣机以及如何使用其他的功能。
\par 所有的东西都买到了,只差链条护罩。卖零件的人说他们没有,很可能也不会进这样的货。我想前面所剩的路程不多了,但是如果没有链条护罩就会溅得一车都是污泥。这样也很危险。既然有这种可能,我就不能撒手不管,所以还是决定要把护罩修理好。
\par 我看到一家焊接店,就走了进去。
\par 这是我见过的最干净的一家焊接店。店铺后面是高大的树和长长的草,让人觉得像是一间乡村的铁匠铺。每一件工具都被小心地挂在墙上。一切都很干净。店里没有人,所以我打算过一会再来。
\par 我骑回洗衣店,看Chris是否把衣服洗好了。然后慢慢地沿着宜人的街道找地方吃饭。这里的交通很拥挤,大部分的车辆都保养得很好,驾车人也十分机警。这就是西海岸。
\par 在镇边上我们找到一家餐厅,我们坐在一张铺了红白相间的桌布的桌子旁。Chris打开一份摩托车杂志,那是我在摩托车店买的,然后大声念出各项比赛的优胜者和骑摩托车横越大陆的消息。女服务生有些好奇地看了他好一会儿,然后又看看我,把视线移到我的靴子上,然后记下我们点的菜。她回厨房报了菜单,之后又出来在旁边看我们。
\par 我猜是因为这里没有别人,她才对我们这么注意。菜上来的时候,她往自动点唱机里投了几枚硬币。早餐端上来了,是松饼、糖浆和腊肠。啊——还有音乐。
\par Chris和我在聊摩托车杂志上的消息,为了压过点唱机的声音,我们尽量提高声音说话,就好像所有在路上旅行很久的人放松了聊天一样。我从眼角看见有人一直在注视我们。过了一会儿,Chris又问我一些问题。由于受到别人目光的干扰,我很难集中精神注意听他在说什么。点唱机播出来的音乐是西部民谣,关于一位货车司机……我和Chris结束了谈话。
\par 结账出来之后,我们骑上摩托车,服务生仍然在门里面望着我们,一副很寂寞的表情。她可能还不了解,有这样一副表情,她很快就不会再寂寞了。我用力踩发动器,然后猛冲出去,仿佛受了些挫折。去找焊接工之前,我需要一段时间抚平自己的情绪。
\par 老板已经回来了,他大概有六七十岁。他有一点轻蔑地看着我——和那女服务生的态度完全不同。我告诉他链条护罩的问题,过了一会儿他说:“我不替你卸它,你自己卸。”
\par 我照着他的话做,然后拿给他,他说:“里面都是油渍。”
\par 我在后面的板栗树下找到一根树枝,把所有的油渍都刮下来,弄到一个垃圾桶里,他站得远远地说:“那儿有溶剂。”于是我就用溶剂把护罩上剩余的油渍清除干净。我把护罩拿给他,他点点头,然后慢慢走过去,把焊接枪的调整器装好,然后看看火焰的大小,然后选择了另外一把,不紧不慢地。他拿起一支钢棒,我想他是不是要去焊接那片薄薄的护罩,一般像这样的金属片我不会去焊接的。我会用铜棒去铜镀。我会在上面打一个洞,然后用焊棒把它补上。
\par 我问他:“你难道不打算铜镀吗?”
\par 他说:“不打算。”他肯定觉得我是个多嘴的家伙。
\par 他点燃了焊枪,然后维持小小的蓝色的火苗。当时的情形很难描述,事实上,焊枪和焊棒晃动的节奏不同,然而焊枪一靠近,焊棒就立刻滴下橘黄色的溶液在护罩上。然后再换下一个地方。
\par 没有任何坑洞,你几乎看不出焊接的痕迹。我说:“焊得真好!”
\par 他说:“一块钱。”脸上毫无表情。
\par 我在他眼睛中看到一丝疑问,难道他在怀疑自己收费过高吗?不是,是一些别的东西……就是寂寞,和那位女服务生一样。或许他认为我在胡说八道,谁还真正懂得欣赏这样的手艺呢?
\par 我们把行李收拾好,现在离开旅馆大概也该到结账的时候了。很快我们又沿着海岸边的美国杉林区由俄勒冈州进入加州。路上车流汹涌,非常拥挤,我们几乎没机会切入。天气又转凉了,而且暗了下来。我们把车停下来,穿上毛衣和夹克,但是还是觉得寒风刺骨。气温大概只有十度左右。天气冷得让我的思想也冻僵了。
\par 在城里我也看到了寂寞的人,在超级市场里,在洗衣店里,从汽车旅馆结账出来的时候,甚至在美国杉林里,那些来自各方的露营者,处处都看到寂寞的人。他们有的是退休了,径自看着树,看着海。你会在一张陌生的脸上突然捕捉到一丝搜寻的眼神,然后立刻又消失了。
\par 我现在已经看过许多这种寂寞的表情。而这里正是号称世界上人口最拥挤的地区之一。这似乎有些矛盾。在东西两岸的大城里,寂寞的情形最严重。然而在人口稀少的俄勒冈州、爱达荷州、Montana和Dakota,你本来很可能以为人们会更寂寞,但情况并非如此。
\par 我认为这是因为身体上的距离和寂寞毫不相关,造成寂寞的原因是心理的距离。在Montana和爱达荷州,身体上的距离虽然很遥远,人们心理的距离却很近,而在这里正好相反。
\par 我们现在身在美国的主要都市里,这儿有纵横交错的高速公路,还有大型的飞机场,以及电视和电影明星。然而在这里,大部分的人很可能对周遭的一切毫无知觉,大众媒体让他们以为身边的事物是不重要的,这就是他们寂寞的原因。你可以从他们脸上看到寂寞。先是他们眼中闪过一丝搜寻的神情,然后一旦看见你,你对他们来说便只不过像一个物体,算不得什么,而不是他们想要寻找的对象,因为你不是电视上的人物。
\par 但是在我们经过的美国其他地区,比如说像穷乡僻壤的地区,像曾经有中国人挖过壕沟的地区,还有乘马车的地区,整片是山脉的地区,人们有更多沉思的机会,孩子会去玩松果,也会有大黄蜂在四处飞舞。我们头上顶着一片绵绵无尽的蓝天。周围的一切都深深融入我们的生活之中。所以从来不觉得寂寞。
\par 一两百年前人们的生活与这种情形比较相像。人口少多了,而寂寞的感觉也不会这么强烈。当然,我这么说毫无疑问地是太简略了,但是如果举出一些确切的证据,你就会明白我所言不虚。
\par 造成这种寂寞的主要原因就是科技,就是科技的产品,像电视、喷射机、高速公路等等——但是我希望说明一点,真正的祸首并不是科技本身,而是科技所带来的一种趋势,物化了人与人之间的关系。也就是在科技背后截然二分主客观的看法造成了这种现象。这就是为什么我要费尽心力借着科技来改变这种现象。一个知道如何怀着良质去修理摩托车的人,要比不具有这种情怀的人有更多的朋友,而且他的朋友不会把他视为一个物体。良质总是能够消灭主客体之间的距离。
\par 如果有的人工作很枯燥——或者手中的工作迟早都会变得很枯燥——为了让自己过得愉快些,他会开始选择良质,然后悄悄地为自身着想而追寻这个目标,因此使自己手中的工作变成一种艺术。他很可能会发现,自己成了一个更有趣的人。而对他周围的人来说,他也不再是物体,因为他选择了良质。不只他自己和工作受到影响,在他周围的人也会逐渐改变,因为良质会像水波一样荡漾开来。他手中的工作具有良质,于是会让人有不同的感受。感受到的人会觉得这种感受不错,就可能会把它传播给别人,这样一来良质就会不断繁衍开来。
\par 我个人的感觉是,这就是世界不断改进的方法:让个人越来越珍惜良质,仅此而已。我不愿意通过大规模的社会运动聚集许多人,进行一些大型的活动,然而却忽略了这种个人的良质。让我们重新考虑这些活动。它们当然很重要,只是我们必须意识到,它们要以具有良质的个人为基础。过去我们曾经拥有这种属于个人的良质,但是却把它当作一种天生的感受力,而没有深入地去了解它。现在它快要枯竭了,几乎每一个人都要丧失良质了,我想此刻我们应该开始重视这项极大的资源——个人的价值。许多年来,有一些政治家曾经提出过相似的学说,但是我并不是他们的一分子。当然,他们提倡个人真正的价值,而不把它当作输送更多金钱给有钱人的途径,就这方面而言,他们是对的。我们的确需要重新珍惜个人的操守、对自我的信赖以及老式的进取心。我们真的很需要。我希望在这一次Chautauqua当中能够指出某些方向。
\par 而Phaedrus却从这个观念出发,走出了不同的路。我认为那条路是错误的。
\par 但是如果我在他的情况下,很可能也会走相同的路。他觉得解决这个问题需要创造一套新的哲学,或者他认为比这个范围更广——创造新的理性——那么二元化的科技思想所造成的丑陋、孤寂和空虚的心灵就会消失。理性不再和价值无关,理性必须受制于良质,他确定他会找到其中的原因,这并不需要完全深入到希腊哲学家的思想当中。这些希腊神话曾经深深影响我们的文化,造成今日科技的趋势——做虽然合理但是没有任何好处的事。这就是问题的根源。就在这儿。许久前我曾经提到,Phaedrus在追求理性的鬼魂。这就是我的意思。理性和良质分家了,而且互相对立,良质被迫屈居于理性之下。
\par 开始下了一点雨,但是还不需要停车,这只是下小雨之前的一点毛毛雨。
\par 我们已经离开高大的森林,眼前是一片广阔无垠的、灰蒙蒙的天空。
\par 于是,我又一次重读亚里士多德的著作,想要从中找到Phaedrus提过的可怕的思想,但是一无所获。我看到的多是一堆乏味的分类,用现代的知识似乎很难加以评价。其间的结构非常薄弱,就好像博物馆里希腊人的陶艺品,显得很原始。我相信,如果我了解得更深入,我会发现它一点也不原始,但是没有全盘的了解,我看不出亚里士多德的著作为何能称得上是伟大的书,或者为什么会引起Phaedrus的震怒。很明显,我看不出来亚里士多德作品的价值所在。然而伟大的书籍早已为世人所知,Phaedrus的作品却没有出版,所以我的责任之一就是要把他的思想详细地写下来。
\par 亚里士多德认为,修辞学是一种艺术,因为它是一种理性的知识系统。
\par 亚里士多德的说法让Phaedrus非常震惊,他已经准备好要深入研究这位世人公认的伟大的哲学家,了解他极为复杂的思想体系,研究其中深刻的意义。然而他却受到了这种说法的迎头痛击。竟然会出现这种胡说八道的论调!真让他大大吃了一惊。
\par 他又继续读下去:修辞学可分成特定实证和一般实证。特定实证可分成实证方法和实证种类。实证的方法有人为的实证方法以及非人为的实证方法。人为的实证方法包括道德实证、情感实证、逻辑实证。道德实证有应用知识、美德及善意,而善意需要有情感方面的知识。针对忘记情感方面内容的人,亚里士多德列了一张清单。其中包括生气、轻蔑(分成轻视、憎恨和侮辱)、温柔、爱、友谊、恐惧、信心、耻辱、妒忌、施舍、仁慈、怜悯、义愤、无耻、竞争和忽视。
\par 你还记得我在南Dakota对摩托车进行的描述吗?我把摩托车的各种零件和功能详细地列出来。你发现其中相似的地方了吗?Phaedrus现在深信,这是用这种方式写作的起源。亚里士多德一页又一页地重复这种文体。就像三流的技术指导员,把一切都列出来,然后仔细解说其间的关系,还不时自作聪明地指出其中一些新的关系,然后就等待下课,这样他就能在下一堂课重复同样的讲述。
\par 对这些讲述,Phaedrus没有产生任何怀疑,也没有任何敬畏之心,只是嗅到了学究陈腐之气。亚里士多德难道相信他的学生读过这些名词和彼此之间的关系之后就会变成优秀的修辞学者吗?如果他并不这样认为,那么他真的认为自己在教修辞学吗?Phaedrus认为他的确是这么想的。从他的文章里面嗅不出他对自己有任何怀疑。Phaedrus反倒看见他极为满意,不断叙述这些名词,做各种分类。在阅读亚里士多德作品的过程中,Phaedrus自始至终都十分震惊。如果不是亚里士多德早在两千多年前就已死去,他很可能会痛快地把他给宰了。因为他发现亚里士多德树立了一个坏榜样,在历史上有数以百万计的无知而自满的老师,他们运用这种愚笨的分析模式,这种盲目而机械的命名,无情地把学生的创造力给抹煞了。现在进入任何一间教室,你都会听到老师不断地分析又分析和解释其间的关系,然后树立许多的原则,研究各种方法。你听到的只不过是亚里士多德数千年前的鬼魂在说话——那是一种缺乏生命力、赞成二元论思想的声音。
\par 研究亚里士多德的课在一间阴暗的教室里进行,教室里有一张非常大的圆桌,对街有一家医院。午后的太阳从医院屋顶上斜射过来,阳光似乎穿不透玻璃窗上厚厚的灰尘和都市里污浊的空气。所以这间教室给人一种沮丧的感觉。
\par 上课的时候,Phaedrus发现木桌正中有一道裂痕,好像已经裂了许久,但是没有人想去修理它。一定是太忙了,有更多重要的事要处理。在快下课的时候,他终于问老师:“我可以问一些亚里士多德修辞学方面的问题吗?”
\par 教授说:“你得先读他的作品。”Phaedrus从哲学教授的眼中看到和他注册的时候相同的神情。似乎是在警告他:你最好回去把他的作品仔细读过。
\par 现在雨下得越来越大,我们停下来盖好头盔面罩,然后慢慢地向前骑去。
\par 一路上必须特别留意马路上的坑坑洞洞、沙石和被油污染的路面。
\par 下一个礼拜,Phaedrus读过亚里士多德的作品,准备好攻击他的理论,就是说:修辞学能够成为理性的思想体系,是一种艺术。如果用这种标准来衡量,通用公司所制造的汽车就是一种艺术,而毕加索的作品反倒不是艺术了。如果亚里士多德的作品果真有深刻的意义,那么这正是让它们显现出来的大好时机。
\par 但是他的问题一直都没能提出来。
\par Phaedrus举起手,他突然看到老师的眼睛里闪过一丝恨意,然后突然有一位学生打岔说:“我想这里有一些模糊不清的地方。”
\par 教授制止他说:“先生,我们来这里不是要研究你想什么,而是要研究亚里士多德想什么。”教授当面羞辱他,“一旦我们要研究你的思想时,我们会在学校里专门开一门课。”
\par 没有人说话,这位学生吓得说不出话来,别人也是一样。
\par 然而哲学教授还没有攻击完,他手指着这位学生,问他:“根据亚里士多德的说法,特定修辞学分成哪三种?”
\par 大家更沉默了,这位学生不知道答案。
\par “那么你没有读过他的作品了,是不是?”
\par 从他的眼神当中,可以看出他早已准备这么做,接着他就指向Phaedrus。
\par “你,先生,特定修辞学分成哪三种?”
\par 但是Phaedrus是有备而来,他很平静又沉稳地回答:“讨论的、审议的以及展示的。”
\par “什么是展示的技巧?”
\par “就是分辨异同、赞美,以及详述的技巧。”
\par 哲学教授慢慢地说:“啊——是的。”
\par 然后整个教室一片寂然。
\par 其他的学生吓了一跳,他们都奇怪究竟发生了什么事。只有Phaedrus知道,或许哲学教授也知道——这位无辜的学生替他承担了原来准备给他的打击。
\par 现在每一个人的表情都变得小心翼翼,想要预防老师提出更多同类的问题。
\par 哲学教授犯了一个错,他把自己的权威浪费在这名无辜的学生身上。而Phaedrus这个有意攻击他的人却逍遥法外,似乎越来越嚣张。由于Phaedrus没有问任何问题,所以他无从下手。现在他看到了自己提出的问题Phaedrus是怎样回答的,所以他不愿意继续问下去。
\par 那位无辜的学生低头看着桌子,脸涨得通红,然后用双手盖着脸。Phaedrus看到他受窘的情形十分气愤。他在自己的班上从来不会这样对待学生。原来这就是他们芝加哥大学教古典文学的方法。Phaedrus现在认清了哲学教授的面目,但是哲学教授却没有认清他。
\par 天空仍下着雨,一路上到处可见各种标志。我们来到了加州的克雷森特城,这里既寒冷又潮湿。Chris和我看着码头那灰色建筑另一端的海洋。这是我们这些天努力的目标。然后我们到一家铺着红色华丽地毯的餐厅吃饭。在镶着花边的菜单上,每一道菜的价格都非常贵。
\par 我们是这里惟一的顾客。静静地用过餐、付过账后我们就上路了。现在我们向南走,一路上很冷而且雾很大。
\par 下一堂课那位无辜的学生就没有再来。这个结果并不令人意外,整个班的学生都吓坏了。一旦发生这种情形,这个结果是不可避免的。因此以后的每一堂课,都只有哲学教授一个人唱独角戏,他自顾自地说着,大家脸上毫无表情,看不出是同意还是反对。
\par 哲学教授似乎很清楚发生了什么事。原来他看Phaedrus时总是怀有敌意,现在则带着一丝恐惧。他了解目前教室里的情形,一旦时机一到,他很可能会得到报应,而不会有学生同情他。所以既然无法避免被攻击,他就尽可能地不让这种情况发生。为了不让事情恶化,他必须更小心谨慎,每一句话都得完全正确。Phaedrus也了解这一点。虽然他保持沉默,但他明白自己处在非常有利的地位。
\par 这期间Phaedrus很努力地研究,而且学习的速度非常快,但是再也不开口说话了。然而如果认为他是好学生,那么你就错了。一个好学生会用公正的态度去面对学问,但是Phaedrus却不然。他别有企图,所以他要找的资料都是对自己有帮助的,以及如何击倒对方的方法。
\par 他对其他人所著的伟大书籍不感兴趣。
\par 他之所以在这里,只是他想要写出一本属于他自己的伟大书籍。他对待亚里士多德的态度有些不公平,就好像亚里士多德对他的后继者不公平一样——因为亚里士多德的言论破坏了他的思想。
\par 亚里士多德如何破坏了他的思想呢?他视修辞学为自己思想体系中非常不重要的一环。它是实用科学的一支,和亚里士多德主要的研究——理论科学,有些关系。作为实用科学的一支,它和其他主要真理、善与美之间就隔绝了。所以,在亚里士多德的体系当中,良质和修辞学完全分离了。亚里士多德如此轻视修辞学,但写作时又极尽华丽之能事,这让Phaedrus在读他的著作时,忍不住要想办法轻视他、攻击他。
\par 要做到这一点毫无问题。亚里士多德在历史上一直受到各种挑战和攻击。
\par 攻击他的荒谬和可笑就好像在桶中射鱼,轻而易举,无法让人满足。如果Phaedrus对亚里士多德的成见不是这么深,他也许能学到一点亚里士多德创新知识的方法——这正是委员会存在的目的。
\par 然而如果不是如此致力于研究良质,那么一开始他就不会来这里。所以他实在没有机会去研究亚里士多德的方法。
\par 在哲学教授讲课时,Phaedrus注意听他说有关古典形式和浪漫形体的问题。
\par 在讨论到辩证法的时候,教授似乎非常不安。虽然用古典的说法找不到原因,但是Phaedrus浪漫的直觉告诉他,他闻到了一样东西——问题的根源。
\par 辩证法吗?
\par 亚里士多德一开始就用非常神秘的方式讨论辩证法,他认为修辞学是辩证法的一体两面,他这么一说,显得它仿佛极为重要,至于为什么会这么重要,亚里士多德却没有进一步解释。接下来有一堆并不连贯的议论,让人觉得似乎有一大堆思想被遗漏了,或是把资料拼错了,或是印刷的人遗漏了,因为不论他读了多少遍也无法完全了解。人们只发现亚里士多德非常注意修辞和辩证之间的关系。就Phaedrus看来,这表示他和教授一样不安。
\par 哲学教授也曾为辩证法下过定义,Phaedrus专心听着,但左耳进右耳出,这好像是哲学叙述不完整时的必然现象。
\par 后来有一位学生和他有同样的困扰,于是要求哲学教授重新替辩证法下定义。
\par 这时教授瞄了Phaedrus一眼,仍然有些害怕,变得非常急躁。Phaedrus开始怀疑,辩证法是否有一些特殊的意义,才构成了它的重要性——即能够左右论点的成立与否。的确,它有。
\par 辩证法一般是就对话本质的层面而言,也就是两个人之间的对话。在今天来说,它表示逻辑的讨论,两人经由交互盘问而找到真理。这正是苏格拉底和Plato在《对话录》里的讨论方式。Plato深信辩证法是追求真理的惟一方法,惟一的一种。
\par 这就是为什么辩证法是个关键词。
\par 亚里士多德攻击这种信念,认为辩证法只适用于某些场合——追寻人类的信仰,找到事物的永恒形式,也就是真理。
\par 对于Plato来说,它是固定不变的,是既成的真实。亚里士多德认为还有一种科学方法,它能研究物质的形式,然后找到物质的本质,但是它是可变的。这种形式与本质的二元论,以及这种科学方法,是亚里士多德的中心思想。于是,对于亚里士多德来说,把辩证法从苏格拉底和Plato所尊崇的高度拉下来,是非常必要的。辩证法曾是,并且仍然是一个关键词。
\par Phaedrus猜测,亚里士多德贬低辩证法,也就是从Plato认为的它是找到真理的惟一方法,变成认为它是修辞学的另一面,这很可能会激怒现代的Plato主义者,甚至会激怒Plato本人。由于哲学教授并不清楚Phaedrus的角色,因而十分不安。他很可能害怕Phaedrus和Plato的拥护者会一起攻击他。如果真是如此,他倒不必担心。因为Phaedrus并不会因为辩证法被视为与修辞学平等而愤怒。他气愤的是修辞学反倒被拉下来与辩证法平行。这就是当时二者的状况。
\par 而把这个状况理清的当然就是Plato。
\par 但很幸运,在南芝加哥这个阴暗的教室中,下一个出现在中间有裂痕的圆桌上的会是Plato。
\par 我们现在沿着海岸前行。空气不但寒冷而且潮湿,让人的情绪无法振作起来。雨停了下来,但是看天色仍然会继续下雨。走着走着,我看到了沙滩,有一些人在上面走动。我觉得有些累了,就停下来。
\par Chris从车上下来,说:“我们停下来做什么?”
\par 我说:“我累了。”迎面吹来的海风十分寒冷,还在沙滩上形成了不少沙丘。
\par 因为下雨的关系,天色十分昏暗而且很潮湿。我想必须在这里停下来。于是我找了一个地方躺下来,才觉得温暖些了。
\par 但是我却睡不着。有一个小女孩站在沙丘上望着我,似乎希望我过去和她玩。过了一会儿,她自己走开了。
\par 这时Chris回来了,要我继续向前走。他说他在岩石上看到一些很有意思的植物,碰到它们的时候触须会动。我就和他一起去看。在海边的岩石丛里有海葵。它们是动物不是植物。我告诉他海葵的触角会让小鱼昏过去。潮水一定完全退了,否则我们不会看到这些动物。
\par 这时我瞥见那个小女孩站在岩石的另外一边,手上拿着一个海星,她的父母手里也拿了一些。
\par 我们又骑上摩托车朝南去。雨下大了,于是我把面罩拉下来,以免雨打在脸上。但是待在里面很闷,我不喜欢,所以雨一变小,我立刻把面罩打开。傍晚之前我们必须要抵达阿克塔城,但是路太潮湿,我不想骑得太快。
\par 我记得是柯勒律治曾经说过:“一个人如果不是Plato的信徒,就是亚里士多德的信徒。”不能忍受亚里士多德永无止尽的分析,必然会喜好Plato天马行空的概念。不能忍受Plato高远的理想主义,必然欢迎亚里士多德的实际。Plato认为得道非常重要,每一代都不断有这样的人出现,殚精竭虑地寻找宇宙存在的源头。而亚里士多德则代表了维修摩托车的技术人员,他喜欢世间万象。
\par 从这个角度而言,我自己就属于这一派,我喜欢从周遭的事物里找到佛性。然而Phaedrus天生就是Plato一派,所以课堂上开始讨论Plato的时候,他就觉得愉快多了。他所谓的良质和Plato所谓的善非常相似,要不是他留下了很多笔记,我很可能会以为二者是相同的,但是他否认了这一点,而且我及时发现了这个差别有多么重要。
\par 然而,在课堂上大家所讨论的并非Plato的善,而是Plato对修辞学的看法。Plato认为修辞学与善无关,它是属于恶的一部分。Plato最恨的人,除了暴君之外就是修辞学家。
\par Plato在《对话录》中首先提到的是高尔吉亚。Phaedrus终于找到了感觉,他知道这正是自己追寻之境。
\par 他一直觉得自己在被某一种未知的力量——弥赛亚\footnote{Messiah,犹太人盼望的复国救主}的力量向前推送。十月来了又去,日子变得十分缥缈,断断续续。除了谈到良质的时候之外,对他而言,什么都不重要。重要的是他有一个全新的真理将要产生了。这真理足以粉碎许多学说,而让世界大为震撼。不管世界喜不喜欢它,都必须接受它。
\par 在《对话录》当中高尔吉亚是一名智者\footnote{Sophist,古希腊的修辞和演说老师。他们教导的是人类所有的知识来自于感官,由于感官经验人人不同,所以“人是衡量一切的标准”,真理只是各种不同的意见,道德也是相对的。他们强调要在政治和社会上获得成就,因而收取高昂的学费教导人们文法和修辞学},他接受了苏格拉底反覆的盘问。苏格拉底非常了解高尔吉亚的工作,以及他是如何工作的。但是在《对话录》的二十个问题当中,他一开始就问高尔吉亚什么是修辞学。高尔吉亚回答他,修辞学和讨论有关。在另外一个问题里他回答,它的目标就是要说服别人。在回答另外一个问题的时候,他说明了修辞学在法院和其他场合的地位。在另一个回答当中,他说它所探讨的问题就是公正与否的问题。这些都是一般人所说的智者的工作。现在苏格拉底运用辩证法,把它转化成了别的东西。修辞学变成了物体,物体就必然划分成各个部分,而各部分之间必然互相有关,这些关系是无法改变的。你可以很明显地从这些对话当中看见,苏格拉底如何运用分析的刀把高尔吉亚的艺术劈成碎片。而更重要的是你可以看到,这些碎片就是亚里士多德修辞学的基础。
\par 苏格拉底曾经是Phaedrus幼时心目中的英雄,看到这样的对话时,他不但很震惊而且很气愤。他在书旁写下自己的答案。这些对话一定让他非常沮丧。看到对方的回答,你不知道对话会如何接下去。有一次苏格拉底问高尔吉亚,修辞学家所用的词汇和哪一等级的事情相关?高尔吉亚回答,最伟大的和最好的。
\par Phaedrus立刻知道,这答案之中蕴藏着良质。于是在旁边写下“说得没错”。但是苏格拉底却认为这样的回答模棱两可,含混不清。他仍然无法明白。Phaedrus在旁边愤怒地写道:“骗子!”他还参照了另外一则对话,苏格拉底对那个答案很明白,一点儿都没有不清楚之处。
\par 苏格拉底并没有运用辩证法去了解修辞学,而是运用辩证法去摧毁修辞学,最起码是去破坏修辞学的名誉,所以他的问题根本不是真正的问题——它们只是言语的陷阱,让高尔吉亚和他的同道掉进去的陷阱。Phaedrus对这一点非常痛恨,希望自己当时就在现场。在课堂上,哲学教授注意到Phaedrus良好的表现和勤奋,于是认定他很可能不是一位坏学生。
\par 这是教授犯的第二个错。他决定要跟Phaedrus开一个小玩笑,问他对烹饪的看法。
\par 苏格拉底曾经告诉高尔吉亚,修辞学和烹饪都是煽动人的学问——是很卑微的思想——因为它们所诉求的是人的情感而非真正的知识。
\par 在回答教授的问题时,Phaedrus以苏格拉底的回答为准。
\par 这时从教室后面传来一位妇人偷笑的声音。Phaedrus十分不高兴,因为他知道教授想要用辩证法来打击他,正像苏格拉底打击他的对手一样。所以他的回答一点也不好笑,只想摆脱教授的阴谋罢了。Phaedrus早已准备好要朗读苏格拉底的论点。
\par 但是这并不是教授所要的,他想在教室里进行一场辩证法的讨论,而Phaedrus就是那位修辞学家,他会被辩证法玩弄。教授皱皱眉然后又试着问:“不是,我的意思是,你真的认为我们应该拒绝在最好的餐厅里享用一顿丰盛的美食吗?”
\par Phaedrus问:“你是问我个人的意见吗?”好几个月来,由于那名无辜的学生不再来上课,已经许久没有人敢在班上表达个人意见了。
\par 教授说:“没错。”
\par Phaedrus不吭声,想要找出答案。全班都在等待。他的思想在飞驰,不断过滤辩证法,仿佛一直在开棋局,发现这一手输了,然后又开另外一局,速度越来越快。但是全班的同学都静默无声。
\par 最后让他很难堪的是,教授放弃等待开始上课。
\par 但是Phaedrus听不进去,他在不断思索。借用辩证法他不断探测各种事物,发现新的分支和其他的分支。于是在不断发现辩证法中间所隐藏的邪恶和低级之后,他十分愤怒。教授看到他脸上的表情吓了一跳,然后有点惴惴不安地继续上他的课。Phaedrus仍然继续不断地搜索。他终于发现有一种邪恶深深地植根在他自己身上,就是假装想要去了解爱、美、真理以及智慧。但是它真正的目的不是去了解而是去利用,以让自己登上宝座。辩证法——就是这个篡位者。这就是他所看到的。这个暴发户和所有所谓的美善相斗,想要涵盖它然后加以控制,这就是它的邪恶之处。教授提早下课,然后火速离开了教室。
\par 在学生们静静地离开教室之后,Phaedrus独自坐在大圆桌旁,一直到太阳下山。教室里逐渐暗下来了。
\par 第二天他很早就到了图书馆,等着它开门,一进去,他就开始仔细地重读Plato的书,然后又去找那些一向不被人了解而且他一直轻视的修辞学家的书,而他接下来所发现的开始证实他的直觉是对的。
\par 已经有许多学者对Plato诅咒诡辩学家感到十分不安,委员会的主席自己就曾经提出,不能确定Plato的含意的批评家,同样也不能确定《对话录》中苏格拉底的对手所说的含意。亚里士多德曾经说过,Plato借用苏格拉底的名义把自己的话说出来,所以,我们大可以怀疑Plato也可能是通过别人的嘴把他自己的话说出来。
\par 其他古代作家的作品似乎对智者有不同的评价。许多老一辈的智者被派驻别国任大使,这当然表示他们有崇高的地位。而这一批智者对苏格拉底和Plato也没有不敬之处。后代的历史学家曾经认为,Plato之所以对智者恨之入骨,是因为他们无法和他的老师苏格拉底(这位实际上最伟大的智者)相比。这种解释很有意思,但是Phaedrus并不满意,因为人通常不会反感老师所属的宗派。
\par 然而Plato真正的含意究竟是什么呢?
\par 于是Phaedrus不断研究苏格拉底之前的希腊人的思想,想要找到答案。最后他终于发现,Plato对智者的恨牵涉到当时一场思想上的争斗。代表善的智者和代表真理的辩证学者为人类未来的世界走向而争斗。真理这一方赢了,而善输了。
\par 这就是为什么我们今天接受真实很少有困难,而接受良质的阻力则很大。
\par Phaedrus是如何得到这样的论点的呢?想要了解这些就需要解释:首先,你必须放弃这样一种观点,即最近的山顶洞人和第一位希腊哲人相距时间很短。由于这一段时间缺乏历史记载,所以往往让人产生这样的幻觉。但是早在希腊哲学家出现以前,也就是大约在我们现有记录的五倍时间之前,已经有很文明的社会了。他们有村庄、城市、车辆、马匹、市场、划分好的田野、农业的工具和家畜。他们所过的生活和今日农村一样丰富而充满变化。就像今日活在这些地区的人一样,他们不明白为何要把生活记载下来,或者他们曾经这么做过,只不过他们的记载从未被人发现,因而我们对他们一无所知。然而这个自由自在生活的黑暗时期却被希腊人无意中给打断了。
\par 早期希腊的哲学思想代表人类开始有意识地寻求不朽的事物。在那之前,所谓不朽的事物在神话的范围之内。然而这时,由于希腊人开始冷静客观地去观察周遭世界,因而培养出了抽象思考的能力。这让他们可以将古希腊的神话视作想像的产物而非真理。这种思维的能力从来没有在世界上出现过,因而将希腊文化提升到前所未有的境地。
\par 但是神话并没有结束,毁掉古神话的变成了新神话,而爱奥尼亚\footnote{Ionia,古代小亚细亚西部沿爱琴海海岸的一个地区}的哲学家将新神话转化成了哲学。它由新的角度显现出自身的永恒性,于是永恒不再是神明的专利。
\par 你也可以在不朽的法则之中找到永恒。
\par 重力定理就是其中之一。
\par 永恒的起源起初被泰勒斯\footnote{Thales,640?-461?B.C.,希腊哲学家,奠定几何学基础,致力天文学研究,认为水是万物的根源}学派的学者叫做水。阿那克萨哥拉\footnote{Anaximenes,公元前5世纪希腊哲学及科学家,认为空气是万物之源}学派的学者则叫它空气。毕达哥拉斯\footnote{Pythagoras,公元前6 世纪希腊哲学及数学家,相信灵魂不灭和轮回之法,主张“数”是万物的根本,万物因数的关系才产生了秩序}学派的学者则叫它数。他们是第一批不把永恒的起源视为物质的人。赫拉克利特\footnote{Heraclitus,535?-475?B.C.,希腊哲学家。著有《自然论》,主张万物轮回,火是不断变化的典型,是万物的根源。倡言生命短暂的悲观论}学派的学者叫它火。同时也把火的变化当作是起源的一部分。他认为宇宙的存在就是一种对立,以及对立二者之间的互动。他认为宇宙之间存在着一,也存在着万物,而一是宇宙的起源,隐藏在所有的事物当中。阿那克萨哥拉则首次认为一就是人类的心灵。
\par 巴门尼德\footnote{Parmenides,510?-450?B.C.,希腊哲学家,存在学派创始人}说得更清楚,他第一次提到这个永恒的起源,这个一、真理、上帝是和现象以及意见分开的。而这种分开的重要性,以及它对后世的影响难以言喻。这是古典的思想第一次承认它浪漫的根源,而且宣称:“善与真并不必然同一。”然后继续独自前行。阿那克萨哥拉和巴门尼德有一位信徒叫做苏格拉底,日后完整地诠释了他们的思想。
\par 在这里需要了解,直到这时为止并没有所谓的心与物、主体与客体、形式与本质。这些划分只不过是日后辩证法所发明的玩意儿罢了。现代人很可能会替这二分法辩护:“这种二分法原本就在那儿,只等希腊人去发掘。”然后你问道:“在哪儿?请指出来!”现代人很可能会迷糊了,心想究竟这是在干什么,然后依然相信这样的二分法。
\par 但是Phaedrus认为它们并不存在,它们只是鬼魂,是现代神话中不朽的神衹。
\par 由于我们活在其中,因而认定它是真实存在的。事实上它们正如被它们所取代的神人同形同性论一样,只不过是人的艺术创作。
\par 截至目前为止所提到的这些生于苏格拉底之前的哲学家,都企图在他们观察的世界中找出永恒的起源。这些学者可以统称为宇宙学派。他们都承认宇宙中有这样的起源存在,至于这个起源是什么则众说纷纭。赫拉克利特学派认为永恒的起源是变与动。而巴门尼德的门徒芝诺\footnote{Zeno,公元前5世纪希腊哲学家,为巴门尼德的弟子}则通过一连串矛盾的议论证明,动与变是幻觉,真正恒常存在的是寂然不动。
\par 而宇宙学者之间的争议却因为另一派人士的出现而得到解决。Phaedrus认为他们是早期的人道主义者,他们是老师,但是他们教导的并非定理,而是对人的信仰。他们的主题不是绝对的真理,而是人的进步。他们认为所有的定理真理都互相有关,而人是衡量一切的标准。
\par 这些就是著名的智者,古希腊的智者。
\par 对Phaedrus而言,了解智者和宇宙学者之间的冲突使他对Plato的《对话集》有了全新的了解。苏格拉底不仅仅只是在真空的环境当中陈述他的理想,他身处两派的斗争之中,一派认为真理是绝对的,一派则认为真理是相对的。他使出浑身解数去战斗,而敌人就是智者。
\par 这样一来,Plato对智者的敌意就有意义了。因为他和苏格拉底都在为宇宙学者的永恒起源进行保卫战。他们认为智者是一种堕落,他们所保卫的真理和知识超越任何人的思想。这正是苏格拉底为之而死的理想——是世界上起初只有希腊人拥有的理想。它仍然是一种非常脆弱的学问,很可能会完全消逝。
\par 于是Plato毫无顾忌地对智者大加挞伐。并不是因为他们是卑微而不道德的人——因为在希腊很显然还有更低级更不道德的人,他却完全忽略了。他之所以诅咒智者,是因为他们威胁到了人类刚开始的对真理的追逐。就是这么回事儿。
\par 于是,苏格拉底壮烈的牺牲和Plato蹩脚的文章所带来的世界,就是我们今日所知道的西方世界。如果不是在文艺复兴时期重新发现了科学的真理,我们和史前时代人类的水准差不了多少。
\par 而科学思想、科技以及其他人类系统化的作为就是其中的中心思想。
\par 然而Phaedrus明白,他有关良质的理论和这一切是冲突的,反而与希腊的智者较为接近。
\par “人是衡量一切的标准。”的确,这就是他所说的良质。人不像唯心主义者所说的那样,是一切的源头。它也不像唯物论者和物质主义者所认为的那样,是被动的观察者。创造世界的良质呈现为人和自身经验之间的关系。人类是创造万物的参与者。人类是衡量一切的标准——这一点很吻合。而他们也教导修辞学——这也很吻合。
\par 而惟一和他所说的以及和Plato对智者的评论有出入的是,他们教导伦理道德的职业。所有的情况都显示,这是他们教导的核心。但是如果他们所教导的伦理道德是相对的,那该如何教导呢?如果说伦理道德暗示了什么,它就暗示着绝对的伦理道德。如果一个人对正确行为的认识每天都在改变,或许我们可以敬佩他头脑灵活,但是他的道德却值得怀疑。这样一来,他们如何从修辞学中找到伦理道德呢?这一点从来没人解释过。有一些东西遗失了。
\par 为了寻找答案,Phaedrus又去读了许多古希腊历史。同样地,他还是寻找对自己有利的条件,然后把不利之处都排除掉。他读到季多所著的《希腊人》,这一本蓝白相间的平装书是他花五角钱买的。他读到一段描述荷马英雄精神的文字。他们生在苏格拉底之前的时代。这些篇章使Phaedrus突然开了窍。只需稍加回忆,他仿佛就能看见他们仍然活着。
\par 《伊里亚特》就是叙说特洛伊城被围困的故事。这座城最后被攻陷了,而保卫家乡的人也在战争中阵亡了。赫克托是领袖,他的妻子对他说:“你的抵抗必然导致灭亡。你对襁褓中的儿子和你忧郁的妻子没有怜悯之心。她很快就会变成寡妇,敌人很快就会把城攻破,杀掉你。要让我失去你,还不如死。”
\par 她的丈夫回答她:“我很清楚这一点,而且很确定的是:圣城特洛伊即将灭亡。城中人也即将毁灭,包括普里阿摩斯王\footnote{Priam,特洛伊末代国君}和富裕的百姓。但是我并不会为了特洛伊的百姓、赫卡柏皇后、普里阿摩斯国王以及我那许多高贵的弟兄们而过分哀伤。他们都会被敌人屠杀然后躺在沙土之中。至于你,深褐色皮肤的敌人会把你带走,让你哭着离开,结束自由的日子。之后,你会来到阿戈斯,然后在另一个女人的主宰之下工作,过着替别的女人挑水砍柴的日子,在监禁之中忍受痛苦:你会受到各种奴役。然后有人看到你在哭泣就会说:‘这就是赫克托的妻子,他曾经是特洛伊人中最高贵的勇士。’而他们会这样说:丧失了这样一个丈夫,然后还要面对这样的奴役,真是太不幸了。但是我宁愿自己死去,宁愿厚厚的黄土覆盖在我身上,也不愿听到你的哭泣,听到那些施加在你身上的暴行。”
\par 英姿勃发的赫克托这样说着时,伸出手臂搂着他的儿子,但是孩子吓得尖声大叫,拼命地躲回奶妈的怀里。因为他很害怕父亲此时的样子——他十分激动,他头盔上的马鬃晃动得非常剧烈。
\par 他的父亲大笑起来,他的母亲也笑了起来。于是赫克托把头盔拿下来放在地上。
\par 把儿子抱在怀里逗弄,并且亲吻他。他向宙斯祈求,也向其他的神祈祷:宙斯和所有的神明啊!请保佑我的儿子,让他成为所有特洛伊人当中最勇敢的、最孔武有力的勇士,让他能够统治这个城市。当他从战场上回来的时候,愿百姓们会说:“他远胜过其父。”
\par “是什么使希腊的战士表现得这样神勇?”季多提出这样的疑问。“并不是我们所认为的责任感——对别人的责任感,而是对自己的责任感。他们努力追求的目标被我们翻译成伦理道德。然而,希腊原文却是指卓越……这个词有许多值得讨论之处。它贯穿了希腊人整个的生活。”
\par 这就是良质的定义,早在辩证学者运用文字陷阱之前的一千年就已经存在了。如果有人还不了解它的意义,那么他要么是在撒谎,要么是对人类的命运从来就漠不关心。我们不值得为这种人进行任何解释。为什么会产生“对自我的责任感”?关于它的描述让Phaedrus也很感兴趣,它几乎完全与印度教所说的“惟一”相对应。那么,是否印度教的“惟一”与古希腊的“伦理道德”就是同一体呢?
\par 这时Phaedrus迫切地想继续读下去。
\par 于是他读到……这是什么!?……“我们翻译成伦理道德的希腊原文是指‘卓越’。”
\par 他像被电击了一样。
\par 良质!卓越!印度的惟一存在!这正是希腊智者所教导的!并不是相对主义的伦理,也不是原始的道德,而是卓越。早在理性教会之前,早在本体出现之前,早在形式之前,早在心物之前,早在辩证法之前,良质就一直是绝对的存在。他们是西方世界最早的一批学者,就已经在教导良质了。他们所选择的媒介就是修辞学。这正是他一直在研究的范畴。
\par 雨小多了,所以我们能看到地平线,遥远的天边有如此明显的一条线,清楚地区分开了浅灰的天空和深灰的海水。
\par 季多针对古希腊人所谓的卓越进一步讨论。“在我们读到Plato作品当中的这个词时,”他说,“我们把它翻译成伦理道德,因而完全丧失了它的原意。伦理道德,至少在现代英语中,完全是一个道德方面的词语。但是它的希腊原文几乎没有别的意思,只是指卓越而已。”
\par 所以《奥德赛》中的英雄是伟大的战士,足智多谋,随时能滔滔不绝地演说。他具有坚强的意志和无限的智慧,他知道要承担神明所指派的工作不可以有太多的抱怨。他也能自己建造并驾驶一艘船。用犁拉出来的痕迹和别人一样直,他能投掷铁饼击败年轻的吹牛家,也会拳击、摔跤和赛跑。他还会剥牛皮、剁牛肉,把牛煮了吃。同时也会因为听到美妙的歌曲而感动流泪。事实上他是一个非常杰出的万能选手,他已经超越了希腊文里的“卓越”。
\par “卓越”暗示着对生活的完整或惟一性的尊重,因而不喜欢专门化。它还暗示着对所谓的效率的轻视——它具有更高等级的效率,它不止要求生活的一部分卓越,而且要求生命的本身就很卓越。
\par Phaedrus想起梭罗曾经说过:“只有在失去的时候才有所获得。”这时他才第一次明白,人们凭借辩证法了解并统治了世界,结果却得到了令人难以置信的损失。他曾经培养了自己在科学方面极高的能力,能够运用自然现象来实现自己力量和财富的梦想——但是同时,他也付出了巨大的代价:他丢掉了一种非常重要的了解,也就是了解自己身为世界的一部分,而非它的敌人。
\par 一个人只要望着地平线,内心就能得到宁静。那是一条几何的线条,完全水平,很稳定而且很明显。或许,欧几里得对线条的认识就是从这里得到的灵感。或许,这是第一位天文学家描绘星图时进行原始计算的依据。
\par 现在环绕在苏格拉底和Plato头上的光环已经消失了。他们一直批评智者学派的行为——用情绪化而具有煽动力的语言隐藏自己的目的,使原本居于劣势的论点,也就是辩证法,能够逐渐强壮起来。而此时Phaedrus发现他们一直在做的也正是这件事。Phaedrus认为,往往我们对别人指责最严苛之处,就是我们最害怕自己的地方。
\par 但是为什么?Phaedrus不断地思考,为什么他们要毁掉卓越呢?他刚开始追问,立刻就想到了答案。Plato并不想毁掉卓越,只是贬低它,把它塑造成固定不变的理念,然后转化成僵化而无法改变的永恒真理。他称卓越为善,是行事最高的指导原则,是所有理念当中最好的,仅次于真理。
\par 这就是为什么Phaedrus在教室里提到的良质,和Plato所谓的善是这样的接近。Plato所谓的善是从修辞学家那里得来的。于是Phaedrus继续研究,但是没发现有任何宇宙学者曾经提过这个词。
\par 这是从智者那里来的。二者的差异在于,Plato的善是一种固定不变的理念,而对修辞学家来说它根本不是一种理念。
\par 善不是真实的形式。它是真实的本体,是在不断改变的。它是通过任何僵化或固定的方法都完全无法了解的。
\par 为什么Plato要这样做呢?Phaedrus发现,Plato的哲学是两种综合的结果。
\par 第一种综合想要解决赫拉克利特和巴门尼德学派之间的差异。两派宇宙学者都支持不朽的真理。为了让支持真理的这一方赢得胜利,Plato必须先解决真理的内部冲突,才能抵御支持卓越的学派。为了做到这一点,他声明,就像赫拉克利特学派所说的那样,不朽的真理不仅仅是改变;同时就像巴门尼德学派所说那样,它也不仅仅是毫无变化的存在。这两种不朽的真理同时以不变的理念和变动的现象存在。这就是为什么Plato认为二者需要加以分离。比如说把马性和马分离,认定马性是真实存在的,而且是固定不变的观念,而马则是毫不重要的一时现象。马性是纯粹的理念。而一般人所看到的马,只不过集合了马不断改变的现象。所以一匹会排泄、会随意走动的、会倒地死亡的马,并不会影响到马性,因为马性是不朽的理念,会永远存在。
\par Plato的第二种综合则把智者所谓的卓越融入二元论的理念和现象之中。
\par 它给予卓越最高的地位,仅次于真理和达到真理的方法——也就是辩证法。然而在他企图融合善与真之时,他利用辩证法所得到的真理篡夺了卓越的地位。
\par 一旦善与真被归类于辩证的理念,那么另外一位哲学家就可以很容易地借用辩证法指出,根据“真理”的次序,它们更应该被赋予一个较低的地位,从而和辩证法的规则相容。这样的哲学家很快就出现了,他的名字就是亚里士多德。
\par 亚里士多德认为马的现象,也就是它会吃草,给人作交通工具以及会生小马,需要得到更多的重视。他认为马并不仅仅有现象,这些现象附着于某一种东西,这种东西是一种独立的存在,就像理念一样,是不会改变的。它就是本质。这时,现代科学对真实的理解就产生了。
\par 因而,在亚里士多德的影响下,读者不具有古希腊人卓越的观念,因而让形式与本质占据了思想。善的观念变成一支被称为伦理学的次要学科,它主要讨论的课题是理性、逻辑和知识。这时卓越已经死了,而大学则以科学和逻辑作为建校的根基:针对现存世界的实际延伸出无穷的形式,然后称其为知识。
\par 而把这些形式传给下一代就是系统。
\par 而修辞学呢?可怜的修辞学现在已沦落为传授写作的各种规矩和形式,包括亚里士多德的形式。就写作来说,这些似乎都十分重要。拼写出了五处错,句子的结构不完整,或者三个修饰词放错了位置,或者……这样的情况层出不穷。任何人有这样的问题就表示他没有学好修辞学。毕竟这属于修辞学的范畴,不是吗?当然,这就是空洞的修辞学,诉诸情感而不具有辩证的真理。但我们并不希望情形是这样,不是吗?这样我们就好像欺骗、亵渎了古希腊人,就是那一群智者——还记得他们吗?我们会从学校其他的课程里学到真理,然后再学一点修辞学,这样才能写出优美的文句,得到老板的青睐,才会得到提拔。
\par 形式和种种的繁文缛节——是最优秀的学生所憎恶的,然而却被最差的学生所喜爱。月复一月,年复一年,坐在前排的学生,脸上带着笑容,轻巧地拿着笔,理应得到他们亚里士多德式的甲等;而那些具有卓越特质的人则静静地坐在后排,思索究竟自己出了什么问题,才无法喜欢这门课。
\par 现在很少有学校愿意继续教授古典伦理学,于是学生们便追随着亚里士多德和Plato,永无止尽地提出古代希腊人永远不需要问的问题:“善究竟是什么呢?我们如何去界定呢?由于每一个人都有不同的定义,我们如何才知道哪里才有善呢?有人认为善存在于快乐之中,但我们又怎么知道快乐是什么呢?
\par 而快乐又该如何界定呢?快乐和善不是客观事物。我们无法用科学的方法研究它们。它们不是客观的存在,只能存在于你心中。所以如果你想要快乐,只需要改变你的心意。哈哈,哈哈。”
\par 这就是亚里士多德式的伦理学,亚里士多德式的定义,亚里士多德式的逻辑,亚里士多德式的形式,亚里士多德式的本质,亚里士多德式的修辞学,亚里士多德式的笑声……哈哈哈哈。
\par 而智者学派人的尸骨早已化为尘土,他们所说的也和他们一样烟消云散。
\par 于是这些尘土被埋在毁灭的雅典瓦堆之中,而雅典也消失在覆灭的马其顿帝国当中。紧接而来的是古罗马帝国和拜占庭帝国的灭亡,然后接着是奥斯曼帝国,接着就是现代国家——他们被埋得这样深,而且被蒙上了一层礼法、虚伪之情和邪恶,以至于只有很多个世纪之后出现的这个狂人,才发现了可以将他们出土的线索,同时恐怖地看清了前人的所作所为……
\par 路上一片漆黑,我必须打开头灯才能顺利地在雨雾中行驶。
\subsection*{30}
\par 在阿克塔我们走进一间小餐馆,浑身上下都湿透了,所以感觉特别冷。我们点了咖喱、豆子和咖啡。
\par 然后我们又骑车上路。现在骑上了高速公路,车速很快,路面潮湿。今天我们不必急着赶路,慢慢骑到旧金山就可以停下来休息了。
\par 在雨中迎面驶来的汽车投射出奇怪的光影,雨滴像子弹一样打在头盔上,把车灯折射成奇特的弧形,这是二十世纪的美国。我们现在身处的就是二十世纪,也该结束这个二十世纪Phaedrus的奥德赛之旅了。
\par 下一堂哲学课是在南芝加哥的有大圆木桌的教室里,助教宣布哲学教授生病了。过了一个礼拜他仍然在生病,留下来上课的学生有些惊讶。人数只剩下了三分之一,他们径自走出去喝咖啡。
\par 在咖啡店里,一位Phaedrus一向认为非常聪明但有些自以为是的学生说:“我觉得这是我上过的最不愉快的课。”他似乎像女人一样小心眼,把责任推到Phaedrus身上,认为是他破坏了他美好的经验。
\par Phaedrus也说:“我完全同意你的看法。”他等着别人对他的攻击,但是没有人这样做。
\par 其他的学生似乎也意识到了Phaedrus是事情的起因,但是他们没有对他怎样。
\par 有一位年长的女士问他为什么要来上这门课。
\par Phaedrus说:“我也在思考原因。”
\par “你是全日制的学生吗?”她问。
\par “不是。我在海军码头那边当专职老师。”
\par “你教的是什么?”
\par “修辞学。”
\par 她停住不再问下去。桌上的每一个人都看着他,大家都不发一语。
\par 十一月逐渐过去。黄色的叶子逐渐飘落,只剩下光秃秃的树枝,抵御从北方吹来的寒风。已经开始下雪了,初雪融化,只剩下单调无聊的城市等待冬的降临。
\par 在哲学教授缺席的这一段时间里,Phaedrus研究了另一段Plato的对话。它的主题是Phaedrus。这个名字和我们的Phaedrus毫不相关,因为当时他并不是用这个名字。这位希腊的Phaedrus并不是智者,而是一位年轻的演说家。在对话当中,他被用来衬托苏格拉底。这段对话是讨论爱的本质以及哲学修辞的可能性。显而易见地,Phaedrus并不是非常聪明,而且在修辞方面相当笨拙。他引用了演说家吕西亚的一段很糟的讲词。你很快就会发现,这不过是替苏格拉底铺路,反衬出苏格拉底接下来的演说有多精彩。
\par 再接下来的更精彩,可以算是Plato《对话集》当中最好的一段。
\par 除此之外,Phaedrus比较突出的就是他的个性。Plato常常根据这些人的个性称呼他们。在高尔吉亚那段对话里有一位年轻、爱说话、天真又性情好的次要角色,他叫做宝勒斯,希腊原文的意思就是小马。而Phaedrus的个性和他不同,他不属于任何宗派,他更喜欢乡林的宁静,而非都市的嘈杂。他的个性很激进,几乎到达危险的边缘。有一次,他差一点用暴力威胁苏格拉底,所以,Phaedrus在希腊文中的意思就是狼。在这段对话当中,他被苏格拉底所提出的爱深深吸引,因而被驯服了。
\par 我们的Phaedrus读了这一段对话之后,被其中诗意的意象所感动,但是他并没有被驯服,因为他在其中还找到了一丝虚伪的气息。对话本身并不是目的,而是用来批判修辞学所诉诸情感的世界。热情被视为了解的毁灭者,而Phaedrus在想,是否从这儿开始,对热情的批判就深深埋藏在西方思想之中。古希腊人思想和情感之间的冲突,在其他地方也曾被描述成希腊人性格和文化的基础,这一点很有趣。
\par 下一个礼拜哲学教授仍然没有出现,于是Phaedrus利用这一段时间加紧在伊利诺伊大学的工作。再下一个礼拜他在芝加哥大学对面的书店里,正准备去上课,突然看到两只黑碌碌的眼睛穿过书架望着他。当他看到脸的时候,他发现那就是早先在教室里替他受过的无辜学生,后来就没有再来上课。他脸上的表情似乎透露出一些Phaedrus不知道的事。Phaedrus想走过去和他说话,但是他转身走开了,留下困惑的他。这时他只觉得很疲倦,他要在伊利诺伊大学教课,还要在芝加哥大学和整个西方思想体系抗衡,这逼得他每天必须研究二十个小时左右,因而疏忽了饮食和运动。或许只是因为疲劳,他才觉得对方的表情很怪异。
\par 但是当他过马路到对面教室去上课,对方却尾随在后面二十步左右的地方。似乎有什么事要发生了。
\par Phaedrus到了教室等教授进来,很快那位学生也跟进来了,他悄悄在教室后面坐下来。都很多个礼拜没来上课了,他现在不可能得到任何学分了。他似笑非笑地看着Phaedrus,似乎在对着什么微笑。
\par 从门口传来一阵脚步声,Phaedrus突然明白了——他的腿紧绷起来,双手也在颤抖。在门口出现了一副仁慈的笑脸,站在那儿的正是委员会的主席,由他来接替下面的课程。
\par 这就对了。现在就是他们把Phaedrus赶出去的时候了。
\par 这位主席大方地在门口站了一会儿,然后和一位似乎认得他的学生谈了一会儿。他面露微笑,然后把视线转开,巡视了一下,似乎在找寻熟悉的面孔,然后他点点头,又低声笑了一下,等待上课铃响。
\par 这就是那个学生又回来上课的原因,他们已经向他解释过为什么会突然攻击他。然后为了表现他们是好人,就让他坐在旁边看他们攻击Phaedrus。
\par 他们要怎样进行呢?Phaedrus早已知道了。首先他们会在学生面前运用辩证法,以显示Phaedrus对Plato和亚里士多德的了解是多么薄弱。要做到这一点并不困难。很明显地,他们对Plato和亚里士多德的了解,比Phaedrus多上百倍,因为这是他们一生的研究。
\par 然后,当他们运用辩证法把他完全击倒之后,会告诉他要不就乖乖听话,要不就滚出去。然后他们又会再问一些问题,而且他也不可能知道答案。于是他们就会宣布他的表现太差劲,根本不需要再来上课,必须立刻离开教室。当然,很可能会有一点变化,但是这是基本模式。要做到这一点非常容易。
\par 然而毕竟他已经学了很多,这正是他来这里的用意。他可以用其他方法表达自己的论点,这样一想,他就不再紧张,平静下来了。
\par 在上次看到主席之后,Phaedrus把胡子留了起来,所以主席一时没能认出他。
\par 但过不了多久,主席很快就会发现他。
\par 主席小心地放下大衣,在大圆桌的另一边拿了一把椅子,然后拿出一只旧烟斗,把烟丝塞进去。塞烟丝的动作足足持续了半分钟。你可以看出来,他常常吸烟。
\par 在他巡视班上同学的时候,他微笑着用一种几近催眠的眼神注视每一个人。他觉得教室里的气氛似乎有些不对劲。但是他又加了一些烟丝,一点儿都不慌张。
\par 很快,最后的时刻来临了,他把烟斗点燃。不久整个教室都充满了烟味。
\par 他开口了。“根据我的了解,”他说,“我们今天要开始讨论不朽的Phaedrus。”
\par 他一个一个地看学生,“对吗?”
\par 班上的学生有些羞怯地点头。他十分具有震慑力。
\par 于是主席为哲学教授的缺席道歉,然后提出自己讲课的方式。因为他已经研究过这一段对话,所以他会提出许多问题,然后从学生的回答中去了解他们研究的情况。
\par Phaedrus认为这个方法不错。通过这种方法,教授很快就会认识每一位学生。
\par 很幸运地,Phaedrus研究得很透彻,几乎要把它背下来了。
\par 主席说得没错,这是一个不朽的对话。开始可能会觉得很奇怪,但是它会给你越来越强的冲击,就像真理一样。
\par 在这里,Phaedrus所提出的良质似乎被苏格拉底形容为灵魂、自动自发的能力,以及所有一切的源头。这二者之间没有冲突,因为在一元论的哲学思想当中,是不可能产生任何冲突的。印度的一元思想和希腊的一元思想是一样的,如果不一样,那就是二元了。而一元论之间所产生的差异主要在于“这一位”的特性,而非“这一位”的本质。由于“这一位”是万物的源头,包含了一切,所以它不可能用这些事物来定义,因为不论你用什么去定义它,你所用来定义的事物都无法达到“这一位”的层次。“这一位”只能通过比喻来描述,而苏格拉底则选择用天地的比喻让人明白,如何利用两匹马拉的车把人拉向“这一位”。
\par 但是主席现在要Phaedrus旁边的同学回答问题,他是在用饵引诱他,刺激他反击。
\par 然而因为他问错了人,这个学生并没有对他进行攻击。主席觉得很生气,斥责他下回应该把材料研究清楚。
\par 现在轮到Phaedrus了,他很冷静,现在该由他来解说这一段对话了。
\par “是否能让我换个角度回答。”他说。
\par 这也是因为他没有听到前面那位学生说了什么。
\par 主席认为他这么说无异于是对前一位同学的指责,就笑着但是语带轻蔑地说:“这个主意不错。”
\par Phaedrus继续说:“我想在这一段对话里,Phaedrus的特征就和狼一样。”
\par 他说的时候声音很大,语气也有些愤慨。主席几乎被激得跳起来了。
\par “没错!”主席说,从他的眼神里可以知道,他现在认清楚这个留了胡子的学生就是要攻击他的人。“Phaedrus在希腊文里的意思的确是狼。你说得很对。”他开始恢复平静,“继续说下去。”
\par “Phaedrus见到苏格拉底的时候,苏格拉底只熟悉城市的生活,于是Phaedrus就带他到乡间去,然后开始背诵一段他崇拜的演说家吕西亚的讲词。苏格拉底要他念出来,他照着做了。”
\par 主席说:“且慢!”这时他已完全恢复了冷静,“你说的是情节而非对话。”
\par 于是他叫另外一位同学回答。然而似乎没有任何人知道怎样的回答才能令主席满意。于是主席带着略为悲哀的口吻说他们下次必须预习好,这一次只好由他来替他们解释。于是,他造成的紧张终于得到巧妙的缓解,而整个班级也在他的股掌之间了。
\par 于是主席继续专心地解说对话的意义,Phaedrus也十分注意地听。过了一会儿,有一件事使他分了心,因为有一种错误的思想悄悄地溜了进来。刚开始他不知道究竟是什么,然后他才知道,主席完全忽略了苏格拉底对“这一位”的描述,而直接跳到马与车的比喻上。
\par 在这个比喻里,追寻的人想要接近……“这一位”,他由两匹马拉着,一匹是高贵的白马,性情温驯,而另外一匹当然是顽固热情的黑马。这匹白马永远帮助他奔向天堂之门,而黑马则永远带给他挫折。主席还没有说出来,但是他即将宣称,这匹白马就是温驯的理性,而这匹黑马就是黑色的热情。他正要进一步说明,但是,突然,错误的思想涌现出来了。
\par 他坐正了然后重复地说:“现在苏格拉底向神明发誓,他所说的都是真的。
\par 他已经发誓,自己所说是实话,那么,如果接下来他说的不是实话,他就无异于丧失了自己的灵魂。”
\par 这是陷阱!他用对话来证明理性的神圣,如果这个论点得以建立,他就可以直接研究理性究竟为何物,然后看啊,我们又落入亚里士多德的国度之中了。
\par Phaedrus举起手来,手心向前,肘放在桌上,他的手还没有发抖,他现在显得很平静。Phaedrus知道,自己这么做就是签署了自己的死亡宣言。但是他也知道,如果把手放下就是签署另外一种死亡宣言。
\par 看到他举起手,主席有些惊讶,有些困惑。但还是让他发言。
\par Phaedrus说:“这一切只不过是比喻。”
\par 大家都没有说话,主席很困惑地问:“什么?”他的法力已经被破解了。
\par “有关车子和马的描述都是比喻。”
\par “什么?”主席又问了一句,然后大声地说,“它是真理。苏格拉底曾经向神明发誓它是真理。”
\par Phaedrus回答说:“苏格拉底自己说这是比喻。”
\par “如果你读过对话就会发现苏格拉底特别强调它是真理!”
\par “是的,在这个之前……我想是第二段……他说过这是一个比喻。”
\par 教材就放在桌上可以参考,但是主席十分明白,在这个节骨眼不可以去参考,如果去翻阅了,而且证明Phaedrus是对的,那么他在班里就颜面扫地了。他曾经对学生说过,他们中没人仔细地研究过这本书。
\par 修辞学得一分;辩证法得零分。
\par Phaedrus想,太棒了,他记得苏格拉底这么说过。他完全贬低了辩证法的地位,这正是重点所在。它是一个比喻,所有的一切都是比喻,但是辩证学家不知道这一点,这就是为什么主席忽略了苏格拉底的这一段话。Phaedrus抓住这一点并且牢牢地记住了它,因为假使苏格拉底没有说它是比喻,他说的就不是“真理”了。
\par 还没有人看清楚这一点,但是他们很快就会明白。主席在自己的课堂里被攻击得体无完肤。
\par 现在他无话可说。刚上课的时候,他靠让大家保持沉默建立起了自己的形象,现在这沉默反倒把他给毁了。他不知道攻击究竟从何而来,他从来没有遇到过活着的智者,只有死去的智者。
\par 现在他想要抓住什么,但是没有东西可以让他攀附。他自己的动力把他拖向深渊,当他终于找到可以说的话时,听起来好像来自于另外一个人;像是一个小男孩忘记了自己要背的课文,或者是完全背错了,但是还希望我们能放过他。
\par 他想指责班上没有人好好研究过这段对话,想以此来吓唬他们,但是坐在Phaedrus右边的人朝他摇摇头,很明显地,有人仔细读过。
\par 于是主席支支吾吾地犹疑起来,似乎有些害怕学生们,也想在心理上和他们保持距离。Phaedrus在想,这一场戏究竟会怎样收场?
\par 然后发生了一件不妙的事。那位曾经被攻击的学生现在已经不再天真。他开始嘲讽主席,然后问他一些讽刺的问题。主席本来就已经被Phaedrus攻击得瘸了腿,现在可以说被打倒在地……但是Phaedrus知道,这一切都是冲着他来的。
\par 他不觉得难受,只是很厌恶。当一个牧羊人杀了一匹狼,然后带着牧羊犬去看狼的尸体时,他必须小心谨慎,避免犯任何错误,因为这只牧羊犬和狼之间仍然有某种血源,这是牧羊人不该忘记的。
\par 一位女孩子替主席圆场,问了他一些比较容易的问题。他很感激地接纳了这些问题,然后用非常冗长而缓慢的语调回答,想要恢复冷静。
\par 然后有人问他,“什么是辩证法呢?”
\par 他想了一下,然后转向Phaedrus,问他是否愿意回答。
\par “你是问我个人的意见吗?”Phaedrus问。
\par “不是……就算是从亚里士多德的角度吧。”
\par 现在他不再闪躲了,他就是要把Phaedrus拉到自己的国度中,然后再攻击他。
\par “就我所知……”Phaedrus说,然后停下来。
\par 主席面带笑容地说:“然后呢?”这一切都已经设计好了。
\par “就我所知,亚里士多德认为辩证法先于所有的一切。”
\par 主席脸上的表情由原先的感激变为震惊,然后再变为暴怒。说得没错!你可以由他的表情知道他心里在呐喊,但嘴里没有说出来。Phaedrus又落入了他的陷阱。他不能因为Phaedrus引用了《大英百科全书》中他文章里的一句话而攻击他。
\par 修辞学得二分;辩证法得零分。
\par “然后由辩证法产生了形式,”Phaedrus继续说道,“然后由……”但是主席打断了他的话,因为他发现Phaedrus并没有按着他的路子走,于是就结束了对话。
\par Phaedrus想,他不应该打断的。如果他是真正追寻真理的人,而不是专门宣传某一种观点,就不应该打断他的话。
\par 他本来可以学到一点东西。一旦这么说:“辩证法先于所有的一切。”这句陈述本身就变成了辩证的实体,隶属于辩证问题。
\par Phaedrus原本想这样问:“认为利用辩证法问与答的模式达到真理,这种方法先于所有的一切,究竟有何支持的证据?”然而毫无证据,所以一旦把这句话孤立起来接受严密的检视,它就会变得荒唐可笑。而这个像Newton万有引力定律一样的辩证法,下面没有任何支撑物,却是世间万物的根源,嘿!这真是愚不可及的事。
\par 辩证法是逻辑的源头,但是却来自于修辞学,而修辞学则是神话和古希腊诗学的传承。这在历史上和常识上都确有其事。而诗与神话则是史前人类对周遭世界的反映,而且以良质为根基。所以,归根结底,是良质而非辩证法酝酿了我们所知的这一切。
\par 下课的时候主席站在门口回答问题,Phaedrus也想过去说几句话,但是他没有这样做。这一生他受过无数的打击,因而对可能带来更多打击的讨论没有兴趣。主席对他并不友善,甚至没有一点表示友善的暗示,反而有相当的敌意。
\par Phaedrus是匹狼,这个形象颇为适合。
\par 他轻巧地走回公寓,发现越来越适合。
\par 如果他们过分赞成这样的论点,他也不高兴。他最明显的个性就是充满敌意。
\par 真的是这样。Phaedrus这匹狼从山上下来,就是要猎杀知识领域当中这批天真的居民,他完全符合狼的形象。
\par 理性教会就像所有有组织的机构一样,并非源于个人的优点而是源于个人的弱点。理性教会要求的并非能力,而是无能。一个无能的人才容易受教。而一个真正有能力的人总会带给别人威胁感。Phaedrus明白,他已经错过了融入这个组织的机会,因为他拒绝臣服于亚里士多德的思想。但是这种思想似乎不值得他去尊敬,因为它是一种劣质的生活方式。
\par 对他而言,在雪线以上的良质比这儿烟尘满布的窗户和听不完的言语要好多了。他明白,自己所说的永远无法被这里的人接受。因为要接受他的思想,这个人就必须摆脱社会的权威,而这里到处都充满了权威。绵羊能过怎样的生活?决定权在牧羊人。如果你在晚上把一只羊放到雪线以上,狂风吹来时,羊可能会吓得半死,然后会一直哀嚎到牧羊人找到它为止。当然,来的也可能是狼。
\par 下一堂课,他想表现得和善一点,但是主席似乎并没有这种意图。Phaedrus要他解释一处自己不甚明白的地方。他想这样可以缓和两人之间的对立。
\par 然而得到的回答却是:“这下你可累了吧!”主席尽可能地辱骂他,但是却伤害不到他。因为主席拼命谴责Phaedrus的,正是他自己最害怕的地方。Phaedrus望着窗外,为这位老牧羊人、教室里的羊和狗而悲哀,而且也为自己永远不可能成为他们的一分子而悲哀。然后下课铃响的时候,他离开了,永远不再回来。
\par 然而在伊利诺伊州的教学却像野火一样旺盛,学生现在非常专心地倾听这位奇特的、留着胡须的人的讲述,他从山上来,告诉他们宇宙间有所谓良质的存在。他们知道他说的是什么,但是他们不知道该怎样形容,所以有些不确定。
\par 还有一些人则对他有些畏惧,他们知道他有点危险。但是大家都深深地为他着迷,想要听更多的讯息。
\par 但是Phaedrus并不是牧羊人。如果故意去扮演这样的角色,那会把他给毁了。
\par 这时课堂上经常会让他有一种很奇怪的感觉。坐在后排不那么守规矩的学生往往对他所说的十分投入,而且也是他心爱的学生。坐在前排像小羊一样柔顺的学生却常常被他所说的吓住了。但是学期结束的时候,这些像小羊一样的学生总是能通过考试,而后排的却无法通过。
\par 虽然到现在Phaedrus仍然不想承认,但是直觉上他做牧羊人的日子快结束了。他越来越好奇,不知道接下来究竟会发生什么事。
\par 他害怕教室里会出现沉寂,就是那种把主席给毁了的沉寂。按他的本性,他并不喜欢连续几个小时不断地讲话,那会让他很疲劳。然而现在没有其他的事来转移他的注意力,于是他开始注意这种害怕。
\par 他来到教室的时候,上课铃响了。
\par 他坐在那儿一言不发。整堂课他都静静的,有些学生想要刺激他,使他清醒些。
\par 但是之后他们也不说话了。有许多学生因为惊慌过度而不知所措。下课铃一响,全班同学立刻冲出教室,于是他又去上下一堂课,重复同样的情形。接下来的几堂课他都是用同样的方法去上。然后他就回家了。他越来越想知道接下来究竟会发生什么事。
\par 感恩节到了。
\par 他连睡四堂课的本事已经缩减到两堂课,然后是一堂课也没有了。所有的一切都结束了,他既不会回去上亚里士多德的修辞学,也不会回伊利诺伊大学教这门课。所有的一切都结束了。他走过街道,内心在翻腾。
\par 现在城市的身影笼罩在他身上,在他奇特的观念中,这个城市变成了他信仰的对立面,并不是良质的大本营,反而是形式与本质的大本营。像钢筋水泥的船坞和道路、砖块、柏油路、零件、老旧的收音机、铁轨、动物的尸体;形式和本体,没有良质。这就是这个城市的灵魂。盲目、巨大、邪恶而没有人性;夜里你可以看到南方有大火炉燃起熊熊的火焰,而在啤酒、比萨和洗衣店招牌之间是浓厚的煤灰,沿着街边则是许多不知名而没有意义的招牌。
\par 如果到处都是砖块和水泥,物质的纯粹形式,既清楚又开阔,他就有可能存活。正是对良质所做的那些卑微而悲惨的努力,才足以致人于死地。就拿那间公寓中石膏制的假壁炉来说,它被用来容纳那从来不曾存在过的火焰。或者像公寓与前面的树篱之间那一片数英尺见方的青草地。在Montana之后,数英尺见方的青草地。如果他们忽略树篱或青草地,那就没事。现在它的作用就是提醒人们去注意那些已然失去的事物。
\par 沿着公寓附近的街道,他无法从砖头、水泥,或霓虹灯的间隙中看到任何东西,但他确知,其中埋藏的是怪异的、扭曲的心灵,始终尝试着借某种方式来证明自己拥有良质,它们从梦幻杂志或其他大众媒体上学来各种奇怪的姿态与神色,而且还要把钱支付给物体的卖方。
\par 他整夜整夜地想着这些,想着豪华炫目的鞋子、网袜,以及褪去的亵衣,他注视着被煤烟熏黑的窗户,旁边露出的奇形怪状的贝壳,当表态逐渐褪去而真相愈见分明时,此地仅存的真理就是——哭喊天堂,上帝啊!这里只有死气沉沉的霓虹灯、水泥,以及砖块。
\par 他对时间的感觉在逐渐消失。有时候他的思想快得像光速,但是一旦要他做什么决定的时候,却又好几分钟想不出任何事来。有一个念头在他心里出现,是从Phaedrus的对话当中抽出来的一部分。
\par “写作的好坏我们需要向吕西亚请教,或是向任何一位诗人和演说家请教吗?”
\par 什么是善,Phaedrus,什么又是恶——我们需要别人来告诉我们答案吗?
\par 这就是几个月前他在Montana的教室里说的,这是自Plato之后的每一位辩证学家所忽略的。他们每一个人都想从知识的角度去界定良质,但是现在他发现自己和良质的距离非常遥远,因为他也在做同样的事。他原来的目标是不要让良质被界定,但是在和辩证学家对抗的过程中,他提出了许多论点,每一个论点都是他在良质旁边建立的砖墙。
\par 一旦想通过系统的思考去界定良质,就会破坏它最原始的目标,所以他所做的实在是一桩愚不可及的事。
\par 到了第三天,走在一条不知名的十字路口,他突然什么都看不见了。等到恢复视觉的时候,他发现自己躺在人行道上。旁边有人在走动,好像完全无视于他的存在。他很疲惫地爬起来,然后费力地回想回公寓的路。他的思路越来越慢,越来越慢。之后他就再没有离开过公寓。
\par 他双脚交叉,望着墙壁。在一间没有床铺的房间里,地上铺着毛毯。所有的桥都断了,没有回去的路。而现在连前进的路也没有了。
\par Phaedrus盯着卧室的墙壁看了三天三夜,他的思绪既未前进也未退后,只停留在那一刹那。妻子问他是否生病了,他没有回答。她很生气,但是Phaedrus却没有任何反应。他知道她在说什么,但是无法回答。不只他的思考停顿了下来,他的欲望也止住了。最后一切变得一团糟。他觉得沉重、疲惫,但是并不想睡。
\par 他觉得自己好像是巨人,有好几百万英里高。又觉得自己在永无止尽地融入宇宙之中。
\par 他开始扔东西,把携带了一生的东西都扔了。他要妻子跟小孩一块走,去替自己做别的打算。他的尿液流满了房间的地板,他也不觉得讨厌和羞愧。香烟一直烧着,烫到了手指,然后手指起了水泡,水泡破了才把香烟给弄熄了。
\par 对他而言,这一点都不痛苦。他妻子看到他受伤的手和地上的尿液,就赶紧打电话求救。
\par 但是在别人赶到之前,Phaedrus的整个意识开始慢慢地毫无知觉地整个瓦解……然后他不再思索接下来究竟会发生什么事。因为他知道了接下来要发生的事。于是他为他的家人、为他自己和这个世界流下泪来。这时他想起一首圣诗的片断,“你必须要经过那死荫的幽谷”。这句话把他向前推进。“你必须要独自经过那死荫的幽谷”。这首诗还提到,没有人能替你去走。它的内涵似乎超过了字面的意义,“你必须要独自经过那死荫的幽谷”。
\par 他走过了这一段死荫的幽谷,走出神话,仿佛像从梦境中走出来。他整个的意识就像是一场梦,不是别人的梦而是他自己的梦,是他现在必须独自支撑的梦。然后他自己也消失了,只剩下他的梦和在梦中的他。
\par 而他曾经这样辛苦地保卫、牺牲,从来没有背叛过的良质,原来他从来不曾了解,现在却了然于心,他的灵魂得到安息了。
\par 这时路上的车很少,路面一片黝黑,头灯似乎很难透过雨水照射到路面。这真是非常危险的状况。任何事情都有可能发生——突然的煞车,或是路上有漏油和动物的尸体……但是如果你骑得太慢,后面的车就会一直催你。我不知道为什么我们还在继续走着。我们早就该停下来了。我也不知道为什么要一直骑下去。我想我一直在找汽车旅馆的招牌,但是因为思想不集中而没看到,如果我们一直这样骑下去,它们会都关门了。
\par 我们从高速公路的下一个出口下去,希望能通往某处。但是很快地我们就骑上一条颠簸不平的柏油路,上面有一些碎石子。我慢慢地骑着。头上的街灯透过雨水散发出来黄色的光晕。光晕摇晃着,我们一会儿身在亮处,一会儿又在暗处,一会儿在亮处,一会儿又在暗处。没有看到任何旅馆的招牌。在我们左边有一个暂停的标志,也没有指示该从何处转弯。每一条路都一样漆黑,我们很可能永无止尽地骑下去,但什么也找不到。现在甚至连高速公路都找不到了。
\par Chris喊着:“我们到哪里了?”
\par “我也不知道。”我的头脑变得十分疲惫,缓慢下来。我似乎连正确的回答也想不出来……更想不出接下来该做什么事。
\par 现在我看到前面有一点白色的灯光,而且有加油站醒目的标志,就在往前一点的路上。它还开着。我们在路旁停下来。服务生看了看Chris,很奇怪地打量着我们。他不知道哪有汽车旅馆。
\par 于是我走到电话簿旁边,找到一些汽车旅馆的地址,然后告诉服务生。他想指引我们方向,但是他也说不清楚,于是我就打电话到他说的最近的一间旅馆,订下房间,然后向对方确定路该怎么走。
\par 雨中漆黑一片,虽然有对方的指引,我们也差点找不到旅馆的位置。因为他们把灯关了。我们登了记,没有说什么。
\par 旅馆房间布置得像三十年代,但是已经有些破败和肮脏,能看出来是不懂木工活的人布置的。但是里面还算干燥,而且有暖气和床铺,这就够了。我把暖气打开,坐在前面,很快地,刺骨的寒意和湿气就不见了。
\par Chris没有抬头看我,只是瞪着墙上的暖气片。过了一会儿,他说:“什么时候回家?”
\par “到旧金山之后,”我说,“为什么要问这个?”
\par “我一直坐着,坐得很厌烦……”他的声音逐渐小下来。
\par “然后怎么样?”
\par “我……我不知道……只是坐着……
\par 好像我们哪里也不去。”
\par “我们应该去哪里呢?”
\par “我不知道,我怎么会知道?”
\par “我也不知道。”我说。
\par “那么你为什么不知道呢?”他说。
\par 然后哭了起来。
\par “Chris,怎么回事?”我问他。
\par 他没有回答我。他把头埋在手里,然后前后摇摆,这给我一种很奇怪的感觉。过了一会儿,他停下来又说:“当我小的时候,情形不是这样。”
\par “那是怎么样?”
\par “我不知道。我们总是一起做事情,都做我想做的事。现在我什么事都不想做。”
\par 他又开始很奇怪地前后摇摆着,脸埋在手里。我不知道该怎么办,这是一种很奇怪的、无法形容的摇摆,是一种把别人摒弃在外的自我封闭,像是回到了我不知道的地方……海洋的深处。
\par 现在我知道曾经在哪里看过他这样了,在医院的地板上。
\par 我不知道该做什么。
\par 过了一会儿,我们爬上床,我已经想睡了。
\par 然后我问Chris:“我们离开芝加哥之前情况比较好吗?”
\par “是啊。”
\par “怎样好法?你记得那时怎样吗?”
\par “很有意思。”
\par “有意思?”
\par “是啊,”他说,然后静下来。之后他又说:“记得我们一起去找床的事吗?”
\par “这很有意思吗?”
\par “当然,”他说,然后又沉默了好长一段时间。之后他说:“你不记得了吗?
\par 你要我到各个方向去找回家的路……你过去常常和我玩游戏,告诉我各种故事,然后我们一起骑车出去。但是现在你什么都不做了。”
\par “我在做。”
\par “没有,你没有。你只是坐着发呆,你什么事都不做!”他又哭了起来。
\par 窗外的雨突然下大了,这时我感到一种非常沉重的压力。他是在为自己哭泣。他想念的是他自己。这就是那个梦,在梦里……
\par 我似乎有很长一段时间都在听墙上暖气里的声音,还有风雨吹打屋顶和窗户的声音。然后雨逐渐小了下来。除了偶尔风吹过,雨从树上滴下来打在屋顶上,什么声音都没有了。
\subsection*{31}
\par 早上醒来的时候,地上有一只绿色的蛞蝓。我吓了一跳。它大概有六英寸长,四分之三英寸宽,而且全身像橡皮一样柔软,表面覆盖着一层黏液,好像动物的内脏。
\par 四周一片潮湿,而且雾气很重。但是视线还算良好。我看见旅馆坐落在一座小山坡上,周围有一些苹果树,树下青草如茵,上面缀着露珠。然后我又看到另一只蛞蝓,然后又有一只——天啊!
\par 整个地面都爬满了蛞蝓。
\par Chris出来的时候我指给他看。蛞蝓像蜗牛一样慢慢爬过一片树叶,但是Chris什么话也没说。
\par 我们离开这家旅馆,到路旁的一座小镇韦奥特吃早点。他仍然很冷淡,不太想说话,我就随他去。
\par 在莱吉特我们看到一个开放给观光客的野鸭池,于是我们买了饼干喂鸭子。
\par Chris丢饼干的时候还闷闷不乐,是我看过的最不快乐的神情。然后我们又沿着崎岖的海岸前行。突然,四周全是大雾,温度立刻降下来,我知道我们又靠近海边了。
\par 离开雾区之后,我们从一座高崖上望海,海是那样深沉而遥远。我们一路骑去,我觉得越来越冷,于是就停下来拿出夹克穿上。Chris走到悬崖边上,悬崖起码有一百英尺高,这样太危险了!
\par 我大声喊他:“Chris!”但是他没有回应。
\par 我跑过去,一把抓住他的衬衫把他拉回来。
\par “不要过去。”我说。
\par 但是他用很奇怪的眼神瞄了我一眼。
\par 我又拿出他的衣服交给他,但是他一直拿着没穿。
\par 没有必要催他,我让他自己决定要不要穿。
\par 但是他迟迟不肯穿上。十分钟过去了,接着又过去了十五分钟。
\par 我们似乎在比赛耐性。
\par 吹了三十分钟的冷风之后,他问我:“我们要往哪里走?”
\par “往南,沿着海岸走。”
\par “我们回去吧。”
\par “回哪儿?”
\par “回比较温暖的地方。”
\par 这样又要骑好几百英里。“我们现在必须往南走。”我说。
\par “为什么?”
\par “因为回去的路程太远。”
\par “我要回去。”
\par “不行。把你的衣服穿上。”
\par 他不肯,只是坐在地上。
\par 又过了十五分钟,他说,“我们回去吧。”
\par “Chris,骑摩托车的人不是你,是我。我们要往南走。”
\par “为什么?”
\par “因为回去太远。而且这是我的决定。”
\par “那么为什么我们不回家?”
\par 我生气了,“你并不是真想知道原因,是不是?”
\par “我想回家,告诉我为什么我们不回家。”
\par 我快要爆发了,“你真正想要的不是回家,而是激怒我。你再这么闹,我可真要生气了。”
\par 这时他有一点儿恐惧,这就是他要的。他想恨我,因为我不是他。
\par 他很气恼地望着地面,然后把衣服穿上。我们走回车子那儿,又沿着海岸继续骑下去。
\par 我能假装他理想的父亲形象,但是在潜意识里,他真正的父亲并不在这儿。
\par 在这一趟Chautauqua之旅当中,我们曾经攻击过虚伪。我不断地提醒要消除主客观的二元对立观念,然而我一直没有去面对最大的对立,也就是他和我之间的对立,一个和自己对立的心灵。
\par 但是是谁造成的?不是我。现在没有办法解决它……我现在在思索,到海底究竟有多远……
\par 我们来到门多西诺县的海边,这里的景观十分原始而且辽阔,非常美丽。
\par 山坡上长满了绿草。但是在山坡的凹处和岩石下,有一些很奇怪的灌木丛,被下面吹上来的风雕塑成奇怪的形状。我们经过一些老旧的木篱笆,它们被风雨吹打成灰褐色。远处有一座历经风雨的、灰色的老农庄,怎么会有人在这儿耕作呢?篱笆有许多地方都已经破损了。真可怜。
\par 路开始从高崖降到海岸,我们停下来休息。我关掉发动机,Chris说:“我们停下来做什么?”
\par “我累了。”
\par “但是我不累,我们继续走。”他仍然在生气。我也生气了。
\par “你到海边去玩,我在这里休息。”
\par 我说。
\par “我们继续走。”他说。但是我走开了,不管他。他坐在摩托车旁边的石头上。
\par 海风中有一股腐朽生物的气味。冷风吹得人无法休息。我来到一块大岩石后面,这里还可以晒到太阳。我把注意力集中在暖洋洋的阳光上,对眼前的情景我已经很感激了。
\par 我们又继续往南骑。现在我明白了,他是另外一位Phaedrus,有像他过去一样的思考模式,像他一样的行为,不断地找麻烦,被他自己内心一股莫名的力量驱使着。这些问题……同样的问题……他想要了解一切。
\par 如果他得不到答案,他就会打破砂锅问到底,直到他满意为止。而这又导致了另外一个问题的出现,于是他又一直追问下去……永无止尽地追问,不懂得问题其实也是永无止尽的。这里面丢掉了一些东西,他知道,但是如果他试图寻找,可能会因此而丧命。
\par 我们在一座高崖上转了一个大弯,眼前的海洋无边无际地向前延伸出去。
\par 大海寒冷湛蓝,很奇怪,却让我有绝望的感觉。住在海边的人永远不会了解海洋对于住在内陆的人的意义——它代表了如此遥远而庞大的梦想,虽然就在眼前,但是在最深的潜意识里,你却看不见它。当他们到达海洋的时候,将意识与潜意识的梦境相比较,就会感到挫败。
\par 他们走了这么远的路,却到达了一个永远无法探知深度的神秘之处。它是一切的源头。
\par 过了许久,我们来到了一座小镇。
\par 街上起了一层薄雾。在海上看起来十分自然的现象,此刻突然出现在小镇上。
\par 阳光照耀,一切都蒙上了一层古老的情调,仿佛让人回到许多年以前。
\par 我们在一家拥挤的餐厅前停下来,只剩下最后一张桌子了,靠着窗边,我们望着街道上的灯光。Chris低着头,不想说话,或许他已经意识到我们剩下的旅程很短了。
\par “我不饿。”他说。
\par “那么我吃的时候你在旁边等?”
\par “我们继续走,我不饿。”
\par “但是我饿了。”
\par “但是我不饿,我的胃在痛。”这又是他的老把戏。
\par 于是我在邻桌的谈话和刀叉声中吃午餐。我看见窗外有一个人骑自行车经过,我觉得好像到了世界的尽头。
\par 我抬起头来,看到Chris在哭。
\par 我说:“又怎么了?”
\par “我的胃在痛。”
\par “就是这样吗?”
\par “不是,我真的好恨这里的一切……
\par 我很抱歉,我不该来的……我很讨厌这次旅行……我原本以为会很有意思,但是完全不是……我很抱歉我来了。”他像Phaedrus一样诚实,而且像他一样对我的恨意越来越深。时候到了。
\par “Chris,我一直在想,也许你可以从这里搭公共汽车回家。”
\par 他脸上没有表情,然后有些惊讶和失望。
\par 我又接着说:“我会自己继续骑下去,然后一两个礼拜之后再和你会面。
\par 没有必要强迫你跟我一起走。”
\par 现在轮到我惊讶了。他的表情显示他一点儿都没有释怀。他越来越忧伤,然后低下头来什么话也不说。
\par 他现在似乎心理失去了平衡,而且很害怕。
\par 他抬起头来,“那我住哪里?”
\par “你现在不能住在我们原先的房子里,因为有别人住进去了。你可以跟爷爷奶奶住。”
\par “我不要跟他们住。”
\par “你可以跟姑姑住。”
\par “她不喜欢我,我也不喜欢她。”
\par “那你就去跟外公外婆住。”
\par “我不要。”
\par 我又提了其他几个人,但是都被他否决了。
\par “那么你要跟谁住呢?”
\par “我不知道。”
\par “Chris,我想你自己很清楚问题在哪里。你不想旅行,你讨厌它。然后又不想和他们住一起,我提的这些人,不是你讨厌他们,就是他们讨厌你。”
\par 他默不作声,眼眶里的泪珠在打转。
\par 邻桌的女人很生气地看着我,她张嘴想说什么,但是我瞪了她好一会儿,直到她闭嘴继续吃她的东西。
\par Chris放声哭了起来,别的桌上的客人都看着我们。
\par 于是我说:“让我们散步去。”然后起身去付账。收款台的女服务生说:“真不巧孩子不舒服了。”我点点头,付了账就出来了。
\par 我想找一把椅子,但是没找到。于是我们骑上车,继续往南前进,想要找一个可以休息的地方停下来。
\par 马路又开始向海洋的方向伸展,一直通往一处高地,它很明显地突出在海洋之上,但是现在四周是一片浓雾。过了一会,我从散去的雾中看到有一些人在沙滩上休息,但是很快雾又涌上来。
\par 那些人又看不清楚了。
\par 我看了看Chris,看见他眼睛中只有空洞而又困惑的神情。但是当我要他坐下来时,他的怒气和恨意又出现了。
\par 他说:“为什么?”
\par “我想我们该好好地谈一谈了。”
\par “好吧,谈就谈。”他说,所有的愤怒都涌上来了。他不能忍受我这种温和的态度,因为他知道这些温和都是假象。
\par “未来会怎样?”我说。这样问真愚蠢。
\par “什么?”他说。
\par “我的意思是你对将来有什么打算?”
\par “顺其自然。”他有些轻蔑的口吻。
\par 雾又散开了一会儿,露出我们站着的高崖,然后雾又涌过来。这时我觉得,该发生的事还是必然会发生。我被推向什么东西,而眼角的事物和眼中央的事物,现在完全一样重要了,完全融合为一。于是我说:“Chris,我想我们该谈些你不知道的事。”
\par 他听了一会儿,知道有些事要发生。
\par “Chris,你眼前的父亲曾经精神错乱过好长一段时间,现在他又快要发病了。”
\par 并不是快要发病了,根本就已经发病了。像海一样的深。
\par “我要把你送回家并不是因为我在生你的气,而是我害怕如果和你继续旅行下去,不知道会发生什么事。”
\par 他脸上的表情没有任何变化,他还不明白我在说什么。
\par “所以我现在就要和你说再见。Chris,我不确定我们是否还会再相见。”
\par 我把该说的话都说了,现在该发生的就让它发生吧。
\par 他很奇怪地看着我,我想他仍然不明白,那种眼神…… 我曾经在哪里看过……在哪里……在哪里……
\par 在晨雾当中,沼泽里有一只小鸭子,它是一种小凫,它的眼神就像这样……我射到它的翅膀,所以它不能飞了,我跑过去抓住它的脖子,我想弄死它,但又停了下来。然后,似乎受到了宇宙中某种神秘力量的驱使,我看着它的眼睛。
\par 它的神情就像此时的Chris一样……平静、毫不知情……然而却又十分明了。然后我用手遮住它的眼睛,扭断它的脖子,我的手指可以感觉到断裂的震动。
\par 然后我移走手掌,小鸭子仍然看着我,但是眼神呆滞,不再随着我的动作而转动。
\par “Chris,他们在说你。”
\par 他看着我。
\par “这些问题你自己都很明白。”
\par 他摇头否认我说的。
\par “他们所说的似乎是真的,感觉也像是真的,但实际上不是。”
\par 他的眼睛睁得大大的,然后还是摇头否认我的看法。但是多少有一些了解我的意思。
\par “现在事情越来越糟,你在学校里有问题,和邻居也处不来,和家人,和朋友,到处都有问题。Chris,是我在替你挡回去,告诉他们你没有问题。但是现在不会再有人替你挡了。你明白吗?”
\par 他很惊讶。从他的眼神里可以看出他有些畏惧。我没有给他力量,我从来没有给过,我像对待那只小鸭子一样,慢慢地把他给扼杀了。
\par “Chris,这不是你的错。从来都不怪你。你要了解这一点。”
\par 这时他似乎意识到了什么,于是闭上眼睛,哭了起来。哭声很奇怪,好像从很遥远的地方传来。他在路上跌跌撞撞地走着,有时候又跌坐在地上。然后又把身子弯起来跪在地上,前后摇摆着,头掴在地上。四周的微风吹着他身旁的青草。有一只海鸟在旁边停下来。
\par 我听到雾里有一辆卡车驶近的声音,心里有些害怕。
\par “Chris,你得站起来。”
\par 他哭的声调很尖锐,几乎不像人在哭,而像一个遥远的水妖。
\par “你一定要站起来。”
\par 他还是赖在地上摇摆,不肯起来。
\par 我现在不知道该怎么办,一点都没有头绪。这一切都结束了。我必须先把他送上公共汽车,离开悬崖边就好了。
\par Chris,现在一切都没有关系了。
\par 这不是我的声音。
\par 我不会忘记你。
\par Chris停了下来。
\par 我怎么会忘记你呢?
\par Chris抬起头来看着我,他透过散开的薄雾看着我,但雾一会儿又聚拢了。
\par 我们将会在一起。
\par 卡车的声音来到我们身边。
\par 现在站起来!
\par Chris慢慢地坐起来望着我,这时候卡车出现了,停了下来,司机探出头来问我们是否需要载一程。我摇摇头,然后挥挥手叫他继续走。他点点头,然后就消失在雾中,只剩下Chris和我。
\par 我把夹克罩在他身上,他把头埋在膝盖中间,又哭了起来。现在他只是啜泣,比较像人的声音,而不像刚才哭得很奇怪。我两手很湿,觉得额头也湿了。
\par 过了一会儿,他哭着问我,“你为什么离开我们?”
\par 什么时候?
\par “在医院的时候!”
\par 没有办法,警察把我带走。
\par “难道他们不让你出来吗?”
\par 不让我出来。
\par “那么,你为什么不开门?”
\par 什么门?
\par “那扇玻璃门!”
\par 这时我觉得像被电击了一般,他在说什么玻璃门?
\par “难道你不记得?”他说,“我们站在门这边,你站在门那边,妈妈在旁边哭。”
\par 我从来没有告诉过他那个梦,他怎么知道的呢?喔,糟糕了。
\par 我们在另外一个梦里。这就是为什么我的声音听起来这么奇怪。
\par 那扇门我打不开,他们不让我打开它。我必须照着他们的话做。
\par “我以为你不想见我们。”Chris说,他把头低了下来。
\par 他眼中出现了这些年来一直存在的恐惧。
\par 现在我看到那扇门了,它是在一座医院里。
\par 那是我最后一次看到他们。我就是Phaedrus,我就是他,他们因为我说实话而想把我给毁了。
\par 这一切都对上了。
\par 现在Chris哭的声音渐渐小下来,但还是没有止住。海风吹在我们四周长长的野草上,雾逐渐散去。
\par “Chris,不要哭了。只有小孩子才哭。”
\par 过了好久,我给他一块破布擦睑。
\par 我们把东西收拾好,然后放上摩托车。
\par 现在雾突然散去了,我看到他脸上的阳光,看到我以前从未见过的表情。他戴上头盔,然后系上带子,抬起头来。
\par “你真的精神错乱过?”
\par 他为什么这样问呢?
\par “没有!”
\par 他吃了一惊,但是眼睛里闪烁着光芒。
\par “我就知道。”他说。
\par 然后他爬上摩托车,我们出发了。
\subsection*{32}
\par 我们现在沿着曼扎尼塔的海岸前进,路旁的灌木丛叶子好像涂了蜡,这时我又想起Chris说“我就知道”时的表情。
\par 车子很顺利地转着弯,不论角度如何,总是能顺利地转过去。路的两旁到处是野花,还有令人讶异的景色。一个接一个的大转弯不断出现,整个世界好像在不断旋转,山坡也在不断地起伏着。
\par 他说:“我就知道。”这句话不断出现在我的脑海里,好像鱼钩上有东西上钩了,想引起我的注意。这件事已经埋藏在他心里很久,有好多年了。现在想起来,他所制造的那些问题都可以谅解了。他说:“我就知道。”
\par 很久以前他一定就听说过什么,在他小时候把这一切都弄混了。这就是Phaedrus经常说的——我经常说的——许多年以前,Chris一定相信,然后一直埋在心里。
\par 往往连我们自己都无法了解彼此之间的关系。他就是我要出院的真正理由,因为让他独自长大是不对的,而且在梦里他总是想把门打开。
\par 我根本没有把他带到哪里,是他在带我。
\par 他说:“我就知道。”仍然有东西上钩,轻拉着鱼线,表示我以为严重的问题可能并不严重。因为答案就在眼前。
\par 看在老天的分上,卸下他的重担吧!我又成了一个完整的人了!
\par 我们嗅到清新的空气,还有野花和灌木丛散发出来的香气。离开海岸边,寒意就消失了。我们觉得又热了起来,把夹克和衣服里的湿气都蒸发掉了。原来潮湿而沉重的手套也变轻了。我好像被海洋的湿气冻得太久,因而忘了温暖是什么滋味。我觉得有些睡意。前面的一条小溪旁有张野餐桌,到那里的时候我关掉发动机停了下来。
\par 我告诉Chris:“我很想睡一觉。我先睡一下。”
\par 他说:“我也睡。”
\par 于是我们睡了一下,醒来的时候觉得很舒畅。许久都没有这种感觉了。我拿起我们的夹克,把它们夹在车子上绑东西的绳子里。
\par 天气太热,在这种天气里是不需要戴头盔的。我把头盔拿下来,绑在绳子上。
\par Chris说:“把我的也放在那儿。”
\par “你要戴它才安全。”
\par “你也没有戴啊。”
\par “好吧。”于是我就把他的头盔也收起来。
\par 眼前的路仍然永无止尽地向前伸展着,风从树林里吹过来,我们又转了许多大弯,眼前不断出现很多新的景观。
\par 然后我们看到前面出现一座峡谷。
\par 我大声叫Chris:“你看好美啊!
\par “你不需要吼!”他说。
\par “喔。”我笑了起来。拿掉头盔之后,你就可以恢复平常谈话的音量,这么多天之后终于可以把头盔拿掉了。
\par 我说:“真的很美。”
\par 我们又经过了很多树林、灌木丛。
\par 天气越来越暖和。Chris靠着我的肩膀,我把头转过去,看见他站在踏板上。我说:“这样有点危险。”
\par “不危险,我自己会注意。”
\par 他可能会注意,但是我还是说:“还是小心点。”
\par 过了一会儿,我们在树下来了一个大转弯,“喔!”他说,然后又叫,“啊!”
\par 然后又是,“哇!”路旁的树枝非常低矮,几乎要打到他的头。
\par 我问他:“怎么回事?”
\par “风景太不一样了。”
\par “什么?”
\par “所有的一切都变了。我以前都不能越过你的肩看出去。”
\par 一路上树枝在阳光下摆出奇怪而美丽的图案。它们倏倏地在我眼前忽明忽暗地闪过。然后我们又来了一个大转弯,才摆脱了这些树影。
\par 没错,我从来没了解过这一点,这些日子以来,他都坐在我背后。我问他:“你看到了什么?”
\par “情形真的太不一样了。”
\par 我们又来到了一座小树林里,他说:“难道你不怕吗?”
\par “不怕,你已经习惯了。”
\par 过了一会儿他说:“等我长大我可以拥有一辆摩托车吗?”
\par “如果你会照顾它的话。”
\par “那要怎样照顾呢?”
\par “要做许多事情。你看我一直做的就是。”
\par “你会全部教我吗?”
\par “当然。”
\par “很难吗?”
\par “如果你有正确的态度就不难。事实上难的是要有正确的态度。”
\par “哦。”
\par 过了一会儿,他又坐下来,然后说:“爸?”
\par “什么事?”
\par “我会有正确的态度吗?”
\par “我想会吧,”我说,“我想不会有任何问题。”
\par 于是我们又骑过尤凯亚、霍普兰,以及克洛弗代尔,一直来到美酒的家乡。
\par 高速公路十分顺畅。载我们几乎横跨过半个大陆的摩托车依然低低地吼着。于是我们又经过亚斯提和圣罗莎、佩塔卢马和诺瓦托,现在高速公路变得更宽阔,车流也增加了不少,到处都是小气车、卡车和公共汽车。不一会儿路旁就出现住宅、船只和海湾了。
\par 当然,试炼永远没有了结,人只要活着就会发生不愉快的事和不幸的事。
\par 但是我现在有一种以前没有过的感觉,这种感觉并不只停留在表面,而是深入内里:我们赢了。情况正在慢慢好起来。
\par 我们几乎可以这样期待。
\end{document}