\documentclass[UTF8,10pt,a4paper,openany]{book}
\usepackage{ctex,enumitem}
\usepackage[compact]{titlesec}
\usepackage[a4paper,left=2cm,right=2cm,top=1.5cm,bottom=1.5cm]{geometry}
\usepackage[colorlinks=true,linkcolor=blue,urlcolor=blue]{hyperref}
\setenumerate[1]{itemsep=0pt,topsep=0pt,partopsep=0pt,parsep=\parskip}
\setenumerate[2]{itemsep=0pt,topsep=0pt,partopsep=0pt,parsep=\parskip}
\setitemize[1]{itemsep=0pt,topsep=0pt,partopsep=0pt,parsep=\parskip}
\setitemize[2]{itemsep=0pt,topsep=0pt,partopsep=0pt,parsep=\parskip}
% \titleformat{\section}{\centering\normalfont\Large\bfseries}{\thesection}{1em}{}
\titleformat{\chapter}{\centering\normalfont\Large\bfseries}{第\thechapter 章}{1em}{}
\setlength{\parskip}{3pt}

\title{精神分析入门\\ An Elementary Textbook Of Psychoanalysis}
\author{作者:Charles Brenner\\ 译者:杨华渝~沈德灿~陈仲庚~万文鹏~张伯源~钱铭怡\\ 排版:章小明}
\date{}

\newcommand{\signature}[1]{\begin{flushright}\slshape #1\end{flushright}}
\newcommand{\quan}[1]{\textcircled{\small #1}}
\newcommand{\signatureA}{\signature{(陈仲庚~译)}}
\newcommand{\signatureB}{\signature{(沈德灿~译)}}
\newcommand{\signatureC}{\signature{(张伯源~钱铭怡~译)}}
\newcommand{\signatureD}{\signature{(万文鹏~译)}}

\begin{document}
\frontmatter
\maketitle

\chapter*{感谢}
\begin{center}
汉堡促进科学文化发展基金会理事长\\
The Hamburg Foundation forthe Advancement of Science and Culture\\
席佳琳女士\\
Ann Kathrin Scheerer\\[2ex]
德中心理治疗研究院主席\\
German-Chinese Academy for Psychotherapy\\
玛加丽女士\\
Margarete Haa\ss{}-Wiesegart\\
\end{center}

\chapter*{译序}
1986年6月,应德国汉堡促进科学文化发展基金会的邀请,本书译者中万文鹏、沈德灿、张伯源、杨华渝一行四人在德国友人席佳琳和玛加丽两位女士的陪同下,参观考察了德国法兰克福、海德堡、汉堡、慕尼黑及瑞士的苏黎世等城市有关精神分析的研究及治疗机构,受到热情的接待。基金会还赠送我们不少Freud的经典著作及有关通俗读物。Charles Brenner的An Elementory Textbook of Poychoanalysis就是其中的一本,并特别为席、玛两位女士所推崇。建议由我们译成中文出版。

翻阅之后,我们共同感到这确是一本值得推荐的读物。它篇幅不大,却概括了弗洛依德的基本思想与精神分析的主要内容,阐述明确贴切,言简意赅,在同类书中是极为难得的。于是决定把它翻译过来,以便国内读者学习借鉴。

本书自翻译至出版一直得到德国汉堡促进科学文化发展基金会席佳琳女士及德中心理治疗研究院玛加丽女士的关心和支持。在此表示真挚的感谢。

北京出版社决定出版此书,使本书得以问世。在此亦深表谢意。

本书各章的译者为:第\ref{1}、\ref{2}章陈仲庚,第\ref{3}、\ref{5}、\ref{10}章沈德灿,第\ref{4}章张伯源、钱铭怡,第\ref{6}、\ref{7}、\ref{8}章杨华渝,第\ref{9}章万文鹏。全书由杨华渝统核定稿。本书译文中的错误纰漏之处,欢迎广大读者及同行专家不吝赐教。
\signature{杨华渝\\ 1999年7月}

% \newpage
\chapter*{原序}
本书试图将精神分析的基本原理以浅显易懂的形式呈现在读者面前。读者事先不一定要掌握精神分析的知识,而通过阅读本书,可以引导和帮助读者去进一步了解精神分析的著作。主要读者对象是医生、精神病学家、心理学家、社会工作者及社会科学家。通过为读者提供近年来精神分析假说中值得信赖的探索,通过向读者提供精神分析不同发展阶段的某些观点,本书必将有助于读者对精神分析有全面的了解,帮助读者澄清某些混淆和误解。这些混淆和误解容易导致对Freud在其40年活跃的精神分析生涯中不同时期理论的不同产生错误的认识。

我曾先后在New York医院的Westchester分院和Yale医学院的精神病学系研究生专业训练中,对从事精神病学工作的有关人士进行过多年的培训,本书材料的组织即源于这种教学经验。书中每章之后都列出了建议仔细阅读的著作,这是对书中已有内容的补充和扩展。这种阅读必将给精神分析的初学者打下坚实的基础。

\chapter*{第二版前言}
本书写作的目的是向感兴趣的读者介绍精神分析的原理。从本书这些年来颇受欢迎的情况来看,似乎已达到了预期的效果。

第二版在原书基础上增添了新的内容。新加入的两章,一章讨论了在正常的心理机制而不是在病理心理机制基础上产生的心理冲突的作用;另一章则对精神分析迄今为止已获得的成功及未来的前景进行了剖析。

新的版本对文体和内容还进行了某些小的改动。纠正了原版中的错误。对阅读书目和参考文献也进行了修正,以收入原书目中的最新版本和此前仅见于外国版本的某些著作的美国版。此外,还增加了一些参考文献。
我希望这些修正和增补有益于拓展本书的范围。
\signature{Charles Brenner\\ New York, 1972.6}


\tableofcontents

\mainmatter

\chapter{两个基本假设}\label{1}
精神分析是一门科学体系,由Sigmund Freud所创立,至今它的发展仍与Freud的名字联系在一起。很难确定到底精神分析是从什么时候开始的。其原因是,它经过若干年的发展才逐渐成形。到了1895年,精神分析才发展成了一个独立体系。像其他科学体系一样,精神分析理论的产生是来源于观察,然后将这些材料整理和解释。因此,我们把精神分析理论视为人类心理功能及其发展的一系列设想。它是普通心理学的一部分,是迄今为止对人类心理学做出的最为重要的贡献之一。

精神分析理论既涉及正常心理机能,也涉及病理心理功能。认清这一点甚为重要。但是,精神分析又不仅仅只是一种精神病理学的理论而已,在实际应用时,可以治疗精神疾病或心理异常。这种既涉及正常心理,也涉及变态心理的精神分析理论,主要来源于对变态的研究和治疗。

像任何科学体系一样,精神分析理论的各种假设是相互关联的。其中有些假设比其他的假设更为基本,更为确定,获得更多的证据,并且就其意义来讲,已被我们视为确立的心理规律。

两个充分得到证实的基本的假设是:心理决定论原则或因果原则,以及意识是一种特殊的、非同寻常的心理过程。如果把后一种主张用另一种话来说明、那就是:根据精神分析理论,潜意识心理历程在正常及变态心理机能中均占有非常大的优势,并有着重要的意义。本章将阐述这两个基本假设。我们也会看到,这两者是相互联系的。

先讨论心理决定论原则。这一原则的意义是说,心理现象与我们的躯体现象一样,没有任何事情是偶然的或碰巧发生的。每一心理事件的产生,都是一些先前的事件所决定的。我们心理生活中的事件看起来似乎是偶然的,好像与过去的事没有关系,其实不然。心理现象与躯体现象是相同的。心理现象同样不能缺乏与其先前事物的因果联系。在心理生活中,不存在分离中断的情况。

这个原则的理解和应用,对于探讨人类正常心理及病理心理方面,都从本质上提供了明确的方向。如果能够正确地理解和应用这个原则,我们就不会忽视任何心理现象,错把它们看作是没有意义的或者偶然发生的事情。

对于那些我们感到兴趣的心理现象,我们要经常问自己:它是什么原因引起的?为什么会这样发生?我们之所以问自己这些问题,是因为我们相信确实存在有答案。至于是否容易找到答案,则是另一回事。但是毫无疑问,我们知道是有答案的。

下面举用这种方式探讨心理现象的一个例子。日常生活中我们常常忘记了或者放错了某些东西。一般的看法是把这种情况看作是一种“偶然”或“碰巧”发生的事情。然而经过精神分析家75年来对这种“偶然”事件细致的研究(由Freud本人开始),表明它们绝不是一般所说的偶然发生的事。相反,每一个这种“偶然”的事件都能够找到当事人的愿望或意图。这与上面所说的、心理机能的决定论原则是完全符合的。

还可再举出日常生活中其他方面的例子。比如睡眠现象中,常见而又神秘的、引人注意的梦,也遵循这种心理决定论原则。每一个梦或者说每一个梦中的每一个意象,都是由其他心理事件的后果而产生的。而且每个梦和意象都与做梦者的心理生活,在意义上有着联系。

有关梦的问题,我们将在第\ref{7}章中进行详细讨论。读者需要知道,这种关于梦的观点与70年前心理学研究者的观点完全不同。当时的心理学研究者们认为,梦是睡眠中大脑不同部位零乱和不协调活动的结果。显然,这种观点与心理决定论原则相违背。

对于精神病理现象,这一原则也是适用的。精神分析学家实际上也证实了这一假设。每一种神经症症状,不管其性质如何,都是由其他心理历程引起的,尽管病人本人常常把症状看得与自己毫不相干,与自已心理生活没有联系。实际上这些症状与其他心理历程是有关联的,而且可以得到证实。

谈到这里,我们不能不承认我们讲的已不只是第一个基本假设,即心理决定论原则了。我们也涉及到了第二个基本假设。这就是存在有一种个人自己并不知道的心理历程或潜意识的存在及意义。

两个假设的关系十分密切,实际上在讨论第一个假设时人们很难不涉及到第二个假设。确切的事实是,在我们内心所经历的大都是潜意识活动。这是我们所不知道的部分。于是就造成一种假象,似乎我们的心理生活是不连续的,我们的思维、情感、偶尔的遗忘、一个梦或一种病理的症状,似乎与先前心理中所有过的现象并无联系。而事实上,这些联系存在于心理历程中的潜意识部分,而不是在有意识地进行。如果能够发现潜意识的原因,那种表面上的非连续性就不存在了。因果联系和顺序也就立即显示出来了。

举一个简单的例子。有人在哼唱着一个曲调,自己却不知道从哪里学到的这个调子。此人的心理生活,从表面上看似乎失去连续性。一个旁观者作证,告诉我们说,人以前听到过这个曲调,感觉印象即听觉的印象使得这个人哼了出来。由于此人并不意识听到过这个曲调,他的主观经验就表现为思想上的不连续性。这种情况需要旁观者提醒,才可改变表面上的中断现象,从而使人看清因果之间的联系。

上面讲的是个简单的例子。实际上潜意识心理历程,很少像此例的听知觉那样简单而易于被人发觉,人们很自然地会问,有没有用以揭示个人自己不知道的心理历程的一般性方法?能否直接观察这些心理历程?如果不能,那么Freud又是如何发现这些心理过程的呢?

事实上,我们还没有直接观察潜意识心理历程的方法。观察这些现象的所有方法都是间接的。我们可以由此做出推论,以表明这些现象确实存在,确定其性质,并阐述潜意识对于人们心理生活的重要意义。我们现有研究潜意识心理历程的方法,是Freud经过多年发展而来的。Freud称之谓精神分析(psychoanalysis)。这种方法十分有用又十分可靠,通过应用此方法,他辨明和发觉那些原来曾经隐藏的、出人意料的心理历程;通过应用这个方法,他也认识到潜意识心理历程对每个人(不论是心理健全还是心理不健全的人)心理生活的重要性。下面对Freud技术的发展作一简要回顾。

Freud的自传(1925)曾记述,他先是作为一名神经解剖学家从事医务职业。这是一种甚有竞争性的工作。为了谋生他操业神经科,并对现今称为神经症和精神病的病人进行治疗。当时。神经科的每一位医生都要治疗这些病人。只有专职的研究人员或专家不用处理这些私人治疗业务。Freud开始行医之时,尚没有理论上的、即以病原学为指导的精神疾病治疗方法。当然,那时整个医学领域对此也所知不多,细菌学尚处于初级起步阶段,无菌外科学刚刚开始发展,生理学和病理学方面的进展还未能对治疗病人做出实质性贡献。我们至今相信,一个医生的医学训练愈完备,他的医疗效果愈佳。由于医疗效果的提高,临床医学才发展成为一门科学。在一个多世纪以前,情况则不是这样,一名有完好训练和有学问的医生,虽然在诊断上有一套,在治疗疾病时却不比无知的江湖庸医强多少。读到Tolstoy小说中关于蔑视医生的描写时,我们会感到奇怪,以为这是作者的癖性所造成的。这正像后来另一位著名小说家Aldous Huxley所深信的那样,认为不再需要用矫正性透镜来治疗近视眼。实际上,在Tolstoy的早先时期,连得到最好训练的医生也确实不能治好病人。由此,他的批评性嘲笑完全有其道理。直到十九世纪后半叶,在大学里所讲授的医学,在疗效上才优于自然疗法、宗教科学、顺势疗法或迷信偏方。

Freud是一位受过正规训练的科学家,他采用了十分有效的治疗方法。例如,对于癔症症状,他采用当时著名神经病学家Erb提倡的电疗法。Erb的许多方法现今仍在临床电生理学中应用。然而,Freud告诉我们说。他的结论是,这种治疗对癔症无效。1885年Freud来到Paris,在Charcot的诊所学习了几个月。他掌握了催眠术,以此方法引起和消除癔症的症状,通过催眠后暗示来消除病人的症状,并取得不同程度的疗效。就在此时,他的朋友Breuer告诉他一种治疗癔症的经验,这对后来精神分析的发展至关重要。

Breuer是一位天赋极高的临床医生,得到良好的生理学训练。他曾与人合作,发现了呼吸系统反射,即Hering-Breuer反射。他首先应用吗啡(Morphine)治疗急性肺部水肿。Breuer告诉Freud,几年之前,他用催眠术治疗一例癔症妇女,发现这位妇女在催眠中重新回忆起过去的经历,与症状有关的情绪再一次得到体验,这时她的癔症症状就消失了——这表明病人在催眠状态下可以将产生症状的原因谈出。Freud热切地把这一方法用于自己的病人,得到了同样良好的疗效。他的这些医疗结果,后来发表在与Breuer合作的论文和专著(1895)中。

随着Freud工作的深入进行,他发现不是对任何人都一律可以诱发催眠的,而且治疗的效果也不稳定。另外,有一些女病人在催眠治疗中对他产生了性的依恋,Freud对此非常厌恶。这时,他回忆起法国催眠师Bernheim的一次实验,从而解决了他的困难。Bernheim有一次当众表演催眠性遗忘病(Freud是这次的观众之一),只要催眠师对被催眠者强调说将会失去回忆,被催眠者在催眠时所经验的遗忘就可以持续下去。如果反复坚持这个要求,即使病人不进入催眠状态之中,也能记住他应该忘掉的内容。在此基础上,Freud证明他可以不用催眠解除癔症性遗忘症。精神分析技术就这样开始了。它实质上是病人在解除了意识的指向或稽查的束缚的情况下,向分析者全盘托出他心里的一切想法。

在科学史上,一种技术的革新必然开拓出一个新的世界,从而使以前认识不正确的或了解不全面的东西,得到解释或提出有效的理论假设。Galileo发明了望远镜,通过这一技术使天文学领域发生了巨大的进展。Pasteur应用显微镜研究传染性疾病,也使这一科学领域发生了变革。精神分析技术的发展和应用,使其发现和应用者——Freud本人对精神病学的理论和实践产生了革命性的变革,尤其对于心理治疗是一项重大变革。从而对整个人类心理学做出了重大的贡献。

病人解放了思想中意识控制的重要价值在于:在这种情况下,病人想的和说的都是潜意识的思想和动机。因此,通过倾听病人的“自由”联想,Freud可以掌握病人思想中的所有潜意识过程。这里说的“自由”,是指从有意识的控制中自由出来的意识。Freud处于这种得天独厚的地位来研究病人的潜意识心理过程。通过多年与病人的接触及细致地观察,他发现不仅仅癔症病状,许多正常的和病理的行为以及思维活动,也是内心潜意识表现的结果。

在研究中,Freud发现潜意识心理现象有两大类。第一类包括思维、记忆等,通过注意的努力,可以将它们转回到意识。Freud称这一类易于进入意识的心理成分为前意识(preconscious)。任何思想如果在某种情况下是可能意识到的,那么不论在意识到之后或在没意识到之时,都是前意识。另一类更为重要的就是潜意识(unconscious)现象,需要付出巨大的努力才能使它们回到意识中来。换言之,有一种极大的力量在阻碍着。只有消除了这一阻碍才能使它们回到意识中来。癔症遗忘症就是这种例子。

Freud把第二类心理现象称为严格意义的潜意识。他指出,这种潜意识过程对心理机能施加十分重要的影响,这种潜意识过程在精确性和复杂性方面完全可以与意识过程相媲美。

如前所述,我们还没有办法可以直接观察潜意识的心理活动。我们只有观察被分析者向我们叙述他们的思想、情感以及行为表现。这些所观察到的资料都是由潜意识心理活动派生面来的。由此我们可以推断出潜意识活动的本身。

采用Freud设计的分析技术,所得材料将会特别全面而清晰。然而,还有另外一些资料来源也说明了我们的基本论点,即潜意识心理过程可以影响我们的思想和情感。下面再讲一下它们的性质。

为人们熟知的催眠后暗示,可以表明这些活动的性质。暗示处于催眠状态中的被催眠者,让他在脱离催眠状态之后做某一件事,例如告诉他:“在时钟敲两点的时候,你从椅子上站起来,把窗户打开。”同时,在回醒以前,也告诉被催眠者,要忘掉催眠状态中发生的一切。之后,让被催眠者回醒。在被催眠者回醒后不久,时钟敲了两下,他就站起来打开了窗户。如果问他为什么要这样做,他或者回答说;“我不知道,我觉得这样好”,或者更为常见的情况是进行合理化地解释,如说他觉得很热。这里的关键是,在他做这些活动时,他并没意识到催眠者曾命令他这么去做;也不能通过任何简单的记忆活动或内省,意识到自己的真实动机。这个实验清楚地表明,真正的潜意识心理历程(此例表现为遵守命令)对于思想和行为具有动力性的或动机的效应。

也可以从临床或一般的观察中得到其他的证明,例如某些梦的现象。关于梦和整个做梦的研究,必须应用Freud所设想的研究方法,即精神分析的方法。实际上,Freud对梦的研究和技术是他的主要贡献之一。他的《梦的解析》(Freud, 1900)在任何时代都被列为最伟大而具有革命性的科学著作之一。如前所述,有关梦的心理将在第\ref{7}章进行详述,本章不需要对梦的解析做细致的介绍。这里只需要阐述如下的一些观察。

在有关北极探险的杂志文章或航海记录中,可以看到某些资料,说明饿得要死的人总要梦见食物或者梦中大吃。我想,我们很容易地认识到是饥饿引发了这样的梦,人们在醒来的时候也完全会意识到自己处于饥肠辘辘。可是,梦见在宴会上狼吞虎咽之时,他们并没意识到饥饿,只是在梦中他们觉得饱饱享受了一顿。由此我们可以说,正在做梦的时候,做梦者的心理正在进行潜意识的过程,从而产生梦的表象。梦则成为被意识到的经历。

另一些与生活需要有关的梦,如醒来时感到口渴的梦见喝水、醒来时有如厕需要的梦见小便或大便,都表明睡眠中的潜意识心理活动能够产生意识的活动。上述的这些例子,就是潜意识里伴随躯体感觉而产生的愿望所导致意识里满足需求的梦。这些情况不用特殊的观察技术也能获知。然而,通过精神分析技术,Freud能够证明在每一个梦的背后都有活跃的潜意识思想和愿望。因此,它确立了一条放之四海而皆准的规律,认为梦的产生是由于做梦者潜意识心理活动的结果,如果不采用精神分析技术,它们不可能被意识到。

直到十九世纪的最后十年,Freud才开始给对梦进行研究。在此之前,梦的问题一直被人们所忽视,不作为严肃科学研究的对象。加上在Freud以前从未有过有效的研究技术,因此,虽然有人对梦做了认真的探讨,终也不能对此问题有所裨益,精神分析方法建立以后,Freud对梦的现象能够做出比任何前人更为丰富的发现。

还有一些现象也引起Freud的兴趣。这些现象表明,潜意识心理活动如何影响我们的意识行为。它涉及心理生活的正常领域。像梦一样,过去这些现象往往被人忽视,因为在精神分析方法出现以前,无法对它们进行研究。有关这些问题在第\ref{6}章中将详细讨论,在这里只做简短的介绍。我们称为日常失误却是在醒觉时出现的,而不像梦那样在睡眠时进行。失误包括口误、笔误、记忆错误,以及与此相似的一些过失动作。像梦一样,有些失误十分浅显,我们可以准确地猜到并确信其潜意识意义是什么。比如对于不愉快的和厌恶的事情(如付账单)非常容易遗忘是众所周知的。与此相反,多情的恋人不会忘记相爱者与他的约会。如果他忘了约会,他也能找出原因来,忽视她的潜意识如同有意识的意图。一个青年对是否要着手结婚犹豫不决,在他驾车去参加婚礼的路上,他看到交通信号就停车,在信号灯改变颜色时,他才发现自己停在绿灯前面,而不是红灯使他停下来。另一个更易识破的例子,可以称为症状性行动,而不被看作一种失误。一位精神分析家为了自己的方便,取消了先前的预约。病人在应该去进行治疗的那一天,感到无聊之极,就设法买来一对古代的决斗手枪。这样,一旦在他应该躺在治疗家床边进行分析的时候。他就可以用手枪对准一个目标进行射击。不需要病人的联想就可发现,他对分析家不让他去做治疗甚为愤怒。我们可以说,与梦的分析相同,Freud采用的精神分析技术表明,潜意识心理活动对一切失误的产生都起着作用。不仅在意义明显的活动中,即使在意义不明显的活动中也同样有潜意识的作用。

还可以举出别的证据来说明潜意识心理过程对个体心理生活具有意义的观点。一个人行为的动机。本人虽然不知道,旁观者则往往清楚。可以从临床的和个人的经验中遇到这种事例。一个企图控制并过分要求自己孩子的母亲,会自认为她是最自我牺牲的母亲,她只为了孩子好,而毫无为己着想的意愿。对于这个妇女,我想我们大多数人都会说这个妇女在潜意识里有管辖和控制她孩子的愿望。对此,她自己不但不得而知,而且还会否认有过任何这种要求。另外一个令人感兴趣的例子是,和平主义者惯于与任何人剧烈争吵,如果有人反对他强行的观点,他是决不饶人的。很明显,他的和平主义意识伴随着要争斗的潜意识愿望。在此例子中,这种要争斗的潜意识愿望,正是他在有意识的态度中所反对的。

至此为止,我们列举了在正常心理生活中,关于潜意识心理过程确实存在的证据。然而,实际上Freud最早并最主要的是在精神疾病患者身上的症状中,发现了潜意识心理活动的重要性。在症状中,包含有病人不知道的意义。Freud的这个发现,已为人们普遍接受和理解,不需要做更多说明。如果一个病人患有癔症性失明,我们当然可以假设他潜意识地有某些不愿看见的事物,或者他的良心禁止他看到这种事物。当真说来,要对一个症状的潜意识意义做出正确的猜测,决不是轻而易举的事。对于一个看来甚为简单的症状,产生它的潜意识根源可能有许许多多、相当复杂。因此,即使能够正确猜到它们的意义,猜到的也只是全部真实情况的一部分,有时则是很小的一部分。这里所讲的仅是某些不重要的情况,列出来的目的是为了说明潜意识心理过程这一基本假设。

回过头来,通过上述的例子,我们可以认识到潜意识心理活动既影响健康人的有意识思想和行为,又影响心理障碍人们的有意识思想及行为,就像在催眠实验情境时的影响一样。虽然不用精神分析技术我们也可以看到潜意识的力量,然而我们应该记住,由于应用了这种技术,这些首创性发现才有可能;而且对于彻底地研究潜意识心理现象来说,这是最基本的手段。

这些研究使Freud坚信,心理机能的绝大部分是在潜意识中进行的,而意识则是一种异乎寻常的(而不是寻常的)心理机能。这种观点与Freud以前普遍的看法大相径庭。那时,人们把意识与心理机能看作同义词。我们今天深信并非如此,意识虽然是心理作业的重要特性,但并非必然特性。我们相信,意识并不需要,而且常常也并不依附子心理作业,甚至在决定个体行为的心理作业中也没有依附关系,就连那些最为复杂的和精确的行为也是如此。这些心理作业虽然复杂而重要,在相当程度上是潜意识的活动。
\signatureA

\chapter{内驱力}\label{2}

前面所讨论的两个假设是整个精神分析的基本理论或理论基础。精神分析理论的其他部分都建立在这两个假设之上。也就是说,所有的有关心理结构的假说,以及这些心理结构功能作用的假说、都是由这一理论所产生和决定的。

精神分析理论认为,本能的内驱力是产生心理活动的能量。下面我们将通过本能内驱力来继续对精神分析理论的心理模式进行探讨。

Freud所建立的心理学理论一直是由生理学所指引的,而且他总是尽可能地这样做。实际上,从最近出版的他的书信来看,Freud在十九世纪90年代的早期就致力于建立一种神经病学的心理学。尽管由于事实上这两个学科之间没有令人满意的关系。致使他放弃了这方面的努力,Freud却仍然像现代的某些精神病学家以及心理学家那样,坚信心理现象总有一天要用大脑机能方面的活动进行描述。虽然在这方面已经做了某些努力,至今依然没有什么满意的发现,也没有谁能说清楚到底哪一天这种努力会获得成功。就目前来说,精神分析与生物学的其他分支之间还没有什么理论上的联系。本章将要讨论其中与心理机能有关的两个主要方面,即感知和本能的驱力、即“内驱力”(Triebe/drives) 。

首先要谈一下术语的问题。这里所谈的内驱力在精神分析文献中通常称为本能(Instinkt/instinct)。虽然目前“本能”这个词要比“内驱力”更广为人知,在这一章里还是用内驱力这一慨念更为恰当。因为、尽管它们之间紧密相关,本能在低等动物就有。而人类的心理机能则最好用内驱力来形容。这里两者的区别在于,本能是对某一特殊刺激以某种模式化的或恒定的方式做出反应的先天能力——这种反应模式通常是复杂的行为,比简单的反射、如膝跳反射,要复杂得多。然而,像简单的反射一样,具有中枢神经系统的动物的本能也包括刺激、中枢兴奋以及由此而引起的运动反应几个部分。可是,我们称之谓人类内驱力的,并不包括运动反应,仅存在对刺激做出反应时的中枢兴奋状态。这种兴奋状态之后的运动活动,是由心理的一个高度分化的部分。即精神分析术语中的“自我”(ego)所中介的。因此,组成内驱力兴奋状态的反应(或本能的紧张)就可以受到经验和反射的矫正,而不是像低等动物中的本能那样,以一种前定的方式进行反应。

这种人类的本能生活和低等动物的本能之间的区别也不应太绝对化。例如,成人中的性驱力与性乐高潮这一先天的反应模式之间,就有着密切的联系。另外。在人类的本能驱力中,运动反应一般是由基因因素预先决定的。然而,人类的行为受本能影响的程度与其他运动相比要小得多,而环境和经验对人类行为的影响比较明显。为了说明这一区别,我们在这一章中采用了“内驱力”这一概念,而不是采用“本能”。

内驱力是一种先天决定的心理成分,当它发生作用时就产生一种心理兴奋状态,或我们通常所谓的紧张。这种兴奋或紧张能够推动个体的活动,这种活动一般也是由先天所决定的,但它也可以由于个体的经验不同而有一定程度的变化。这种活动要么导致某种兴奋或紧张的终止,要么导致某种满足。从字面上看,前者是一种比较客观的概念,而后者的主观性则大得多。由此我们可以看到,内驱力发生作用的特点,表现为一个连续的序列。我们称这种序列为紧张、运动活动及紧张的消失,或者称它们为需要、运动活动以及满足。前面的几个术语忽视了主观体验的成分,而后面的几个则特别强调这一一点。

内驱力所具有的推动个体活动的能力,使Freud深受启发。他把内驱力作为一类心理能量,并把它定义为进行工作的能量。Freud又假定有某种心理能量是内驱力的一部分,或从某种意义上来说,内驱力来自这种能量。这种心理能量与物理能量有着根本的区别。我们还没有对它们进行这种类比。心理能量的概念也和物理能量的概念一样,也是一种假设,其目的是简化我们所能观察到的有关人类心理生活的事实,以便于理解。

Freud还通过对某一特殊客体或个人所投入的心理能量。将他的心理学假设与物理学做了进一步的类比,对于这一概念Freud采用了一个德文词Besctzung,翻译成英文即cathexis(投注)。“投注”的确切定义,是指投向某个人或某件事的心理代表物的心理能量的数量。这就是说,投注是一种纯心理的现象,是一个心理学概念,而不是物理学概念。心理能量不能通过空间直接投入或传递给外部的客体。它所能够传递的只是各种各样的记忆、思想,以及组成我们所谓的心理代表物的客体的幻想。从心理学意义上来说。投注越大,客体就越“重要”,反之亦然。

让我们通过一个孩子的例子来说明投注的定义。这个孩子的许多重要的、本能的满足均由他的母亲提供。用我们新的术语来说,这个孩子的母亲是他的内驱力的重要客体,这个客体得到了心理能量的高度投注。这也就是说,这个孩子对有关他母亲的思想、表象以及幻想等(即她在孩子心目中的心理代表物)是予以高度投注的。

再强调一下,心理能量的概念已经在精神分析学家中引起了很多的争论,并且已经引起了不小的混乱。其中的很多问题似乎均起因于“能量”一词,在物理学中有许许多多种能量:动能、势能、辐射能等,似乎心理能量也像是这几种物理能量形式的一种。但事实并不是这样,心理能量只是心理学概念中的一个术语,而不是物理学中的术语。实际上,从某种意义上讲,心理学是中枢神经系统活动的一个方面。它是动物生物学的一个分支,因此也是物理学和化学的分支。正如前面所指出的,目前我们对于两者之间的关系还所知甚少。我们还不清楚什么样的大脑活动,什么样的物理过程会伴随某一希望、渴求,或某一特殊满足的需要。在我们真正了解这些以前,我们无法将物理能量和心理能量进行类比。我们必须认识到在目前情况下我们知识水平的局限性,并且尽量不去解物理能量和心理能量之间的毫无意义的方程。某些人把热力学定律应用于心理能量,热衷于讨论心理过程的热力学函数,都属于胡闹。

下面讨论内驱力的分类和性质。Freud关于内驱力的分类假设在1890年至1920年的30多年间有着较大的变化和发展。在随后的10年中,别人又在他的观点中加进去了很多重要的内容。最早的分类中,他把内驱力分为性驱力(sexual drive/ Sexualtriebe)和自我保存驱力(self-preservation drive/ Selbsterhaltungs-triebe),很快他就放弃了关于自我保存驱力的假设。因为他认为这是一种不能令人满意的假设。通过多年的研究。他发现所有本能的活动是性驱力的一部分,或者由性驱力所派生。然而,在对各种各样心理现象的研究中,特别是关于施虐狂和受虐狂的研究、最终使得Freud又一次修正了他的理论,在《超越快乐原则》中他所阐述的关于内驱力的理论,尽管在他提出的当时并没有被所有的分析学家完全接受,可如今已被精神分析学家所广泛接受了。

在他最后的分类中,Freud提出心理生活的本能存在两种内驱力,即性驱力和攻击驱力(aggressive drive)。虽然从字面上的意思看,这种二分法与我们通常所说的性和攻击有着一些相似,实际上简单地给这两种内驱力一个定义是不可能的。如果我们说其中一种内驱力说明了心理活动的性欲成分,而另一种代表破坏成分。就有点接近它们的意义了。

Freud驱力的二分法理论中最重要的假设是,在我们所能观察到的所有本能现象中,不论是正常的还是病理的,性和攻击两种内驱力都参与其中。用Freud的话来说,就是这两种内驱力是有规则地融合在一起的,尽管两者的数量不一定相等。

因此,甚至在最冷酷无情的故意伤害中,虽然从表面上看来它只是满足攻击的行为而已,但做出这一行为的人在潜意识水平上也获得了一定释度的性欲满足。同样,不论多么温柔的性行为都不可能不同时伴随潜意识中攻击驱力的释放。

换句话说,我们所假定的那些内驱力。在人类行为中不可能以单独或非混合的形式出现。它们是从经验中所抽象出来的,是一种假设,一种采用当前时髦的术语进行命名的操作性概念。我们相信,它们能帮助我们以最简单最系统的方式去理解和解释我们的经验。我们不可能在临床病例中找到一个完全孤立于性驱力而只存在攻击驱力的例子,也不可能找到一个完全孤立于攻击驱力而只存在性驱力的例子。攻击驱力并不等同于我们通常所说的攻击行为,而性驱力也不同于对性交的渴望。

在我们目前的理论中,我们区分了两种不同的内驱力。我们称其中一种为性或性欲驱力。另一种为攻击或破坏驱力。为了说明这种区别,我们又假定有两种心理能量,分别与性驱力和攻击驱力相联系。前者有一个特殊的名字,称为libido。后面一种没有这样的名称,尽管有一段时间人们建议把它称为类似于destroy(破坏)一词的destrudo。人们通常习惯于把它称为攻击能量,有时也简单地称之为“攻击性”。但这后一种用法却带来了一些麻烦,因为正如我们刚才所讨论的,攻击能量和攻击驱力的含义是不一样的,它们也都不等同于我们通常所说的攻击行为。因此,用同样的词去描写它们,只能带来一些不必要的混乱,会模糊了它们之间应有的区别。

应该认识到的重要一点是,我们现在的理论中划分内驱力为性驱力和攻击驱力是基于心理学的证据。Freud最初的分类中,试图把内驱力的心理学理论与生物学的基本概念联系起来,分别把这两种驱力称为求生驱力(Eros)和求死驱力(Thanatos),大致与合成代谢和分解代谢的过程相一致。这些代谢过程比心理学的过程要明显得多,具有所有有生命物体的本能特点,即原生质本身的本能。

不管Freud的生物学设想是多么正确或多么不正确,它确实造成了许多误解。我们不应该过分强调我们目前采用的对内驱力的分类仅仅是根据临床上的资料。不论Freud关于求生驱力与求死驱力的观点是否正确,都与此无关。实际上,目前有一些分析家接受了求死驱力的概念,而另一些人(可能占大多数)则不然。但是,无论接受与否,他们在临床水平的评价上都认为本能现象是由性驱力和攻击驱力组成的混合体。

最初,Freud把内驱力定义为一种来自体内的对心理的刺激。因为在当时他所关注的只是性驱力,则这一定义正好与以下的事实相符,即不仅性兴奋和满足与躯体各个部分的变化和刺激有着明显的关系,各个内分泌腺体所释放的激素也与性生活和性行为有着直接的关系。然而,攻击驱力与躯体关系的证据仍不清楚。人们开始时认为,攻击驱力与骨骼肌收缩之间的关系,类似于机体的性驱力与性兴奋敏感区之间的关系。由于目前还缺乏生理学、化学或心理学等方面的证据来支持这个设想,因此多数人已经放弃这一假设。也就是说,依然假设攻击驱力的机体基质是以神经系统的形式和机能实现的。有些分析学家不希望走得这么远,他们把攻击驱力的生物学基础,看做是目前还疑而不解的间题。

不过分纠缠于这些理论上的间题,而将内驱力的各个方面与我们可以观察到的事实联系起来,这种做法会更为有效。显然这样做的方法很多,其中对理论和实践都有好处的方法是讨论内驱力的遗传发展。

为简单起见,让我们先讨论性或情欲驱力,因为我们对它的发展和变化要比对攻击驱力更为熟悉。精神分析理论假定这些本能力量在婴儿时期就已发挥作用,影响婴儿的行为,并且要求得到满足。这些本能力量后来变成给成人带来幸福以及痛苦的性欲。实际上。这里用“假定”一词是不太合适的,因为它已经得到充分地证实。

证明这一命题的成立、至少有三方面的证据。首先来自对儿童的直接观察。如果不带任何偏见,客观地与幼小儿童谈话或观察他们时。就会发现他们存在性的愿望和行为。难就难在,每个人都需要忘记或否认自己儿童早期的性的愿望和性的冲突,于是乎在Freud以前的研究几乎没有可能发现在儿童中的性愿望。另外两方面的证据来自对儿童和成人的分析。前一证据可以直接地由观察婴儿性欲的表现及其意义得来,而后者则由推断得来。

必须明白,3-5岁儿童的性愿望与成人的性愿望十分相似。了解到这一事实之后,就会毫不犹豫地将用于成人的词汇也用于儿童。我们怎样才能观察到童年早期性驱力的派生物或现象呢?按照Freud的看法。我们可以根据如下的观察:\quan{1}在正常的发育过程中,某些童年早期可以得到快感的行为后来变成为能够得到性兴奋和满足的行为,如亲吻、窥视、爱抚、裸露等。\quan{2}在某些性心理发展异常(性变态)者中可以看到,童年早期某些种类的性兴趣或性活动(常见的有肛门、口以及窥视等)变成成人性满足的主要方式。\quan{3}对神经症病人采用精神分析法进行心理治疗的证据表明,病人心里这种“变态”的欲望很是活跃。然而,这些神经症病人不像性变态者那样能意识到“变态”的欲望,并且引起性的兴奋,这些神经症病人的“变态”的欲望是在潜意识中进行的,并且引起他们的焦虑和内疚。

我们现在可以谈谈自婴儿期开始性驱力现象的图式化序列。Freud在他1915年《性理论的三篇论文》中对这种序列做了最初的描述。

必须明白,这里所描述的各个阶段并不像图式里介绍得那样明显,两个相继的阶段会有一定的重叠。因此,两个阶段的转换是一个渐进的过程。还应注意的是,这里所绐出的每一个阶段的持续时间只是个大略的平均数。

在生命的第一年到一年半的时间里,婴儿主要的性器官是嘴、唇以及舌头。这意味着,这时婴儿主要的欲望和满足在口部。这种设想的证据可以通过对婴儿的直接观察而来。但在很大程度上由对较大的儿童及成人的分析推断而来,因为即使年龄大了仍能从吸吮、口衔及叼咬中获得快感。

在生命的第二年到第二年半之间,消化道的另一端,即肛门又成为性紧张和性满足的最重要的部位。这时愉快和不愉快的感觉都与憋便和排泄有关。这些机体过程,连同排泄物本身及其气味都成为儿童非常感兴趣的客体。

随着生命第三年的到来,性的部位开始转向生殖器。在正常情况下,这种转化将会一直保持下去。性发展的这一阶段被称为性器(phallic)期有以下两个原因。首先,不论什么性别的儿童在这时都把阴茎作为主要感兴趣的客体。第二,这一年龄阶段小女孩的性兴奋和性快感的器官是她的阴蒂,而在胚胎形成的过程中它相当于男性的阴茎,即使在她以后的生活中也还是如此,只是阴道的作用逐渐取代了阴蒂而已。

这就是儿童性心理发展的三个阶段,即口期、肛期以及性器期。性器期在青春期的性器官化中再行出现,成为成年人的生殖器(genital)期。童年期的性器阶段和成年期的生殖器阶段是有区别的,因为只有到了青春期才具有性乐高潮的能力。然而,在精神分析文献中,这两个阶段名称的应用并不都是恰当的,如把“性器”(phallic)误用于“生殖器”(genital ),尤其是童年的口欲和肛欲期往往被称为“前生殖器期”(pregenital phase),而不是“前性器期”(prephallic phase)。

儿童性的表现除了上述的三个主要阶段之外,这时期的性驱力还有一些值得注意的现象。其中之一就是窥视的欲望。这在性器阶段表现得最为明显,同时还表现出窥视的反面,即裸露的欲望。此时,儿童一方面希望窥视别人的生殖器,另一方面又希望把自己的生殖器也裸露出来。当然,这种好奇心和裸露欲也表现在身体的其他部位以及躯体的其他机能方面。

儿童期另一种较为常见的性表现与尿道及排尿有关,被称为尿道欲。排尿时的皮肤感觉及气味、声音也与此有关,而在这些方面,不同的儿童又有所不同。这种相当重要的性方式的变异,是由于儿童之间的体质差异造成的,或由于儿童所处环境包括挫折或诱惑等的差异造成的,到目前为止还没有一个满意的答案。精神分析学家,包括Freud在内,都认为某些情况下体质因素为主,某些情况下环境因素为主,而在大多数情况下这两种因素同时都起作用。

我们已经描述了儿童期性驱力正常发展时的阶段序列。由此可以设想,不同阶段的儿童满足性驱力的客体和方式会有不同。比如,在口欲阶段,奶头或乳房要远比在肛欲和性器期重要,而吸吮作为性满足的一种特殊方式在口欲早期也特别重要。我们也可以看到,这些变化是渐进的,面不是跳跃性的。旧的客体和满足方式甚至在新的客体和满足方式已经建立以后,还要保持一段时日才会逐渐被抛弃。那也就是说,前一阶段对某一客体的libido投注随着下一阶段的到来而逐渐减小。尽管这一投入减小,在新的阶段到来以后仍然还要持续一段时间,然后逐渐将投注转移到在这一新阶段中的客体上去。

心理能量理论为我们提供了一个既简单又符合事实的能说明其间变化的解释。那些在先前阶段中贯注到客体上或者满足方式上的libido会逐渐地脱离出来,并转向下一阶段中的客体或满足方式。因此最初集中在乳房(确切地说是乳房的心理代表物)上的libido会逐渐转向粪便,进而会转向阴茎。按照这个理论,在性心理发展的过程中,libido会从一种客体转移向另客体、或从一种满足方式转移向另一种满足方式。这种转移在很大程度上是由人人均相同的体质因素决定的,也可能因人而异。

然而我们应当知道,所有真正强烈的libido投注都不会被完全放弃。正常情况下,在大部分的libido转移向另一客体之时,仍然还会有一部分滞留在原来的客体上。这种婴儿期或童年期libido的投注滞留在后来生活中的现象,称之为libido的“固着”(fixation)。如果一个男孩仍然固着于他的母亲,那么在他的成人之后就不能将其情感正常地转移到另一个妇女身上去。另外,“固着”一词还用来表示一种满足的方式。因此,我们可以说某人固着于口欲或肛欲的满足方式。

“固着”一词通常意味着一种病理现象。这是由于童年早期投注的滞留现象,最早是由Freud在神经症病人中发现并描述的。如前所述,固着似乎是心理发展的一种一般性特点。当它过分时可能会引起病理现象,也有可能是其他一些目前还没有发现的因素取决了它是否与精神疾患有关。

无论是对某一客体的固着还是对某一满足万式的固着,它通常完全是潜意识的或至少部分是潜意识的。我们可以设想比较强烈的固着,即比较强烈的投注的滞留可能是能够意识到的,而比较弱的固着可能是潜意识的。可是,从已有的证据来看,实际上并没有发现在持续性投注的强度与意识的接受程度之间有何关系。例如,童年期性的兴趣尽管有着非常强烈的投注,随着儿童的长大,这种兴趣的绝大部分都被遗忘了。在描述这一过程时,“遗忘”一词是非常暧昧的。准确地说,是由于这些兴趣的记忆转入了意识而阻断了它们的能量。其他的事情也有同样的情况,而那些后来发生的固着也是如此。

除了我们前面描述的在性心理发展过程中libido向前流动以外,libido可能还会倒退,这种倒退称之为“退行”(regression)。这里,这个词被用来与某种驱力相联系,指的是本能的退行,指的是又重新回到原先的客体或满足方式。

本能的退行与固着有着密切的联系,因为当发生退行时,往往会倒退到个体已经发生固着的客体或满足方式上去。如果一种新的快乐不能得到满足从而被放弃掉,个体就有可能再回到原先能够得到满足的快乐上去。

一个说明这种退行的例子表现于小孩子对他新出生的弟(妹)的反应。由于弟(妹)的出现会分享每亲对他的爱及注意,尽管在弟(妹)出生几个月之前他已停止了吸吮大拇指,而当他的弟(妹)出生以后他却又开始了吸吮。这个孩子退行到libido满足早期的客体——大拇指,而这种早期的满足方式就是吸吮。

就像我们在例子中举的那样,退行往往是在不利的条件下产生的,但并不一定都是这种情况。儿童或者成人在玩肛门、开玩笑时,就有以退行行为取得快感的表现。退行并不等于就是病理现象。在某些条件下,它是心理生活的正常现象,而在另外的情况下,可能是一种不利的或病理的现象。

这里还应该特别提到婴儿性活动的一个特点,即他们与所渴望的性客体(通常是人)之间的关系。举一个简单的例子,如果婴儿不能经常吃到母亲的奶,那么他就会吸吮自己的拇指或脚趾。这种通过自身去满足性需要的能力称为自我情欲(autoerotism)。这样、儿童就不一定非得从外界环境中去获得满足。可是,这种情况容易使儿童与外部现实世界脱离,造成对自己的兴趣过分集中,甚至惟一独占、排斥其他,产生像精神分裂症那样的严重病理表现。

我们现在再来讨论攻击驱力。我们必须承认,关于攻击驱力的描述远要比对性驱力的描述少得多。这很大程度上是由于,在1920年以前Freud一直没有把攻击驱力当做人类心理生活中一个独立的本能的驱力,使得它没有像生驱力那样很早就被提了出来,并且一直是个特殊的研究课题。

攻击驱力的表现与我们已经描述的性驱力一样,有固着和退化的能力,也有从口向肛门再向性器的转化的序列。也就是说,幼小婴儿的攻击冲动是通过咬这样的口部活动来释放的。稍长大一点,憋便和玩便就成了释放攻击驱力的重要方式。而更大一点的儿童则把阴茎以及阴茎的活动作为(或至少在幻想中认为)破坏活动的武器和手段。

显然,攻击驱力与我们前面描述的身体各个部位之间的关系不像性驱力与身体各部位之间的关系紧密。例如,五六岁的儿童很少把他们的阴茎当作攻击冲动的武器,他们一般是用手、牙齿、脚以及言语等作为攻击冲动的武器。然而在游戏和幻想之中,所使用的矛、箭、枪等,可以通过分析发现它们在潜意识思想中代表了他们的阴茎。这表明在潜意识想象中,他们是以自己强有力的而又危险的阴茎去征服他们的敌人。尽管如此,我们还是要承认性驱力与攻击驱力相比较。性驱力与躯体的性感带——无论是躯体的同一部位抑或任何相似的部位——的关系要比攻击驱力密切得多。这种区别在最早的口期还不怎么能被看得出来。因为。只有几个月的婴儿除了口以外没有什么其他可用的东两,从而我们可以设想这时也跟性驱力的释放(吸吮、口衔)一样,口的活动也就成了婴儿攻击驱力(咬)的主要方式。

有意思的是,在攻击驱力与快乐之间的关系仍然没有什么明确的解释。我们可以毫不犹豫地承认在性驱力和快乐之间的关系。性驱力的满足并不是简单的紧张的释放,而是一种快乐的获得。即使快乐可以被内疚、羞愧或厌恶所干扰或取代,也不能改变我们对性与快乐之间关系的看法。但攻击驱力的满足(或换句话说,攻击性紧张的释放)也能带来快乐吗?Freud认为不能。但是,后来的绝大多数的精神分析学家却认为它能带来快乐。

顺便提醒一下,精神分析文献中常常发生对“libido”或“libido的”一词的误用。必须认识到,libido不仅是指性驱力的能量,也指攻击驱力的能量,在攻击驱力这一概念还没有成形以前,文献中也应这样使用这一词。那时,“libido的”就相当于“本能的”意思。由于最初应用时的作用如此之巨大,以至于甚至在今天还要告诚人们“libido”应同时包括性的和攻击的能量。

\signatureA



\chapter{心理结构(之一)}\label{3}

现在,让我们问自己“通过对精神分析理论的讨论,目前我们所获得的心理图像是什么样子?”

回答这个问题,我们要以两个基本的、较好建立的假设为出发点,这两个假设是关于心理功能的假设,基本上是描述性的。一个是心理决定论,另一个是心理活动主要是潜意识的这一观点。

我们认为,这两个假设是我们进一步讨论精神分析的指导思想。正像我们刚才所说的那样,它们基本上是描述性的。但在下一个专题——内驱力中,我们立即就会发现我们所接触的概念基本上是动力性的。如,我们讨论心理能量。即驱动有机体去活动以至获得满足为止的能量;随着婴儿的成长。由遗传决定的本能,从一个阶段到另一个阶段的变化模式;在此模式的大范围内可能产生的全体差异;在发育过程中,libido及攻击能量从一个对象转到另一个对象;固着点的建立;心理能量返回那些固着点的现象,即我们所谓的本能的退行。

事实上,精神分析理论的特点就是向我们描绘了一幅运动着的、动态的心理图像,而不是静态的、死气沉沉的心理图像。它试图向我们解释心理的成长与功能,以及心理各部分的运作、交互作用及冲突。即使是把心理划分成不同的部分,也是在动力的、功能的基础上而划分的,在本章及下面两章讨论Freud所说的心理结构的要素时,我们将会明白这一点。

为了建立心理结构的模型,Freud把心理结构的模型描绘得像一台复式光学仪器,就像光学显微镜或电子显微镜一样,由很多光学元件按顺序组装而成。同样,心理结构也被认为是由一些心理成分连续地排列而成,它的一端是知觉系统,另一端是运动系统,中间是各种各样的记忆与联想系统。

即使是在极早期的心理图式中,也可以看出那些划分是功能性的。结构的部分对感觉刺激起反应,一旦启动另一与之密切相关的部分,就会产生意识现象,其他部分储存记忆痕迹并重视它们,等等。从一个系统到下一个系统流动着某种心理兴奋,依次供给系统以能量,就像神经冲动从反射弧的一个要素传到另一个一样。

另外,Freud在他早期的图示中,建议划分三种心理系统,他在记忆与联想系统插入了一个系统。但即使在他第一次对这三个系统进行讨论中,它们的出现仍是极其重要且非常新奇的。后来他把关于这方面的观点更精细化、系统化。现总结如下。

心理的内容与操作可以根据它们是否是有意识的来划分。这样,就区分了三个系统,潜意识系统、前意识系统与意识系统。

乍一看,似乎Freud关下心理结构的第一个理论尽可能地不带有动态性与功能性,而是在纯粹的、静态的、质的基础上,即以“它是否有意识”来划分心理的各部分的。但事实上并非如此,第二个理论基本上也是功能性的。下面就讨论这个问题。

Freud指出,光有意识的属性不足以区分心理内容与过程。理由是,有两类内容与过程都不是有意识的,却可以由动态的、功能的标准来区分。第一组(前意识)在任何主要的方式上与现在正在发生的有意识的任何内容没什么区别。它的要素可简单地靠努力注意就可成为有意识的,相反,一旦撤消注意,就不再是有意识的了。第二组心理过程与内容(潜意识)是没有意识的,但与第一组所不同的是,经任何注意的努力它们也不会成为有意识的。它们在此时被内心的一种力所排斥而不能接近意识。

第二组的一个简单的例子是在第\ref{1}章所描述的在催眠状态下所给的一个命令,嗯嗯嗯让被催眠者在催眠中“醒”来以后服从在催眠时给予的命令。在此案例中,被催眠者在催眠过程中所发生的一切都由于催眠者让他忘记的命令所排斥。或更精确地说,催眠中事件的记忆被服从于忘记这些事件的命令的心理活动排斥在意识之外。

在此功能性的基础上,Freud划分了潜意识系统与前意识系统。那些被意识系统所排斥的内容与过程叫做潜意识系统;那些可由注意变成有意识的叫做前意识系统;那些有意识的过程与内容叫做意识系统。

由于意识与前意识系统在功能上的接近,常把它们连在一起作为意识-前意识系统,与潜意识系统相对。意识与前意识密切的关系很容易理解。某一在目前属于意识系统的思想,一会儿就可能是前意识系统的一部分。当不再注意它时,它就不再是有意识的了。相反,在某一时刻属于前意识系统的思想、愿望等会变成有意识的,成为意识系统的部分。

远在Freud以前,心理学家就很熟悉并研究过意识过程。因而,Freud所做的主要的、新的贡献与发现是关于潜意识系统。事实上,在精神分析产生后的许多年中,它一直被叫做“深层心理学”,即潜意识的心理学。它主要是有关被某一心理力排斥于意识之外的心理内容与过程的心理学。在精神分析产生的过程中,我们刚才所总结的有关心理结构的理论起了很大的作用。

随着Freud对潜意识系统理解的加深,他认识到潜意识的内容并不像他所预期的那样有规律。已证实,除了主动地被排斥于意识之外这标准可运用于心理的内容与过程之外,还有其他新的标准,而且用新的标准比用旧的标准划分的心理内容与过程组更同质,更有用。为此。1923年Freud就心理系统提出了新的假设。这个理论常被称为结构假设,以区别于前一理论,常被称为拓扑理论或假设。

结构假设代替了前一假设,试图把功能性相关的心理内容与过程组织在一起,并在功能差异的基础上划分成不同的组。Freud在他的新理论中所提出的每一个心理“结构”,事实上是在功能上相互联系的一组心理内容与过程。Freud划分了三个功能上相关的“组”或“结构”,分别称为本我(id)、自我(ego)与超我(superego)。

对于Freud最后一个理论,我们可以初步地、粗略地说,本我包括内驱力的心理代理物,自我由必须处理的个人与其环境关系的那些功能组成,超我包括对我们心灵的道德知觉及我们的理想抱负。

自然,我们假定内驱力是随出生而出现的,但对环境的兴趣或控制,以及道德感或抱负都不是随出生就出现的。所以很明显,自我与超我是在出生以后某一时间才产生的。

Freud假定本我是由出生时所有的心理结构组成,自我与超我则源于本我,但随着成长过程而分化出来,成为独立的功能性实体\footnote{后来有人假定新生儿的心理结构是未分化的结构,从它分化出本我、自我和超我。这种观点要比假定本我是先存在的、是后两个的渊源更有利。}。

这种分化第次的发生与自我的功能有关。我们都知道,在婴儿发展出任何一种道德感以前,婴儿对他的环境就具有兴趣,并能够对环境施予某种程度的控制。事实上,Freud的研究使他认识到,超我的分化并不是在5或6岁才开始的,也可能不是在几年以后才牢固地确立,超我的分化可能要到10或11岁才开始。另一方面,自我的分化开始于生命的头6或8个月,在2岁或3岁才牢固确立\footnote{一些分析学者(如著名的Melanie Klein及其同事)假设,超我在一岁以前就开始作为独立的心理系统而活动。但目前,这观点不被大多数精神分析学家所接受。}。

因在发展时间上存在着这些差异,对我们来说,分别讨论自我与超我的分化将是比较适宜的。从时间的差异性看,我们最好从自我谈起。

在下面讨论自我的分化与发展时,读者要记住一点,即在一本书中,必须依次逐个地介绍和讨论这些发展的各个方面,虽然在现实生活中它们是同时发生的、是相互影响的。为了得到自我发展比较全面的图像,我们必须熟知它的所有情况,不能在一个时间里只介绍一个方面而忽略其他。我们应该同时讨论所有的情况,但这是不可能做到的。因此,读者就应该在阅读某一方面的同时,认真考虑其他所有的方面,除非读者先前就很熟悉下面讨论的材料。这就是说,读者至少要读两次以上。只有反复阅读,读者才能较清楚地理解自我分化与发展的各个万面的密切关系。

我们已经说过,我们称为自我的这组心理功能,主要的或在重要的程度上与处理个体与其环境的关系的功能相似。对成人来说,这概括性的阐述包括许许多多的现象:感望的满足、习惯、社会压力、认知上的好奇、审美或艺术上的兴趣及其他一些现象。这些现象中,某些之间有很大差别,某些只有细微的差别。

然而在儿童时代,尤其是在婴儿早期,没有那么多的理由对环境感兴趣,他们的性情也不是那么变化不定与难以捉摸。幼年时儿童的态度是很简单、很实际的:“给我我要的”或“做我想做的”。换句话说,对儿童而言,环境只是作为一个满足或释放(来源于内驱力并构成本我的)愿望、冲动及心理紧张的对象。如果我们想把我们的观点讲得更全面一些,我们还必须加上相反的方面,即环境也会成为儿童痛苦或不愉快感受的来源,面这些感受却是儿童试图逃避的。

重复地说,婴儿最初之所以对环境发生兴趣,是把环境作为可能获得满足的源泉。心理用于对付环境、利用环境的那些部分逐渐地发展成为我们所谓的自我。也就是说,自我即心理为了获得最大限度的满足或释放本我而应付环境的那一部分。就像我们在第\ref{2}章所说的那样,自我是内驱力的执行者。

自我与本我之间合作得这样好,在我们的日常临床工作中并不常见。相反,我们日常所处理的都是自我与本我之间严重的冲突。它们是神经症的来源。作为临床医生,由于在工作中必须时时处处想到这种冲突,以至于忘记了冲突并不是自我与本我之间惟一的关系,自我与本我之间的冲突,确实不是主要的关系,我们刚才所说的合作关系才是重要的。

我们不知道,在心理发展的哪一阶段,自我与本我之间开始出现冲突,并对心理功能具有重要的意义。似乎很有可能发生在自我已经发生较大程度的分化与组织之后。我们暂不讨论这些冲突,待到讲解自我与本我的发展时再说。

在生命最早的几个月,自我应付环境的活动是什么?对我们成人来讲,这似乎没什么意义,但瞬时的反射证实了它们的重要性。我们可以肯定,尽管它们表面上很不重要,但它们与我们以后生命中所获得的成就比起来,则是一个相当主要的里程碑。

对骨骼肌的控制是一组明显的自我功能,即我们常指的运动控制。各种感知觉的形式也同样地重要,因为它提供了有关环境的基本信息。如果一个人想有效地影响环境,有必要得到一个记忆库。显然,一个人知道过去发生的事情越多越详尽,他经历的“过去”也越多,他将会更好地利用现在。随便提一句,最早的记忆似乎就是那些有关本能满足的记忆。

这些功能之外,婴儿早期必定有某些心理过程与后来生活中我们所谓的情感相对应。这些原始的情感或情感的前体是什么?这是一个很有趣的问题,但至今仍没有令人完全满意的答案。最后,在婴儿早期的某个时间必定进行着所有人类的自我活动;初次发生于冲动与行动之间的犹豫不决,释放的首次延迟。这些都将会逐渐发展成为无边无际的我们称为是思想。的复杂现象。

所有的这些自我功能——运动控制、知觉、记忆、情感、思维——正像我们能够看到的那样,是以原始的、基本的方式开始的,只是随婴儿的成长而逐渐发展。这种渐进的发展是自我功能一般的特征,而与自我功能的渐进发展有关的因素又可分为两组:第一组是生理上的成长,在此是指主要由基因决定的中枢神经系统的生长;第二组是经验,或经验因素。为了方便,我们把第一个因素称为成熟。

我们可以很容易理解成熟的重要性。如直到脊髓中锥体束髓鞘形成之后,婴儿才能对四肢的运动进行有效的控制。同样,双眼视觉的能力必须依赖于适当的眼球调节运动与斑状图像的融合。显然这种成熟因素对自我的发展在速度上及顺序上均有很大的影响。我们从发展心理学家那里学到这方面的知识越多越好。然而,Freud的兴趣在于经验因素对自我发展的影响,虽然他很清楚地意识到基因因素的重要性及机体构造与环境交互作用的复杂性。

Freud(1911)的一个经验是,在形成自我的最早阶段,婴儿与他自己躯体的关系起着十分重要的作用。他指出,我们自己的躯体在我们整个一生的心理生活中占据着非常特殊的位置,而且在婴儿很早的时期就占据了这一特殊的位置。他认为,原因不只一种,比方说,躯体的一部分与婴儿环境中其他任何物体都不同,当儿童触摸或轻咬自己的躯体时,产生了两种感觉而不是一种,即不仅有感觉,同时还有被感觉。这是不存在于其他任何物体的。

此外,更重要的是,婴儿自己躯体的一些部位为本我的满足提供了既方便又时刻都可获得的东西。如,由于成熟及某种程度的经验,婴儿在3-6周时能够把他自己的拇指或食指放进嘴里,使得自己在想吮吸的任何时候都能够满足这个愿望。我们相信,对这么小的婴儿来说,在心理上任何东西都比不上伴随吮吸所带来的口欲满足更为重要。我们可以想象,各种自我的功能(运动控制、记忆、动觉)也十分重要。它使得吸吮拇指的满足以及内驱力本身的对象——拇指与食指——的满足成为可能。我们必须记住,同样的道理,吮吸(口腔)器官在心理上也十分重要,即它们与由吮吸而产生的所有快感、经验都有着密切的关系。这样,躯体上被吮吸的及吮吸的部位,在心理上均其有或将变成为十分重要。它们的心理代表物在属于自我的那些心理内容中也就占据着重要的位置。

我们应补充说一下,由于躯体的一些部位常常产生痛苦或不舒服的感觉,而且这些痛苦感觉往往是不可逃避的,因而它们在心理上很是重要。譬如,一个婴儿饿了,喂了他之后他才不感到饿。我们不可能把饥饿的感觉“拿走”,就像把手从引起疼痛的刺激物上拿走那样。

通过这些因素以及其他我们较模糊的因素,我们可以看出,婴儿的躯体——首先是各个不同的部分,而后是整体——在自我之中占据了特别重要的位置。躯体的心理代表物,即与它有关的记忆与观念,由于内驱力能量的投注,可能是婴儿早期阶段自我发展的最重要的部分。Freud(1923)说,自我最初就是整个躯体的自我,这表达了上面的事实。

在自我的发展中,另一个十分重要而又依赖于经验的过程被称为认同(identity),通常是与人、与环境中的物体认同。靠“认同”,我们经过行动或加工,使思想或行为的一个或几个方面变得像某事或某人。Freud指出,倾向于变成环境中的对象,是个体与对象的关系的重要部分,尤其是在极早期的生活中。

早在生命第一年的中期,人们就能在婴儿的行为中看到这种倾向。如,靠模仿成人的微笑,婴儿学会了微笑;模仿人们对他说话面学会了说话。在依赖于模仿的这段时间里,成人总在不断地带领正在成长的孩子玩大量的模仿游戏。只要提到“藏猫猫”及“拍拍手”,人们就可想到这类游戏在孩童时期的这段时间占据了多大的部分。

认同重要性的另一个例子是婴儿对语言的获得,但这要晚一些时间。观察表明,儿童对运动言语的获得依赖于对环境中的物体的模仿,也就是大量地应用认同这一心理倾向。只有当中枢神经系统已足够成熟,儿童才能学会说话,而且语言的获得从总体上来讲并不只是简单的模仿过程。不过,儿童常常是在模仿中学说话的,至少在最初是这样的。也就是说,他们重复成人对他们说话的声音,在模仿成人的过程中学会了说话,而这往往是作为游戏的一部分。在观察每个孩子以环境里的成人及大孩子一样的“腔调”来谈话这一事实,是会受到很大启发的。如果孩子听力正常的话,音调、音高、发音与风格都学得准确无误。由于模仿得如此的准确,使得人们怀疑我们一般听说的“音盲”(即分辨不出音调的相对差异)可能真的是天生的。我们可以确定,“认同”在获得我们称之谓运动言语的这一特殊的自我功能中起了很大的作用。

言谈举止,体育或智力上的兴趣与爱好,本能驱力的随意表现的倾向(如发脾气)、或对这些表现进行控制的倾向,及其他一些自我功能也是如此,即认同起了重要的作用。这些方面中,有一些很明显,有一些则较细微、较不明显。但把它们放在一起,就能很明显地看出,经验对自我形成的作用确实非常重要。

当然,与环境中高度投注的人或事认同的倾向,并不只限于儿童的早期。如,青少年的穿着打扮、言谈举止往往与娱乐界的偶像或体育明星相似,也就是说,此时他已与那位偶像或明星认同了。这种认同在青少年中可以是短暂的,但也并不总是如此。教育工作者很了解这一点的重要性,即青少年的老师不仅要教得好,而且必须以身作则,成为学生的“好榜样”,使他的学生都变成他、与他认同。纵然,我们并不总是同意教育工作者关于好榜样的模式,我们仍然应该同意,学生倾向于与他们的老师认同的事实。

这种倾向确实在人的整个一生中都持续地存在,但在后来的生活中,大部分是以潜意识作为表现的。换句话说,成人常常并不知道,他在思想及行为的某些方面正变得像另一个人,即模仿另一个人,或已变得像他人。在生命的早期,想变得与他人相像的愿望是很可能接近于意识的,虽然并不总是这样。如一个小男孩毫不隐瞒地想学他的父亲,后来又想学“超人”等。长大之后,他会像他的新老板那样留起胡子,但他并没有意识到有要与新老板认同的愿望。他这样做的愿望是潜意识的,尽管他留胡子的决定已表达了这种感望。

到现在为止,我们讨论的是个体与环境中libido高度投注的人或事认同的倾向。从我们的讨论中已有证据表明,这种倾向是很正常的,虽然它们似乎在早期的精神生活中比晚期要更显著、更重要。

有趣的是,还有一种与攻击能量高度投注的对象进行认同的倾向。如果被争论的对象或人是强有力的。那么就会发生一种叫做“与攻击者认同”的认同,在这种情况下、个体满足于(至少在幻想上)分享了对手所具有的力量与荣耀。不论是儿童还是成人,他在一个被崇拜的对象上投注了大量的libido,与此对象认同,就会为他提供同样的满足,就像我们先前所举的与父母、教师、受欢迎的偶像及老板的认同那样。

我们所掌握的最有力的证据表明,认同是继发于某些幻想。在这些幻想之中,人们希望拥有所钦佩的对象的权力与财富。毫无疑问,在许多事例中,有强有力的动机在起作用。可以简单地说,与一个对象的认同是libido投注的结果。因为,远在类似于羡慕这类动机或取代被羡慕的对象这类幻想还没有被确信为在起作用之前,在婴儿期的某个时间就已能够观察到这种倾向。认同是否也可能是攻击能量高度投注的结果,至今仍是个有待回答的问题。

Freud强调在认同过程中起重要作用的另一个因素,叫做对象的丧失(object loss)。它可以指对象实际的死亡、对象已死亡的幻想、与对象长时间的或永久的分离、或者这种分离的幻想。他发现,在上述任何一种情况下,都存在着与丧失的对象认同的倾向。后来的临床经验重复地验证了此发现的正确性和重要性。例如,一个人的父亲死后,此人的生活道路改变了,他变成了他父亲的复制品,像他父亲过去所做的那样接替了父亲的企业,一下子好像自已长大了许多。Freud还引述了一个病人自认为犯了罪,而事实上是她死去的父亲做的。第一个例子,我们应称之为是正常的。第二个例子显然是有严重精神疾病的病人。

就像例子所揭示的那样,被高度投注的人的丧失(死亡或分离)对个体自我的发展可能具有关键性的影响。在这类例子中,存在着一个持久的要求,即想模仿或变成已丧失的人。最常被精神分析实践所研究的这一类案例是抑郁。这些人的精神病理学里,潜意识地与丧失的对象认同起了很重要的作用。

这样,我们可以看出,认同在自我的发展中所起的作用不只一个方面。首先,它是个体与其高度投注的对象之间的关系的一个固有的、内在的部分,尤其是在生命早期。另外,我们已提到过,与所钦佩的以及所憎恨的对象之间的认同,即Anna Freud所说的“与攻击者的认同”。最后,也就是我们最后提到的那个问题,高度投注的对象的丧失导致了与丧失的对象或多或少地认同。不论认同发生的方式是怎样的,其结果是更好或是更坏,它总是使自我变得更为丰富。

现在,我们想讨论与区分自我与本我这一主题密切相关的另一个主题,即我们称之谓初级与次级过程(the primary and the secandary processes)的那些心理结构的功能。

之所以命名为初级过程,是因Freud认为它是心理结构起作用的原始方式。我们认为,本我在整个一生中所起的作用与初级过程相一致,而自我的活动在生命的早期,即当自我的结构还没有成熟,还很像本我时,也是如此。另一方面,次级过程是在生命的第一年中逐渐的、渐进地发展起来的,它是以相对成然的自我的活动为特征的。

“初级过程”与“次级过程”这两个术语在精神分析文献中,只是指两种相互联系而又有区别的现象。“初级过程”可以指儿童在自我还没有成熟时独有的某一类思维,也可以指我们所认为的驱力能量——libido的或攻击的——在本我或不成熟的自我中移动或释放的方式。次级过程可以指成熟的自我所特有的某一类型的思维,也可以指我们所认为的在成熟的自我中产生的心理能量的结合与动员过程。这两种思维在临床上更有其重要的意义,也容易被研究。这两种处理及释放心理能量的方式在我们的理论上很重要,但比较难研究。它们对于我们所有的有关心理能使的假设都是适合的。

讨论初级或次级过程时,先谈谈在心理能量操作中的现象。

对于初级过程,它基本的特征可以用我们以前关于内驱力能量的理论公式来简单地描述。简而言之,与初级过程有关的内驱力投注具有高度的活动性。我们认为,这种投注的活动性可以解释初级过程两个显著的特征:\quan{1}本我以及不成熟的自我具有要求得到立即满足(投注释放)的倾向;\quan{2}纯投注能够从受阻的或不可能达到的最初目标或者释放方法上转移,并能够用相似的、甚至不同的方式来释放。

第一个特征,即要求得到立即满足或投注释放的倾向,在婴儿期与儿童期很是明显,这时自我的功能还没有成熟。此外,后来的生活中,它比我们的虚荣心所愿承认的要常见得多。用精神分析的方法对潜意识心理过程的研究,特别是对我们称之为本我的那些过程的研究表明,立即释放投注的倾向是我们整个一生中本我的特性。

对于第二个特性,即用一种投注释放的方法取代另一种,可以用一些简单地例子很好地解释。婴儿提供给我们这样一个例子。当婴儿得不到乳头或奶瓶时,他就吮拇指。这时,与吮吸冲动或欲望有关的内驱力能量的投注,指向了乳头或奶瓶的心理代表物。可以看出,这种投注是灵活的。当得不到乳头或奶瓶而无法实现释放时,此种投注就会转移到婴儿可获得的自己的拇指上,靠吮吸拇指,实现了投注的释放。

另一个例子是儿童玩泥巴。玩粪便不再是投注释放的一种可行的方式,因那是被禁止的,故而与粪便的心理代表物有关的投注具有灵活性,作为儿童,他可以把投注转移到泥巴上。由玩泥巴来达到投注的释放,以获得相同的满足。同理,我们可以理解到为什么当儿童生妈妈的气时就咬或戏弄小弟弟了,或者,一个人在白天不敢向老板发泄他的愤怒,在晚上。却朝孩子大喊大叫了。

当我们转过来考虑次级过程时,我们会发现。情况与初级过程十分不同。这里,强调要延迟投注能量释放的能力。我们可以说,关键似乎是只有当环境条件非常令人满意时才可能延迟释放。诚然,这有些像把人比做神,毕竟我们是在谈论自我,即人本身。无论如何,延迟释放的能力是次级过程的一个基本特性。

次级过程的另一个基本特性是,比起初级过程来说,投注更紧密地与投注释放的某个特定目标或方法维系在一起。至于第一特性,即延迟满足的能力,在初级与次级过程之间只有量的差异,没有质的差异。

同样的理由,从一种过程到另一种过程的过渡是逐渐进行的。无论在追踪描绘某一个体的成长与发育过程中,还是在企图研究某一个体的心理活动,都不可能在初级与次级过程中间划一条截然分开的界线。常常可以说,某一思维或行为具有这样那样的初级或次级过程的痕迹,但没有人敢说,“这儿是初级过程的结束,那儿是次级过程的开始”。从初级到次级过程的渐进的变化,是我们称之谓自我的那些心理过程分化与成长的一部分。

就像我们以前所说的那祥,初级与次级过程也可以指两种不同类型的思维或思维方式。我们认为,生活中初级过程思维比次级过程思维出现得要早些,后者是作为自我发展的一个部分或方面而逐渐发展起来的。

现在,如果我们想定义或描述这两种思维方式,我们会发现,次级过程的思维要比初级的容易描绘些,因我们对前者较熟悉。它就是我所知道的来自内省的有意识思维,它主要是有声语言,遵循逻辑与句法规则。它是我们普遍认为相对成熟的自我所具有的思维方式。因为我们对它比较熟悉,就不需要具体的、进一步的介绍。

另一方面,初级过程的思维是自我还没有成熟的那些年龄的儿童所特有的思维方式。它与我们所熟悉的有意识思维,即我们称之为次级过程的思维有很大的不同。由于如此的不同,以至于使读者可能会认为,初级过程的思维是一种病理的心理功能,而不是正常的心理功能。因此,强调下面的论点是很重要的,即初级过程的思维对不成熟的自我而言是很正常的,是一种主要的思维方式,而且在某种程度上很正常地存在于成人的生活中。这一点不久我们就会看到。

我们接着介绍初级过程的思维。它往往产生陌生而费解的印象。它缺乏任何负性的、有条件的、有其他修饰性的连接词。如果陈述某事,只能从上下文来看出,它是指正性的抑或负性的、或有条件的抑或表示愿望的。这里面,互为相反的事物可能相互取代而出现,相互矛盾的观念可以和平共存。我们似乎很难相信这种思维不是病态的。

在初级过程的思维中,往往用暗指或类似物来表达,也可能用部分的物体、记忆或观念来代表全部,也可能用某一种思想或意象来表达几种不同的思想或意象。事实上,在初级过程的思维中,言辞的表达并不像在次级过程的思维中那样,是专用的。视觉或其他感觉的印象可以代替一个词或一段话或一个章节。还有一个特征,即时间感或对时间的考虑在初级过程思维中是不存在的。没有“以前”或“之后”,“现在”或“后来”,“首先”、“其次”或“最后”之类的东西。在初级过程中,过去、现在与未来都是一样的。

初级过程的思维在一些严重的精神疾病中十分明显,而且由于它在精神生活如此的突出,以至于形成那些精神病人的主要症状。在中毒或脑器质性疾病造成的各种谵妄案例中,在病因不明的严重疾病如精神分裂症、躁郁症案例中都可以见到初级过程的思维。但初级过程的思维本身并不是病态的。在这些案例中,病态表现在缺乏或者未出现次级过程的思维,而不是因为有了初级过程思维的存在。成人的生活中,初级过程的思维占优势或占了全部的活动时,才构成了病态。尽管初级过程思维给我们留下的最初印象十分奇怪,下面的讨论将有助于对它更好地理解,使我们认识到,它比我们所想象的要熟悉得多。

如我们知道,幼儿的智力发展的过程中往往缺乏时间感。要经过几年的时间,儿童才会有时间感。在此之前,儿童只能理解“这里与现在”。由此可以看出,初级过程的思维是儿童早期的一个常见的特征。

同样,用非语言的方式来表达观点的倾向也是儿童早期的特征之一,这是会说话以前的儿童所必须有的思维方式。

至于我们所介绍的混乱的、不合逻辑的、概要性的特征,以及修饰性连接词的应用,基至是反向虚词的应用,在书面言语中要比在口头言语中更为常见。在口头言语中,大量的意义可用上下文、手势、面部表情及声调来表达。而且,说话的方式越口语化,越不正规,句法越简单。一旦离开了上下文,说的话本身越易模糊。如,说“他是一个伟大的人物”这句话,说话人如果用严肃的、开玩笑的或讽刺性的方式来说,就分别指几种不同的含义。如果说话的人是在讽刺,“伟大”一词则表示字典所给出的定义的反义。这种用反义语来表达,乍看似乎是初级过程的思维中最令人难懂的一个特征、但在日常生活的运用中却很常见。由于如此之常见,以至于我们不意识到它的频繁性,除非我们特别注意。

类似的,用整体来代表部分,或用部分来代表整体,或用暗示或类似物来代表都是在诗歌中所追求的思维方式,在玩笑与俚语中也常见。甚至用非言语的方式来表达观点也常常潜人我们的意识生活。虽然我们的艺术修养可能不很高,我们也会用诸如“整个故事用语言难以描述”之类的话来形容些图画。我们将会认识到,这种企图在幽默的动画片、漫画及广告图像中很是常见。

这些例子都表明,初级过程的思维并不像我们起初所假想的那样,与成人有意识的思维不相容。显然,在整个一生中它都存在,并持续地起着虽然是从属的但相当重要的作用。另外,就像我们在下一章将见到的那样,在正常情况下,自我保持着暂时地逆转到儿童所特有的不成熟模式的能力。这常见于不管是否以饮酒来增添情趣的成人在玩游戏、开玩笑之时,也可以在睡觉、做梦或醒时做白日梦时出现。在这些情况中,初级过程的思维比次级过程的思维有明显的短暂的增加。后者就像我们已说过的邢样,在成人的正常生活中占统治地位。

虽然我们到目前为止已概括了初级与次级过程思维的基本要素。但还要谈谈以下几点,以使读者更好地探讨有关这些主题的精神分析文献。

首先,为了更好地界定初级过程的思维。精神分析文献中用了两个术语来描绘这种思维的一些特征。这两个术语分别是“替代”(displacement)与“缩合”(condensation)。

技术上,精神分析的“替代”是指用整体代表部分,用部分代表整体,或用与某一个观念或意象有联系的观念或意象来取代之。Freud假定,这类取代是由于或者依赖于投注,即心理能量释故的转移,使它从某一个思维或观念转到另一个上去。为此,他选择了“替代”一词,被替代的就是投注。另外,此术语显示了初级过程的思维与内驱力能量(也叫初级过程)特有的调节方式之间的密切联系。这种情况下,初级过程的思维所特有的替代的倾向与我们已介绍过的初级过程所特有的投注的移动有关。

“缩合”这一术语,用于指单一的话或意象,甚至是其中的一部分,来表示几种观念或意象。像替代一样,“缩合”这一术语选用来指过程所依赖的能量的替代。Freud假定,当许多心理代表物被一种所代表时,对这些代表物的投注就缩合在一个上了。

另一个初级过程思维的特征是象征性的表达(symbolic representation)。虽然它很像我们已经讨论过的“替代”这一特征,我们还是把它当作独立的、特殊的一个特征来考虑。

Freud对梦及神经症进行研究的早期(1900)发现,梦或症状中有一些成分对不同的病人来说具有相似的涵义,而这涵义与他们日常生活中的意义又不同。尤其奇怪的是,病人自己并不知道其涵义!如,在梦中,一对姐妹常代表乳房,旅游或缺席代表死亡,钱代表粪便,等等。这就好像人们在潜意识里运用一种秘密的语言,无法有意识地去理解它,也像没有语言的字母,Freud称之谓象征。换句话说,在初级过程中,可能用金钱作为与粪便完全一样的象征,用旅行来代表死亡等。这确实是一件很奇怪的事实。因此,这一发现激起了人们强烈的兴趣,同样也激起了强烈的反对。之所以造成这么多人的兴趣或反对,是因为被象征所表达的事物及观念是被禁止的(如性的或“下流的”)。

可用象征来代长的事物并不很多,包括躯体和它的各部分,尤其是性器官、屁股、肛门、尿道与消化道及乳房;最接近的家庭成员,如父母、兄弟姐妹;某些躯体功能与体验,如性交、排尿、排粪、吃、哭、发怒及性兴奋;出生,死亡;以及其他一些。读者会注意到,这些事情都是小孩感兴趣的,换句话说,它们对自我还未成熟的个体是很重要的,此时,初级过程在个体的思维中起主要的作用。

对初级过程及次级过程的讨论就到此为止。现在,我们转向内驱力能量理论的另一个方面,去了解自我如何从本我中分化出来,以及随后的发展。

这一方面叫内驱力能量的中立化(neutralizaltion)。中立化的结果是,迫切地、尽快地需要释放的内驱力能量(如所有的本我投注)变得可被自我所接受,在自我的支配下,按照次级过程来完成它的各种工作与愿望。这样,我们把未中立化的内驱力能量与初级过程相联系。虽然我们并不肯定中立化与次级过程的建立与操作之间的准确关系。

首先,中立化是渐进的,而不是突然的变迁。其次,被自我功能所接受的能量对自我是必需的。没有能量,自我就失去功能。

我们说中立化是渐进的,指的是这种转化是在相当长的一段时间内一点一点地发生的。就像与自我发展有关的其他变化那样,它是一种逐渐发生的变化,而且与自我的成长是平行的。中立化对自我的成长来说是极重要的。

如果我们试图给中立化的能量下一个定义的话,一个最简单、最易理解的定义是,它是一种从原始的、性的或攻击的能量明显地转变而来的能量。这一内驱力能量去性(desexualization)的概念,是Freud(1905)认为性驱力是惟一的本能驱力时所提出的。近些年来\footnote{指二十世纪四五十年代——译者注},作为伴随物,“去攻击性”(desaggressivizalion)一词已被引进。但是我们认为,不论是对性能量还是攻击能量上,似乎简单地称中立化更可取。

中立化一词是指一种个体活动。这时,个体停止了通过投注释放而获得内驱力的满足,转而开始为自我服务;对任何事情(哪怕是接近原始的、本能的形式的事情中),都明显地或完全地脱离满足或投注释放的需要。下面的例子可以帮助我们更容易理解之。

儿童最初的讲话尝试,就像不成熟的自我的其他活动那样,对各种内驱力的投注提供了一种释放途径。很难或不可能全面、准确地知道幼小儿童在讲话时所释放的内驱力能量是什么,但我们可以肯定其间有下列几种:情绪的表达;与或人或年长兄弟姐妹的认同;与成人玩游戏及赢得成人的注意。然而,这时的语言运用很大程度上不在于这种满足,即使缺乏最初所伴随的那些直接的满足之时,也可用语言来交流思想。那种最初的原始的内驱力能量已被中立化,并为自我服务了。

我们要强调,像谈话这一类的活动与内驱力满足之间的关系,在生命的早期是正常的。没有内驱力能量的贡献,语言的获得将会受到严重的妨碍。我们可以在临床的实例中,如在孤独的缄默症儿童、精神病儿童中看到这一事实。这些儿童与成人没有令人满意的关系,他们的言语有在治疗的过程中,即重新开始有了或第一次有了满意的关系时才能恢复或首次产生。另一方面,如果涉及到的内驱力能量未能充分地中立化,或者在后来的生活中未进行中立化,讲话或供它利用的中立的能量被重新本能化了,那么神经症患者的冲突就会干扰自我的功能。不管是否有内部冲突,个体都具有这种功能。从儿童的口吃(不适当的中立化)及癔病失语(重新本能化)可以见到这种本能化的结果。附带地,我们可以加上,重新本能化(去中立化)是退行现象的一种表现。关于退行,在第\ref{2}章我们已提过,在第\ref{4}章还要讨论。

中立化的能量是在自我的支配下执行一些自我功能的。这符合自我的操作是自主的这一事实,因为这些操作一般地不受内驱力的变迁(至少在幼儿期之后)以及由内驱力所引起的内在的心理冲突所干扰。从这种意义上来讲,这些操作是自主的。但是,它们的自主性是相对的,不是绝对的,而且如我们上面所说的那样,在一些病态的情境中,它们支配的能量可能是重新本能化的,它们的活动受来自内驱力的欲望所影响,或者受这些欲望的冲突的影响,甚至任这些欲望所摆布。

\signatureB



\chapter{心理结构(之二)}\label{4}

在第\ref{3}章里我们已经对本我与自我不同的形成过程及机能等诸方面进行了探讨。我们也谈及了由“自我”所组合起来的各种基本心理机能、如动作控制、感知觉、记忆、情感和思维等。同时也讨论了影响自我发展的两个因素,即我们称之为成熟以及环境或体验这两个广泛的范畴。我们还重点讨论了后一个范畴,并指出了婴儿自身环境对自我发展的重要性。此外,我们也讨论了通过环境中其他人的认同,对儿童自我发展的重大影响。我们还转而讨论了什么是心理结构各个部分的机能模式;什么是初级的和次级的心理过程,特别是思维过程。最后我们又讨论了来自内驱力的心理能量的中立化在自我的构成及其功能中所起的作用。

在这一章我们将讨论两个密切相关的重要问题,第一个题目是关于自我如何了解及控制周围环境;第二个题目是关于自我如何控制和约束本我由于内驱力的作用而产生的愿望与冲突。这第二个题目十分复杂而又十分重要。因为,自我一方面作为本我与环境的中介,要同外部世界作斗争;另一方面又必须对本我进行控制,即同内部世界作斗争。

现在开始讨论第一个题目,即自我对于环境的控制。正如前面我们讨论过的,自我至少有三个与此有关的重要功能:一是感知觉,它向自我通报环境的直观情况;二是记忆能力、比较能力和按照次级过程进行的思维能力,它们提供了比原始感受印象更为高级的关于环境的知识;三是动作控制和技巧,它们通过活动方式使个体承担起改变身体环境的任务。正像人所共知的,这些功能是互相关联而不是相互分离的,例如:在获得感觉印象时,手的触摸,动作技巧是重要的,正像立体视觉的与用手触摸时所获得的感觉一样。然而,除了这些不同的和相互关联的自我功能以外,我们还认为自我在协调与环境的关系上起着特别重要的作用,称之谓现实检验(reality testing)。

所谓现实检验,一方面可以理解为自我区分由外界所产生的刺激或知觉的能力,另一方面也可以理解为自我区分由本我产生的冲动与欲望的能力。假如自我能成功的完成任务,我们就认为个体具有良好的或适当的现实感,否则,就认为他的现实感是不良的或缺陷的。

那么现实感是怎样发展起来的呢?我们认为,就像其他自我功能一样,它是婴儿在一个相当长时期的发育成熟过程中逐渐地发展起来的。我们设想,婴儿在出生的最初几周,根本无法区分来自躯体的及本能驱力的刺激与来自环境的刺激。这种能力的发展是渐进的,部分是由于神经系统和感觉器官的成熟,部分是由于经验的积累。

Freud(1911)把注意力放到挫折的研究上来,并作为后来的研究重点之一。他认为,现实检验的发展在出生后的头几个月至关重要,并指出:婴儿多次体验到的得到满足的重要刺激(如来自乳房和乳汁)有时也会丧失。一旦婴儿发现了这些,即使这个特殊刺激的投注再高(此例中婴儿处于饥饿状态),也能使他认识现实状况。

Freud认为,这种挫折的经验在婴儿期以各种方式不可避免地一再重复出现,是婴儿现实感发展的最重要的因素。也就是,通过那些经验,婴儿学会了去认识在世界上时隐时现的某些事物。对这些可能不在眼前的事物却又老是向往它的存在。这是婴儿认识事物(如妈妈的乳房)的起点,而这些事物来自“身外”而非来自“自身”。相反地,有些刺激,不管婴儿多么希望它不存在,但却又老是存在。这些刺激来自“自身”而非“身外”。这是婴儿认识这一类事物(如胃痛)的起点。

区分某些事物是“自身”和“非自身”的能力,部分来自现实检验的一般功能,部分来自自我界限的确立。实际上,使用自身界限(self-boundaries)可能要比用自我界限(ego boundaries)更确切一些,只是在文献上后者的使用变得固定起来。

在这种经验的影响之下,成长中的儿童的自我逐渐形成了检验现实的能力。我们知道,在儿童期,这种能力是随时间的迁移而不断变化的。例如,孩子们有把所做的游戏想象成为现实的倾向。然而,我们必须认识到,即使是正常成年人的生活也是随愿望、恐惧、希望和记忆而变化的,很少有人认为世间是安宁和稳定的。我们绝大多数人对世界的看法或多或少都受着我们自己内在精神生活的影响。

举一个简单的例子,当我们看到各自的国家由热爱和平的关系转入到战争状态时,那些以往被认为可敬的人就可能变成可鄙和凶恶的人。是什么造成我们对他们的品质评价的转变呢?我们必须承认,是我们自身内部的心理过程的改变成了决定性因素。无疑,这些心理过程是十分复杂的,但它的最重要的作用之一,是唤起了我们对敌人的仇恨并要求去消灭敌人。但它的另一种作用是使人产生罪恶感,以致害怕惩罚和报复。由于这种起伏的感情,使得原先在我们眼里的可尊敬的邻居,变成了可鄙、可恶的人。

自我用作现实检验能力的这种不全面不可靠的特性,影响到我们前面讨论到的流行的偏见。显然,这也可以来自迷信和巫术的信念,以及来自一般的宗教信念。不过,正常成年人的日常生活中,现实检验的能力可达到相当程度的成功,只有严重的精神疾病才会使它缺损或减弱。精神病患者的现实枪验的能力所受到的障碍比起看来正常的神经症患者要严重得多。一个精神病人坚信他的妄想或幻觉是真实的,其实他的妄想或幻觉只是他自己曾经历过的恐惧和愿望。

对于各种严重精神疾病来说,现实检验的障碍具有如此的规律性,因此可以作为诊断精神疾病的标准。这种障碍的严重后果提示我们:在自我作为本我代言人的正常角色中,现实检验的能力至关重要。完整的现实感能使自我为本我的利益而有效地作用于环境。自我与本我联合起来为了得到满足的机会而力图去利用环境。这对自我来说是很宝贵的。

现在让我们来看看,作为本我和环境之间起中介作用的自我角色的其他方面。我们发现,自我能延缓、控制或阻止本我能量的释放,而不是促进或加速它的释放。

当我们理解了自我和本我之间的关系后,我们就知道自我对于本我能量释放的控制能力在有效地利用环境方面是头等重要的事情。假如一个人稍稍耐心地等待一下,他就能经常地避免某些由于不满足而引起的不愉快,并能增加获得满足以后的愉快。举一个简单的例子,一个1.5岁的孩子想要撒尿,如果他的自我能坚持到他在进到厕所之后再撒尿,那么他就能避免大人的责骂,同时会得到赞扬而感到十分愉快。此外,我们发现,某些内驱力能量释放的延缓是次级过程和次级过程思维发展的重要部分,这对于自我利用环境极为可贵。

因此,我们可以将自我发展的每一过程,都理解为自我在某种程度上延缓了本我能量的释放,以及在某种程度上对本我进行控制。Anna Freud(1954)将本我与自我的关系比作现代国家的个人与政府机构之间的关系。她指出,在一个复杂的社会中,公民要想让公仆们有效地工作并代表他们的利益,就必须让公仆承担许多任务。政府机构为个体公民创造了许多好处,为他们带来欢乐和享受,但同时也会发现有某些弊病。因为政府机构在满足个体特殊需要上往往收效太慢。同时,每个人都有自认为最好的理想,而这样的理想此时又得不到。在类似的情况下,自我可能强行延搁本我的内驱力,借口环境的要求而进行遏制,并以中立化的方式为自身的利益而适当地使用内驱力的某些能量。

到此为止,我们已对自我与本我之间的关系有所了解,认识到自我与环境之间的关系,还远不足以使自我有能力与本我的本能愿望长期对峙。毕竟我们已经反复讨论过自我与现实的关系。自我最初是为本我服务的,因此我们会期望,在本我欲望与环境现实之间存在重大冲突之时,自我应与本我更紧密地结合起来。然而,我们所发现的情况与我们的期望竟有天渊之别。在某些情况下,自我可能自己起来反对本我,甚至直接阻止它的内驱力能量的释放。自我对本我的否定,只有在自我功能得到某种程度的发展,并对它加以组织以后才明显地表现出来。当然,自我的发展和组织在生命之初就已经开始了。我们知道,很小的孩子常有攻击其兄弟姐妹的行为,但随着时间的推移和外界环境对此行为的压制,自我也就否定了本我的这一欲望,似乎攻击行为已不存在。这只不过是由于自我占了主导地位并压抑了本我欲望,使其得不到显现的机会而已。

由此我们看到,自我最初虽然只是本我的执行者,并在生命的各个方面表现出来,但自我同时也从很早就开始逐渐学会了去控制本我,并同本我对抗,甚至发生公开的冲突。这样,自我就从对本我的顺从和助长,变成了与本我对抗,甚至控制了本我。

我们在对自我作用进行回顾之时,也同时提出了一些需要回答的问题。我们如何解释最初作为本我的一部分,并服务于本我的自我是怎样扩展为对本我进行控制的呢;当自我成功地控制本我冲动时,它是使用了什么样的手段呢?\label{label1}

对于第一个问题的回答,部分地取决于婴儿与其所处环境关系的特性;部分地取决于人类心理中的某些特性。这些特性有些是新的,而有些是我们前已述及并十分熟悉的。一般说,它们都与自我的功能有关。

就环境而言,我们知道,婴儿的环境十分特殊,有明显的生物特性,决不单单只是环境而已。如果没有这些环境——先是母亲,然后是双亲——婴儿就不能生存。因此,我们也就不难理解婴儿对双亲的生理需求为什么与其对心理的需求旗鼓相当。儿童的欢乐大多来源于其父母。我们知道,由于多种因素,婴儿的母亲可能成为婴儿环境中这样一个重要的客体,在母亲的要求和婴儿的本我欲望之间发生冲突时,自我则站在母亲一边而反对婴儿的本我欲望。例如,若母亲要制止孩子破坏性冲动的表达,诸如撕毁书页等,这时自我就顺从母亲的意旨而抑制本我的破坏性冲动。

对于这部分问题的回答是很容易理解的,无需作更多的技术性讨论。接下去我们要更多地讨论一下上述已讨论过的第一个问题中的其余部分。

首先,我们要再次强调的是,自我的形成及自我的功能所使用的能量整个地或大部分地来自本我:因此,我们可以假定,本我是心理能量的一个巨大储存库,自我的存在是消耗减弱本我的内驱力能量,以建立起自我来。因此,自我的发展势必导致本我的某种程度的削弱。甚至我们可以看到,在我们当中的许多人身上似乎已经完全没有了本我的痕迹,因为所有的心理能量都参与了自我的形成。从这里可见,自我的生长就像寄生虫一样在消耗着本我的能量。而当自我发展到足够强大时,就变成了控制本我的力量,而不再是完全地为本我服务了。虽然我们前面说的似乎不能充分地解释这种结果,但事实只能就是如此。

我们在这里所提及的自我形成及其功能的重要性,本我的心理能量的削弱和自我的增强过程,都具有重要的作用。

我们所看到的自我发展的主要形成过程是内驱力能量的中立化。正如在第\ref{1}章曾详细描述过的这种去性过程,明显地造成了libido和攻击的能量的减弱以及有利于自我的能量的增强。

在自我的发展过程中,心理能量从本我向自我的转移的另一个重要作用是认同过程。认同作用在第\ref{3}章也已讨论过了,读者将会记得它使个体变得像是外部世界的某个客体(人或物)。而这个客体对个体来说,在心理学上具有重要意义,也是内驱力能量高度投注之所在。

正如我们所看到的,所谓“变得像”就使自我产生了变化。这种变化的后果之一是由先前附着于外部客体身上的内驱力能量全部地或部分地附着于或存在于自我之中的客体的副本。于是乎,某些本我能量就附着于自我,壮大了自我所支配的能量,同时削弱了本我的能量,改变了自我和本我的力量对比。

值得我们注意的另一方面是,本我需求被削弱,因而自我控制力得以增强,可以成为一种幻想的满足过程。显然,人们可以在白日梦的幻想或睡梦的幻想中使本我的某些欲望得以满足。但实际上,这只不过是有关的本我冲动的不完全的实现及其能量的不完全的疏泄而已。正如一个口渴的人可以梦见喝到了水,不再感到口渴而继续睡眠一样。

梦幻在我们的精神生活中起着重要的作用,在此我们不想多加论述。我们只想指出,幻想的效应之一是本我冲动虽仍受自我的监督与控制,但相对来说是较容易获得满足的。因而,在幻想中使本我部分地接受自我的控制是可以实现的。这样,我们就应该清楚,实际上幻想在我们正常的精神生活中是经常发生的。

现在我们来讨论最后一个心理特征,即自我对本我的调节在什么范围内起作用的问题。在整个情况中,这种特征是决定性的,真正取决于自我在某种程度上和某些时刻对本我的调控。人具有在某种情况下产生焦虑的倾向,如果没有Freud所阐述的快乐原则作基础就很难理解焦虑的精神分析理论。下面我们就对此进行研讨。

所谓快乐原则,简言之即人们都具有获取快乐和避免痛苦的心理倾向。Freud曾用德文名词unlust表达并译成英文pain(痛苦),因此我们也常常称之为快乐-痛苦原则。然而pain(痛苦)一词具有双重含义,即包括了身体上的痛感觉,同时又是快乐的反义词。为了避免意义上的含混,我们选用含义较明确的“不快乐(unpleasure)”一词而不用“痛苦(pain)”。

Freud还认为,人在初始阶段,其获得快乐的倾向极端迫切且直截了当。由于年龄的增长,人们不再那么露骨地追求个人快乐了。

到现在为止,快乐原则的概念听起来很像我们曾在第\ref{3}章讨论过的初级过程的概念。按照快乐原则,在生命的早期阶段,人有一种追求愉快避免痛苦的倾向,而且刻不容缓。按照初级过程,内驱力能量的投注必须尽可能地释放。这样我们就可进一步假定,在生命初始阶段的心理功能里,这个过程是占优势的。此外,与快乐原则相关联,Freud还断言,随着年龄的增长,个体延缓获得快乐、避免痛苦的能力逐渐增强;还有,与初级过程相关联,他又系统地阐述了次级过程发展的观念,认为随着年龄的增长,个体也延缓了投注的释放这一过程。

因此,Freud早期的快乐原则概念与后来的初级过程观点是一致的。它们的真正区别只不过是在术语的使用上,即把快乐原则看作是主观的术语。而把初级过程看作是客观的术语,也就是说,所谓“愉快”或“不愉快”这些词说的是主观现象,在这种情况下表述的是情感;而“投注的释放”或“内驱力的释放”说的是能量分配或释放的客观现象。我们应该注意到,根据我们的理论,情感或情绪是一种自我的表现,因而更多地取决于本我的过程。

Freud在实际上很清楚地意识到,快乐原则的公式和他称之为初级过程的本我功能的公式之间有很大的相似性。事实上,他曾试图把这两个观念统一起来,但终未成功,所以我们还需把这两个观念分别加以讨论。

统一这两个观念的企图是在非常简单的假设基础上作出的,即如果在心理结构中投注不能够得到释放、那时就会感受到不愉快(痛苦);而当这种投注得到充分疏泄时,就会产生愉快的感受。最初,Freud(1911)曾假定精神紧张增加就会造成不愉快,相反就会产生愉快。如果这个假设是正确的,快乐原则和初级过程就应该是在相同假设下的两个不同的术语。

对该观点的论述如下:所谓快乐原则,说的是在很小的孩子身上有一种通过不推延欲望的满足而获得愉快的倾向;所谓初级过程,则说的是在很小的孩子身上有一种只要有投注就释放,即不推延内驱力能量释放的倾向。按照Freud最初的假设,由于满足而产生的快乐也就意味着内驱力能量的释放。如果这一假设是实在的,那么前面的两种说法所表达的是同一件事情。也就是说,快乐原则和初级过程只不过是同样假设的两种不同的说法。

Freud(1924)曾断言,虽然在大多数情况下,快感的产生与心理能量的释放相伴随,不愉快则是这种能量积蓄的结果,有时却不完全如此。他甚至认为,相反的情形也是存在的,因为有时性紧张的增强也能体验到快感。

Freud最后的结论是,在内驱力能量的释放与积蓄之间,以及愉快与不愉快的感受之间,这两方面的关系并不简单、且难以预测。因此,他曾提出一种称之为能量释放增加比率和节律的假设。然而,往后虽然有人力图去发展关于快感和内驱力能量积蓄与释放之间关系的满意的假设,都没有得到广泛的承认。

我们到现在还不能用心理能量去满意地解释有关的“快乐原则”。因此,我们必须把握早期的关于“快乐”与“不快乐”的主观体验的解说,即人的内心或人的精神生活总是倾向于追求获取“快乐”而避免“不快乐”的。

读者将会回忆起,我们之所以介绍快乐原则,是要把我们的注意力转向一个新题目,即焦虑这个主题。在焦虑的精神分析理论中,快乐原则的重要性,会通过我们的讨论变得更为明显。

Freud最初关于焦虑的理论认为,libido的过度控制和不适当的释放都将引起焦虑。精神上libido的异常积蓄,是由于外来障碍而影响它适当的释放造成的,还是由于内在潜意识冲动或性满足的抑制造成的,对于焦虑理论来说都不那么重要。由以上情况造成未释放的libido的积蓄可能转换为焦虑。这一理论既不能解释这一转换是怎样发生的,也不能说明由什么因素决定转换发生的准确时间。根据这一理论,“焦虑”这个术语指的是病理性的惧怕。这种惧怕,在现象学上与对外部危险的正常惧怕有关,但在起因上则截然不同。对外部危险的惧怕是一种习得性反应,即基于经验的反应。而焦虑则是libido的转换,即内驱力能量的一种病理性显现。

直到1926年,有关焦虑的精神分析理论的状况大概如此。同年Freud在美国发表了一篇名为《焦虑问题》的专著,在英国则定名为《抑制、症状与焦虑》。在这篇文章里Freud指出,焦虑是神经症的主要症状,运用结构性假设,Freud又对焦虑提出了新的理论。下面我们将对此进行概括性介绍。

认识抑制、症状和焦虑三者之间的关系极为重要。这是Freud关于焦虑的第二理论。这在我们经常引用的Freud名为《超越快乐原则》和《自我与本我》这两本早期著作中已有所阐述,对此我们在本书的第\ref{2}、\ref{3}章也曾作过介绍。这两本专著阐述了对焦虑的基本观点,也包含了有别于现代精神分析理论的基本概念。这些概念涉及到了内驱力和结构假设的双重理论。这使人们得以寻求一种比过去更一致和更方便的方法去观察心理现象,同时了解其内部复杂的相互关系。有关的新理论也大大地促进了精神分析的临床应用。自我分析的发展,以及从这些新理论提出以后而形成起来的精神分析自我心理学的发展,就是一个突出的例子。

Freud还在他的几篇论文中分别阐明了这些新理论应如何有效地应用于临床的问题。《抑制、症状与焦虑》一书是这一临床应用最成功的范例。书中,Freud成功地促进了焦虑理论的临床应用。这一理论是建立在由结构假设而产生的领悟的基础上的。

为了对这一新理论有更好的理解,我们首先要认识到的是,Freud认为焦虑是有生物遗传基础的。也就是说。他相信人的器官先天就具有我们称之为焦虑的、利用心理和生理表现来进行反应的能力。他还认为,在人类,也如低级动物那样,这种能力对个体具有确定的生存价值,至少在人类的“自然”状态下是如此。假若一个人没有双亲的保护就不能战胜任何危险的事物,那么,很快他就得毁灭。

因而,在其焦虑理论中,Freud既没有说明焦虑的性质,也没有说明焦虑的起因,而只是阐述了焦虑在人的精神生活中的重要地位。正如我们所了解的,Freud在《抑制、症状与焦虑》一书中所提出的观念部分也包括了他早期的理论,同时又部分地大大超出了这些观念。

Freud改变了自已早期的重要理论。例如,他放弃了libido释放的障碍转变为焦虑的观点。他把研究方向转向临床,并运用两例儿童期恐怖症的详细描述与研讨来证明他新见解的有效性。

在新理论中,Freud提出了焦虑的外在表现和他称之为“创伤情境”或“危险情境”之间的关系。首先,他将之定义为一种情境。在这种情境里,人的心理被一种过分强烈,以至于既不能控制又不能释放的刺激流所冲击。这时,焦虑就自然地产生。

由于控制外来刺激和有效地释放这种刺激都是自我功能的一部分,因此,在生命的早期,自我仍很软弱且未成熟之时,创伤性刺激就常常发生。Freud认为,婴儿出生时所经受的体验,是创伤情境的原型。那时,婴儿就受到体外和内脏感觉的刺激,并产生Freud称之为焦虑表现的反应。

Freud把出生过程看作是伴有焦虑的创伤情境,可见于后来在心理学上有显著意义的创伤情境的原型,并与他的新理论相符合。Otto Rank(1924)试图在临床上比Freud更广泛地应用这理论,甚至超出了Freud本人所提出的观念和做法。他认为,凡是神经症者的病源都可以追溯到出生时的创伤,并且可以通过重新呈现这种创伤让患者意识到它,病就可以得到治愈。兰克的观点在当时引起了精神分析学界的轰动。但很快就平息了下来。

Freud在他的论文中,对新生婴儿发生的心理创伤给予了很大的注意。下面是他所阐述的这种创伤情境的例子,即婴儿对母亲的依赖,不仅是为了满足大部分的身体需要,而且也是为了本能上的满足,至少在出生后的几个月内,婴儿的体验主要是与身体需要的满足相联系的。例如,给婴儿喂奶,不仅需要吃饱,而且还要体验到与口腔刺激相联系的本能快感以及双亲爱抚、怀抱和温暖等所体验到的快感。起初婴儿还没有能力由自己来实现这些本能的快感,这时只有靠母亲才能实现这些快感。假若没有了母亲,那么,婴儿那些只有通过母亲才能得到满足的本能需要往后就会发展成为一种创伤情境。因为,婴儿的自我还没有充分发展到能够通过控制内驱力欲望以推延满足,以避免刺激的冲击而崩溃的程度。由于既不能控制,又不能适当地宣泄这种刺激,焦虑就会随之发生。

以上所列事例都是具有代表性的。由刺激的冲击所增强的原始而自发的焦虑属于内源性焦虑。它来自内驱力的作用,或更精确地说是来自本我的作用。据此,我们所谈及的自发型焦虑有时就被视为“本我焦虑”。然而,这个名称至今已很少延用了,因为这会引起误解,以为本我是焦虑的发源地。事实上,Freud在其结构假设的理论中,认为自我是所育情绪的发源地,对任何情绪的体验都是自我的功能。按照这个理论,自我必然也是焦虑情绪的发源地。那种认为本我是自发引起的焦虑的发源地的观点,容易使人产生误解,使得自我很难以不同的形式而存在。实际上,正如我们前面所述,在小小的婴儿身上只能有自我的雏形,或者说自我与本我相差无几,很难区分开来。不过,无论如何,对早期儿童来说,自我在一定程度上还是可以区分的。它也确实是焦虑的发源地。

Freud还认为,心理结构对刺激进行反应的倾向和能力,即产生焦虑的倾向和能力贯穿整个人生。换言之,创伤情境可以发生在任何年龄。这种情境,在生命的早期阶段由于自我尚未成熟而会经常发生,随着自我的发展,对来自内部或外部刺激的控制和宣泄的能力也就越来越强。读者应当记住,只有当刺激不能被适当地控制或疏导时,这种情境才会变为创伤性的,并且随之而产生焦虑。

假如Freud有关出生过程是后来创伤原型的假定是正确的话,那么,出生时的体验主要是一种外来刺激所引起的创伤情境。另外的情况下,由内驱力所产生的刺激则来自内部。如同母亲没能满足婴儿的需要,本我就会喧嚷起来,并要求母亲给予满足。

就我们所知。本我欲望未实现而产生创伤情境在生命早期最常见,也最重要。Freud认为,这些情境发生在生命的以后阶段就表现为现实焦虑性神经症(见第\ref{8}章)。那些病人所患的焦虑,事实上是由于性驱力能量因外部的干扰没能得到适当的释放所造成的。

然而,由于现实神经症十分罕见,Freud的这个观点是缺乏实践意义的。这一基本观点的另一方面的应用对临床实践可能更有意义一些,即成年人的所谓创伤神经症。例如战争神经症和弹震神经症,确是由外部刺激而产生焦虑的结果。Freud提出了这观点并得到了很多作者的证实。事实上,Freud(1926)认为:创伤神经症并不是由单一因素造成,而是有较深层次的人格参与的。

Freud关于创伤情境和在此情境中焦虑自发产生的观点,可以看作是他的新的焦虑理论的第一部分。虽然焦虑理论有别于他早期有关焦虑产生原因的理论,它们还是有密切关系的。Freud早期的观点认为,焦虑是libido的转化。而他后来的观点则变为焦虑可以是,也可以不是来自内驱力的刺激。

我们将Freud新理论的第一部分总结如下:
\begin{enumerate}
    \item 当人的心理受到的刺激太强以至于不能控制和宣泄时,焦虑就会自发产生。
    \item 刺激可能来自于体内或体外,但更多地来自本我,即内驱力。
    \item 当焦虑按照这个模式自发产生的时候,这种情境称为创伤情境。
    \item 婴儿的出生被认为是这种创伤情境的原型。
    \item 自发焦虑的发生是婴儿的特点,因为在婴儿阶段,自我还未发展成熟;如果焦虑发生在成年人身上,我们称之为现实焦虑神经症。
\end{enumerate}

Freud新理论的第二部分,是儿童在生长过程中学习应付创伤情境的来临,并在创伤到来之前以焦虑的形式作出反应,Freud把这种超前焦虑称为信号焦虑(signal anxiety)。它是由危险情境或对危险即将来临的感受而产生的,是自我的功能,并用于启动自我控制力量,以应付或避免创伤性情境的发生。

对“危险情境”意义的阐述,Freud又以离开母亲的婴儿为例。他认为,婴儿会以母亲的存在作为满足的对象,以达到其个人需要的满足,否则创伤或焦虑就会自发地产生。Freud指出:婴儿的自我发展到一定阶段,将会认识到母亲的离去和在母亲离去之后所自发产生旳焦虑的极端不愉快体验这两界之间的关系。换言之,自我能意识到,如果母亲就在跟前,焦虑就不会发生;假如母亲离去,焦虑就会发生。其结果是,自我把与母亲的分离当作“危险情境”,即当母亲不在时,来自本我的为满足需要的迫切欲望就变成了创伤情境。

对于这样的危险情境,儿童将怎样反应呢?由于每个人都有过儿童时期的体验,因而对此并不陌生。儿童表现出各种不快以阻止母亲离开或呼唤母亲回来。然而,Freud对于婴儿的内在心理活动,比起试图去改变外部环境的各种自我活动更感兴趣。他指出,在危险情境之中,自我以自身产生焦虑的形式进行反应。由于这时自我所产生的焦虑是作为危险信号的,因而他称之为信号焦虑。

在继续讨论之前,让我们先来看看,自我怎样主动地产生焦虑,是作为一种信号,抑或是为了其他目的?对这个问题的回答取决于我们对自我的所有相关的功能的回忆情况如何。我们相信,在危险情境中,自我的某些功能,如感知觉、记忆、某种思维过程等,与对危险的认知有关;而自我的其他部分或者其他自我功能对危险所进行的反应被称为焦虑。诚然,我们可以从临床经验中推测,对危险的知觉也许会产生对创伤情境的幻想,而这种幻想就是信号焦虑的起因。这个推测是否正确暂不提,我们可以说,某些自我功能承担着对危险进行认知的责任,而另外的功能则用焦虑来对危险进行反应。

让我们继续用Freud的观点讨论。当自我认识到危险情境,并发出信号焦虑来进行反应时,将会发生什么。我们以此时有快乐原则的参与为出发点。信号焦虑意味着不愉快,而且越是焦虑越是不愉快:可以说,焦虑的强烈程度与自我对危险的严重程度和/或危险的紧迫性成正比。因此,我们可以预期,在任何情况下,危险情境会引起相应的焦虑和不愉快,而这种不愉快又会自发地转化为动作。这就是Freud所说的万能的快乐原则。快乐原则的作用是赋予自我必要的力量,以阻止可能会强化危险情境的本我冲动。以一个离开母亲的婴儿为例,这些冲动可能表现为渴求得到母亲的照护和抚爱。

Freud依次罗列了可能发生在儿童生活中的典型危险情境。首先是儿童与能给其带来满足的重要人物的分离。这在精神分析作品中,被描述为“客体的丧失”或“所爱的客体的丧失”。因为,当第一次觉察到面临危险时的儿童,年龄还太小,还不能产生我们称之谓爱的那种复杂的感情。第二个典型的危险情境,是儿童所处环境中能满足其需要的人所具有的爱的丧失。换言之,即使爱他的人就在身边,孩子也会害怕这种爱的丧失,我们用“客体的爱的丧失”来描述这种情境。第三,两种性别之间典型的危险情境是有区别的。对男孩子来说是惧怕阴茎被割去,即在精神分析文献中所说的“阉割情结”;对女孩子来说则是惧怕外阴的损伤。最后一个危险情境,是所谓负疚感或者说被超我否定与惩罚。

以上典型危险情境的第一种情况,发生在孩子从出生到1.5岁之间,是自我发展最早期阶段的特征。然后是第二种危险情境的出现。第三种典型危险情境则要到2.5-3岁的孩子身上才会发生。最后一种情境只发生在5-6岁以后,这时超我已经形成。对神经症患者来说,所有这几种危险情境潜意识地存在于他们的整个生命过程之中,但每种危险情境的强度和重要性则因人而异。显然,让病人了解潜意识中哪一种恐惧为主,在临床工作中具有重大的实践意义。

Freud断言,焦虑是精神疾病的主要问题,这一观点今天已被人们广为接受。但我们偶尔发现也不尽如此,在《抑制、症状与焦虑》一书出版以前,从理论和临床实践上看,精神分析认为神经症是libido的迁移,特别是libido的固着。在那时,正如我们早就说过的,焦虑被认为是libido不适当地释放而发生的迁移。当然,在理论研讨中,我们的重点应放在libido上,因而临床医生应当设法消除libido的固着,以确保libido的合理释放。这并不意味着只有消除libido的固着才是最重要的,别的并不重要。而是因为,我们现在无论在理论及临床上,既从自我又以本我两个方面,进行考察,而不是只从本我单方面着眼。

在现代精神分析文献中,特别强调了焦虑对精神疾病的重要性。由于焦虑能够使自我对来自本能的欲望和冲动进行检查和制止,从而避免了危险。因此,焦虑功能并不都是病理性的。相反地,它在人的精神生活和成长过程中是必需的。人若没有焦虑,就不可能接受任何形式的教育,就会受到来自本我冲动的支配而力图去满足它们。除非这些努力造成了被焦虑所困扰的创伤情境。

有关信号焦虑的另一点是,它不像那些伴随创伤情境的焦虑那样强烈,换句话说,自我所发生的信号很少会造成人的痛苦和产生创伤情境。信号焦虑具有减缓焦虑的作用。

现在我们对焦虑新理论的第二部分加以概括:
\begin{enumerate}
    \item 在自我发展的过程中,当危险情境出现时(即受到了创伤情境的威胁),需要借助于产生焦虑作为危险的预告。
    \item 由于快乐原则的作用,信号焦虑使得自我能够去检查或制止危险情境中的本我冲动。
    \item 从儿童的早期起,危险情境就或多或少地在潜意识里持续存在,并延续终生。
    \item 信号焦虑对焦虑有减缓作用。它在人的正常发育过程中起着重要作用。信号焦虑是精神神经症特征的焦虑形式。
\end{enumerate}

到此为止,我们已经完成了在第\pageref{label1}页所提的两个问题中第一个问题的回答,即虽然自我开始时是作为本我的部分并为其服务,随着时间的推移,自我最终成了本我的驾驭者。下面我们将回答在第\pageref{label1}页中所提出的第二个问题,即自我怎样成功地实现对本我冲动的控制。

我们从对焦虑的讨论中可以了解到,自我之所以同本我冲动的出现相对抗,是由于预感到这种冲动的出现将会引起危险情境的来临。然后,自我产生焦虑作为危险的信号,以这种方式取得快乐原则的帮助,从而能成功地对抗危险冲动的显现。用精神分析的术语把这种对抗过程称为“防御”或自我的防御作用。那么,我们就可以这样来提出问题:自我用以对抗本我的防御措施是什么?

对这个问题的回答十分一般化,也非常简单,即自我可以使用任何一种为其目的服务的手段。自我的任何态度,任何知觉、注意的转移,强化另一个比危险冲动较为安全的本我冲动,抵消强烈的危险驱力能觉的意图,认同作用以及幻想等,都可作为一种防御方式,单独地或联合地运用。总之,“自我”能够运用正常的自我形成过程及自我功能,一次又一次地达到防御的目的。

然而,除以上那些自我防御措施外,自我也采取我们已经讨论过并已熟悉的过程,其中某些过程是自我最初用以对抗本我的防御手段,Anna Freud(1936)定名为防御机制(defense mechanisms)。我们下一步将探讨这些自我防御机制。

我们对防御机制的解释可能是非常不全面并可能要遭到非议的,因为在精神分析学者中对于什么是防御机制,什么不是防御机制的问题还存在着争议。由于自我对本我冲动的控制的解释已经相当圆满,因而我们所要做的事情是尽力给这些机制下定义,并进行研讨、承认它们在心理功能中的重要性。

\section*{压抑(repression)}
在精神分析文献中,我们早已认识到的和已经深入讨论过的防御机制之一叫做“压抑(repression)”。压抑存在于自我的活动中,它阻止不利的本我冲动或者这种本我冲动所派生的记忆、情绪、需求或随心所欲的幻想等的出现。所有这些,就像在个体的意识生活中根本不存在似的。所谓压抑的记忆,是一种来自发生了压抑的个体主观意念中的遗忘。诚然,我们可以说,除了压抑,我们确实不知道是否还有任何其他形式的遗忘。

压抑一直作用于精神活动。至少在一个相当长的时间内,自我与本我之间在压抑的领域互相对抗着。我们相信,一方面被压抑的东西不断地由某些要获得满足的内驱力能量的投注而得以补充;另一方面自我又不断借助于精神能量的行使而维持着压抑状态。我们称这种能量为抗投注(anticathexis/countercathexis),因为它具有对抗内驱力能量投注的功能:正是由于这种内驱力能量,使被压抑的东西不断得以补充。

在投注和抗投注之间的平衡并不是一成不变的。这是对抗力量之间的平衡,而这种平衡随时都可能被打破。只要自我的抗投注力量大于被压抑的东西的投注作用,后者就会被压抑。如果抗投注的力量变弱,被压抑的东西就会进入意识并见诸行动。这就是说,压抑作用将要失败。假如内驱力的投注增强而自我的抗投注作用没有增强,那么压抑也会不起作用。

对以上的各种可能性加以说明还是有价值的。由自我而产生出来的抗投注作用可能会从几个方面被削弱。似乎在许多中毒和发热的情况下都能发生,酒精中毒就是大家非常熟悉的例子。此时,一个人可能会在其外部行为上或言语中表现出性的或攻击的倾向。在清醒状态下,自己知道原本没有的事,在喝醉的状态下就会信以为真。抗投注作用的相对削弱,常常发生在睡眠状态。正如我们在第\ref{7}章里所看到的,人们把压抑的愿望和记忆以睡梦的形式在意识中表现出来。而这愿望的实现,在觉醒时是根本不可能的。

相反地,我们有理由相信。青春期本我能量增加,使在生命中已固着多年的压抑,部分地或全部地被冲破。此外,我们也观察到,满足的缺乏,可增加本我冲动的强度,就像一个饥饿的人会吃那些平时感到恶心的东西一样。因此,严重性剥夺的人,比起那些没有长期或严重性剥夺的人更容易陷于压抑的失败。诱惑,则是由于本我冲动强度的增加而削弱了压抑的另一个事实。

必须指出,如果压抑被减弱以至于将要失败或在某种程度上的失败,并不意味着在自我与本我冲动之间冲突必然结束,而是在此以后,冲动变得更直接、更随便地进入意识,并在自我的帮助下达到满足。这个结果是可能的。例如,由儿童向成人过渡之时,必须使许多性的压抑全部地或部分地受到解除,以使成年人性的调节机能趋于正常。然而,另一种结果也是常可见到的,即只要本我冲动一进入意识并企图达到满足,作为一个新的危险,自我便进行反应,并一再发出焦虑的信号,用这种方式动员新的、强有力的防御措施,以对抗有害的和危险的冲动。如果自我的企图是成功的,不论是用压抑还是别的方法,一个适当的防御就会重新被建立起来,进而为维持自我,又需要抗投注的能量。

自我与存在于压抑中的本我两者间平衡的变换是可能的,能够使欲望被完全地压抑,即造成欲望的消失,能量投注的废止,或至少是完全投注到其他精神内容中去。实践中我们知道,根本没有所谓完全理想的压抑。事实上,在临床工作中我们曾处理过这样的病例,由于压抑的不成功而产生精神症状(详见第\ref{8}章)。就我们所知,只有很少的病例,在他们身上被压抑的东西能继续以内驱力能量的形式进行投注,而这种内驱力能量必然要被抗投注所抗衡。

关于压抑机制,还有两点需要弄清楚。第一,压抑的整个过程都是潜意识的,而不仅仅是压抑的内容。担任压抑的自我的活动也完全是在潜意识状态下进行的,本人并不晓得什么是遗忘了的东西,什么是被压抑的东西。人们所能知晓的只有一个结果。有些意识的活动与某些压抑相类似。在精神分析文献中常使用“压制(suppression)”这个词来描述这一活动。这是说,人们常常有意识地去忘掉某事。在压制和压抑之间有中介机制但没有截然界限。而当我们使用“压抑”这个词的时候,就意味着排斥在意识之外,而在潜意识中发生了持续的抗投注作用。

第二,被压抑的东西并非不能进入意识界。同样重要的是,要认识到,被压抑的东西变成了在功能上完全与自我分离,并成为本我的一部分。

关于以上叙述还需做某些解释。在对压抑的讨论中,我们已经认识到自我作为一方与本我作为另一方之间的对抗和冲突。不一定说压抑使本我冲动变成本我的一部分。在这两者关系中,我们必须认识到,记忆、幻想和情绪同本我冲动有着内在的联系。在压抑发生以前,在本我冲动中就包含了自我的许多成分。首先,在压抑出现以前,自我功能既服务于这种特殊的本我冲动,也服务于其他的冲动,使得本我冲动与自我功能之间形成一个和谐的整体,而不是相互冲突的两部分。当压抑发生时,整个被压抑的东西从自我中抹掉而加入到本我之中。不难理解,如果一个人把这些东西存在内心里,会有损于自我的完善。现在我们认识到,每发生一次压抑,实际上就削弱了自我的强度,因而也就减少了它的效能。压抑削弱了自我的效能或称强度,是因为每一次压抑都需要自我进一步消耗有限的能量储备,以维持抗投注作用的需要。

\section*{反向形成(reaction formation)}
我们要讨论的第二个防御机制称为“反向形成(reaction formation)”。这是一种相互矛盾的态度,例如在潜意识中的恨表现为爱,于是本来是恨却为爱所取代,残酷为温存所取代,执拗为依从所取代,不修边幅为清洁所取代,而原有的态度则仍然存在于潜意识之中。

虽然我们非常习惯于上述的反向形式,人们还总是喜欢抛弃那些社会所不认可的行为,而代之以他们的父母或老师乐于接受的行为,当然也可能用相反的反向行为方式,用恨取代爱,以执拗取代依从等。反向形式的精确性质是由什么决定的呢?每一个特殊的案例都可以回答这个问题,即“自我惧怕什么样的危险?对于什么样的焦虑信号作出反应?”假如自我惧怕憎恨的冲动,或者更准确地说,假如自我惧怕与憎恨有关的冲动,作为反向形式的防御机制将检查这种冲动,并通过强化爱慕的态度而阻止这种冲动。假如惧怕爱,那么,反过来就产生了恨。

例如,一个人可能对人或动物产生一种非常温柔慈爱的表现,以检查潜意识中残暴的冲动,并使潜意识中这些冲动维持下去。这种情况也可能发生在精神疾病或精神分析治疗过程之中。病人对治疗家的有意识的愤怒来自于自我潜意识的需要,即自我为了防止自已对治疗家的爱慕之情和幻想而产生了恨。认识到这种防御机制之后,每当我们现察到那种不真实的、过分的态度之时,我们就会想到事情的反面。我们会想到和平主义者或者反对活体解剖者的潜意识里存在有残酷和仇恨的幻想,这些对于他们的自我来说,是十分危险的。

我们确信,就像前面所说的压抑一样,反向形成也是在潜意识中发生的,即使不是全部,至少绝大多数的自我防御机制也是如此。认识到存在于我们的有意识的精神生活中反向形成机制的类别是有好处的。这些在潜意识里发生的反向形成,至少跟那些有意识的阿谀奉承者、伪君子、“好主人”有相似之处。那些专门吹捧人的伪君子可能会说:“尽管我的真实感情是厌恶他,但我假装作喜欢他。”我们一定不要把这两种相似的情况相混淆。这种有意的过程只是表明暂时的调适,而反向形成则与压抑一样,在个体的自我与本我之间永远是交替进行的。

我们在对下一个防御机制进行讨论之前,还要阐明一下自我活动的复杂性及其内部关系,同时谈谈企图简单地讨论自我防御机制的困难所在。

一个两岁的孩子,他的母亲又生了一个弟弟或妹妹。他不愿意别人从母亲那里夺走了他所需要的母爱和注意。这个孩子的所言所行表示出他的敌对情绪,甚至会对弟(妹)造成危险。但不久,他发现他的这种敌意事实上并不受母亲欢迎。由于害怕失去母亲的爱,他就采用防御措施来抵御这种敌意的冲动。其中采用的就是压抑。这时敌意冲动被排斥于自我之外,而加入到本我之中,并通过持续的抗投注作用而阻止它进入意识。

此外,儿童对于弟(妹)的敌意的消失和对于弟(妹)的某种程度的爱的表达是常常可见到的。这种爱或憎的程度也不尽相同。这也是自我防御特别是反向形成作用的结果。自我常常采用两种机制来抵御由本我冲动所产生的敌意。它不仅采用压抑而且也采用反向形成。

临床经验告诉我们,防御机制很少单一使用,往往是两个或两个以上的综合使用。但在综合使用过程中,总是有一个或两个是最重要的。

在我们的简短的事例中,可以了解到孩子之所以压抑其敌意情绪,是因为母亲常对他说:“如果你欺侮小弟弟(或小妹妹),我就不喜欢你了。”他也回答说:“我不恨小弟弟(或小妹妹)。我也无需害怕你不喜欢我。”“我不恨弟(妹)”一语提示有压抑的情绪。事实上,孩子与其母亲的谈话并不意味着他们之间的相互理解。即使所讲的话本身并不全面,他们所讲的话必然与真正发生的事情相关联。我们刚才提到的那些话不只是压抑,反向形成也是孩子所采用的防御机制的一部分。通过反向形成。孩子才有效地说出了“我不恨他,我爱他”的话。“我爱他”这话从何而来?这是孩子动用了内部的防御机制而发生了情感的转移,即把原来憎恨的对象转而违心地说成是爱的对象。事实上,母亲们不但说:“你不要恨小弟弟(或小妹妹)”,而且表白得非常清楚的是“你必须爱小弟弟(或小妹妹)”。所以,孩子的“爱弟(妹)”是害怕失去母爱的一种合乎逻辑的违心的表达。然而,分析的经验告诉我们,一个两岁大的孩子“爱弟(妹)”通常是以一种十分特殊而有意义的形式来表达的。他常常模仿母亲的行为和态度对付弟(妹),换句语说,就是潜意识地向母亲认同。

由此,使得我们得到一个意想不到的结论,即认同的过程可能是反向形成的一部分,或必要的序幕。我们还会想到,防御机制可能不止两种类型,即一种是基本的不能再分小的类型,另一种是可以再分小的类型。这是一个有待回答的问题。Anna Freud(1936)在她的名为《自我及其防御机制》这一经典著作中提出,压抑是防御机制的基础,其他的所有机制若不是去强化压抑,就是在压抑失败以后而起作用的。Anna Freud通过对防御机制的一般特性和发展基础的研究证实了这一点,即由最原始的防御机制开始,发展到防御机制的先驱形式,再一步步地发展到最后即较高阶段的防御。

回过来再说,关于压抑是一种防御机制,其他所有的机制都不过对其起辅助作用的看法,我们还难以作出最后的结论。困难在于我们不能把握住它的特点并对其过程加以描述。我们只知道压抑的结果是对某些事情的遗忘,这就是它不能进入意识。诚然,其他的每一种防御机制都是为了阻止某些事物进入意识,然而是否就可以认为压抑是所有防御机制的专有名词?我们这样说恐怕还为时过早。

\section*{隔离(isolation)}
让我们继续阐述防御机制的有关条目。隔离(isolation)一词在精神分析文献中表述了两种不尽相同的防御机制,虽然在病人身上两者都具有强迫性神经症的症状。Freud最初把它定名为情感隔离,但把它叫做情感压抑或情绪压抑更为合适。在这种情况下,一种与欲望相联系的幻想,或者过去经历的关键性记忆已经进入意识之中,但那些痛苦的情绪则不能转化为意识。这种病人通常会尽力去抑制各种过于强烈的感情。情感压抑的过程常常表现为痛苦的或恐惧的情绪被排斥于意识之外,而以快乐原则的利益取而代之。对这些不幸的人而言,这种情况太过分了,以至于他们始终都没有察觉这种情绪的存在。

隔离的另一个含义十分少见。Freud在他的《焦虑问题》(1926)一文中是作为强迫症的病理心理学来进行讨论的。这是一个潜意识的过程。被隔离的思想与先前的那些思想隔离开,继之以短暂的心理空白期。对于这些在心理上剥夺了任何联系的被隔离的思想,自我竭力限制它再度进入意识。这种思想,是作为“不可接触”的思想来对待的。

\section*{抵消(undoing)}
如前所述,隔离的两种形式都以强迫症状为特点。另一种与这种症状有关的防御机制是“抵消作用”(undoing)。这是一种动作,其目的是抵消伤害,而这种伤害可能来自个体潜意识想象中的欲望,具有性或敌意的内容。例如,一个敌视兄弟姐妹或双亲的孩子会产生焦虑并会有以下的行为。他先打他愤怒的目标,继而又吻了他们。他的第二个行为是为了抵消第一个行为。不难发现,这类行为在较大儿童和成年人中也很常见。

儿童及成人的许多仪式行为可以在这个基础上加以解释,也就是有意识地或无意识地去抵消自我以为是危险的那些本我冲动。有时仪式的含义是明显的,就像上面所举的那样,尽管病人本身并没有意识到。但抵消机制的含义通常不容易被觉察,这是因为在进入意识以前它已经被歪曲或伪装。我们能够讲的是,整个抵消观念是神奇的,而且可以推测它是起源于儿童的早期,此时神奇的观念在精神生活中占有主导地位。

\section*{否认(denial)}
另一个重要的防御机制,Anna Freud (1936)称之谓“否认”(denial)。它是借助于随心所欲的幻想或行为,对不愉快的或不希望的外界现实加以否定。例如,一个小男孩非常害怕他的父亲,他可能会说他是世界上最强壮的人,并获得了世界重量级拳击冠军,他还会系上象征冠军的皮带在房子里走来走去。这就是小孩子否认他比父亲矮小、软弱的例子。这样,那些真实的事实就被否认了,而由能够满足孩子在身体上超过他父亲的欲望所产生的幻想和行为所取代。

“否认”这个词也被人用来描述内部体验的态度。上述例子描述了一个小男孩对内心害怕的不认同。此处。使用“否认”这个词是很不情愿的。因为它与压制(suppression)的概念非常相似。“否认”的原意是对外界某些印象的阻抑。如果不把这些外界的印象从意识中否认掉,就会引起注意,并造成痛苦。

在对防御问题的讨论中,“否认”这个词的使用有时会造成某些混乱。这是由于某些事物之被否认,就像阻止某事进入意识的防御机制一样自然。在每一次防御进行时,本我说“是”,自我说“不是”。也就是说,自我常对本我所肯定的东西加以否定。由此推断,Anna Freud所描述的,通过幻想而起作用的否认可发生于每一个防御机制。我们可以说,否认的防御机制与我们一生中的游戏及白日梦紧密相连。所有的娱乐活动都是否认的防御机制在起作用,而使我们避开日常生活中的挫折和担心。

\section*{投射(projection)}
下面我们要讨论的另一个防御机制称为“投射”(projection)。它使得个体将个人的欲望或冲动投射于他人、他事或外界的其他事物。这在精神病人身上屡见不鲜。病人将他自己的冲动投射于外,误认为他自己的身体受到旁人的伤害,这样的病人通常被认为患有偏执性精神病。

应当注意,尽管投射作用在偏执性精神病患者中起着重要的作用,在没有精神障碍的正常人的心理上也会发生作用。精神分析的经验表明,很多的人总是将自己的愿望或冲动借助于投射机制而外射于他人。这些愿望或冲动则是此人不愿接受的,或者是试图在潜意识中一笔勾销的。人们会潜意识地说:“不是我有这样坏的或危险的欲望,是他。”可是,对这些个体的分析表明,在战争年代我们把犯罪和邪恶的东西投射于敌人,我们对于陌生人、外国人或与自己肤色不同的人的偏见,以及许多迷信和宗教观念,都是我们自己的欲望与冲动部分地或全部地潜意识投射的结果。

从以上事例我们知道。投射作为防御机制在成年人生活中被非常广泛地应用着。这时,他的外部现实知觉就被严重歪曲,或者换句话说,自我的现实检验能力将受到明显的损害。只有在自我已经没有能力正确地去检验现实时,投射才会被广泛地应用。

投射作为一种防御机制,在生命的早期阶段起着很重大的作用。非常小的孩子常常很自然地投射于他人、他事、动物或非动物的客体。他自己所经历的感情和反应,他并不打算违背的真实的思想情感与欲望,就通过投射来实现其目的。这在小孩子是常见的事。当孩子由于欺骗行为而受到责骂时,就常辩解说,那不是他干的,而是别的孩子干的。事实上正是他所为。我们见到有些成年人原谅了一些孩子的有意识的欺骗,但儿童心理学家认为,年幼的孩子确信自己的投射是真实的,并期望他的父母或保姆也按他的愿望去做。

有关投射机制可能的起源问题,有人认为,把人的某些想法或愿望从其精神生活中分离出去,并投射到外界中去,是从婴儿期就存在的生理经验的净化。由精神分析指导的观察研究告诉我们,小孩子常常把他的排泄物认作是自己身体的一部分。当投射作为一种防御机制来运用时,使用的人就会在潜意识中试图抛弃自身不愿要的精神内容,尽管它们是肠子的内容物。

\section*{反向自身(turning against the self)}
另一个防御机制称为本能冲动转向反对自身,或简称为“反向自身”(turning against the self)。我们可以用儿童期的行为来表明其意义。孩子在盛怒时,例如与另一个人相对抗却又不敢去表达自已的反对态度之时,可能转向于拍打、撞击或伤害自己。这种机制尽管看起来似乎陌生,但在正常精神生活中却有很大的作用。它常常伴随着对于冲动指向的客体的潜意识认同。在上述事例中,孩子在自已打自己时会说:“我就是他,我要打他。”

\section*{认同(identity)}
读者还可能回忆起我们在第\ref{3}章已经用相当长的篇幅对认同进行了讨论。我们认为,它是自我发展的最重要因素。为达到防御的目的,我们常常使用认同,但现在人们不大同意把它归结为防御机制,更确切地说,只是在防御途径方面它有经常被使用的一般倾向。自我可以运用任何东西去减少或回避来自本能内驱力需求的危险。

当认同作用被自我作为防御途径而使用时,它往往在吞咽或吃食等身体动作以后被潜意识地模式化。这意味着,采用防御机制的人在潜意识中想象着成为他正在吃被其认同的人,或正在被其认同的人所吃。这样的想象正好与投射机制相反。这里,潜意识中的模式似乎是一种行为的净化。

“内射”(introjection)和“合并”(incorporation)也是在文献中发现的,用以表示与消化相联系的潜意识幻想。一些学者试图去区分这几个术语,但在一般使用上,它们也是“认同”的同义词。

\section*{退行(regression)}
我们应该提到,在自我防御作用中占有重要位置的另一个防御机制,名为“退行”(regression)。虽然它像认同一样,是一种防御机制,但严格来说,它有更广泛的意义。我们可以假定,退行倾向是本能生活的基本特点,正如我们已经在第\ref{2}章谈到的那样。作为一种防御,本能的退行可以避免面临严重冲突时而发生的焦虑。比如,在本能发育的性器期,某些欲望可能要部分地或完全地放弃掉,个体必须退回到或退行到较早的口期或肛期。这样,就避免了性器期欲望持续存在所造成的焦虑。本能的退行往往只是部分的而不是全部的,只要足以使自我在与本我的冲突中占上风而形成心理环境的相对稳定就行。这样,性器前期的欲望就能或多或少地代替了性器期的欲望。其他的例子表明,退化达不到防御的目的。代替相对稳定的平衡,却产生了性器前期的冲突。这种本能的退行虽已发生,却没能产生有利于自我的缓解内部心理冲突的情况,通常见于临床上严重精神障碍的病人。

这种伴随有自我功能和自我发展的不同程度退化的本能生活的退行,可见于许多病例。当这样一个自我功能的退化在个人的精神生活中持续下去并进入成人之时,它就被视为病理性的了。

对于防御机制我们已作了罗列:压抑、反向形成、隔离、情感的隔离、抵消、否认、投射、反向自身、认同或内射,以及退行。它们在正常心理发展和机能方面,以及在心理病理状态方面都或多或少地发挥着作用。

\section*{升华(sublimation)}
同以上防御机制有关,又不同于以上心理机制的是“升华”(sublimation)。起初人们认为,心理防御机制是与精神功能障碍相联系的,而升华则与心理防御机制相补充。而今则认为升华是自我正常功能的表现形式。在第\ref{3}章及本章里我们反复提到,自我的正常功能是减少环境的压力,使内驱力达到最大的满足。为了说明升华的概念,试举一例,婴儿玩大便的欲望是内驱力所促使的。在我们的文化中,这是孩子的父母所不允许的,孩子只好玩泥饼面不玩大便,后来以玩橡皮泥或塑料物品代替了泥饼,有些人则由此成为了业余爱好者或专业雕塑家。精神分析的研究表明,这种替代活动达到了玩大便这一儿童本能冲动的满足。因而,这种原始需求的活动已被调整为社会所能接受并加以赞许的行为活动。这种冲动在精神生活中是潜意识的,并使个体致力于塑造、雕塑、泥塑或塑造塑料制品,最终,在这些源于婴儿的冲动及活动中,次级过程占了优势。

升华是一种替代活动。它协调了环境对个体的要求,并在潜意识里满足了个体对不被承认的原始形式的婴儿内驱力的需求。在我们的例子中,玩泥饼、塑造、雕塑等都是玩大便欲望的升华。也可以说,这些都是发生在不同年龄水平上正常的自我功能的表现,以尽可能地满足本我的需求,并且与环境尽量地协调起来。

\signatureC



\chapter{心理结构(之三)}\label{5}
在涉及心理结构的所谓结构假设的最后这一章,我们将讨论个人与其环境中其他人之间的关系,以及超我的发展。我们照例从一生中的早期事件开始,后讨论儿笔期及以后的发展过程。
% 客体关系
Freud最先给我们指出,我们与其他人之间的关系对我们的精神生活以及我们的发展十分之重要,其中最早的当然要数儿童与父母的关系。在大多数情况下,这种关系局限于母亲或母亲的替代者,不久就会扩展到兄弟姐妹、亲密的玩伴及父亲。

Freud指出,儿童早年曾依恋过的那些人在他精神生活中总占有一席之地。不管儿童对他人的依恋是由爱还是由恨,或者由二者联系在一起,有一点是确定无疑的,那就是最终形成的是由最常见的联系在一起。早期依恋的重要性部分地由这些事实来决定,即在早期出现的各种关系影响儿童发展过程,而晚期出现的关系由于出现得晚,所以不能起到同样程度的作用。此外,早期依恋的重要性,部分地是由于儿童的早期有很长一段时期内不能自助。这种情况的延续,儿童只得依靠环境来保护自己和得到满足,并使自己比其他哺乳动物活得更久一些。换句话说,生物因素在决定人际关系的特点上起着重要的作用,因为它造成了我们称之谓延长的产后胎儿化。这是人类发展的特有内容。

在精神分析文献中,“客体(object)”这个词习惯上用于指那些对人们的精神生活有显著影响的环境中的人或物,而不管它们是有生命的或是无生命的。同样,“客体关系(object relations)”这一术语则是指对于这些客体的个体的态度和行为。
% 自恋
如在第\ref{1}章里讨论过的那样,我们假定在生命的最初阶段,婴儿认识不到有客体,他只是在发展的最初几个月里才逐渐把自身与客体区别开来。我们也曾经陈述过,儿童把自身的各个部位,如手指、脚趾和嘴,也看做是重要的客体中的部分。所有这些都是使儿童感到愉快的极为重要的来源。因此,我们假定它们都与libido的投注密切相关。更确切地说,儿童身体这些部位的心理代表物与投注密切相关。我们不再像某些精神分析学家以前所认为的那样,认为libido像激素一样可以转移到身体的某个部位并固着在那里。Freud(1914)把这种libido朝向自身的情况称为自恋(narcissism)。这个名称取自青年Narcissus的希腊神话,在这个故事里,这位青年爱上了他自己。

自恋概念在现今精神分析中的地位,在某种程度上还没有确定,因为早在本能的双重理论被提出之前,这个概念就被Freud所发展,结果在自恋概念中只有性驱力找到了一席之地。无论用本能的双重论还是用结构假设,都未能使自恋产生清晰的线索。我们应该考虑一下,由攻击驱力而增加的朝向自身的能量是否也是自恋的部分?再者,心理结构的哪一部分是由具有自恋性质的驱力能量所投注的?它是自我的固有属性,还是自我的特殊成分,或者是尚未明确的心理结构的其他部分?这些都是尚未得出确切答案的间题。

然而,尽管自恋概念还没有被提到议事日程上,仍可以认为它沿袭了精神分析理论中有用的和必要的假设。这个概念运用于成人时,一般说来至少可以表示三种不同的意义:(1)过分地投注于自身;(2)过少地投注于环境的客体;(3)与这些客体存在病理的不成熟的关系。当这个概念运用到儿童身上时,一般指儿童早期发展的一个正常的阶段或特点。值得一提的是,Freud认为libido大部分保持在自恋中,这意味着libido一辈子都是朝向自身。当然,这里指的是“正常的”、“健康的”自恋。Freud也认为投注在外部世界客体里的libido力量与自恋libido之间的关系,就像变形虫的伪足与它的身体之间的关系一样,这意味着客体的libido由自恋的libido所驱动,如果以后因某种原因使得客体被放弃,则客体的libido还能回归到自身。

% 早期客体关系
现在让我们回到客体关系发展这个话题上来。儿童对他意识到的第一个客体的态度自然是排斥的和自我为中心的。儿童首先只与客体提供的满意程度相联系,即只与客体能满足自身需要的方面有关。除非客体对婴儿来说是在物理上不存在,否则当婴儿通过客体体验到满足时,就对客体投注了。我们假定,同客体持续的关系只能逐渐发展起来。此处我们指的是,即使在要满足的客体不存在的情况下,也有持续的客体投注。对于那些不能从中得到愉快和满足的客体,只要它持续存在于环境之中,婴儿也会逐渐地对它发展起兴趣来。基于这种看法,我们就可以用更主观的术语来表示同样的含义。例如,早期的婴儿只有在饥饿或其他的原因需要母亲时,才对她感兴趣,但到了婴儿晚期和童年早期,母亲在儿童心里的重要性则是持续性的了,而不再是一过性的。

到底与客体的持续关系如何得以发展,我们还不很清楚。换句话说,我们对这种关系的发展阶段,特别是早期阶段还不清楚。有一件事值得一提,最早的客体是那些被我们称为部分客体(partial object)的东西。例如,在母亲成为儿童单个的客体是经过了一段很长的时期、在这段时期内,母亲的乳房或奶瓶、手、面孔等,在儿童的心目中都是分离的客体,或者是同一物理客体的不同的方面。这些物理客体对儿童来说,可能是单独的而不是统一的。例如,对儿童来说,母亲的笑容与她的怒容可能是两个不同的客体;母亲慈爱的声音可能是一个不同于她责骂声音的另一个客体,儿童在一段时期以后才能把这两种面容或两种声音知觉为单一的客体。
% #矛盾情绪
我们认为,持续的客体关系大约在出生后第一年的后期开始发展,这种早期客体关系的特征之一便是我们称之为“矛盾情绪”(ambivalence)的高级层次。这意味着,根据情况,爱的感情可以被同样强烈的感情(如恨)所取代。我们可能会提出疑问,是否对客体破坏性的幻想和欲望会被认为是敌意的(这种客体出现在出生后的第一年的后期)。事实上,如果这种破坏性的幻想和欲望出现的话,会导致客体的毁灭。婴儿想吞掉乳房或要妈妈的欲望和幻想,既是爱的前身也是恨的先兆。然而,毫无疑问,生命的第二年,儿童开始对同一客体既有狂怒的情感又有愉快的情感。

这种早期的矛盾情绪在某种程度上贯穿着人的一生,但一般说来,在儿童后期矛盾情绪的强度比2-5岁的儿童要小些,在青春期和成年期强度也比较小。确实,矛盾情绪的减轻往往比实际更显著,尽管对客体意识到的情感对个体的精神生活有重大影响,通常只有一半表现为矛盾情绪,另一半则保留在潜意识中。正像我们所预料的那样,这种持续的矛盾情绪常常与严重的神经症冲突和症状密切相关。
% #认同
早期客体关系的另一特征,是对客体的认同现象,这些我们在第\ref{3}章曾经话及。当时我们指出,在自我复杂的发展过程中,认同作用扮演了极为重要的角色。我们阐明了任何客体关系往往都有认同的倾向,也就是说想变得和客体一样。在自我的发展中,时间越早认同的倾向就越显著。

由于在某种意义上说自我的一部分是这些关系的积淀,我们可以认为,在生活的早期阶段,客体关系在自我发展中起着特别重要的作用。另外,最近几年来,与客体的不适当或不满足的关系得到重视。这意味着,自我机能的正常发展在生活的早期可能受到外部环境的阻碍。从这个意义上讲,儿童后期或成人的严重心理障碍,可能是在儿童早期种下的。

正如我们在第\ref{3}章所述,对高度投注的客体的认同趋势,在潜意识中贯穿人的一生。尽管它在个体的后期生活中并不占据支配地位,但这种地位却已在童年早期就被确定了。虽然这种与客体认同的趋势比意识中的精神生活发展更快,而且也不依赖于我们对持续存在和操作的感知,认同趋势潜意识地持续是许多早期心理机能的一个例子。
% 不定
然而、如果在成年生活中,认同在客体关系中占主导地位,我们就应当把它视为自我发展障碍的证据,并从病理学角度来看待它。这种发展障碍的第一个例子是由Helene Deutsch(1934, 1942)提出的,她把这种现象称为“不定(as if)人格”。有些人的人格随客体关系的变化而改变,就像变色龙一样。如果这样的一个人与聪明人谈恋爱,他的人格和兴趣也适合了聪明型的人。如果他放弃了这种关系,转而加入一个团伙成为一名暴徒,那么他也会全身心地去适应这种生活态度和方式。正如我们从先前讨论中所预料的那样,Helene发现,这些患者早期的客体关系,如与父母的关系,总是不正常的。关于自我发展停滞和障碍的相同病例,Anna Freud及别的学者均报告过。

客体关系的早期阶段通常涉及前生殖器客体关系,有时特别与肛门或口唇的客体关系有关。在这种联系中,使用“前生殖器”(pregenital)一词不够恰当,更准确地是应该用“前性器”(prephallic)。总之,在精神分析文献中,儿童的客体关系通常按性感带来称呼。这种性感带,在儿童当时的libido生活中,起主导作用。

这些名称主要具有历史意义。弗洛伊徳研究libido的发展阶段,先于研究早期精神生活的其他方面。因此,自然libido发展阶段的名称被用于描述儿童期的所有现象。然而,一旦把它们用于客体关系,libido的术语就不仅仅限于历史的意义了。它提醒我们:无论如何,libido只是一种内驱力,可能主要是性驱力,它只有通过客体才能得到释放或满足,因而它把寻求客体放在第一位。客体关系的重要性,主要由本能需要的存在而决定、内驱力和客体之间的关系影响人的一生。我们之所以强调这件事,原因在于:客体关系与自我发展之间的联系有时会被忽略。

通常2.5-3.5岁之间的儿童,就开始了他一生中最强烈和最致命的客体关系。正如读者在第\ref{2}章了解到的那样,从内驱力的观点来看,从肛门水平向性器水平发展时,儿童的精神生活也发生了变化。这就是说,儿童从本能生活的客体上体验到的强烈的或主要的欲望和冲动,从此以后将会变为性器水平。这并不是说,儿童迅速而完全地放弃了较早时期支配他的精神生活的肛门和口唇欲望,恰恰相反,正如我们在第\ref{2}章所述,前性器欲望一直要持续到性器阶段。但在这一阶段中,它们居于从属的地位而不是支配地位。

无论从自我的观点还是从内驱力的观点来看,性器阶段都不同于以前的阶段。然而就自我而言,这些不同是由于自我机能的进步和发展所造成的。自我的机能在整个童年期特别在童年早期就已经性格化了。当时,这种变化发生在本能生活内部,即在本我内部。我们认为,从口唇到肛门再到性器的变换,主要是由遗传的生物趋势造成的。

3岁或4岁时的自我更有经验,更进一步得到发展和整合。因此,与1-2岁时的自我相比,存在着多方面的差异。这些差异部分表现在我们曾涉及到的自我机能的各个方面,即表现在与自我有关的儿童的客体关系的那些特征上。如果能正常发展的话,儿童到这个年龄就不再有部分客体关系。例如,母亲身体的某几部分:她的各种心境;能满足儿童愿望的“好”妈妈角色与使儿童失望的“坏”妈妈角色,都被这个年龄的儿童重新认识,并与称为妈妈的单一客体进行比较。此时,儿童的客体关系已达到相当程度的持久性和稳定性。即使客体暂时不存在,指向某个客体的投注仍然保持。这在自我发展早期是不可能的。甚至在客体长期地完全地不在的情况下,这种投注仍然固执地保存。另外,在性器期儿童至少能非常清楚地区分客体和自身,并且以同样的情感和思维把客体构想成自己。的确,正如我们在第\ref{4}章所见,这种构想过程有些不真实。在这个过程中,动物和玩具被认为与人相似,并且儿童容易把自己的思想和冲动以错误的方式投射到他人身上。然而,我们在此要指出一点,在性器阶段,儿童的自我发展已达到了这种水平:与童年后期及成人生活相比,客体关系即使不能在各个方面完全一致,至少也很相近。四五岁儿童的自我知觉和客体知觉已发展到一定程度,致使某些情感的存在成为可能。这些情感包括对特定客体的爱或恨,以及嫉妒、恐惧和愤怒等。此时的情感已具备了以后生活中出现的同类情感的全部基本特点。
% 俄狄浦斯
性器期最重要的客体关系被组合在一起成为俄狄浦斯情结(俄狄浦斯 complex)。正如把2.5-6岁称为性器阶段或性器期一样,这一阶段也常常被称为俄狄浦斯阶段或俄狄浦斯期。由俄狄浦斯情结构成的客体关系对正常的或病态的心理发展都兵有极为重要的意义。尽管我们已经知道,越是早期出现的事件对个体的决定作用也越大,因而似乎俄狄浦斯阶段出现的事件,在生活中的重要性比不上前俄狄浦斯阶段或前性器阶段。但是Freud认为,事实上俄狄浦斯阶段的事件仍然对绝大多数人具有决定性的作用,并且其作用对所有人都是显而易见的。

关于俄狄浦斯情结的知识,我们可以由下述方面得以增加。Freud很早就发现,对双亲中异性者的乱伦幻想与对双亲中同性者的嫉妒和谋杀冲动,有规律地出现在他的神经症病人的潜意识精神生活中。由于这种想与古希腊传说中的俄狄浦斯的经历十分相似(他杀死了父亲并与母亲结婚),Freud把这种发生了显著变化的情意丛(constellation)称为俄狄浦斯情结。在本世纪初前十年或十五年内,人们认为俄狄浦斯情结不再为神经症患者的潜意识生活所独有,认为正常人也同样具有俄狄浦斯情结,儿童期存在这种欲望和由此引起的冲突实际上是全人类所共同具有的一种经验。正像许多人类学家阐明的那样,与我们有文化差异的人,其精神生活和儿童期的冲突与我们有所不同,但他们对父母的乱伦和谋杀的冲动及冲突却存在于一切文化之中。

除了认识到俄狄浦斯情结具有普遍性之外,在二十世纪前二十多年里,我们对俄狄浦斯情结欲望本身的理解也有所加深,包括所谓逆反的或负俄狄浦斯情结,即对父母中同性者的乱伦幻想和对父母中异性者的谋杀欲望。起初,这种幻想和情意丛被认为是很特别的,但很快被认为是普遍现象。

下面对所谓俄狄浦斯情结给予简要而全面的陈述。它对父母具有双重态度:一方面,怀有嫉妒,想排除可憎的父亲,并与母亲发生肉体关系以取代父亲的欲望;另一方面,怀有嫉妒,排斥可憎的母亲,并代替她与父亲发生关系的欲望。

让我们看一下是否有可能追溯俄狄浦斯情结典型的发展,以便给予更真实的阐述。先要提醒读者一句,切莫忘记,俄狄浦斯情结最重要的是包含于其中的情感的强度和力度,这种情感是一种实实在在的爱。对许多人来说,它是整个一生中所经历的最强烈的事件。读者在阅读时记住下面的内容还不够:在儿童中普遍存在的爱和恨、向往和嫉妒、暴怒和恐惧等情感躁动的强度,这正是我们试图描述俄狄浦斯情结时所谈到的东西。

在俄狄浦斯阶段的初期,不论是男孩还是女孩,一般都同母亲有最强的客体关系。据此我们知道,这时除了儿童自己的身体以外,儿童对母亲心理代表物的投注比任何其他东西都更强烈。我们不久将会谈到,这是一个重要的特殊情况。就我们所知,俄狄浦斯阶段的第一个时期对两性是同样的,由业已存在的与母亲的关系,扩张和延伸到儿童醒觉时的性冲动的满足。与此同时,要求母亲对他们爱得专一的欲望也在发展,这种欲望与想“成为父亲”、或者“做那些父亲对母亲所做的事”的愿望有关。这个年龄的儿童,对父亲所做的那些事情当然不很清楚。但是无论有无机会对父母进行观察,儿童从自已的生理反应出发,会把这些愿望与性器官所产生的令人激动的感觉联系起来。就男孩来说,总是与他的生殖器勃起的感觉和现象联系起来。Freud在早期从事治疗神经症病人的工作时就曾发现,儿童对父母的性活动会产生各式各样的幻想,并希望能与母亲重复这些活动。

正如我们所知,这些幻想或推测通常与儿童和成人之间令人愉快的体验有关,这种体验在俄狄浦斯阶段开始时就已经被儿童通过自淫活动而熟悉了。毫无疑问,随着年岁的增长,儿童的性幻想也随经验和知识的积累而发展。另外,我们还应该谈到,儿童希望像父亲那样也给母亲一个小孩的欲望是俄狄浦斯愿望中非常重要的一种。这个阶段的性理论主要涉及到如何去做这件事,以及小孩形成后又如何能够出来等问题。

伴随着对母亲性的渴望和得到母亲专一爱的欲望,还存在希望竞争对手被歼灭或消失的欲望。竞争者一般指的是父亲或兄弟姐妹。兄弟姐妹竞争者公认不止一个来源,而实际上最主要的是占据父亲特有的地位的欲望。

嫉妒﹑谋杀的欲望造成儿童严重的冲突,主要有两方面,其一是对报复特别是来自父母的报复表现明显的恐惧,因为对那种年龄的孩子说来,父母似乎是真正全知全能的人;其二是由爱和敬慕的情感所产生的冲突,表现为对父亲和兄姐的渴望和依赖,以及想消灭弟(妹)的愿望为父母所不允而引起的恐惧。换言之,作为嫉妒欲望的结果,儿童既惧怕报复又惧怕失去爱。

以上论点,为我们把女孩和男孩的俄狄浦斯情结分别考虑提供了便利条件。下面我们先讨论后者。

从对许多成人和儿童进行分析所得到的经验,以及从人类学、宗教、神话、艺术、创造和各类其他资料得到的证据,说明幼儿惧怕的报复——作为他的俄狄浦斯情结的结果——是他的阴茎的丧失。在精神分析文献中有一个专门术语来表示这种现象——阉割(castration)。不论儿童期个体或文化环境如何,他都具有这种恐惧。已有许多不同作者系统地阐述了儿童为何惧怕阉割,在这点上我们无须参与他们的讨论。

现实中确实有些人没有阴茎,即女孩和妇女。儿童观察到的这种现象,使他相信他自己有可能被阉割。由于担心丧失他极为珍视的性器官而引起的恐惧,使强烈地冲突沉淀在他的俄狄浦斯欲望中,这种冲突最终导致对俄狄浦斯欲望的否认。它们部分被抛弃、部分被压抑,这意味着它们流入了难以接近的儿童潜意识的幽深峡谷之中。

幼小的男孩由于想专有地享有母亲的身体和爱抚的欲望遭到拒绝,有可能激起对母亲的嫉妒及愤怒,导致除掉母亲,以及以父亲之爱来取代母亲位置的欲望,这样就使情况更加复杂,因为这种情况非常容易导致阉割恐惧。于是乎,一旦儿童了解到妇女没有阴茎,这些欲望就会最终被彻底地压抑。

由此我们可以看出,俄狄浦斯阶段不论是男性的还是女性的欲望都会激起阉割焦虑。同时由于小男孩无论在生理方面还是在性方面都不成熟。只能以放弃欲望,或利用各种防御机制及自我的其他防御操作来控制欲望,以解决由这些欲望激起的冲突。

至于小女孩的情况则稍微有些复杂。她想以男人的位置与母亲相处的欲望,不是在阉割恐惧基础上形成的,而是由于她天生没有阴茎所引起。她缺少这种器官的现实造成了不幸,这种事实引起了强烈的羞耻感、自卑感、嫉妒(阴茎妒忌和羡慕),以及由于允许她没有阴茎而降生所引起的对母亲的愤怒。女孩在愤怒和绝望中,通常把父亲作为主要的恋爱对象,并希望以他取代母亲。当这些欲望受到的挫折过于强烈,在必定会成为女孩的事实面前,她又有可能把早期的依恋对象转向母亲,而把希望有阴茎和想成为男人的欲望一辈子保持在性心理行为中。然而更为常见的是,当小女孩把父亲作为惟一的性对象遭到拒绝时,她被迫放弃或压抑自己的俄狄浦斯欲望。对女孩来说,与在男孩俄狄浦斯欲望中起着具有决定性作用的阉割焦虑相类似的东西、首先表现为“阴茎羡慕(penis envy)”的禁欲和嫉妒,其次表现为由希望被父亲阴茎插入并授精的欲望引起的生殖器受伤害的恐惧。

读者将会明白,对俄狄浦斯情结说明是高度概括化的,实际上这个时期每个儿童的精神生活都与旁人不同。这些精神生活深受生命的最初两年那些先于俄狄浦斯时期的经验的影响,同时也受俄狄浦斯时期事件的影响。例如可以想象,在俄狄浦斯时期,一个人由于父母或兄弟姐妹的疾病、离开或死亡,由于弟(妹)的出世,由于看到父母或其他成人的性交,由于成人或年长儿童的性诱惑等,都会造成极严重的后果。

除了这些环境因素之外,我们相信儿童有可能在体质上或素质上均不相同。Freud(1937)提到了本能禀赋的变化,例如双性化(androgyny)的倾向,即男孩女性化的素质,女孩男性化的素质。他假设在心理范畴内,双性化的倾向对所有人都是正常的表现。这实际是来自正常的俄狄浦斯情结的一个推论,它包括与父母双方都有性关系的幻想。然而很清楚,在性驱力中男性化成分与女性化成分相对强度的变化,对各种俄狄浦斯欲望的相对强度会有很大的影响。

例如,男孩强烈的女性化体质倾向被认为有利于俄狄浦斯情结的发展。在这种俄狄浦斯丛中,取代母亲而与父亲发生性关系的欲望,要比那种取代父亲而与母亲发生性关系的欲望更加强烈。反之对女孩说来,强烈的男性化倾向所造成的与上述男孩相反的情况也同样成立。至于是否在任何情况下都是这种结果,那就要由对这种体质倾向有利或不利的环境因素的多少来决定了。此外,我们目前还无法对体质或环境孰为重要做出使人满意的预测。在临床工作中,我们也常常对体质因素和倾向一无所知。因此,在与环境因素进行比较时,我们可能会疏漏掉一些重要的因素,而环境因素通常是显而易见的,因此容易给人留下深刻的印象。

在俄狄浦斯期中,至少有个重要方面我们还没有提到,但它又不应该被忽略,这就是生殖器手淫,它通常构成儿童在这一时期的性活动。手淫行为和伴随它的幻想,在很大程度上取代了直接的性表现和儿童对父母的攻击冲动。从长远来看,这种取代了在现实环境中与他人进行真正性活动的自淫方式,其结果究竟对儿童有益还是有害,部分地取决于个人所选择并接受的道德标准。但无论如何,这种评价似乎没有什么意义。代替是不可避免的,因为就最近的分析来看,它是由于生物学上不成熟而强加给儿童的。

随着俄狄浦斯期的推移,生殖器手淫或者停止,或者明显地减少,直到青春期才重新出现。最初的俄狄浦斯幻想遭到了压抑,但它们经过伪装,以儿童熟悉的白日梦方式保留在潜意识里,继续对精神生活的各个方面施加重大的影响:影响成人性活动的形式和对象;影响创造性的、艺术性的、职业的和其他升华了的活动;影响性格形成;影响个体可能产生的所有的神经症症状(见第\ref{8}、\ref{9}章)。
% 超我的形成
然而,这俄狄浦斯情结并不是影响个体将来生活的惟一因素。另外有一个非常重要的因素,对日后的精神生活起着作用。我们现在就来加以讨论,它就是“超我的形成(superego formation)”。这是Freud在其所谓心理结构的结构假设中所假定的心理机能中的第三种。

正如我们在第\ref{3}章所述,超我大体上与我们称之为良知的内容相应,它由人格的道德机能构成。这些机能包括:(1)在正直的基础上对行动和愿望的赞成或反对;(2)批评性的自我观察;(3)自我惩罚;(4)对错误行为补偿和悔过的愿望;(5)对合乎道德或需要的愿望和行动进行奖励的自我赞扬或自爱。然而,与一般意义上的“良知”相反,我们认为,超我的功能通常大部分甚至全部都是潜意识的。据Freud说,一方面精神分析表明,在个体有意识地拒绝和否认时都有潜意识欲望的存在,因此人类比自所了解到的更缺乏道德;另一方面精神分析又表明,我们每个人所具有的道德要求和禁令要比我们能意识到的要多而且严格。

让我们回到超我起源话题上来,目前,一般都同意超我最早起源于(或者说它的前身出现在〕前性器期或前俄狄浦斯期。父母或代替父母行使职责的保姆、保育员和老师的道德禁令和要求,在儿童早期就开始影响儿童的精神生活。当然,这些影响在一岁末才表现出来。如果以成人的标准来判断,我们就会意识到儿童早期的道德要求非常简单。其中最重要的是与排便训练有关的要求,Ferenczi把这种超我的前身称为“括约肌道德(sphincter morality)”。

然而在前俄狄浦斯期。儿童把针对他的道德要求视为环境的一部分。如果母亲或其他道德仲裁者在那儿,为使她高兴,儿童会控制自己不去做违反道德的事情。如果儿童独处或者生母亲的气,他就会不惧怕惩罚而使母亲不高兴或自己为所欲为。到了俄狄浦斯期,超我在这方面有所改变。在儿童长到五六岁时,有些道德变成内部的东西。我们相信,儿童首先体会到的道德标准和要求是做了错事要受到惩罚,并体验到后悔,这些折磨并不是来白他所服从的客体,而是来自儿童自己。除此而外,我们还相信,尽管内化过程在青春期和成人生活中不断地增补和修饰,但在9-10岁的儿童就已基本上稳定下来并长期地存在下去。

这种事关重大的内化作用是何产生的?就我们所知,在放弃、压抑或拒绝俄狄浦斯情结里乱伦和谋杀欲望的过程、儿童与这些欲望的客体之间的关系,大部分转化为与这些欲望客体的认同,他不再因父母反对或惩罚这些欲望而爱父母或恨父母,代之以变得像父母一样去拒绝这些欲望。超我最初禁忌的核心,是要求个体拒绝那些由俄狄浦斯情结构成的乱伦及敌意欲望。这种要求是超我的本质,并以潜意识过程贯穿人的一生。

由此我们认识到,超我与俄狄浦斯情结有特殊密切的关系,并且是通过与父母的道德和禁忌的认同而形成的。而在俄狄浦斯情结解体的过程中才在儿童的心里产生这种认同。我们可以说,超我最早是由性器期或俄狄浦斯期父母道德的内化而构成的。

现在,让我们更仔细地考察一下认同过程的某些方面。在谈到认同的发生时,我们必须记住,自我此时的主要任务是防御俄狄浦斯情结的抗争。我们知道,在此年龄阶段,儿童精神生活中占主要地位的抗争,男孩是阉割焦虑(女孩则与此相似),其余的一切只不过是这个主要问题的一部分、延续或附属。

从自我的观点来看,形成超我的认同作用十分重要,它有助于对不服从控制的本我冲动建立防御机制。也就是说,父母的禁令永久性地保存在心里,以便对本我起警戒作用。儿童以这种方式与父母认同之后,一且本我冲动出现,就像父母在旁边一样,及时地制止了这些欲望的实现。

从防御观点来看,超我的认同对自我很有利。实际上,我们完全可以说超我的认同支撑着自我。然而,从自我独立和自由地享受本能满足的观点来看,超我的认同又对自我很不利。从超我形成之日起,自我就丧失了活动的自由,而永远地处于超我的支配之下。自我并不是得到了超我这个同盟者,而是得到了一个主人。自我必须服从于超我,并且在协调本我和外界环境之外又增加了一项要求。自我通过认同也能得到父母的一些力量,但只能在永远隶属于他们的条件之下才能成立。

Freud(1923)对认同的形成做了两次更深入的观察。第一个观察是,儿童体验到的父母的禁令大多数是言语上的命令或训斥,由此推断,超我与听觉记忆特别是口头语言的记忆有密切的关系。这种事实的直觉是那种与“良知的声音”有关的言语形式。在精神压抑的状态下,比如在梦中、在某些严重的精神疾病之时,超我的机能以言语的形式被觉察。这时,个体感到似乎是来自外部的声音,就像他小时候听父母的命令一样。然而,不应该相信超我只限于听觉和记忆有关,其他各种感知觉的记忆,如视觉和触觉的记忆,也同样与超我有关。例如,一个被自己敌意幻想惊吓到了急性焦虑发作的程度的病人,只要他一想到生气,就感到脸被别人扇了一巴掌。这个病例,超我的活动被体验为来自外人的体罚,就像他在童年时受父母惩罚一样。

Freud(1923)的第二个观察是,通过内射作用而形成超我的父母表象,在很大程度上就是父母的超我。换句话说,父母在养育子女时,总是以他们自己儿时父母对待他们的方式来训练他们的子女。他们把自己早期所获得的道德要求又用到自己的孩子身上,结果造成孩子的超我反映了或者说类似于他们的父母。正如Freud所指出的那样,这个特点已成为一种重要的社会后果。它导致社会道德准则永恒存在,并且造成保守思想的存在,因为它阻碍了社会结构的变革。

现在,我们考虑一下超我形成的某些方面,它们与本我的关系比与自我更为密切。就像Freud指出的那样,超我的认同在某种程度上是放弃俄狄浦斯情结中乱伦客体关系的结果。在此意义上,认同部分地是由于客体的丧失所造成。读者可能还记得我们在第\ref{1}章讨论过,这是认同作用的机制之一。正如我们所理解的那样,当本能投注从原先的客体退回时,它们对一个客体的不断地搜寻导致与存在于自我之中的客体认同,而投注随后即朝向此客体。被投注的客体变成了自恋的对象。于是,自我中所形成的这种认同构成了特殊的自我部分,我们称之为超我。

从本我的观点来看,超我是俄狄浦斯情结的客体关系的替代物和后继者。为此,Freud认为超我的根深深地扎在本我里。我们由此认识到,超我的形成导致大量的客体投注转为朝向自身和自恋。通常,最公开的性投注以及最直接、最强烈的敌意投注被放弃了,而温和的不太强烈的敌意情感继续依恋着原先的客体,即儿童继续对父母怀有少量的温和的、不太强烈的憎恨或反叛的情感。为避免理解错误,我们应该清楚这一点,所有儿童对他们父母的直接的乱伦和谋杀冲动一点也没有放弃。恰恰相反,其中相当一部分,乃至大部分儿童的这些冲动只是被简单地压抑下去或者受到防御而已。这种继续留在本我中的部分,就像其他被压抑的欲望一样,针对着原先的客体,在行为、意识、思维和幻想中公然表现出来,只有由自我指挥的抗投注与这些欲望对抗。然而,这些被压抑的俄狄浦斯欲望随着投注,不再对超我的形成有帮助。为此,即使它们十分重要,我们也不再讨论它。

一个奇怪却又很容易观察到的事实是,个体超我的严厉程度并不一定与儿童期父母反对其本能欲望的严厉程度相一致。在目前讨论的基础上,我们希望出现这种情况。由于超我是内化了的父母。我们希望有严厉父母,儿童也会有严厉的超我,反之亦然。在一定程度上,这无疑是正确的。例如,在俄狄浦斯期男孩的阉割恐惧或女孩类似的恐惧会形成严厉的超我,结果在以后生活中,性以及攻击都受到严厉的禁止。这种超我,以及这些禁止,都不是个体所情愿的。

然而,除了父母的严厉程度之外,其他因素似乎在决定超我的严格程度上起更重要的作用。其中,主要是儿童自己俄狄浦斯欲望中攻击成分的强度。如果不太严格地看来,我们可以说,决定超我严格性的重要因素是儿童在俄狄浦斯阶段对父母的敌意冲动,而不是父母对儿童的敌意或严厉性。

我们相信,我们可以用以下情况理解或解释这点。当俄狄浦斯客体由于超我的认同作用而被放弃和被取代时,从前投注到客体上的驱力能量将至少部分地受自我新形成的部分——我们称之为超我——所支配。受超我支配的攻击能量,来源于投注到俄狄浦斯客体上的攻击能量。这两种能量即使在力量上不相等,也至少是成比例的。这就是说,投注到俄狄浦斯客体上的攻击能量越多,则以后受超我支配的这种能量也越多。无论何时,只要某人被强迫服从超我的命令,或者由于违反了禁令而受到惩罚,则攻击能量就会针对他自己。换言之,超我的严厉程度由它所支配的攻击能量的多少来决定。回过头来,与俄狄浦斯期父母对儿童的禁令的严厉程度相比,超我与儿童对父母的俄狄浦斯冲动的攻击投注的关系更为密切。儿童的俄狄浦斯期幻想越强烈,越具破坏性,就比破坏性幻想轻的儿童有更强的罪恶感。

我们从本我的角度看,超我形成的最后意见是这样的。阐明俄狄浦斯阶段的冲动的一种方式是,认为本我的冲动与这一阶段的客体有关。即父母会使儿童的身体造成伤害。对男孩,这种恐惧是怕失去他的生殖器;对女孩,类似的恐惧则是害怕性伤害,或由缺乏阴茎引起的强烈的不愉快的羞耻感。总之,在客体投注的欲望与自恋的或向自身投注的欲望之间存在着冲突。问题在于,是否有利于自恋投注,注意到这点对我们很有益。当自恋投注基本保持完整时,危险的客体投注就被压抑、被放弃,或者以其他的方式受支配或被拒绝。因此,我们再一次强调这种观点:虽然儿童本能生活中的客体关系很早就被观察到,而且似乎更引人注目,但儿童本能生活中的自恋成分却比客体关系更为强烈。

在讨论超我的形成时,我们必须讲到它在儿童后期、青春期以及成人生活时的修饰和扩展。每一次这种附加和改变都是与儿童或成人环境中某一个客体认同的结果,或者是与这个客体的道德形象认同的结果。最初,这些客体是儿童生活中像父母一样有影响的人们,如老师、神职人员和家里的仆人,以后儿童可能将没有接触过的人,甚至历史上、小说里的人物内化。这种认同在青春期以前和青春期尤为普遍,他们塑造了个体的超我,使个体与那些道德标准及观点相一致,而成为其中的一员。

当我们考虑存在于各社会群体道德准则之间的明显差异时,我们会认识到,成人的超我中很大一部分是认同作用的结果。超我在成年期可能会产生变化,例如由于宗教信仰改变而导致的变化。然而,在俄狄浦斯期形成的超我的原始核心总是保留着其中最稳固最有效的部分。因此,禁止乱伦和弑父母是个人道德中最主要的部分。这种个人道德完全被内化,且似乎很少被违反。如果时机适当或者诱惑极为强烈,超我的其他禁令更有可能被违反。

现在我们讨论,当超我形成以后,超我在心理结构机能中所起到的几方面的作用。一般来说,我们可以认为,在俄狄浦斯期结束以后,超我发动并强迫自我针对本我冲动的防御活动。由于儿童在俄狄浦斯阶段害怕被父亲阉割,并压抑或放弃他的俄狄浦斯欲望以免遭伤害,所以在俄狄浦斯阶段以后,儿童或成人潜意识地惧怕他所内化的父母的表象,也就是说,他的超我控制本我冲动,以免使超我不愉快。我们在第\ref{4}章讨论过一系列能引起自我焦虑反应的危险情境,其中最后一项是超我的反对。现在我们把第\ref{4}章的有关问题重新列举一遍,按年龄顺序分别是(1)丧失客体;(2)丧失客体的爱;(3)阉割恐惧或类似的生殖器伤害;(4)超我的非难。读者可能还记得,相继出现的各种危险情况并不是相继地消失。实际的情况是。各种情况依次作为焦虑的主要来源,以及作为自我反对本我冲动的防御措施。

超我所反对的东西有些可以被意识到,我们所熟悉;有些则是潜意识的,只能通过精神分析才能被了解到。例如,我们都熟悉内疚或懊悔的痛苦情感,并且毫不犹豫地把它们与超我的操作联系在一起。然而,还有一些与超我密切相关,却没有得到足够重视的心理现象。因此,Freud指出,超我的反对是自卑感之所以感到痛苦的原因。实际上这种自卑感与内疚情感是一样的。这种现象在临床上十分重要,因为它告诉我们,一个严重自卑或自尊心过低的病人可能是在潜意识里谴责自己错误的行为,而不像病人所说的那样。

正如超我对自我的反对导致内疚感和自卑感一样,高兴、快乐及自满的情感可能是超我对自我的赞同,即超我对自我的行为和态度赞同的结果。像内疚感一样,这种“美德”也是人所周知的现象。这两种对立的情感或心理状态,可以比作小孩的行为,既受到父母的赞扬和疼爱,又遭到父母训斥和惩罚。换句话说,一旦我们认识到超我是投注了的父母的象征,并且在一生中超我和自我的关系就像父母和孩子一样,那么在以后的生活中,由于超我的赞成或禁止而产生的有意识的情感就很容易理解。

在成人生活中,超我通常是潜意识的,它与儿童早期的心理过程密切相关,并且就产生于其中。超我有两个特点,其一是报复法则,其二是对欲望和行为缺乏辨别。
% 法律报复
法律报复(Lex talionis)的意思很简单,是使作恶者遭受到他施加给别人的同等的伤害。这在圣经的要求“以眼还眼、以牙还牙”中表现得更清楚。在两种意义上,它具有原始的裁判概念。第一种意义,它是古老而原始的社会结构中特有的裁判概念。这个事实很重要,却又与我们的现代社会结构无关。第二种意义与我们有关,即报复法则在本质上是小孩的裁判概念。令人觉得有趣并感到意外的是,这种概念潜意识地持续到成人,并决定着超我的功能。我们可以通过对许多例子的分析,发现超我所强加的潜意识的惩罚。

由于在欲望和行动之间缺乏辨别,在心理分析中常常可以看到,超我对各种情况惩罚的严重程度几乎相似。显然,这不仅是针对做被超我禁止的事,而且也针对被禁止的欲望或冲动本身。超我的这种态度,由一个四五岁或更小的儿童很难区分自己的幻想和自己的行动这一事实得已证实,他主要受“欲望造成这样”的信念所支配。在以后一生中,这种态度由于超我的潜意识操作而永远保留。

超我的潜意识操作的另一特性是,它可能在潜意识里导致赎罪或自我惩罚,这种情况只有通过心理分析才能发现。一旦认识到这种事的存在,并且对它注意的话,一个人就会发现许多过去没有想到的现象。例如,监狱里的精神病医生读一读如何抓获重罪犯的官方报告,就很容易理解这一点,罪犯自己在潜意识里对惩罚的欲望,是对警方破案非常有力的帮助。罪犯常常潜意识地提供了那些自己知道会导致被发现、被逮捕的线索。当然,对罪犯进行精神分析一般是不可能的,但在有些情况下,报告中的这些事实足以说明问题。

例如,一名溜门撬锁的盗贼,一年多来一直很顺利。他时常出没于中下阶层的住宅区,在那里,他能轻而易举地通过公寓的走廊和楼梯到任何房间里。每天上午,他躲葳在暗处,专等某个主妇外出购物时,他才进到空无一人的房间。他不留任何痕迹,除了警察无法追踪线索的现金外,不拿其他任何东西。显然,这个盗贼很清楚他在做什么。在几个月内,警察很难发现他的活动。看来只有他运气不好才有可能结束这种行当。突然,他完全改变了自己的习惯,不仅拿现金,也偷取珠宝,并且拿了一小部分到附近的当铺去典当,几天后他就落入警察手中。以前多次行窃时。他都不动珠宝,那些珠宝的价值不次于他最后所偷取的那些,因为这个盗贼知道,处置这些珠宝而不引起警方注意,几乎是不可能的。由此我们可以说,这个罪犯潜意识里安排了自己被逮捕入狱这件事。鉴于我们对潜意识心理活动的了解,不难看出,他之所以这样做,是出于对惩罚的潜意识需要。

当然,对惩罚的需要不见得都跟刚才提到的例子那样,一定要与不端行为相系在一起。它可能也是意识的或者潜意识的幻想和欲望的派生物。正如Freud指出的那样,一个人的犯罪生涯可能始于惩罚的需要。也就是说,由于对俄狄浦斯欲望的压抑而在潜意识里产生的对惩罚的需要,造成了必定会受惩罚的犯罪活动。从犯罪的观点看来,这个人就是罪犯。

然而,我们还应当补充一句,潜意识里对惩罚的需要,不一定都会导致遭受法律上处罚的那一类犯罪活动。在潜意识里,也可能导致其他形式的痛苦或自我伤害,比如说职业生涯的失败(所谓的“命运神经症”)、“意外的”受伤等。

我们业已了解,从自我的观点来看,坚持自身惩罚或自身伤害的超我本身已构成一种危险。因而,自我可能会与超我的防御机制,以及那些反对本我的防御机制相对抗。下面的例子可以说明我们的意思。一个在童年有很强的窥淫倾向的人,成年以后,会成为反对不道德行为组织的积极支持者。他特别地热衷于刺探和检举淫秽照片的传播,因为这种活动使他能够不断地找到裸体男人和女人的照片。显然,这个人在潜意识中为满足他的窥淫癖提供了机会。然而,上面是用本我和自我之间的防御斗争或冲突的观点来解释的,而不是用自我和超我之间的冲突来解释。从自我和超我冲突的观点,可以看到两点:首先,成年人看裸体照片时,并没有童年时观看裸体而引起的罪恶感。他的自我已将罪恶感排除在意识之外,并投射到别人。这里的别人是指有窥淫罪的人,更确切地说,是那些因窥淫欲望及活动而受到惩罚的环人。另外,此时的自我已建立了反向形成以对抗罪恶感,这样在他的意识里体验到的不是罪恶感,而是刺探和发现裸体照片的优越感和美德。

自我对超我的防御,是有规律的,是正常心理机能的一部分,并可能在许多精神疾病中起重要作用,因而在临床上不容忽视。

Freud在专门的文章中曾指出,超我和集体心理学之间有重要的联系。集体,是靠集体中的每个成员对领导者的内射和认同而结合起来的。这种认同的结果,使领导者的形象成为每个集体成员超我的一部分。换句话说,集体的每个成员都有共同的超我成分。这样,领导者的愿望、命令和知觉就成为下属的道德准则。虽然Freud的专题著作早在Hitler崛起之前就已写成,他对集体心理学的分析,却能完满地解释Hitler影响成千上万追随他的德国人的道德标准。

在宗教集体或教派中,也有同样的机制。在这些情况下,集体的各个成员都有共同的道德标准,即共同的超我成分。它产生于对同样的神灵或宗教领袖的认同,从心理学意义上说,神灵在此起到了与非宗教集体中的领导或英雄同样的作用。我们知道,在人们心目中,神灵和英雄之间有密切的关系,甚至像有着高度文明的罗马帝国的臣民,也把自己的皇帝奉若神明。

我们再一次谈谈超我的起源和性质,来对超我做一个总结。超我是俄狄浦斯期父母的禁令和劝诫的结果。在整个一生中,超我的潜意识本质依然是对俄狄浦斯情结的性欲望和攻击欲望的禁止。无论超我有多大变化或添加,人们在儿童后期、青春期乃至成人生活,都会有这种禁忌。

\signatureB



\chapter{过失和诙谐}\label{6}

在本章及随后的两章里,我们将把前面所讨论的心理功能的知识,应用于人类精神生活的某些现象中去。我们打算讨论的现象有:(1)失误、过失、忽略和遗忘。大家对这些现象均很熟悉,Freud将这些现象归纳成日常生活的病理心理学;(2)诙谐;(3)梦;(4)神经症。之所以挑选这些题目,是因为它们都是精神分析理论的经典论题,经过Freud及追随他的精神分析学家们多年的研究,内容广博而且可信。此外,神经症是精神分析治疗的主要对象,因而对这种精神疾病进行讨论,就具有十分重要的实践意义。

我们先从日常生活的病理心理学谈起。这里边包括口误、笔误、遗忘,以及许多生活中的不幸之事。一般,我们总把这些情况归咎于偶然发生的,称之谓意外事故。其实,在Freud系统地研究这些现象之前,人们已经模模糊糊地觉察到这些现象的发生有其目的性,而不是偶然发生的。一个古老的谚语就说:“口误是内心真实想法的流露。”说明并不是所有的口误都被认为是无意的事情而对待的。比如,某S君把J小姐的名字给忘掉了,或者“误”称她为R小姐,J小姐则会认为这是有意瞧不起她,或者根本没有把她放在心上,S君自然也就得不到她的好感。还有,假如某人对他尊贵的主人讲话时“忘记了”应有的规矩,不论他怎样地辩解说这纯属无意的,终归还是要受到惩罚的。即使他确是无心,他的主人也会认为他的举动是有意的。30多年前。有一次圣经中十诫“你不得\ldots\ldots”被无意地误印为“你得\ldots\ldots”,结果以有意亵渎神圣之罪严厉惩处了排字工人。还有两种看法:一种看法认为,这些现象纯系出于无意;另一种迷信的看法认为,这是邪恶和魔鬼在作祟。比如说,排字妖怪会将排字工人的头脑搞乱,造成错误百出。Freud第一个严肃而坚定地提出,即使本人没有意识到,失误及与其有关的现象也是个人有目的、有意图的行为。换言之,这是潜意识的行为。

最容易理解的失误或过失是遗忘,这种失误往往是由于压抑而直接造成的。在第\ref{4}章中已经讨论过,压抑是自我防御机制之一。在进行精神分析的过程中,我们可以观察到它简单而明显的表现。此种现象发生在,当病人谈到他认为是重要的并且希望记住的事情时,有时会停顿片刻才能想起来,因为在这种情况下,遗忘的动机也会出现。尽管每个人遗忘的动机不尽相间,其基本目的是一个——防止产生焦虑及罪恶感。

举一个例子,某病人多年来一直巧妙地使用合理化的手段以避免因自己的性行为而感到惧怕及羞愧,在进行精神分析治疗时,他虽然没有强烈地体验到这种情绪,但还是领悟到了他内心的惧怕和羞愧是由于自己的性行为所造成。他对于这个新获得的领悟印象极深,认为这对于了解自己的神经症症状十分重要,因为事情确实如此。但是,仅仅一两分钟之后,当他正在谈论这个领悟有多大的价值之时,他突然想不起来刚才在谈些什么问题,而且将自己五分钟以前谈的所有事情也都忘得一干二净。

这个例子,近乎戏剧性地说明了人类心理活动中遗忘(更确切地,应说为压抑)的能力。显而易见,多年来存在于病人内心的那一股能使他避免因自己的性行为而感到惧怕及羞愧的力量,将他新获得的领悟迅速地压抑下去了。于是谈话时,病人就不至于为自己的性行为而感到惧怕和羞愧。可以说,此例自我的压抑更多的是反抗超我而不是反抗本我。病人的自我,压抑了他当时感到害怕的听觉记忆和思想,因为这些听觉记忆和思想,将会导致由于异常的性行为而感到惧怕和羞愧。当然在别的情况下,自我的压抑主要是反抗本我。

有人可能会认为,我们所举的是一个特殊的不典型的例子,跟通常忘了去做一件原本打算做的事情,或者忘了一个熟悉的名字或面孔完全不同,因为我们很容易发现这个病人忘事的原因。那么,为什么有人会忘掉“不应该”忘掉的事情呢?

其答复是:绝大多数是潜意识的缘故。通常只有运用精神分析技术,在遗忘者充分合作的情况下,才能发现之。如果此人能够合作,将所有想到的与失误有关的事情:毫无顾忌地都谈出来,我们就可以知道其意向和动机。否则,我们则只能靠偶然的机会去获得足够的事实,用来猜测潜意识动机而产生的失误的“含义”。

有一个病人在社交场合上遇到了一个熟人,可就是想不起那个人的名字。如果病人不进行联想,就不可能知道为什么会发生遗忘。联想逐渐深入,病人说到那个熟人跟他认识的另一个人同名,而他非常恨后者,此时病人流露出十分内疚的感情。他补充说,由于那个熟人是个跛子,使他想起曾经产生过伤害他所痛恨的那个人的念头。由病人联想而得到的材料,解释了究竟是什么原因使他忘记了那个熟人的名字。在潜意识里,这个跛子熟人使他想起了另一个同名人,他恨那个人,并想把那个人弄成残废。为了避免意识到自已伤害人的幻想,为了避免由此而产生的罪恶感,他将这两个人的名字都压抑下去了。压抑起到了防止伤害人的幻想进入意识的作用。这种伤害人的幻想是本我的一个组成部分,进入意识则导致罪恶感。

上面所举的两个例子里,记忆障碍或“失误”是防御机制作用的结果,称之为压抑。压抑的动机及其实现都是潜意识的。当事人对于自己记忆的失误无法解释,只好归咎于不走运、疲劳或其他所能找到的借口。除压抑外,其他的心理机制也可能造成失误,但是所有的失误均发生于潜意识之中。

口误和笔误也是对某些潜意识欲望不能完全压抑所致,尽管说话人和写宇者极力隐藏,这些潜意识里想说和想写的东西到底还是流露出来了,有时隐藏的含义在失误中公开地表露出来,听者和读者对此可一眼看穿,有时失误的含义则模糊不清,只能通过此人的联想才能发现背后的意思,

下面是一个含义显现的失误。某律师在炫耀他的当事人对他如何信任时,原打算说他们告诉了他“他们极其秘密的麻烦事”,却说成了“他们极其冗长的麻烦事”。这个口误暴露了隐藏于他内心的愿望,他时常被当事人冗长乏味的谈话弄得厌烦难忍,希望他们把话说得简短些,不要占他那么多的时间。

有人可能会由这个例子而得出一个结论:失误的含义明显,说明其潜意识欲望被压抑得浅而短暂,而这些欲望进入意识只能引起轻度的惧怕或内疚。事实并非如此。例如,某病人在第一次治疗的交谈中,无意地将妻子说成母亲,他会解释说,这个失误不含有任何意思,并举许多的事实来说明他的妻子与母亲有多么的不同。经过仅仅几个月的分析,这个病人就认识到,在幻想中他的妻子是他母亲的再现,而许多年前的俄狄浦斯情结使他极想娶母为妻。显然,此例的失误,揭示了本我对许多年来一直极为强大的自我的反抗。

应该指出,不论失误的含义多么明显,听者或读者对其潜意识含义的解释,在未经失误者的联想证实之前,仅仅只是个猜测而已。可能这个猜测是基于确凿可信的事实,如掌握了发生失误时的情境、失误者的人格及生活经历,看起来似乎是无可反驳了。但是原则上,任何失误的含义,只有通过失误者的联想才能确立。

对于那些一时还搞不清楚其含义的口误或笔误,则绝对需要依据失误者的联想才能确定。发生这一类口误或笔误时,潜意识心理过程侵入想说或想写的内容之中,造成一个音节或几个音节、一个词或几个词的遗漏、插入或歪曲,听起来或谈起来似乎没有什么意思寓于其中。有些人对Freud对于这些现象的解释一知半解,往往不同意他指出的这些失误也必有其含义的说法。于是,他们将含义明显的失误称为“Freud的”失误,将那些含义不清的失误称为“非Freud的”失误。事实上,运用适当的检查技术,运用精神分析的方法,就可以发现隐藏在这些尚不清楚的失误背后的潜意识心理过程的性质和意义。

人们往往将口误或笔误归咎于疲劳、不注意、慌忙、心情兴奋等情况而造成。Freud是否考虑过将这些因素纳入造成失误的原因呢?Freud认为,这些因素对失误的产生仅仅起辅助的作用而已。对某些人来说,如果他们不是处于疲劳、不注意、慌忙之中,可能不至于发生失误,也就是说,这些情况使潜意识的过程更容易侵入意识所想说和想写的内容中,而造成失误,因此主要是潜意识心理过程的作用。他举了一个类似的情况来说明这个观点:假如一个人在黑暗而空无一人的街上被强盗拦劫,他决不会说是被黑暗或空荡所拦劫。拦劫他的是强盗,而黑暗及空无一人的情况帮助强盗成了事。这里,强盗相当于潜意识心理过程,是它造成了失误;黑暗及空无一人的情况相当于疲劳、不注意等情况,起辅助作用。我们可以说,潜意识心理过程是产生所有失误的必需条件。有时,仅只潜意识心理过程,就构成了产生失误的充分条件;有时,仅只潜意识心理过程,尚不足以构成产生失误的充分条件,还需要上述谈到的辅助因素的帮助,才能让潜意识心理过程侵入意识。这样,产生失误的条件才能达到充分的程度。

讨论口误或笔误,不能不谈到初级心理过程的作用。例如,某病人谈到他年轻时对体育活动如何地喜欢时,将“体育”说成“体阅”。他解释说,“育”宇在发音主跟“阅”字相近,故而造成口误。经过联想,发现他潜意识欲望是想显露赤裸的身体给别人看,同时也想看到别人赤裸的身体。这些潜意识欲望是他喜好体育活动的直接原因。我们特别要注意的是,这个失误是病人潜意识中露阴及窥阴的欲望,在病人的意识打算说“育”之时暂时地侵入,结果说成了包括“身体”和“看”两方面意思的混杂词“体阅”。这个混杂词将两个词的意思凝缩合一,与次级过程的语言规律不符。第\ref{3}章中,我们讨论了思维的两种形式:初级思维过程和次级思维过程。初级思维过程的特征之一是缩合。我们认为,就是这个特征的作用,使得“身体”与“看”结合成了“体阅”。

在别的失误里,还可以发现初级思维过程的特征,如精神分析术语中的转移、以部分代表全体、以全体代表部分、以相似物代表、以对立物代表、象征等等。可由其中任何一种的作用,或几种同时作用而造成失误。

需要指出的是,虽然口误或笔误主要通过初级思维过程而形成,但初级思维过程的作用绝不仅仅只限于此,它往往也是造成其他过失的主要原因。比方说,前面谈到的忘掉熟人名字的那个人,由于熟人是个跛子,使他想起潜意识里想伤害同名的另一个人。其实,那个熟人有一只胳臂在出生时受了伤,既短又瘫,而他在潜意识里是想切掉与熟人间名的另一个人的阴茎。这样,熟人胳臂的残废就象征为阉割,这可作为初级思维过程的一个例子。

现在,让我们来看看平时称之为意外不幸的种类:是自己造成的,抑或是由于自己“不慎”而由别人造成的。从一开始就必须明确,此处我们所讨论的均是自己所造成的不幸之事,虽然自己不是故意这样做的。我们不想讨论那些不由自己所控制的不幸之事。

决定一个人是否对他的不幸之事负有责任往往不难,但事情并非总是如此简单。例如,一个人在雷电交加的暴风雨中被闪电所击,我们通常会认为,这确是意外的事件,受害者不可能有什么潜意识意图,因为谁也说不准闪电将击中什么地方。当我们知道受害者是站在一棵四周空旷的大树底下、离他几尺远的地上又有一根大粗铁链一直挂到树权上的时候,我们不禁要想:在意外发生之前,他是否意识到在那种情况下会有多么的危险。在他康复后,我们得知他完全具备这一方面的知识,可是他“视而不见”,故意去冒险。于是我们可以断言,这个受害者的被闪电所击,是在潜意识里精心策划的。我们说,车祸可能纯系机器故障而造成,与驾车者的潜意识意愿无关,也可能是驾车者潜意识里故意疏忽而造成的。

可能有人会问,难道每一件不幸之事都是由于潜意识的意愿所造成的吗?难道人类就没有不完善的时候呢?比如说,难道除了潜意识愿望外,就没有人会出车祸了吗?

回答是:原则上来说,无一例外,迄今所有被认为是由于“人类不完善”而造成的不幸之事,都是这些行为者潜意识意愿的结果。疲劳、单调乏味等因素确实或多或少地会增加这些不幸的发生,但是就像我们在口误或笔误中所讲的那样:潜意识的意愿是造成不幸之事的必须条件,往往也构成了充分的条件,而疲劳、乏味等只是辅助因素而已。

有人会问,为什么我们这么肯定地认为那些可以由人们自已控制的不幸之事必然是潜意识所造成的呢?答复是:我们的结论,直接来自对于造成不幸者的研究。这里所指的直接研究,也如同对其他过失的研究一样,指的是精神分析技术。如果能取得造成不幸者的合作,由他的联想,可以了解到造成这起乍看起来完全是无意的不幸的潜意识动机。对这些不幸之事进行分析时,当事人往往会想起,在即将发生不幸之事以前,自己已察觉到将会“出事”。显然,只有在他故意这样做的情况下,才可能察觉、才会出事。这种对自己意愿部分地觉察,在不幸之事发生之时或发生之后,通常以遗忘的形式压抑下去,只有在进行分析之时才又重现于意识记忆之中。因而,不经过分析,当事人总是确信纯系无意而引起不幸之事,不知道实际上是自己有意造成的。

精神分析治疗过程,提供了许多直接研究这一类不幸之事的机会,而不是仅仅根据偶然得到的外在的证据去推测。我们的例子绝大部分来源于此,但这并不等于被分析的病人生活中发生不幸之事的机会比其他人多。

一次,某病人在开车去上班的途中,经过一个颇为繁忙的交叉路。由于不少行人穿越马路,他将车速减到每小时5公里。但左转弯时车的左前杠还是撞倒了一个老人。第一次谈起这件不幸之事时,他说当时根本没有看见这个老人。以后他记起来,当他感觉到车子撞着什么东西时,并不以为然。换言之,他已模模糊糊地觉察到自己在“意外”发生时,用前杠去撞人的潜意识意识。根据他对各种情况的联想,发现在这起不中之事中,病人的主要潜意识动机是伤害自己的父亲。虽然他父亲已故去多年,但被压抑下去的幼年时的俄狄浦斯情结依然活跃于他的本我之中,使他产生伤害父亲的欲望。由于初级心理过程的作用,他的这个欲望转移到一个素不相识的过路老人,于是这个老人就成了他看来似乎完全无意的车祸的受害者。尽管老人没有受伤,而且病人自己也有保险,病人仍然为这件小事而感到十分害怕及强烈的内疚。了解到病人撞倒老人的潜意识动机,使我们明白了,是这些动机造成病人的罪恶感和惧怕。换言之,他对车祸的反应,表面上看来不成比例,但对所压抑的杀父欲望来说,则是相符的。

在第\ref{1}章中我们提到过一个例子,事情不大,还够不上不幸之事。说的是一个年轻人在举行婚礼那天早晨开车去他未婚妻家,在绿灯下他把车停下了,直到红灯亮时仍未觉察自己的错误。由他的联想而知,他潜意识里不愿意结婚。在他潜意识里,俄狄浦斯性质的施虐和乱伦的性幻想,使他对结婚产生罪恶及恐惧之感。

这两个例子中,前一例不幸之事是由于不能完全压抑敌意的本我冲动,使本我冲动部分地由压抑中逃逸而出所造成,精神分析著作中经常描述这种情况。后一例的过失,或是防御反抗本我冲动,或是超我对本我冲动的阻抑,或者两者皆有,很难将这两种情况明确地区分开。在这一类过失中,超我的潜意识活动经常起着重要的作用。许多不幸之事,是潜意识里故意使自己受损或受伤的缘故。这些人的动机中,潜意识里需要惩罚、需要牺牲、需要偿还早先的行动或愿望占很大的成分。所有这些动机都是属于超我的。

下面我们举一个例子来说明这种动机。一天,我们第一个例子的那个病人在停车时将汽车的右前轮重重地撞到路边石上,以至于将车胎撕裂。对于一个有经验的司机来说,鲜有发生这种意外。更奇怪的是,事情居然发生在他自家门前的路边石上。他在那里停过不知多少次车,从未发生过这种事情。他的联想提供了解释。几个月前他的祖父病逝,那天上午他去祖父原来的住处看了看,出事前刚刚回来。潜意识里,病人感到自罪,因为他曾愿他祖父早死,并认为他祖父的死与他的愿望有关。这种潜意识的欲望也指向了他的父亲。撞坏自己汽车的车胎,满足了潜意识超我的愿望,是对潜意识幻想中盼望祖父早死的想法的惩罚。

有时,不幸之事包含了犯罪和惩罚两种成分。我们也可以设想,这个人被压抑的、想伤害他父亲的幻想转移为毁坏他的车而得以满足,或象征性地得以满足。这个病人在联想时没有谈到这一方面的内容,我们只得怀疑和猜测了。然而,在别的例子里,犯罪和惩罚两种成分无疑会存在于一个行动之中。

例如,一个女病人驾着她丈夫的车,在路上行驶时突然来个急刹车,结果她后面那辆车撞上了她的车子,将车后杠撞弯了。精神分析所发现的潜意识动机十分复杂。有两种既不相同却又有关联的情况:首先,由于她的丈夫待她不好,使她极为气愤,但又不敢公开地、直接地进行反抗,损坏他的车表现了潜意识里的愤怒;其次,潜意识里她又为自己向他大发脾气感到自罪,把他的车弄坏是得到他的惩罚的一条捷径。事情发生后,她立即就意识到自己已经“身临其境”了;第三,病人性欲很强,她丈夫满足不了她的要求,她只得强压下来。一个男人“撞进了我的屁股”,象征性地满足了她潜意识里性的欲望。

我们不打算把各式各样的过失通通都罗列出来,因为造成它们的原因及发生的机制都一个样,至少十分相似。有意思的是,很难明确地在过失及所谓的正常心理行为之间划条线以区分两者。比如,口误与隐喻大不相同,后者是经过深思熟虑而有意说出来的,但是在人们谈话时,有些隐喻不经思索就脱口而出。对这些自发而生的隐喻,讲话者有时感到高兴,有时感到懊恼,有时则毫不介意,认为“这正是我想说的”。这样,我们发现,虽然区分仔细思索而生的隐喻与口误并不难,可它们之间还有一个中间状态。如何将这种不受欢迎的隐喻(讲话者立刻会说:“哦,不,我不是这个意思。”)与口误区分开呢?拿走路做个比方,假如某人沿着条十分熟悉的路线步行去一个地方,可是他却拐错了弯,走到另一个地方去了,我们称之为过失。然而,他有时也会不假思索地稍稍改变一下路线,仍然走到了目的地,我们能称之为过失吗?这个人只不过没有意识到自己改变了习惯的做法,换换新样而已。此时,如何能在过失与正常行为之间划一条线呢?

事实上,在过失与正常行为之间不存在什么明确的界线。它们之间只是程度上的差异,并没有本质上的不同。由本我、自我及超我的潜意识部分所产生的潜意识动机和冲动,在产生和形成所谓正常心理行为中的作用,并不比产生过失的作用小。在正常心理行为中,自我能够协调各种各样潜意识的影响,从而控制它们,使它们之间互相和谐。也使它们与外界环境达到和谐。这样,就把实际上来自许多不同方面的潜意识影响组成一个单一的协调的整体而进入意识。在过失时,自我不能对潜意识里各种心理力量成功地进行整合,于是这些力量的一种或几种在某种程度上独立地活动并表现出来。自我整合活动愈成功,心理行为愈正常。相反,自我整合活动愈不成功,过失则愈明显。

扼要地说,日常生活中的过失是自我在不同程度上不能将心理活动各种力量整合为一个和谐的整体所致。发生过失时,潜意识心理力量或多或少地与整合对抗。并在不同程度上直接地单独地影响于思想或行为。这种潜意识心理力量有时来自本我,有时来自自我,有时来自超我,有时来自其中两者,有时则三者共同作用。偶尔,观察者只要通过表现出来的现象就可以敏锐地推测出潜意识力量的性质,然而绝大多数情况下,需要该人主动合作,通过精神分析才能发现是什么样的潜意识力量在作用。即使对推测很有把握,也只有通过运用精神分析的方法才能肯定自己的推测正确与否或者全面与否。

下面,我们将讨论诙谐的问题。像过去一样,诙谐是大家日常生活中所熟悉的现象,Freud很早就对它进行了研究。他成功地说明了潜意识心理过程在谐语的形成以及由谐语而带来的乐趣中的性质和重要性。他同时提出一种理论,用以解释“好”的谐语能够通过发笑使心理能力得以释放。

Freud说明了,每一个谐语的产生均是基于初级思维过程的作用,他以一个十分机敏的技术证明之。他用次级心理过程的语言将谐语的内容再讲一遍,结果是,除了使人感到有趣、聪明、难堪、讥讽、不合时宜之外,完全失去其诙谐的特点。让我们拿一个大家所熟悉的诙谐的政治警句为例,它是“自由党人是双脚牢固地站在半空中的人。”乍听起来,似乎听不出初级思维过程在这句话中起着很重要的作用,但是用严格的次级心理过程的语言再说一遍,这个警句则成为“自由党人力图成为坚定和现实的人。但是,实质上都做不到。”变成了批评人的话,完全失去其诙谐的特点。

用次级心理过程的语言重讲警句时,我们立即就会发现,在这个警句的原先形式中,其严肃的含义是以初级思维过程而不是以次级思维过程表达的。原先的形式,以一个牢固地站在半空中的自我标榜为“自由党人”形象来表达次级思维过程的意思。通过类推,人们理解到“双脚牢固地站着的人”意味着“坚定或果断的人”,而“站在空中的人”意味着“不切实际和不果断的人”。加之,警句的原先形式又缺少重述时的解释词及连接词:“力图成为”和“但是,实质上”。在第\ref{3}章中已提到,以近似物来代表,以及省略掉连接词和解释词,以使句子极为简单化,是初级思维过程的特点。

自然,其他谐语也显示了各种别的初级思维过程的特点,如精神分析术语中的转移、缩合、以部分代表整体、以整体代表部分、以相反物代表、象征等。由于诙谐主要是语言现象,人们经常可由分析谐语而看到它是依初级思维过程而用词的。例如,抽取不同词的某个部分,组成一个新词。这个新词具有原来两个词的意思。我们可以认为,这是词的缩合过程。此外,某个词的一部分可用来代表整个词,或用另外一个在声音上或形状上相似,而实际意思完全不同的词来代表某个词。所有初级思维过程的特点都被包括在所谓的“词的游戏”之中。双关语,是我们所最熟悉的用于玩笑的词的游戏。从格言的角度来看,它是一种最低形式的诙谐。尽管它的价值被忽视了,许多精彩的谐语中,还是包含了双关语。

从发展的观点来看,初级思维过程是儿童期的思想形式,年龄增大,则逐渐被次级思维过程所代替。由此观点我们可以说,像诙谐这样的活动,是创造者和听者两者均部分地及暂时地复现了初级思维过程。换言之,自我部分地及暂时地倒退了。诙谐之时,是自我本身引起的退行作用,至少是自我促使了退行作用。Kris(1952)认为,这是自我造成的退行作用,而且是个可以控制的退行作用,不同于各种各样病理的退行作用。后者是不可控制的,并大大地破坏了自我的功效,乃至自我的完整。

总之,我们可以说,谐语的创造者依部分退行作用表现了初级心理过程的观念,而其形象或概念,则用次级心理过程的语言来表达,即用词来表达。同时,听者也暂时地退行到初级思维过程去理解谐语。应当知道的是,这些退行作用是自幼产生的,并没有引起创造者和听者的注意。比如前面举的这个例子,其创造者希望用诙谐的方式来表达自由党人力图成为坚定和现实的人,但是在事实上则做不到的观点。部分退行到初级思维过程后,这个观点就表现为一个双脚牢固地站在空中的人。这个观点用词表达,就成为了谐语。听者和读者通过部分退行作用所产生的初级思维过程来理解创造者的意思。

诙谐形式上的特点就讲到此。Freud用许多例子来说明构成谐语必须有条件,一旦失去了这些条件,就失去了诙谐的特点。然而,Freud还提出,仅仅靠这些形式上的特点,还不一定能构成出色的诙谐。比如,复杂而含有多种含义的双关语可能被许多人简单地由其技巧和形式上的出色而认为诙谐。它们不光“只是双关语”,不仅在形式上极为聪慧,而且在用词上“诙谐”。下列诗句可说明这种观点。
\begin{quote}
    霍尔是个小伙子,万物回春却摔死(died in the fall\footnote{英语fall兼有摔倒及秋天两种意思,诗中用之,一语双关。})。\vspace{-6pt}
    
    春天死人太痛心,他是亡命秋季里(died in the fall)。
\end{quote}

还有一点使诙谐能够成功的因素是,听者应易于乐得起来。每一个讲笑话或讲故事的人都知道,一旦听众真心地笑起来了,几乎任何事情都能引起发笑,甚至那些平时不值得报之一笑的事情,也都能够引起发笑。酒后欣快,可以增加这种效果。反之,如果一个人心情不好,“没有情绪”,多么逗乐的事情都不会使他感到诙谐。

例外的情况是存在的,但没有多大的意义。总而言之,上述形式上的特点是必须的,但仅这些还不足以构成诙谐的充分条件。Freud指出,内容也很重要。从特点上看,诙谐的内容由敌意或性的想法所构成,而当造谐语或听谐语之时,自我或多或少地对这些内容进行抵抗。此处,“性”指的是精神分析的含义,即包括口、肛门、阴茎及生殖器的性欲成分,诙谐的技巧有助于潜意识意愿的释放,而在别的情况下,这些潜意识意愿是不允许表现出来的,至少不允许它们完全地表现出来。

为说明之,我们举一个三十年代非常著名的谐语:“如果所有的姑娘们在Yale大学的舞会上首尾相接,我也不感到奇怪。”这句谐语的内容很明显:“如果所有的姑娘们在Yale大学的舞会上性交,我也不感到奇怪。”但是,对于社交聚会的内容采用后一种直接的描述,会遭到听者心理中超我的谴责。听者由此所想到的是,造这句话的人过于粗俗,以及语言上的下流、不堪入耳,根本体验不到由这句话所引起的性幻想或欲望会带来什么愉快的感情。然而,同样的内容用诙谐的方式表达出来,可以大大避免超我的谴责,并且产生性兴奋的愉快。也就是说,诙谐的技巧可以得到在某些情况下所得不到的性满足。

如果我们回到自由党人的警句,同样可以发现,应用诙谐的技巧,创造这句话的人可以对他所瞧不起的人表现出比直截了当地批评更为轻蔑的效果。在初级思维过程的帮助下,他们似乎自始至终都在恭维自由党人,丝毫没有辱骂他们的意思。而听者被禁止表现出来的那股冲动,此时则得到满足或愉快地得以释放。当然,这是一股敌意的冲动。

释放被禁止表现的冲动,产生了愉快。这股冲动到底是故意的,是性的,还是两者皆有,则要看通过谐语主要得到什么样的体验。真正好的谐语,不光只是聪明,而且要有“观点”。除了诙谐的鉴赏家只注意形式之外,光靠形式出色不能代替其内容或意义。换言之,谐语在技巧上所带来的愉快,远不如从自我防御的压抑下逃逸出来的冲动所带来的愉快大。

我们必须认识到,不论在数量上多么不同,事实上诙谐的愉快来源于两个不同的方面。其一,退行的初级思维过程代替次级思维过程,这是诙谐的必要条件。这种退行而产生的愉快,是抛弃了成人生活中的压抑,回到了儿童一样的行为而得来的。其二,平时被控制或被禁止的冲动释放或逃逸的结果。前一个来源主要起诙谐的作用,而愉快的体验则主要来自后者。

下面几段将用主观上的词语,即依据愉快的体验进行理论上的讨论。Freud在关于诙诸的专著中,进一步说明了伴随诙谐的笑声和愉快,是基于心理能力的释放。他认为,以初级思维过程代替次级思维过程,可以通过笑声将心理能力释放出来。自我防御的暂时解除,使平时被禁止的冲动一下子得以释放,产生大量的心理能力。Freud提出,就是这股子能力,通常被自我用来反抗这些冲动。而在诙谐时,这些冲动突然地暂时地得以自由,在笑声中释放出来。

在结束本章之前,让我们把在诙谐中所谈到的东西,与在过失中所谈到的东西进行一下比较。虽然这两种现象有相似之处,如,两者均为潜意识意愿的暂时流露,均由初级思维过程起主要作用。然而,过失时潜意识意愿的表现,或者是由于自我暂时不能控制潜意识意愿,或者是由于自我不能将潜意识意愿与其他同时活动于心中的心理意愿进行整合成正常的形式而造成的。因而,过失的发生是不顾及自我的。诙谐则是自我暂时及部分地退行到初级思维过程,造成自我防御活动暂时解除,使得潜意识冲动表现出来。因而,诙谐是自我制造出来的,受自我所欢迎。还有一个区别:在过失时,暂时表现出来的潜意识意愿可来自本我、自我或超我;而在诙谐中所表现的潜意识意愿均来自本我。

\signatureA



\chapter{梦}\label{7}

梦的研究在精神分析中占有特殊的地位,1900年Freud发表的《梦的解析》一书对心理学来说,就像早它半个世纪的《物种起源》对生物学一样,具有革命性的划时代的意义。1913年,Freud在A. A. Brill所译的《梦的解析》第三版的序言中写道:“即使按照我目前的看法,本书包括了我有幸发现的最有价值的东西,像这样的洞察力,在一生中只有一次而已。”本世纪初,当他的职业生涯必须与医学界同道们分道扬镳时,对梦的理解大大地帮助了他。在那个困难的时期,他通过医疗实践竭力探索治疗神经症的方法。从他的信件中,我们可以发现,当时他经常失去信心,有时甚至灰心绝望了。然而,梦研究的发现,又鼓舞了他。由此,他知道自己的事业有着坚实的基础,从而给了他继续向前的信心。

Freud把梦的研究放到这样高的地位是完全正确的,因为在日常心理生活中,再也没有别的现象有着这么多表现明显而又易于研究的潜意识心理过程了。梦,确实是通向心理中潜意识领域的捷径,对于精神分析学家来说,它将永远是重要的和有价值的。梦的研究,不仅能了解一般情况下的潜意识心理过程和内容,而且能了解那些被压抑的、被排斥于意识之外的、在自我防御活动时才表现出来的心理过程和内容。由于这些心理过程和内容属于本我,而被排斥于意识之外,存在于病理过程之中,并导致神经症甚至精神病,使得梦的这个特点成了精神分析研究的另一个十分重要的理论。

下面谈谈梦的精神分析理论。睡眠中出现于意识之中的主观体验,醒后睡眠者称之谓梦。这种主观体验,是睡眠时潜意识心理活动的结果。这种潜意识心理活动,从本质上及强度上打扰睡眠,但睡眠者仍然处于睡眠之中而不醒转。睡眠时的意识体验,不论睡眠者在醒后能否回忆,均称为显梦。它的各种组成成分,称为梦的显意。促使睡眠者醒转的潜意识思想及欲望,称为梦的隐意。使梦的隐意变成显梦的潜意识心理过程,称为梦的工作。

将这些区别搞清楚并记在心里,极为重要。做不到这一点,就会对梦的精神分析理论混乱不清,以至于产生误解。严格地说,在精神分析的术语中,“梦”这个词指的是整个现象,而梦的隐意、梦的工作及显梦只是其中几个组成部分而已。实际应用时,在精神分析著作中,“梦”这个词往往指的是“显梦”,如果很好地掌握了精神分析理论,应用起来就不会出现混乱了。比如说“病人做了这样一个梦”,接着就说出显梦的内容,无疑此处“梦”这个词指的是“显梦”。在著作中,在讨论时,还会遇到这样的提法:“梦的含义”。确切地说,这仅仅是指梦的隐意而言,那些对于梦的理论尚不谙练的人,在阅读精神分析著作的时候,每遇到一个“梦”字都要好好地想想作者所指的意思。我们应当准确地掌握这些术语,以免发生误解。

知道了梦的三个组成部分的定义之后,下面将讨论梦的起始过程,即梦的隐意。梦的隐意可分为三个主要范畴。第一个范畴比较明显,包含了夜间感觉的印象,这些印象不断地侵扰睡眠者的感觉器官。有时,其中某些感受印象还形成了梦的部分隐意,加入了梦的起始过程。我们对这一类感觉都很熟悉,如闹铃声、渴、饥、尿意或便意、外伤或疾病引起的疼痛、身体某个部分受压、过热或过冷等,这些均可成为梦的隐意的一部分。在此应记住两个事实:(1)大多数夜间感觉刺激物并不侵扰睡眠,即使它们加入了梦的形成,也不侵扰睡眠。同时,睡眠时由感受器而来的绝大部分冲动,对我们的心理没有什么作用。事实上,甚至我们处于醒觉状态时,那些感觉刺激物只有在相当强烈时才能引起我们的注意。比如,有些人听力完全正常,却能在雷鸣电闪的暴风雨之夜安然熟睡,既不被吵醒,甚至连梦也不做。(2)睡眠时侵扰的感觉印象,可以直接将睡眠者弄醒而一点儿梦也不做。这种情况可见于孩子有病时父母的睡眠。这时,父母虽然睡了,却“留一只耳朵”注意孩子的动静,只要有一点点声音来自孩子那里,他们都会立即醒来。

梦的隐意的第二个范畴,包含了与梦者当时醒觉生活中的活动及先占密切相关的思想和观念,而这些思想和观念仍然在梦者心中潜意识地活动着。由于它们仍然是活动着的,则可使睡眠者醒来,就像侵扰性的感觉刺激物在睡眠时所起的作用一样,如果睡眠者不是醒来而是做梦,这些思想和观念就成为梦的隐意的一部分。这一类的例子多得数不清,包括平时进入自我的所有各种各样的兴趣和记忆,以及伴随它们的任何希望或恐惧、骄傲或羞辱、爱好或厌恶的感情,如回忆前一天晚上的娱乐,担心一件未完成的任务、预测将来的一件高兴事情,或者任何一件睡眠者生活中所关心的所能想到的事情。

梦的隐意的第三个范畴,包含一个或几个本我的冲动。这种冲动(至少在以原始的婴儿的形式表现时)被自我防御排斥于意识之外,并在醒觉生活中不能得以直接的满足。Freud在他的专著中,将这一部分本我称为心理结构中“被压抑的”。后来,他仍然喜欢这样看,但现在的心理分析学家们一般都认为,压抑并不是自我用以抵抗本我冲动进入意识的唯一防御方法。虽然如此,至今仍然沿用最初所应用的术语“被压抑的”来指这一部分本我。明白了这一点,我们可以说,在任何梦中,梦的隐意的第三个范畴都是来源于本我被压抑的冲动。在用以抵抗本我的自我防御中,最重要的而且影响最为深远的部分自儿童生活的前俄狄浦斯期及俄狄浦斯期即已存在,因而早年产生的本我冲动就成为被压抑的主要内容。由此可知,那些来源于被压抑的冲动的梦的隐意具有儿童或婴儿的特点,也就是说,它由儿童早期的欲望所组成。

我们可以说,梦的隐意的第三个范畴不同于前两个范畴。前两者包含的是当时的感受和当时的想法。在儿童期,当时的想法均带有儿童的特点。然而,大孩子和成人生活中,梦的隐意则有两个来源,其一是当时的,其二是过去的。

三部分梦的隐意,到底哪一个比较重要?Freud明确指出,来自被压抑的部分是梦的隐意的主要部分。他认为,哪一部分组成主要的心理能量,哪一部分就是做梦所必须的,若没有它的参加,就不会有梦。像Freud所说的那样,一个夜间感觉刺激物,不论其强度有多强,只有在得到一个或几个被压抑的欲望的帮助,才能导致做梦。梦中的事情,正是睡眠者醒觉生活中所考虑的事情。至于这些事情有多大的力量,则要看睡眠者的注意和兴趣了。

是否在每一个梦的隐意中都可以找到这三个部分呢?前面讲述了,每一个梦的隐意的主要部分都是被压抑的欲望或冲动。同样,梦的隐意中,至少有一部分是来自当时醒觉生活中的事情。至于夜间的感受,虽然在某些梦中起了明显的作用,但并不是每一个梦中都需要它的作用。

现在谈谈梦的隐意和显梦之间的关系,或者更明确地说,谈谈显梦的成分和内容。这种关系依梦而异,可以非常简单,也可以十分复杂,但其间有一个成分是不变的:隐意是潜意识的,而显意是意识的。两者间最简单的关系是,梦的隐意可以变成意识的。

睡眠时,在感受到刺激物的情况下,偶尔会发生这种情况。例如,某人早晨醒后,别人告诉他:夜里他睡着后,有救火车队他房前驶过。他可能会想起来睡眠中听到了火警笛声。这种体验,大凡可被视为介于平时清醒状态下的知觉和典型的梦之间的边缘或过渡的体验,不能称之为真正的梦。当然,也有可能睡眠者在听到警笛声的一刹那醒来了。

为了说明梦的隐意与梦的显意的关系,我们还是应当用真正的梦才行。早期儿童的梦,往往可以作为说明隐意和显意之间关系的最简单的例子,理由是:其一,在这个年龄阶段,不需要去区别什么是做梦者当时所考虑的问题,以及什么是婴儿期所考虑的问题,两者是一回事;其二,这时区分不出什么是被压抑的部分和什么是本我的部分。因为,这么小年纪的儿童的自我,还没有发展到能够抵抗任何本我冲动的程度。

下面举一个两岁小男孩的梦。他妈妈出了医院,抱回家一个刚出生的小弟弟。第二天早晨,他诉说前一天晚上“看见小弟弟走掉了”,这是个显意。什么是这个梦的隐意呢?一般情况下,只有应用精神分析的方法,通过梦者的联想,我们才能知道。两岁的孩子肯定不懂这些,也无法合作。然而,通过这个孩子梦中对刚出生的小弟弟的行为和态度,我们了解到他的敌意和抵制,就像通过联想而了解到梦的显意一样。结论是,梦的隐意是对刚出生的小弟弟的敌意冲动,并且希望小弟弟死掉或滚开。

这个例子中,梦的隐意与梦的显意之间到底有什么关系呢?两者之间有如下之不同。首先,梦的显意是意识的,而梦的隐意是潜意识的;其次,梦的显意是一种现象,而梦的隐意如同欲望或冲动;其三,梦的显意是一种幻想,再现了梦的隐意中欲望或冲动的满足。我们可以说,梦的隐意和梦的显意之间的关系是,显梦是以视象或体验来表现意识的幻想,满足了或正在满足隐意中的欲望。在这个例子中,梦的工作包括了:形成或选择一个满足欲望的幻想,并以视象的形式将它再现。

这些就是儿童早期所有的梦中梦的隐意与梦的显意之间的关系,它也是大一些的儿童以至成人梦中这种关系的基本形式。即使有些梦很复杂,也只不过是下面我们要谈到的那些因素将这种形式弄复杂化罢了。

首先做梦的过程实质上就是在幻想中满足本我冲动的过程,由本我而产生的侵扰性的欲望和冲动,变成了梦的隐意的一部分,并在幻想中得以满足。这样一来,也就在一定程度上失去了它的侵扰性和弄醒睡眠者的能力。这就是为什么睡眠者不被这种侵扰性的潜意识心理活动弄醒,而在梦中继续睡眠的缘故。

我们知道,显梦通常是欲望的满足,是由梦的隐意的性质所决定的。梦的隐意,就像它是心理能力的主要来源一样,是梦的发动者。在梦的隐意中起这个作用的本我成分来源于本能的内驱力,因而只能不断地得到满足才行。在睡眠的状态,身体活动受限,不可能通过适当的行为使它得到充分地满足,因而只能以幻想的形式来代替,得以部分地满足。从心理能量的角度看,在梦的隐意中,本我的力量促使心理结构进行梦的工作,并通过满足欲望的幻想性意象,即显梦,而得以部分地释放。

在大一些的儿童及成人梦中的大多数显意,不可能一下子就被认识到是欲望的满足。在过去的五十年间,有些人以某些在显意上属于悲伤的甚至是惊吓的梦,来反驳Freud所断言的:每个显梦都是幻想性欲望的满足。我们如何理解这个理论与显而易见的事实之间的不一致呢?

回答很简单。前面说过,儿童早期的梦中,梦的隐意通过梦的工作变成了显梦,以幻想的形式来满足构成隐意旳冲动或欲望。幻想是以梦者感觉印象的形式来体验的。梦的隐意与梦的显意之间这种明显的关系。也时常在日后生活中的梦里发现。这些梦与童早期简单的梦非常相似。然而,在日后生活中,梦的显意更多地是以伪装和变形的形式,来表现满足欲望的幻想,主要体验为视象或视象系列。伪装和变形往往太大了,以至于完全认不出显梦欲望满足的样子。就像大家知道的那样,显梦常常表现为亳无相关的片断的大杂烩,似乎毫无意义,一点儿也显不出来是欲望的满足。有时,伪装和变形会大到使显梦产生惊恐的体验和不受欢迎的地步,从而感受不到幻想中欲望的满足所产生的那种愉快。

伪装和变形是由梦的工作造成的,并在大一些的儿童及成人的显梦中十分突出。那么,梦的工作到底都有哪些过程,而每一个过程又如何去伪装梦的隐意,使得它在显梦中难以识别的呢?

Freud指出,有两个主要因素及一个辅助因素与梦的工作有关,第一个主要因素是梦的工作的本质,也就是将梦的隐意中还没有用初级心理过程的语言表达的部分,转化成初级心理过程的语言,将梦的隐意的所有成分通过缩合变成满足欲的幻想。第二个主要因素由自我防御作用组成。自我防御深深地影响着转化和幻想形成的过程。Freud将它比作为新闻检查官,有权将要不得的内容删掉。第三个是辅助因素。Freud称之为后期修饰。下面将一个个地讨论这些因素。

首先,如同我们已经讲过的那样,梦的工作包括将梦的隐意中原来应该按次级思维过程表达的部分转化为初级思维过程。其中包括了那些当时生活中的事情及兴趣。Freud指出,这种转化是按一定方式进行的。使转化的结果以视象的形式表现出来。当然,这种形式的表现,就是梦的显意的主要意象。与其相似的表现形式,可以意识地存在于某些正常醒觉活动中,如字谜、创作动画和画谜。

在梦的工作中,影响这个转化过程的还有隐梦成分的性质。而这种隐梦成分已经变成初级语言过程了。从本质来看,是与被压抑的欲望或冲动密切相关的记忆、意象和幻想。换言之,梦的工作是将当时醒觉生活中的事情,转化为与被压抑的事情密切相关的词语或意象。不过,在满足被压抑冲动的许多幻想中,梦的工作只选择最容易被带入当时醒觉生活的幻想。梦的工作。以最近似的方式,将那些必须的梦的隐意转化为初级语言过程、同时创造或选择一个再现满足被压抑冲动的幻想。这些冲动当然也是梦的隐意的一个组成部分,像我们前面所说的,所有这些都要与视觉再现性有关。梦的工作以最近似的过程进行转化,使得一个单一的意象同时能再现几个隐梦的成分、其结果则形成高度的“缩合”。也就是说。绝大部分的显梦是构成梦的隐意的思想、感觉、欲望等高度缩合的形式。

在讨论自我防御在梦的工作中所起的作用之前、我们要问,是否梦的工作参与显梦的所有伪装及变形?如果是的话,它起了多大的作用?

我们知道,将醒觉生活中的事情用初级心理过程的语言表现、将会使其意思及内容大为变形。有人可能会问、为什么心理机制要把这些事搞得使梦者那么难认呢?因为,尽管动画、字谜、画谜是以初级心理讨程的语必、仅创作人能理解这些形象的意思、事实上,许多别的人也能掌握其意思。同样、还有一些以初级心理过程的语言表现的观点,也能被我们所了解,就像在第\ref{6}章中所说的谐语一样。难道仅仅是由于显梦所包含的现。点是以初级心理过程表现的,于使得显梦晦涩难懂吗?

回答是:读谐、动画、画谜。甚至字谜,都是为了特殊的需要而创造的,必须使别人知道它的意思、因而明显易俺。然而,对于显梦则无此要求。它的最终目的是欲望的满足,或者说,使与梦的隐意有关的心理能量得以充分的释放。为了不使这些内容舜醒睡眠者,一般不可能立刻了解到显梦的意思,甚至睡眠者本人也不了解。

参加梦的工作的第。二。个主要因素,在伪装梦的隐意以及使显梦晦涩难懂上,起了更为重要的作用,这个因素也就是自我防御的作用。应该注意vFreud在描述这个因素之后很久,才提出了心理结构的假说,才用到诸如“自我”、“防御”等术语。因此、他选用了“梦的检查”一词来描述这个因素。

为了清楚地了解自我防御在形成显梦过程中的作用、我们酊先必须认识到,它在不同程度上影响着梦的隐意的不同部分。梦的隐意中,夜间感受的那一部分,一般没有自我防御的作用。除非自我为『睡眠而全部否认这些感受。但是,我们还不能确认,睡眠者对于夜间感受的态度,就是通常所指旳H我防御、进而不在此问题上过多地讨论。

与夜问的感受不司、包含在梦的隐。意中的被压抑的欲望和冲动。叫直接地被自我防御所l反衬:这种迂对是经年累月水恒不变的,故而被称对“被压抑的”、在醒觉生活中、自我防御始终反对这些I压抑的欲望和冲动在意识里出现。由此不难理解,为什么自我防御不让这一部分梦的隐怠在意识的显梦里出现了。自我防御反抗这一部分梦的隐意、结果造成显梦往往难以理解,并且完全认不出是个为满足欲望的幺想的意象。

梦的隐意中,其余的部分就是当时醒觉生活中的事情。它与自我防御有关。许多醒觉生活中的事情,除了可能干扰睡眠外,与自我无何冲突,而且某些事情,在自我看来龙愉快的。但是,仍然有一些当时生活中的事情,对自我来说是不愉快的。引起焦虑和罪恶感。睡眠时,自我防御机制力图阻止这些不愉快的事情进入意识。在第\ref{4}章中已经讲到,通常是由于不愉快或不愉快的前景而造成自我防御活动的。[因而,自我在潜意识里反对隐梦成分的程度,依与隐梦成分相联系的焦虑或罪恶感的程度而定,也就是依其不愉快的强度而定。

由此可见、自我防御不但强烈地字对梦的隐意中来自被压抑的那部分进入意识,同时也反对醒觉生活中属于隐意的那些事情进入。意识。事实上,我们称之为梦的隐意的潜意识的思想、努力和感受总是会强行进入意识,而表现为显梦的。虽然自我抵抗不了这些情况,它却能影响梦的工作,使显梦变得晦涩难懂、无法辨认。因此。显梦之所以无法理解、并不简单地由于它是以无法理解的初级心理过程的语言来表示的缘故,主要是由于自我防御使它变成为那个样子。

Freud(1933)将显梦称为“妥协形成”。意思是。它的各种成分可被视为,梦的隐意的力量为一方与自我防御为另一方之间的妥协。就像在第\ref{8}章中将说到的那样,神经症的症状也是如同被压抑成分与自我防御之间的妥协形成。

下面这个简单的例子能够说明之。假设梦者是个女人,来自被压抑的那一部分梦的隐窟是俄狄浦斯期的欲望,即与她父亲发生性关系的欲望。根据她当时生活的幻想、在显梦中可能表现为梦者跟她的父亲打架,同时伴有性兴奋的感觉。如果自我防御不让不加伪装的俄狄浦斯欲望表现出来,性兴奋就不可能进入意识。结果v其显梦的成分仅表现为与她父亲打架,而不含有性兴奋。如果这还是太接近于初级幻想,以致引起焦虑和内疚,而为自我所不容,父亲的形象可能不出现,代之以梦者与别的什么人打架,比如说与她的儿子打架。如果打架还是太接近于初级幻想,就可能代之以别种形式的身体活动、如跳舞。于是,显梦的成分就是梦者与她的儿子一起跳舞。即使这样,也可能遭到自我的反对。于是上述的显梦成分可能变成为,一个陌生女人与她的儿子呆在一个地板光潸的房间里。

对梦的隐意的真正性质进行伪装的例子。多得数不清,无法在此一一枚举。实际上是防御强度与隐梦成分之间的平衡关系,决定了显梦与隐梦之间的远近。也可以说,决定了梦的工作对隐梦成分做了多少伪装。从上面这个例子也可看出每一个显梦的总象都可被分开,并在适当的情况下在某一个梦里表现出来。比如。在一个梦里首先试试显意“A”。假如自我不能容忍“A”,则以“B”代替之。锻如自我仍不能容忍“B”、则以“C”代替之,如此依次进行。按照防御和隐梦成分之间力量的平衡关系,在显梦中出现的或是“A”、或是“B”、或屡“C”、或是\ldots\ldots

防御和隐意之间的“妥协形成”是个十分复杂的问题,不可能在本章里详细叙述。但其中重要的或典型的几点,还是要提一提的。其一是,属于隐意的事情可以表现子显意的各个部分。比如上面所举的例子里,梦者见到她自已在显梦中的一个部分跟人打架,则她的父亲则在完全不同的另一个部分。这种联接上的分裂现象通常是梦的E作所造成的。

另一个普遍的“妥协”现象是:部分显梦甚或全部显梦表现得模糊不清。Freud指出,这说明防御对隐梦成分反对的力量是很大的。虽然防御的力量不足以整个地阻止显梦进入意识,但是这种防御力蛩还是能够只让其中的一部分进入意识或模模糊糊地表现在意识之中。

从属于梦的隐意的各种情感或情绪的变化、也由梦的工作所决定。我们已经说过,上面举的那个例子里性兴奋的情绪、在显梦中可以根本不出现。另外,这种情绪也可能表现得非常轻,或者表现为别的情绪。例如。隐意中的狂怒,在显梦中可以表现为生气,也可以表现为轻度的不喜欢,或者可能一点儿也觉不出生气。包含于梦的隐意中的情感,在显梦中甚至会以相反的情感方戊表现出来,如盼望表现为厌恶、恨表现为爱、悲伤表现为高兴。Freud认为,这些变化说明了自我与隐意之间的“妥协”,使得许许多多伪装成分进入显梦之中。

讨论梦中的情感,必须包括焦虑的情感,本章汗头已经讲到,某些Freud的批评者们力图反驳他所提出的每一个显梦都是欲望的满足的论点。他们的依据是,有种梦的显意是以焦虑为主线。心理分析著作中、通常称之谓焦虑梦。在非精神分析著作中,这种梦当中最严重者称为噩梦。Jones(1931)对此进行过广泛地精神分析研究。一般说来,焦虑梦表示自我防御作用的失败,表示梦的隐意的成分冲破自我防御而闯入意识,即进入梦的显意。由于它的形式过于直接,非常容易被认识,使得自我无法忍受。故而产生焦虑反应。就像琼斯指出的那样,俄狄浦斯幻想不带什么伪装地表现在噩梦的显意中。确实,在这一类梦中,常可见到性满足和恐惧一同出现。

还有一种与焦虑梦密切相关的梦,常被称为惩罚梦。这些梦中、如果来自被压抑的那一部分隐意在显梦中表现得过于直接,不但自我会感到罪恶,超我也要谴责之。在大多数梦中,自我防御会阻止这一部分隐意出现。然而,在所谓的惩罚梦中,显梦表现的不是伪装的被压抑欲望的幻想,附是伪装的惩罚欲望的幻想。这是一种自我、本我及超我之间非常特殊的“妥协”。

我们讲过,梦中被压抑的潜意识欲望和冲动,通过或多或少地伪装,组成了显梦这个满足欲望的幻想窟象而进入意识。但是、从定义上来看,被压抑的冲动是不可能育这些活动的。也就是说,按照规定,“被压抑的”本我冲动及与其直接相联的幻想、记忆等,均被自我防御永久地阻遏而无法直接地接近意识,既然如此,这些被压抑的又是如何在梦中进入意识的呢?

Freud在睡眠的心理学中,对这个问题已有答复。睡眠时,身体活动受限,自我防御的强度明显减弱、就像自我在说:“我不要为那些令人讨厌的冲动而担忧。只要我躺在床上睡觉,它们什么也不干成。"可是Freud又认为,睡眠时被压抑冲动的内驱力,也就是被压抑冲动侵入意识的强度并不减弱。这样,睡眠时在防御减弱的情况下,被压抑的冲动就比醒觉时更有机会进入意识。

我们必须认识到,睡眠与醒觉生活之间只是程度上的差异,而不是性质上的不同。虽然,睡眠时被压抑的成分较之醒觉生活时更容易进入意识,但睡眠时自我防御通过梦的工作能使被压抑的成分大大地伪装和变形、当它们进入意识时已面目皆非。相反、在某些情况下,被压抑的成分却会以较直接的方式进入醒觉生活的意识中。比如第\ref{6}章里所举的那个开车在繁华的路口“无意地”撞倒老人的例子,说明了即使在醒觉生活时,被玉抑的俄狄浦斯冲动还是可能暂时地控制一个人的行为,以较为直接的方式表现出来。用以说明这种观点的现象还多得很,由此,我们不能认为睡眠和醒觉生活是截然不同的。然而,总的来说,被压抑的成分表现在显梦中还是比表现在醒觉生活的思想或行为中更为直接。

参加梦的工作,除了上述两个主要因素外,还有一个过程。这个过程使得显梦最终形成,并使之晦涩难认,故而被认为是梦的工作的最终阶段。Freud将此过程称为后期修饰。这时、自我力图使梦的显意变得符合逻辑、紧凑一致,变得“合乎情理”、就像自我使任何印象在其领域里都“讲得通”一样。

下面从纯描述的角度谈谈显梦的特点。显梦主要由视觉印象所组成。仅次于视觉体验的是听觉。然而,别的感觉也可以作为显梦的部分。成人之后,常町有思想或片断的思想出现于显梦之中。例如,一个梦者说:“我看见了一个长胡子的人,并且知道他是去拜访我的个朋友。”当这种思想出现在显梦中时。大多并不从属于感觉的印象。

人家从自己的体验里可以知道,当我们睡着后。是完全相信显梦的感受印象的。这些感觉印象,就像我们醒时的感知觉那样逼真。由此看来。显梦的这些成分可以比作严重精神疾病时所出现的幻觉。Freud曾经将梦比做过性的精神病,虽然梦本身不是病理现象。因此,梦的工作的结果表现为显梦,即使显梦是出现在睡眠中的正常现象,从本质上来看它是个幻觉。

Freud从1900年初建梦的心理学起,就用第\ref{2}章提到的心理结构分域(topographic)理论来解释显梦。依此理论,心理释放的正常过程是从心理结构的知觉末梢到运动末梢。在运动末梢,心理能力丧现为行动而释放出来。无疑,这是一个反射弧,即神经冲动的过程,从感觉器官开始一经由中枢神经元,沿运动通路而出。Freud指出,睡眠时由于运动通路的释放受阻,梦的心理能力则必须采取逆行的方式通过心理结构,其结果是,在心理能力释放的过程中,心理结构的感觉末梢被激活,在意识中产生了感觉意象,如同知觉系统被外界刺激物激活一样。Freud最初以此理由,解释了显梦中表现的感觉意象为什么对梦者如此之通真的道理。

按照目前精神分析关于心理结构的理论,即所谓结构假说。我们对显梦本质上就是幻觉的解释如下。睡眠时,许多自我功能或多或少地停止活动了。比如由我们以前举的许多例子里可以见到,睡眠时自我防御减弱了,自主的运动也几乎完全停止了。更重要的是,我们认为,自我有一种检验现实的功能,能够分辨刺激物是由内而生的,抑或是自外而来的。睡眠时,自我检验现实的功能受到明显的破坏。加之,睡眠时自我功能退化到早年生活的水平。如,以初级心理过程的形式思考而不是以次级心理过程的形式思考,而这种思考大多数不是以语言的形式表现,而是以视觉意象为主的感觉意象的形式表现。自我检验现实功能的丧失也可能是睡眠时自我明显退行的结果。总而言之,睡眠时的思考多不用语言的形式。而大多数是以视觉意象的形式思考,同时自我又不能区分这些意象是来自内部还是来自外部的刺激物。由于这些因素作用的结果,因而我们相信,显梦本质上是视幻觉。

下面举一个很容易观察到的事实,作为支持结构假说的解释,及不支持分域假说的根据。在许多梦中,检验现实的能力并不完全丧失。梦者在梦中能在一定程度上觉察到他所体验到的不是真实情境、而“只是一个梦”。这种检验现实功能的部分保存,很难用分域假说来解释,而与结构假说完全吻合。

我们已经讨论了梦的三个部分:隐意、梦的工作和显意;讨论了梦的工作是如何进行的;以及影响梦的工作因素。在实践时,为了研究某个人的梦,必须通过显梦而去确定其隐意。了解到隐意之后,我们才能说是完成了梦的解释或发现了梦的含义。

释梦仅限于精神分析治疗之时。因为,为了达到此目的,通常需要运用精神分析的技术。我们不在这里讨论梦的解释,因为事实上它只是个技术过程,是精神分析实践的一部分,而不是精神分析理论。

\signatureA



\chapter{病理心理学}\label{8}

近60年来、关于精神障碍的精神分析理论,也像内驱力和心理结构的理论一样,巴发生了变化和发展。本章将概要地介绍其发展状况,并以目前的观点对精神障碍的精神分析基本理论进行一般性的讨论。

Freud第次开始治疗患有精神疾病的病人时,精神病学也是处于早期阶段。早发性痴呆这个诊断术语一刚刚见于精神病学的著作;神经衰弱这个时兴的标签,贴到大多数今天称之为神经症的病人身上↓夏科(Charcot)运用催眠疗法成功地消除或减轻了癔症的症状;神经结构的病理变化被认为是精神疾病的基本原因。而这些精神疾病又是由于不近人情的劳累,生活节奏的紧张,也就是说由于工业化的都市生活所致。

第\ref{1}章中提到,Freud最初注意到的是癔症。当时在Breuer的建议下,Freud采用种被称为疏泄法的改良了的催眠疗法,治疗了几种癔症。根据病人们的体验,他认为癔症的症状,是由于潜意识里想起了带有强烈情绪的事情所致,而在那件事情发生的当时,这种情绪不能得以适当的表现或发泄。Freud关丁癔症的最初理论是:癔症的症状足心理创伤的结果,这种病人可能有先天性或遗传性的神经病态,他在1906年说。这是一种纯心理学的病困理论。

在此之前,他从治疗被诊断为神经衰弱的另一组精神疾病病人的经验里、得出了病因学上完全不同的理论,认为神经衰羁只能是不卫生的性活动的结果(Freud, 1895)。按照Freud的观点,这些活动分两种、各自产生不同的症状群。手淫过度或梦遗属于性活动不正常的那一种,产生疲劳、萎准、腹胀、便秘、头痛和消化不良。Freud认为、“神经衰弱”仅限于这一组病人。第一种不卫生的性活动,虽能产生性兴奋状态。却得不到适当的发泄,如性交中断或得不到性满足的性行为。这一类性活动造成焦虑状态,多数以焦虑发作的形式表现。Freud将这一类病人诊断为焦虑性神经症。直到1906年,他还认为神经衰弱和焦虑性神经症的症状,是性代谢紊乱而产生的躯体反应。这种生化紊乱如同甲状腺毒症和肾上腺皮质缺乏症。为了强调它们的特殊性,他将神经衰弱和焦虑性神经症归并,称为现实神经症,而将癔症和强迫症称为精神神经症、

由此可见,Freud所提出的分类是基于病因学而不是简单地基于症状学。他特别提到、有典型的症状,又们手淫或梦遗病史的病人才能被诊断为神经衰弱(Freud, 1899)。假如没有这样的病史,它们必定是由别的原因所致,如麻痹性痴呆(梅毒性脑膜脑炎)、癔症。应该强调,甚至在今天,精神病学中对那些不是由于中枢神经系统疾病或损伤所造成的精神障碍、通常仍是依症状学而分类。但是,在精神病学以及医学的别的分支中,依据症状描述进行疾病或障碍的分类已经价值不大了。因为,治疗适当与否主要依据于症状的原因而不是症状本身。况且,同样的症状在两个不同的病人身上,可以有完全不同的原因。了解Freud早期对于精神疾病病人的工作还是很有用处的。他力图摆脱纯描述性的分类法,把有共同病因的,或者说有共同发病心理机制的精神障碍归为一类。直至今口,精神分析关于精神障碍的理论v还是注重于病因学和病理心理学,而不仅仅是描述性的症状学。

大约自1900年起,Freud在临床上的主要兴趣就集中在他称之为神经症的精神障碍上,而不再研究所谓的现实神经症了。然而,1926年他在有关焦虑的专著中,再次主张分类中应当有焦虑性神经症(他没说神经衰弱),并认为它是由于性兴奋没有得到适当满足而造成。他不再主张焦虑性神经症实质上是生化内分泌紊乱了,而认为焦虑的出现纯粹是心理机制的作用。他设想,在性高潮时内驱力应该得到释放。假如此时内驱力不能得到释放,将产生一种心理紧张状态。这种紧张状态如此之强,使得自我无法控制,于是就自动的产生了焦虑。这些情况,我们在第\ref{4}章已经讲过。

很难说当前的精神分析学家们对神经衰弱和焦虑性神经症是怎样看的。Fenichel(1945)在他临床精神分析标准教科书中、将它们作为真E的实体而讨论。但在精神分析的期刊中,罕有涉及这方面的文章,而且除了Freud最初描述的以外,再也没有报道过这种病例。这似乎表明,现实神经症在精神分析的疾病分类中已不再是那么重要了。

精神神经症是精神分析研究的主要对象。Freud关于这种精神障碍的早期理论,30多年来不断地发展和修订。这些理论上的改变,来自于对病人进行精神分析治疗实践的摸索,使病理心理学得以不断地充实。

最初的那些年(1894-1896)研究进展得十分迅速。首先。是认识到心理冲突对产生精神神经症症状的重要性。Freud在与Breuer共事时就认为,癔症症状和强迫症状都是由于被忘记的过去的事情所引起的问题在于,发生这些事情的当时,所伴发的情绪没有得到适当地发泄。在进一步观察研究的基础上,他很快地又补充说,任何成为致病的心理事件或经历都必须是该人的自我所厌恶的,而且必须强到足以使自我力图避开它或防御它的程度。必须认识到,Freud在此时所用的“自我”和“防御”,虽然与他30年代在心理结构假说中所用的是同一个词、可是其含义却大不相同。此时、“自我”指的是意识的自我,特别是指意识的自我的伦理和道德标准;“防御”指的是意识的拒绝,而不是我们在第\ref{6}章讨论的那样,有着特殊的含义。

Freud认为,这能够很好地解释癔症、强迫症,以及“许多恐怖症”的病人。他将这些症例归为一组“防御性神经精神病”。Freud力图在病因学基础上建立分类系统,而不是仅仅依据病态精神症状的描述而分类,由此可见一斑。Freud当时认为、某些恐怖症(如旷野恐怖)以及某些强追症(如强迫性怀疑)是焦虑性神经症的症状,是由于性兴奋得不到适当地发泄而造成的,同时也造成身体里的性代谢紊乱,而不是像防御一个厌恶的经历那样,纯粹是心理机制的作用。

Freud关于神经症的病理心理学的下一个说明来源于他的经验。他研究了遗忘了的致病事件、发现均可追溯到病人儿童期一个与性生活有关的事情。于是乎他提出,精神疾病是儿童期被某一个成人或某一个大孩子性诱惑的心理结果。基于他的经验,他进一步提出,如果病人在儿童期致病性的(或者称为创伤性的)性经历中处于主动的角色,日后的症状则表现为癔症性的。好莱坞、百老汇以及那些畅销书的作者们,根据儿童期特殊的心理创伤事件通常是造成日后生活中神经症症状的原因这个理论做了许多文章,但必须明白,遵照一些公共道德监察者的意志,这些作品中通常不允许描述合乎上述理论的创伤性的性经历,

Freud从来也没有放弃,任何神经症均是源于儿童期性生活障得的观点。在今天看来,这个概念仍不失为精神分析关于神经症理论的基础。然而。Freud很快发现,病人所谈到的儿童期的性诱惑,即使病人信以为真,事实上有许多是幻想出来的,面不是真实情况。这个发现对Freud来说是个极其沉重的打击。他怨恨自己那么轻易地被病人所欺骗。在绝望和羞愧之下,他几乎打算放弃精神分析的研究,重新回到受人尊重的医学界。然而,他生活中最大成就之一,是他能够很快地从绝望中走了出来,用新的知识重新检查自己的材料,继续精神分析的研究,并且迈出了一大步——认识到、决不仅仅限于像儿童期性诱惑那样的创伤性的事情,甚至从婴儿的最早期起、性的兴趣及活动就己成为人类正常心理生活的个部分(Freud, 1905)。一句话,他建立了婴儿性欲的理论。我们在第\ref{2}章中已讨论过这个问题。

根据这个发现,那些纯属意外的创伤性经历﹑在神经症的病因学中就显得不那么重要了;而在致病因素中、病人性的素质和遗传则显得更为突出。Freud没想,事实上。神经症是由素质和经历两种因素引起的。某些病人以素质因素为主,而某些病人则以经历因素为主(Freud, 1906)。他一生中均持此观点。一般说来,这个观。点也为当今的精神分析学家们所接受。我们说,虽然自 1906年以来精神分析的观察已经大大地充实了经历因素的知识,但是这些观察从本质上来说并没有增加我们对素质因素的知识。Fries(1953)对儿童发育的研究,旨在阐明这个素质因素的性质,但是依然是处于探索的阶段。

发现了婴儿性欲是个正常的现象、也导致产生一些新的有趣的概念。其一,缩短了正常人与神经症病人之间的距离。其二,说明了性变态的原因,以及性变态对正常人及精神神经症病人的关系。

Freud认为,正常人发育过程中,某些婴儿性欲成分(我们在第\ref{2}章中讨论过)被压抑了,剩下的在青春期被整含成为成人性欲。这样、他们就可以认识到在性兴奋和性满足中的作用,可还是认识不到继发于生殖器官的性兴奋和性满足作用,这些般如接吻、窥视、抚摸、闻嗅等。在神经病人的发育过程中,压抑的过程太过分了。这种过分的压抑造成了一种不稳定状态。在生活中、由于某些事情的诱发。压抑不住了,这种婴儿性冲动就逃逸出来,至少部分地逃逸出来,造成了神经症的症状。性变态者的发育过程中,婴儿性欲的某些成分不正常地持续到了成人,如露阴或肛欲。结果,性变态成人的性生活被特沫的婴儿性欲成分所支配,不能表现为正常的生殖欲望。

应当注意到两点,第一,压抑是正常心理发育的特点,也是异常心理发育的特点。我们在第\ref{4}章中已反复地讲述了这个观点、不但要考虑到压抑,也要考虑到自我防御机制。第二,被压抑的冲动逃出压抑而造成神经症,与我们在第\ref{6}章中讨论的睡眠时被压抑的冲动逃出自我防御而产生显梦非常相似。

Freud清楚地看到这个相似性、并据此而提出,神经症的症状如同显梦,是被压抑的冲动与反对这些冲动进入意识思想及行为的人格力量之间的妥协形成。只有一个不同,梦中隐胾的本能欲望可能是。也可能不是性欲的欲望;而导致神经症症状的被压抑的冲动则都是性欲的冲动。

Freud也提出,神经症的症状像显梦的成分一样、有其含义、有其隐藏的或潜意识内容。可以把这些症状看成是伪装和变形的潜意识性幻想,即神经症病人部分或全部的性生活被表现于他的症状之中。

到此为止,我们已经追踪到1906年Freud关于精神障碍的观点。这显示了他的天才,显示了他所创造的并作为研究技巧的精神分析法的丰硕成果。我们今天所阐述的主要内容都已包括在他当时的理论之中了、有的充分地发展了,有的尚处于萌芽阶段。我们已经知道,弗洛伊筏开始对精神疾病进行研究时,精神病学界所流行的看法是:精神障碍是与正常心理功能毫不相干的心理i的疾病;分类上,以描述性的症状为基础;病因上,或者直率地推说不清楚,或者归咎于模糊而笼统的因素,如紧张的现代生活、心理紧张、疲劳、神经结构病理改变等等。1906年他对许多精神障碍成因的心理过程的了解,使他有充分的理由在病理心理学的基础上进行精神障碍的分类,而不是依症状学而分类。而且,他认识到,正常人与神经症病人之间并不存在不可逾越的鸿沟,他们之间在心理学上,只是址的差异而非质的不同,他开始从心理学的角度去了解人格上的障碍,如性变态,并且认识到这些心理障碍也与正常情况有联系,并不是明显的质的不同。

Freud 1906年以后的研究。补充和修订了他以前关于精神障碍病理心理学的理论,但是基本原则并没有改变。

当今的精神分析学家们。仍然是注重于症状的心理学原因而不屉症状本身;仍然以本能力量与反本能力量之间的心理冲突来考虑这些原因;仍然认为人类从正常的到病理的心理功能和行为,就像白炽光谱从红到紫一样,邻近两色间没有明显的界线。我们知道,每一个正常人都有一些Freud称之为神经症的冲突和症状。心理“E常”只是指数量上相对的正常而已,具有主观性。精押分析学家们仍然认为、婴儿和儿童期的事件和经历直接造成日后生活中的精神障碍,至少参与精神障碍的形成。

依照现代精神分析理论,最好是将临床上的精神障碍,理解和解释为心理结构不同程度和不同形式的功能障碍。如果我们能运用发生或发展的方法,则再好不过了。

从第\ref{2}章到第\ref{5}章谈到的内容里可以清楚地了解到,儿童早年各种心理结构的功能均处于发展过程之中、非常容易发生障碍。例如,1周岁以内的婴儿如果在身体上得不到母亲形象的接触和刺激,许多自我的功能将得不到适当地发展,与外界环境联系的能力以及对付外界环境的能力可能被损坏到愚鲁的程度(Spitz, 1945)。甚至在1岁以后,由于挫折过多或过分放纵,使必须的自我功能得不到发展,尤其是必须的认同得不到发展,结果使自我不能很好地在本我与环境之间斡旋。也就是说,自我既不能控制和抵消内驱力,又不能寻找机会以满足愉快。

从内驱力的观点看,困难在于必须控制内驱力,却又不能控制得过分。一个人的内驱力控制得不够、就不能跟大家样、成为社会的一员。但是,内驱力过分地被压抑,则会以不良的方式表现。如果性的驱力被压抑得过分·儿其是如果过早地发生这种情i况的话,则会严重地破坏一个人的欣赀能力。如果攻击的驱力被过分地控制,一个人将难以跟他的伙伴们进行正常的竞争。同时,由于不能对别人攻击,就经常转而攻击自己、长现为或多或少地公开的自毁。

超我的形成过程也可能发生异常。也就是说,旨在结束俄狄浦斯期的复杂的心理变革也可能出现某些失误,造成超我过分严厉,过分宽容,或时而过分严厉时而又过分宽容。

事实上,内驱力惇自我和超我都有可能发生问题。概括地说,如果内驱力控制得不够,自然是意味着自我和超我的功能缺乏;相反,如果内驱力被控制得过分,则可以推想到自我太强了,以及超我太严厉了。

第\ref{3}章中提到,许多自我的兴趣,也就是说许多作为内驱力出路和快乐来源的活动、是基于认同而被选择的。然而在选择一种特殊的活动中好外还有一种因素有时会比认同更重要。在这些情况下、主要是由本能的冲突来决定选孝。例如一个孩子对造型和绘画旳兴趣,可能由一种特殊的希望被粪便弄脏的冲突所决定,而不是由想跟画家认同的需要或欲望而决定。同样,科学上的好奇心、可能来自于早年对性的强烈的好奇所致。

我们认为,上述的两个例子对个人的发展是有利的,是本能冲突升华作用的结果。我们在第\ref{4}章已讨论过这个问题。然而,本能冲宾更多的是被自我以限制或抑制的方法来解汕,至少,使之处于静止状态,而不是像升华作用那样使它更加发展。举一个简单的例子、一个在别的方面都很聪慧的孩子就是学不了算术。因为他哥哥在这一方面特别有天赋。这种自我欺骗式的智力活动抑制,使他避免了由于妒忌而与哥哥竞争所带来的感情上的痛苦。

自我兴趣或活动的这种限制,对一个人的生活可能没有什么影响,也可能极为有害。例如、有些人潜意识地避开终生事业的成就,坚决得就像前面举例的那个孩子避开算术一样,而且从本质上来看抱有同样的目的,也就是说坚决地限制了能够引起极不愉快的本能冲突。自我严格的限制,往往也就满足了超我惩罚或忏悔的要求。还要把问题看得复杂些、并不是所有对来自本能冲突的自我限制都能使孩子与环境产生矛盾、犹如不会做算术的那个孩子那样。又如、一个小孩子表现得听话守规矩,很可能是属于自我欺骗性质的行为、目的是想得到别人的疼爱、而实际上他并不愿继续忍乏强烈冲突所带来的不愉快。这种情况对孩子到底是有益还是有害呢:这跟“正常的”优良行为又如何区别?

或发生于自我,或发生于本我,或发生在两者的退行作用和固着作用,也有同样的问题。例如,某个很特殊的人。只有将他的本能生活退行到肛欲的水平。才能解决俄狄浦斯情结。也就是说、在他日后的生活中,将对他自己的肛『门过程和产物保持异常强烈的兴趣、喜欢收集和收藏他所能得到的东西。第\ref{2}章中谈到,这些本能的退行通常退到早先的一个固着点,而且、实际上固着也帮助了退行。这个例子说明了此人的肛恋是退行的作用。同时,这个例子也可用来说明固着的作用。从本质上来看其结果是一样的。将这个例子用子自我的范畴,可以看出,作为俄狄浦斯冲突的结果,在自我与客体之间的关系上也有部分的退行,于是他杷周围环境中的客体都看成重要的了。他所收集的这些东西仅仅起到满足他的欲望的作用,没有一件东西能使他长久地投注。这个例子也像我们第一个例子一样,说明同样的情况,在另外一个人就可能是固着的结果而不是退行的结果。

自我的限制,以及自我和本我的固着和退行,产生性格特质。如果这些性格持质对个人的处理愉快的能力,以及避免与环境产生严重冲宝的能力干扰不大。我们称之为正常。如果它们对愉快的干扰过大,连且使个人与环境产生严重的冲疼,我们称之为异常。我们在此必须再次强调、在正常与异常之间没有明确的界线。这种区分完全凭实际观察,做出的决定也必然带有主观性。我们认为,超我的形成是俄狄浦斯期严重的本能冲突的正常结果。同时,超我确也是某些抑制或限制不断地强加在自我和本我之上而形成的、用以阻止来自俄狄浦斯冲突的危险。

从纯理论的观点来看,只要我们将前儿段所讲到的各种可能性都认为是心理结构发展和功能的不同途径、而不勉强去分什么是正常的什么是异常的,我们就可以避免被责备为主观武断。然而,临床医生面对着处于痛苦之中或与环境产生严重冲突的人,只能主观武断地区分他们是正常的还是病理的。若是正常的,则不予理睬,也不用治疗;若是病理的,则需要予以关心并进行治疗。我们已经讲过,在心理结构的发展形式和功能形式之间区分什么是正常的什么是病理的,必顽依据个人处理愉快的能力受了多大的限制,以及他对环境运应的能力受了多大的破坏而决定。当一种心理功能形式被认为是异常时。临床术语通常称之为性格障碍或性格神经症。这种情况下,心理结构的功能形式是病理的,对个人大对不利,但在心理上还是相对地稳定和平衡。这种平衡,就像任何心理平衡必须做到的那样,是来自心理的各种力量与成长过择中外部影响力亳之间相互作用的结果。

各种性格障碍或性格神经症对治疗的效果大不相同,般来说,病人愈年轻,由特殊的性格特质和性格结构晰造成的痛苦愈严重,治疗效果愈好。必须承认,对这些病例,我们还缺乏十分精确的或十分可靠的预后标准。

现在谈谈Freud早期通过研究癔症和“防御性神经精神病”而深刻了解到的心理结构功能障碍的形式。在这些障碍中、有下列情况。首先、在儿童早期自我和本我之间有冲突,尤其在俄狄浦斯期及前俄狄浦斯期。这种冲突,由于自我能够建立某些稳定有效的方法,以检查危险的内驱力衍生物而得以解决。这些方法很复杂——包括防御和自我变换两种、如认同作用、限制作用、升华作用、达行作用等。不论什么方法,只要某些随后而来的事情不至于破坏其平衡,并且不影响自我结构继续有效地控制内驱力、这种方法将令人满意地工作下去、当然时间上可长可短。诱发因素到底是通过强化和加强内驱力起作用,还是通过减弱自我起作用。我们目前仍不清楚,重要的是、自我相对地减弱了,使它对内驱力控制的能力遭到损害。在这种情况下、内驱力(更确切地应说是内驱力的衍生物)不顾自我的制止而侵入意识,并直接地转变成公开的行。在自我处于相对不利的情况下,自我和本我之问出现了剧烈的冲突、最终达到了妥协。这种安协被称为神经症的症状。虽然它与Freud所说的现实神经症毫不相干,却也常被称为神经症的症状,其至Freud自已在后期的著作中也这样称呼之。

在心理功能障碍的形式中,不论什么诱因、均存在自我防御的失·结果是,原先自我还能有效地掌握的本我冲动不能再被自我适当地控制了,于是形成了妥协。妥协形成,潜意识地表现了两种成分:一方面是内驱力衍生物;另一方面是自我防御反应,以及自我被内驱力部分地冲破后,对出现的危险所表现的恐惧和罪恶感的反应。这种妥协形成被称为神经症或精神神经症的症状,而旦如Freud许多年前指出的那样,它与显梦或梦的成分非常相似。

举几个例子可能有助于说明之。

第一个例子是一个患有呕吐症状的年轻女子。通过精神分析发现,这个病人潜意识里被压抑的欲望是受孕于她的父亲。这个欲望及对抗它的投注始于她生命的俄狄浦斯期。直到她20多岁父母离异及父亲再娶之前,她都一直能够满意地解决俄狄浦斯冲突、父母离婚及父亲再娶,使自我力量再也不能适当地控制她的俄狄浦斯欲望,于是再度激活了她的俄狄浦斯冲突,破坏了她多年来所建立的内心平衡。病人妥协形成之一,表现为呕吐的症状。症状潜意识地再现了被压抑的受孕于她父亲的俄狄浦斯欲望的满足,一若病人对呕吐所麦示的那样:“瞧,我成了孕妇了,大清早就作呕。”同时,由于呕吐所造成的难受以及伴随它的焦虑,丧现了自我潜意识的恐惧和内疚,这种恐惧和罪恶感与俄狄浦斯欲望有关,此时,自我还能有效地压抑,使得婴儿欲望没能够进入意识。使病人意识不到她的呕吐是怀孕幻想的一部分,更想不到是跟她父亲而受孕。换言之,表现为呕吐症状的心理结构功能障碍,使得欲望所贯注的内驱力得以释放,但这是由自我防御作用所伪装和变形了的释放,带来的是不愉快而不是愉快的感觉。我们说,神经症症状的解释通常都过于简单化了,也就是说,它们往往并不只是由一个这样的自我和本我之间的潜意识冲突而造成的。以这个病人为例,欲望被表现为幻想;“母亲死了或者离开了。我已经取代了她的位置。”于是罪恶感和恐惧感应运而生,也就产生了我们所描述的症状。

另一个例子是个青年男子,其症状如下:他必须肯定所有的灯均已关了,才能离开房子。理由是,他有一个可怕的幻想:假如家里没有人。灯又不关,就可能发生短路,引起火灾,将房子烧毁。此例的病因也还是起源于俄狄浦斯冲突。这个病人的俄狄浦斯冲突一直没有得到稳妥的解决,自我防御及调节机制由于青春期心理躁动的开始而遭到破坏、于是从那时起,妥协形成或精神神经症的症状在他的心理功能中就显得突出了。

精神分析过程中发现、这个青年人的症状有下述潜意识的意思或隐意。这个病人在潜意识里想以他父亲的身份跟他母亲相处。潜意识电,他幻想出一条达到他日的的途径:这所房子烧毁了,他的父亲由于失去了房子而颓废不振,整天沉湎于杯中之物,不能继续工作,他就可以取而代之成为一家之主。此例,以两个事实再现了本我欲望的侵入:\textcircled{1}经常以取代他父亲的幻想作为先占、而意识中只有在房子被烧毁之后才有可能实现;\textcircled{2}他离家前巡视时,除了将电源插头拔出外,还要将电源插头再插进去试试。尽管他意识的先占是必须防止火灾,这种举动还是表现了他使房子烧毁的欲望。另外,症状里的自我清楚地表现为:抵消、压抑、焦虑和内疚。

第。三个例子是个患有病理性恐癌症的青年男子。同样,也是由于婴儿俄狄浦斯冲突造成的。诱发因素是病人成功地完成了职业学校的学业,以及过早地盼望结婚。这两者从潜意识的角度对他来说,都是危险的俄狄浦斯幻想的满足。病人的症状表明、他潜意识里幻想做一个女人,被他父辛所爱,而且受孕于其父。症状的一部分是白期望或害怕得了不治之症所构成,象征着被阉割以致变成女人的幻想;症状的其余部分是想到在他身体内部有个东西不断地长大,表现了他已有身孕以及胎儿在他肚子里不断长大的幻想。此时,他的自我仍十分完好,可以一辈子防御那些可怕的俄狄浦斯欲望。这些欲望仍然被压抑着、至少仍处于它们原始的婴儿的形式。欲望是以变形的样子表现的,病人自己不可能有所认识,因而不能进入意识。他根本没有意识到自己存在着变成女子,而且还受孕于他父亲的欲望。然而、尽管他的防御使出了最大的努力、仍然不可能使他避免产生焦虑。即便他的俄狄浦斯欲望以伪装的形式出现,也难免使他感到焦虑。于是。担心得病和死亡也就成了他症状的一部分。

对于神经症的症状形式,Freud创造了两个术语:疾病或症状形式的原始获益,以及疾病或症状形式的继发获益。现在谈谈Freud所说的个人由症状形式而得到的现实获益或利益是怎么一问事。

Freud认为,疾病过程的原始获益在于至少有部分的本能释放,即至少部分地满足了一种或几种本能欲望,向时又不至于;卢生严重的罪。恶感或焦虑。如果不是这样的话、这种罪恶感和焦虑将妨碍病人由神经症症状而带来的哪怕是部分的满足。这种说法似乎很奇怪,因为事实上焦虑总是伴随神经症症状的﹑而且是这些症状的主要组成部分。但是,似是而非的情i况总是比真实的情况更明显。Freud认为,自我相对地减弱。使得婴儿本我冲动最大限度地侵入意识。如果发生了这种情况,伴随而来的将是由这个冲动丽产生的最大限度的婴儿罪恶感和恐怖感。通过神经症症状这种妥协形成,内驱力的衍生物以伪装形式部分地出现,自我就可以在一定程度上甚或全部地避免了不愉快之感。由此可以看出、神经症的症状与另外一种称之谓显梦的妥协形成是何其相似。在显梦中,自我无法阻止被压抑的冲动进入意识,只得允许冲动通过适当地伪装和变形而得到幻想性的满足或释放,同时,自我也能得以避免体验不愉快的焦虑情绪,或不至于醒来,

从本我的角度来看,神经症的症状代替了被压抑欲望的满足。从自我的角度来看,神经症的症状是危险的。不受欢迎的欲望慢入意识,虽然要受到检查,并且只能得到部分地满足,至少比这些欲望以原来的形式出现所引起的不偷快要小些。

继发获益是自我眯坌堂寻找能够满足愉快的努力。一以症状形成了。病人不仅觉得症状能带来痛苦、也发现症状可以带来某些好处。举一个明显的例子、一个士兵在战时出现焦虑状态,就能得到他的伙伴们得不到的切实好处:被撤到后方、减少了被打死的洁险。虽然事情十分明显·但i这到底并不是一个很好的例子,因为、被撤向安全地带的念头可能会潜意识地影响焦虑状态自身的发展。然而,无疑我们还是可以找到许多说明通过神经症而得到好处的病例的。

从神经症症状的理论现。点来看,继发获益不如原始获益重要。但从治疗的观点来看,继发茯益则可能十分重要。因为,继发获益过多,而这些症状的存在付病人又有用处、就使得病人潜意识地愿。意神经症长期持续下去。比如,对严重肥胖的治疗,本来就是个上分困难的事情。假如这个病人是马戏闭里的华女人,她以自己的病态肥胖而谋生·就不存在治疗的可能了。

在所举的这些神经症症状形成的例子里,还没有包括一种可能性,即可能是自我防御中退行的作用,包括自我功能的退行和内驱力的退行。从理论的观点来看。退行作用只不过是自我可以运用的许多防御方法的一种。然而,从实际结果的观点来看·它是特别严重的一种。总的来说退行的程度愈厉害,引起的症状愈严重,治疗的效果愈差、病人愈需要住院治疗。

关于由自我防御失驳而造成的功能障碍的形式,还有一点需要谈谈。作为功能障碍的神经症症状、对个人的自我束说是外来的。并且是不愉快的。比如、那个必须将灯全部检查一遍才能离开家的年轻人,他不愿意那样做,但又控制不住地那样做。他的症状,对他的自我来说是外来的。同时也是令人不愉快的。但是,那个患有呕吐的年轻女子,并没有感到她的症状是外来的,她毫不杯疑是她胄里感到作K,就像患了急性胃炎而必然会恶心一样。她的症状无疑还是令人不愉快的。

可见,由于自我相当的薄弱、不能建立或维持一种稳定地控制内驱力的方法而造成的妥协形成中,有一些对自我来说既不是外来的,也不是令人不愉快的。其中,最严。重和最明显的是各种性变态及性癖好者。在这些人身上我们可以观察到两种情况。第一、他们显然是处于性格障碍和精神神经症症状之间。很难将它们与这两者明确地鉴别;第二,构成性变态和性癖好的本能的满足,被自我用作为防御的方法。以检查其他的内驱力衍生物,因为这些内驱力衍生物的出现或满足对自我来说太危险了。从自我的观点来看,这些妥协形成是利用一种内驱力衍生物控制另一种内驱力行生物的范例。由此可知,这些妥协形成与我们在第\ref{4}章讨论的反向形成很榭似,这样,就大大地修订了Freud原先提出的神经症的反面是性变态的说法。

我们不打算详细地讨论什么毕的内心冲突和妥协形成会导致什么样的症状。临床上将这些症状分为癔症、强迫症、恐怖症、躁狂抑郁症、精神分裂症、性变态等等。我们的目的是,使大家了解一般的、基本的、理论的知识。这些知识既普遍地存在于所有这些临床类型之中,又可从病理心理学的角度鉴别这些临床类型。最重要的是。我们力图使大家明白一个事实,即在心理功能的领域中、正常的和病理的之间不存在什么明确的界线。我们称之为止常的,以及我们称之为病理的、应该被理解为人与人之间在心理结构的功能上的差异——程度上的差异,而不是本质上的不同。

\signatureA



\chapter{心理冲突与常态心理功能}\label{9}

在前一章里,我们把注意力主要集中在心理冲突的那些被回类为病态的后果。而在这一辛中,我们将要集中研究精神详的另一端,将要研究人格发展的与心理冲寒密切相关的那些方面、就归类而言、它们无疑属于常态而非病态。

前面已经提到过,在此一范畴之中的常态与病态只有程度之别,并非质的不同、心理冲突的若干后果处于汇常和病态两町之间,要想明确划分几乎是不可能的、除非武断,常态与病态之交错有如彩虹之颜色互相渗入。

但是,又有许多心理冲突的后果无疑属于常态。在分析神经症病人时常有机会观察到这样的正常现象。在精神分析的过程中可以发现,一些正常现象的潜意识涵义及其复杂的来源,要不是该人接受精神分析的话。这些情况绝不可能被发现。基于这样的分析经验、我们在本章将讲述人格发展的若干正常方面,者如性格特征、职业选择、择偶等与心理冲突之间的联系。我们也将讨论正常精神生活的其他一些方面。它们也和心理冲突确凿相关,但从对个人精神分析所得资料尚不及从童话、神话、传说、宗教、道德等方面所得资料那样丰富和满意,对后面这些情况我们所作的推论一部分基于对个体病人的分衍经验·一部分基于对人性这个总体所作的精神分析。

精神分析学家对性格特征产生兴趣,最初是由于注意到性格特征与幼儿期的本能愿望有关联。Freud指出,幼儿期肛欲的起伏变化与日后的爱好整洁和条理、过度节俭和吝啬,以及顽固和执拗是有联系的·性器期愿望与成年后的勃勃野心也同样有联系。其他精神分析学篆步Freud之后尘从事这方面的研究、其结果是、由性格特征与libido发展的特定阶段之间通常可以观察到的关联出发,发展出了一套性格类型术语,诸如口腔性格、肛门性格和性器性格。对许多病人的临床经验证实了Freud的早先印象,即。上述的肛门性格特征通常是由幼儿期的肛门愿望和冲突导生出来的。肛门性格也用于称呼凌乱不整、污秽肮脏和不修边幅的人。基于同样理由,自信、乐观、大度。以及与之相反的特点,都被称为口腔性格;面野心勃勃、追求认可和称赞,则被称为性器性格。

这套术语是基于本能、特别是性本能、所作的分类。它反映出对于精神生活中本能方面的强调,而这一点正是精神分析心理学早期发展阶段的特征。后来。逐渐积累了更多的知识,才认识到,从幼儿期的本能慰望以及由之产生的心理冲突出发、到成年期的精神生活和行为之间的道路是错综复杂的。由以下的实例,我们试图指明这条道路的复杂性以及个体的生活经验如何参与形成后来的结果,

第一例妇女正当20来岁,在其生活风格中,仁慈大度的性格特征非常鲜明。她因为相当严重的神经症症状而做精神兮析治疗。在分析过程中发现,她的仁慈宽厚正与其神经症症状一样、都和童年时期的心理冲突有密切的关系。然而她的性格特征理应认为是正常的,因为仁慈大度使她自己感到快乐而不是引起自我折磨,这一性格特征也是社会所乐于接纳的,现将情况简述于下:

病人小时候就被与其生母长期分隔开了,这是因为母女间无法维持令人满意的关系、母亲使女儿遭受挫折。病人对母亲怀有强烈的矛盾情感,由之产生的心理冲突在其神经症的每一症状的出现上都起了极为重要的作用。然而,又正是这相同的心理冲突决定了她有一副宽厚心肠。她自己还在小小年纪,就要充当弟(妹)们的保护者。可怜的更加幼小的弟(妹)们也像她一样,遭受着母亲那不可预测的情绪和行为的析磨。这个家庭里的几个孩子是在接连几年中生下来的,她只比几个弟(妹)略长一点,可是、她得做弟(妹)们的支柱。为他们辩护,庇护他们少受惩罚,安慰他们的苦恼,似乎她就是孩子们的妈妈而不仅是姐姐。对于这些孩子,她的所作所为算得是一个“好妈妈”。及至成年以后,她仍怀着强烈的愿望要去帮助这个大千世界上的那些孤弱无助的、受尽折磨的“弱小者”。她满腔热情地投身」这·一类的慈善工作、毫不吝惜自己的时间、精力和金钱。与对被压迫者慷慨大度相对应的是、她对于压追者有强烈的轻蔑和憎恨。她在潜意识中把·受其援助者与儿时的自己及弟(妹)等同起来,把被她憎恨的压迫者与儿时的母亲等同起来。当她小的时候曾渴望报复母亲,现在通过对压迫者的愤怒她解了恨。通过对孤弱者的慷慨帮助,她潜意识地为自己和弟(妹)们树立起一个可信赖的、有献身精神的母亲形象以取代那个现实的自我中心的、不堪信赖的母亲。病人成年以后对此一类型的慈善工作特别有兴趣,起决定作用的因素在于她毕生渴望有一位可爱的、可信赖的母亲,同时又存在憎恨和复仇的愿望。别的慈簪事业对于她就缺乏吸引力了,即使稍有捐助,也是敷行塞责、与其热切献身于帮助孤弱者时的仁慈慷慨形成强烈对比。这是一个很好的例证,清楚地表明正常的性格特征起源于幼儿期的本能需要和受到的挫折。

第二例是一位30岁的男子,他秉性愉快、举止文雅、通晓事理、有舍作精神。与前一例病人同样,他也是在生活中出现了显著的神经症性障碍。但是他又有上述的性格特点,行为大方可人,无愧出生于教养良好、道德高尚的中上等阶层家庭及著名学校。潇洒翩翩的风度对于他就像呼吸一样自然。要不是对他精神分析有所了解的话,还会以为他那“天生”的良好风度就是从小接受风度教育的结果。无论说到哪里去,这样的性格特征肯定属于正常范围、它是社会能够接纳的,也不会引起本人的痛苦和忧伤,他对待生活的那种愉快、通情理的好脾气总是给他带来好处。正像每个人都会遭到失败和受到危险威胁。他也有烦恼和沮丧的时候。不过,这样的情绪从不持续很久、他很快就会采取通情达理的态度。困难解决不了,那就忍受,乐观总比抱怨强,只要努力坚持对付下去,事情总归会有一个满意的结果。人们会说:“这岂不是哲学家伊素第二吗!”然而这关系不到哲学才能的问题,而在他童年时冷酷现实使其所在文化环境中的传统美德得到强化、以致形成其人格中至关重要和必需的一个部分。

病人9岁时曾面临一次严重威胁,对他至关重要的一位家庭成员要抛弃他。连续三天他陷入了深重的急性抑郁。幸而这威胁没有兑现,但危险依然存在、被遗弃的可能从此长留在他的心中,对此他作出了两手反应。一方面用自己的行动争取不被遗弃;另一方面作好思想准备,万一被遗弃时不致感到孤弱无助、不知所措。第一手反应的基本表现是,把出于本能的愿望和行为一扫而光。在发生这次威胁以前,病人本是个急性子、还偶有狂怒发作,白那以后再也没有了。进行精神分析时他能团忆到的真正发火只有一次。他的性活动也减弱,但程度还不是太严重。他变成了一个非常温K的男孩。自9岁起遭遗弃的威胁之后,他再也没有性方面的越轨表现以及攻击行为,在此之前偶然还是有的。第二手反城的实质表现是认同于那个威胁要遗弃他的人,从此他变成像那个人一样的性格:愉快,乐规,通情达理,务实。由于认同之故,他变得会照料自己了。等他成为少年,离家读寄宿学校时,独立生活能力真的不错。由这个病人及前‘例病人可以看出,童年时的心理冲突和精神创伤与长大成人后的正常而有利的性格特征之间存在着密切关系。这个病人的好脾气、好态度和乐观主义并非只是良好教养的结果,它们还出自以下强烈的动机:病人深信,要是脾气不好、行为出错、他就会邀到遗弃,当他9岁时差一点就被遗弃了。还有,它们同时也是向那个孔乎就要遗弃他的人认同的结果。换句话说,病人的正常而且有利于适应环境的性格特征,以及他的神经症症状、都与儿时的精神创伤租心理冲突紧密相关。它们的动机也都出自同]一个根源。

在第\ref{5}章里我们]已经讲过俄狄浦斯阶段形成超我时认同机制的重要性。在此章的后面我们也还要回到超我形成这个题目上来。现在要强调的是,并非俄狄浦斯阶段的所有认同都和超我形成有联系。等如、有一些认同只是儿童性愿望和竞争愿望的略加伪装的表现。小男孩有意识地模仿他的父亲是一种通常现象,因为他对父亲又羡慕又嫉妙r、这样的愿望有不少持续到成年阶段,以致在诈多方面儿子成了父亲的心理复制品。在母亲与女儿的关系上也常常存在同样的情况。这样一来、父母亲及其子女可以有相同的姿势,相同的面部表情、相同的步态、话语、笑容,巧至与人相处时相同的含蓄或者热情等等。亲『之间不仅存在着形体上的相似,还可以有行为上的相似。后者不是休质特征的遗传、而是心理特征的仿效,是儿童期潜意识的认同、因为儿童期望像他们的父母一样,所以就在各方面向其父母认同。

儿童的羡慕和嫉妒并非全部都指向其父母亲、当然。父母亲是其指向的主要目标。他们对同胞的兄弟姊妹也怀同样的感情。往往非常强烈。这样的感情在儿童的本能生活中,以及随之形成的冲突和妥协中,可能起重要作用。有这样一个例子,一位年轻女士对音乐有强烈的业余爱好。她喜欢听音乐,按业余的际准而言、受过了良好的音乐教育,她学过好几年的大提琴,乐于演奏,但并未达到精通的地步。所有这切都是仿效·不是仿效母亲·而是仿效姐姐,其姊是位有造诣的职业音乐家。她们的父亲高度赞赏姐姐的音乐才能和成就。我们的病人从孩提时起就感觉到姐姐由于有音乐才能成了父亲的宠儿、所以她也学习音乐,希望在父亲面前与姐姐争宠。这个病人对酱乐感兴趣之后的发展情况应该认为是正常的,音乐对于她是件乐事,是生活中的重要部分。在我们社会中、大多数音乐爱好者都是这个态度。然而无疑的是,这个病人对音乐的爱好是由于俄狄浦斯情结而产生的,是由于儿时在父亲面前与姐姐争宠而产生的。通过精神分析发现、在她成年之后的音乐活动中、也存在着潜意识的俄狄浦斯涵义。譬如有一次病人报告了一个梦。说她在一个管弦乐队中演奏。通过自由联想,她回忆起一位音乐家,若干年以前他俩曾经相爱,最近听说那人当了一个著名管弦乐队的指挥。据病人说,这位音乐家无论在年龄上还是长相上都与她的父亲大不相同,可是这个人总是使她想起自己的父亲,或许是因为这两个男人刮完胡子都用同一个牌子的花露水吧。由病人的自由联想我们查明了两点:第一、其梦的隐意是与父亲作性交合的俄狄浦斯愿望;第,此愿望以幻想(梦)的伪装,表现为与她在“很久以前”相爱过的男士一起“演奏音乐”。音乐对于她,在潜意识中仍然保持着原来的俄狄浦斯涵义。

在幼儿期本能生活与成年期的一些正常活动(诸如选择职业、选择配偶)之间,常可观察到同样的联系。要在职业选择。上举出满意的例子是有困难的,因为有很多因素影响职业选择。不过,就由下面三则节略的、有所伪装的实例,读者至少也会有几分相信我们论断的正确性。

一例病人是一位40岁的男性产科医生,在同胞6人中居长,他和弟(妹)们都是在乡村出生的,并且在那里度过童年。他记得母亲生下弟(妹)时的情景,每一次对他都是印象深刻的大事。他对母亲的分娩有强烈的好奇心,可是每次分娩时都不许他进房间观看,虽然从很小时候起,他就把动物的分娩看得多了。孩提时代的性好奇是决定其终身职业的一个重要因素。但除了满足其好奇心之外,他选择这个职业还有其他潜意识的目的。其中之一是满足他超越父亲的愿望。因为每逢病人的母亲分娩,父亲对请来的医生总是必恭必敬。谦卑之至。所以自己也要当这样的医生。第二个原因是、每逢他的母亲再次怀孕,他总是对母亲和其腹中的孩子产生恼怒的感情,而他选择产科医生这个职业能够加强自己对恼怒的防御机制,作为产科医生,他对产妇和婴儿们既亲切又爱护备至,不再像童年时那样出现弒亲性的恼怒以及随之产生的内疚。最后还有一点,作为产科医生,每当一个婴儿降生时他都感到自信并意识到自己的重要,不再像童年时那样,每当母亲又生一个时就感到自己的重要性下降、变得孤弱无助。

第二例是30多岁的男子、职业也是医生。这个病人4岁时,母亲住院动大手术,因而与他分开了好几个星期。这次分离在他的生活中引起的重要后果之一是,他决心当一个医生,而且要当外科医生,每逢有人问他长大了干什么,他总是回答:“当医生,拿刀割他们。”他对母亲的矛盾情感是很明显的,30年后在对他作精神分析时得到了充分肯定。当外科医生有正反两方面潜意识的涵义:一方面,当医生就能为母亲治病,做教她性命的英雄,同时也不至于与母亲分开;另一方面,母亲离开自已去住院是对自己的不忠诚、拿刀子意味着要刺痛她作为惩罚。

第三例是一个不到30岁的劳务协调员,男性。像上面一例一样,这个病人很小时被迫与母亲分离,造成重大精神创伤。因为他的父母吵架后分居,才6岁的他被送进了寄宿学校。成年以后,这个病人出现很多神经症症状,但是作为劳务协调员他却得到了明显的成功。他总是不倦地努力协商解决把他自已也卷入其中的劳务纠纷,经常能使争执的双方弥合分歧,避免公开破裂。他有一个鲜明的论点:只要争执的双方能够坐到一起进行对话、就没有不能解决的纠纷。他这样热情地调解纠纷,其根源在于:自从他与父母分离就产生了强烈的愿望,希望双亲停止争吵并重新和好,以使他能重新和父母住到‘起,特别与心爱的母亲住到一起。所以,他无时不在努力工。作。说合人们避免分裂,不要像自己的父母那样做,不要把工入们弄得像他自己6岁时那样有家难归。这一例再次说明,童年时的精神创伤可以潜意识地影响成年后的职业选择,而且是有利的影响。

当转而考虑幼儿期本能生活与往后配偶选择之间的联系时,面临的困难是现象太多太复杂。。二者之间的联系是多方面的又是很密切的、主要的困难是把复杂的现象梳理清楚。要证明二者之间存在联系看来用不着,也无需感到惊奇。只要回想就能发现,每个人的第一个性对象都是在童年时代就有了,日后再有的性对象往往是童年时第个性对象的复制版。对严重的神经症病人进行分析时发现如此,轻度神经症病人和正常人亦复如此。所以只需举一两个例子便足以说明。

有位年轻男子与一女士恋爱然后结了婚。他自己清楚地觉察到这位女士在体型、身高,面貌等方面都与他本人相像,然而他未能意识到这种体质上的相像正是这位女士对他有吸引力的原因之一。直到后来对他进行精神分析的时候。他才明了这个事实,即他俩看来有如兄妹这一点,在潜意识上对他起性兴奋作用。潜意识中,这位女士代替了他的一位姐妹。童年时,他非常喜欢和依恋这位姐妹,曾幻想与她结婚,做她的孩子的父亲。及至成年,他潜意识地实现了这个幻想,与一位外貌酷似其姐妹的女士结了婚。有趣的是,我们发现丫他潜意识幻想的令人印象深刻的后果,他把妻子当成了自己的姐妹。在对他进行精神分析的头几个月里,他经常提到妻子,却用了他姐妹的名字、要是分析者不提醒他注意,他从不觉察自己的错误。

另一例女病人对配偶的选择受到童年时代与姐姐关系的影响,情况较为复杂一些。当这两姊妹还是小女孩时,她们最亲密的玩伴是邻居的一个小男孩。病人的姐姐与这个小男孩很投契,以致人们都说他们俩长大了会结婚的。而病人那时常有被他们俩排斥在一边的感觉,并且嫉妒他们。病人对自己的父母亲也有类似的被排斥感和嫉妒感。童年尚未结束、这两姊妹就都见不到以前的邻居了,因为两个家庭不再住在一起了。又过了若千年,两家再次搬到一起,三个童年时的伙伴重新聚首时已都是青春年华。这时,病人趁姐姐离家上大学的机会,有意地让那年轻人喜欢上她,也成功地达到了目的。但是,当那年轻人进一步亲热的时候,她又拒绝了,说是只能做“好朋友”。又过了不久,她就把才从姐姐那里瀛过米的年轻人丢开,与这个年轻人的一位好友恋上爱并且结了婚。后者在遇见病人之前本来有未婚妻而旦就要结婚的,却被病人枪过来了。病人在成年后的爱情生活中,成功地实现了做小女孩时的渴望——在三角关系中要做一个成功的、而不是失败的女人。她终于向童年时代喜欢姐姐而不喜欢她的那个男人报了仇。把他丢在一边而与他的朋友结婚。对她的姐姐及其玩伴、她取得了潜意识的完全的胜利。

以上实例表明,幼儿期本能愿望对于日后的精神生活有强大的和持久的影响。它们可以决定人的职业选择、成年后性生活过程、兴趣爱好、举止风度、奇习怪癖等等。在许多情况下,更准确地说,这些影响还不是本能愿望和心理冲突本身的直接作用,而是由之产生的幻想在起作用。后一例女病人便是实现了在性的竞争上胜过姐姐的幻想,也许可以称之为灰姑娘幻想。那个外貌与其妻相似的年轻人也是实现了童年时的幻想。产科医生、外科医生、劳务协调员的情况也都是‘样。在每一例中,都是由幼儿期本能愿望及与之相连的恐惧中产生幻想、这幻想在病人的生活中变成了重要的潜意识的内驱力。

迄今所举的例子都得宫于临床实践。对每一例都可以应用精神分析方法。每一例都是病人,他们与分析者充分合作,尽可能完全和自由地谈出他们的思维、联想,过去和现在生活中的详情细节,甚至不可告人的隐私秘密以及令人痛苦的回忆。因而,一方面有幼儿期心理冲突的资料,它们往往是潜意识的,另一方面有成年生活中清醒的思维、愿望和行为的资料,将二者联系起来所得出的结论,其可信度是相当高的。在以下所作的大部分讨论中,我们也要提出类似的结论。可是,这些结论不是全部基于临床资料,不完全是用精神分析方法对个体进行治疗性分析时所得的资料。所以,我们将留有余地,约束自己只在有充分证据时才作出推论,并且时时留意(当得不到由用精神分析方法才能获得的资料时)所存在困难的性质。

*

在幼儿期本能幻想的后果之中,包括有一切类型的白日梦和故事:如童话、神话、传说以及经过不同程度加工的文学作品。儿童对童话以及类似的故事最感兴趣。童话广泛流传,从古至今不衰,表明它们所涉及的主题几乎投所有幼儿之好,事实也是如此。童话以非常直接的方式涉及幼儿期本能生活的主题,主要是俄狄浦斯阶段的主题。几乎在每一篇童话里都有一位年轻的男主角或女主角,战胜并且杀死了一个邪恶的老坏蛋(当然也有男的或是女的),其后,男主角与美丽的姑娘结婚,女主角与英俊的王予结婚,并且永远幸福地生活下去。每则故事自有其变化和发展,不过它们的基本模式永远是同样的。而每一种变化发展又会引起某一类儿童的特别兴趣。譬如《灰姑娘》的故事就特别对做小妹妹的儿童有吸引力。在这个故事里,被人瞧不起的小妹妹灰姑娘得到了王子的爱,与他结了婚当了王后,向恶毒的母亲和姐姐们报了仇。这里还可以看到,童话中总是涉及由俄狄浦斯愿望引起内疚的问题。由于童活故事的对象是小孩子和有童心的单纯的成年人,用一点简单的设计便足以安抚听众们良心上的内疚。童话中的男主角或女主角肯定都是好人,常常遭到虐待,就像灰姑娘那样。和主角作对的一定是一个卑鄙、邪恶和可恶的家伙、理应遭到恶报。童话《灰姑娘》有了许多改写本,把原本中的生母和胞姊改写为继母和异母姐姐,让隔了一层的亲戚遭报应。良心上就会舒服一些。

另一个广泛流传的童话是《杰克和豆茎》,又名《杰克和吃人巨怪》·其实后面这个题目更贴切。童话《灰姑娘》的主题恳爱情和婚姻,而童话“杰克”的主题则是弑亲和阉割,不过作了充分的伪装、所以孩子们听起来只感到兴奋愉快,而没有恐惧。在故事里,没有说杰克杀死的是他的父亲,只说那是一个吃人的巨怪,杰克先要把巨怪的法宝愉出来,然后才能杀得死他。要不是一个傻女人相救·连杰克也被巨怪吃了。童话“杰克”也有改写本。说是巨怪先吃了杰克的父亲,这样一改,杰克非但不是弒亲的逆子,而是成了复仇的孝子。又说那法宝本为杰克的父亲所有、于是杰克把它偷回来也成了正义举动。真正的强盗和谋杀者便不是杰克,而是吃人巨怪。

篇篇童话里的角色郴离不开主角(男的或女的)及其父、母、兄弟、姐妹。主角和他的朋友一定是好人,与他作对的一定是坏人。结局一定是大团圆:主角胜利,作对者死亡;一定有性结合,即主角和所爱的人结婚;故事还一定指出前景,那就是他们将会有许多儿孙,并且“永远生活得幸福”。成人对童话一类的文学作品不再感兴趣,但童话永远使孩子们着迷。如果成人改变阅读的目的,不作为文学欣赏和娱乐,而是从中了解儿童心理,了解他们的期待、愿望、情感、抱负和恐惧,那就会读得非常有兴味。童话为了解幼儿期本能的精神生活以及日后潜意识精神生活的许多表现提供了一条有用的通道。

神话和传说的起源与童话相同。诚然在若下基本方面它们的目的不一样。譬如神话和传说的读者对象是成人而不是儿童,因而涉及的心理活动也更为复杂。就某种意义而言,它们较为现实,比起童话来更多地反映了成人怎样看待环境的复杂性,反映了人在复杂的环境面前相对地孤弱无助的感觉。就某种意义而言,它们也较为现实地试图解释人类环境的起源,解释其性质和功能模式。它们不像童话那样单纯为了取乐,而r认真地试图解释宇宙,因而可视为科学的先驱。可是、神话和传说基本上衍生于幼儿期的本能生活、衍生于本能生活中的热情、恐惧和心理冲突,这一点。又和童话是一样的。

臀如,希腊神话的荷马编写本在公元前1000年稍后便已流传。荷马把男女神祇们描写成一个大家庭,居住在奥林匹斯山巅的宫殿里,父亲是天神宙斯,母亲是天后赫拉,他们有许多孩子。在荷马描写的奥林匹斯山上,乱伦、嫉妒、争斗和阴谋诡计全都习以为常,有如在任何儿童的俄狄浦斯幻想中一样,惟独谋杀不可能发生,因为诸神祇都是注定不会死的,而天神宙斯又最为强大。他永远是战胜者或最后主宰者。荷马神话预先排除了弑亲,它不以父亲的悲剧作为结局。

然而在其他神话中,也包括许多希腊神话在内,又把弒亲的主题表现得直截了当,把父神的命运弄得和“杰克”童话里的吃人巨怪相同。父神被杀死,被阉割,还往往是被自己的孩子们吃掉,孩子们述经常是得到母亲的帮助才战胜父亲,于是孩子取代父亲的权力,转回来又毁在自己的孩子手里。大约在公元前500年,古希腊的欧里庇得斯写出了关于俄狄浦斯的故事、年轻的俄狄浦斯于不知情中杀死自己的生父并娶生母为妻,后来为了惩罚自己无意中犯下的滔天罪行,他刺瞎了自己的双眼。由于俄狄浦斯的故事把乱伦、弑亲、阉割和悔恨的主题表现得如此清楚明白、所以Freud才创制了“俄狄浦斯情结”和“俄狄浦斯阶段(期)”这两个专门名词用于libido发展中。

现在、我们由古希腊神话转到犹太教——基督教神话上来。我们发现,后者和前者一样,与幼儿期本能生活有同样的联系。i旧约中的主要英雄是摩西,他是摩西律法的创制者,是尘世间上帝意志的代表。摩西本被作为埃及王子养育,但是他反叛埃及国王,打败了国王,自己成为人民的领袖。我们可以看得出来,这个反叛和弑亲的主题起源于竞争、仇恨和嫉妒,而此三者正是发展处于俄狄浦斯阶段的小男孩对父亲的感情。然而,正如我们在第\ref{5}章中已经提到过的,处于俄狄浦斯阶段的儿童对其父母亲的感情是矛盾的,对父亲和对母亲都有爱和恨的交织,但是爱和恨的比例可有不同变化。在摩西故事中,在摩西和其神圣的父亲,上帝的关系中。清楚地显示出那种小男孩对父亲热爰的感情以及对父爱的渴望。摩西被描写成虔诚侍奉上帝的人,对于背叛上帝,崇拜邪神异教行·举西必加惩处。一句话。摩西彻底地认同于上帝,并服从上帝。

这样对待上帝的。态度,已成为西方宗教传统中不可分割的组成部分。在此传统中养育出来的任何人,都把这种态度视为理所当然。但是,在宗教神话里,一般而言,这种态度不是普遍现象。甚至就在摩西故事里,也暗示过一点对上帝的反叛。说是由于某些小事遵从上帝意旨不够,摩西便被罚不许进入迦南福地。不过,基本上摩西还是被描写成受上帝宠爱的忠诚仆人,他的反叛看来只是针对低层次的父亲——法老王。

在基督故事中,也描述了父子间的矛盾情感,然而是用曲折伪装的手法描述的。与在摩西故事中一样,主要是强调圣子对圣父的爱,以及对圣父意旨的服从。故事中、把耶稣和圣父上帝描写得如此等同,以致实际上是合二为一。耶稣永不背叛,岂止于不背叛。他服从圣父的意旨已达到被杀死而无悔的程度。被杀之后,耶稣与上帝,圣子和圣父,便永恒地合在一起了。基督故事里,也出现了弑亲和乱伦的情节,但只是作为外围情节。当然不是说耶稣有这样的动机和愿望,与此相反,是一些可恶的犹太人和罗马人把圣子耶稣钉上了十字架。是他们弑亲而不是耶稣,耶稣自己是他们的栖牲品。至于说到乱伦,故事里也有一点暗示,说是耶稣之被钉上十字架是赎人类之原罪。所谓原罪是亚当和夏娃犯下的,他们违反上帝的禁令,在伊倒园里作性的结合。

关于宗教神话的讨论自然地把我们引向从总体上对宗教的讨论。在社会生活之中、可能没有另外的题目比宗教更富于心理学兴趣了。特别是我们注意到,宗教和心理功能的某些方面有联系,这些方面恰是我们兴趣之所在:部起源于幼儿期本能生活的潜意识动机和心理冲突。

幼儿的家庭——他的父母亲和兄弟姐妹——本质上就是他的全部世界。他的性冲动和攻击冲动指向家庭成员,由之产生愿望和心理冲突,构成幼儿期精神生活的特点:这里有热情的爱,揪心的嫉妒,狂怒,恐怖,悔恨,控制自己惊惧冲动和抚慰双亲的急切努力。对于幼儿来说,双亲是无所不知和无所不能的。宗教把整个世界变成了幼儿家庭的翻版:信徒有如幼儿,上帝和神职人员则有如幼儿的双亲。神职人员像双亲指点幼儿那样,齿诫信徒们举止应该如何,哪些是可以祈求的,哪些是不该想的,他们还负责回答有关世界的问题。特别是世界起源的问题。成人之关心世界的起源与幼儿想知道他那个小世界的起源是一样的。幼几总是想知道他自己是怎么生成的,他和别的婴儿是从哪里来的。Freud(1933)指出过,宗教对于信徒有三重功能:提供一种宇宙论,一套行为准则、和一个奖惩系统。父母亲提供给幼儿的也正是相同的功能。

因为,在心理上信徒和上帝的关系在许多方面都相似于幼儿与父母亲的关系,可以想象得到,信徒和上帝的关系上带有其起源的烙印。可以看到相同的矛盾情感,相同的爱与恨的混合,顺从与违抗的混合,虽然作了最大的努力去排除,但还是有相同的感官成分的混合。有些病人,宗教在其心理上起重要作用、对这些病人进行精神分析时,可明白无误地观察到宗教信仰中的如上表现,因而是有个案基础的。加之,在宗教仪式的惯常做法里也能辨别得出如上的表现,只是人们一向像儿童一样天真地对待仪式,照本宣科,亦步亦趋,没有去思索一下仪式的寓意。在潜意识上,宗教仪式是和幼儿期愿望和恐惧有联系的。

为了说明这一点,我们可以考虑一下弥撒和领圣餐这两上互相密切联系的仪式。它们作为宗教礼拜的中心组成部分。已经由大多数基督徒实行了大约1500年。受圣餐者被告知说,由于神力的作用,面包和葡萄酒已经转变为上帝的肉和血、供教徒们吃和喝。没有比领圣餐更清楚、更直接地表现出弑亲愿望的了。诚然,反叛并且战胜父亲的情感表露,在这里全然被否认了。这不是反叛、而是遵从上帝的意旨。这不是战胜,而是继悔罪、苦行赎罪和禁食之后,祈愿通过食圣父之内和血,变得像上帝那样心灵圣洁。授圣餐时的话是说得非常清楚的:这是圣父的肉和血,你们是他的孩子,吃下这个,喝下那个吧。仪式同时是为了纪念耶稣,他为了博得圣父的爱宠,甘受残害和死亡。授圣餐时都要提醒受圣餐者,耶稣是死在十字架上,并赞美他是服从上帝意旨的楷模,上帝是一切人的圣父。受圣餐者还被告知说、无论降临的命运为何,他们都应怡然承受,因为那是圣父的意旨,他们应该像耶稣一样忠诚地遵从,即使是受苦受难、受残害和死亡,也不能抱怨。如果他们像耶稣一样恭顺,上帝就会降爱于他们,并像对待耶稣那样,让他们永生于天国,与。上帝同在。这一宗教信仰(教义)与小男孩在发展的俄狄浦斯阶段所常有的幻想是何等惊人地相似啊!而后者是我们在精神分析实践中熟知的。俄狄浦斯阶段的男孩往往幻想自己是一个女孩。对于他,这意味着遭到阉割,也就是身体伤残,以此来博取父亲的宽恕和慈爱、并分享父亲的权力。与此类似的是成人的信仰,它已经在宗教活动中仪式化了、它允诺把圣父的爱赐给众信徒,因为后者也像伤残的圣子耶稣一样顺从于上帝了。

举这些例子是为了说明,虽然各种宗教彼此有所区别,但在此主要方面,它们全是一样的。它们以不同方式反映了一个事实,即宗教的根源在于幼儿期的心理冲突,诸如乱伦和弑亲的心理冲突,爱、镞妒和恨的心理冲突,同性恋和异性恋愿望的冲烫,阉割恐惧、阴茎嫉妒、悔恨和自我惩罚的冲突。在每一种宗教里,倍徒们潜意识上是儿童,神和神职人员在潜意识上是他们的父母亲,是他们既爱又恨、既惧怕又蔑视、既服从又反抗、既祟拜又想摧毁的父母亲。每一个社会群体的历史渊源,决定其宗教信仰及仪式活动的诸般表现。即使是不具备精神分析知识的社会学家和历史学家们,也一再指出过此一事实。精神分析学家能够增加的一点意见是:无论该群体的历史如何,农业或采集,定居或游牧,好战或和乎,每一个群体的宗教都是为了应付潜意识的心理冲突,而后者的根源则在于幼儿期的本能愿望和恐惧。

在讨论过宗教之后,必得再对道德稍加议论。前面已经提到过,任何宗教里面都包括一个道德准则,亦即奖励遵守规定和惩罚违反者的系统。在每一个宗教里都有“你应该做”和“你不可以做”的规定条款。这些社会性的规劝和戒律是怎样与每个人对自己的规劝和戒律,亦即个体的超我联系起来的呢?关于超我的形成和作用,第\ref{5}章已经讲过了。

在当今大多数有组织的社会里,道德被认为是宗教信仰的理想结果。例如,按基督教义的论点而言,当·个儿童被教育得敬畏和热爱。上帝,与十字架上的耶稣同样,于是他就会遵守上帝的道德法规、成长为一个好人。换言之。宗教信仰被认为有使人们道德高尚的力量。据我们所知·许多社会·特别是文明社会。都相信这是真的。虽然这一信念已被广泛接受,但是,由采用精神分析方法进行精神分析治疗所得的资料来看,那并非事实。事实是,个体的道德观念,亦即个体的超我形成,出现得更早。个体的道德观念是宗教教育的前驱,而不是其结果。个体的道德观念主要是由幼儿期,特别是俄狄浦斯阶段,本能生活中的心理冲突形成的。它也打上了随后发生的一切心理冲突的烙印。它终生持续存在,然而在很大程度上是潜意识的。事情很奇怪然而确乎是事实:没有一个人全部知道他自己的道德准则为何,甚至不知道其中最重要的那些部分。所以,有的人自信他的行为合乎道德标准,实际他的行为也为社会所肯定和赞誉,但往往还会出现罪恶感。有的事情明知不道德,也受到社会的谴责,但人们做得习以为常。

由精神分析所得的资料,证实了宗教批评者们的观察,即一切信经、一切教义问答、所有刻在石头上的戒律圣训,都不能使人道德高尚。精神分析并为此作出了科学解释:道德是个人的事,是超我形成的结果、它源于激情和势不可挡的恐惧。这些恳幼儿期本能生活的一部分,它并不源子主日学功课。个人的特定经验很重要,在形成道德观念时它能动地起决定性作用。可以非常真实地说、任何宗教的道德清规都起源于信徒们]幼儿期愿望和心理冲突,它和宗教神话及传说的起源是完全一样的。

但同时也得认识到,这并不是个体道德和社会道德之间关系的全部故事、这只是故事的一个重要部分。一个由精神分析作出了特殊贡献的部分。除了别的方面之外。每一种宗教都要致力于平息信徒们的焦虑。而同时又要容许他们的本能活动得到菜种程度的满足。要做到这一点,就得由道德法规来回答如下的问题:“我必须怎样做,上帝(我的双亲)才会赐爱于我和保护我。才不会因我有性的及弑亲的愿望和行动而恨我和惩罚我?”对于每个成长中的孩子,长辈们,尤其是父母亲,都得为其本能的冲突提供一个现成的解决方法。这个方法总是孩子的父母亲认为可以接受并且自己行之有效的,于是他们又把这个方法介绍给自己的孩子。按这种意义而言,如果某一种道德准则对社会的所有成员或大多数成员算得是满意的解决方法,这个准则就富有生命力。如果不满意,那就会这样变或者那样变,直至达到满意为止。如果变不成功,达不到满意的地步,就会被抛弃,而由另外的信仰和实行体系取而代之。社会为个人提供道德准则和宗教,个人对这准则遵从到何种程度,对宗教是否信仰、取决于个人从中是否能得到可行的方法以解决其潜意识心理冲突,这冲突是植根于其幼儿期本能愿望中的。

作为一种社会制度的宗教,到近代已经在走下坡路。大体而吾,宗教的衰落可以归因于近三个世纪以来科学技术发展带来的心理影响。然而,在反对宗教的社会里,原本会表现为宗教行为和信仰的潜意识倾向,可能以一种类似宗教的政治及政治家表现出来。应该说,这样的发展既非预谋,也无法预见。反对宗教社会的改革者们从来没有打算把他们的社会弄成他们所厌恶的那种类似宗教的性质,然而却事与愿违。

一定程度上说,在社会结构中发生这样的事并不新奇。自远古以来。民众总是要把他们的统治者予以神化。据我们所知,最早的王国起始于埃及,以及底格里斯河和幼发拉底河两河河谷。在那些地方,国王、高级祭师和神,集于一人之身。而古希腊,甚至就在产生了理性主义的黄金时代,作为亚里士多德学生的亚历山大仍然被神化了。继亚历山大之后,数不清的希腊和罗马统治者相继被神化。我们想想也觉得奇怪,怎么··个活着的人能宣布自己是神,怎么别的人会相信他是神。他究竟与众人有什么区别呢?然而直至今日,世界上仍有许多地方由国王或女王统治,至少是名义上的统治,他们仍然宣布王权神授、反抗他们或是不服从他们就是不尊上帝。反对宗教就是犯罪。保守派仍忠诚地认为罗马教皇是。上帝在尘世的直接代表。从心理上讲,离活神仙也不是太远,虽然不能与诸神并列,不过也只是稍差一点。

事实上,由对病人进行精神分析的经验已经充分证明,凡是被视为年纪大的、智慧高的、权势重的、能力强的人,会在潜意识中被人当作父亲。任何官僚都不全然是上面强加的,同时也有下面的捧场。地位低的人对统治者的谦卑态度有其潜意识的根源,它出自俄狄浦斯愿望和心理冲突。一位共和国的总统在民众的潜意识中也是被视为父亲。这和上帝、独裁者、神权帝王、帝国神王之类并没有什么两样。其区别似乎仅在于,一个特定的社会或社会组织是否一定要强追人民承认,某个人,或某几个人,确实具有幼儿赋予其父母的那些品质:他们非常聪明,所以无所不知;他们非常强大,所以无所不能;他们非常之好,所以没有罪过和缺点。于是,爱他们和顺从他们就是正确的,就会得到他们的爱和回报;不爱他们和不顺从他们就是错误的,就会受到他们的惩罚。

一个宗教组织或政权系统越是接近以上的标准,就越是明显地看得出那是成人重现幼儿期的精神生活。大多数成人都把幼儿期的精神生活有意予以遗忘,但它仍然活跃在潜意识中,并通过无数途径驱使人在一生之中反复地重现幼儿期的精神生活方式。至于成人世界里的政治和宗教制度,那也是幼儿期家庭境遇的不断复制。这一倾向在当今社会里明白可见,有如在50个世纪以前的社会中一样。

此刻我们要从宗教转到与之密切相关,但更为普遍的魔法和迷信上来,在当今科学时代,说到“魔法”一词时通常是指戏法技巧的表演,故意搞出些违反常识和实用知识的现象以娱众。严肃的成年人绝不会相信魔法师有真正的超自然力量,能把人分成两半又合起来。只有小孩子才会受到戏法的愚弄。但是、现在仍然有一些社会或社会团体,相信科学不及相信魔法,甚至成年人也会认真地相信某些人法力无边,这些人被称为巫师、巫婆、神汉、神婆、圣人、圣女之类。甚至就在西方社会,也只有知识分子才认为魔法、圣迹、巫术之类不合时宜,还有更多的人在虔诚地求教于占星术,信仰“疗病者”的生意依然兴隆。无需为此感到惊奇,在历史的长河中,不信魔法的人多起来也才只是晚近的事。诚然,早在公元前5世纪,古希腊就有少数伟大的哲学家反对魔法,以理性的观点看待世界,不过,终究他们人数太少,他们的同时代的绝大多数人,以及继后许多世纪中的绝大多数人都相信魔法。即使现在,虽然科学和理性主义高度受到重视,魔法信仰和神奇性的思维依然繁荣昌盛。可能他们还会永远昌盛下去。魔法和迷信如此源远流长、广泛分布、当然值得加以研究。

魔法和迷信被简单地定义为一种信仰的结果。这种信仰认为,某个人的思维和言语能够影响、甚至控制别的人和环境中的物体。精神分析学家发现,所有儿童都要经过一个对魔法坚信不疑的阶段。精神分衍文献中时常能遇到“思维全能”的说法。一定程度上,幼儿有此信仰是当然的,因为每个儿童在学会说话的同时,惊奇地发现自己具有支配环境的力量。这非常符合我们刚才给魔法下的定义。此时,他生平第一次能够把自己的想法告诉父母,而父母就会照他所说的做,或为他把所要的东西拿来。这简直就像阿拉伯故事《一千零一夜》里讲的:说到做到。加之、儿童的愿望要比成人的愿望强烈得多、至少相对如此。由强烈的愿望产生幻想,但儿童却以为非常真实。如果现实世界中的严酷事实与他的愿望性白日梦不相一致,比起成人来,儿童更容易不顾讨厌的现实而坚持认为他的幻想才是真实的。每个儿童要等再长大一些才逐渐学会把外界现实与自己的愿望性幻想区别开。精神分析学家称之为现实检验(见第\ref{5}章)。然而,甚至当个体的现实检验能力已充分发展以后,童年时习惯了的魔法式思维倾向依然或多或少的存在。对大多数人而言,是相当的多。

幼儿期思维的另一种表现,在魔法和迷信中也起重要作用。幼儿当初总是把周围环境的一切东西看成与他自己样地有思想、感情和愿望,他以为自然界的一切都是有生命的。后来经验增长,又通过父母开导,才知道不是那么回事。但是,这一信念的痕迹依然存在于成年生活之中,与魔法并存。臂如有些宗教信仰就是强烈的惟灵论体系。万物有灵的信念在艺术作品(如雕塑、绘画)和文学作品(尤其是诗)中都常有所表现。

总之,魔法和迷信都依赖于思维全能,特别是愿望性幻想,以及对自然的泛灵论观点。它们往往和幼儿期本能愿望的此或彼方面有明显的联系,而这种联系的性质是潜意识的。我们已经指出过魔法在宗教里所起的巨大作用,但其作用范围绝不局限于宗教。至丁迷信,则常可观察到,这个人的宗教,会被那个人指为迷信。对于非信徒而言,任何宗教都只不过是迷信。宗教之所以能显得神圣庄严,与对它的虔诚信仰是分不开的。

儿童到了6岁前后,会无·一例外地受到各种魔法和迷信的吸引。在西方城市儿童中有一个常见的迷信,说是走人行道时若踩着板块之间的缝隙就会交“厄运”。有这种迷信的儿童一走上人行道就会半开玩笑半认真地努力避免踩缝隙。如要查明此迷信对某特定儿童的潜意识涵义,那就得对他进行精神分析。要想查明两个以上的儿童其涵义是相问还是有异,实在是很困难的。完全可以想象,相同的迷信和相同的魔法仪式对于不同的儿童可能会有不同的意义。然而,又有别的证据,提示存在相同的意义,因此推想它有普遍性、至少有许多情况如此。

有一个小调是儿童们走人行道时有时喜欢唱的,歌词如下;

踩着缝隙,

折断你妈(或你爸)的背脊。

踩着裂纹,

你爸(或你妈)的腰杆断得成。

也就是那些平时避免踝缝隙的孩子、有时会欢快地一路跑跳过去,每一脚都踩在缝隙上,一边还高唱着这个小调。

看来由以上的现象中可以得出一条合理的结论,就是这些唱小调踩缝隙的儿童,其对父母的敌意,乃至于弑亲愿望。压抑下去还不久,小调里表示得相当明显,足以听得懂。当他们欢快地在缝隙上跳上跳下时,表现的就是这个愿望,只不过有所伪装罢了。婆不是有这段歌词、很难推测此游戏的潜意识涵义。为了对此迷信作出解释,必须假设由对父母的敌意引起了罪恶感。因而、潜意识的想法应该是:“我对于父亲(或母亲)有这样歹毒的愿望,应该受到惩罚。”于是,由此产生了意识中的迷信:“如果我踩在缝隙上(这就是满足歹毒愿望的潜意识象征),我就会遭到厄运。”为了避免惩罚(在意识中就是厄运),这孩子就使用了魔法:只要避免踩在缝隙上。、就能迫使命运之神赐给他“好运”。这件事在潜意识概念之中本来并不神奇,本来的意思是说:“父亲(或母亲),我已经学好了、我知道你不会惩罚我,要是我坏你才会惩罚我。”这全然是现实的观念。反映了儿童的体验和愿望。但是,在意识的迷信和魔法仪式中,双亲已经转变成为全知和全能的“命运”,而迷信者的良好愿望需要以自已的某种毫不现实、毫无实际价值的行为(如不踩缝隙)来加以确证。这种行为的根源在孩子的潜意识思维之中,他在踩缝隙和满足对双亲的敌意之间划了等号。就此例而言,搞出一套魔法仪式并非基于在现实世界所获得的经验·而是源于由幼儿期愿望和恐惧而产生的潜意识思维。

另一常见的迷信关系到“13”这个数字。此迷信流传十分广泛,许多人都认为13不吉利,以致许多高层建筑里找不到“第13层楼”,许多剧院里找不到“第13号”座位。这是为了照颜人们怕触霉头的心理。

有趣的是,这本是一个基督教迷信。其最初的表现形式是,认为13个人坐在一个饭桌会带来厄运。原因在于,最后的晚餐上有13个人:耶稣基督及其12广J徒,与此密切相关的另一个基督教迷信是说星期五也不吉利,因为耶稣基督是在星期五那天被钉上十字架的。这两个迷信有时会结合在一起、如果13日恰逢星期五,那就是加倍的不吉利。

要说虔诚的基督徒当想到耶稣被钉死于十字架时感到悲哀,那是容易理解的。可是,为何要迷信“13”会带来厄运呢?由前面的例子我们可以推想,它必然以某种方式与由于耶稣被钉十字架而产生的潜意识有罪感相关连。让我们就从这迷信的最初表现形式试加解释吧。最初是说,要是13个人一起共餐,就会有坏事发生,说不定有一个人会死。表面看来,这是一个理由不充分的推理,或者说,经不住统计分析的推敲。推理是这样形成的:因为福音书说,耶稣基督是在13个人一起吃过晚餐后被扣留并送上十字架的,所以,只要是13个人同桌吃饭v就会发生同样的或者类似的事。于是,最好的办法就是避免参加这样的宴会,只要人数不是13,参加者就安全得多。非基督徒当然没有理由相信此中的联系,可是,他们也照样迷信13人共餐会带来厄运。

然而,这里显得不充分的只是意识中的理由,与别的迷信一样、它也有更为深在的原因。最后的晚餐就是第一次的圣餐。马太福音第26节写道:“耶稣拿起面包加以祝福v掰开分给众门徒并且说:‘拿去吃吧,这是我的肉体。’他又举起酒杯加以祝福,递给众门徒并且说:‘拿去喝吧,这是我的血液。’”

因此,事情很明显,在最后的晚餐上,基督的众门徒吃了他的肉,喝了他的血。众门徒的食品就是上帝本身。这里我们又一次遇到了一个幼儿期本能愿望:杀死并且吃掉自己的父亲。13人共餐这件事,对于一个基督徒象征了潜意识中的幼儿期愿望:杀死并且吃掉自己的父亲。这是一个充满有罪感和恐惧感的愿望,迷信的基督徒们有如前一例中的孩子们那样,潜意识地告诉自己的父亲,保证自己是个好儿子,决不会干杀死并吃掉父亲的坏事,要是自己干坏事的话,就一定会受到惩罚。于是,他对象征满足自己恶愿的行为认真避免去做,由此便避免了厄运的报应。

许许多多的人对于“13”迷信的起源了无所知,不知道它和最后的晚餐有关系,甚至不知道耶酥基督何许人也,然而这些人照样害怕“13”会带来厄运,研究一下“13”对这些人的潜意识意义那是很有意思的。无论何人,只要在潜意识中感到有罪,在其意识中就会产生一种莫名其妙的、非理性的对厄运的期待,他也会试图用某种魔法式的神奇手段来减轻这种期待的痛苦。由此我们也可以设想。无论何人,要是对“13”或者诸如此类的预兆产生强烈情感反应,以为它会带来厄运的话,这个人大概是因为什么事在潜意识中产生了罪恶感。至于一个人为什么害怕这种恶兆而不害怕那种恶兆,只有对具体的人进行精神分析,然后才能得到满意的答案。有兴味的是,一些精神分析学家在临床实践中观察到,那些在意识中强迫计数的病人v或是在其仪式行为中涉及数字的病人,在其潜意识中常有手淫的先占观念。其幻想也与手淫有关。不能肯定说二者之间有恒定的联系,但似乎比较经常。

总的看来,相信预兆和算命是迷信的重要组成部分。在大众的心目中,魔法师、占星术家、算命者都是一路的角色。无论过去还是现在。这样的人在许多社会里都是重要人物并且受到尊重。算命人通常执行双重任务。他们要预卜未来;并就某项活动是否宜于进行,向社团或者个人提供建议,选择吉时。咨询内容可以由恋爱婚姻、经商贸易、直至外敌入侵的可能性等等。几乎任何一种自然现象都可用作预言的根据:诸如星象变化、鸟群飞迁、日蚀月蚀、掌纹线路、动物肝,、乃至杯中茶叶的浮沉等等。就迷信者看来,一定是有某个人、或某种“力量”,才会使得星移、鸟飞、日月无光等等,用这些现象向人们预示凶吉,告诫人们可以做什么和不订以做什么。基于我们对潜意识精神生活的了解,有理由认为、迷信也和成人对上帝的宗教信仰样,是由童年期对待双亲的态度行生出来的。童年期总是由父亲吩咐什么可以做、什么不可以做、要是不听父亲的话就会受到惩罚。对于儿童来说,是父亲决定他的未来,他的愿望是得到满足还是遭受挫拆,大校全掌握在双亲之予。迷信者在凶吉预兆和`算命人的面前,潜意识中仍然是一个孩子,采取的是恭顺服从双亲的态度、揣度并迎合双亲的意识,博取他们的慈爱和帮助。要是接受这样的解释,那么,占星术家和算命人就是决定人们命运之“力量”在现实生活中的代理人。他们通常年纪很大。迷信者自己像是儿童,而他们就像是儿童的父母亲。

前面我们已经指出,宗教、占星术以及诸如此类迷信的信徒们对待上帝、牧师、魔法师、算命人所采取的是服从和敬爱的态度,他们被当作是幼儿期双亲的成年期代表。然而,由临床经验以及直接观察得知,幼儿对待双亲的态度是矛盾的,既有敬爱和恭顺的一面,也有愤怒和反叛的一面,甚至还有弒亲和阉割愿望存在,有时是前一面占优势,有时是后一面突出。前一面方才已经谈过了,现在再来研究成年生活中敌意愿望突出明显的一面。这样的意识态度和行为大部分起源于幼儿期的对抗和反叛愿望,这愿望在成年生活中仍然存在。不过是潜意识地存在。

在政治生活中经常可以见到新一代和老一代发生或多或少的冲突。这一熟知的现象通常被称为“两代人的冲突”或“代沟”。当今的政治权力属于公众,至少在理论上是这样说的,所以两代人的冲突应被看作是群体现象。在另外一些场合下、两代人的冲突是个人的冲突。或者少数人的冲突。现时,这两个名词也用于政治舞台之外,指家庭里双亲与成年或青年子女之间的争论和斗争。然而,直至精神分析方法被采用之后,才发现两代人之间的严重冲突不是子女到了青春期以后才发生的,早在子女的幼儿期就已起始,通常是在发展的俄狄浦斯阶段。以后的冲突只不过是原始冲突的二版、三版、四版罢了。每一次都有新增添的材料,但主要部分仍是原来的那些东西。很重要的一点是,原来的那些东西在成年生活之中一再作潜意识的重复。有时,就旁观者看来,成年人的行为显得荒谬无理,莫名其妙,与现实环境不合拍。这样的情况一旦发生·只有根据其潜意识的遗留物才能得以理解。潜意识的遗留物在两代人的动机中都起重要作用。常可在年轻一代的身上看到这样的现象,就是他们用以批评和攻击老一辈的那些意识性的理由,全然解释不了何以要表现得那样尖锐和激昂。他们攻击老一辈时明显的激情状态必定有别的原因。

由精神分析的经验指明,这决不是仅用年轻人的鲁莽、急躁、不成熟,或诸如此类的品质就能解释得了的。应该特别注意到,起始于幼儿期的嫉妒和弑亲愿望,此愿望潜意识地持续存在于往后的生活之中。诚然,此愿望对于个体是独特的,但所有的人都有相似的愿望,因而可以将“两代人的冲突”看作是一种循环往复的、多少显得普遍的现象。

至于老的一代,同样是受到潜意识愿望的促动,他们的愿望也是植根于幼儿期的本能愿望。譬如,老一代人可能潜意识地认同于自己的父母亲,仍然潜意识地相信自己的父母亲是全能的,胆敢违抗其权威、必会遭到毁灭牲打击或阉割的威胁。老一代人也可能在潜意识中把年轻一代(他的孩子)与他自己的双亲等同起来。他自己的双亲虽然已经年老或者死亡,但在其潜意识幻想中再生为年轻的一代。无论是两代人的政治冲突还是家庭内亲子之间的冲突,缺乏理性和难以解释的激情都不是只出现于一方。双方都是人,双方都受到源于幼儿期的潜意识本能愿望的强烈驱动而本人可能浑然不觉。至此、我们当然能够理解马克·时温的一则含有深刻的心理学真谛的笑话:“当我17岁时,常为父系的愚鑫无知感到惊愕。当我21岁时、却诧异于短短4年之中他有了如飞的进步。”另一方面,我们又会同意下述观察实为敏锐的直觉:各个园家的领导人总是准备打仗,其原因之一是急切地为年轻一代提供机会·使他们j成为战死的英雄。

在人类的活动中还有另外一个领域,心理因素在其中明显地起电要作用。一般而言,与前面讨论过的严峻的革命﹑代沟和政治相比,对于大多数人这个领域的实用。意义小得太多。不过,它仍然栩当重要,值得注意·特别是精神分析对于它有新颖而坚实的见解。这个领域就是文艺。

在文艺心理学中,潜意识精神生活起怎样的作用呢?首先、在文艺创作或表演过程之中,其次,在文艺欣赏过程之中,也就是说、在文艺工作者的活动和文艺欣赏者的活动两个方面。潜意识精神过程在此发挥了什么样的功能呢?如要严格科学地回答这些问题,必须对文艺工作者和文艺欣赏者都进行精神分析研究。对文艺欣赏者进行精神分析的机会是比较多的,因为许多作治疗性精神分析的病人本身就是文艺欣赏者。虽然,病人对文艺作品的反应,常常不是需要进行分析治疗的核心问题,但分析者总是有机会窥视病人的潜意识精神生活与其意识的文艺欣赏体验之间的关系。如要就二者之间的关系作详尽的、系统的探讨,这样的机会当然就少得多了。如要对文艺专业工作者进行精神分析,这样的机会就更少。当这样的机会实际出现时,出于职业卷的保密习惯,精神分析学家几乎不可能将自己的发现进行交流。在医学文献中、做躯体疾病的个案报告不难,只要隐匿病人的姓名即可。但是,即使隐匿了病人的姓名,要发表精神分析的个案报告那也会困难重重。病人的知名度越高,困难也就越大。如渠病案报告的内容主要涉及病人之所以知名的那一方面的活动、这篇报告肯定别想被发表。精神分析文献中,关于潜意识精神过程对文艺活动影响的资料十分缺乏,其原因就在这里。大多数作者都限制自己只讨论未曾经过精神分析的文艺名家,根据众所周知的传记资料和其他史料,作出有关潜意识因素的推论。Freud的关于达·芬奇和歌德的开拓性的研究论文就是如此写出来的。其他作者们也发表过这方面的推论。据信,其中至少有一部分是基于对文艺界人士进行精神分析的临床经验,但是不可能展示这些个案证据。

尽管存在众多的困难,经过多年的努力。终究还是得到了有根据的若干重要成果。让我们首先讨论文艺工作者这一方面。而在文艺的众多行当之中,我们挑选文学作为研究对象。

文学作品与幻想之间的关系是显而易见的。还在精神分析方法被采用之前、人们便已了解并且常常提到这一事实。而在精神分析学产生的初期便已作过详尽研究的课题之一便是幻想,这是因为神经症的症状和幻想有关。精神分析学家对于夜间的幻想(梦)和白天的幻想(白日梦)都有浓厚的兴趣。而此刻我们最感兴趣的是白天的幻想,因为白日梦和文学作品的关系太密切、太明显了。

由于神经症和文学创作都与白日梦有关,Freud探讨了其间的联系,以图阐明文学创作的若干方面。他的开拓性研究澄清了事实;在梦和白日梦的产生中,起重要作用的潜意识本能愿望和心理冲突在文学创作中同样地起着重要作用。这就是说,作家将其白日梦,将其幻想,制作成文学作品、他希望自己的作品能引起别人的兴趣,能受到别人的欣赏。白日梦一般是为自己,而写作一般是为读者。正如画家和雕塑家的技艺在于善用画笔和凿子、作家的技艺仍在于将能引起读者兴趣的构思制作成小说、诗歌、戏剧之类。作家的构思因使用的媒介(臂如口语、书面语)不同以及文化背景不同而产生变化。可是,无论构思如何变化,作家总之是在为读者编造自己的白日梦。文学创作的核心,其起始点和主要内容,只能是作家的白日梦。因此,认识白日梦的性质和功能,研究白日梦的心理学,对于了解文学创作而言是十分重要的。

白日梦通常关系到未实现的愿望:恋人在白日梦里作爱;儿童在白日梦里成长为端庄漂亮、有造诣、有成就的人;饿汉在白日梦里进食美餐;渴者在白日梦里痛饮甘泉;疲倦者在白日梦里得到休息。通过对自我的观察或询问别人,这样的例证比比皆是、并无需于精神分析学的专门T知识。一个人只要是渴望于什么东两而又有足够的时间沉思,他的愿望就会在白日梦中得到实现和满足。例外的情况当然也不少,同样有不愉快的白日梦,基至非常可怕。不过,在绝大多数白日梦中,意识的愿望得到了意识的满足。此一事实众所周知而且易于确证。精神分析学所能添加的一点意见是:潜意识愿望同样也是白日梦的熏要源泉。在精神分析治疗的过程中,只要有机会对病人的白日梦进行分析,总是能观察到潜意识愿望对其形成起重要作用。有许多幼儿期本能的愿望,程度不同地一直未得到满足,它们一直驱使着个体程度不同地去寻求满足、但是个体本身并不觉察其存在,不知道自己为之寻求满足的愿望究竟是什么。白日梦是使愿望得到某种程度满足的途径之一。

从临床实践中举几个例子,有助于说明问题。一位成年男性病人,有一次到诊室去作分析治疗,就在走向诊室的路上他有了一个白日梦,于是立即向分析家报告。他幻想着自己刚走到街的拐角,就看见诊室的大门前面停着警车和一辆救护车。原来那里发生了可怕的事故,有个病人突然狂暴起来,开枪将分析家射倒在血泊之中。这时,白日梦的报告者对自己的幻想作了修改,他又幻想自己正置身于诊室之中,。与疯狂的攻击者格斗起来,在其开枪射击之前成功地把武器夺下。

这位病人的自由联想是从头天晚上看过的一幕电影开始的,那电影里充斥着暴力和谋杀的场面,兼有赤裸裸的色情镜头,使病人产生了性兴奋。情景之一是谋害者杀死了一个男人然后诱奸其妻。这场面使得病人毛骨悚然又神魂颠倒。随后,他又提到影片里有个角色年纪比较大,使他联想到自己的父亲。并不是这角色相貌像他的父亲,仅只是由于戴的那副眼镜像他父亲的。接下去病人又讲起他的父亲如何可以信赖,自己有问题总能得到父亲的帮助。至此一转,谈起他对分析家随意更动了他的预约治疗时间而不满。

在这一例的白日梦里显示出病人对分析家有意识的矛盾情绪。病人恼怒分析家图自己的方便更动他的预约治疗时间。觉得自己是不受欢迎的人。但同时他又为自己的恼怒害羞,因为他体会得到分析家对他的帮助,对后者总的印象是好的。这些意识的态度在白日梦中的表现形式明显受到前晚所看电影的影响。也就是说、他的恼怒情绪由于有人枪击分析家而得到了满足,而他的友好感情又由于自己救分析家于死地也得到了满足。想必、他对自己的恼怒情绪感到内疚、所以谋害分析家的不是他,而是另外一个病人,而他在幻想之中甘冒危险与攻击者格斗,也表达了友好感情,减轻了内疚。

以上这些动机都是意识性的,病人自己心里明白。然而,情况不止于此。当病人刚人少年,他的父亲便在办公室被一患精神病的雇员开枪打死。他对横死的父亲怀念至深,常常想入非非。他幻想自己当时正在父亲的办公室里,在紧急的一瞬间打掉了攻击者手中的枪,于是救了父亲的命。看来,病人在其白日梦中表达了对父亲的潜意识弑亲愿望和爱慕愿望,同时又表达了对分析家现时存在的意识的矛盾愿望。可以认为,病人潜意识地把分析家与其父亲等同起来,把仍然藏匿在潜意识中的对父亲的情感和愿望转移到了分析家的身上。他的自由联想并且表明,这遭到转移的愿望源于俄狄浦斯情结:其白日梦的促成因素之一是头天晚上所看影片中的性刺激场面、谋害者杀死了一个男入然后诱奸其妻,又有某个情节使他联想起了自己的父亲。换言之,病人对父亲的性嫉妒、弑亲激情、随之产生的悔恨,所有这些幼儿期俄底浦期阶段遗存在潜意识中的愿望,在其走向分析家诊室的路上所进行的白日梦里都得到了意识的满足。

日常生活中的意识愿望因周围环境、日常需要、对事物的印象和兴趣而发生改变。幼儿期本能愿望则持续终生而无改,虽然大部分是在潜意识之中。这样一来,一个人的白日梦既随其意识愿望经常在改变,又随其潜意识本能愿望和冲突而万变不离其宗。前面介绍的这一例病人从少年期开始。反复在白日梦中幻想拯救父亲的性命,他的幻想也总是涉及弑亲的主题。

另一例病人、自从童年开始便反复幻想自己是一名军人并且开一挺机枪。在白日梦中。他杀死过成千上万的想象中的敌人。与此同时,他又有一个“亲密的伙伴”,在每一次白日梦中,这名“亲密的伙伴”都受到了致命的创伤,而因病人的自我牺牲的英雄行为得到了拯救。此例病人的白日梦中,充斥军事场而是受到外界事件的影响。当时正值第二次世界大战,病人有意识地渴望自已成长为有男子气概的军人。但是,潜意识的决定因素却更为复杂和更为重要。病人在现实生活中的玩伴,他的“亲密伙伴”,实际是比他年幼4岁的胞妹。这个妹妹是母亲最为宠爱的宝贝。病人的强烈嫉妒之情及于全家,但无从公开表露,而是在爱国主义的屠杀幻想中得到了宣泄,同时还表现为各种临床症状以及竞争活动的抑制。它又表现为希望自己是个女孩。在他的幼稚的想象中、变成。女孩意味着失去阴茎,此一想象又引起他的强烈焦虑。这样一来,在他的重复呈现的白日梦中、便不是他变成女孩、而是他的妹妹变成了一个男人。除此之外,他还有一挺巨大的机枪在手,这也象征性地使他消除了失去阴茎的疑虑。最后,为了断然否认他对妹妹的仇恨和希望她死的动机、在每一次的白日梦中,他都必冒生命的危险去把她救出来,然增又细心地护理她的伤情。

这些有关病人白日梦的知识对于理解文艺创作心理学有何帮助呢?我们能就此作出哪紫有根据的结论呢?有一点可以肯定下来,就白日梦与潜意识愿望的关系而言,作家与普通人不会有什么不同之处。作家的白日梦,同样会受到幼儿期未现能愿望的激发,至少部分如此,作家的幼儿期本能愿望也仍然活跃于其潜意识之中,虽然本人觉察不到。作家的文艺创作,其素材来自于白日梦。因而,由对其作品的考察,往往可以推断出某位作家幼儿期愿望以及其心理冲突,至少在许多情况下,这样做取得了成功。当然,考察内容不局限于公开发表的作品,还应包括作家的写作提纲、笔记本、原始草稿等等,越多越好,因为这些东西才是最接近其白日梦的素材。研究中有时可以发现、某位作家偏爱一种或几种题材。一旦找出这些题材与作家幼儿期本能愿望的潜意识遗存之间的联系、真相立即大白。以海明威为例,他偏爱男子汉、大丈夫气概的题材。他的小说情节里充分显现出坚韧不拔和刚劲有力的风采。而由我们的临床经验证明,每当病人在白日梦里过于坚持强调男子汉、大走夫气概时,其潜意识幻想中总是充满阉割恐惧。海明威是否也如此呢?我们猜想情况可能是这样。在他的小说里至少也能找到一条证据,书中有一位主角在战斗中打掉了生殖器。再以陀思妥耶天斯基为例,他永远执着于犯罪、悔恨和惩罚的主题。《罪与罚》是他的最伟大的作品之一,即使以此作为其全集的题名也仍然显得贴切。他为什么毕生离不开这个题材呢?由其悲惨的身世,我们可以得到部分的理解。他在童年便目睹了父亲遭谋害而死。

别的作家们不像海明威和陀思妥耶夫斯基那样执着子单一的题材,据其作品判断,在他们的白日梦里潜意识主题有比较广泛的演变。然而,令人印象深刻的是,成人文学作品的基本主题与本章前面讨论过的神话和童话主题都一样,都源子幼儿期本能愿望和心理冲突。无论作家怎样艺术加工,精心掩盖其本质,无论作家怎样有意识地相信其作品具有这样或者那祥的目的意义,他们实际呈现于读者面前的永远只能是对其潜意识本能愿望的反应,永远只能是对其白日梦的改造加工。作家也是人,他们不可能做出除此之外的事情来。

在文艺领域之内的其他艺术创作也和文学作品一样,与白日梦有密切的关系。不过,那些不用语言表达的艺术形式比文学作品更难说得清楚,不易使人信服。若非亲自对艺术家进行过精神分析,对其白日梦中的潜意识决定因素有所了解,自难消除f疑。意识的决定因素容易被发现、潜意识的决定因素就很难得到确证。因此,精神分析文献中关于文学作品及其作家的研究较为多一点,对于别的艺术形式,研究论文就很少了。

文艺欣赏者的人数大大超出于文艺家的人数。因此,直接用精神分析方法研究文艺欣赏者的潜意识动机,机会远多于对文艺家的研究。诚然如前所述,病人对于文艺作品的反应通常不是其自由联想的重点所在。可是。病人有时还是会谈到他对一本书、一场电影、一幕戏剧或其他表演的感受、从中便可以得出一定的推论。部有强烈持久感染力的文学作品,必定是它的情节能够激动并满足读者们潜意识俄狄浦斯愿望的某些重要方面。如果这作品是一部悲剧,那必定是它的情节与读者们的潜意识恐惧及自我惩罚冲动合了拍,而这些东西又是和潜意识本能愿望密切联系的。

精神分析学家很早就发现幼儿期性欲这个主题在文学中的重要性,随后又有许多作者对此加以肯定。诚然,它只是使一部分文学作品得到成功的条件之一。单有这一条是不行的,要使一部作品具有强烈持久的感染力,还必需熟练掌握语言艺术,精于安排情节结构。善于塑造人物性格、懂得突出戏剧效果,有描写景物的能力,能使对话切合情景,有独创性等等。所有这些条件无不重要、但是,真正使作品永垂不朽的核心条件是,它的情节能够满足读者们强烈而急切的幼儿期愿望。

回顾伟大文学作品的情节便足以证明此说不虚。不朽的《哈姆雷特》讲的就是胞弟谋害亲兄并娶嫂为妻v毫不含糊地描写了乱伦关系。王子哈姆雷特杀死叔父和母亲以报父仇,自己也死在叔父手里。再看托尔斯泰的名著《安娜·卡列尼娜》,女主角离开了自己的独生子和老得可以当父亲的丈夫,去和年轻的恋人住在一起,可是她又自毁幸幅,不断折磨恋人以使之和自己疏远,最后自尽于火车轮下。《卡拉玛佐夫兄弟》是陀思妥耶夫斯基的力作,其主要情节是对弑亲罪的惩罚,由此可以清楚地看出,人们对于乱伦和弑亲这两个主题有永恒的潜意识偏爱,然而在意识中却企图掩饰此一事实,甚至激烈地予以否认。

*

以上我们讨论了性格特征、认同、兴趣爱好、职业选择、配偶选择、童话、神话、传说、宗教、道德、政治、嵬法、迷信、代沟、革命和文艺,为说明常态l神功能列举了广泛的样本。我们试图指明,植根于幼儿期本能愿望的潜意识心理过程,由愿望产生的恐惧、悔恨和自我惩罚冲动,以及因愿望和恐惧相击而产生的心理冲突,在以上的每一方面都起了重要的作用。我们希望以上样本的讨论足以说明幼儿期本能生活的力量、普遢性和恒久的影响。幼儿期本能愿望终生持久存在于潜意识之中,由之产生的心理冲突在心理功能的各个方面一而再、再而三地反复呈现出来,有的属于常态,有的属于病态,直至生命完结之时才会终止。

\signatureD

\chapter{当今的精神分析}\label{10}

本书最后这章,既是概括性的评论、总结,也是对未来的展望、使读者『解当今精神分析的地位,『解精神分析对当代所做出的贡献,并且指出精神分析在将来可能起的作用。因此,这一章比前几章更能广泛地表现作者个人的观点、经验及预见。

每一个科学的发现都改变着世界。某些发现改变得多些,某些改变得少些,但每一次科学发现之后、世界绝不会同以前完全一样。有时,一个发现的影响具有实用价值、像蒸汽机的发明,使十九世纪的工业革命成为可能;有时则是对思想界,对人本身以及宇宙的看法造成影响,而不是对周围的物质世界。就精神分析来说,兼有下面两方面的意义:在实践方面,是一种治疗的方法,另一方面,是了解人类天性最好的信息来源。

早些时候,在评论精神分析对思想界的作用时,Freud将精神分析的发现与哥白尼、达尔文的理论相比拟。而Freud正好出生在达尔文的《物种起源》发表的那一年。哥白尼的日心说,表明我们所在的地球不是宇宙的中心,仅仅是围绕太阳运转的行星之一而已。进化论也从生物学角度将我们置于适当的位置,即我们不是像《圣经》中所说的那样,被创造出来统治世界,而是亿万年前最初的蛋白质分子所进化而成的千百万个物种中的一个。像Freud表述的,精神分析告诉我们,我们甚至不能把握自己的心理活动。我们被潜意识的心理过程、被欲望、被恐惧、被冲突以及被幻想所支配、所指挥。

众所周知、这种对公认的信念的挑战会使大多数人感到不安。大多数人会因为这种观念而不愉快。为了使他们自己感到舒服,为了避免心理上不舒服,或将心理不适减到最低限度,他们就要抵御新的思想。对人们这一类行为的潜意识里的恐惧进行探讨,将会是有趣的。Freud强调上述提到的情况是自恋的作用,认为、当一个人自以为了不起的感觉受到伤害或威胁时,就会感到不愉快,会把从儿童期就有的潜意识里的自卑感和无助感,连带它们所导致的冲突一并激活。

可是,到目前为止,精神分析已不再是个新鲜事物了。当人们接受了达尔文和哥白尼的正确思想成长起来之后,这些思想便产生了愉快的感觉,而不会像最初提出来时那样的不愉快。学习了物种起源,了解了自然界,知道了宇宙的大小,对大多数人来说,是令人兴奋、令人高兴的事情,就像科学普及书籍和文章所起的巨大作用所证实的那样。另外,人们还可以探讨潜意识里的欲望,以及它们在儿童期的起源。

这里我们要指出的是,像本书中所描述的精神分析的知识,与物理学和生物科学中的基本理论是一样的重要。这些物理学和生物科学的基本理论开阔了我们对世界的认识。一旦我们研究了化学、物理学、生物学、天文学或地质学之后v我们对世界的认识就完全改观了。拍打着海滩的潮水、池塘里的冰块、我们脚下的土地和岩石,以及头预上的银河都成了新的、与以前不同的东西。尚样,精神分析能使我们比以前更好地了解周围的人,为我们观察了解包括我们自己在内的人类提供了新的方法。

精神分析的发现,使我们能更加准确、更加充分、更加完整地了解作为一个独立的人的心理活动和行为。从物理学,我们知谊,自然界的客体并不是像我们的感官所感觉到的那个样子,甚至我们自己的身体也不是像我们感觉到的那样,而是像其他一切物体那样,是由无数的分子聚集而成的,而每个分子则是由原子、电子、核微粒所组成。这些物质都处于持续地高速运动状态。完全一样,由精神分析我们知道,每一种想法、每一个行动比起Freud用精神分析调查法之前人们所想象的要复杂得多。我们之所做、所想,部分地是由于本我的力量,也就是儿童时期的本能愿望所支配;部分地是由对那些愿望抵制的防御机制的作用(自我);部分地是由于道德上的要求(超我);部分地是由于我们所处的外部环境的紧急事件及机遇所构成。掌握了精神分析的知识,人们就能知道内驱力以及冲突在人类动机中所起的重大作用。Kris(1947)写道,精神分析就是将人类的行为看成是冲突——这是对人性极其深刻的理解。人类具有动物的天性、其儿童时期的经历所形成的欲望构成了驱使人一生活动的主要动机。内驱力、执行或防御内驱力的自我功能,焦虑,内疚,冲突,以及在心理生活中潜意识过程的巨大作用,都是精神分析的立足点。这些观点在当前依然是首屈一指的。应用新的研究方法将会给今后人类心理学带来什么,只能由人来猜测了。可到目前为止,精神分析的方法是最好、也是最有效的。在精神分析中还有许多无法解决、令人怀疑的问题,但有一点是肯定的、它的应用为曾经是模糊不清的人类心理学领域带来了光明,使得人们第一次真正地更好地理解人类自己。

精神分析的前景如何?精神分析尚未被发现的领域,或者说当前精神分析学家所争论的是什么?当前精神分析学家积极调查的领域是什么?

在科学领域里,预测未来的前景总是要冒风险的。就在写书的这一时刻,某些新的发展,某些无法预料的发现可能正在进行着,在完全无法预料、不能预测的范围内影响着事态的全过程,这种可能性在科学探索中是很自然的。科学家都相信,科学是一种无止境的追求,无休止的探索。“无止境”、“无休止”意味着无限,而这一概念对于一个人来说,确实是难以把握。然而,有一点是非常清楚的,科学上隔不多久就会有新的发现。人们的努力已经触及到这个世界的表面,而他们的探究将会很快地被完成。

目前(1974)。对精神分析的兴趣不断扩大。例如在美国,受过训练的精神分析学家是1940年的10倍,是1930年的20倍,当然总的说来仍是个小数目。1971年美国精神分析学会花名册上列出了1332名成员,对于一个拥有两亿多人口的国家来说这不是个大数目。1930年,全世界为数不多的精神分析学家多分布在维也纳、柏林、纽约和伦敦。今天,精神分析学家的数目不断增加,他们工作在拉丁美洲、西欧的很多国家,以及加拿大、美国、澳大利亚、以色列、印度、日本。从特拉维夫到奥斯陆,从布宣诺斯艾利斯到蒙特利尔都有精神分析的教育、实践和研究中心。从事于心理卫生工作的精神科医生及有关人员对精神分析的兴趣还会升温。精神分析的基本知识对于任何形式心理治疗的应用都是合理的。没有这样的知识,人们如同在黑暗中打枪,失去了立足点。此外,如果一个人想实践心理治疗的话,要建议他尽可能多地理解自身的心理冲突。对自身缺乏充分的了解、自己内心主要的心理冲突没有得到圆满的解决,他会很容易地受到病人的冲突、潜意识愿望和恐惧的影响,也就很难或者无法控制住自己。这对病人时常是有害的。也就是说,想要从事心理治疗的人·首先应当分析自己。对自己的精神分析在训练中十分有用,而且往往是最基本的过程。

前不久,由于心理治疗应用的开展,心理治疗的重要性被越来越多地认识。如果这种情况持续下去,可以有把握地预测,精神分析的教学和实践会继续增加。只要任何形式的心理治疗广泛地被应用、精神分析在治疗上以及知识来源上都将起到重要的作用。

精神分析已经远远地扩展到心理疾病领域以外了。它大量地涉及了正常心理生活的许多方面。事实上,精神分析向那些对社会科学或行为科学感兴趣的人,向法律上的专家和文学艺术界提供了人类的心理知识——有关人的需要、恐惧和冲突,儿童时期发展来的动机,以及成年后它们所起的作用。在上述的领域中,精神分析的知识对专家们是十分有价值的,尽管其价值刚刚被认识到。那些将精神分析应用于他们感兴趣的领域的人是开拓者。可以预期,在这方面的工作会有很大的进展。有朝一日,精神分析的知识会成为对从事人类及其行为的研究人员进行教育的内容。

再简单说说精神分析本身目前特别有意思的领域。目前,主要的兴趣依然是临床实践和精神分析的教育,即训练临床实践的分析者。多数的精神分析学家热衷于在精神分析法的应用中如何提高自己的技能,以及准确地对临床观察材料进行心理功能和心理发展方面的阐述。他们主要的兴趣是怎样更好地理解、更好地治疗那些寻求帮助的人。另一个与之密切相关的兴趣是·帮助那些想实践分析自身以获得必要的知识和经验的人。

除了上述这些主要兴趣外,与精神分析有关的专业人员的精神分析教育越来越受到重视,特别是美国和德国的精神分析教育研究所。然而,到目前为止,这类计划仍腐早期发展阶段。人们期望这一领域的活动会大大增加。

近些年来,一部分精神分析学家致力于研究儿童发展、并吸引了某些人的兴趣。在托儿所中对儿童行为的观察,对人们早年生活的心理学做出了重要的贡献,并对儿童和成人的精神分析治疗产生了影响。这类工作的中心是在伦敦或美国的一些城市。这是一项耗时耗力的工作,因为它需要花费许多年去观察儿童及其家庭。

与儿童发展相联系的另一件有意思的事情是,精神分析学家近年来越来越注意在最初两年中儿童所受到的成人照顾对儿童心理发展的作用。到目前为止,我们有关这方面的知识还不多,因而还不能下什么结论。随着知识的积累,那些模糊的令人烦恼的问题会有清晰的一天。比如,我们知道恋母阶段(大概2.5-6岁)对每一个儿童来说都是困难的和躁动的时期,是心理发展的关键阶段。在儿童那些年代所发生的事件,会影响其今后的发展,可能是正常的,也可能会是病理的,并持续一辈子。每一个儿童都有恋母冲突。为什么它对某些儿童所造成的影响要比另一些要不利,为什么它使一些儿童产生心理疾病,而对另一些儿童的影响却属于正常?

回答这个问题,似乎要通过恋母期自身的事件、性体验、恐怖事件、死亡或遗弃、身体疾病来提供。可事情并非总是如此。Freud在早期的。工作中,就指出了体质因素的重要性,而这种体质因素与刚刚提到的经验因素不同。除了体质因素之外,儿童生活的最初两年中发生的事件决定了儿童后三四年中对应激事件的反应方式。也就是说,在最初的年代。照顾儿童的成人所采用的抚养方式起了作用。由此可以更好地理解为什么恋母期的心理应激对某些儿童产生更为不利的影响。

毫无疑问,别的有价值的成果还会从目前及今后对儿童头两年的精神分析研究中获得,而这些成果又反过来对于照顾儿童有直接的实用价值。

对于事件,精神分析都是研究个人的生活史、生活中的重要事件、一件事件与另一个事件之间的联系,以及它们的心理原因及心理结果。由精神分析而得到的个人史与文学传记的个人史不同,更不同于讣告或颂词。它较少涉及可见的生活部分,主要关心的是人们隐藏起来的那部分,不仅对于周围的人是隐蔽的,对于自己也是隐蔽的。这些隐藏的力量和事件构成了每个人现实生活的基础,决定了人们的现实生活,形成了人类的生活方式。

为此,精神分析的真正目的是要得到精神分析式的个人史,进而去认识人的心理和行为。我们试图根据所了解的过去和现在的情况来预测未来。这种预测可能是有趣的,但它毕竟比不上真正发生的事情那样具有魅力,毕竟比不上看到将来变成现实那样具有魅力,毕竟比不上透过目前的情况而了解到过去更具有魅力。
\signatureB
\end{document}