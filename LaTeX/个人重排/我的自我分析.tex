\documentclass{article}
\input{newcommand.tex}
\title{我的自我分析}
\author{章小明}
\date{\today}

\begin{document}
\maketitle
\tableofcontents

\section{序言}
    说实话,我说不准我今天能写多少东西,因为我平日在脑子里总是想着要把这些东西写下来,但是实际上本人还没有一个清楚的脉络使得我可以写作。另一方面,我写这些东西也是为了我能整理出我的一切逻辑。而这份总结我也希望时时更新,作为我对自己一份完善的整理。而我或许应当深入我的日记进行搜寻。

    我试着把它们写成描述性的文本,尝试站在一个理性、中立的态度去阐述本人、以及本人做的事、想的事。

\section{基本内容}
\subsection{家庭、父母与童年}

\subsection{精神状态}
    \paragraph{2022年12月15日} 本人从高二(2018年)确诊抑郁症开始,便一直处于心理感到痛苦的状态。这种状态是由神经症引起的还是有明显的生理因素干预,我更倾向于前者猜测。自那时至现在,本人的精神状态在幅度不大的不稳定中逐渐变差,而现在已经变得非常痛苦了。本人目前还没有出现特别差的症状,如过分的厌食、呕吐、过量服药、躯体化、暴力倾向、自杀尝试。已经有过自残行为,但很长时间内都没有再做过了。

\section{人际关系与亲密关系}
    我会觉得我的人际关系和亲密关系很烂,因为即使我有或有过很多朋友,但我与它们的联系并不持续或足够频繁到我和它的关系很亲密,往往都是隔了很长时间才会回复。因此我会觉得,哪怕在我最熟悉的网络世界中,我的朋友也很少,或者说我的网友和我的联系非常短暂。我所关注的人往往是我所不那么熟悉的女性,因为我一方面有着对女性的渴望之外,我又恐惧更亲密的关系。我不明白为什么,我会把我所不熟悉的人看作未知的渴望的对象,或许是因为那是我的投射。但另一方面,我想我需要更多的和我有联系的人建立联系。
    
    但坏消息是对很多人我做不到这些。我会觉得如果把我的某件事告诉某个人,那么其他人便得不到了,也就是说,我放弃了与其他人建立关系的潜在可能。因此我常常不和别人建立联系。也有可能仅仅是我怕麻烦——一方面我渴望着人际关系,另一方面我害怕进入那样的关系。

    我也不太明白如何和一个人建立好的关系,我会急切于与之建立好的关系,但实际上投入时间和精力是一个非常漫长的过程。我想我只是一个所有人的一般朋友,一个在我的社交圈中游荡的幽灵。

    现在,我会在很多展露自己的内心的时候害怕遭到伤害。而在过去,我更希望把这一切都给别人看,希望别人能理解、关心我。我想现在是因为,本人遭遇了很多社交上(与人相处上)的挫折,伤害了我敏感的内心。

\section{女权主义}
    感觉在很多遇到``渣男''等男性伤害到女性的情景下(这就是她们所说的性别暴力)本人会闭口不谈。究其原因还是我是个男的,我会为此感到羞愧和被攻击。我会觉得我也参与其中,我也对女性施暴了;也有可能是,我感觉在这个场景下,女性也会责难我这样的男性。而让我去否认男性的本质、本体,我会觉得我被否定了一切,因此我尝试对抗它,比如保持沉默或反抗之。暂时,我不知道有什么好的办法能告诉我这些东西和我容纳在一起。

    我时常会陷入这样的心态:在普遍不信任男性的环境下(厌男、认为男性是怎样坏的、``女人的不幸从心疼男人开始''、以及诸多激进女权radfem),本人会感到无法融入,尽管尊重、理解,但是我依然会觉得这些东西让我受到了屈辱、不快。我会感到被排斥、不友好、被攻击,因此我回避这些内容,为了我的身心安全。
    
    是的,我知道很多人遇到过并厌恶诸多甚至全部男性,遭遇过那样的性别歧视或暴力。但我是男性,我天生就秉持着这样的``特权'',好似从出生开始,我就背负着在父权制社会中的得利者的身份,而我无法和她们——受害之人达成共识。而我作为一个有同理心的人,一方面的确同情并支持很多女权事业的进展,一方面被排斥,而另一方面我也的确做过很多不好的事。是的,我没有主动做过,我也没有越界而保持尊重对方的态度。·我曾经在被询问``要不要看看我的逼''的时候选择了同意,对方是一个初三的女生。·我承认,我对很多女性都抱有``要是我能得到她就好了''的想法,而这种``玷污友谊''的想法使得我反向形成为和所有女生都保持恰当的距离。

    我希望能加入,或者说被接纳。这里有两条道路:要么我选择认为,她们所给我的认可是不值一提的,首先将她们认定为远方的、非人的,我没有必要追逐她们的标准,而做好现在这个社会中的自己就好,在生活中自然流露自己的同理心
\end{document}