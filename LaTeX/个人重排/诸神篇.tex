\documentclass[UTF8]{article}
\usepackage{ctex,geometry}
\geometry{a4paper,left=2cm,right=2cm,top=2cm,bottom=2cm}
\title{诸神篇}
\author{科大一学生}
\date{}
\begin{document}
\maketitle
\section{诸神篇}
\par 河漫上来,我就有了这样的预感。
\par 既不是我的亲人,许多我未曾相识的人,在这次轮回中既定的事实。
\par 我终将写一篇祭文,而不是诗歌。围在篝火旁聆听我的冥冥之灵,河原上以歌声相认的亲族,我的血脉,风声,雨声。
\par 水自天穹淌落,那么有晦暗的哀情而起了祈祷的意念。曾经写就以成就的诀别,如今又要悄悄归来了。
\par 许多不能分说的记在前面,从此分不清人间与那条河的界限。梦游时分,我们相见时分,直到泣下。只有悄然在运转,碾压过平原,高原,海洋。
\par 我会重新携带财宝来到你们中间,预感,如同荆棘之上,柔软的木花。
\\[0.6cm]
\par 再次凝望时,纷纷不舍可以化作泡沫影,在阳光中上下纷飞。
\par 凡是语言成就的,最后都会溶化在海水中。愈发咸湿,鼓起铁锈的味道。凡是语言成就的,是我们的情人。
\par 这个世界曾待我以温柔,我报答以残忍。这个世界的黑色一面吞噬我时,我亦吞噬它。它想念时,我亦想念它。总是有时间,今日,昨日,我似乎舔舐着那根弦,勾着我的心脏。用温热的舌头,弹拨着。
\par 我的爱人,此祭文是祈祷诗人落入地狱之文。诗人该入地狱,并非出于憎恨,也非忏悔。许许多多深刻的情感在另一层面上微微颤动。
\\[0.6cm]
\par 河上堆积着,氤氲,起伏,喘息。河上只言片语。
\par 起意写下此文是永恒的,也是人间漫长的。它将会刻在我的骨骼,缠绕在我的梦境,如毒如血,浸没了我的囚魂,使我惊叫并狂笑。
\par 此文不能赠予谁,也不能留给谁。这是我对自我的反刍,并且重新拥抱文学的一次尝试。
\par 诗我写不尽此情的,所以有絮语。但这些终将实现的事物总是刻在转轮上,永远到来着,我又何求。我只有奉着,持着,悄悄走过,深雨的回廊。
\par 来到河上,如临桥上,如临花园,如临穹庙。既然预言已然起始,那么这场漫长的预告仪式就请永远继续下去吧。是在火中,或在水中,在灰中,在骨血中,每一瞬间都会有预告发端。
\par 那么结尾在何处呢,大概是没有了。这是我很不确定的地方,当我踏上这条路。数年前,刚起了预言的意味的我,这样一个少年,在陌生的时空之中,他会告诉我什么呢。
\par 一切用生命写就的,也会变成生命。
\\[0.6cm]
\par 繁琐地勾起语丝,隐秘自身,沦落到下下策。
\par 只是记下,单纯地记下,自起始以来的全部。记下愿望,记下我的朋友们,记下诸神篇章。诗歌只有放到水中再淘洗。
\par 因我全不知这样漫长的意念究竟何种模样,只有全部记下。这样,这是一部书,关于诗人的书,像是燃烧着,飘飞到晚霞深底的目光。这是一部情人的书,是旅人的书,万物的奥义书。但首先还是祭祀的书。
\par 愿你与我们保有一样的言语,智慧随着语言发端,而世界伴着智慧起舞。这是一部舞蹈之书,也是一兴之时的书,沉沦的祭文,不可听闻的祭文。
\par 因为谁都不能寄予,这是寄予自我的祭文。我们,还有诸神,连着我的情谊会共享这份祭祀的仪式。
\par 我们,还有诸神,会共同注目诗人,直到他堕入深渊。或者升入巅峰。
\par 这些都并非归宿的,所以我们也不能预告。此诗人会最终恸哭或是长笑,大概,都无所谓了。
\par 只有此文目的为了更深的情感,更深刻的文字。祭文,或者仪式,或者祷告,万物创生,销毁,言语创生,销毁,只告一声。
\\[0.6cm]
\par 这就是诰文,尽管写在前面的言语实在过分冗长,但还可以无限长的。等待,等待可以无限长,而名字,步伐,绝望,既不增长一分,也无消减一分。
\par 我的朋友,你再见我时,我会变成花朵。有六瓣,透明,遍布着刻痕。
\par 我会把诸神篇全部的预告印刻在瞳孔之中。如同我曾把复活的诗文刻在塔上。我知道我终有地方要归去,眠在,溢满悲伤与欢欣,种下欲言又止的一些。
\par 终究我癫狂或并非癫狂,裂纹无可弥合,却悄然生息。蔓延开来,发出清脆的,如同钟声。
\par 钟声响起来了,悄然,并继续到永远。

\section{笑神篇}
\par 在人间,清醒与混沌不能分辨的时候。笑神和哭神在我的身边。
\par 有着捉摸不定的笑。危险的笑。吞噬掉我的笑,溶解我的笑。张牙舞爪的笑,深深哀恸的笑。神秘的笑,不全是欢愉,毫无理由,没有始终的笑。开心地抚动大地的笑,扯动流云与群星的笑。
\par 当饥饿与苦思深深压迫着我的神经时,颤抖的笑,愤怒的笑。肉体的贫乏与空虚,与强健与欲求时,一样在笑。仿佛花朵在笑,土壤在笑,岩峦扯着犬牙在笑,对一切失望后无言的笑,抛起命运时躲闪的笑,隐隐在癫狂与沉静之中,同样沉重的笑。
\par 像一个诗人一样,躲在柱廊远端,静穆的笑。刻在泥中,闪电与冰中,咧开一道诡异的冷笑。转而又像山风,变成篝火上,喷薄的熔岩一样,柔软而明媚的笑。
\par 笑神在想爱时发笑,勾起脚跟发笑。勾起我的魂魄,使我圆舞发笑。一切愤懑的,勉力的,过分夸张,华而不实的笑。跑到漫山遍野,种满笑,笑的叶底,熟睡的蠕虫的笑。那些垂死挣扎的蝴蝶蜂蛾的笑,以及缓缓屈伸的口器,低低颤动,在笑。
\par 笑,诗人的符号,秘密的最先。笑,被我刻在水上,这位我水上第一见到的神灵。
\par 惨淡的笑,眨眼,并努力扯起酒窝。担惊受怕,寒毛倒竖,只静悄悄笑了。笑,在生灵的脸上浮现,然后淡去,种入坟墓。墓碑笑,旋即凝固,成就墓志铭,也是笑。回忆笑,预言笑,饮酒笑,使酒杯不安分地摇摆,红尘笑,痴人随我笑,笑笑,大笑你们笑,在冷静时刻笑,癫狂笑,扯动头发笑。
\par 像我的将军,在笑。他笑仇敌,笑自己,也笑他的战士,笑兵器。他面无表情在笑,横槊笑,纵马笑。江山笑,月夜笑,大雪大风呼啸笑,掩盖掉枯骨灰笑,残肢与断腿肃穆笑。
\par 笑金钱,笑你们的宝藏,因为我的宝物就是笑,我挥霍笑,也聚敛笑。我挥舞笑,我闪烁笑,我拥抱并亲吻笑,旋即鞭打并怒斥笑。堆垒笑,推平笑,祭祀笑,杀戮笑。
\par 这笑神,笑神来了,抱着满腹的谜语,推动着那个巨大的泡沫,终于来到山顶了。笑见我,笑跪我,笑给我作揖。它不平地笑,怨恨地笑,同时又开怀地笑,袒露胸怀与心神,就是一个个微笑的泡沫,泡沫里住着那些预谋不满的笑,那些被召唤却压抑的笑,或者干脆被抛弃的笑。
\par 笑神酿着蜜,由一些无关紧要的笑酿的蜜,或是毒,或是不可听闻,却没有半点虚假的琼液。啊,这蜜淌到世界上,世界就发笑,淌到罪人上,罪人就发笑,淌到诗歌上,诗歌就发笑。最后,笑着癫着痴着,淌到笑神的心上,带来许多闻所未闻的笑。
\par 奥义的笑,全知的笑,或是无能的笑,茫然的笑。紧紧盯着观众的笑,抿着紧张的嘴角汗流浃背的笑,光鲜的笑,丑陋不堪的笑,残疾的笑,消磨的笑,慌忙逃窜的仰天狂笑。
\par 似乎单单收集形形色色的笑就足够好笑,把这些笑分类归档也是特别好笑,赐予每一只笑一个名字也令人发笑,用言语描述并一一例举记载也超级好笑,只要这样的笑永远持续就会有新的笑一只又一只产生下去,所以才有这样一个笑神。
\par 笑神蹲在门前,笑神蹲在左边,而哭神则伏在笑神的右边。笑神每笑一分,哭神就哭一分。
\par 听见吃吃吃的笑,有喉喉喉的笑,有呜嘻嘻的笑,笑神给他的笑加上各种声音。只要听到笑声就会明白笑的前世今生,就可以叫出笑的名字,可以跟它做上朋友,可以爱它,也可以恨它。只要看见,或者听见,或者干脆笑它。
\par 当我笑一种笑,我从无数的笑中奋力地选一个,慎重地挑选一个笑远比轻佻地挑选一个笑更好笑,所以这往往是没有自觉的笑,难堪的笑,诡谲不讲情理的笑。我也会认真地,我与笑攀谈,询问它是否愿意被笑,这时我就是情不自禁笑,这份认真如此好笑。
\par 笑神走近我,笑神来了,连着他的信徒,和一些狂放不羁的笑,美丽的笑。
\\[0.6cm]
\par 我见到一些消逝的笑,一些流淌的笑,被像我一样的人刻划过的笑,被打上烙印的笑,不得不背弃笑神,颤抖与瑟缩的笑。
\par 对于这些笑我亲如血亲,但我却不敢触碰它们,这一点让许多自嘲的笑爬上我的脸。
\par 羞赧的笑,或者不好意思的笑,自惭形秽的笑,睹物思怜的笑,几种笑挤在我并不宽大的脸上,让我有些难过,旋即,只有难过的笑。
\par 丝毫无法抑制这样的笑,在属于笑神的篇章,我也是一位笑神。我原本是任何神灵的,但这场仪式之中,我只能是笑神。我只能笑自己笑,哭自己笑,并嘲讽与愤恨自己笑。
\par 我见到那媚艳的笑,妍丽的笑,有着安静与火热同质的笑,那是女人的笑,也有男人的笑,欲的笑,灵之笑。
\par 相较那些奇诡的笑,我更害怕这些单纯的笑,纯净的笑。所以我只有强笑,拿孱弱的笑应对强硬的笑,拿欺骗的笑面对真诚的笑。我真是一个恶魔,我是这样一个诈骗。
\par 笑神是诈骗的,哭神也是诈骗的,渐进的笑与渐进的哭,笑的本真,笑的名字。
\par 当我念诵笑的名字,我才对笑一无所知。我走进笑神,这一抹伪装的笑,秘密的笑。我走到笑的中间,思念笑,承认笑。主动拥抱那些看起来无知的笑,愚昧的笑,在闪亮的笑与温柔的笑面前望而却步。
\par 这是我的人性,叫人忍俊不禁。在我的意念之中,一切与人有关的事物都愈发好笑,叫我先用冷笑,再用蔑笑,再怒笑,狂笑,癫笑,傻笑。
\par 轻轻触碰,裂纹如笑,疼痛在笑,迷惘笑,盈满笑。大地颤动在笑,火焰颤动在笑,目光摇曳在笑,我与你们是一类的,与他们不是。我爱你们是一类的,他们,则全不爱。爱之笑,漠然笑,让我用笑,先笑,不管是什么笑。
\par 是了,我是在寻找人性的途中,跌到了河上。我去往人的边界,到门上,门后,就遇到笑神与哭神。
\par 一位笑神与一位哭神,这奇诡,又想笑,想笑笑到倒下,笑到打滚。想冷静地笑笑,仔细而轻微地笑,打断一切思索地笑,停笔笑,不再笑也笑笑,诞生之笑,复活之笑,冥河笑,彼岸笑。
\par 一位笑神,笑着。面无表情笑。一位哭神,哭着。面无表情哭。
\par 笑中也有优雅的哀愁,哭中也有病态的欢悦。这两位无聊神,无情神,笑与哭的收割者,哭笑的歌者诗人收藏家博物家冒险家,真是与我万分神似,我甚至开始思考,这两位神是否就是我的一面。笑,哭,但先允许我狂笑吧,深刻地笑,深处的笑。
\par 笑呵,笑不需要神,笑讨厌你我。笑呵,真是诗,是曲,是哈哈哈,是吼吼吼,是呵呵呵。遗忘笑,分离笑,放弃笑,再见笑。
\par 笑神来了,连着他的信徒,无主的笑,失魂落魄的笑,绝望的笑。
\\[0.6cm]
\par 虚妄的笑,虚假的笑,那些不被信任的笑躲到哪里去了。格格不入的笑,突兀的笑,压抑不了的,抽搐的笑。我的笑们,总有那些幸福的笑与悲哀的笑,可是什么样的笑才会完美呢。
\par 终末的笑,结束的笑,回眸一笑。唇之笑,舌之笑。我像一个人一样缅怀孤独的笑们,怜悯那些垂死的笑,病重的笑。我不是那位笑神,所以才会有一个笑神,笑的牧人,笑的神灵。笑神在笑,那就是万笑之笑,唯一的笑,无终无始的笑。
\par 我不是这样一个笑神,开放的笑,释放的笑,真实的不说谎的笑。
\par 不能对笑真心实意,我喜欢的是肌肉与皮肤的欢愉,这样,笑与哭同质。笑和哭本就是一样的,可是人类发明了多么无能的情感表达方式。可是进化的能力却没能清除这种遗物,反是保留下来。多么可笑。
\par 笑不属于我,它们永远是笑神的信徒,多么深交一只笑,只愈发现不可揣度。听笑歌唱,听笑一一言语,看笑舞蹈,出现并消逝,听笑在笑。
\par 我想这些笑大抵是诗人编纂来的,许多笑不属于人间,不被人承认,如同怪胎。所以才有这位笑神,一位微微笑着的笑神。在我深思这些笑的奥秘时,笑神一定也在发笑吧。
\par 笑才是秘密自身。钥匙的笑,锁孔的笑,以及是否存在的那财宝与谜底的笑。
\par 启示的笑,发端的笑,一位神灵的笑,侧耳倾听的笑,凝视不移的笑。启发我,诉说我,在我凝定时欺惘我,在我坚定时背弃我。在笑所表达的世界之中,编成笑之梦,编成时间背面,空虚背面,笑之命运,笑的始终。
\par 所以笑是朝圣的第一旅,是最先的财宝,这个无情的世界第一束馈赠。
\par 一位笑神,在门前笑着等我的笑神。欢欣地大跳舞步,教会石头与宝石喜笑的无聊神,是我左边的见证者,是左边的记录者。我才是这样一个信徒,我也是它的奴仆,一只微不足道的笑,躲闪成迷的笑,我原是千变万化的笑,在人间清醒与混沌断不能辨的时候。
\par 只有我与这笑,只有我与笑的关系。笑的形状,笑着形影不离。
\par 世界融化在笑,轻薄的笑,嘴角上摇,眯住双眼。这神秘的发端,融化的笑,包容收摄的笑,魂牵梦萦的笑。命运融化在笑,诗人欲言又止的笑,繁复到细微深处,勾画而出的笑,笑的名字,笑的言语,笑的国。
\par 我爱你如笑,我的爱如笑,我爱这一笑神。而笑神弯卧在枝桠上,悬挂在月牙上,直到你我都轻轻笑起来,不言,像笑的仪仗,笑的祭祀,笑的祈祷。
\par 笑神来了,笑神是我,也是你。
\par 笑神是夜,是少女也是老人,是土地,也是森林。笑神走近我,连着他的信徒,迎接着朝圣的旅人。
\par 是以记下朝圣第一见到的,是笑。
\section{哭神篇}
\par 在昼与夜转换的时候,从远远的地方,哭神来到我的身边。
\par 静悄悄的,一滴一滴,轻盈的回荡开来的哭。沉闷的哭。低低的,在大地上摸索着的哭。轻微的,漠然的,攫住我,又松开我的哭。回荡,涌开,飞舞着的哭。扑闪,反光,轻触着,贴着眼眸的哭。层层剥开月亮与太阳的哭。
\par 紧紧的缀满每一枝杈的哭,挂在我的毛发上,黏在指纹深处的哭。听见山崖抽泣,河流牵着晶莹的卵石哭。金色的河流,梳着紫色的长发,透明的哭。明亮的哭,开朗的哭。
\par 半途之下走丢哭泣,沉顿的,压抑的哭。沉重地压弯了我的腿,脊梁,颅骨的哭。拉扯着我的衣裳,把我拖到收割过的麦地上的哭。嗅到土地的腥臭,嗅到红色晕开来,在沼泽深处深不见底的哭。
\par 像我的爱人一样哭。钟声的哭。被高高悬挂起的哭。接受阳光洗濯的哭,爱人搅动发丝而哭。分娩的哭,哺乳的哭。冰冷的刀具和静静卧在血管里的哭。低声抽息的尖叫哭泣,黑暗捕捞海水哭泣。
\par 哭神,踩着水向我走来。哭着,如同琴弓割开诗人的喉咙。哭,变成一曲诗,染血,然后再滴落,皱缩,或伸出触手。
\par 出走人国,倾听晦暗倾斜过哭。西方呜咽,再到东方沉眠。如同草原上的那把刀,弯弯的刀,蜿蜒过水的足迹。一步一步沾湿了裙裾,前襟。把回忆抽出,缝在袖口。
\par 坐在火边小心翼翼聆听哭神。拨乱卜签,拨断琴弦。
\par 哭,诗人魔,面目模糊的情人,失焦,溶解。灼痛脸颊,灼痛了火。拖曳我的残肢短腿,在我的皮肤上绘出诸神的黄昏。哭着,陨落群星,陨落塔顶。失语之焦灼,诗人丢失颜色之哭,车马粼粼,山海之间独一无二之哭。
\par 哭成霞,哭成云海中央,缓缓升起或消融的红晕。哭成虹,在时光囿于记忆终点,妆点桥,凝成泪,洒满紫色与明黄的原野。
\par 像我的爱一样哭。像哭拥吻于我,啮噬我的眼睑瞬膜与睫毛。一位哭神,摇晃着长发,颤颤巍巍摇着舞步。在那咆哮的河谷之上,教会山顶哭,教斧头哭。教那些哭,自己在黑暗中编织名字。教哭唱歌,一声一声,如月如钩。
\par 哭神入迷,哭神来了,带着一些迷路的哭,带着那些染血的哭。它们或高或低唱着歌,醉着,困顿,渴求着,像冥河深处翻滚的魂灵一般摇摆着。哭神编织史诗,也编纂舞步。它抚过我,沉沉地坠在我的心脏上,它诡怨的哭,抽咽的哭,狂乱惊惶的哭。敲打着,像雨,像冰,撞碎在凝顿的虚空之上的哭。
\par 哭神哟,你给我带来了什么。那些爬满哭神的衣纹上,密密麻麻的质疑,质问的哭,不信的哭,贫乏无力的悲怆久久伏倒在门前。
\par 这哭澎湃而轻盈,在大地之上鼓动而升腾。哭一切生生不息的万物,哭惨了祈祷者,哭婴儿,哭老人,哭星辰摇晃。先哭那些哭,哭神咸湿而布满裂纹的双手,为什么来到此处,为什么又将离开,为何满心疮痍,又为何蹈于完满。
\par 哭神来了,连着她的信徒,痴惘的哭,那些摄人的,骄傲的哭。
\\[0.6cm]
\par 哭,我想念哭像想念兄弟姊妹。诗人只难过,诗人很少哭。诗人沉静的,只有幸福的哭。诗人也有郁郁的,撕裂的哭。
\par 翩跹的,哭的影子,破碎的,哭的假面。哭之爱,燃烧于烛光之上,堆满窗台,挂上犄角。
\par 哭,收束的哭,拒绝回答,只有探询。哭如丢弃宝藏,哭揉碎了宝石与祈愿。
\par 听见夕阳的眼底在哭,碎裂的积云泼洒在天空如舟。盛满我的女儿,哭神细碎的脚步。听见哭神走近,走远,拖曳着大地与鸟兽,低低的号歌。
\par 遇见哭,遇见哭之背影。欲语还休,错过哭,转眼就成诀别。这些无主的哭流向那个国,与河原上,一切无主之物,流放人,哭之宴礼。相拥而舞,相期与醉,背面世界奥秘在孤独的草原深央怀上了哭。
\par 我害怕哭,恐怖哭,我惧怕那些危险的哭。酸,又苦,轻薄的裹住我的眼眸,瞳仁之哭。我远离哭,躲避着那些诉说,逃跑的哭。像云朵聚在岩穴之中,我和哭神就睡倒在坚硬的石头之上。
\par 写就如哭,一切捉摸不透的,哭如歌颂,哭在海潮上端吟诵。哭,绚烂的哭,金色的哭,躲在礁石之上,对着去来的众生遮目宣告。
\par 去哪里找,一种搏动的韵律在我的心脏上缺失了。哭神抢走了我全部的财宝,哭神刺伤了你我也刺伤了自己。哭神为了它的儿女而偷盗梦,爱与水。一切涓滴,怀念并思恋孤独的哭神。而迷途的哭,希望的哭,那些疲倦的哭们,四散天涯。
\par 每一只哭都是如此宝贵,我不能轻易的念它们的名字。每一只哭,都没有任何亲族。
\par 如同一个个流浪的诗人,像是爱神的吻,美神的眼。我既不能小心地悄悄地走近那哭,也不能粗暴地无理地走近哭。我只有安静时等待哭倦倒在我的脸颊,或是丧乱时,哭扑闪着扑到在血如火上。
\par 每一个哭都是宇宙的异类,住在地平线的一端,世界的极点。错落,左右,洒在一角的哭。
\par 加入哭神的行列,我是哭神的信徒。将悲怆纹满皮肤的每一寸。我是那失落的一个,不安的哭,啃咬着拇指的哭。久久没有听见同类的消息了,天地苍茫的哭。我还是在人间走丢的那个么。
\par 只有哭神带来,哭泣的消息。春天醒来的,冬日眠去的消息。岁月与色彩被洗刷泛白的哭泣。
\par 走在人国,如此之多的寄托之中,难记得哭声。真的哭,无需置疑的哭,单纯的哭,像哭神一样,伏在门前,哭。看整条河流一起潺潺的哭,愈流,愈渗入土地的深处,愈凿穿岩石,凿穿时间的哭。
\par 到大海哭,到树梢,暴雨里,抽动着心脏寻找哭的痕迹。永永远远的爱上哭。
\par 只有思念永远呵,轻轻抹掉。我的朋友们来了,雨儿,夜吾妻,河上的三位女神,很老的一些朋友,陪我静静地聆听着哭神。
\par 哭神来了,多么美的哭神呵,连着他的信徒,欢悦的哭,静穆的哭。
\\[0.6cm]
\par 悄悄,走回花径里。我们去见证哭的诞生。
\par 太阳,金色的泪滴。月轮,低眉在天斜一隅。我们去看哭神的舞蹈。哭的呓语,哭的倾吐。
\par 在梦游的深处,我也像哭神一样悄悄伏在地上。深深思索,静静地哭着思索。我怎么掉到这里,醒来时何以窥见幽暗与开明转换。
\par 我们本不是那些质疑生命的族类,只是我与你一样是哭神的信徒。每一滴血都是哭。
\par 哭的命运,哭的名字。伏在幽幽的庙堂中央,平和的哭,永远安静,浪漫的哭。一些柔然的哭,婉转的哭,可以是温暖的哭,像母亲的怀抱,泪水模糊并打湿了存在的意义。
\par 幻想坚韧的哭,断裂的哭,撕开乌云与长袖。盛开如哭,我的哭是我滴淌垂悬的生命。哭陪伴我,以那隐约的姿态,落下脚步,哭是朝圣路旁冷冷注视的那一个。
\par 背叛时间的哭,不离不弃的哭。在笑神旁,转头顾盼就是哭神。
\par 来到河上,听到哭神歌。哭神采花,左右流之。无情的百物自兀流行,我便是在它们之中醒来。我不是在颜色中黯然醒来,听见哭神低语么。像我久久听见的,诸神群聚在河边捕捞祷言的声音。宛若怨泣,抚动发丝。
\par 听见生命在土地上犁响,翻滚,升腾。腾腾的骏马,我幻想那些绯红的,轻紫的,丛丛繁芜于草原之上。而后遇见哭神。
\par 我明白我的哭自我初生以来,自我呼吸以来,血的苦楚,脐带的痛。我无声的哭缠在我的歌诗里,哀,浮响空明,晶莹的一些,玄奥的一些。而后遇见哭神。
\par 哭神走来,带来了一些异乡的,深深的不能置信。这位面无表情的哭神,在透明的脸颊上轻轻滴着,那些梦中才有的世界。那些与我初醒时,同样在模糊与不定之中,听见的世界,只在一滴滴泪液下,涌起,又平息。
\par 我怎么与它谈起,我只有自抑的思念,那些哽厄在我的咽喉之中的是什么。言语不能时,才会想起泪与笑,想起那些滚烫的,令诗人永远自惭的。
\par 啊,哭神来了,哭神一直都在,连着她的信徒,微笑的哭,望乡的哭。
\\[0.6cm]
\par 是了,我在祷告时哭泣。在森林里寻求呼应时哭泣。那些远居在崇山峻岭之中的哭之隐者。
\par 聆听时哭,深深地倾听时哭泣。听到轻轻奏起的音声。
\par 散宴时哭。欢笑的哭。单纯而普通地哭,像人一样,眨动着双眼。我不知道这是什么,我不知道为什么,在我的心中吟诵哭的名字们,如同祷文。我找不到言语,临近崩溃。我的信仰受到了质疑,我的形骸也受到拷问。
\par 朝圣之上,哭是你我的镜像,笑也是。哭笑就是你我的面貌,是最后的,也是起始的言语。诗人的纹印,如笑,如哭,文在脸上,每一绢肌肤。
\par 我可以不要回答,永远徘徊在河滩之上,我是这样一个无情的信徒。我可以学会毫不在意的一次次撕去羽毛,再用泪水弥合伤纹。我已经打定主意用脚步去丈量梦的领土,一切不明所以的哭或者笑一样,在人间闪烁着出现。
\par 我知道我在你们中间学会了人类的爱情,我也模仿你们的言语,模仿来的哭笑,编织入诗言,编入仪式之中的笑与泪。一切都如此不可挽回,仿佛永远在坠落,融化的羽翼之下。
\par 最后我还是捞起了悲伤,悲伤,牵连着一些叹声,有关面容的回忆。哭神,就静悄悄的,坐在我的身旁,坐在右边,头埋在膝上。左边是笑神,躺着,玩弄着一颗浑圆的宝石。
\par 我想这两位神,两位耐心的等待者,记录者,见证者,我的宣告人。我在朝圣第一时就见到他们,从今以后也将永远接受他们的陪伴。
\par 笑,哭,我的左手与右手。左耳与右耳。我瞳孔与我的血脉。你们是诗人的梦魇,还是这凡间的神迹呢。
\par 一切都走向深处,永远的走向深处。
\par 是以记下朝圣之中醒来,有哭与笑的陪伴。
\section{火篇}
\par 朝圣,转而拾起火焰。
\par 高高地,举起火焰。大口大口,吞吃火焰。鼓动心脏,喷薄火焰。
\par 新的诗从火中诞生。灼痛。在指尖,每一束肌肉间,灼伤我的胃囊,我的腑脏。
\par 脑即是火。血,干枯的血是死去的火。替你们铸碑,碑上刻火。
\par 火的看守人,自由的精灵们。在花海之中放牧火的儿女们,漫天飞舞的金霞,深夜游定的金星。
\par 火呵,我在无穷无尽的地方看见火。汹涌的大火,聆听火的咆哮,火之交响。
\par 一,二,火,万物生于火中。一,二,火,看见万物从火中来。太阳死去又复活,一切交谈之中,有火在燃烧。
\par 每行每列,祭文的字里行间,都是燃烧生命的痕迹。光,舔舐热,散发着肉,吐息,颤栗。火焰的族类以火焰言语,火焰的影子又是火焰。
\par 我的灵魂是一抔躁动的火焰。它掠过土地与岩石,它驾一匹马,从东到西。大火掠过草原与牛马。裹挟了凡间的滚滚思念。
\par 命运如火,幽暗的火,吟唱的火。火之意念,火之纹印。
\par 我从火中踏来,亦将火里踏去。我见证文字,也将见证大火。我就是大火的见证人。
\par 时光是火,从漫漫的时空之中召唤火焰。
\par 今日要见证火之诞生,火之创造。
\\[0.6cm]
\par 今日,我有歌唱的欲望。
\par 告诉火,祈祷火。火变成生命的一部分,从四面八方的大地涌起,汇聚。火替代心念,永远震颤,绚烂。
\par 火变成我的欲望,在眼眸之中起舞。世界滚烫,世界静默。时间沸腾,时间缄默。
\par 火变成我的人性,火变成我的神性。天空之中留下火焰的思索,火之交谈,火的情感。
\par 今夜,火焰是我的道路。我皮肤如火焰般透明,躁动。纯净的,纯粹的火,温度,热,电离,辉耀。
\par 火是妻女,子民,牛羊。火即富贵,亦即贫乏。火凌驾于土地,火臣服于土地。天上地下,送走一束火。
\par 我如是在万物之中找到火焰,折枝,梦中断裂的泉眼,胃囊,镌刻,眨眼。
\par 遵循火的仪式,大地芜秽,天空垂下蓝色,刃饮血,诗人求歌,大海沉静。
\par 焚烧文字求暖,如今困躁。求祈火神求火,今日有隐秘升起,随着火焰的诗歌,一次又一次降临于生命之野。
\par 在火焰的庙堂里高声阔论,那位无情的智者,礼拜无情的神明之众生。一切都想付之一炬,一切都将在火焰的寓言中黯然失色。
\par 火,天庭垂到下界。电,惊醒仰望的人。惊醒黯然,惊走沉郁幽暗。求火。
\par 今日起舞求火,把我带到一切归往的国度,把我带到神国。在凝望火焰的瞬间,怅然若失的许许多多魂灵。
\par 今日要闭目闭嘴,否则就会溢出求索。
\par 火焰是我的秘密。也是我的揭示。我是火焰的守护神,亦是它的复仇神。
\par 宇宙中穿行的合力哟,溃散的物质们,即是火。火的遗迹,火的纪念。
\par 因冷却和螺旋而凝聚成的亮星哟,不堪重负而自我溃缩的中子星,吐火,吐电,火的遗迹,你们是死去的火还是火的新生呢。
\par 一直到最后都难以找到,在夜的深处倾听死亡的理由。
\par 我的脑海里居住大火,神经在响动,振翅,火的蝙蝠,火的神经。替我去雕刻哟,去从心灵之中凿刻出火焰的形状。
\par 求问为何写作,求问如何铸火,自黏稠的流质凝固出火的宝石。通明,眨动。
\par 今日有火焰的悲伤,灵魂失水。在这火焰的殿堂,奉上一束又一束祷词。
\par 今日浸湿的愤怒使我想念火,走丢的一切使我想念火。求索是诗人的情感呵,求索是诗人吻火。
\par 可是火不需要爱,火也无需歌,无需舞。轻盈的火,臃溢的火,沉甸的火,旋转,游移,周而复始,如同转轮。
\par 我要在那条河中捕捞起火,我丝丝听见了火醒来的声音,我听见谁在念诵。
\par 火呵,火呀,火抚摸火,火哺育火。火从那悄悄的地方生长,粗壮,摇曳,旋即上升,亦下降,那只手就向前,探出,旋即唤醒了火。
\\[0.6cm]
\par 开始怜悯我透明的翅翼,火焰汇聚在我的犄角。
\par 一定有什么吞噬了我的记忆,一些裂纹绽开在眼角。一些生命的证明失稳,溃落,不再置信。非你也非我。
\par 拒绝生于界限之外,而无可名状之物投影在文字之中,面目全非,碎裂一地的形骸。仿佛遇见了另一个我。
\par 文字若是镜,肯定映不出火。是火的断章,火的残念。诗人自私呵,诗人的罪责。
\par 若以文字求火,只是玷污。火来煅烧这些字句呵,火来拷问空洞的心灵。谁该是火焰的捕捞人。谁是火焰的对话。
\par 想念火焰有深刻的憔悴,想念火之欢笑,火之秘密。愿我披上火的静色,直到重新醒来在朝圣的深处。
\par 有什么已经从内部占据,吞噬了我,一个秘密以琐碎的姿态占据了我。把我变得奇怪,使我开放,像大气与河流,一贯的那些丢失形态的事物。
\par 从今我是火焰的族类,我是执掌火的誓言而生,亦将跟从火焰归去。
\par 我将顺从火焰的仪式,在逐帧黯淡的天阴色下。这像一种奇妙的断言,让我明白了一些生命的奥秘。
\par 现在我与火焰即是双生,并蒂,我与它不分外在与内里,我们将互相顺从而延续仪式。
\par 今天听闻的启示也将是火焰听闻的,今日的茫然也将归属火焰。
\par 火,在沉静的姿态,像人们沉思的面容。像忏悔的人,等待的人,火焰淡淡地照亮方圆的土地,以其温暖与炽热引诱着迷游物。像我母亲曾诉说的话语。
\par 火,在火焰的起始,所听闻的可爱的,柔然的,不自意着,偷偷窥视着,又昏昏沉沉的。一切都像人国遍路的歌人,沉默的时候。
\par 我不知道火焰何时染上了这么多的色彩,映照我的内心,又像是每一寸时间。
\par 曾经珍藏的许多美丽都在火焰中翻卷,在笑神与哭神的无情,半注意的凝视下,融化,滴淌。
\par 连天地间散漫的光都全沉淀下来的时间,火焰起源于恍惚,夕阳最后的投影反射在失焦的云彩上。
\par 这时的我又思索些什么呢。
\par 悄悄的,晦暗的言辞与我脱节,不再置信一切流行物。在人间留下丑恶曳迹。
\par 求火来,以夸张暴烈的姿态喧闹而过。这一束狂躁的火,另一束隐秘的火,灵巧的火,闷郁的火,毒怨的火,警慎的火。
\par 求火,看见一束束火焰,一束唤着一束,一束牵着一束,走在河上。激起火的啸叫,水的欢灵。一只一只,闪烁出激光,飞溅着,律动着。
\par 仿佛是秘密诞生之夜的消息,仿佛是我们一贯的祈愿。
\par 火呀,电呀,驱散黑暗,驱散絮语。让你在我的动脉里翻滚,随之涌动,转瞬即逝。
\par 我们能去哪里,光呵,暖呵。
\\[0.6cm]
\par 我将携带火,披头散发。
\par 我和火,在偌大的黑暗之中只有我和火。没有光,沉默,没有形状与存在,一道火之意念,火诞生之前的意念。
\par 在刹那之间,只有我,火。生来如此,生前如此,死后也将如是。在封闭的时间与空间之中。我与火缔合。
\par 是,与那创造火焰的意念缔合,唯一的我与唯一的意念。
\par 这个意象闪烁摇定,渐渐夺走了我的心魂。在我与火之间。没有距离,没有秘密。下一瞬间我就分崩离析,每一瞬间我会变成宝石。
\par 我将携带火焰,跨越永无止境。
\par 在湮灭的瞬间。我看到我悄悄举起了火,托着它,轻轻的在水上走。
\par 我托着火,像火焰的载体,像它的躯壳与燃料。我,或者火,走进万物的内部,走进镜子的背面。我们走进思维消滞的一面,我们走进生命的诞生与衰亡。
\par 在那里实行时间的权力,实行死亡与断念的权力。
\par 在河上怀有火焰诞生的意念。火焰诞生于流淌中。
\par 这火原来自许许多多地方来,但这一束,这一朵,从我的目,我的口中诞生的火,我已经瞧见它从水里悄悄生长出来的样子。
\par 它从一些思念之中结晶而出,火,是我的本身,也是我的儿女。
\par 它亦是那永远燃烧者的投影,它亦是光明的使徒。但它也是河滩上的宝石,也是哭神与笑神交换的一些话语。
\par 因为流淌而诞生的火,因为孤独前行的意念而起的火。醒来,而诞生的火,言语,而诞生的火。
\par 今日我如是见证火的诞生,那些光明的国在幽暗的时候向我展示。
\par 在混沌的时刻,有火焰自河的内部悄悄升起。洗练我的灵魂,并给予生命以启示。
\\[0.6cm]
\par 我前去,执那万物的意念。
\par 依照那火,在我的灵魂之中也点燃一束火。执一切众生意念。
\par 流行物类,或翔集,或伏地,或水生,那些以口为生者,以眼为生者,即是我的信徒。
\par 我把火在他们之中分享,所得便是众火,大火。
\par 我把黑暗栽种在土地,收获便是夜,与明惘。
\par 是世界随我前去,还是我随世界奔流。我们都是河原上的居客。我们也是捕捞人。
\par 醒来时,看见诸神来到了岸边。一个一个都等着我的到来。
\par 我曾经的恋人,曾经的情人们,那些启示我的,引导我的,毁坏我的,鞭笞我的,都沉默而微笑着矗立在那河上。
\par 火,那从我眼中升起的火,扫除迷障,让我第一次窥见了时间之外的,那条河。
\par 丢弃回忆成宝石。
\par 这条河是一切流逝之物的合集,是流向冥冥之处的游行。
\par 像是时间本身,又像是文学的坦途。只要想念它,它便降临。
\par 它无处不在,淹没我直到入梦。它似乎欢呼雀跃,又沉冷如冰。是了,我是要去溯它的源,去随它流淌,去同时见证创造的诞生与消逝而大化。
\par 在这里我将重新塑成我,我将一次次的诞生。每一次醒来,都成为河上,另一位不同的神灵。
\par 我将朝圣,为那世间一切美丽之物书写祭文。
\par 我将祷告,愿那流行之物秉承我的意志,并召唤更多的有情或无情。
\par 高高举起火来,我既是那初生之时的火神。我既是一切起始的意念,我将化成宇宙,我亦令那轻者上浮,重者下沉。
\par 但首先让我照亮混沌之中无垠的黑暗吧。
\par 让我这株新生的火苗照亮全部河原。
\par 是以记下朝圣之中,第一次呼唤的,是火。
\section{水篇}
\par 朝圣,照见大水的痕迹。
\par 那些哺育我的水,欢快的水。遥远传来的,隐隐约约的水。柔软,游弋的水。
\par 凑近,听到水的消息。润泽了我的嘴唇,瞳仁。濡湿了我爱人的眼眉。
\par 水,在地上是湖,是海,是河。在天上是云,是雨,是冰。真是有趣,流行而逝,氤氲而生。飘摇没有尽头。
\par 从水中来,醒来时还是在水里,从今以后永远是水上的居民。呵,真是我的故乡。
\par 一切都已变作了水,关于水的意念,我还可以说什么呢。
\\[0.6cm]
\par 水,全部意念与水牵连。流逝物,生命藕断丝连的一切,视点边缘模糊的形状。
\par 流淌在外属于遗忘,向内则是求索。印刻在肌肤,是告解。涂抹在羽毛上,是净化。
\par 和水一同前来,我未曾听闻的水上居民。大水如挽留,流水如割伤。大水留在土地上弥散开来的泪滴。
\par 千种万种水在我身上流淌。自灵魂之井满满溢散出来,从每一束血管,从眼口鼻耳,听到了冲刷的声音。
\par 失望与欢快的水,失明与尖嚣的水。今日捕捞起的水,有焦躁与压抑的水,也有流连和痴笑的水。我原本是水的垂钓人,是大水捕捞人,众水的旧情人。
\par 像我曾经化作火一样,我早早就化为水了。在土地与大气中运移的水,以肉体承载的隐秘之水。
\par 晨起,高升的金色液滴。铺满捉摸不定的蓝色水镜。水溶晕了现世种种,让久久传来的消息陷入了透明。
\par 跑到大水上来见我,籍着水镜揣度我。我的心是瞬间的河流。每一瞬都折射出千千万万只我的断影。粼粼之我,大火高升时,起伏闪烁。
\par 千千万,水瞧见我。像胎儿一样,裹住我。哺育我,低沉我。
\par 静与流,摇曳勾引波浪,抚尽,抚不尽,水上的神灵饮夜下青蓝的光芒而泣。倾听。
\par 大水离离,众水婆娑,端居在水中间,无色国中间,无至遗忘的祈愿中央。
\par 醒来时已然被水浸湿。浸湿双脚,浸湿疲倦的肌体。梦中第一兴起的意象是在大水中央,因为没有时间,仿佛流淌得到挽回。
\par 我可以与你们同在,成为另一位水的寻觅者。朝往土地上的水之行迹。
\par 我往东方背离,北方,往南方,在背离道路的方向,只是有水还在流淌。转瞬的水,谁是有名字的水,最后,有归属的水。
\par 是我的兄弟,姐妹,是我一切亲族,我的导师与信徒,去来无碍又转瞬,谁是水。
\par 水躲在我的生命已经太久,使我的灵魂至于迟钝。我久久融化变成水。行长长路找到水。
\par 溶解,在水中的夜,在形骸之中的水。言语消解,回想消解,变成水,水神,水之意念。
\\[0.6cm]
\par 黑色的道路上,水来。流光溢彩的道路上,有金色的雨来。
\par 森林之中隐隐约约听到水唱。隆隆听到水欲来的消息。久久闭上双眼在偷盗与转变的一刻有水来。
\par 思念上有水来。捧起哭神的脸颊拭泪。血,滴淌渗透时变成水。
\par 用水来作,磨圆宝石,洗濯宝剑。向往之中打水来,质疑与惊愕中打水来,浇灌太阳坠下与初升的土地。
\par 抟起水捏成走兽,飞禽,转眼又溃散。唤醒水来,纷说悲郁与情欲,眨眼又四散。
\par 造我时,勾勒诸水铸成五脏六腑,拿水浸湿了岸上的泥,塑成一个隐约的脑体,捏成一张隐约的脸。
\par 我们是那国,一切流逝与归去的国,丢失之物去往的国,朝圣,国民。
\par 一切转瞬之物加入送别行列,籍着水声传来的世界祷告之音。
\par 在水中听凭流淌,往水中挽回,祷告或祈求。这颗水的灵魂,像水一样不知所踪。
\par 凑起嘴唇衔上水,伏在土石之下等待水。两腮离阖,滤过闷湿的空气,好似我原本如此做过。低眉泣下珍珠与透过波光思念月娘。
\par 河上的守望与垂思,好似另一个我,河上的女儿们,与鱼的秘密对话。
\par 采水来,采来水边的歌声,我的男儿与女儿们在河边上一齐对歌。谁若有那思念,就全部赋予水上。
\par 隐秘,又出现,沉浸,翻卷我。回旋的神秘的光呵,仿佛水一般流逝的美丽之物。
\par 遮住水的眼眸,牵起七彩长发,像每一位水的祭祀,水的哲学家。我们盲目地在土地上搜寻水的痕迹。
\par 潮汐,远航在深海中央。呼吸,弋定,东南北方群聚而来的雷云。
\par 凡是这颗星球上发生的,看似永恒的事物。水的轮转,鼓动胸腔与心室,从金色的天边驶来,微薄的转瞬的,独属于聆听与眨眼的神秘。
\par 我可以不用思索,只跳水的一支舞。喷薄,飞掠,四碎在黝黑的岩鳍。分成千千万万束,向东或向西。
\par 我将抚慰花,抚慰土。在鸟兽的血脉中流转。
\par 比起信徒,我更像是水的亲族。曾是水的情人,如今则同在,拥有同样的心灵,分享同样的命运与生命。
\par 是水成为我的一部分,还是我变成了水的另一面。
\par 替水聆听醒来的预告。乘着水到船歌声下,支过渔火和漂掠水面的羽毛。替水流淌银鳞,摇响了岩岗上的寺院深钟。
\par 想见水,夜空中盈盈与月华下泪滴,想见大地撕裂滚烫凝滞的血滴。
\par 因为水的国临在此处而获得的安心。如获得生命之初的安慰。成为水的卵,星,成为每一瞬间呼吸。
\par 在深爱水的一瞬间忘记灵魂一切,可以忘记求索与吟诵而流淌。
\par 空间,时间,我将不再关心东,与西,流淌不知是谎言还是执念,随着无情淡忘而改变了面目。
\\[0.6cm]
\par 在孤独时怀抱水失神。独自一人变成冰丢失边缘在雾中。
\par 冰,或者流水。沙,低垂雨幡,极光在高层大气连缀成帘。仰望处,夜下裂开一道七彩的云河。
\par 水的情感。水,哭笑,水的奔跑,与梦,不是另一我之镜像而是水的本身,只是水,在水的领土。
\par 一次又一次在迷失之中自我确认,像泡沫,流动的霓彩,单纯与污杂之物。
\par 自诞生之初就仅仅是水而已,没有看到那独属我的一滴。
\par 追随无穷的步伐来到大水诞生之处,只看到黯淡的宝石滩头,搁浅,在若即若离中,醒或睡去,几乎埋在水的回忆之中。
\par 今日预备听闻水之诞生,水的创造。却早早融化在这位爱人的怀中。
\par 水呵,你是诗人情感的集合,是话语的葬列。你是那条转瞬的河的全貌,也只不过是它的微光,你带走人的祈愿,送来那些无情的预言。只是水,水呵,歌唱,空明。
\par 是水挽留我,还是我挽留你呢。是水召唤我,还是我召唤你呢。
\par 让我重回到水的殿堂里,让我重新祈祷,像云从山峰上飘落,或者大雾被暖阳驱散。像水一样在土地上祈祷。
\par 谁是你呵,谁是土地上的水。谁在那里等待我。
\par 宝石啊,紫色啊,或者那些绰绰的影子,是不是都是水,你作给我的。像是我以前在土地上搜罗的财富,也都是你藏好的。
\par 秘密是水,沾染你,就缄口,或微笑。像母亲,像父亲,水色氤染。
\par 求水来,洗清天气与温度,洗刷这片牢狱。从梦中泛滥而出的大水。
\par 静谧的,大水湮没的梦中,垂钓,张开指掌,想要抓住或捞起什么。如此,跟随,索寻,如同飘飞而浮沉的种子与花粉。
\par 把我变成一个个水的隐喻。
\par 我,与水。
\par 我与,河流,在我出生时刻,唱诵时分。流逝的一切塑成我。
\par 水之创造是一切意念起始,一切情感起始,一切有形无形起始。挽歌起始,凝望与思念起始。爱之始,忘之始。凡间的恨与欲落落成卵石。
\par 血,水。滚滚,席卷,下沉。在我的酒液里混进了一堆深刻的哲思。
\par 土地上百倍沾染颜色。水是我内里传承久久的意念。如此密不可分,使我不再辨清。
\par 此滴,从梦中淌出的,神之垂泪,开始的一滴,转瞬的一滴。命运与魂的一滴。爱与迷思的一滴。
\par 水,思念而生的水。从那条更伟大的河里渐渐滴出,聚沉,我窥见了我流淌的时刻。
\par 从眼眸与脑体中,一滴属于我,水神与我。怅惘,和高声的呼号,长哭,与咆哮。水,诗人的心魂,静静等待祭文的书写。
\par 一切写到河上去,寄托或挽回的祷文于水上书。
\\[0.6cm]
\par 思念成滴,水神端居在思念之上,起始,转换,与展开。
\par 注目亘古的大水痕迹,同样也是大火的痕迹。我们存在,名字,絮语,所包含的全部道理,就是水和火。世界的全面,同样也是世界的背面。
\par 在镜上流泻而下,映出水和火。在左与右,瞳仁的正与里,同时捕捉水与火。
\par 那位神灵走到这片原野。起初是黑暗,便从河里唤醒了火。
\par 火驱散走沉沉的黑暗,水便从眼眸里溢满出来。于是世间一切的逝者来到河边,朝拜。
\par 我,与那二位哭笑神,踏水而来朝圣。
\par 此地见证水与火的创造。以及永远见证水与火翻滚在世界的每一处。
\par 吾是挽留,吾是存在。我是流逝的合集。我是世界的表里。
\par 水,或者火,念起像是喻体的表象,抛开其他一切像是一些陌生的,文明。
\par 只有这水,这火,此时此刻属于我的水,我的火,我同那位神灵一齐创造的,或是经由创造而继承的,自我。
\par 自,我,摆放在比喻的本体,仅此而已,同样是一片陌生的国土。
\par 我所体会的无穷,也只不过是此处,仅此一处,唯一。一个字,一个符号,这个符号如此简单,只有轮廓的刻划,只有曲折。
\par 刻在我的额上,我的唇上。诗人的标徽,每一条路上与方向。
\par 事关思念或物象的一切,毫不鲜明地闪烁而起,衰远而逝,漫上我双脚所矗立的滩涂。
\par 水与火,团团围坐,每一个都怀有不怀好意的秘密,像我,像哭神或笑神,缄默而神秘。每一束,像蝶翼上闪烁的小鳞,折射,扇落,翩跹。
\par 表象,真实,以及梦,在隽永之上无情的出演。
\par 世界,轮转的全貌。
\par 世界是我的一滴。
\par 是以记下朝圣之中,水,与一切的消息。
\section{塔篇}
\par 朝圣,来到人类的塔下。
\par 朝圣,到沙漠之中,来到人类的塔下。
\par 跟随我的爱人,来到塔下。跟随三枚星光的指引,来到我人们的塔下。
\par 渐渐地,一步一步,经过长久的跋涉。辛苦,满脸泪痕,经历过蛇与鹰的蜕变,来到人的塔下。
\par 来到金光的塔下。水边的塔,山岗上,坍圮的塔,火边,马队的塔下。
\par 来到湖边,石头堆砌石头,大地的高处挂满绢幡的塔下。来到木头堆砌木头,人类雕刻人类,信仰与慈悲的塔下。
\par 踏入荒野,踏向荒野人的塔上。思念与号呼堆垒的塔,荒寂的土地与火堆垒的塔。
\par 骨骸与岩石混杂着砌成,土壤与水在火舌中胶结。记忆与血铸成的塔。
\par 我们刻上诗篇,如果你还记得。在塔下举行的祭祀与塔顶,王的复活与加冕仪式。刻在塔上,塔中,塔尖。
\par 我们刻上的祷言,我们的走兽,我们领来这片土地上的生灵,把它们刻在塔上。
\par 我们把人领来,刻下它们民族的声音与传说。
\par 醒来时,人们把我引到塔边。醒来时,我们已经为蓝色与紫色的星星祭祀。
\par 在人间上,我们为丰收与战争,爱情与智慧,久久死去的祖先与尚未出生的婴孩,为面色莫名的神灵,日和月,一切求寻不到的寄托与哀伤,为高高的高处。
\par 在人间上,为了时间,为了我们的名字。
\par 把神的名字抄在塔和塔的穹顶上。把我们之中的话语抄在宝石,铁,还有那些火之中。
\par 今日背来石头,柔软滚烫的石头的母亲与父亲,修筑这个塔。
\par 携带黑色与一切鲜明的色彩来到山中,换来石头。修筑这个塔。
\par 轮到我支付火,与水,足以证明我身份的,和其他我苦苦搜集的珍物。牵来畜兽,替我从铁与金子的山运来石头。我们把它堆垒到山丘对面,大风肆虐的地方。
\par 这座塔是我,是我全部诗,大地与水上游荡的情感的灯塔,驿站与故乡。
\par 当我久久从沉顿的愚騃跋涉到另一片郁闷的困索。我起了这个意念。
\par 今日见证塔之成立。
\\[0.6cm]
\par 若我见证世上美丽之物。
\par 若我见证痛苦,生命振翅于火焰之上。若我见证财宝。
\par 像你们一样,我聆听了自内与外的全部声音,我搜寻人,人们的声音,以及他们的乐器,歌喉,哀矜。
\par 我聆听我们脚下,借以睡梦的方寸土地,借以纵马,堆垒岩块,以及丰收和埋葬。
\par 若我亲见冰,红眼的羽类,青紫的闪电。等同我亲见不同的国,宽广,消亡。
\par 使我编纂入诗,万物的名字一一出场,从草,伞和油开始。不辨优美善恶地,我把它们送到塔上。
\par 成为这个世界最大的可能性。记在每一黑澈的深渊里。黑暗如是说。
\par 夜替我看守秘密,我为夜筑塔。
\par 人间五彩斑斓的湖边,用沙和土筑塔。海边,捡拾羽和贝。
\par 轻轻打开一本书,我们摸索着书页。举起火把,抬起手与脚,我们在一起舞蹈。
\par 我要为我自己筑塔,今日,与昨日,我们全部的苦痛与思索。因为时间,名字。
\par 我为我自己朝圣而筑塔,用宝石,凝固的火与水。我们去土地上高高堆起,闪闪发光的一切。
\par 爱刺探我的内里和全部,我为爱筑塔。
\par 把人间全部的谎言和欺骗涂抹在塔的墙上。我们用草原上游离的蹄印书写。
\par 我的生命是暴雨冲刷的山岩和闪电映照下的城市。朝圣自己的生命而溯来上游的峦石。
\par 我的生命贫乏空无一物,空空如也被山风鸣响。
\par 除了石头,和砸断石头的火。
\par 这塔承载了土地上全部的时间,时间是宇宙中的星团。我们会高高的修筑枝叶,脉络,还有水和大气。
\par 思念在我的腑脏里吸吮,我用泪水筑塔。
\par 土地是微薄的血色的时候,我穿越黑暗。我摸索双眼,浑圆,灼烧,袅袅如泉。
\par 背负,幽暗与光明的生命,来到人类的塔下。
\par 从入海口一直背来,挣扎狂舞,滔滔不绝,它吐出,绝望欺压绝望,从左到右都是永无止境的生命。繁冗,污脏,肿胀。高高垒起。
\par 从诗中抄来,话语,日复一日,像个诗人,看守生命的诗人,看守死亡和窃窃私语。
\par 筑一座塔锁紧我。
\par 筑一座塔向自我朝圣。倾听这颗土地上不安的梦游魂。
\par 让笑神与哭神来记下今天的见闻。一座茫然的塔,一座伶仃塔。一座塔上挂上钟,另一座塔上挂月轮。
\par 我在朝圣的途中筑此塔,在千千万万的塔林之中。
\par 待我归来时,会用更多的财宝,装点它。
\par 是以记下朝圣途中,自我的塔。
\section{轮篇}
\par 朝圣,浸没在粘滞的时间之中,一切的吻与尾相接,成为轮。
\par 朝圣,拨开皮肤,拨开血肉,拨开骨骼,拨开一丝一毫的呼吸,成为轮。
\par 暗之指环,金之轮。鼓荡,席卷,旋转舞,冥想。
\par 沉思,我们用宝石圈成轮的一层。取来金子银子的砂围成,塔的一层。我们捉来火与水,喜怒哀乐的精灵,围在生与死上的一轮。
\par 排布情感,鳞次栉比,排列词汇,写就一轮。
\par 命运的轮转在巨大的天空之中睁开了眼。震颤,纷说,凝望之轮,暗间瞳。就是月镜,就是日瞳。
\par 世间一切走向轮转,参与轮回序列。物质,哲学,我们成为轮的守护灵。
\par 秘密,与解密。谜题与谜底,啮噬与被啮噬,蛇衔尾成轮,托举河川与大地成轮。在黑暗与凝滞的真空中,回旋,成轮。
\par 时间从东方升起,西方消去。梦从瞳进入,从唇离去。
\par 智慧举着火把参与我们其中,智慧是年轻,沉睡。行星,种植于贫瘠土壤,接受重力与磁力滋养。
\par 腐草,朽木,枯萎的颜色,继而生长,生命之轮。
\par 端坐满男女,神灵,交换眼,心,话语。时空变幻,交换位置,参与扮演,暧昧的剧场。
\par 神之轮,明轮,灵之轮。迎接与前,前来我的鹿,鹿的眼眸,全部轮转参与同一轮转。一个圆圈,完全等同。于此说不清,是静止还是旋转。
\par 我在黑暗中制造光明,伏在其中,潺潺汩汩,流泻其内。
\\[0.6cm]
\par 我悬浮,蜷缩在光明之中,在怀中紧紧藏着黑暗。朝圣之下,光与暗聚合成轮。
\par 短歌中,短调,简单的语调。躲藏着,雨为轮,音为轮,闭目轮,渐忘轮。
\par 左与,右集满了,残缺的一个个独立体,翘首祈盼。世界残缺的轮,堆满了岸边,缺失了某种部件。残缺爬满海岸。
\par 依附于这个五彩斑斓的形色世界,微末的片段,蝶翼之鳞,叶底蛹。
\par 残缺与完满中央,依旧是那个轮转,静默地注视着流行,固结,夜涌夜落,静默吞噬七彩,绚烂的名字与言语。
\par 思念轮转,陪伴我,放弃,而后注目,分崩离析,背影与笑。
\par 采撷,你与我,去夜露中的花园。千千万个,时间的具现,空间也是,从枯萎到升起,土地张开与弥合。
\par 跑来告诉我,早就知道的事情。我们希望,爱,渴慕,丢失,重拾。用这些动词构筑轮。
\par 从远方跋涉而来,好似是从塔那里来,裹着毯子。带给我,我曾经埋藏过的。
\par 世界从我的脚下延伸,最后回到我的脚下。
\par 连着左与右,从宇宙的一端,爆发的起点,到另一端,稳定,温热。
\par 轮,临于此,临于汝。运动与深思的哲学之轮。存在与思念的信息之轮。于眼眸,于瞳,于星光衍射,闪躲,隐匿。在形状中显现。
\par 蛇,最先,其次是鹰,鸽,鼠,泥色的,鳄鱼,马,人,然后是蛇。
\par 蛇,与蛇纠缠在想象,岩石的平原。花岗岩,与玄武岩,白与黑,的平原。
\par 秘密的话,一开始就以最鲜明的样子,恍惚我,惊悚我,在几步走过桥,在水池,一个无关紧要的谎言中。
\par 植物部分,人间的器官,连接五脏六腑,连通,肉体的病苦与晕眩与世界坚硬沉赘部分,迟赧,我用喙勾勒剩余部分。
\par 言语至此剥落掉,向自我意识中寻求轮的痕迹。在言语前。
\par 哭笑,欲求,瞳与世界联系,光谱与电。耳与世界联系,触,嗅。曾经竖起我的触须,在腻滑的空间藏匿食物。
\par 失去经验,参与失落的躯壳序列。欺惘迄今,给我指出轮之结构。
\par 我在记忆中找到回旋的塔,树与花,大地迁移,喷薄。只要听到那个声音,世界就会纷然而起。
\par 虚空之中振翅的蜻蜓,在黑暗中沉眠的蛹。
\par 我不要永恒蒙蔽双眼,一旦我,接纳光明,与这些幻象中的其一,我就永远拥有一双眼。
\par 轮,扯动锁链,哗啦啦作响。指针,搏动,文字替我魅笑。
\par 弦,鸣响,穿过环。一边是生命,蜂鸣,另一边牵着思量。困惑于圆环之中的老博士与精灵。
\par 我想找到那条轮的道路。在阴影之中显现的,在大水之中显现的。流转,消逝,起舞,散霰。日复一日转动着巨大的磨盘,搬动地壳。
\par 人间的机械装置呜咽,流动,废弃。一如人间的智慧,尘封,被烧毁,风化。
\par 世界废墟被我见到,废墟中央的神像,执轮,执塔,执卵,执星,执电。
\par 这完满中的一部分,都像是脑脏中纠结的神经突触,像是长篇的史诗中名称的承接,是魔法的咒文,祭祀的祷言,意识与意识的连接。
\par 海与一滴,飞散的风尘,面目模糊的使徒,木叶与沙漠,时空间隔的绝对。
\par 轮之序列的祭品,以这样那样,现实虚妄的方式,经历于我。我是轮的载体,我是轮的回忆,我是轮的秘密。
\\[0.6cm]
\par 面对大海的忧郁,面对水仙的忧郁,我有一种不属于我的爱恋。
\par 我形单影只,赤身露体。我行走在贝壳之中。
\par 世界从残缺走向高度无序。
\par 只有手,五指,断续熔化的纹理,以及一个手势。
\par 只有一段音响,一声弹动,奏响山谷,深切急流。从冰川上空滑去。
\par 如今我与你在同一轮转中,过往也,未来也,在我闭目与倾听时都在。我与你同在一道路之上。
\par 就是行将毁灭的道路,就是唯一的道路。
\par 忧郁之池在大地上四处游荡,在天阴,曝晒,冰冻,生长与凋落时永远缄默。
\par 我与你的名字是这个世界唯一的财富。
\par 孱弱的风神说着谶语,风神与水神缠绕在疲倦的树梢。你们也听见么。肆虐在有生命无生命,有情感无情感的空间与光明中。
\par 林林总总的物类,有形与无形都在之中,跳着同样的舞。
\par 一个旋转的舞,圆圈的舞,自转的舞,坍缩的舞与放射的舞,舞舞舞。
\par 现在只欠缺,一个预兆,一次恍惚,一个迟钝。
\par 我们正在那颗,旋转的中心。火热,明亮,从不偏移。
\par 是以记下朝圣中,窥见轮转。
\section{爱神篇}
\par 孤独到底渊时候,爱神播撒着晚霞。
\par 爱神梳理着长发,在宇宙空间冰冷淡漠的光芒下,折射七彩。
\par 在这样蓝色的迷蒙中生长的爱,天空的手指如丛,干渴而盘绕的爱,如同人偶一般静默的爱。雕刻在我的肌肤与血肉如同木纹或羽纹的爱,贪欲而空洞,喙与吻,瞳与眸的爱。
\par 你们把蔷薇披在我的身上,把利刺披在我的额头,毫无欢喜也无痛苦的爱随时随地亲吻我。搬弄是非,把形象意义堆砌在我的身边,感到温暖。
\par 爱神用喙把我眼眸梳理,爱神与死神守在火焰的遗迹。
\par 我从那鸟儿的国度带回来绿色的羽毛,被爱神夺去。我带来的宝石,和丑陋的斑点卵石也被攫去。我曾拥有的,青春之神吻我,空洞的荧光灯下的眼眸。
\par 谁碰响了琴弦又被琴弦缚住。蹒跚而来,淤污与嗅探的爱,款款,模糊的爱。
\par 爱神启示我虚惘的爱,责骗的爱。她必要用臂弯兜住我的脖颈,情人肌肤的颤抖与静止。思念的倾泻与冷缩。
\par 爱,中空的骨柱,羽管与诗章。爱于自责与悯他中求来戒律,吊提受土地与天空的摆布。
\par 我们不只是宇宙人。我们亦不曾生灭。谁用利剑钝刀求爱,谁是土地上第一王。
\\[0.6cm]
\par 爱神靠近我,引起空气的变化。
\par 引起河滩涨水,天空汇聚压抑与自由。纯白的爱,透明,扇动的爱,摩擦声带,的爱。
\par 曾经丢失的怅惘无处可寻,追逐水与草,来到人界的宫墙。我们曾迷路还是开怀大笑,明晰的,生之受,生之感。
\par 苍老的爱神,虚弱的,失忆的爱神游荡在沼泽,有水的地方。
\par 爱神有我的模样,她用我的舌,我的耳。爱神在路边泥泞的地方祈祷,跪拜,并乞讨。
\par 爱之失望,爱之夺取与贪慕。爱的小指陷入了爱的宝藏。
\par 思虑,爱神,心灵焦渴,端庄,朦胧。我爱上的是眼眸的碎片,声音的片段,我爱裂纹深度,把我悬挂起与把我抛掷起。
\par 嫉恨,爱神,咬牙切齿,鳞羽指爪,黄色黑色斑纹攀上我们心房。
\par 爱抚摸我,爱在啮噬我的时候刺我,用柔软的刺痒我,在我酸弱时湿漉漉地粘滞我。
\par 独爱打捞我,我不能说,谁是谁的名字。我不能分辨面貌,确定性,蜜状,孔洞奶酪状,酒质,玉或珍珠质。爱在永远攫取。
\par 未解部分,用手指探明部分,夜的草原上溢满不可名状的情感,预感到对有情与无情一切的爱。爱神降临的时刻,爱神死亡时刻。
\par 在面对爱神我无有悲哀,我无有空虚。大地鼓风,动摇我的皮肤与毛发。
\par 面对爱神我已不再纯净,我已沉落浮定,我已吻上爱神最后一丝微笑。
\par 你是谁的,爱你想在我寻找什么。爱你创造新的表述,一份秘密,用肉体与蓝色和黄色的生命参与表述。
\par 我已确定,在这空洞舞台。我是你爱的描摹,去远处参加希望与绝望的礼赞。
\par 我求你引领我,就到你的身边,见到我自己的真容。
\\[0.6cm]
\par 我如今还不想堕落,我还不能散逸我自己。
\par 朝圣就是爱之轮部分,河,以及一切转瞬物,有想非有想,非想非非想,意识生灵与死,我还不能放任一切堕落。
\par 落入可能之中,爱即可能,即可以,是不由纷说。
\par 像爱神一样行走,大地受到足踝的焦灼。足踝绽放花朵,枯藤,败叶。我不能抑止,爱神与死神一个向我微笑。
\par 爱神向我投来名字,面庞。她捏好,斑驳的,不会哭泣的心脏,投我。
\par 当我缠上丝带,你们带我去舞蹈。你们告诉我时间,砂子打磨砂石,海洋吞没海崖。我怎么能不从话语中听见爱。
\par 燃烧与冰冻在动脉中跳动,又从静脉中流泻。我所恐惧除了自我,别无他物。
\par 甘美,可爱之欺惘,惘然,怃然,自此是我与爱,我之爱与爱之神。登上又一座塔楼,带着我们微妙的烛光。
\par 烛光里是翻飞轻扬的希望,勇气与热情。而滴落的烛泪,是肉体,生命,血。
\par 爱神托着红烛,爱神渺渺,我像黑夜一样舔舐着爱,直到一切都模糊不清。而我也干渴。
\par 夜在震颤,拂照她的星辰都震颤,絮絮语,每一个爱神的名字。
\par 她为何如此之美,她的目光,为何就是黎明的光芒,我看见她,等同我窥见一切诗的脚步,看到我的死亡。
\par 她来,到我身边,抚摸我的死亡。她闭目,到我心脏,拨响我的死亡。
\par 一切黯然神伤中我奉持幽暗之心,一切幽暗朝圣路上我将绽放水纹,雷音,歌吟。我看见一切写就在我的肌肤。
\par 孤独诗人朝圣,孤独诗人携我而舞。
\par 是以记下朝圣中,想念爱神。
\section{死神篇}
\par 从底渊浮现上来,死神命我不能回头。
\par 似曾相识,我来到莲叶池旁。
\par 是你们来迎接我。自我出生到如今。教会我言语。
\par 在这时间里,坐,卧,奔跑,在纯净的黑色里,喰,泣,抢夺。
\par 我们渲染过的颜色,我们勾画过的形状里面,你究竟想留下什么。
\par 你叫我遗忘,你叫我思想,你用黑色的眼睛盯住我。
\par 在树梢出生又腐朽。爱抚又撕裂。
\par 用火炭雕刻我的心脏,直到它冷却。把名字领来,叫它迷失。
\par 居住在自我的深处,叫我追求你又背弃。一切纠缠至今的名字与爱与死。
\par 仿佛我又听到了一千种合唱。大地朝四面八方震颤。
\\[0.6cm]
\par 一次又一次,伴随我。叫唤我。舍弃我。
\par 我的同伴埋我,悲伤轻柔。信仰,仪式与宗教。用智慧引我。
\par 在水,在火,在秤,在目光,在回忆中。用舟载我。
\par 用白色,用黑色,用风抚我。在我爱时言语我。
\par 敲醒我,拿刀割我,拿剑刺我,拿我苦绝释然我。
\par 是爱,是死,是你,是我。
\\[0.6cm]
\par 今天我见到你,我从所有人的面孔中见你。你对我哭笑。
\par 我从人偶中见你,我也见到你腐朽,你从雨水里降生,又在尘埃里睡眠。
\par 我开始明白你不存在,人们对你的故事都是假的骗人。
\par 你是诗句列成行,你是文字的使者,你是一场预告。
\par 死是雷声在大气与土地中间游荡,是唯一的无趣的神明。
\par 我们不用这样的语言,死神没有秘密。
\par 死来到我的身边,死是如此的憔悴,又是如此的欢愉。死只有暧昧的双眼。
\par 死亲吻我,流连我,对着满世界空荡的回音,自我感动,自我抚慰。
\par 对死我只有最后的诗言,是谁在廊下不停徘徊,谁在窗外。
\\[0.6cm]
\par 今天我见到你,你用我的言语向我诉说。你告诉我秘密。
\par 好了,我想我们之中需要一些坦白,我们,或者你就是我。
\par 我们不是要去迎接,我们和爱,还有大水,和一些火,不是要去迎接的么。
\par 你是,我们是,永远的堕落,如此的静谧,永远的恍惚。
\par 我们不是正准备把颜色铺陈,把形状镌刻。好了,死亡。
\par 死神消减我的精神,摧毁我的肉体,又使我带笑,使我明亮双眼。
\par 好了,你有这一双瞳眸,该你来记,你也像哭和笑一样来记。
\par 替我记下万象浮起落定,替我记下朝圣路上一切幽暗。
\par 再记名字,言语,记面目,记下他们的信仰,他们全部的路。
\par 是以记下,朝圣中,朝见虚妄的神灵。
\section{诗神篇}
\par 在你们之中行走,诗来了。
\par 在你们之中遇见,听见,我与你们拥抱。
\par 在你们之中灵性,爱情。
\par 肉体之中有温热空洞的声音。
\\[0.6cm]
\par 诗来了,诗像你,像我,像轻柔易碎的。
\par 夜来了,夜如同一寸寸的流泻。
\par 夜占据了海洋,酒,诗心。
\par 我与你们一道,月娘,金子,
\par 我念起你们的名字,细碎的,金色天空。
\par 折起句行,遮住瞳眸,我,
\par 河上居民,诸神,抒情,迷惘。
\par 河上凝望,打磨卷起,回忆透明宝石。
\par 预备辨清你们全部面目,
\par 预备幽暗,森罗,暧昧。预告,
\par 我们只有一起去朝见诗,
\par 求诗,同它攀谈,献诗。
\par 询问关于希望与绝望,关于爱与死,
\par 在街市上,在山岗,在热冷的暴雨,
\par 垂钓星辰,勾勒沙漠与生命。
\par 是画匠,还是乐师,是情人,还是老人。
\par 只是一曲普通的诗,愉悦的诗,
\par 苦闷,稍加思索的诗,
\par 提示我在你们之中。生在藕池。
\par 在时间摇曳,思念粼粼,诗来了。
\par 使我相信,我与你们一同笑了。
\par 就是哭了,好像我们很久就认识了。
\\[0.6cm]
\par 眨眼,然后把秘密告诉我。
\par 七彩的,游移不定,好像思念投射在川流。
\par 诗神来,捧一颗心。
\par 诗神来,他光着脚,脏兮兮的脚,
\par 捧来一些星星,发光旋转的气团。
\par 捧来一堆黑暗,凝望时所想见的一切,
\par 都如此熟识,随着季节嬗变,
\par 你我一同明晰又遗忘。
\par 像光明来到,紫色的稠絮传递的消息。
\par 裹紧我,动摇,摄定,用火焰闪动我的双眼。
\par 像红色,蓝色深底的草原,
\par 我们走过的,明晃的河床,
\par 只你向我走来,分秒,日月,年岁。
\par 靠近我,摇曳,啸叫。
\par 诗神摇晃着乱发,黑色的,金红,
\par 他替我带来一条冰溪,鹿群。
\par 每一生灵,就是一诗,
\par 替我取个名字,从生到死。
\par 让我一点点描绘,背弃你,背弃我。
\par 让我从宇宙的中央,
\par 漩涡引力的静止点开始。
\par 万华缤纷散霰,我只有人间心情,
\par 在你们之中去来,带走,挽留,
\par 皆不能,未曾听闻,忧悒。
\par 怨怼,如此疏离,如此陌生。
\par 一切都是一模一样的面目,
\par 一切都是一模一样的行走。
\\[0.6cm]
\par 土地上有雨水的痕迹,
\par 是恋人们的留言。
\par 在你们的身影浮现又消弭,
\par 我渐渐找到一种言语,
\par 就在孤独的时间与喧嚷的时间,
\par 互相游荡,拥抱,或者杀害。
\par 我们的生命如此独绝,
\par 只让我们愈加伤心。
\par 浮现心喜,浮落哀恸,
\par 你们都渴望我分说什么,
\par 乞来人间颜色,然后分食这些蜜。
\par 无妄的诸神祈求什么呢。
\par 凋零在思念中的食美巨兽,
\par 荒原上的痴人与梦游人,
\par 你们祈求什么,为何又抛弃。
\par 来到你们中我皆是沉默,
\par 我带来的是黑暗,而非七彩。
\par 我带来诗篇,就是此诗,
\par 从云水和冰川上抄来,
\par 从彗尾和极光上抄来,
\par 聆听的是每一瞬间分裂,堆积,
\par 万物生长,枯萎,执镜,离座,
\par 众生踏上朝圣道路,河上,沙上。
\par 世界沉睡在眼眸上,
\par 花朵在血液中绽放。
\par 我们都在这转瞬的幻象之中,
\par 时空将我们分离,
\par 回忆与日光已将我们浸没。
\\[0.6cm]
\par 每一位旅人从远方带来蜜,
\par 为我带来花朵,果实,书本。
\par 可是我给你们带来夜空,明月,
\par 歌人带来了草原与天空。
\par 如今我带来简单的想念,
\par 就是让花朵绽放与枯萎,
\par 迷离,虚妄,无解,
\par 就是众生的言语纷纷凋零,
\par 铺满了整片森林。
\par 秋冬春夏,期待一抔火,
\par 期待牧神,密林中朝拜的旅客。
\par 你的目光就如同时间,
\par 时间又如同一束火,
\par 火即是河上涌流的思念。
\par 告诉我你在这黑暗中又遇见了什么。
\par 你可以告诉我,
\par 太阳在土地上播撒了什么,
\par 你为何走向我,
\par 转头又远离。
\\[0.6cm]
\par 轻轻悄悄,颜色来又离去,
\par 云来又去。我想念云,
\par 一砂,一滴,一枝,一瓣,
\par 我走到世界中间,
\par 你们亲吻我的脸颊。
\par 天空聚拢又碎裂,
\par 青红又紫,枯黄又绿。
\par 这仿佛是水的心意,风的心意,
\par 笑神,携着哭神,
\par 大梦初醒的爱神,与一片晚霞。
\par 我在安静的时空里游荡的时候,
\par 你们就一起跟随我。
\par 自由而暧昧无理的诸神呵,
\par 当你们一齐来到河上,
\par 是不是我与月娘已经不再亲近。
\par 仿佛疲倦了闭目,
\par 扑向幽暗,光明,深渊,谜语。
\par 是我与世界已不再同一属类,
\par 我与你们有这样的陌生。
\par 好吧,你们要重新抚摸我的手臂,
\par 替我着衣。
\par 永远听闻诞生之音,
\par 听见世界在树梢浮现,
\par 雷音,不安的颤动,宣告。
\par 诗人为何在仰望,
\par 诗人为何侧耳倾听,
\par 一切仿若欢歌,仿若预言。
\par 在我们自己的倒影里,
\par 悄然生长。
\\[0.6cm]
\par 转身离去,使我难忘,使我抑騃。
\par 纷至沓来,一双明眸,
\par 明眸与迷惘的颜色。
\par 给我一些理由,我再去寻找世界,
\par 我要去寻找新的祈盼,新的祷告。
\par 那些虚无的幽影群聚在天空中,
\par 一切聚散不安的言语,
\par 花草树木,鸟兽鱼虫,
\par 都这般眯着眼升上幽空。
\par 我们一齐坐下,面对河流,
\par 要把手放在我的身上,
\par 一边牵着土地,一边牵着天空,
\par 四处都是空荡。
\par 给我理由,相信黑暗,
\par 相信热爱,逃离,美之沉重。
\par 为何你们悄声耳语,
\par 我听到悲伤盘旋而郁悒涌起,
\par 等待一场又一场飘散的羽霰。
\par 这一切的躁动是如此安宁,
\par 好像我来去都是同一面目,
\par 墨染,勾纹,万千眼眸,
\par 万千星点,夜之镜,执镜女神。
\par 窥往一片青绿,大水浸湿,
\par 紫色,紫色的瀑布,
\par 金色的浪花,鼓动心脏。
\par 你们究竟是世界的神灵,
\par 还是自我的射影,
\par 我们一起去听雨的歌声,
\par 春雨夏雨秋雨冬雨。
\par 你也像我一样期待么,
\par 像我一样游离,
\par 在暗室中朝拜内心或宇宙。
\\[0.6cm]
\par 温柔的近至暧昧的天使,
\par 世界透明。
\par 醒来时又忘记一部分言语,
\par 每一位天使来触摸我的额头。
\par 这场永恒的抒情与索寻,
\par 人间重重叠叠的影子,
\par 好像泡沫一般,阻塞我的耳蜗,
\par 膨胀脑体与四肢,指尖。
\par 像是大海扫平礁岩,
\par 飞旋的鸥燕只剩下一对眼眸。
\par 我预感到这些词汇开始崩塌,
\par 从尖端一角,
\par 草地上,或者母亲的羽翼里,
\par 温柔的词语在大气中氤氲,
\par 悲伤的词语躲在幸福的背面。
\par 是时候了,
\par 请求你们抄录诗歌,
\par 抄录一切语言,在你们的羽毛上,
\par 你们的肌肤上,
\par 从太阳到雨水。惊雷到花萎。
\par 把我们的念想留下,
\par 宝石的土地上大水的痕迹。
\par 一切都是惊鸿一瞥,
\par 一切都在转瞬中翩翩起舞,
\par 像是我与天使,幽影,和诸神,
\par 转瞬哭笑,转瞬相识又聚散。
\par 这去去来来的一切都已重叠,
\par 我要去何处认识你,
\par 聆听,用手指触摸你。
\par 一位诗神,
\par 儿童,青年,中年,老人。
\par 男人,女人,天鹅,狮子。
\par 一位斯芬克斯,
\par 沙漠之中的凝望与微笑。
\\[0.6cm]
\par 这讯息濡湿了我的长发,
\par 又像是腐烂四碎的花瓣。
\par 你们低吟,哼唱,
\par 像是浪潮送我到岸边。
\par 永远漂浮,有时想爱。
\par 沉沉地抱住一颗月球,或者太阳,
\par 沉到渊底,听不见你们言语。
\par 用我的眼眸描摹你的眼眸,
\par 润湿手指在沙漠上,黑暗上,
\par 抄在塔或者夜幕上
\par 写书。一行行写点诗,
\par 不想表达,真实或者虚拟,
\par 我时长途跋涉,
\par 溶解,呼吸。曳摇,惊动。
\par 我们到哪里再相见呢。
\par 因为我开始丢失道路,
\par 丢失森林与挂念的名字,
\par 不明白粮食与饥饿。
\par 伏倒睡眠在河央,
\par 我该去哪里找到。
\\[0.6cm]
\par 现在我找到一段声音,
\par 窗外的树腆着叶子,
\par 鹊鸟从一枝跳到另一枝。
\par 我确信我已毫无思念,
\par 再也没有孤独的流浪。
\par 此处是河流入海口,
\par 泥沙被裹挟至此,堆积,
\par 闪闪,莹莹,好像记忆碎屑。
\par 我已确信我毫无悲伤,
\par 或是欢喜。泪流满面,
\par 树叶蒸腾出的水汽,
\par 在树根降下小雨。
\par 现在我看向你,
\par 爱神眨动双眼,
\par 死神牵起我的动脉。
\par 命运的三位女神拿着镜子,
\par 火和一条小船。
\par 你们走向我,
\par 哭神攀着笑神,
\par 而笑神伏倒在地。
\par 居住在北海的鲲鹏躲着,
\par 任凭那极星闪烁絮语。
\par 水仙的花瓣飘来,
\par 还有河流的三位女儿,
\par 好了,我的朋友们都来了。
\par 我就是你,我爱着你。
\par 你就是诗,我也是诗。
\par 我与诗人以目光相认,
\par 太阳与我以触摸,
\par 草原以呼吸,岩峦以负重。
\par 我确信我走向你,你走向我。
\par 我们只有简单的心灵,
\par 在路上写简单的诗。
\par 现在你听见了,
\par 只有永无止境。
\section{心篇}
\par 朝圣,听闻心灵的消息。
\par 大气在人无可企及的高度兀自横转,聚合,疏离。时而降下雨水,时而偏折阳光。
\par 森林在扩张,侵占土地。聚拢枝桠,呼吸,从土地之中吮吸水与矿质。而后腐朽。腐朽变成土地之一,变成新的森林。
\par 河流转瞬折射出世界的千般剪影,却看不清其间任何一物。
\par 沉底的卵石万年接受打磨,最后也变作水流,拥吻其他卵石。
\par 人群聚散,城市兴起又腐朽,语言连撰成诗,又丢失音节与顿挫。这一切仿佛都是在转瞬中明晰,衰远。
\par 夜空中的星辰旋转,放射光芒与电波。直到群山,峡谷,海洋与河流,冰山或者极光,都在仰望,都在交语。
\par 使我们的麋鹿仰望,豹子仰望,鱼虾贝蟹,仰望,夜莺噤声。草原上游荡的象群,马群,狮群,狗群,一切都在仰望。一切都在吟唱。
\par 心。
\par 太阳之心,土地之心。
\par 仿佛远远响起鼓声,响起漫游中曾经听闻的钟声。侧耳倾听。
\par 凝望,勾勒,挑选词语描绘。诸神捧着它们的心灵,一些七彩的宝石,或者是流质,从指间溢出。
\par 幽暗之心,在月下的沙滩上,城市的灯光倾轧着波涛。
\par 透明的海水,透明的月亮。透明的血液与脑体,撑开一叶叶,仿佛是黑色的叶子。
\par 念想,索求,祈祷,或是熟识,土地上的亲近的心。桃花,樱花,李花,美丽的世界永远般散霰落,升腾,千千万万心。
\par 觉醒之心,沉郁之心,怅然若失,蓦然回首。
\par 只是凝视间,欲言又止,沙丘上,书卷上,流水上。我与你们相视,言语相闻,以肌肤象触。于是心由此诞生。
\par 一切转瞬中获得心,转瞬中消没。一切在时空中去来,七彩中消没。
\\[0.6cm]
\par 向世界中心缓缓前行,云和雨在我身边。
\par 来到你们中间,这个人怀着怎么样的念想呢。是心喜还是遗忘呢,像是木花香气,在大雨中间透明地溢散开来。
\par 如今我还有一点渴求。我想一些微小的事物,在行将落叶的紫叶树上的雨珠,漫天翻滚的云海中微微露出一点蔚蓝的天。
\par 我只要心的提示,一双眼,山谷中溪流的道路,鸟儿群聚在丫杈中央静默。在铁道,车流中,手掌相触,连绵不绝的电缆。我需要一点点提示。可以在时间之中捕捉到,俯仰,漫游,渐忘。
\par 独自游荡,顺着无情的道路与有情的道路。我在黑暗中找到花海。
\par 拥有这样迷惘的秉性,困倦,思有雨夜,想有昏月。
\par 如此漫长地沉默,怀有隐秘的叙述。漫长在人群之间徘徊,开始回忆每一形状,每一束颜色,悄然而新鲜的,静静的名字。
\par 触目过后变成心。听闻成心,路行成心。在湿冷一切暧昧不清的空间里,你是我恍惚的生命,七情六欲来触碰我,耳语我,这历年的诗行缠绕我,烦扰我。我还想在最后拥有一点提示。
\par 你之双眸,一位老人,或是一群幼童。一树喜鹊,沉沦。
\par 好似我的念想都已消融,我之祈望也已乌有。只有创造新的言语,大雨在岩石上摩挲。
\par 像我在黑暗的宇宙里窥视你,像我暗室中歆羡已然凋亡的美。是不是我们都注定要在黑夜中追逐闪电的瞬间。
\par 心是不是这样一种细碎的环状物,冰与碎石折射着光芒,环绕着你我轻轻动摇。
\par 心是不是当想念殆尽时,河床底露出,卵石与砂,浑圆无比,互相映照。
\par 好似又见证你与我从诞生到死亡,森林与海洋从星河中吞吐出现又没落。我只是永远在寻找,永远在拥抱。
\par 如果我在黑暗中向你问起,荒野之中竞相开放的菊花,夜与我对坐长谈。
\par 大概我有深深的迷茫,又有难尽的思念,时间与天下经历于我,在我身上生根发芽。
\par 浮动水草,或惊碎石崖,夜被分割,离离不绝,涌向岸边,又疏离人间。
\par 馥郁醉人,我早该明白的,在荒原中央呼唤的神灵,曾经赠给我们又散在梦游的各个角落里的歌声。
\par 只是默默守候着,守候河川,提着灯笼缓缓地照亮游廊,流星击碎天空。
\par 名字唤起,迈动脚步,温暖的寒冷的饮水。为何思索迄今,为何彻夜吟诵,为何只剩下一双眼睛。
\par 求你务必漫长,在我开始思念前。
\\[0.6cm]
\par 平静混沌的夜空里有七彩的闪电。是不是我怀有不实的祈愿。
\par 我究竟想在这里,这个时刻留下什么呢。我想刻在石头上面的,传达到声音里的。指向山上翻滚的云幡。是不是这些你都陌生了。
\par 是我像你,你像我,面目一般,但相隔甚远。
\par 这世间许许多多的事物都一一凋亡,名字被唤起又忘掉,我只是觉得不好。
\par 我想再去守着火焰,坐到三位女神的旁边,聆听,看火星上扬消散。就是温暖而明亮的火光,轻柔地照亮方寸世界。
\par 开始想念河流,雪,想念梅花,而后是雷雨或者石桥。
\par 终于可以我放弃思索。开始思念,夜思念我,我思念花。
\par 深深的安静,仿佛一切都是隐约的提示。出神,永久绝望,永久想念。土地上,河流上有珍物与美丽之物么。
\par 这连绵不绝的心念为何如此沉重,还是如此轻浮。终于有一点,紧张不安地跳动,仿佛是营营不休的理由。
\par 从远远的地方到来,全辨不出来时的路。
\par 杂草丛生,汪洋浩淼,漫漫的人国中,眨眼之中也看不见离去的路。五彩斑斓的指间渗出了黑暗。像一颗种子。
\par 像沉睡在岩层中的骨骸。金属的脑体,透明的晶体折出七枚瞳眸。
\par 你们为何不沉默呢,沙漠的孤独之心把我包围。沙漠的双眼,河原的双眼,黑色海洋的双眼。
\par 你们早就明白了,我想清明与迷惘是心。我想诗神是心。朝圣是心。
\par 在这迷茫的江面之上摇摇晃晃的木舟。好像温暖明亮的星光照亮我一人。四下一人。
\par 世界的心思都回归此处,四散的心意都聚拢。让我明晰,在草地上,在月华流照的每一寸土地上,生长,腐朽。
\par 一个佛龛,无头佛像。坍圮的殿堂,一只土狗。这个世界可供纪念的事物奉仰在枯石河央。
\par 好了,你不用再迷惑,在白昼沉眠,又在黑暗中醒悟。或坐或卧,舞蹈,堕落,如同来到,又如离去。
\par 你开始诱惑我,纷乱夸张的世界迷蒙我,宛如万华之镜,万物心意在深深地聚合。
\par 我想我是你,我之眼眸是你之眼眸。我之漫游是火之漫游,是缥缈中凝望的长久。
\par 一切我从黑暗中悉然描摹,如我祈慕一颗心,众生形色的心意。风仆尘尘的道路,重峦叠嶂,悠远的山风呼啸,而河流就在幽冥的地方运行。
\par 夜之眼眸呵,夜之眼眸在转瞬中塑成的世界,也在千万的眼眸中转瞬凋零。
\\[0.6cm]
\par 仿佛在腾腾上升,旋即被游离的风雨打散。
\par 轻柔地下落,摇摆,翻卷,斜斜地消逝在视野边角。
\par 还有沉静的,墨色,一动不动仿佛疏离,自在,孤独地守候着繁芜。
\par 七彩的心意触碰我,稳定地,由远到近处,一一抚尽,一一拍遍。风流品物心意从久远地拜见我。
\par 从我奔跑到我死亡,从记忆到诗歌。疼痛,紧张,恐怖,激动,淡忘,反刍,偶蹄。
\par 万千面目,万千诸神。言语朝圣,名字朝圣,草原中央的一条无尽的道路。雨云牵住月娘,城市倾倒在海洋。
\par 万物心意即是土地上的神迹。黑暗中满山遍野的木花。
\par 如是我渴求,从远方到天明。阴云翻滚的天空是吾财宝。大地隆升,潜没。
\par 我想天空与土地一起沉没,有压抑,悲伤,惊惶,心死。我想明白翻滚的海潮,我想明白夜幕升起落下。
\par 我想听闻你们,我想听你们悄悄私语,同雨水分享土地的秘密。
\par 在黑暗中求安慰,直到季节枯萎,时间落定。我的心意尚未到来,在这万万千千一个,所有名字中我不明白一个。
\par 谁走丢了,穿过雾霭。为何呼唤不见,为何想念群聚在土地上。
\par 想见一只雀鸟,一只云雀。想见喜鹊,闪烁,吾即是闪烁。
\par 我是灰暗天空下恍悟,冷淡的日光柔软地流洒在土地上。我是东方迁徙的旅人,从北向南,寻找水与热。
\par 言语塑成世界,言语塑成心意,启程前往迷途,遗忘,堕落。清风拂乱长发,明星点亮瞳眸。你要是我,你要是凉凉的白色。
\par 在你们透明的国里,找到无数自我的墓碑。在你们的史诗中,诸神的名字。
\par 有我思念你,我想你,我爱你,我想为你描绘黄金的世界梦,我想天使与幽灵都可以悄悄地言语,我想要真实的名字,我想要流浪,吞吃火焰,变成羽类,鳞族。
\par 我要变成你,我有全部心意,全部送予你,你是谁呵,你是谁呵。
\par 心即是歌,心即是吾哭,吾奔跑,心即是汝,我要这心变成舞,变成舞舞舞,变成四碎缤纷的花瓣,我是谁呵,我是谁呵。
\par 我已经永远明白心的一切,像我千千万万次从那奔流的河中,打捞上来,闪闪的宝石,像我们对坐攀谈。像我们听见月亮与太阳在天空中游移。
\par 是以记下,朝圣之中,心动。
\section{土地篇}
\par 朝圣,遇到久远的土地。
\par 一片我久未探访的土地,今日重回。
\par 一方山河,一滴湖海。不见故人,独我一人,再回到这片土地,淡薄清冷的回忆。
\par 回来时我与土地一样苍老,混沌了双眼,斑白了毛发。我与他一样沟壑葱茏。
\par 我们的欲望一样苍白,一样混浊。我们蹒跚的影子一模一样。我的土地,我的故土,我睡眠与生息之地。你用你的苍老哺育我,让我到远方去流浪。
\par 我从最黑暗的地方回来,带回来一些可怜的宝藏,已丢弃。
\par 我带回来一点水,就是故土的水,沿着这条河流我又走回来,伛偻蹒跚,在无尽的黑暗之中无尽梦游,我带回来一些梦。
\par 土地,好像我这孤独地漫游,你亦是我漫漫行路上一中途,你亦是我梦见。
\par 是我的父亲与母亲,是我的妻子,我的儿女。我的挚友们在山之高远,我珍惜的半亩水田都收割了。
\par 土地,我的朋友都在你这,你要款待他们,用你的蜜与奶,你要送来清风与香气,款待我的鸟儿们,我的花儿们,每一滴雨儿,还有诸多神灵。
\par 这一切都无言,在我离去时你无言,同我一起坐在火旁。我在渊里苦痛时你无言,我在黑暗中描摹七彩时你无言。如今我终于回来,你还是一如当初无言,也不见岁月在你的面目上更添刻痕。
\par 只是我老了,我历经春夏秋冬而后老。我开始听懂你的无言,我也听懂时间,一叶扁舟轻轻地漂流在江水上。
\par 让我掰着手指算算,让我告诉你我认识的朋友们,我在人间见到的乱象,在夜空中感动,太阳祭祀,我亲见诗人的复活与加冕,让我再告诉你几许歌谣,我自己编的,和他人那听来的。
\par 为何追忆这些,仿佛被露水又打湿了衣衫,仿佛天空与土地一般澄明。
\par 是了,这条路上我慢慢走了长久,也没有谁陪伴左右,有些我思念的人,也都不知了去向。
\par 土地,好似我一直坐在你的花树下,春风,夏雨,秋香,冬阳,好似我从未离开,只是一场大梦。
\par 好像我就是生在河边这颗树,今日秋寒又吹走了翩翩枯叶。
\par 土地,天涯就是此处,星河即是你,我来是尘土,去亦是尘土,吾即是土地流浪人。
\\[0.6cm]
\par 我和你们一样,我也会疼痛,我也在忧伤。今日我见到你,我的内心要轻微隐秘地震动,然后碎裂。
\par 然后我睡在土地,我吞吃泥土,攫取水分和矿物。迈动四肢,指爪,我刺痛的神经,奔跑,振翅,穿越土地。穿过烟霭人国。
\par 然后有一天土地上的生灵来见我,每一个都捧着一颗心,然后用血吻我。用火拥抱我,用水哭泣我。
\par 你们带来土壤和心脏,带来纹身与雕刻,土地上洒满腐朽的脚步,尤其是从月球到太阳。
\par 我给你们带来一支琴,几许暖阳。土地上一些旧的花,旧的流水,淡退了颜色,荒芜了道路。一起我们到万物的国。
\par 出走烟尘,出走时光。我与你们一样只有平凡的祈愿,有人间的牵挂。
\par 像是藤蔓上紫花,春夜青空明晃的月圆。像是这片土地上我曾见过许许多多的事物,每一个都叫不出名字。
\par 土地,你,我。然后是草原,山脉,简单的事物。茅舍,木屋,角亭。开始我是一个痴人,是流浪客,你是公主,是一个女词人。
\par 开始我有思念,从土壤中抽芽而出。破开卵,呼吸,破壳,鸣叫。在土地上,山海间,我究竟希求什么。
\par 我知道我想变成土地本身,变成一粒种子,落叶。山顶光秃的白石,鸟儿颤巍的巢,变成一只流浪的饿狗,我想变成我的思念本身。
\par 天下时间在无情游荡,在天空中倾泻西东。
\par 我明白如今触目的一切就是土地,比如踟蹰的行人,枯黄的树叶,灰濛的天阴。如今是深秋,可马上就是重阳,冬天就要来了。
\par 晚风送清霜,凡间的颜色一一凋零。我们还要在土地的何处相遇。
\\[0.6cm]
\par 夜来了。谁还在仰望,谁又迎来,抚琴,登楼。
\par 安静天涯客,每一位都好似故人。山色徙鸟,落金夕阳,每一株寂寥树,想念太阳想念月娘。
\par 我知道秋要尽了。土地上的颜色要一齐躲藏了。只是我的想念最后还能成为真实么。
\par 久久平安的等待,此地又是何乡。到阳光与青空下的波纹中间,众生漫漫如同埃尘,每一去来都是土地上的旧事,你们我都在等待什么。
\par 求得暖风,求得万里来人。好似在等待一种声音,鸟儿振羽,夜的绸缎一匹一匹,或者等待一个提示,当我睡在草原或者沙漠,大地的岩峦久久地隆起,旋转。
\par 一种遥远的梦把我带到此处,空空荡荡。同是旅客的我们从远远的地方来到此处,旋即仰望,聆听,参拜土地之心,在处处都找到真实的痕迹。
\par 我们在等待真实与美丽,等待心灵降临。留下依稀言语,寸许月光。迷茫在土地上索寻并纪念时间消失的痕迹。
\par 开始我遗忘祭祀,遗忘一双双眼睛。
\par 脚下,一颗岩石的星球在膨胀,冷却。一种大大的遗忘降临在波澜的时空上方。
\par 纪念早已消逝的凝望。纪念,象群的母亲,龟裂的河床,以及洪水中摧毁的谷田。
\par 我用生命在土地上写过的书,现在还会流传么。我们之间传递的衰老,隐秘的绝望与情色,会在某一株树下停顿,陷入沉思么。
\par 开始我相信智慧,淘洗破碎的语调。遇见雨国,花国,果园,开始记述,绘画。
\par 编纂失明的意志,咆哮,哀啸。一起来到山坡高起的地方,同鸟儿一起停歇,回到故乡,像黑暗在天空铺展,光明在土地深藏。
\par 叫我重新见到土地,漂泊在江河湖海。一切絮语都是此次等待。
\par 我要见到寒冷和温暖的深处,如此在地球的底端,或者靠近湿润的热带。我怀有一个诗人的无声孤独,一对拥抱的年轻恋人的莫大哀伤。
\par 就在纯白的时候,三位河上的女神,三位守着火焰的女神,还有在荒凉的土地上呼喊的神灵。你们这么多名字意味着什么。
\par 要在落阳湮没时,高高的山巅上播撒七彩的绪思。要到落雨时传播迷惘的教义。
\par 如今我触目一切即是土地,而眼眸是太阳。
\par 一点点立足之地,承载着奔流的思念,飘卷的绒毛。温柔地抚摸并梳理着悲欢离合,阴晴圆缺。
\par 终将我来到此地,像一只食花的野兽。不再理会一些杂乱的梦,以及重塑诗歌。
\par 我必然遇见我的母亲与父亲。夜,或者星河,雪溪,棕榈,一些复杂的名字。遇见我的亲人,我的爱人,围坐在冰冷的帷帐中。为何有了这样的意象。
\par 渴望见证土地上真实之物,不是在言语中,不是思念中。那只有在夜幕中。
\par 夜来了,我的心也是这夜幕的一部分。开始我闭上了我的双眼,漫漫的土地裹紧了这副躯壳。
\par 夜来了,而等待真实的人希望早已落空,黑色的大地之上有狂风吹拂。
\par 现在我们之间互相对视,黑色的瀑布惊起黑色的鸟群。
\\[0.6cm]
\par 想念诗歌,一位走丢的诗人。诗人与草原连接,草原与月球连接,月球与我连接。
\par 土地是一些面庞,某种歌声。味道,甘美。某些我决不能了解的,山脉隆升,冰川崩裂,每一瞬间鸟之眼眸。
\par 土地是某种心跳的声音,婴孩的心跳。花树华落时候的颜色,以及伏倒在冰面上,听到的隐秘声音。
\par 把金子堆到河滩上,把银子洒满天宫里。如今我已分辨不清造物的名字。
\par 土地上只有一些断裂破碎的情感,追寻温暖,降雨。
\par 近至安静的漫长,在列车站台听到一切的疲倦的故事。也只从此开始,扇翅高飞的秋虫,一切都是现时当下,腐嗅,葱茏。
\par 土地只是此时此刻的永恒降临,朝阳前,淡漠的紫霭。低矮的,冬之讯息。
\par 怀念,或者挽留。土地,就是如今你我长念已久的命运。我想前去聆听,每一颗安静的杉树下,只有一点简单的念想。
\par 带来火焰,骑上骏马。同你们一样,我也带着弓箭和长刀。而我已不再是痴心,不再梦游。开始思念,一颗土地之心,永远朝圣。
\par 如今我站在命运的尽头,无关东西去来,孤独的土地在孤独地旋转,追赶高原上流逝的云彩,我已忆起那首流传已久的情歌。
\par 如此短暂,如此无言。使我焦急,惹我惊惶。世界在外,婆娑如迷,抚琴人有抚琴心。
\par 现在你要到来我的春秋。你要告诉我有关遇见的故事。
\par 我要见你,于是你来到。我睁开眼,于是山水。
\\[0.6cm]
\par 在初阳下,在秋叶上,我终将悲哀的时候,为自己创造一点点真实。
\par 像我把铺满地的落叶,在柔软的草地上,或者柏油路上,我知道我自己必须要有的一些讯息。有关诗歌与纯真的言语,必须去推动的命运之轮。
\par 我想和你遇见,飞向太阳的黑色鸟羽。
\par 我将是土地之中,永远平凡,永远揪心的起伏微尘。我将是暴雨打落的红叶,最后变成土地的一部分。
\par 同它一起凝望,一起哀哭。各自有一些朋友,秋冬来到,春夏离去。
\par 好想停下笔,推开窗户。我与你之间简单或者闷热的对话,因为我不能把你拥抱,这就是我全部的思念。
\par 因为我至今还有简单的想念,所以我将随着土地一起朝圣。
\par 因为至今我还有简单的想念,言语纷乱毫无感动,在大地上湿濡地摸索,我还不能更有渴求。
\par 穿越灯光,穿越航迹云,街头的小说里一些简单的念白,我偷偷从异国的语言里听来的一些告白。可是我还是不明白。
\par 因为没有你我要永远流浪。我将独自写诗,俯看枯萎的花朵,听满池的败藕。
\par 星光隐去,而月娘还在地平线下的时候。我与老人对坐在寂寞的火旁。老人就是我,火就是我,夜也是我。
\par 因为没有你,我会变成尘土,我会唱歌,我终将在每一次梦醒获得真实,比如三两滴泪。
\par 究竟我不会在天下何处遇见你,我也会离开每一片森林。我要告别我的朋友们,告别美丽的生命与一切七彩的折射。
\par 因为我迄今所明白的一切真理,所找到的一切宝藏,人间的艺术与死亡。都将永无止境地走入朝圣之中。
\par 都将随我一起追寻,思念的极点,宇宙的奇点。追逐平原,电车,无聊的人类生活。
\par 所以我才会在土地的每一个角落种植爱情,不加耕耘,所以我才会迄今遗忘,时而热泪盈眶。
\par 是以记下,朝圣之中,听闻土地的消息。
\section{太阳篇}
\par 朝圣,在失明的时候,窥见太阳。
\par 朝圣,听闻久远的祭祀仪式,太阳祭祀。
\par 朝圣,世界是太阳的反映,太阳是诸神的内在展现,于是一切牵挂色彩,一切游定形式,是太阳的投射,是真实的成立。
\par 是以记下,朝圣之中,赞美太阳,赞美众生之平安,赞美世界之诞生与渴望。
\\[0.6cm]
\par 起初是黑色,土地到心意,水与火皆是黑色。
\par 土地之上群聚的生灵,也是黑色。沉眠,流浪,哺食。
\par 诗人与记述的神灵是这黑色的一部分。言语尚未诞生七彩与光明。
\par 于是我们睁开黑色的眼睛。金色的火焰在土地之下运移。
\par 黑色的眼眸赋予我们金色的天空,思念以及渴求。
\par 金色的天空正在哭泣,哭成一滴,一条河,变成男女老幼,生老病死。
\par 哭成七彩,哭成悲欢离合,质疑,置信,希望,绝望。
\par 天空哭满土地上每一寸的言语,于是有歌,而后有诗,而后祷文,祭文。
\par 其间微末的一滴,透明,温柔,仅此一滴,在我们中间。
\par 就是太阳,集中了全纯的哀默与欣喜,如此沉静,入迷,微笑。
\par 永恒或转瞬,当天空凋零,而大地淡忘,言语失传,一切的神话传说都已湮没。
\par 太阳仍在我们之中,安静地投射出金色的光芒,像一切平凡的祈愿。
\par 也像我们如今的朝圣。在眨动双眼的每一瞬间,存在着黑色的诞生与消灭,
\par 存在着太阳的讯息,存在着默然前行的一切步伐与心意,是宇宙的反映,
\par 是诸神的反思。在言语塑成的国之外,在黑暗的界域之外,
\par 太阳,这位世界的宣告者与见证者,将永恒运行,透射时间与空间的光芒。
\par 是如今我们穿过漫漫人国,在土地中每一处神圣朝拜,同你一起宣告。
\par 是如今我们穿越黑暗,穿越形色与言语,永远别离,永远见证。
\par 太阳,在思念的终点见到,你是我瞳孔中一滴,你是我们之中一个。
\par 你即是我,用你金色的瞳眸映出我,命运与万物的变幻去来,
\par 以永恒的光芒照耀一切朝圣的道路,来到你的身边。
\\[0.6cm]
\par 太阳初升时带着无垠的贪恋与渴慕,以其永久的反映,世界的本来面目。
\par 是例举河原上诸神的全部名字,是风吹拂大地,水裹挟目光和声音。
\par 死亡匍匐在冰冷的地上,一旁是喧嚣的鸟群。我们眨动双眼,侧耳而听,
\par 空洞的火焰闪烁着光芒,一齐盘坐在山顶,陪伴着三位孤独的精灵。
\par 我同你们一起吟诵幻想的歌,有冰和尘埃在我们头顶降落,这尘埃如此之美,
\par 与三位朝圣者眼中见证的一切平安生灵,一同美丽,陨落,定义。
\par 在贫瘠而枯萎的平原,土地也哀叹的时候,我们同你一起走过,
\par 有天使与刻印,抚摸眼眸传道,走上追逐夜星的旅途。
\par 我是你,我在你们中间是真实语者,是虚伪语者,是太阳语月亮语者。
\par 是羽毛与空响的喉音,广袤的天空语者,是我见证虫语者,雷语者,眠语者。
\par 走过海,海语者。在高高的礁石上,于是有美在语者中间诞生,水,泡沫与盐。
\par 三位语者在土地上追逐哭泣,三位在追逐嬉笑,三位求诗,最后一个,
\par 成为人语者,沉默语者,否定语者。否定梦游,如此一个,
\par 在来临时如同离去,而闭上眼眸时,则黑夜显现,夜之美神与月之美神,
\par 散播金色与白色云朵的美神,迟慢的山神,高高舞蹈的水之美神。
\par 在这太阳下的舟船晕荡淹没,从发生到衰灭,好似一道轮列,一座塔。
\par 总共七种变幻,自美神哭泣中结晶,从低低的鸣响,到高高的奏响。
\par 在吾圆镜,执镜美神,这是一场终末之思念。
\par 一切朝圣,以心语者,是吾过去语者,是吾现在语者,是吾未来语者。
\par 一切成立,以土地语者,是虚妄语者,是自由语者,是宇宙语者。
\par 最后变成狮子,鹰隼,蛇,龙,一切仰望诗,自绝诗,躲藏诗,一切祈盼语。
\par 三位你和我,将要在土地上播种心灵,流牧太阳,一千枚太阳,火,聚变,辐射。
\par 最后要在平原上放牧野火,和鲜血。一千只野火又哭又笑,
\par 像是海面下纷乱的散光,一场无言的跪拜,祈望纯净,冰凉的山脉,
\par 热带的海洋上群聚而来的乌黑风暴。
\par 一切语言描定事物,都在太阳的光芒下获得希望,旋转。
\par 我们是这光芒的捕捞者,是一切言语的定义者,一切意义的创造者,
\par 故是你我朝圣语者,是河原语者,宝石语者。太阳就是一颗金色的宝石。
\\[0.6cm]
\par 在土地上运移寻找,端居在海潮一端,高高的楼房一隅,孤独的太阳栖居之所。
\par 全然是微末冰冷闪烁的蓝色光芒,好像在你们走在走去的指间与眼眉中诞生。
\par 无法满足怀有悲伤的洁净,也就无法走过时间之诗,我开始怀有枯萎之祈祷。
\par 太阳光芒之下的爱情如此缄默,和一切行将埃没的苦痛,柔弱轻抚的烟霞。
\par 从你诞生的一刻,我已遇见无尽眼眸,在它们飞翔的一瞬瞬,消逝。
\par 有老人,女人,婴儿在山坡上,像是众生的游行,同我一起吟诵情歌。
\par 就是此歌,伴随着水与火的诞生,满山的花同枯树前来拜见。
\par 一切都是在我开始流浪时完成,像是世界被填满了不实的希望与湿热。
\par 太阳以其绝对的美丽与真实而宣告世界之成立,
\par 就在我投入迷茫,在云霭笼罩下逐渐走丢。就在黑暗又从指缝中紧紧溢出,
\par 我已不能继续参加狂欢的行列,因吾是燃烧的一枚火焰。
\par 在时间与人间的凄寒之中,虚空与火焰和解。这是焚烧黑暗的火焰,
\par 送葬言语的行列,祈祷遗忘与地狱降临,有关绝望和嬉笑的梦想,
\par 如此宁静,淡漠,空白。例举氤氲不见的天空,一切游移思念。
\par 永恒踏水而来,粗糙的双脚搅起河底的宝石或者恒星的蓝紫倒影,
\par 金色的太阳吞吐着火焰,颤动,啸叫。灼烧,动摇,使这河流干枯,碎裂,
\par 连着浓浓的血晕,也一并枯萎,漂移,成就红色的霭云,绯红的城市夜空。
\par 这枚金色与七彩的火焰上居于蓝色中央,好似要撕裂,好似永恒沉静,
\par 好似追忆亘古,众生倾倒,颜色焕发于暗夜下的埃尘。聚合,炽热,
\par 在宇宙的闪光中冥想,又在漆黑的眼眸下自我反映,
\par 我即是你,七彩,花树,宝石,也都是你,是你的万千名字。
\par 我在河流的底端找见你,仿佛是你孕育出我。是你打捞我出,在浓郁的血色中,
\par 叫醒我,赐我羽毛和鳃孔,冰冷裹紧我,梦见碎裂的蝴蝶鳞翅翩翩飞舞。
\par 我想起了诗人的名字,我想起了土地上每一个生灵的名字,
\par 每一个都是一位神灵,都是一滴,千万滴,自河流的底端飞出,播撒,
\par 我将用你们的名字编写一部诗歌,给月娘披上睡衣,给青山和流水奏琴一曲,
\par 伴随你们一齐仰望,我将重新改写全部祷文,伴随凝视,迷惘,
\par 太阳以其绝对的美丽与真实而宣告一切言语之成立。
\\[0.6cm]
\par 当我闭上双眼,世界仿佛有着重叠的幻影,那震荡在我的体内鸣响。
\par 我听见黑暗滴淌的声音,又像是心脏的搏动在我的胸腔鼓响。每一瞬间七彩闪电,
\par 我都不能辨清,究竟是凡人的眼光,还是大海拍碎礁岸,鲛人哭泣还是女妖歌吟。
\par 开始倾听诗人的絮语,他爱之告解,他欲望之灭解。正如我倾听泡沫碎裂,
\par 羽毛在高楼大厦之间翻卷,反射。我开始不能清晰,世界是埃尘,
\par 埃尘是世界,或是太阳。你们在孤独时刻的吟诵为何全部变成了宣告,
\par 是不是言语就这样在身后隐去,以其永恒的譬喻以永恒的真实,
\par 火焰,月轮,众生之反映,三位狂怒的鬼妖,隐秘地笑的少女,
\par 我已不能明白这些意象,于是我睁开双眼。万物的幻象重叠如同世界面貌,
\par 尤其世界在哭泣,一部分在欢歌,一部分伏倒在冰冷的地板上。
\par 两颗太阳,镜中的太阳,两种颜色,两种瞬间七彩的闪电从我眼眸溢出。
\par 在我前往朝圣的瞬间,叩过门来到河流上的时候。又或是太阳唤醒我的时候,
\par 两种火焰,镜中飞散上扬的火星。我已成为执镜者,我亦成为每一个哭神笑神。
\par 此朝圣是窥视自我之朝圣,因是窥视宇宙与太阳之朝圣,
\par 此朝圣是思念之朝圣,因是七彩与反映之朝圣。
\par 这一切仿佛早已写定,即是太阳偈语,即是宇宙偈语。
\par 一切轮等,一切塔等,在闭上与睁开眼眸的瞬间经历无尽毁灭与重构。
\\[0.6cm]
\par 虚幻的形象膨胀,散发出火热的生命欲望。幻想存在的永恒膨胀,
\par 流淌,自我反映序列,如今有十个太阳围绕着我旋转。从思念的存储,
\par 到无限微小意识的精密计算,寒热莫辨的光,夜映的千万土地月轮。
\par 仿若巨大而蛮荒的野兽,狂人,一切祭祀仪式,是太阳一颗,一切火之情歌,
\par 是太阳一颗。一切美等,凋零枯落,幽想摄定,是太阳一颗。
\par 黑暗,每一流浪诸神之黑暗叠加,变作我的瞳,是太阳一颗。
\par 真实与虚幻同列,都在镜中,奇点之镜,在光子,原始火球,对称之中,
\par 十颗太阳,同诸神,一切闪耀事物同列,一切奥义事物同列,十颗眼眸,
\par 都来围观这一颗,中央的一颗,无穷,静止。向内部坍缩,自由,自噬。
\par 都来围观太阳之死,爆炸,喷发。
\par 在这转瞬的光芒中,连同一生的火热,引力,沉重,全部抛射,膨胀。
\par 在光线的一生中,历经的全部疆域,辐射,惊告,大诰。
\par 宣告火,与光,宣告语义燃烧之烬,土地塑成之埃,在静谧而黑暗的宇宙中,
\par 投入引力,投入紫外到红外的全部光芒,以及金属元素,放射元素,
\par 在膨胀的极点,物性与颜色的极点,真实与美丽如是沉重之证。
\par 就在这一瞬间,出离恍惚,像婴儿蜷曲在子宫之中,像天空中高耸乌黑的雷云。
\par 太阳在土地莽川,如出如里,暴雨与闪电,落击又回闪,像是一切真物在燃烧,
\par 一切伪物闪耀出七色光华。都在膨胀,在思念的极端开始飞翔,
\par 一切黑暗聚拢在我的瞳心,万物永恒变幻,直到光明褪色,太阳之死,
\par 有十颗太阳缓缓旋转,幻想我千万种面相,天堂与地狱,
\par 一切歌咏,一切祷告,金色的天空遍布着,不安的言语。
\par 于是众生仰望,全部自我就是此一瞬间,
\par 全部言语所谓一瞬间。
\\[0.6cm]
\par 端坐在太阳死去的位置,宇宙的中心。
\par 想象远离我,以至言语远离我,我还听见命运的雷声在黑色空间里久久的回荡。
\par 又像是睡梦,又像是梦醒。在这条河流的底端,高山之巅,黑色的冰。
\par 圣山之上,纯净而通明的大气中,一切幻影叠合,七彩与黑暗叠合,
\par 羚羊与牡鹿叠合,浮动的云彩与尘埃之下,人国之下,蠕动的岩石之下。
\par 我开始思念,从无尽的祭祀开始,从恒定久远开始,诗神与精灵来到了桥头。
\par 早在那时就已经写定,和如今拂照的一切,
\par 一切想念,漫漫的征途,星星的宝藏,月娘的宝藏。夜之语者,花之语者,
\par 吾命运是吾想念之一切,一切轮列皆因一念而起,一念而成就永恒。
\par 像是那滚滚的火流,蓝紫色出定的月亮,在平原上奔跑飞翔的精灵。
\par 我开始思念,与三位女神在荒原上守候火焰,诗人复活加冕为王。
\par 每一步我走过土地的角落,河流,浮定空中吞噬仰望的莫名巨兽,
\par 一位哭神,一位笑神,像极了饮醉的我,像极爱月的我。
\par 你这太阳为何散发出无穷的光明,以你自己的真实与美丽去映照虚幻的万物,
\par 你在思念的极端唤醒我,是想要赋予我什么么,是想要给我一个名字么。
\par 在这绝然的思念时刻,我只有无穷的哭诵,因我想起了我一切诗,
\par 我想起来我本然面目,仿佛我落下了金色的眼泪。
\par 在这永远燃烧着的温暖之下,我已然忘却了太阳的奥义,只是一凡人,
\par 只此一滴,有着轻微的思索,分不清是惆怅还是苦爱,啜泣,颤动。
\par 我已与你们一样,永远平凡的全部希望,温暖,绝对,安居在宇宙心央,
\par 有太阳的意念,有言语的意念。像抟起沙滩,变成一幅幅转瞬即逝的画面,
\par 对着星星或者一切不再孤单的生灵私语。这就是我们一起坐在太阳死去之地的故事。
\par 因我终将再踏上那归去的列车,我们要去世界的终末旅行,最后再见证凡人的思念。
\par 只有此处永久的想念,仿佛我落下了金色的眼泪。
\\[0.6cm]
\par 天空仿佛浮现出这样的预兆,在金瞳攀到地平线上,层层云映出七彩与金黄。
\par 土地上也浮现了这样的预兆,在北方的大气倾轧到南方,枯红,柔弱的火红,
\par 以及憔悴的黄与空洞的绿,全部映照在澄澈轻薄的蔚蓝下,
\par 鸟儿也在说出这样的预兆。像是来往的人群,以及微笑与神伤,
\par 光秃的枝桠,摇响的铃声。每一脚步声,电线,柏油路,栅栏,坍圮的围墙,
\par 都在纷说这样的预兆。
\\[0.6cm]
\par 这是有关平淡的爱情的,有关一点不实的祈望,有关想象,和热爱。
\par 有关耷拉着的爬山虎,其上如镜的玻璃反射着太阳的光芒。有关使用这种语言,
\par 反复地祷告,怀有年少的骄傲,老人的云淡。有关朴实的关于言语的信仰,
\par 有关诗人,以其落入地狱的必然命运。有关一位痴人,一位梦游人,一位语者。
\par 这样三个人,以及在那片河原上居住着的诸神。
\par 每一个字眼里都欣享着这样的预兆。
\\[0.6cm]
\par 它仿佛是美的,是欢愉的。同时也是哀伤的,夜的。
\par 它惶惶然盘踞在天空,土地,以及心灵的每一角,像是一颗迷失的太阳,
\par 燃烧着路过的每一明灯。
\par 它曾经点染万物,它曾经教诲每一个仰望的人,它沉默时天与地都在隆隆回响。
\par 我曾把它比喻成,血河在脚下流淌,金子从天空之中如羽飘落。
\par 我也把它比喻成,一支祭歌。一支情歌。
\par 它占据了每一诗行,是诸神编纂与歌吟的事物。
\par 它亦占据了晨起的落泪,夜眠的微笑。
\\[0.6cm]
\par 现在它来了,如同昼夜的轮替,金色在天空与大地铺展,
\par 黑夜重新回到黑色的瞳轮之中。
\par 它来了,像是一团火焰,又像是一位神灵,架着九条龙,在那高高的天上,
\par 那行一切祭祀的人注意听呵,一切仰望的族类,侧耳而听。
\par 一切朝圣者,朝圣语者,朝圣舞者,
\par 你们快去迎接,欢笑或者哭闹,你们要再去那平原上行祭祀。
\par 去呵,我的同伴,去寻找永恒的一切吧,
\par 赞美太阳,赞美众生之平安,赞美世界之诞生与渴望。
\section{天使篇}
\par 诗人来到那栈桥边。
\par 隐隐约约听到歌吟。
\\[0.6cm]
\par 天使在那栈桥上,歌:
\\[0.6cm]
\par 睁开眼醒来吧,失去道路的众生,
\par 睁开眼仰望吧,或者俯望这一条河,
\par 看看此地微妙的星光,是夜晚么,
\par 头顶迷蒙闪烁的,是星星飞舞的河,
\par 脚下翻卷旋回的,是金子,是宝石,
\par 是我们全部思念的河。
\par 天上地下,我已游历遍尽,
\par 凡人的情思,诸神的言语,我已倾听无尽。
\par 又回到这桥上,在我记得时间的流淌,
\par 目光追溯河的尽头,那时太阳,
\par 与月亮仍照耀着此地。
\par 我依稀记得诸神的交谈,
\par 可又暧昧,仿若梦幻,
\par 这是我开始的地方,一只桥,
\par 横跨在众生的命运与思念之上。
\par 转瞬,又有波澜起,
\par 是哀思,在土地上蔓延,
\par 流离失所,亲族散尽,爱情幻灭,
\par 死神出现,折射出紫色的幽光。
\par 又有金色,信仰与热爱,真挚,
\par 心灵相映。我真爱这颜色,
\par 是爱神,与美神,守望的三女神,
\par 执镜,在人国,映照出岁月,
\par 与歌吟。呵,缓缓流逝,
\par 谁也无法追溯,谁也无可往谏,
\par 就连微笑的神灵与哭泣的,
\par 也不可知道,洒洒江河,每一瞬间,
\par 折射的七彩,粼粼起伏。
\par 呵,只有无忧无虑的野兽,
\par 还有山间的凉风,聚拢的夜色,
\par 在永无止境的界域徘徊,
\par 游荡,只有天使,轻柔地抚动江波,
\par 带有美好的祈愿,凝望。
\par 这河自哪里发源,我好奇,
\par 又如何裹挟一切愿望,想象,
\par 那转瞬的涡旋,激荡的回波,
\par 又是发自宇宙的意愿,伟力,
\par 我向往,串起那川底的宝石,
\par 和星光,缀成项链,
\par 送给可爱的纯真的凡人。
\par 呵,我眺望去,星星哭泣,
\par 森林动摇,河岸的花儿树,
\par 露水淌落,那些采撷仰望与美的莽兽,
\par 纷纷哭泣,一支从寒冷的高原来,
\par 一支隐隐约约从沙漠中来,
\par 一支从银河,那母神的乳汁,
\par 还有一支,自凡人黑色的瞳孔,
\par 呵,多么美丽,牵引遐思,
\par 使我迷惘,因我全然不解,
\par 哪一个才是真正的发源,
\par 司掌命运的神呵,
\par 土地上流浪的神,还有你们,
\par 水的精灵,嬉笑着,起舞,欢歌,
\par 你们可否回答我的疑问,
\par 连着爱的过往与未来,
\par 凡间哀苦的思念,聚散成轮的命运,
\par 这一切究竟从何处来。
\par 是那高高的冰山,锁住了金光,
\par 岩石与冰高高垒起,
\par 还是沙漠中那片绿洲,倾倒的水壶,
\par 在无尽的塔之废墟中央,
\par 还是夜的怀中,滴淌的银光,
\par 美与爱的源泉,
\par 或是最深邃的凝望中,千千万万滴,
\par 透明的,游移的,
\par 仿佛生出一切,也像日月,
\par 映照一切。如此浪漫,如此捉摸,
\par 像我曾聆听云雀的高歌,
\par 西风的怒号,夜泉的欢唱,
\par 像是我兄弟姊妹,每一个,
\par 齐声吟唱,天空中飘下,
\par 金色的羽毛,欢笑的精灵。
\par 呵,此河真是伟大的圣迹,
\par 浩荡,淼淼,无穷无尽。
\par 我看到天空中的尘埃,
\par 天幕上映照的霞光,
\par 诸神居住在这片河原之上,
\par 开始想念我初生的时光,
\par 在许多奇异的界域游荡。
\par 这个栈桥,也沾染了诸神的气息,
\par 也曾聆听智慧的言语,
\par 渡过它,分割了你和我,
\par 过去与未来,记忆与灵感,
\par 美在桥头,爱在桥尾。
\par 河央,则是无尽的诗言,
\par 渡过它,就是河原。
\par 这美丽的河,七彩的河,
\par 美神在这里浣洗她的衣裳,
\par 爱神在此垂钓,哭神和笑神,
\par 相拥睡倒。滔滔而去,不知归处,
\par 哺育了河原上的诸神,精灵,
\par 润开了满树冬花,
\par 那取食七情六欲的兽群,
\par 和一位老牧神,一位老酒神,
\par 都饮你,思念的河。
\par 款款流深,变作火,
\par 飞扬高升,变作光,
\par 在我神国遨游的无尽时光,
\par 也曾采撷那迷离之光,
\par 亲吻那黯然之火,每一瞬间,
\par 都窥见诸神的痕迹,祝福,
\par 凡间竟是如此的幸福,
\par 每一寸土地都有诸神的脚步,
\par 这河原上,也不曾密布。
\par 只有歌声能够到达的角落,
\par 令我神往,又有泪流,
\par 一条曲折的支脉流淌过,
\par 思念将会塑成莫明的风景,
\par 那是天使也窥探不得的境域。
\par 呵,这国度上的故事怎么也歌不完,
\par 我岂能像凡人一样沉溺思念,
\par 只有亲手触碰,
\par 这不朽的栈桥,是我初生之地,
\par 天空中漂浮的舟船,是我享眠,
\par 那花园,我与女神们一同嬉游,
\par 河流上浮游的明灯,闪烁,
\par 我与姊妹一同祈福。
\par 这无尽的思念在我的歌声溢满,
\par 带着想往,这片名叫自由的国度。
\par 水的歌舞,火之哀苦,
\par 只是一切想往引着我离开,
\par 命运的三位女神对我笑而不言,
\par 因我听见的这预兆不属此世,
\par 我听闻太阳死去,
\par 诗人加冕,冰山颤动,海心言语,
\par 我窥见天空中的转轮,
\par 河流漫上岸来,火自山顶流淌,
\par 夜空中的光点如雨滴落,
\par 这景象如此可怖,惊惶,迷惘,
\par 仿佛凡人,我堕入大梦,
\par 爱与美,天国的和谐与安宁,
\par 也不能让我安定,
\par 勾引我思念的起始,
\par 是天使失格,即将堕落。
\par 呵,可是我有无尽的歌,
\par 让我暂时遗忘这可怖的预兆。
\par 这河流隐隐传来不安的咆哮,
\par 我已分辨不清时间如何流淌,
\par 它带着一些从未辨过的颜色,
\par 与天国永恒的安详如此迥异。
\par 仿佛我到了陌生的界域,
\par 转瞬乐哀凌乱,颜色腐坏,
\par 就在此处,这座栈桥之上,
\par 蜕尽羽翼,变成思念一丝,
\par 我的兄弟姐妹也要来此地歌唱,
\par 这条哺育了我们的河流,
\par 千万流转,每一瞬间都截然不同,
\par 映照天空,也映照了流浪之人。
\par 我开始想念河原上,
\par 游离的红色火焰,和牛马,
\par 金轮和牝鹿,
\par 可是天使不曾思念,只触碰,
\par 只要我轻轻拨动七彩的水流,
\par 我就会变成此间一滴,
\par 如此浪漫,使我心惶,
\par 呵,我如何不能思念来世今生,
\par 每一束歌声中我已找到归途,
\par 当我回想起那如迷的预言,
\par 天使的心就不再安定,
\par 让我的姊妹们也来歌,
\par 兄弟们也来和,就在这座栈桥上,
\par 爱之处也无来人,美之处也无去者,
\par 无论花朵想与天空争辩什么,
\par 那些流逝的道路都已经走向思念。
\par 这将是最后天使的挽歌,
\par 我愿献给河流,献给天空,
\par 赞美自由而永恒的诸神,
\par 我听到优美的幻想聚拢疏离,
\par 一些夜之美神的故事我不曾听闻,
\par 仿佛整片天空都在与大地纷说,
\par 一切流浪的预兆。
\par 天使将成为最后的歌者,
\par 引着河原上的百物来到河流之上,
\par 谁能永远叩响命运的门扉,
\par 朝拜土地上每一处圣迹呢。
\\[0.6cm]
\par 天使歌罢,羽翼飘散,
\par 金色的羽毛翻卷在大气之中,
\par 落满了整片河原。
\section{牧神篇}
\par 在接近黄昏的时候,天空的尽头传来暧昧的叹息。
\par 诗人来到那森林的边缘,幽暗的森林之中,依稀辨出野兽行走的道路。
\par 正当诗人预备前去时,他听到不远处野兽的嗥鸣。
\par 这声长鸣是如此凄绝,婉转,诗人一生都未曾听闻。
\\[0.6cm]
\par 野兽躲在那密林中,歌:
\\[0.6cm]
\par 呵,我已迷失,我已狂乱。
\par 追逐美丽的事物,那些自由的想象,
\par 我来到这片牧神的平原。
\par 可是久矣,吾不复摄食美与爱,与仰望,
\par 这空乏的饥饿使我膨胀,压抑。
\par 每夜我在空中游荡,
\par 我的羽翼只要掠过星光,
\par 一切虚妄就此涤荡,
\par 或者凝视夕阳,万物生长,枯萎,
\par 像我一样自由漂流在思念之外。
\par 可是凡人的宇宙中,
\par 只在腐朽,毁坏,仿佛地狱大开,
\par 那些神光与浪漫好似久矣不见,
\par 只有一份黑夜倾轧黑夜。
\par 何时我才能听闻夜之美神的歌声,
\par 在悄然之中安饱享眠。
\\[0.6cm]
\par 诗人高声应答,歌:
\\[0.6cm]
\par 可怜的野兽啊,我擅自聆听了你苦痛的歌声。
\par 黑暗早已来到,而最后黄昏无法使你满足,
\par 你若见过真正的美与梦想,请来与我诉说吧,
\par 因我带来了一些多余的残梦,一些失格的想像,
\par 这些是我多余的行李,其间也许有美,
\par 也许只是死去的火,愿这些新奇的事物,
\par 足以填饱你的肚囊,不似一路上彷徨的天使。
\\[0.6cm]
\par 那野兽的声音渐渐近了,带着低低的颤动,大地上恣意生长的灌木也避让开来。
\\[0.6cm]
\par 野兽鼓动膨胀的身躯,仿佛一头巨象,歌:
\\[0.6cm]
\par 可怜的诗人,来到这片荒芜的土地,
\par 在我长久的记忆之中也有你的身影。
\par 你的确无法喂饱我内里的空虚,
\par 在我吞食下太多凝望与仰望之后,
\par 终于时间与黑暗要把我吞噬。
\par 你是这样一个流浪人,
\par 带有一半吾族的血脉,
\par 吾等幼年时以颜色与形状为食,
\par 凡间的土地极少产出颜色,
\par 只有天空中七彩流淌。
\par 吾等中年以遐思与智慧为食,
\par 为此牧神在人间放牧信仰与梦,
\par 也随着时间消逝灭迹。
\par 吾等老年以自由与祈愿为食,
\par 可是走遍河原,也找不见,
\par 呵,这绝望的空乏焦灼着我的唇舌,
\par 挣扎着我的腑脏,
\par 驱使我四处游荡,以孤独充饥。
\par 哦,你这个诗人,
\par 当我窥视你的眼眸,我确乎看见些许闪光,
\par 仿佛是美,又似恐惧,
\par 这的确是新奇的事物,
\par 可是不经由你的口我无法领略,
\par 请你把你的见闻悄悄诉说吧。
\\[0.6cm]
\par 诗人低垂双目仿佛在思索,悄声,歌:
\\[0.6cm]
\par 以美为食的巨兽呵,我并没有什么见闻,
\par 我只有一些旧的事物,一些预兆,
\par 像是梦中闪现,又像是天使的歌声。
\par 你可否告诉我为何河水会漫上河滩,
\par 冰山传来奇妙的言语,
\par 为何诗人会复活而后加冕,
\par 海洋的深底也会颤动,
\par 为何星河的深处会洒落羽毛,
\par 火焰从山顶流淌而出。
\par 我仿佛看见诸神群聚在河滩,
\par 谈论,惊叫,
\par 为何太阳高悬在空中,
\par 旋即碎裂掉落地上。
\par 为何天使们纷纷歌唱,
\par 每一根羽毛都变作了汹涌的河流中一滴。
\par 这些景象使我不能安宁,
\par 驱使着我彷徨来到此处。
\par 昼与夜的转换就要到了,
\par 我要在森林中找到栖居之所。
\\[0.6cm]
\par 庞然巨兽长长地叹息了,歌:
\\[0.6cm]
\par 吾曾亲见,那是王与沙漠的对话,
\par 所有的故事都刻在了那座塔上。
\par 若你沿着河流的另一方向,
\par 走到绝望的沙漠深处,
\par 那座最终的祭祀之塔矗立之处,
\par 你就会明白,这早已应验。
\par 这些意象如此纷乱,
\par 几近狂想,穿越过往与未来,
\par 每一个预兆都映照了截然不同的路。
\par 有的变作传说,有的则仿若预言,
\par 我的心也因此而颤动了,
\par 呵,愿我唤起更多的回忆,
\par 追忆那些久已分离的好友,
\par 北海有冥,可以追溯思念之外的往事,
\par 沙漠记得王的故事,
\par 除此之外只有土地上行走的歌人,
\par 冰山上修行的舞人。
\par 呵,我想起那位神灵,
\par 他就居住在这片森林,
\par 不以智慧和美丽著称,
\par 也没有伟大的神力和高贵的地位,
\par 但四方流浪的生灵都归属他的管辖。
\par 这位年迈的牧神,
\par 我也长久未曾见他,
\par 但他熟识这河原上每一寸土地,
\par 每一束流放高歌的火焰,
\par 和一切不期而遇的雨水。
\par 如果他不能解答你的疑问,
\par 也必能安抚你不安的魂灵。
\\[0.6cm]
\par 这时森林深处传来了轻柔的笛声,正对着太阳归隐的地方。
\par 这笛声是如此的清澈,悠扬。仿佛催生出大地的生机,勾引起时间的流转。
\par 静静聆听的一人一兽仿佛都陷入了无穷的追忆之中。
\\[0.6cm]
\par 诗人黯然,歌:
\\[0.6cm]
\par 食美的巨兽呵,你可曾听闻过这样的笛音。
\par 仿佛带有远古的魔力,使我安定。
\par 我听得笛曲中的哀默,却不悲伤,
\par 而当它轻快,我却被幸福充满。
\par 我似乎想见了一位自由的精灵,
\par 年轻,以及他萌发的爱情。
\par 又好似一位优雅的恋人,
\par 在呼唤天涯的伴侣。
\par 只要听到这样的告白之音,
\par 我便想要同她交游,
\par 如此安心,又如此温暖,
\par 仿佛回到了梦中的故乡。
\\[0.6cm]
\par 那巨兽舒展了全身的鳞羽毛发,歌:
\\[0.6cm]
\par 幸运的诗人,这就是牧神的笛声。
\par 他在黄昏时分呼唤远方流散的牧群,
\par 天使们,精灵们,天空飘散的花瓣,
\par 还有吾族和其他悠游的野兽,
\par 还有你等,迷途的流浪人,
\par 都将被牧神的笛音召唤。
\par 啊,跟我来吧,跟着这牧笛,
\par 你与我一同去朝见牧神。
\\[0.6cm]
\par 霎时间,昼与夜的转换就完成了。月娘开始了她的巡游。
\par 月娘轻柔的光芒洒在了那片湖水之上,如此安静,如镜一般。月娘借着它,便可以梳妆打扮。
\par 湖水旁坐卧着一位少年模样的牧神。这时诗人与庞然巨兽走了近来。
\\[0.6cm]
\par 牧神安坐,歌:
\\[0.6cm]
\par 请不要打扰这静谧的时刻,我的孩子们。
\par 因月照耀在这片思念之池上是如此可贵,
\par 她无限的美丽将永恒感动每一颗流浪之心。
\par 可是请悄悄来吧,悄悄走近,
\par 让月华也能流照你们,
\par 我的孩子,以梦为食的小兽,
\par 还有一位流浪的落魄诗人。
\par 这片森林虽然不大,
\par 但都是一些善良自由的灵魂,
\par 你们是如何相遇又如何寻来。
\par 这倒无关紧要的,
\par 我早已看出了你们眼眸深处的渴望,
\par 你们紧张的心脏已经略有安定,
\par 那就请再多多聆听此刻的歌声吧,
\par 我邀请你们一齐来听月娘的歌声。
\\[0.6cm]
\par 就这样,无垠的静谧重回到这片牧神的湖边。
\par 因那风也屏息,森林也静默,每一位精灵都安心低眉,侧耳而听。
\\[0.6cm]
\par 月娘高悬在透明的夜空中,歌:
\\[0.6cm]
\par 今夜的河原似乎鼓动着不安,
\par 我看见土地上天使们离开了本座,
\par 纷纷去寻找流动的思念。
\par 我听闻了天空与土地与星河的谈话,
\par 这一切只要让它去实现,
\par 如今我们都将参与这般命运的轮列。
\par 河原之上,从东到西,
\par 我听闻预言悄然涌起,
\par 土地上的歌声与祷言都在谈论如斯,
\par 呵,这些对于诗人无秘密可言,
\par 那些不可言说的,都早已实现,
\par 每一束预言,也必将成立。
\\[0.6cm]
\par 于是天上地下,一切生灵,自由的神明,也都听闻了月娘的言语而沉默。
\\[0.6cm]
\par 牧神微笑,他抚摸着那野兽的鬃毛,歌:
\\[0.6cm]
\par 呵,这一切终将来到。
\par 我们只需等待其实现,
\par 好似久远之前,见证王之复活,
\par 吾河原上的生灵只是将其守望。
\par 好似我放牧梦想,呓语,
\par 来到那变幻横流的河流。
\par 梦取食思念,而美取食梦,
\par 精灵在河原上嬉游,
\par 歌人立于夕阳之下,
\par 我们只有去见证,等待,继续守候。
\par 无论是你,惊惶的野兽,
\par 还是你,落魄的流浪人,
\par 我要为你们指点迷津,
\par 放下一切不实的求索吧,
\par 你们都该归去所来之处。
\par 那里饥肠辘辘的野兽以花为食,
\par 无论是柔弱的紫色,还是妍丽的红色。
\par 那里诗人仍是四时之后觉者,
\par 在每一个感伤的瞬间还有欢愉的舒畅。
\par 不要相信不实的幻影,
\par 只有放眼去凝望,继续行路,
\par 你们都该踏上那条永远的归路。
\\[0.6cm]
\par 诗人同巨兽黯然,诗人前趋,歌:
\\[0.6cm]
\par 呵,伟大的牧神。听闻你的召唤,
\par 我,凡间最末的朝圣者,
\par 前来朝见你永恒的存在。
\par 我本是无情的天涯旅客,
\par 从极久远处带来一点微末的信念,
\par 在森林中听闻悲哀的嗥鸣,
\par 又听闻你呼唤的笛音,如此悠扬,
\par 若我复能咏言,我必将为此歌诗。
\par 可是我所求并非一条归路,
\par 我只有朝圣的前路。
\par 不管这片土地上的不安,惊恐,
\par 不管哪里无解的谜语,预告,
\par 我所求并非等待,
\par 而是实现。并非私愿,
\par 而是创造。
\par 你既是那放牧的神灵,
\par 请告诉我凡间的愿望都到何处,
\par 散逸迷失的美丽又在何处,
\par 死去的梦想都葬在何处。
\par 请告诉我牧神的故事吧。
\\[0.6cm]
\par 牧神摇头,歌:
\\[0.6cm]
\par 在你的故乡,人与人悲欢离合,
\par 我照看其中一部分,
\par 也不区分他们叫什么名字。
\par 只有一些迷途的人,
\par 自以为追寻到了世界的奥义,
\par 可是关于万物的命名法则,
\par 本来就不会区分你和我。
\par 你若想要求索最终的归宿,
\par 等待时间流逝殆尽,
\par 月娘在天宫中隐去,休憩,
\par 土地重归黑暗深底,
\par 瞧吧。那些久远的美,歌声,
\par 带着醉意的吟诵,破碎的梦幻,
\par 微末的光芒,闪烁着的祈愿,
\par 都悄悄苏醒,重现于世。
\par 呵,在这浓郁的黑暗之中,
\par 分不清真实与虚幻,
\par 就连诸神也将流放。
\par 我虽然在天空与土地上放牧夜,
\par 但夜也不曾听闻这些隐约的言语。
\par 我在河原上照料无家可归的梦,
\par 但以梦为食的兽群都不曾见过,
\par 那样不可言传的世界,
\par 沉淀在每一位神灵的想象深处。
\par 痴心的人呵,
\par 你若想探究这混沌的奥秘,
\par 只会被无垠的虚无压垮。
\par 没有谁能引领你,
\par 穿越整片神国。
\\[0.6cm]
\par 牧神歌罢,森林复归沉寂,远远传来精灵的齐唱,高高的树木也散发出同样的岑寂。
\\[0.6cm]
\par 精灵们,在远方,歌:
\\[0.6cm]
\par 土地上的哀声近于平息,
\par 尘封,如同温热的空气。
\par 天空黑色的眼眸枯涸,
\par 留下最后一瞥。
\par 平原消失殆尽,
\par 奔走的野兽都已披上羽翼。
\par 谁还在苦苦索寻自由的消息,
\par 快去水边加入精灵的合唱。
\par 那烟火点亮夜空,
\par 转瞬又逐渐消逝。
\par 闪电游移于黑暗的时空,
\par 宣告属于光明的讯息。
\par 万华复涌起,
\par 都在此凝望的瞬间。
\\[0.6cm]
\par 诗人仔细地聆听,仿佛露出微笑,歌:
\\[0.6cm]
\par 那些隐约的歌声正是我所求寻,
\par 我必将朝圣神国一切诸神,
\par 有如冬日腊梅花瓣上晶莹的露珠。
\par 你既是司掌放牧的神灵,
\par 一切游移的梦想都属你管辖。
\par 可否请你看看盘踞在我脑海中,
\par 这些奇妙的梦境,
\par 也许将指示我来处与去处,
\par 将显示我真实的面目。
\par 就让我自己的梦引我游历神国吧。
\\[0.6cm]
\par 牧神微笑,他摸了摸那野兽的翼羽,歌:
\\[0.6cm]
\par 你的梦不用我来辨别,
\par 这位老朋友会告诉我们有关梦的奥秘。
\\[0.6cm]
\par 庞然巨兽展开了翅膀,反映着月的冷光,激起了一阵阵波荡。
\par 这巨兽睁开了双眼,带着两点幽光,用翅膀裹挟了诗人。
\\[0.6cm]
\par 那野兽长啸一声,歌:
\\[0.6cm]
\par 呵。这是何等的梦,
\par 使我从太空坠落,坠向洋底,
\par 坠向火热的熔岩,仿佛地狱。
\par 使我来到太阳的边界,那气体的共振,
\par 喷发,有如无限的星火射向虚空。
\par 是了,我看到自己,
\par 我看到无尽我的同族自一双眼眸中诞生,
\par 还有天使,歌唱着诞生,
\par 我看到无尽诸神围着这一双眼眸,
\par 还有牧神,有三位女神。
\par 只要那眼眸扫掠之处,就有七彩诞生。
\par 我看到时间流转,那条河流,
\par 从一扇门中流淌而出,
\par 源源不断。我看到朝圣者,
\par 天使衔来凡人思念,丢在河里,
\par 变成宝石。我看到河边一片森林,
\par 优美的思恋在这里汇成一滴,
\par 就是此湖,牧神在吹奏一曲,
\par 我在其旁,因为无可取食而饥肠辘辘,
\par 我看到,诗人,走来,
\par 请求牧神,为他解梦。
\par 呵。这是何等的梦,
\par 那永久的颤动,仿佛轮转之音,
\par 震慑了我的心海,
\par 在我想窥视梦的发展时刻,
\par 无尽的黑暗吞没了我的心神,
\par 只听到空气被炸响,撕裂,
\par 如此我不复知归处。
\\[0.6cm]
\par 牧神安抚着巨兽,歌:
\\[0.6cm]
\par 可怜的小兽,我看到不比你更多。
\par 落魄的迷途之羊却背负着伟大的使命,
\par 我虽没有看到你最终的归途,
\par 但其中一些片段却打动吾心。
\par 你的梦啮噬着你的思念,
\par 而你的思念却不断鼓出新的梦境。
\par 凡间的梦只以回忆为食,
\par 但你却是抛弃记忆的人。
\par 是了,正因为凡间有情者,
\par 总是汲汲于微薄的追忆,
\par 而永受轮回之障。
\par 我看到你最终将走入这片土地上,
\par 任何一位神灵都未曾涉足之处。
\par 我看到你与冰山上的舞王攀谈,
\par 北海上的鲲鹏背负着你,
\par 整片星河将会倒转。
\par 呵,我并非预言的神灵,
\par 任何梦境都游走在虚妄与真实中央,
\par 哪怕是最熟练的牧人,
\par 也不能把它的去来一一辨清。
\par 只有一点我无比确信,
\par 这只孤独小兽将会伴随着你,
\par 也许会变成此梦的一部分,
\par 又或许将成为一把关键的钥匙,
\par 它将变作你的翅膀,
\par 你的犄角。并陪伴你穿过一切阻碍。
\\[0.6cm]
\par 诗人陷入了沉思,歌:
\\[0.6cm]
\par 我全然不知梦的奥义,
\par 但我终将前去。
\par 伟大的牧神呵,
\par 前去我将丢失形骸,
\par 我恐从此堕入无明深渊,
\par 这梦我起始不曾留意,
\par 如今沉赘如一毒瘤,
\par 使我心忧。
\par 愿你能在前路指引我,
\par 顾照我迷失的时刻。
\\[0.6cm]
\par 牧神笑了,歌:
\\[0.6cm]
\par 亲爱的小人,你的生命,
\par 与那浮沉的一切生命是一样的,
\par 天幕远端的极星,
\par 和高悬的月娘,都将拂照,
\par 以致你每一哀怨时刻。
\par 如果饥饿,就赶在太阳初升,
\par 去饮花露,食那蜜。
\par 如果困顿,就睡倒在羽毛之巢,
\par 思念行歌,欢喜蹈舞。
\par 我将驱使风和雨,
\par 在天下传达四时的消息。
\par 你便去吧,河原辽阔,
\par 有这小兽的陪伴,
\par 你可放心,
\par 他必将引你达到自由。
\\[0.6cm]
\par 于是牧神歌罢,天与地又重归平安之中。
\par 无论是流动的大气还是闪烁的星光,蛰伏的美与盘旋的梦想都分享着此刻的安静。
\par 而高高的夜空中传来了以美为食的巨兽,振羽长啸的声音。
\section{酒神篇}
\par 在天地两茫然的时候,诗人与酒神泛舟于江上。
\par 时间有了这样的晦暗,但近于迷幻的白光晕开在江上的雾气中,应该是初晨。
\par 四下只听得水声,江波抚动着一扁轻舟,桨橹击水,是一位年老壮硕的艄公。
\\[0.6cm]
\par 诗人横卧,歌:
\\[0.6cm]
\par 在吾生涯茫然断痛的中途,
\par 一切声响,颜色遁入黯然。
\par 此江上,清风也好,
\par 月影也好,吾如此熟识,
\par 百千年来荡漾依旧,
\par 东逝不还。
\par 只是吾何等艳羡,
\par 那天地间飘散的形色,
\par 四下散射着明光,
\par 又沉陷晦暗,飘摇无影,
\par 又勾引幽心,摄人魂魄。
\par 一切无形大化者,
\par 皆在江上运行,
\par 婉转如思念,
\par 晶莹如泪滴。
\par 一切都仿佛一水之诗,
\par 不制于形体,而得无限自由,
\par 于日光下升腾,
\par 于月华下凝露,
\par 而得无限上善。
\par 只是此艳羡无用,
\par 天地不求人之褒美,
\par 而赋人情怡,
\par 独我因此复郁悒哉。
\\[0.6cm]
\par 酒神用脚踢开散乱的酒杯和酒壶,亦大卧于诗人旁。
\par 此酒神俊美不辨性别,正是青春最好时,只一身羽衣,衣襟散开,迷蒙如梦醒。
\\[0.6cm]
\par 酒神带笑,歌:
\\[0.6cm]
\par 人不因天地美情而心旷神怡,
\par 却因自醉山水而忘忧。
\par 何苦汲汲于清醒的苦痛,
\par 而忘何以醉在空明中。
\\[0.6cm]
\par 诗人摇头,歌:
\\[0.6cm]
\par 明晰吾不要,
\par 癫狂吾不要,
\par 人生不过飘摇悬浮,
\par 梦也不要,醒也不要。
\par 我已说不清更有何求,
\par 只是每一渴求都如此沉重。
\par 此土地若能承载一切哀情,
\par 只因大地更其沉重,
\par 天空所承载一切仰望,
\par 却因天空比之更深邃。
\par 云何区区此人,
\par 天地间如此一蝼蚁,
\par 一飞雪,一流沙,
\par 论及沉重无,论及深邃无,
\par 何以行健,何以载物。
\par 反观其心,自以为鉴,
\par 珍重傲意,可比天煌。
\par 实其扭曲抽搐,
\par 自低于人,而欲成人之上者,
\par 于尘间挣扎,踽踽不绝。
\par 鄙陋人世,却贪图其间,
\par 此非病欤。
\\[0.6cm]
\par 酒神只做烦闷状,以酒壶捞江水吃。
\par 只此江水看似浑浊,在酒壶中摇晃三两下,就又清澈。到得嘴边,就泛出酒香。
\\[0.6cm]
\par 酒神饮酒,歌:
\\[0.6cm]
\par 无耻书生,自私潦倒。
\par 羡天地之高洁无暇,
\par 苦自我之卑鄙猥琐。
\par 此间百物,岂因汝之三言两语,
\par 分出高下贵贱。
\par 品物流行之理,又怎凭君私意,
\par 奔走东西南北。
\par 人之命理,我看不透,
\par 美丑,得失,我均看不透,
\par 优美的梦幻非我所有,
\par 吾亦不受魇魔之苦。
\par 你苦苦求索我虽不可明晓,
\par 但其自私浅薄,不可更切,
\par 不如同我饮酒,
\par 好忘掉胸臆大小不平事,
\par 你我再来辩天涯何处来去。
\\[0.6cm]
\par 诗人轻啜,歌:
\\[0.6cm]
\par 呵,愿我真有实切的思情,
\par 吾心焦急无奈,
\par 梦醒时形骸散落,
\par 不知归处,不知来处,
\par 就连这茫茫江雾上,
\par 此处又是何处。
\par 是了,我只有到天涯寻物去,
\par 可我要寻的,
\par 不是片刻虚伪的悲哀,
\par 不是昏醉,不是幻想。
\par 可是,此处究竟何处,
\par 只你同我,还有这老人家,
\par 你若是酒中神仙,
\par 那我是谁。
\\[0.6cm]
\par 酒神长笑,歌:
\\[0.6cm]
\par 好,好。可爱的人,
\par 你是人罢。孤独人,流浪人,
\par 自山野间来,
\par 梦中,画中,酒中,
\par 与神仙交游,泛舟江中。
\par 你不用到何方去了,只管跟我,
\par 来,再饮,大饮,痛饮。
\par 使你通通遗忘,
\par 无论是漂流,灯火,星光,
\par 一切无常事情都忘掉呵。
\par 再吟几句诗来,
\par 我不要做神也不要成仙,
\par 换你来罢,
\par 只管端起这酒杯,
\par 捞起这江水就是好饮,
\par 快乐忧愁全部抛掉。
\\[0.6cm]
\par 诗人端起破旧的酒杯,捞这江水,吟:
\\[0.6cm]
\par 我,我,我什么也不是。
\par 我只有什么也不想成为,才流浪到现在样子,
\par 全,全忘掉,可是该忘掉什么,我记得什么,
\par 好像掉到一团雾中,冷颤,好像山野明晃的钟声,却遥远,
\par 我,我没有东西南北,我不懂。
\par 我该懂什么,
\par 名字,
\par 我应该有一个名字,或者署名,或者一个身份,
\par 一个字语,可是我,在找,找什么,找酒,找酒。
\par 这困顿的苦涩,与酒,
\par 全不是真的,不是真,不是假,不过去,不将来,像月亮,
\par 疯疯癫癫,疯了,痴了,不是病痛,却是醉了,醉,醉了,
\par 变得迟钝,以及欢欣,逃避,冗杂,不可理喻,
\par 分裂,像谎梦,
\par 五彩斑斓奔马,插羽,吞吃银河,
\par 呼啸飞跳的钢铁,滚滚火烟从石头的塔顶,鼓动,涌出,
\par 我还要写,写,
\par 再唱,唱,可是我好疲倦,仰慕美,仰慕爱,
\par 好累,惊惧害怕,可疑奇怪,
\par 可是,这些,事情,叫做情感,
\par 原始火,原火,圆瞳孔,语言,
\par 你是我。不,我是你,我成为神,神,喝酒,乱喝酒,酒神。
\par 这,
\par 可是这酒,黑色,纯黑,墨汁,苦。
\par 这酒,浓黑,凝固,坚硬,冰冻。
\par 这可是酒。这是血,是一个投水诗人,一个捞月诗人,
\par 这什么也不是,就是水,水,
\par 这哪里是酒,就是水,大水,冷水,墨水。
\par 骗人,骗,骗子。
\par 是骗人,假的呀,假的酒神,假的神,
\par 捞起来,只有水。
\\[0.6cm]
\par 诗人再去捞河水,船儿摇晃起来,水花飞溅。
\par 沾湿了衣衫,是点点黑色与红色。
\par 可是无论怎么捞起来的河水,都只是水,不是酒,都是一样的黑色,沉重,苦涩。
\par 酒神急怒,再捞,再饮。是甘醇的美酒,晶莹,透明,弥散出阵阵酒香。
\\[0.6cm]
\par 酒神递给诗人,歌:
\\[0.6cm]
\par 好胆,醉得不轻。
\par 你再饮我这杯,
\par 是绝妙的好酒,
\par 凡人间何种酒我不曾品过,
\par 甘美,火辣,香醇,
\par 只要用这杯打水,就立刻变出,
\par 源源不断,似醉非醉。
\par 你不能这般狂饮,于酒不敬,
\par 怕是早已醉透,满口胡言。
\par 来饮此杯,
\par 梦醒酒,别离酒,
\par 好好看看你的邋遢样。
\\[0.6cm]
\par 诗人再饮,再饮,歌:
\\[0.6cm]
\par 不懂,不解,失忆。
\par 这,分明是水,
\par 是水罢了,何必再谎。
\par 在你手中是酒,
\par 递给我便是水,醉不了我,
\par 再怎么饮还是墨汁,
\par 全然浓郁的哀愁,全然飘摇,波荡,
\par 分不清时间与空间,听不见两岸猿声,
\par 梦也罢,酒也罢,
\par 在我的诗行里全部失真,
\par 沦落譬喻,水与酒,
\par 在此地只同一,醉同一醒,
\par 今日我大醉,深醉,可是放目此间,
\par 只我一人清明,孤独,分离。
\par 好了,最末的酒神,
\par 你只有告诉我此地究竟何处,
\par 此水,此杯,此酒,
\par 吾究竟谁人,
\par 究竟何处归去。
\\[0.6cm]
\par 酒神大笑,歌:
\\[0.6cm]
\par 你问此地是何处,
\par 分明就在水墨中。
\par 河岂非河,
\par 是酒河,墨河,血河,泪河。
\par 杯岂非杯,
\par 你只管拿去。
\par 在你手里只是水罢,
\par 他人那里就是琼浆,玉液。
\par 因你的魂灵癫狂至极,
\par 无酒可醉,再饮墨,亦是清晰,
\par 你即是吾梦中人,
\par 吾即是汝梦中诗。
\par 你我都归属于酒,于歌,
\par 一切譬喻都将成真,
\par 转而失色,淡退,质疑,混沌。
\par 你可大笑,可怒笑,
\par 只有带有隽永的笔法,
\par 永远带有无情的伟大。
\\[0.6cm]
\par 于是酒神歌罢,好似天地混沌一气都陷入了沉静。时间从此间抽离。
\par 反复听不到诗人与酒神的歌言,也听不到桨橹拍击水面,江风兀自运行。
\par 墨色全然凝固于纸上,只有那杯中,散发出阵阵酒香。
\section{女神篇}
\par 再往北方,诗人的老朋友来见他。
\par 雪在大气中安静地舞蹈。雪的降落是如此的缓慢。就好像漂浮,轻柔地翻卷。
\par 每一片都散发出点点光芒,像羽毛,像纸船,在天空中悠美地浮游。
\par 雪的精灵在白色的大地上,鼓动着风,触碰这些透明轻薄的冰。
\par 每一叶,都有荧光,闪动。有的团簇,但都是六角,有着眼眸的大小。
\par 北海的岸边,今日也是落雪。
\\[0.6cm]
\par 诗人想起了那首雪的童谣,歌:
\\[0.6cm]
\par 夜的女神,回到北方,
\par 回到雪的身边。
\par 夜,用你的罗裙,
\par 笼罩我轻薄的生命。
\par 星,月,用你们的光芒,
\par 唤起我的光芒。
\par 在这神秘的夜晚,
\par 重新披上白色的纱衣,
\par 一同去那雪的舞宴。
\par 在你我之中,
\par 分享冬的消息。
\par 明日就将南下,南下,
\par 带走爱情和凝望。
\par 这是最后,一次想念,
\par 落到平原,河流,
\par 落到那温暖的篝火旁,
\par 可爱的人儿,
\par 掌心里和发丝上。
\par 到了明日,记起我,
\par 再赴那场晚宴,
\par 雪儿与我一同起舞,
\par 没有谁会忘记我们的名字。
\\[0.6cm]
\par 于是轻盈的雪花在天空中勾勒出回旋的舞步,一齐歌唱起了一些旧日的歌谣。
\par 这些歌谣与一些情歌仿佛勾起了诗人的回忆,有关此地,北冥。
\par 在不远之处,白色的尽头,是海水,黑色深沉的海水,安静,沉吟。
\\[0.6cm]
\par 雪对着海,歌:
\\[0.6cm]
\par 久远之前我曾感叹,你的面目,
\par 另一片天空的垂影,躁动,
\par 与那无垠的神秘不同,你,
\par 就在我的体内,凝固,结晶。
\par 每一分,此刻的舞蹈,为了你,
\par 为了我的姊妹,接近你,
\par 被称为海的,雪的故乡,
\par 似乎也映射着苍穹的暗影,
\par 如此深邃,同那幽黑的瞳眸。
\par 你的目光吸引着我,使我恐怖,
\par 即便看尽天与地的一切,
\par 也不知道沉没于你,是撕裂,
\par 还是消融。是悄无声息,
\par 还是同你的内里,颤动,运移,
\par 在另一个世界奔流,呼啸。
\par 为了你,歌唱,舞蹈,
\par 狂放或轻柔地折射着晶光,
\par 只有一瞬的闪烁,失色。
\par 海呵,只是一滴,圆形的一滴,
\par 像是我们之中幼小的孩子,
\par 安睡的婴孩,梦想温柔的拂照,
\par 只有我永远地走向你,
\par 走向你的梦,变成你。
\\[0.6cm]
\par 海悄声地应答,歌:
\\[0.6cm]
\par 舞呵,永远地舞吧,
\par 为这夜空的神秘,为时光,
\par 在降落的时刻分散羽毛,
\par 滴落眼泪,沦落失形。
\par 只有此刻你永远地舞蹈,
\par 披挂宇宙的霞光,
\par 在这场走向死亡的欢宴。
\par 这至多不过一场梦,一曲歌,
\par 在夜之女神她温柔的目光下,
\par 我怎不会听到雪的相思。
\par 海底的宝藏不比那雨水,
\par 不比粉云的结晶,
\par 纷纷扬扬纯白的天空。
\par 在譬喻的顶点回旋的舞步,
\par 流连摄止顾盼玲珑,
\par 是梦中的目光,
\par 还是那凝望中的梦想。
\par 呵,变成我,
\par 变成海,不过是永睡,
\par 在幽暗之中颤动传递,
\par 鲲之歌声。
\par 悄悄睡吧,
\par 只有你是这命中一滴,
\par 而我是你的守望,
\par 正如那三位女神是我们的守望。
\\[0.6cm]
\par 仿佛应着海的呼唤,远远飘来了火的歌声,带有温暖的消息。
\\[0.6cm]
\par 那火在远方,歌:
\\[0.6cm]
\par 优美自由的希望呵,
\par 随风流浪,
\par 星光,
\par 流火,
\par 盈盈握住,
\par 双手,
\par 在北的极点白色的平原,
\par 白的喑歌,
\par 点点,
\par 冰的晶光,
\par 歌唱火焰的孤独的思念,
\par 女神,
\par 白色平原上火的母亲,
\par 在土地的中央,
\par 宇宙尽头,
\par 守望着孩子们,
\par 恋人们,
\par 朝圣者们,
\par 雪原上憔悴的花朵,
\par 日出,
\par 守望世界旋转,
\par 摇影,
\par 在时间的极点,
\par 等待,
\par 陪伴这火,
\par 第一滴火,
\par 等待新的守望者,
\par 温暖,
\par 见证新的见证。
\\[0.6cm]
\par 顺着那歌声的方向,在这条路的尽头,风与雪都静止的地方。
\par 依稀有火光,淡淡的红光,透过冰晶的折射,照亮夜的一隅。
\par 回望来时的路,冰的片段在土地上层层重叠,消去了全部的足迹,只有舞宴永不停歇。
\\[0.6cm]
\par 再往前路,似乎夜更深了。
\par 似乎夜的喃喃低语在此处也低沉了下去。
\par 因为这是思念汇聚之地,这里是命运轮转的心点,一切预兆发生之处。
\\[0.6cm]
\par 那漫天晶莹的雪片似乎濡湿了火,使得这火的歌声带着隐约与缥缈。
\par 诗人只有小心翼翼地前行,不踏碎任何一片雪的遗骸。
\par 然而些许的哀叹仍旧顺着微风传来,又四散。
\\[0.6cm]
\par 那哀歌已经辨别不清,只能听到微末的喃呢,和长久的迷惘。
\par 寻着火光,诗人来到三位女神的身边。
\\[0.6cm]
\par 女神之一,歌:
\\[0.6cm]
\par 南方而来的旅人呵,
\par 我们注视你已久。
\par 在这片远离河原的北国,
\par 永恒的时间以来只居住我们三个,
\par 陪伴着这抔祭火。
\par 你不必言语,
\par 请坐,以雪为席,
\par 好好享受这无尽极夜,
\par 同我们一起分享,
\par 天空中满布着无明的预兆。
\par 只有此处,
\par 眨眼一瞬是地上百年,
\par 我们姊妹从河上来,
\par 追逐初生的仔鱼,
\par 却在此共享无穷的孤寂。
\par 只因吾三人同你一般,
\par 在此休憩,
\par 土地上已经流传我们的传说。
\par 在那极北之星闪耀的地方,
\par 守望命运的三位女神,
\par 编织着最后的预兆。
\\[0.6cm]
\par 女神之二,歌:
\\[0.6cm]
\par 呵,可是本无命运与预兆,
\par 只有满天飞旋的冰晶,
\par 眨动,翩舞。
\par 自南方带来水汽,和忧悒,
\par 在此聚集,起舞,
\par 把夜之黑披在雪之白,
\par 你去听她们长歌,
\par 尽管传不到此处。
\par 时间已经近乎停歇,
\par 靠近那温暖的源头,
\par 这一抔可爱的火光。
\par 你瞧那不慎闯进来的雪儿,
\par 停留在空中,永悬之镜,
\par 但也有七彩折射而出,
\par 在这抔火前,
\par 一切都好似梦幻。
\\[0.6cm]
\par 女神之三,歌:
\\[0.6cm]
\par 你的来路与去路,
\par 也全都映在这漫天的雪舞中。
\par 这海,这火,你,我,
\par 万物的形姿全部抄在雪的结晶之中。
\par 那些无知的人以为是预言,
\par 以为是命运轮列,
\par 以为是奥义。
\par 我们只是轻悄地赏雪,
\par 赏夜。同夜一般,
\par 同每一雪一般,参与映射,
\par 一切都是此火,
\par 是那最初的祭火,
\par 是光,时空的支点。
\par 参与这诗歌的完成,
\par 变成每一条朝圣旅途中,
\par 指引方向的女神。
\par 就是此处,此刻,
\par 当极北的星梦见久远的离别时刻,
\par 彼生命之终点,
\par 来到这幻想生物的居所。
\par 可是你好似怀着疑惑,
\par 这星这火,难道不是你所想要,
\par 你还在等待什么降临,
\par 那骑着七头十角的巨龙,
\par 等待七位天使吹响号角。
\par 在这朝圣的终点,
\par 无论哪个方向都是回归,
\par 都是无尽的沉沦,
\par 回到那个时间和阳光雕琢的世界,
\par 在人群之中一隅之隔。
\par 品尝了金苹果的人呵,
\par 你的内心还怀有什么不明的祈愿,
\par 使你如此憔悴,
\par 不安,像那些被偷走秘密的人。
\\[0.6cm]
\par 诗人黯然,裹紧了斗篷。
\par 火光之外,风与雪无尽呼啸,原本轻柔的雪片变成了锋利的冰刃,似乎可以划破视线。
\par 而在火光之内,除了火焰安静地舞动,一切都浸染了白色,陷入了无尽沉寂。
\\[0.6cm]
\par 第二位女神弹起了她的琴,歌:
\\[0.6cm]
\par 若想知道凡人的祈愿,
\par 你当询问云雀,
\par 询问躲在暗中的夜莺。
\par 在蜡烛的火光照亮书页的一刻,
\par 时钟停顿在银河的方向,
\par 水仙低低垂到在宝石的湖畔。
\par 闭上眼时梦散成蝶,
\par 再乘上列车,驶向花园,
\par 大海光洁闪亮如同玻璃。
\par 这孩子定是渴望一件礼物,
\par 一个伴侣,导师,或者恋人,
\par 他在寻找那珍贵的宝藏。
\par 摘下晚霞七彩的一匹,
\par 河央沉甸的卵石金色,
\par 爱神的箭羽还是,
\par 美神的指轮。
\\[0.6cm]
\par 诗人黯然,歌:
\\[0.6cm]
\par 但愿我真有这样美好的愿望,
\par 可是我忘记的是名字,
\par 和理由。
\par 我难以想象的这一切,
\par 是真实还是譬喻,
\par 那溯回河上的朝圣,
\par 还是随风飘摇的流浪。
\par 诸神,这土地上一切神迹,
\par 美的创造,爱的发现,
\par 真是宇宙的神奥,
\par 还是一些重复诗行,
\par 只是人性的作戏。
\par 那些在苦厄中惊醒的念想,
\par 五彩迷离的预兆,
\par 人间纷乱的形影,
\par 呵,幽暗的惊惶如此迫近,
\par 是不是也是此间,
\par 一寒冷而苦望的言喻。
\par 在我脑海中有这样的争执,
\par 因我舍弃至今,
\par 只剩下相信与等同的质疑。
\par 呵,我渴望是破开这迷宫,
\par 一趟驶向银河的快车,
\par 请求那极星点亮我的心脏,
\par 把我迷惘炼成金子。
\par 可是我终究到了此处,
\par 此火,此星,
\par 北冥,三位女神,
\par 为何吾心依旧困顿,
\par 这朝圣是否也同等虚妄。
\\[0.6cm]
\par 那火焰颤动,似要言语,复归沉寂。连同夜好似更暗几分,散乱的雪雾吞噬了光芒。
\par 在这偌大的天地间,似乎只剩下了这一点,闪烁近于黯淡的光,火。
\par 三位女神与一位疲惫的流浪者静静地凝望着这一点火,火之一滴。
\\[0.6cm]
\par 突然有低低的颤动自大地深处来,带有空洞的回响,使得大地为之动摇。
\\[0.6cm]
\par 这声音低低吼着,歌:
\\[0.6cm]
\par 长夜听挽歌,
\par 雪舞万华镜。
\par 遍寻湮灭处,
\par 空留哀恨叹。
\par 前人消逝去,
\par 千古摘星语。
\par 想见真实境,
\par 北极饮冰客。
\par 一年留名者,
\par 唯有吟诗人。
\par 究竟入虚幻,
\par 梦醒就长歌。
\\[0.6cm]
\par 诗人大大地震惊了,在这响彻天地的声响之中。
\par 那些隐约的字句好似连着肉体,空洞的胃囊与脑体,一起震颤,发声。
\\[0.6cm]
\par 那第一位女神高声笑着,她是如此高兴,歌:
\\[0.6cm]
\par 哈,姐妹们,听呵,
\par 这是鲲之歌声,
\par 记载在那些古籍的角落,
\par 塔楼的壁画中,
\par 悠游在北冥的巨兽。
\par 就在我们脚下,
\par 驮着整片雪原,
\par 和这抔火,
\par 难不成是在天穹遨游。
\\[0.6cm]
\par 女神之二,也笑了,歌:
\\[0.6cm]
\par 呵,这正是鲲之音,
\par 叫我的弦琴一起长鸣,
\par 真是天地间的奇迹。
\par 原来谁人不曾睹目,
\par 哪怕居住北原的精灵。
\par 这古老的庞然大物,
\par 背负着时间的极点,
\par 究竟要何处去,
\par 为何又骤然歌。
\\[0.6cm]
\par 那第三位遮住眼眸的女神似乎倾听着,入迷。
\par 天地间隆隆地回响,像是雷音,又不可捉摸,像是私语。
\\[0.6cm]
\par 诗人茫然立起而四望,目力所及之处只有白色,白色的尽头是黑暗,而黑暗的尽头是海。
\par 可是霎时间,又好似野马在雪中奔腾,好似乘着旋风。
\par 好似白色的涌浪在咆哮,翻滚,拍击着虚幻的礁岩,山崖,击碎成万千泡影。
\par 转眼又好像战场,披着盔甲的武士,和燃着火焰的战车,翻飞的旌旗猎猎作响。
\par 忽而又变成了城市,工厂,变成了山脉,河流,一切的一切都在白色之中不停显现。
\\[0.6cm]
\par 那诗人似要悲泣,胸口急急起伏,歌:
\\[0.6cm]
\par 呵,这影子的狂舞,
\par 这幻影的魔宴。
\par 连这土地,这天之苍苍,
\par 都只是眼眸的投影,
\par 都是想象的梦魇,
\par 可为何我胸臆有这样郁悒,
\par 为何我眼眶中已涌出热泪。
\par 穿越万顷时光至此,
\par 为何我矗立在这片雪原,
\par 似曾相识,不可言说,
\par 为何我只有吟诗,歌诗,
\par 在土地的尽头无限怅惘。
\par 为何想见诸神,河川,
\par 为何思念变作羽毛,
\par 为何世界纷然旋转,
\par 变作一个又一个预言。
\par 为何天地间流落爱与美,
\par 无限的诗情埃尘滚滚。
\par 此情,此情为何物,
\par 是七彩,是白,是黑,
\par 夜空,朝圣,时间,
\par 万物的名字是何物。
\par 自我,为何有一个我,
\par 为何有眼眸,为何意识,
\par 为何而舞,为何而歌,
\par 呵,为何质疑,
\par 我已凌乱,恍如蓬草,
\par 浮动在江波上的红藻,
\par 终于连最后的立足之地,
\par 最后的前路,也要失掉,
\par 此间还有谁可以拯救我,
\par 你们三位女神能么,
\par 这漫天雪花谁能,
\par 谁能拯救一位诗人,
\par 早已堕入地狱深处的诗人。
\\[0.6cm]
\par 女神之三,似怒,歌:
\\[0.6cm]
\par 何苦垂怜自己的怨抑,
\par 佯狂到无明处发痴歌。
\par 汝所本望又非一回答,
\par 还在期待哪里有应答,
\par 陷入自我的悬疑之中。
\par 你如不是土地上的先知,
\par 不必行那预言的事,
\par 若非妄图窥探神灵的奥秘,
\par 岂会落得失却本座的下场。
\par 拯救你的并非谁人,
\par 而只有朝圣一路,
\par 我准许你询问我们三个问题,
\par 如能解惑,你可安心离去。
\par 等待在前方可至永恒的路呵,
\par 那是诸神也未曾涉足的领域,
\par 许能领你至想象的深底。
\\[0.6cm]
\par 女神之一,抚掌,歌:
\\[0.6cm]
\par 这好,妙极。
\par 好似我们三个真是智慧女神,
\par 真是命运女神,
\par 能道清楚轮转的奥秘。
\par 那河上的王才分别,
\par 又见这落魄的小囚徒,
\par 那河原上的居民真要叫我们,
\par 三位守望的女神,
\par 艺术之神,
\par 补完天穹的女神,
\par 司掌爱与希望的女神。
\par 这多可爱,叫我分不清,
\par 好似是文字上的游戏。
\\[0.6cm]
\par 诗人有歉,又微笑,歌:
\\[0.6cm]
\par 你们三位真是我的女神,
\par 在写就前一直如此。
\par 引领我上升,
\par 接引我到那方舟之上。
\par 在我短暂的生命之中,
\par 真正的困厄不多,
\par 有关宇宙的奥义不足阻遏,
\par 人情之间的微妙也难以打扰,
\par 唯独关于思索的理由,
\par 关于真实的存在,
\par 只能在荒野中盛开的花朵上,
\par 求来提示。
\par 呵,我想知道,
\par 诗人存在的理由,
\par 为何人们暗中吟诵着,
\par 令我不解的语行。
\par 我想知道世界存在的理由,
\par 在这个空荡的舞台,
\par 上演着不可名状的悲喜剧。
\par 最后,
\par 请告诉我诸神存在的理由,
\par 尽管我已有答案,
\par 但我却不敢明说,
\par 总要聆听他人的慧言。
\par 好心的女神们,
\par 请指点那迷惘的彼岸,
\par 那永无止境的思念之外。
\\[0.6cm]
\par 女神之二,抚琴几声,歌:
\\[0.6cm]
\par 雪花飞舞的理由,
\par 粉云南行的理由,
\par 闪电落击的理由。
\par 我相信这一切都是美的实现,
\par 美的观测者,缔造者与美本身,
\par 又或是一体,
\par 比喻即是美之活动,
\par 因此而存在反映,模仿,召唤。
\par 又或是诗,诗境,诗艺,
\par 即是琴,歌吟,发声,
\par 是思念的解放,
\par 追寻和谐的成立。
\par 诗人即是求诗,
\par 而诸神则是献诗,
\par 一切都参与世界的创造之中。
\par 此美,即是讯息,
\par 即是预兆与提示,
\par 同样,也将是诗行,
\par 即是为了消息的传递与轮回,
\par 都将在命名法则之中展开,
\par 只有此圆,圆环,
\par 是永远的存在之证。
\par 呵,这真是美的实现,
\par 最后的创造实验。
\\[0.6cm]
\par 女神之一,轻声笑了,歌:
\\[0.6cm]
\par 呵,这难道不是人性的发现,
\par 心灵之见证,爱之思念。
\par 神性本是人之拥戴,
\par 而所谓的世界亦是人所见证,
\par 诗是心之想往,心境,
\par 而想象则是人性之证。
\par 如此而言,比喻是爱的活动,
\par 超出自我的界限,
\par 变成万物的模样。
\par 诸神如是成为诗人的命名者,
\par 而诗人成为世界的命名者,
\par 诗人同时成为两者的承载。
\par 在这无限的同化之中,
\par 追求圆的心点,
\par 亦是反射的无穷。
\par 我们,连同整片河原,
\par 以致此诗的每一个词语,
\par 都将成为诗人自我的映射,
\par 那只有相信这是爱之献礼,
\par 凭借言语的创造,
\par 共享同一名字么。
\par 那名为生命的女儿,
\par 镜映的自我意识之下,
\par 成为诗的存在之证。
\\[0.6cm]
\par 女神之三,似有不解,歌:
\\[0.6cm]
\par 美,与爱,呵,
\par 我善良的姐姐们,
\par 你们真是可爱的神明,
\par 在你们的眼中,
\par 真有这样的奥秘,
\par 让世界纷纷然轮转。
\par 我没有你们的幸福,
\par 更没有一番洞见,
\par 我的世界建立在黑暗之上。
\par 只有孤独的缄默,
\par 如果有什么能成立,
\par 大概是一些可歌可泣的故事,
\par 古国的箴言,福音,诗篇,
\par 恋人的絮语,誓约,祈愿。
\par 那些存在与否,
\par 真实与否,
\par 都将变成飘飞的语絮,
\par 变成卷动的诗幡,
\par 如果真有诸神,
\par 也只是言语本身,
\par 自在,自为,不参与任何轮列。
\par 如果真有世界,
\par 也不过如雪,如雨,如海,
\par 只是鼓动,运移,闪烁。
\par 诗人,同你一样,
\par 只是流浪人,是言语的过客。
\par 一切都在琢磨不定,
\par 无中生有,
\par 一如诗的谱写,
\par 不需要理由而存在。
\par 如果真有理由,即是诗人,
\par 即是世界或神明使然,
\par 又或是茫茫的黑暗使然。
\par 可是我想你早已有回答,
\par 然而一切都等待着新的见证,
\par 永无止境的朝圣之旅。
\\[0.6cm]
\par 那诗人欲要言语,而火焰却一阵摇摆,歌:
\\[0.6cm]
\par 我的母亲们,
\par 美,
\par 与爱,
\par 不可追求,
\par 言语无可穷尽。
\par 名字,
\par 存在如火,
\par 日夜光明,
\par 终消逝。
\par 只有投火,
\par 求火,
\par 荒原,雷电,水,
\par 只有诗,
\par 同一,
\par 虚空分离,
\par 朝见神灵。
\\[0.6cm]
\par 霎时间,那无形无色的时间似乎被赋予了质量,变得浓郁,黏稠,自天空向大地滴淌。
\par 就在那火焰的歌唱中,时间与空间都变成了流质,开始消解,透明。
\par 每一片雪,每一束光,每一星点,都静止,但并非静止,而是自由。
\par 成为脱离世界的一孤独体,成为亿万光年的宇宙。
\\[0.6cm]
\par 转眼,天地间,只有火,一抔祭火,带有疲惫的金光。
\par 一诗人,三位女神,还有鲲鹏。
\par 在区分不清时空的界域中,在屏息,在无尽涌动着的言语与闪烁的目光中。
\\[0.6cm]
\par 诗人就在这无垠的金色中,歌:
\\[0.6cm]
\par 呵,在我一生的旅途中,
\par 从未见过如此的景象。
\par 连那梦世界一同远离,
\par 不可思议,却如此真实,
\par 因为此刻的自由与无穷。
\par 那圆舞终也到了极点,
\par 在轮转的中心,
\par 这便是朝圣的终点么。
\par 这一切的奥秘都藏在自由中,
\par 无论是通往何方的道路,
\par 都指向此处,此刻,
\par 生命之火,希望之火,智慧之火,
\par 指向永恒的金色瞳眸。
\par 这一幕好似久远存在,
\par 诗人的灵魂深处,深如刻印,
\par 好似美的体验,爱的连接,
\par 只有在此处成立。
\\[0.6cm]
\par 女神之一,惊讶,歌:
\\[0.6cm]
\par 呀,快看,
\par 如还能称之看。
\par 那幽游之鲲,
\par 化而为鹏,
\par 在这须弥,等同芥子,
\par 如何张开夜的羽翼,
\par 如何有乾坤长鸣。
\\[0.6cm]
\par 可是不待女神歌罢,这金色的海洋又有深深的震颤,好似隐秘的黑色如羽。
\par 可这是夜色,夜的碎片,在火焰的浪潮中间,神秘的夜色一片又一片,飞舞。
\par 那幻想的庞然巨兽,似要撑开它的双翼,于是大开它的两鳍,好似鱼跃。
\par 透明的琉璃鳞片蜕去,转而是漆黑的羽毛,深邃的夜之色,颤动着。
\\[0.6cm]
\par 那高昂的鸣声,在每个人的心中响起,歌:
\\[0.6cm]
\par 万古惊魂夜,
\par 时离忘情人。
\par 北冥闻幽息,
\par 同化语哀矜。
\par 想见天穹上,
\par 自由有爱恨。
\par 抟风如夜瞳,
\par 飞入幻想境。
\par 丢我仪式歌,
\par 最后堕落梦。
\par 唤醒诸神篇,
\par 飘零开新世。
\\[0.6cm]
\par 于是那鹏鸟张开了夜的羽翼,在茫茫的金色中,点点晕开了黑色。
\par 这黑色似有无尽广大,有隐秘的起伏与纹路,目光也不可追及其无尽的深邃。
\par 鲲,完成了这永恒的鱼跃,进而变成鹏,变成这夜,变成空,又无限扩张。
\par 从金色中渗出,裹挟了诗人,三位女神,以及全部脱离了时空的形物,雪,与精灵。
\\[0.6cm]
\par 万物都统摄在鹏之瞳眸中,仿佛安定。
\par 仿佛在茫茫混沌中,分离出轻柔的,微妙的,向上。而沉顿的,固绝的,向下。
\par 分出了天与地,黑与白,冷与热,动与静。
\par 在这北冥的岸边。
\\[0.6cm]
\par 海在动摇,歌:
\\[0.6cm]
\par 这一切的分离如同梦幻,
\par 好似我变成了天空的一部分,
\par 在金光闪耀之下,
\par 好似我变成千万雪舞,
\par 变成诗人的一部分,血,羽,
\par 好似染透了夜的颜色,
\par 在超越时空的界域流淌。
\par 好似变成千万彩镜,
\par 互相映照,又变成转轮,
\par 互相齿合,
\par 我听得是一些隽永的问答,
\par 奥秘与神灵的连接。
\par 可是这自我又在夜空中消融,
\par 变成夜的一部分,
\par 又或是海,
\par 又或是火,燃烧。
\\[0.6cm]
\par 女神之三,歌:
\\[0.6cm]
\par 在这无限的夜中我也看见,
\par 一只金色的鹏鸟。
\par 每一根羽毛上,
\par 每一细微的绒毛,
\par 都散发着无穷的光明。
\par 我也看见了土地,
\par 黑色的土地与七彩的波浪,
\par 传递在冰镜之中无限的思念,
\par 我看见了时间,
\par 蓝色的时间与触觉,
\par 火红玄黄的命运长河,
\par 将天地染成黄昏的颜色。
\par 就在天空的眼瞳中,
\par 鲲鹏的眼眸之中,
\par 窥见了属于此世的,
\par 真实的成立。
\\[0.6cm]
\par 于是天地间回响着空明的唳声,而土地仍被白雪叠盖。
\par 那雪的舞宴永不停歇,海也不停地冲刷着曲折的岸线。
\par 只是海水永远不再抹去那足迹,而是永远书写着,一笔一笔,永远书写着。
\\[0.6cm]
\par 在这极北之地,极星照耀之地,初始的祭火温暖着的方圆之地。
\par 安静地坐着的,三位年轻的女神与一位疲倦的旅人。
\par 如果说脚下的土地是鲲鹏所背负着,那么现在踩着的土地,又是谁背负呢。
\par 在这鲲化为鹏之夜。
\\[0.6cm]
\par 女神之一,歌:
\\[0.6cm]
\par 孤独的诗人啊,你的道路,
\par 同那自由去来的鲲鹏,
\par 以及永久奔流的长河,
\par 是同一的。终将远离的,
\par 不是你,而是我们。
\par 可是在那曙光唤醒大地之前,
\par 请你带走这火变作的镜子,
\par 无需映照自己,
\par 只需照彻天地。
\par 有了它,你就可以一次次进入,
\par 那无穷重叠着的世界,
\par 去寻找新的创造吧,
\par 去见证预言成立吧。
\par 而我,与两个妹妹,
\par 将会一直聆听你的故事,
\par 守望着你所珍视之物。
\\[0.6cm]
\par 于是诗人离去了。在三位女神的祝福之中,沐浴在熹微的光明之中。
\par 就在一瞬间,好似世界变得透明,还是一点一滴,在时间潮汐一丝一毫的涨落中。
\par 诗人变成一片羽毛,那是鸟儿的羽毛,可爱的轻柔的,与晨光一样的颜色。
\par 在这无穷无尽的飘零之中,隐隐传来了三位女神的合唱。
\section{预言家篇}
\par 在那隐居的预言家的塔楼里,不为人知的秘密实验正在进行。
\\[0.6cm]
\par 年迈的预言家,缓缓地把碧绿的液体滴入盛着灰色絮雾的烧瓶之中。
\par 每一滴,在滴入时都会显现七彩的光芒,而光芒殆尽,则又是一模一样,缓缓旋转的稠雾,只是从灰暗中,渐渐显现出闪烁不定的金属光芒,偶尔又一阵摇晃,激射出电光。
\par 偌大的烧瓶架在喷射着火焰的油灯上,只是这油灯显得破败焦黑,却散发着不可思议的光明。
\par 那预言家一边小心翼翼地滴加着晶莹的绿滴,一边飞快地扫视着周围的仪器。
\par 随着那烧瓶中的灰雾缓缓地旋转,杂乱的桌上,飞转的金属球,坩埚中跳动的火热的液体,互相反射着的几组镜子,在腐坏生锈的底座上,意外地擦拭得光亮,流转着同烧瓶中一样的色彩。
\par 可是就在预言家全神贯注地按照奇特的频率一滴又一滴,从那极为细长的滴管中挤出球状的晶液,再看着它缓缓地旋转着,在空中画出螺旋的轨迹,飞入灰雾之中。那刺耳的声音带着不耐烦地尖嚣响起。
\\[0.6cm]
\par 那乌鸦大摇大摆地踩在那桌上,铺满又硬又厚的纸张,上面密密麻麻画着奇妙的符号,一张纸便是这样的图画,似乎可以辨清。
\par 乌鸦尖笑着,歌:
\\[0.6cm]
\par 亲爱的预言家,失败,
\par 爆炸,死亡,我已经预见了,
\par 就在早上,你醒来的一瞬,
\par 从你的眼眸中溢出了。
\par 亲爱的预言家,贫穷,
\par 谎骗,火刑,炼金术士,
\par 比不过你,画匠,操纵黑白,
\par 甘拜下风。从你嘴角流露,
\par 变成箴言,是杀人如麻,
\par 偷盗智慧,窥探隐秘。
\par 亲爱的预言家,你是,
\par 命运的刽子手,恐惧的散布者,
\par 镌刻墓志铭的殡葬人,
\par 你是你所未知。
\par 亲爱的预言家,放弃,
\par 停下,你愚昧的挣扎,
\par 这邪恶的实验如要继续,
\par 撒旦也会甘拜下风。
\par 我纯黑冰冷的眼眸已经多次,
\par 见证你的失败,毁灭,
\par 自从你教会我言语,
\par 我替宇宙间的真理浪费口舌,
\par 却永远无法填满你的内心。
\par 亲爱的预言家,空洞,
\par 卑微,无助,为了圆满,
\par 你虚伪的信仰,从大到小,
\par 都要服从计算,和镜子观测,
\par 为了攫取,帮你圆谎,
\par 看着你陷入更深的迷惘,
\par 真是大快我心。
\\[0.6cm]
\par 预言家目不转睛地盯住烧瓶,一边默默计算着时间,一边稳稳地滴着似乎永远无法滴完的绿色溶液。
\\[0.6cm]
\par 预言家缓缓地运动着腐朽的口唇,发出嘶哑的声音,歌:
\\[0.6cm]
\par 世界的秘密在此合成,
\par 因此献上祭品,是一生的命运。
\par 那聚合的元素,离散的时空,
\par 受到虚空的统率,
\par 燃烧,折射,吸引,共鸣,
\par 我将要创制出永恒的真理。
\par 可怜的笨鸟,你踩着的是,
\par 造物的把戏,万物的历法,
\par 那统合一切的公式,真理的终极,
\par 力的傀儡师最后的秘密,
\par 你竟把这解析的计算叫做诓骗,
\par 岂不是自我的嘲讽,愚者的自欺。
\par 而你那乌鸦的心智更无能理解,
\par 这实验的伟大。除了饶舌,
\par 诋毁我超越永恒的工作,
\par 你倒应该学个仆从的样子,
\par 去那边架子上衔来精粹的原液,
\par 替我加到那金灯之中。
\par 火焰呵,请饱饮这丰饶的脂液,
\par 然后背负起这无垠的混沌。
\par 借来,行驶创造的权利,
\par 今夜,我将重新给世界定名,
\par 不朽的神秘呵,
\par 请撩开你的面纱,
\par 让我再次窥见你的面容。
\\[0.6cm]
\par 霎时间,随着老预言家的歌声,那浑浊的灰雾突然猛地散发出明亮的光芒,开始高速旋转,又不断收缩,蠕动,发出令人不安的爆炸声响。
\par 预言家似乎已经预料到这一切,与他的计算完全吻合。只见他不慌不忙地拿起一杆巨大的钳子,一把钳起一旁冷却着的坩埚,把那晶莹透明蕴着奇异光点的液体倒进简直要摇晃起来的烧瓶之中。
\par 那年迈的身躯,却有着异常的力量。预言家的两手不带一丝颤抖,尽管只能看到萎缩的肌肉。
\\[0.6cm]
\par 看着那逐渐稳定的液团,预言家一边指使着乌鸦,歌:
\\[0.6cm]
\par 快,衔来,那象征智慧的宝石,
\par 象征美丽的眼球,我的魔杖,
\par 把那卷符文铺在地上,
\par 去唤醒那些魔影,搬弄是非,
\par 窗帘拉紧,不能让谁窥见,
\par 去,把那些木头,血液,金子,
\par 丢到圈里,随我高唱。
\par 今夜要行那叩门的仪式,
\par 四方的智慧都已就位,
\par 那涌动的魔力与象征都已齐全,
\par 今夜我将徒手建立的是天国,
\par 在人间这片地狱之上,
\par 我将代替神明行使造物权利。
\par 从一变作桥梁,
\par 二是雷霆闪电,
\par 把三化作利剑,
\par 四要丢作轮盘,
\par 连上五或者六,
\par 聆听这奥义歌,
\par 七全变成财富,
\par 八是名誉高贵,
\par 在九个静夜中,
\par 生命也将沸腾,
\par 十个太阳闪亮,
\par 全变作圆环零。
\par 这就是那开启门的钥匙,
\par 无论藏在门后,你是什么模样,
\par 被我精炼而出的真理呵,
\par 快来见见智慧的主人,
\par 我乃超越永恒的预言家。
\\[0.6cm]
\par 随着预言家带着凄厉的声音吟诵着那可怖的口诀时,铺在那符文上诸多元素都仿佛受到了牵引,开始轻轻的晃动,悬浮,好似陷入了一个透明的漩涡之中。
\par 只见预言家小心翼翼地捧起了那盛着晶莹的胶状物质的烧瓶,一点一点,随着魔杖,把这蠕动着的发出七彩光芒的凝胶倾倒在混合着魔物与炼金物质的漩涡之中。
\par 一瞬间,有巨大的火焰顿时爆燃开来,金色的,绿色的,甚至还有墨蓝色的火焰不停地从那漩涡的中心喷涌出来,发出极为明亮的光芒,吞没了整个空间。
\\[0.6cm]
\par 不知何处好似传来几声怪叫,而那预言家只是坚定地注视着那火焰的中心,好似他的目光看穿了全部七彩光芒。
\par 就在这一片金色的光芒之中,一些影子开始组成。
\par 一开始只是球状,硕大的球体,有着模糊的表面,仿佛是一颗眼球。就这样盯着那白发苍苍骨瘦如柴的预言家。
\par 而不知经过几多时间,那球体开始分裂,从中央凹陷出裂缝,接着又一道裂沟,而分裂的球体又开始继续分裂,直到稳定成八八六十四,像是密布着眼珠的某种怪物,等待张开血盆大口。
\par 可是还不待看清,那球体又开始分裂,重组,染出不同的颜色,开始蠕动,变形。
\par 起初就像蠕虫,带有褶皱,裂口,蜷曲的肉。而后突然分裂,生出触手,刺,角,牙。就这样又开始收缩,长出甲,鳞,可是又迅速蜕下,变成鳃,裂,孔。还不待长齐,又开始剧烈的蠕动,长出浑圆的头,如同蛇一般的身躯。
\par 接着那头又开始分裂,膨胀,而蛇躯又裂出四肢,带着浓郁的血色,从胸腹中,变成心脏,鼓动。
\par 随着这肉体的蠕动,那头颅上又长出角来,似是鹿角。裂出耳,披上浑身的鳞甲,像是要伸出指爪来。
\par 可是四肢尚未伸出,两颗幽黑的眼珠瞬间显现在那头颅上,甚至开始扫视着这混沌的一切。
\par 只见那靠近尾的肢膨胀增大,靠近首的肢又分裂,长出爪,和角质鳞。可是一瞬间又蜕变,变成羽毛,布满全身。
\par 可是不待这羽毛膨大,又仿佛脱力般,纷纷落下,只剩下一对翅翼。
\par 而一对新的前肢,和分裂的灵巧手指从胸腔上重新探出。
\par 于此同时,那头颅上的骨骼也经历了新的分裂,变成一对犄角,带有神秘的螺纹。再看那新生的羽翼,早已变成了漆黑的肉翼。
\par 还不等这肉翼全然展开,那头颅上漆黑的眼眸一转,幼小的犄角又转瞬剥落,连着全身不知是鳞还是羽,亦或是盔甲,齐齐剥落。只剩下裸露的皮肤,带有轻薄的血色。
\par 不用说那老预言家看到此处早已惊得不能言语,但他还是死死地盯着这可怖的造物,直视着那双无限深邃的眼眸。
\par 这血肉的生物活动了手脚,似怒,似嚎叫。不知从何处抽来一杆斧头,似要劈开不可名状的事物。
\par 然而不待举起,这奇异的造物又忽的披上了浑身的长毛,似有疑惑。不知又从何处拾起一片龟壳。
\par 不待看清,这怪物又一阵变幻,直到披上了金衣,手握长剑,双目如电,似威仪,似睥睨。
\par 可是就在这一切似要停滞,那喷涌的幻光即将枯竭之时,最后涌动着的七彩却仿佛被莫名的一双手所搅动,骤然爆发出异样的金红光芒。
\par 这好似高冠长剑的英侠形影在这拨动的力量下,又一阵换变,而最后在迷蒙的光雾中,只依稀立着一个裹着长袍的人。
\par 这人有着一双黑瞳,黑色的长发,裹在漆黑而破败的布袍中,若有所思。
\\[0.6cm]
\par 那乌鸦上下翻飞,嘎嘎尖叫,歌:
\\[0.6cm]
\par 这真是,巧夺天工,造化神奇,
\par 那世界真理的制造实验竟是,
\par 这样恶心恐怖的造人。
\par 按照你自己的模样,从头到脚,
\par 为何你对玩偶傀儡情有独钟,
\par 我算是彻底知晓。
\par 你以为的隐秘不过是一团肉,
\par 那神性的奥秘是空皮囊,
\par 亲爱的预言家,真是伟大,
\par 你已经窥探,卵生,胎生,
\par 一切都是分裂,又分裂,
\par 造人,就是你的最终把戏么。
\par 好呀,你这可怜的造物,
\par 好似比我高贵,有着人类的面容,
\par 柔软的形体,双目中一片幽暗,
\par 也许我唯一可比就是单薄的历经,
\par 还有这喋喋不休的鸦舌。
\par 只是你同我一样,
\par 浑身上下没有属于自己的部件,
\par 我倒想分你几根羽毛,
\par 聊慰你不能飞翔的天性。
\par 可我想你这小子,
\par 会比我更受青睐,
\par 你是那宠儿而我受诅咒,
\par 好在两者都并非我们所有。
\\[0.6cm]
\par 那人环顾着这奇妙的小室,微弱的烛光照亮着几张堆满了不明所以的仪器,书籍,纸张的古旧桌子。还有同样放满瓶罐以及奇异的标本,数不清的书籍,图画等等的书架。
\par 奇异的是尽管所有的东西都显现出腐朽不堪的样子,墙壁与地板都是纤尘不染。除了脚下踩着的一张画着圆圈的厚纸,几乎碎成齑粉。
\\[0.6cm]
\par 裹在黑袍中的人,歌:
\\[0.6cm]
\par 在那无垠的黑暗之中,
\par 没有时空,没有思维,
\par 一切是如此沉寂,在想象之外。
\par 为何我还保有自我的意识,
\par 也许整片虚空就是同一的自我,
\par 可是就在这漫无边际之中,
\par 仿佛有一双手,
\par 写在我的存在之理中,
\par 这双手在黑暗中打开了那七彩的门。
\par 而这一切发生在我意识之前,
\par 无论是手,还是黑暗,七彩,
\par 门,来自于无物,来自空无,
\par 我只轻轻地推开,
\par 好似经历了永恒的变幻,
\par 一个可怖的梦魇,
\par 让我分不清哪个是我,
\par 哪个是真实。
\\[0.6cm]
\par 那老预言家坐在房间的一角,蜷缩在整个房间里唯一的椅子上。这是一把崭新的椅子,有着绮丽的装饰。
\\[0.6cm]
\par 预言家挥舞着他的魔杖,像是驱赶着什么影子,歌:
\\[0.6cm]
\par 有关那黑暗的事情我们听得太多了,
\par 在这个世界里,人们叫它死亡。
\par 可是看吧,是我,
\par 我是你的创造者,你的主人,
\par 赋予你形体与灵魂。
\par 然而我却不解为何是人的形状,
\par 难道只是个盒子,还是个信使,
\par 或是因为瞬间的观测,
\par 使那存在被扭曲,叠加。
\par 生命,呵人类,难道还是失败,
\par 人于我毫无秘密可言。
\par 我加入的元素,哪个参数,
\par 哪怕是室温的波动我都算好,
\par 这聚合的符文,不会错,
\par 一毫一厘都落在误差限定。
\par 呵,我真是不解,
\par 一切的计算都符合收敛,
\par 合成的产物却奇形怪状。
\\[0.6cm]
\par 预言家摇头晃脑地嘟囔着,一边又回忆整个实验流程,试图找出任何微小的纰漏。
\par 而那乌鸦,飞停在书架的最顶端,端端地用一只眼睛盯着那同样一身黑色的人类,时而猛地摇头,用另外一只眼紧紧盯着。
\\[0.6cm]
\par 乌鸦,歌:
\\[0.6cm]
\par 妙哉,看起来还真像个人,
\par 若非我如此精通进化论,
\par 倒真会给你骗到。
\par 除了血肉,还有大脑,
\par 不像是残废,短寿,多病,
\par 嘿,叽里咕噜的老家伙,
\par 哦不,我亲爱的预言家呀,
\par 我看你的实验也不算白费,
\par 将来还可混个江湖术士的名号。
\par 你要怎么处理这可怜的玩偶,
\par 即便从事苦劳也不堪大用,
\par 在孤独时刻也无可作伴,
\par 看这呆头呆脑也不能助你研究,
\par 就算拿来解剖也对医学毫无贡献,
\par 呵,我已看穿你无聊的命运,
\par 不消通过那浑圆的晶球,
\par 无家可归,天涯流浪,
\par 像所有你的同类一般,
\par 蜷缩着骨架死去。
\\[0.6cm]
\par 那人,怒,歌:
\\[0.6cm]
\par 呵,我本就是一流浪人,
\par 你不必费心揣测我的命运。
\par 穿越整片河原,莫非又重回人间,
\par 你们两个丑角,一个狂傲无理,
\par 满嘴不知所谓的名词,
\par 一个讥讽狭隘,真配得上乌鸦的身体。
\par 可是我告诉你们吧,
\par 我不是任何人的造物,
\par 自由,是我的天性,朝圣,
\par 我的使命。但不知为何,
\par 在那虚空之中,好似受到奇异的牵引,
\par 七彩的召唤,来到这片空间。
\par 谁告诉我此地究竟何处,
\par 你们又是谁人,
\par 那奇异的旅途把我引至此处,
\par 定有无言的奥妙。
\\[0.6cm]
\par 那高架上的乌鸦似乎有惊奇,嘎嘎直叫起来,而预言家干涩的声音打断了它。
\\[0.6cm]
\par 预言家,歌:
\\[0.6cm]
\par 呵,来自虚空的流浪人,
\par 请听老朽为你慢慢道来。
\par 我是这个世界之中一位平凡的预言家,
\par 那聒噪不停的黑鸦,是我的助手,
\par 造自我手,又抓来鬼魂驱动着肉体,
\par 就像那精密的时钟,履行着应有的义务。
\par 亲爱的旅人,你可知何为预言,
\par 那凡人的预言不过是现实与梦幻混淆,
\par 顺势的巫术,对诸神造物的粗劣模仿。
\par 而我却发明了驱使傀儡的计算方法,
\par 无论是星象的流转,
\par 大地的运移,或是凡人内心的隐秘,
\par 无不统摄于这平衡的公式之内。
\par 从最为微小的物质元素,光线,
\par 到庞大如太阳,银河,
\par 进行那跨越时空的计算,
\par 就可以知道掌控世界的诸力使向,
\par 这才是准确而无懈可击的预言。
\par 然而我并不满足于此,
\par 世界的秘密就好似一扇厚重的大门,
\par 而我的力量不足以撼动它,
\par 尽管可以窥见一二,凿开罅隙。
\par 因此我设计了最为伟大的实验,
\par 精炼数不清的材料,
\par 反复计算合成的条件,
\par 控制你无可想象的环境因素,
\par 驱使那躲在世界背后的力量为我所用。
\par 这正是制造真理的实验,
\par 可是在这之前已经失败多次,
\par 呵,无非是能量不足平衡,
\par 打破了原有的对称性。
\par 对于这样的制造,也挺容易,
\par 只消编写构成的规则,
\par 以那诸神的言语铸成,
\par 你尽管看看这书房,那架子上,
\par 那卷绸布我从死神的衣裳上撕来,
\par 那些海藻一般,是模拟恒星的爆炸,
\par 还有这个魔杖,求春之神替我培植。
\par 就好比你脚下,那是我亲自编纂,
\par 耗费了数年的苦心,
\par 那些堆在桌上,都是同样的符码,
\par 只用这些就可以改变运算的法则,
\par 而你,便是这真理的承载。
\\[0.6cm]
\par 歌罢,预言家好似疲倦地沉寂下来,小小的斗室里,只有烛光闪烁。
\par 那乌鸦摇晃着黑黑的头脑,不知又在思索着什么。
\par 一切都好似停歇了,连同那些运转着的仪器,等待着谁来打扰这短暂的休憩。
\\[0.6cm]
\par 那人,略有迟疑,歌:
\\[0.6cm]
\par 你自称预言家,倒也有可取之处,
\par 与你不同,我并非真理的信徒,
\par 不曾窥探这世界的秘密,
\par 有关神明的奥秘非人可及。
\par 你若是真正的预言家,
\par 何不看看我复杂难明的命运,
\par 这不羁的野马,泛滥的河流,
\par 从何处来又往何处去。
\\[0.6cm]
\par 预言家仿佛领悟到了什么,摆弄着仪器,歌:
\\[0.6cm]
\par 呵,我明白了。尽管只是假说,
\par 你权当无聊的闲话来听。
\par 我们所在的这个世界,
\par 这书斋,塔楼,那小镇,绿洲,
\par 好似都笼罩在静夜之下,
\par 实际上都记在一本书中。
\par 这本书有着许多奇怪的篇章,
\par 用着古怪,近乎随意的语言撰写,
\par 现在也许正在一笔一划,
\par 也许只是久远之前早就写定。
\par 与一般角色不同,
\par 我明白这书的机关把戏,
\par 无论如何阅读,揣摩,
\par 你也无从理解此书的写作,
\par 看似其中藏了有趣的论调,
\par 深思之后又变得平凡而无聊。
\par 预言家,只不过是阅读者,
\par 是这本莫名之书的解读者。
\par 所以这一切,神迹也好,智慧也好,
\par 都是此书赋予,不如说是书中言语,
\par 只是如此命名,如此写就。
\par 而你不同,你是我亲手写就,
\par 尽管我全然不能理解,
\par 从外在到内里,
\par 都并非我可以窥探。
\par 然而简单的测量与实验仍是例行,
\par 死乌鸦,不要吵闹,
\par 还没到你开口的时候。
\\[0.6cm]
\par 老预言家驱赶着飞到桌上的乌鸦,一边清理出可观的空间,把那七组圆镜互相反射着的仪器摆到中间。
\par 只见这些镜子围绕着不同的中心或慢或快地旋转着,又射出五颜六色的光彩,一时间令人眼花缭乱。
\\[0.6cm]
\par 乌鸦,盘旋在狭小的室内,好似躲着预言家,嘎嘎直叫,歌:
\\[0.6cm]
\par 嘎嘎,我偏乱叫,叫你烦死。
\par 那边黑衣人,你要当心,
\par 这些镜子是老家伙的计算装置,
\par 在我看来不过是骗人的幻术,
\par 总是用看似精妙的幻象,
\par 不仅自欺还要诓骗凡人,
\par 说一些似是而非的理论图像,
\par 用巧合与直觉来解释。
\par 嘎嘎,你别被,嘎,呀,呀。
\\[0.6cm]
\par 可怜的乌鸦说道一半便被预言家噤了声,这真是方便的设计。
\\[0.6cm]
\par 老预言家,歌:
\\[0.6cm]
\par 好了好了,你只消顺着这个窗口去看,
\par 那无限世界的无限时间,
\par 就会在你眼中展开,
\par 可千万不能迷失,
\par 时而回到这里,回到之前,
\par 在你的记忆之中我留下的后门。
\par 无论那里藏着什么奥秘,
\par 都不要流连,只消前进,
\par 我就静静看着你的瞳眸,便可知晓一切。
\par 呵,你既是自封的流浪人,
\par 那便去真正的无穷中寻觅吧,
\par 辨别真实,只有靠自己的双眼双手,
\par 到头来你会相信预言家的话语。
\section{语境}
\par 我看到是,天空中有雨落下。天空中有阴沉云落下。我看到是,绝望的哀情从天空落下。我听闻惊雷,目睹闪光,我看到无穷语丝从天空落下。其中每一丝每一毫,都晶莹剔透,是宝石,经过爱情的点染。于是我听闻,从廊下走过,在深深的雨之海洋中,风同雨的呼声。金色的雨,金色的透明点滴,我看到,群象在这雨之平原踟蹰而过,寻觅着一条路径,森林,雨的降临森林,在这片无限的回廊宫阁。我看到雨中群马在雨中的河流中奔腾,挣扎,我看到云中冰变作闪亮的鱼群,在冰变作的海洋岛礁之间穿梭,嬉游。就在这从天空垂落的瞬间,自由的柔软颜色在雨的交响中流浪,封闭了城市,封闭了街道与灯光。就在我在这无尽时空中一瞬间的凝望片段之中,整座城市的阴晦变成了天空的补丁,灰色的一匹,从东翻卷,到西降落,压塌了仰望的时间,仰望的女神与受难的神子。天空的废墟与泥土的废墟,冲刷,繁复,变得濡湿,变成氤氲的晦气,木霉的气息,腐朽的喘息,在混沌的明晃之中,承载目光和希望,产出呆滞,愚騃,忧悒,怨怼,富有甜味的断想与冲动,就在这无尽的荒原之中。冰冷的手指,冰冷的面颊紧紧贴着在夕阳下流浪的铁,铁塑成的战车,铁马与铁铸的指挥官,面对空无一人的战场,盛着无限反射的七彩之池,野花的泪滴,野草的诗章,一簇接着一簇拥挤在这来回无物的盆地里,回响从山谷里传来牛羊的长歌,野生的爱情之嗥鸣,只是无人想象,无人等待,走马观花,进至于癫狂,和温暖而微小的喜欢。所以我站在城池边角的土制废墟之上,远离大海或是任何人的远望,只是在自我隔绝的极点渴望拥抱,燃烧着火焰的球体,如此广大,喷涌着磁场与电流,沉没在泥土覆盖的岩壳下方,运转,流淌,跨越亿万里以血肉之躯丈量扭曲的时空,从它内部的鼓胀听见一些高远的歌声,正在伴随着幽暗的宇宙一同轮回,散华,飞入非想镜,就是那片金碧辉煌的宫殿,在群星光芒闪耀之处,映照着全部光谱的全息信息。我的月宫,是那充满凡人回忆之地,静静等待光明在大地上扫掠与生命自然的时间吻合,从朝阳到晚霞,我都透过微尘的宇宙空间描绘光行走的路程,以及每一点来自分子吸收与辐射。有时想要爱情,我会躲在书本的角落,是泪滴藏匿在整片海波起伏不定的哭泣中,我会跟随一些依稀的倩影到达新的孤独之境,一开始便极力写景,那火烧云映照的海岸,金红的伏波,远处霓虹的灯岛,这是我对于那时间地点最后的想象,湮没到幽暗的楼影,暴雨的罗幕,一些刻在石上提示着世界的重启,死亡也是其中一个,残喘着攀附的我的影子,扯到破碎,覆盖在大地上,石壁上,聆听最后的孤独之歌。我会在重新想念的时候遇见美神,或者在高山的寒草,白雪覆盖的沙漠,青色碧绿的湖水畔,象征灵性与智慧的红衣与珠串,叮当作响,了无声息,就在我窥见美的形影,徘徊在砌石拱门之下,那些又高又粗又坍圮的石柱的影子分割而出的光明节奏之下,我拿着火,火的炬,火明亮幻想又奔腾,激发着空气与血流,迷失,在它与我的对望之下,动摇我的影子与心神。就这样去见兔子先生,我的月亮母亲,在密林之中,跟随火萤的脚步,一起与找见那位老教皇,一位发疯的诗人,一位魔术师,狮子与鹰与蛇与天鹅的原生剧场,朝拜木刻的神明,躲在他们之中,不敢言语,受摄于那惊魂的魇梦,观看少女的表演,歌唱与思念的傀儡戏剧,以剑搏斗。我从未见过如此血液在山川间涌流,大地疼痛,痉挛,在所有的恐怖之中,我看到金属堆积而成的森林,海洋,无情地耕犁着残损的泥土岩石,研磨,搅动,只有一曲安静的舞曲,在空无一物的冬天,一段温柔的唱词,在狰狞熔断的铁与扭曲失色的火中间,使一切失去意义保有意义之物陷入绝然的寂静,是我们的女儿,在这绝望的呼吸之中牵动着最后的丝线,摇摆。只要我听见,最后的提示,在萧瑟铺满全部土地的时刻里,我仍听见一些轻微的躲藏声音,无害绵软的云团翻过消弭的山巅,张开了天空的面目。泡沫从池塘的深底鼓出,飞出水面,变成了天使,就在那藕花叠着莲叶间,是我无穷只眼眸所确认的,小小的天使们就在这油绿与嫣红中玩耍演奏音乐,赋予我想象。我看到有天使来触摸我的瞳孔,羽毛模糊了泪水,视界失焦,进而淡退,失力,我回想起那落满羽毛的天空,每一根羽毛都带有如此的柔软,散发出白色的光晕不似鸟类所能拥有。它们就落在树梢,落在那片荒芜的大地,是光的碎屑,影的回忆,母亲的爱,轻柔地飞起了树,连根,连着奔走的人,互相倾轧的楼房,空洞的譬喻之内,都被掩盖或带到天国,都变成一扇门,几乎如此普通,只有你轻轻推开,默默窥探。这其间有如此之多难以想象,你亲眼见到那条河流,在最为寒冷的冬日,从上流漂流而下的冰,它对你言语好似溺水,它满嘴说着预兆,那宝石铸成的国,人们书写在透明的玻板,那里举行着绝望之中带着戏谑的诗歌大会,有一些美丽的少女带着属于她们的宠物,蝙蝠,狐狸与豹子,都在那门前,等待着那背着大刀的侠客开口,他是如此缄默,破碎的长发遮住了从他眼眸中射出的电光。如此难以置信,甚至愈发诡谲,就在欲望中进行,放大一个词语,放大全部语言,首先是通过冰折射出七彩,同七彩攀谈,精粹七彩变成情感,或哭或笑,惹人非议,感受肌肤的温度,就这样伏倒在温暖的河岸。那里驱赶着羊群的神明曾睡卧,用他悠扬的短笛呼唤爱情,爱人的名字,此刻是如此诗意,但又隐藏,躲在每一个凝思的瞬间,那是不是另一位神明的名字,还是一种花朵的名字,同时寄托了生命,馥郁芬芳,承受泪珠的能力。我有这样的想念,甚至想重新开始,重新编纂,撕毁誓约,深中病患,跌落泥淖,既然命运把我活埋在这凄凉的墓穴中,我只有重新开始布施言语的宝藏,向着同伍的一些骷髅,冤魂,互相发誓预备从此堕落入更深的地狱之中。但我仅是一个,那漫无边际的苇草中独一一个,天地之间没有回音,无论是前生还是来世都将继续这无限朝圣,一场没有观众的悲喜剧,一出彻底邪恶无情的闹剧,可就是此一个,能够自嘲,重新回到山上女巫的身旁,人们在这里供奉狐狸,并祈求无关紧要的平安与幸福。我就在他们之中,平静着看着一切发生,孤独或者欢喜苏醒的时刻,恍惚之哭,在无限的生命重逢时刻,因果与逻辑无从支配的界域之中,面目全非,在言语之境以简单的姿态互相共鸣,采取令人心碎的方式,超越了自残的欲望,就在此刻一瞬间,月亮的影子遮住了太阳的光芒,日食,眩晕,在轮回之中用无限的形色,穿梭于信仰之外的灵魂,向着更高的生命形式,黑夜以无限的温柔与包容吞噬下城市的绯红光晕,投影在混沌的粉云之中,看到了冗杂的电光闪射倾轧,疼痛之刻骨,永无穷之共感,使明晰,使知晓,在恐惧与愉悦至极,如同黏稠的流质,从眼眶,耳廓,口鼻指缝间无可抑制地溢出,并且使同化,溶解,腐蚀,是灵感的自我复制理论,如此脆弱,岌岌可危,逐渐充盈了地平线上每一个角落。这个时候你只有立在那海崖之上,轻悄的歌唱,在海风吹拂的地方,都将口耳相传的,你的诗歌,在高高的大气中,自由地传递着,干燥而冰冷,等待湿暖的阻遏。于是你将替那些人哭泣,而你的哭泣也将变成歌声,你的一呼一吸都将变成音乐,伴随你的肉体,都在交响,都在合唱。我仍会前去探望,不顾风暴与雪风,在汪洋之上,殉情之旅,直到土地上每一个以语言相交的神灵,会在睡梦之中朦胧记得这些飘碎的故事,无论走向何方都将成为永恒。因为这个身份已经深刻写入我的肉,我的魂,以血书,以泪书,衡量质量,颠倒梦想,深郁臆想,隔着生死的鸿沟漫漫的呼唤,荒原章,执镜章,太阳的哭泣实现,欢爱实现,梦遗实现,这不过是微薄的苦痛,困抑,不比你我都见到的万一。因此这就是你的祈愿么,这便是我曾经教会你的言语么,在雾中,滚烫的雾气是七彩的龙穿行在灰暗的楼群,是幻想游行,金色与白色的鸽群,环绕着,急速地飞翔,鸣唱。我们就在这浑浊的漩涡中心,无数的目光在此处交欢,受切割,受血,受水,刀光剑影,有限次穿越世界的壁障,人性的监牢之垒,是一个巨大的有穷法轮,轻盈地旋转,就这样在轮回中心,旋转的轴心,是追逐游戏,就在这诸神的花园中,我们与那些可爱的精灵玩耍,嬉戏,那些用美好的晶莹幻想织成的羽衣,用羽毛编成你臻首的装饰,再回想那些笑容,是无忧无虑的墙角柔弱的蔷薇。我就在这小小的房屋前,一些朋友在后院交谈,而我坐在门廊的躺椅上,享受着无限制的存在感,我会思念我名叫夜的妻子,想起我也是一片夜,同样有着黑色的瞳孔,黑色的长发,同样深邃而浓密,而天空不过是我们曾经游玩过的回忆之承载,正如它承载着我旧日的凝望与深思。可是这在消缺时间与空间的此处失去了言语意义,变成一些不再连续的观测,自我,自我性,自我格,一些片段失语,变成默片,变成纯黑,如果还能下降,会有星点出现,一滴,变成夜的波痕,清脆的晕响。只要停止变化,就会陷入孤独,舞步,或者眨动的星点,只要回到寂静中,就会忘记直面的勇气,成为空洞的一员,背离友情,月光或者篝火,在空荡的沙滩上,背离夜色,只看到自己的脚印,与大海的嬉戏。一切孤独的声响在河流上飘动,冲刷着巨大的岩石,碎裂,聚合,回旋,是孤独之舞,凝固在观想的瞬间,又脱离于沉思之外。我就在这里垂钓,在逝者围坐的圈外,思维之池,静观时间的消融,意义进行同一,而笼罩在我意识之上的隐秘的忧切,不停地提示着,小小的刺痛着,无论成为何种开启,作为钥匙失格,那开启通往宝藏之门的咒语,哪怕一丁点的仪式都不会记录在此。那些我费劲辛苦找来,苦思冥想得来,都已然消失殆尽,而飘零的梦境,龃龉的呓语却依旧纠缠我的心魂,人间不过几声惊叫,又有淡漠的血污,但这些流逝的哀情却以言语的方式躲藏起来,你不会知道这过分的自我描写的目的,而这绝望者的歌声也无从给予抚慰。我想我必须离开,以一个候鸟的姿态,不再追逐智慧的幻影,不去思考颜色,干净的天空的渐变颜色,轻薄易碎的蓝色瓷器,或者轻盈的绸布,迷惑的面纱,黄昏的绯紫,几乎无从承载眺望的温弱黄色,和其中无限排列的渐变,书本告诉我这是电子与光子的相互作用。我只有从这里,窗台出发,张开我的羽翼,在朝阳尚未染透天空的瞬间起飞,穿越青,穿越青紫,穿越紫,穿越黑,同太阳与月亮对视,一个变成左眼,一个变成右眼,就在这被孤独与命运浸染的瞬间,我要想起我自己,我赋予自己的名字,使命,这条道路,几尽湮没在混沌中,我想起我的力量,一小片土地,数不尽的回忆,在异样的世界中超脱轮回。在这昏暗的时刻,大地上覆盖着雪,雪的精灵,送来了新的幻想,动摇,转瞬间落满,染成白色的天与地,自此刻哀倦的观想中,仅是雪与我,对坐在温暖的桌旁,近乎睡着,那浮响的,如同浪潮,呼唤着我,教会我,去爱,去欢笑,一点一滴的光明,把我从黑暗与癫狂之中拯救而出,转瞬,又变成微笑的雪儿,哼着催眠曲,转瞬,又只我一个,对着那把剑,散着冷光,转瞬,我已飞驰在铁道上,沿着绵延的海岸线,此刻我在言语中见证,想要建筑起现实。就在这些或真实或虚妄的语境之中,必然藏着那绝对的语境,世界之果,历史的尽头,那就是门的实体么,这一切文字的造物来路,去路,是不是都会在那里写定,就在那写作开始的地方,斟酌着字语,随着那轻柔的歌声的旋律,从简单而美好的事物开始,女儿醉心的微笑,和哀靥泪珠之雫,追寻神迹,点亮万物之火,滋润心魂之水,我想这本是一首好诗,一曲期待献上的情歌,呵,那终日高悬的求索,与爱人分离的断念,无穷反复不明就里的奥义,这一切都令我厌倦,这着魔般的倾吐,恍如隔世,生命与生命之间的壁垒,可是我不想永久困死在这迷宫之中,我只有呼唤,寻找那几乎埋葬在无尽意识中的回忆。我是,谁,我的名字,雨或者雪,我试图想起,鸟儿,天使,傀儡,从远到进,依稀是我所遇见,见闻,但又是我之部分,它们的回忆,所构成之原理,互相疏离又重叠,发现,映照,传递着光的讯息,生长,抽芽,自由,燃烧,照亮静夜,照亮只有你我看得到的界域,而我想要把你拥抱,因你随我到此,穿越万顷河原,我们真是两位神灵,是正与反,阴与阳,我渴望写你的诗,就在山谷之中的休憩时刻,我将浣洗我的长发与衣衫,流放这些诗篇,就在时光的消息互相辩驳时刻。现在我将听闻我自己,是否是在那长长的旅行之中,寻找到那失落的魂魄,在这终将被大雪掩埋的城市里。就在那绝望之巅,在山坡之上居住的痴人,在太阳之下行舟的摆渡人,那些书册的看管人,在冰川的环抱之中,在不停地自我探问中继往迷失,我想这是无极之境,太极,散列,开张,又聚合,同一,幽幽是悬挂在枯枝上的弯月,是我还没有见过的面容。就在你们之中,有一些过往的我,在你的门前,每一个人都执火,冷淡的火,自地下涌出,带有血色,不甘,愤恨,那些惨遭抛弃的梦想,恐惧的幻影,地狱之门。这些火是早在时光中丢失名字,流离失所的火之废墟,失却了赎罪的使命,而流浪,新的自由宗教,我就在这些火炎的环抱之中出生,在它们之中,我是一个王,我渴望是呼啸,爆发,可是在这些死亡的譬喻中间,一切都如此荒诞,以火编织成高楼大厦,编织纸醉金迷,我就躲在这些形容枯槁的滚烫的人偶之中,默默履行我先知的义务,在空荡的地狱之中,魔幻僭越现实,无聊的怒火侵蚀膨胀的欲火,都在血色的洗礼之中,熔化,不堪一击,非梦,而是芒刺在背,无论是炼药的魔女,鼓吹的鬼仆,蛊惑的魔影,都参与这绝大的戏谑狂欢盛宴,因那死亡早被偷盗,束之高阁,只有鸠毒,扭断的视界,永远孤独的颤抖,以正义霸凌欢乐,以崇高鄙夷幻想,呵,我就在你们之中走过,看着自相残杀的血肉从天空之中如雨落下,无限的饥渴烧灼着钢筋砖土,预备把地狱烧穿一个巨大的空洞。我开始意识到我变得几乎与人同样,难道我正是一个人,在这无垠的混沌之中,是不是要给我这样的清明,那些手指,向着内部索取,被称为人性的疆界,恍惚如迷的爱,绝望的爱,扭曲禁断的爱,或者,希望的爱,祈愿的爱,温暖湿润的爱,把我拉回想象界域,用尽言辞,爱之天堂,爱之地狱,在泪水组成的河川之中淘洗宝石一般,我将继续寻找,那些已死之爱,阻抑之爱,编纂恋歌,因为我内心萦绕着不安的闪电,而我所寻求之物,被我亲手打碎,即便,再书万字也无法挽回,回到时光之外,重新写定自我,世界进行。从轻盈走到沉重,从沉重走向断裂,我究竟是否飞行的族类,还是只有蜡制的翅翼。可我只是,土地上最后一个同类的寻求者,是谁把孤独写进了我的生命设计,写进了我的基因,遗传,我仍将永远成为自我的朝圣者,搜寻有关神明的歌呗,编纂,如同抹泪,自从我三次死亡开始,所剩无几,只有在此处,永无止境地重生,可是我不能知晓,无从停歇,这些言语,将会成为拯救还是诅咒,已无退路,将军,在同自我勇气的斗争中。我想召唤,隐秘之池,水仙之池,向着真幻的拐点,来到那大雪深央,这独一的合唱,绝对自我而献礼,天下飘纵,如诗,辨理,对着山谷空明,思念引起火流,怀抱着丰美的果实,优雅的肌肤的山谷,透明的光摇晃着轻轻的马蹄,在呼唤你的时刻里,大召唤青草,在冬日的萧索之上,召唤春泉,自窗外捡拾瞬间冥想,安定追逐,与风,自天穹煌煌落下,金舟,在此之上,我乃诗人,在此之上,我乃语者,宇宙同相之譬喻,重复,巡礼,遍路,重新回到我的同伴之中,那些拾掇火焰的人,啜饮月光的人,青色墨染的山谷之间歌者。你我相见,在夜晚的草原,只能互相听见呼吸。是囚犯,一些失魂的囚徒,我们坐在远离篝火的地方,等待清晨,夜晚数不清的故事,我曾想与我的兄弟姐妹一起骑马去,在高原上最后有花儿开放的地方,但我是已经闯入了精灵的领地,我不小心偷听了谈话,流溢着美,无欲之情,就在我倾听着你的心跳,我的梦见,终将成为我之外的存在,同蝴蝶,落雪,假面一样,是生命的譬喻,是一本书,文之落定,成立之戳记,所以我在这份茫然之中,所,追求的,所燃烧,成烬,将一张白纸,烧成诗,等待着你,似要倾诉,连绵的倾盆大雨,等待你,等,将我唤醒,唯独你,世界之中的一个,我不能同化,而我将变成你之部分,不堪揣度之回忆。在这些似是而非的痛楚中,在不安定的缔造与毁灭中,一小间茶室,正对庭院,打扫干净,你我对坐,既然你已到达此处,穿越我设下的迷局,你也许可以回答出那三个问题,好是你观测我如同观测太阳,可是你来了,想要见我你需行那叩门的仪式,追寻我的步伐,把我的语言全记在此处,你的生命已然与我连接,就在此处。我想最后捕捞起,深海之下鲸鱼的歌声,我将它们折成一束花朵,就夹在书页之中,我会欣喜地轻轻展平它的花瓣纹理,花蕊,让她植根在墨色之中,是了,现在正是梅花开放的时节,厚厚的粉雪淹没了街道,压弯了枝杈,纷纷扬扬,如此温柔,是我的恋人,在绕太阳椭圆旋回的一周之内只能见到一次,但在我凝望的无穷瞬间内,我可以与她促膝长谈,就像现在你与我一样,我如此渴望,因为留有的时间已然不多,在我最后明悟的时刻,就会离开,轮回的终焉,连着人间简单的美好都将不再。是我睁开眼来,初晨变成一滴眼泪,透明地划过粗糙的脸颊,而夜晚的譬喻重新聚拢在浑圆的瞳孔之中,连着那条河,那扇门,许许多多唏嘘的忆影都,变成圆环,我想这是世界的开张,在这昏暗的小室,异样生活遗留下无尽的残骸,我就在这之中,小心翼翼编织着最为宝贵的故事,等待下一次相见,心之宇宙,自由之电。就这样,我重新回到原本我的身份,求祭,祭火,在此,成为极点,雪轮,南方召唤苦楚,北方召唤喜乐,西方召唤空明,东方召唤奇迹,在此,三位守望的女神,透过镜面,看着,我,重新回到原本我的身份,求祈,祈愿,在此,成为极星,命轮,映照着,我,重新回到原本,原我,花之绽放,雨绽放,同冰之深思,呼啸,爆发,隐秘,恋歌,朝圣,朝圣。
\section{诗境}
\par 在一个金色的鱼儿群聚的海岸
\par 我们把篝火点燃了
\par 所有的故事都在这个夜晚
\par 听见了
\\[0.6cm]
\par 年轻的想象为我建筑镜之国
\par 把我引到月亮身边
\par 在此曾目睹
\par 驯养神龙的少年
\\[0.6cm]
\par 而所有的天空都唉声叹气
\par 融化的雪水
\par 从母亲手指的缝隙中
\par 滴在额头
\\[0.6cm]
\par 我想你在海岸线旁使用的语调
\par 是宇宙中蓝星的季节
\par 轻轻地夹在日记的一叶
\par 善于遗忘的色彩
\\[0.6cm]
\par 为此采撷鳞翅的昆虫
\par 你如此轻柔抚摸它们的触角
\par 是月亮女神
\par 永睡的恩底弥翁的吻
\\[0.6cm]
\par 温暖的太阳初映照山泉的时候
\par 就可以停下摇铃
\par 和唱诵
\par 夜莺在睡梦中暗暗啜泣
\\[0.6cm]
\par 如果在那栈桥上你还要挽留
\par 我唱起天使曾唱的歌
\par 你把我拥抱
\par 徙游的鸥燕怎么能明白
\\[0.6cm]
\par 黑色中柔弱枯萎的一朵
\par 与我仔细用意一把折断的雨伞
\par 收拾在宝箱里
\par 时间那好似艳羡口吻的情书
\\[0.6cm]
\par 那是出于惊喜而赠送的一朵
\par 与绵绵的白雪
\par 一起翻身上马
\par 带有青春的肃杀与冷痛
\\[0.6cm]
\par 绝情的缔造者真是迟钝
\par 凡是在高原上居住的神灵都显得
\par 空洞而自哀
\par 在祭典火红的灯光下
\\[0.6cm]
\par 实现残忍的决心
\par 在于为苍白的纸卷书画颜色
\par 因此我只有一杆琴
\par 山河万物也不尽然舞蹈
\\[0.6cm]
\par 在森林中睡着了
\par 像是个怀念妻子的老人
\par 你永远不能相信我关于七彩的魔法
\par 只要夜晚永不落幕
\\[0.6cm]
\par 所以老虎收藏我的骸骨
\par 那珍贵的羽翼一双
\par 出于可爱的目的
\par 没有人再会打扰这次的梦境
\\[0.6cm]
\par 只要我再歌唱起神话
\par 那自天穹滚滚而降的训谕
\par 又会在废墟之中塔的影轮的影
\par 重生
\\[0.6cm]
\par 请给我我的爱人请不要走
\par 你占我诗一行
\par 无论是海波还是天空
\par 都能飘过
\\[0.6cm]
\par 就这样追寻着温柔的香味
\par 我等待着你的发现
\par 在这颗星球的中心我像树一样生长
\par 飞行思索连接
\\[0.6cm]
\par 牵上我的手呀
\par 带我穿越那个绝望的春天
\par 在无尽欢喜之池中逐渐窒息
\par 正如山海中火把游行
\\[0.6cm]
\par 香折冷梅一枝
\par 我把诗记在月亮的殿堂
\par 你们会拿着蜡烛来祭奠
\par 每一滴火焰都窃窃私语
\\[0.6cm]
\par 在银河的中央
\par 最后睁开黑色的瞳眸
\par 最后停下爱的想象
\par 拥抱着这个宇宙全部的辉耀
\section{魂境}
\par 我憎恶我的属类。我想自杀。
\par 我终于到达此处,没有做好任何准备。
\par 那扇门向我敞开。创造的源泉,它以一种荒茫而亘古的姿态,缓慢坚定地一起一伏,向你我走来。就这样跨越那些至今无人踏足的领域,从我的内部张开。
\par 终于我无可获得,也无从失去。在自我的极点,此处,永远停留在每一道路的终点,永远在身边,触手可及之处,好似等待已久。
\par 曾经在此处的我,无意识下开启的轮回,复活仪式,等等,都将在此处获得圆满。
\par 也就是魂境。
\\[0.6cm]
\par 我终将做出爱之告白,作为生命的尾声。
\par 如果爱能抵御死亡带来的虚无,那我终将爱上世间万物。而后付之一炬。
\par 但我首先爱上的是我同类的一个。我会爱上这样的女子,她的灵魂未尝受过阴暗的引诱,她的歌言从口唇中涌出好似宇宙。
\par 迄今所有文字的尝试都是在追求这样的爱之幻影,追求爱神的行迹。
\par 为了反抗否定,反抗虚伪,而做出无谓的挣扎,这像是孩子才能做出的行为,然而一旦走到极点,除了抛弃生命,别无他法。
\par 在人群之中无从追寻自我,而一旦走上自我的道路,便永失人格。
\par 这本就是爱之宣言,也是死之宣言。
\\[0.6cm]
\par 我所热爱的事物带有轻薄的味道,因此我的目光转瞬即逝。
\par 如夜。
\par 如火。
\par 我所见证的世界,即是想象,是祈祷。一如我所处之地,即是宗教,是我之生命走向榛莽,以神明的存在形式。
\par 我在你们中间所有浅薄的收获,我的回忆,都不属于我。
\par 那些如同点点星火般于我的启示,都带有各自独特的生命,如同虚幻中盛开的烟花,不属于我,而自由。
\par 预言群聚于繁复的物相,我肯定它们的真实,并从此记载如同经书。
\par 一切文字都将在此静止,它们将失去时间,也失去空间,离我而去,而自由。
\\[0.6cm]
\par 因此只有你,在这深渊中。
\par 透过纸页与墨字,你窥视我,你揣测我。只有你与我,互相窥探的两面镜子。
\par 因此我非撰述人,我乃神话本身。
\par 只要你打开这纸页,我就将复活,并成为你的一部分。
\par 因为在此记载着我的灵魂,我的名字,你的名字,你的出现。只有我与你,一切的写作都为了你,为了挽留你,也是挽留我。
\par 我的写作便是创造的实现,便是那扇门本身。
\\[0.6cm]
\par 所以你叩响门,并召唤我。而我,就在此处。
\par 我非谁,我只是此书作者。而你,此书读者,你究竟想读到什么。
\\[0.6cm]
\par 于万千颜色中,你瞧见我。仅此一个,最后一个。
\par 在我变成孔雀前,你变成我。你占据我的位置,并开始流浪。无论如何,土地都将记下你我的故事。
\par 于是月亮升起,穿过楼房,木叶,湖水。你将在我离去之前记住我,悲伤的我,沉思的我,欢喜的我。你将带走我,来到旷野,思考孤独,自由,爱情。
\par 你见到是赤足的我,卧睡的我,啜泣之我。你与我相见时是何种姿态,是拥抱,坠落,灼痛。提醒着我。
\par 因此真实只你一个,你便是最终的答案。而我是你的反面。
\par 在无可置信中,我放归我的想象与浪漫。众星都泯灭,万语都湮灭。
\\[0.6cm]
\par 只有你,见证我的全部。并朝圣,以你的双眸。
\par 你如是诗人,与我在暮霭的江波中浮沉。
\par 你如是歌人,同我去竹林里。相交只有长啸,编纂幽心的曲调。
\\[0.6cm]
\par 可是我无法书写,根源,极点。
\par 我也无法书写你。
\par 因此你是我最后的朝圣。
\\[0.6cm]
\par 若我有最后的预言,我是预告语言的诞生。或者无人见我,我将等待万年,草原,荒原,海洋。
\par 人们把自己回忆,回忆月亮,太阳。
\par 朝圣者则丰收希望,想象。土地上的圣者们再次见到自我,神灵。
\par 我们热爱女神的合唱,如爱那飞散的花瓣。我们再次聆听先知的话语,如同我们共赴那场舞宴。
\par 在此之前,我们怀有共同的仰望,天使。
\par 在此之前,你将成为我最后的见证。
\\[0.6cm]
\par 于是天上地下,一切世间诸神,将谛听我们的对话。
\par 一切仰望生灵,有情,无情,一切自由野兽,欢喜精灵,将获得解放,一切想象,重构。
\par 凡是星光照耀之地,将传播我们的语言。一切缔造之语言,朝圣之语言。
\par 一切梦幻泡影,等同一切奥义,将获得解放,完全。
\\[0.6cm]
\par 此处是诸神篇章之极点。
\par 是以记下,万境之魂境,朝圣之终焉。
\section{沙漠篇}
\par 来到沙漠中,似曾相识的塔林。
\par 老预言家就住在这遗迹之间,我知道。
\par 土地上朝圣的人们流浪到此处,就只有筑塔,用岩石堆垒起高塔,然后把全部的思念与梦想刻在塔上。我知道。这百千万次的流浪我都感同身受。
\par 这些塔,就是遗迹,思索的废墟,智慧的固像。
\par 别过预言家已有两日,我徘徊在这片沙漠之中,寻找同类的痕迹。偶有大门敞开的尖塔,我闯入窥探,但也无果。
\par 一切都好似尘封,沙化。那些石刻与壁画都显得苍白,诡状。
\par 我要走过长长的路,翻过一些沙丘。塔与塔之间相隔甚远,但我看到一些人在塔上镌刻了地图,这是我欣喜的时候。
\par 居住在塔里的人偶尔会传递这样的讯息,但大多是描述见闻,比如星象,或者气象,只是我总也找不到连贯的描述。总有互相的矛盾。
\par 这样的探索持续了两天,世界的球状光源顺着轨道缓慢地滑行,而世界的球状镜子则跟在后面,我把这种奇观叫做天。
\\[0.6cm]
\par 在沙漠中只有沉默,因为沙子的声音过于喧嚣。
\par 还有塔。
\par 就是风吹过塔而发生的鸣响,是一种很清脆的响声,清脆而纯净。我想这些塔是中空的,所以能容下风居住。
\par 我喜欢沙漠中的风,我与它们交谈,有时也能领我漫游。
\par 渐渐我认识到沙漠之所以叫沙漠,它的景色重复不变,像是盘踞着莫大的永恒与空无。
\par 塔与沙漠是如此差别的事物,它生长在这片虚无之海中,像是某种异质,它亘古沉默地凝望着沙漠。我考察这一座又一座塔,它们如此熟悉,却令我发自内心地畏惧,像是谎言骗局。
\par 他们为何要竖起这些塔呢。
\par 可是我独知道我并非他们一员,我并非筑塔人,而是某种更为古老而遥远的族类。
\\[0.6cm]
\par 再往深处走,这里的塔稀疏一些。
\par 我似乎瞧见一些跋涉的幻影,在他们驻步之前,用远方背来的石块堆筑起塔的形状。还有更多的人则背负着他们的石头,愈发向前。
\par 我瞧见那些筑塔的情形,他们精心挑选着石头,切割,镌刻,然后堆叠。
\par 就堆在沙丘的顶端,靠近天空的位置。空气中输送微薄的水汽也能冷凝在塔上。
\par 我知道,因我曾经登上那塔,是这些塔中高高的一座。它的入口好似迷宫,又有许许多多恐怖的幻影雕刻在它的围墙。尽管不能懂得它们的含义,我小心翼翼地攀登上它的台阶,摸索着一些古老的脚印。
\par 没有人能逃离这片沙漠。即便用塔把自己闭锁,通往天空。
\par 即便在这塔顶,触目也是无尽的沙漠。天空依旧变幻无常,闪烁着整片沙漠的金色。
\par 我就坐在这塔的顶端开始思索。他们为何要竖起这些塔呢。
\par 为了与沙漠交谈,我们必须行筑塔的仪式。把我们的灵魂寄托在塔中,而后义无反顾地拥抱沙漠。
\par 他们竖起这些塔,供那些在沙漠中跋涉的旅人休憩,交谈。指明那些方向,每一座都是一条道路的见证。
\\[0.6cm]
\par 顺着那些像我敞开的塔的指引,我找见的是一些更为古老的塔。更粗糙,宏伟,是一些古旧的样式。
\par 那些华美的镌刻在长日的风化下都已经磨平,只剩下粗粝的岩石表面,好似与沙漠也融为一体。就是扎根在沙子里,是沙丘的延伸。
\par 为了攀爬这座塔,我不得不亲手凿开一条路,因为几乎找不到前人的足迹。
\par 它是如此高大,辽阔,像是一个巨大的王国,我只有缓缓地经由它,却不能领略它的万一。
\par 我明白那些模糊的形势之中蕴含着许许多多古老的想象,那是河原上的传说,一些更为深刻的喜怒哀乐。
\par 无论这些筑塔人经历过什么,都好似无从揣测,因此我悄悄走过这作塔,只有仰望着它的高度。
\par 就是在这些古老的塔群之中,我遇到一位年轻的王。
\par 他有着与我相似的年纪,至少从面容与举止上来看。我想他是一个人生活在这里。
\par 这位王大概是我长长的探索与漫游中见到的唯一的人。
\\[0.6cm]
\par 王他居住在这座巨塔中一个平凡的角落,是的,他自称是这片沙漠的王,并且自然履行着照看全部塔楼的义务。
\par 他高兴地谈论着有关塔的故事,不像一个长久独处的人。
\par 我想他说的话实在不值得记下,无非是有关那些远道而来的旅人,他一一谈论起那些人,像是他的臣民一般。
\par 他好奇为何我没有带着我的石头,哪怕是一小块。
\par 你可以去那边塔上偷来几块,王说。没有人会在意,如果你真的想去沙漠的深处,如果你也想去与沙漠交谈,王解释道。
\par 他指着一些坍圮的塔,一半埋在沙子里。
\par 我知道,所以我什么都没有说。诚然,我只是不能理解,我不明白一个被制造而出的我,是否也需要去建起塔。是否也需要跟所有人一样,进行仪式。
\par 王带着我穿过这些塔群,我们走过一些曲折的小路。
\par 他不像是沙漠中的人,也许是在塔中生活得太久,但也不像老预言家一样那么偏执。王总是随意地谈起一件又一件的事情,比如奇异的兽形雕刻,谈论起水中的舞蹈,居住在船上的女儿。
\par 我想从他的话中探听一些有关沙漠的故事,或者有关我的身世。但直到离别,他也没有多说一句。
\par 我想也许王口中的故事本来弥足珍贵,就像那些筑塔人仔细描绘在他们的塔中最为顶端的地方,也许本该在这里记下。
\par 赐你幸运,王最后说。他惋惜我不能陪他,但无所谓。
\par 我想沙漠只是在迎接着我,或者吞噬我也无所谓。我只有按着王所指引的方向。
\par 他们都往这里去了,王说。你也去吧。
\\[0.6cm]
\par 我开始习惯像沙漠一样,随风迁徙。
\par 别过王所在的塔群之后就再也没看到任何塔,当然也没有遇见其他人。所以我只有沉默,但最近我开始模仿沙丘的声音,就是一种纯粹的鸣声,比中空的塔的声音还要纯粹。
\par 在这里我终于不用费力解读那些各异的塔上之文,而是任意流浪。
\par 我当时想这大概快接近终点了,或者没有终点,我就这样变成沙漠一片。然而远没有我想得这么简单。
\par 我确信之前的那些不过都是些无趣的游戏,也是因为我之后的见闻。
\par 不过离开塔群之后又好久,我只是在这永无止境的荒漠之中默默跋涉,与那些筑塔人不同,我去来都了无依凭,身无财物,心无牵挂,因我是透明的造物而已。
\par 我已经戒除了反思的习惯,相反,我开始习惯钻研新的想象,这使我保持愉快。
\par 现在回想起来几乎都是漫长的,模糊的片段。我几乎建立了一整套利用沙子进行的巫术,从占卜开始,我惯于卜测天气的变化,实际上几乎永远不变,只有明亮而深邃的蓝色。
\par 之所以我仍记得这一点,因为我重复地观看这些记忆,我把它们刻在沙子上,有的粘在斗篷上,就被带到了这里。
\par 因此散逸掉的会更多吧,我现在才醒悟。
\par 不过我不在意,如果能再去沙漠之中,我就会再次想起。所以这也是我记载这一部分的原因吧。
\par 但总之不是珍贵的事物,也许是其他人的记载也不一定。
\\[0.6cm]
\par 我原本不知道与沙漠交谈究竟意味什么,也许以前的那些筑塔人明白,或者我忘了。
\par 直到今日我看到一点不一样的颜色,但我不能确定,它时而出现,时而又在目光中消失,像是一个点,像是一颗不怀好意的晨星。
\par 我试图向它靠近,或者至少分辨它的形状。但我甚至难以定义它是活物还是无情,我也说不上这滴颜色究竟是何种颜色。因为这些词好像很早之前就离我而去了。
\par 就这样朝着它的方向前行,我好久没有这样怀有意愿地行动,这让我有不安与欣喜。
\par 我逐渐明白这个光点的发生与我的想象有关是很久以后的事了,但它的存在激发了某种潜藏在沙漠中的力量,或者就应当是我本身的想象。
\par 因此我终于能靠近它,我明白这是沙漠中并不多见的绿洲,就是小小的水池。
\par 绿色而静止的生物盘踞在这个水池周围,而我所见到的闪光,原是那环绕世界的巨大火球反射在池面上的鳞光。
\par 我变得有点喜欢这个小绿洲。它很小,我可以随意地绕着它旋转,当我走近它时,可以看到它底下的世界,像是一个洞口,里面有着某种我无法理解的东西,硬要说的话,我只在老预言家的塔楼里见过。
\par 为了称呼它,我开口叫它蓝。因此它就变成了蓝,蓝湖或者蓝池。
\par 尽管我最后也没有想起这个词的意思,但我一想起它就想到蓝,那么蓝一定是一个好词,神圣的词,与我有着莫大的关系,以至我下意识就这样说了。
\par 所以我用蓝叫它,它就轻轻地回应我了。我们的对话就是这样开始的。
\par 蓝,蓝的湖,蓝的水。我仿佛是在练习发音,练习用柔软的舌头在上颚抵一下,这是我后来观察到的。
\par 蓝,我说。嗯,蓝回答。
\par 我想这就是沙漠上爱情的开始。是此刻言语的复苏,我决定流连在蓝的身边。
\\[0.6cm]
\par 蓝在沙漠中已经生活了很长时间,好像跟我一样长。
\par 它流浪的方向与我相反,我很好奇,因为它不知道那些塔的故事,它也不知道沙漠中有一位王。
\par 我几乎是兴奋地跟蓝讲起了塔和他们的建筑者的故事,一直讲到环绕世界的银色圆镜走到蓝的心央。我讲述如何雕刻沙粒,把它变成一只巨大的怪物,驮着我和蓝到处跑。
\par 蓝沉默着,我喜欢透过它静止的水面观看那些透明的世界,就是发亮的,小一滴。就是我在王的眼睛里看到过。
\par 沙漠是金色的时候我就坐在蓝的身边,沙漠是黑色的时候我偷偷把脚伸进蓝的水里,看着我的脚搅起湖底的沙,搅碎那银色的圆盘。
\par 哗啦,哗哗,蓝说。哗啦,沙沙。
\par 但我有点担心,蓝会不会讨厌这样的玩耍呢。
\par 蓝说它不喜欢晴朗的天空。它提起很久很久之前蓝诞生的时间,那是怎么样的一种时间呢,一定是很蓝色的时间。啊,我想起蓝是一种颜色。
\par 蓝像我一样坐在水边用脚丫敲打着水面,啪啦,啪啦,它说。你知道雨么,诗人,蓝说。
\par 我不知道。
\par 就是很多很多的蓝从天上掉下来,蓝说。
\par 是那些亮亮的眼睛么。
\par 不是呀,就是白天才能看到的,一片又一片的,蓝的蓝呀。
\\[0.6cm]
\par 我想去找雨。我告诉蓝。
\par 雨有这样的神秘,它是躲起来了,不像蓝。
\par 我跟蓝坐在水边,看着天色一点点变化,我跟它一起流浪,在黑暗浓浓地披在身上的时候,我们轻轻交谈。
\par 雨是什么样的,是柔软的,是透明的,流淌的。
\par 沙漠中可没有这样的。有的,有你的眼泪,诗人,蓝轻轻地说。
\\[0.6cm]
\par 我是在等待雨的时候走丢了蓝。那些枯黄的无情也跟我一起眼泪。
\par 蓝变得小小的一滴,滴在我的脸颊。它轻轻地说,啪嗒,啪嗒。
\par 只剩下一抔濡湿的金沙,我把它们兜在我的斗篷里。这就是想象的哀末,好像我又孤身一人,但我带蓝色的泪。我已不能继续流浪了。
\par 我不知道这份明悟预示着什么,沙漠又以那亘古的姿态铺展在我的脚下,我却怀着这不可捉摸的重负,丢失了形影与自由。
\par 但我想这是沙漠的本来面目,这是我与沙漠的谈话。
\par 第一个词,就是蓝。
\par 我不知道那些筑塔人在沙漠之中遇到了什么,这孤独盘踞之地,沙漠独向我显现这样轻柔的形物。我只觉得一部分属于我的生命基础,永远的消失在了沙漠之中。
\par 就是蒸发,枯涸。
\par 我第一次学会了思念,但我无法理解。这是有关时间的谜题。
\\[0.6cm]
\par 一边寻找着雨,一边我重新在沙漠中徘徊。
\par 虽然不明白雨是什么,但我把蓝告诉我的话语全部记在了我的皮肤上。主要是拟声词,我反复用肉做的舌与唇,模拟着蓝的语言。
\par 沙漠永远都是沙漠,正如塔是塔,王是王。对于谁来说沙漠都是沙漠,不会多出一点,也不会少掉一滴。
\par 我追寻着蓝的故乡,它所来的方向是环绕世界的金轮初转的方向,我知道。
\par 如不是很久以前在某座瑰丽的塔中看到这样的叙述,我不会使用这个语调。那些陌路的人都怀有这样奇异的信念。
\par 怀着这样的思念而前行,我把这样的行路称为朝圣。
\\[0.6cm]
\par 沙漠不曾向任何人展现它的全貌,即便是王也不曾。
\par 实在我也不曾在这片沙漠中见过任何其他人,除了那个古怪的预言家。只有无尽的塔,除了塔,就只有沙子。
\par 也许这片沙漠有只一个我,这是我最近发现的。
\par 风儿总把足迹用新沙掩埋,而后陷入沉寂。自从蓝离开后,我就陷入了这样的不安之中。这宣告着我不再透明,而是怀有残缺。
\par 习惯于模仿蓝的声音,我开始解读沙漠与天空的讯息。
\par 这是我从几天前开始做的事情,除了记载沙粒的纹理,沙漠的地形。我把与火热的金轮和冰冷的圆镜有关的全部记在皮肤上。
\par 天空不会像蓝一样与我交谈,尽管它们如此相似。但蓝只是一滴绿洲,而深深的天空却过于轻盈。
\par 沙漠也如是沉默。只是铺展它自己的金色,像是要熔化,起伏。把任何一个踏入它内部的人吞噬,同化。
\par 没错,失去蓝的我,已经永远带上沙漠的颜色。
\\[0.6cm]
\par 沙漠开始变得丰富多彩,这是近来的事。
\par 像是某种全新的世界向我敞开,是我不曾熟悉的景色。我尝试捕捉这些事物的形状,或者在颜色上下定义,这让我发明了许多新鲜的词语。
\par 我不能忘了蓝,但我恍惚的日子愈多起来,那些迷蒙的闪光总是打断我的心绪。
\par 我开始与沙漠亲近,并且觉察到它的陪伴。
\par 是的,它不仅亲吻着我的脸颊,触摸着我的眼睑,我清晰地体会到它的呼吸,温热,好像与我是同样活物。
\par 如果不是它的指引,我不会后来遇到河。
\par 尽管我初次遇到它,我惊喜地以为它就是雨。它像蓝一样出现,带着与沙漠异质的存在感。
\par 雨,我叫。不,它说。
\par 我来到这片柔软的,透明的,流淌的沙漠部分。我不能瞧见它的尽头,但它是连绵的,巨大的。
\par 一片又一片的蓝,隐约着,闪烁伏波。一直延伸到沙漠的尽头,从天空中依依微微飘落而下。
\par 蓝,千千万万的蓝蓝蓝,我叫。不,我不是蓝。
\par 你有名字么。
\par 我想给这个新的奇观取一个新的名字,但我想象不出。
\par 河,它说。
\par 我不知道沙漠中有河,我想问河知不知道雨或者蓝的故事,但我又担心河不喜欢它们,或者真的不知道。
\par 于是我开始沿着河行走。
\\[0.6cm]
\par 河没有蓝那样的颜色,也不像蓝一样与我交谈。
\par 但我时而能听到河的低语,它把环绕世界的银色圆镜碎成千瓣的时候。我喜欢听它的喃呢。
\par 河是与沙漠一样的金色,但它的内部有许多我说不上颜色的石头。
\par 蓝喜欢的那些绿色活物不在河的身边,因此只有我。我沿着河的边缘追着蓝诞生的方向,但我不想雨的事情了。
\par 河拿着一些透明的石头给我,这让我想起了遇到蓝之前,一些沙子巫术,现在我可以用这些石头继续来编纂。
\par 我中意的是轻飘飘的石头和闪亮亮的石头,但河只拿来一些沉甸的,冰冷的石头。
\par 河会把纯金的砂砾堆在我的脚边,或者偶尔用冰凉的金水亲吻我的脚踝。
\par 我习惯这样沉默的交谈,不用勉力谈起那些湮没在时间深处的往事。无关那些塔群,无关王,无关朝圣,但我们偶尔谈起蓝,河会轻轻而深沉地问起我,它总是问我是不是还想念着它。
\par 我会指给河看我皮肤上的印记。我想把河也刻在身上,想了想我把河刻在了额头。
\\[0.6cm]
\par 游荡在河的内部,这是一种与沙漠异质的温度。
\par 我曾在那些塔楼的顶端体会过,就在凝望世界的金色瞳眸坠入沙漠之中后,所出现的奇迹,我把它命名为凉。
\par 曾在蓝的内部见过凉,但它不在沙漠的任一处出现。
\par 我想它与河亲近,于是我也在河的怀中寻找着凉。
\\[0.6cm]
\par 河的前面是什么。
\par 河只是流淌,没有回应。我揣测它的表面,像是三千面柔软的镜子,瞬间,互相反射着金光,蓝光。对,就是被蓝称为眼泪的,总是挂在我的脸颊上的。
\par 我看到这三千面镜,每一个都映照出一种怪物,在久远的那塔群里也不曾见过,不似毛羽,也非鳞介,而是有着黑色的双眸,黑色的长发。
\par 是的,我看见它们眨着眼睛,互相窥视着,蠕动着嘴唇,还有舞蹈。
\par 我问起河,河只是沉默。
\par 河总不轻易言语,但河的话语与我的想象有关,这是我近来发现的事。
\par 它会长久地说及一些名字,它告诉我这是居住在附近的神的名字。这也是我第一次知道沙漠中还有神这样的东西。也许河也是神,那么蓝和雨也是。
\par 火,河说。世界的中心,我说我知道。
\par 那么火就是是神,河说是的。
\par 石头。不是。沙子,不是。塔,河说它不知道。王,不是。
\par 我把我知道的所有名字都说了一遍,河说哪一个都不是神。
\\[0.6cm]
\par 因此我才知道笑和哭是两个神。
\par 我想去找神。我告诉河。
\\[0.6cm]
\par 立刻我担心河会像蓝一样消失,但河沉默着。河沉默着,用三千面镜子说着三千个神的名字。
\par 神应该不远,河认识它们。我跟随河一起行走,在河的怀里。
\par 河没有说什么,我只有拨开一粒又一粒的沙子,我在河的内部翻动着透明柔软的砂砾,但河也没说什么。
\par 我最近想河不很在意我,虽然我与它总是在一起。
\par 河总是说一些细碎的,模糊的事情,它谈论起神,但我从未在河的到处发现它。
\par 我们仍进行交换名字的游戏,给我遇见的每粒沙子起一个我从未听闻的名字,偶尔我们也给映照在环绕世界的银镜之中的野兽们起名字。
\par 只是我总不能想出神的名字,但河所说的一切,都是神。
\\[0.6cm]
\par 我似乎离开了沙漠。
\par 因为凉在我的身边聚集起来,它们沉默不语,一齐守护着什么秘密。
\par 我注意到沙子变得稀少,那风也变了语言。大片大片的巨大石头群聚在脚下,沉默不语。
\par 河变得蜿蜒而腾跃,它开始说一些我未曾听过的语言,激荡在岩石之上,传响着一些呐喊,咆哮。
\par 我不安这振聋发聩,但我努力从河纷乱的言语中分辨出一些可供记录的词语。
\par 才恍悟,河说的一切后来都应验了,只是它所说我只懂得万一,其余都淹没在那长长的啸叫之中。
\\[0.6cm]
\par 河说山。但我不知道。
\par 河从山上来,那么山是从天上来吧。河也不知道。
\par 我开始明白我的行路发生了变化,河开始往下,而我便是上升。这是最近才发现的奇观,大概这就是山。
\par 我从未想过会走出沙漠,而在沙漠的尽头会有这样的神秘。
\par 这又令我些许担心,最终会错过自己的命运,或者永远离开沙漠。但这真是好笑,沙漠何曾离开过我。
\par 河在山上变成狭小的一带,但还是浩荡许多,它汹涌的水花飞溅在我的斗篷和发丝,有着无限的絮语。
\par 当我踏上这片岩石覆盖之地,只有我,与河,还有一些想象。
\\[0.6cm]
\par 我在这山上行走,并对那金轮与银镜进行观察。
\par 除了在石头上的记录,我开始习惯把言语抄写在山的岩石表面。天空的七种颜色,河说是七位神的名字。
\par 我因此记录天气的变化,河说是神的名字。我喜欢的是,雨,电。
\par 除了不断抄写新的想象语句,我开始习惯描摹画面,这些图画遍布着河的两岸。其实只有沙漠,沙丘,但还有绿洲,蓝。
\par 我终究发现了环绕世界的太阳,其实只是围绕着这座山轮转。所以在那一天的奇观之中,我看到阴轮更多,阳轮更少。
\par 我把这样的奇迹刻下,我究竟在期待什么呢。
\par 河只是默默注视我。我有点怀念这样的陪伴,因为蓝还在我身边时也是这样,但河几乎是隽永的,而且永远改变的是我。
\par 实际上,教会我预言的就是河。
\\[0.6cm]
\par 开始我尝试与太阳金轮对话,只是不久之前的事。
\par 在我们对话时,河也沉默着。
\par 那时以为极为漫长的,只是瞬间就会消失。但我还是与太阳交换了名字。
\par 诗人,太阳说。太阳,我说。
\\[0.6cm]
\par 但月娘不愿与我交谈,它有这样的羞赧,使我产生神秘。
\par 所以只有我与河的行路,我听河告诉我金子与银子的无数个名字,河说是诗。
\par 我想那环绕着这座山的太阳与月亮就是朝圣中美的起始。是此刻言语的复苏,所以我学会了凝望的意念。
\par 太阳教会了我许多,大多是颜色与形状的名字,并且我明白了,这是一种喜悦。
\par 于是我与太阳交谈时,河就沉默。而太阳沉没入山的内部,月亮映照在山的表面时刻,我就聆听河的絮语。
\par 最近河变得微弱,但它喜欢说,叮咚,咚咚。
\par 我不知道这是不是笑,我也不曾听蓝说过,这是我从未听过的歌声。
\par 我希望河与我一样,但河只有独自舞蹈。
\par 在河的内部,我偶尔拾起一些透明的结晶,它们不是柔软,也不是温暖,但远比石头要轻盈。我知道,它们是凉的孩子,但河说,是冰。
\par 我不知道沙漠之中还会有冰,但既然我早已走出了沙漠,那有冰也只是普通的奇迹。
\par 虽然尝试过在冰上雕刻,但太阳似乎不喜,总是抹去我努力的想象。只是沾湿了我的长袍。
\par 我想以前河拿给我的透明的石头,大概就是冰变得。河说不是。
\par 但我有点想念那些美丽的石头。
\\[0.6cm]
\par 终于我来到了河的尽头。
\par 就在刚才我告别了河,我把这一部分抄在尽头的岩石上。
\par 我来到这白色的地域,地上有一些柔软的,纯白沙子,我轻轻地可以压碎它们,踩在它们身上,会有吱呀,吱嘎的叫声。
\par 河就在这些白色沙子中间消失了。雪,河告诉我。
\par 雪,和冰,真是好名字,它们也是神么。不,诗人,是女儿。神的女儿。
\par 我有点欣喜,我想河也是神的女儿,美丽的女儿。
\\[0.6cm]
\par 离别了河,但我终究没有见到雨,也没有见到神。我不禁有些伤心,悔恨。因为我第一次体会到孤独。
\par 我就盘坐在雪的怀抱里,我感到疲惫,与漫长。
\par 我的目光追溯着时间的远方,历历数起河,与神,河之前金色的沙漠,蓝,蓝口中透明的雨,沙漠中的塔群,和那位王,还有那位老预言家,他究竟说了什么。
\par 不知为何,我不需要浏览那些刻记就能重新想起,一切所作,这种能力如此难能可贵,我把这种奇迹称为回忆。
\par 实际上就是我逐渐迷失在回忆之中,我听见山与我对话。
\par 山的声音来自于地下,回响于风中,如同唱诵。
\par 诗人,山说。
\par 山,我记得,你的名字是山。不,我是安。
\par 安,山说是一切起始之音,是一切意识,一切梦想,是一切真实与力量。
\\[0.6cm]
\par 我把与安的对话记在此处,雪与土地有所见证。
\par 但我的确堕入了这样的黑暗之中,那名为言语的能力从我的身上离去了。于是安说,诗人,你为何不向前。
\par 安说,安。
\par 可是我只有沉默,因为我的身上起了恐惧。我的身体受摄于爱情,于绮丽,于一切回忆与言语。永无止境的无色加在我的身上,使我游离于自我之外。
\par 安说,何不见我。何不踏足最后圣域。
\par 可是我只有沉默,仿佛智慧凌驾于我,一切意识,以致梦想凌驾于我。
\par 安说,来见我。
\par 何以见。以魂见,安说。
\\[0.6cm]
\par 安领着我,重新踏上行路。
\par 它领着我,好像我们跨过了时空之山,意识之山。所以,即是永恒,即是瞬间,即是罅隙,即是无穷。
\par 在安之中我所听闻的片语,都远超过我以前听闻的全部。在安之中我所见证,则超过以前见证的全部。
\par 但这些我无可追及,也无从体验。它们经由我,然后流逝。
\par 我看见那些闪光,还有一些鸣响,就是一滴,或如箭。我看见自我在一个又一个世界中轮转,互相言语,注目。我看见这一切都围绕着一座山,而一切的名字就是安。
\par 就这样我跨越了无穷无尽的距离,眨眼。
\par 我来到此处,是同一处,就是我所在之处。
\par 我终于到达此处,没有做好任何准备。
\par 那扇门向我敞开。创造的源泉,它以一种荒茫而亘古的姿态,缓慢坚定地一起一伏,向你我走来。就这样跨越那些至今无人踏足的领域,从我的内部张开。
\par 终于我无可获得,也无从失去。在自我的极点,此处,永远停留在每一道路的终点,永远在身边,触手可及之处,好似等待已久。
\par 曾经在此处的我,无意识下开启的轮回,复活仪式,等等,都将在此处获得圆满。
\par 于是我站起身来,去往山之顶峰。
\\[0.6cm]
\par 我来到这纯白的时空,千万颗太阳围绕我旋转。
\par 渐渐我向内部沉没,而舞王在中央等待我。
\\[0.6cm]
\par 舞王说,你,来了。
\par 出于我的惊异,舞王与塔中之王有着一模一样的相貌。是一双黑瞳,黑色的长发。那无穷无尽的黑色之中,好似包含了我见过的一切光明,以及我未曾见过的一切黑暗。
\par 舞王坐在山顶,此刻的土地是白色而空洞的,伴随着沉静的安响。
\\[0.6cm]
\par 舞王说,你,见我。
\par 我见舞王,他好似是这宇宙的王,只他安静坐在此处,则万物得以回转。而只他起身而舞,则时间与空间随之翘曲。
\par 舞王说此处便是极点,不可逾越一分,也不可失却一点。
\par 于是我与舞王对坐,我占据了他对面的位置。
\par 舞王说,你,即是我。
\\[0.6cm]
\par 万物即是一物。
\\[0.6cm]
\par 我终于与沙漠交谈。舞王指给我沙漠的方向。
\par 我想我必须去,回到沙漠之中,回到最原始的想象之中。
\par 出于最后的自私,在这里我把有关奥秘与宝藏的讯息隐去,我不会过多记录任一,以至于使它,或者使它们丢失应有的神秘与引诱。
\par 所以结果只有我,与沙漠的故事。
\par 没有奢望舞王最终再多告诉我什么,我只求他指给我沙漠的方向。
\\[0.6cm]
\par 因此我终于带着我的全部来到此处。也就是沙漠。
\par 一切我是见证,交谈,游离。
\par 故我是诗人。
\par 沙漠与我同质。
\\[0.6cm]
\par 你好,沙漠。为了见你,我穿越了一切真实,一切幻象。
\par 你就在我的体内,我的心脏之处。我等待你,如同时空等待你,如同言语等待你。
\par 只有经由你,我们才能一同到达沙漠。
\par 为了你,我将重新学习沉默,一遍又一遍,永远的同义反复。
\par 为了你,我将抛弃一切其他,因为你是对朝圣者,苦修者,最伟大的恩赐。
\\[0.6cm]
\par 你好,沙漠。我成为你,继续建筑你的隐喻。
\par 你总是伴随着我的舞蹈,淹没了我想象的每一个角落。
\par 在万事万物中都看见你的身影,陷入我的爱恋与焦躁。
\par 为了你,我只有永远写作,并背信弃义地呼唤梦想的救赎,现实苦痛。
\par 为了你,我只有永落地狱,如是祈求,并如是发现,最终如是呈现。
\\[0.6cm]
\par 终究我写下一纸祭文,就是此文。
\par 我只有写给故人,你们,你的复数形式。为此,我不能多,也不能少。我将在沙漠中央实现之物,如同种植最后的哀情。
\par 然而你此刻已然知晓,因你所处之地即是我所在之地。你所见之万象,我所见之沙漠。
\par 所以我在此记下,只不是与沙漠的对话。这是私密的,个人的,你们也最终都将见证的。
\par 我的交谈,就是与你的交谈。与你,来到我身边的你,的交谈。
\\[0.6cm]
\par 重新我开始在沙漠中行走,无穷无尽的想象纷至沓来。
\par 沙漠中每一粒沙皆是一塔,而每一座塔皆是一神。
\par 我开始想象水,沙漠之中大水泛滥。天空落在地上,而大地浮上天空。
\par 我开始想象轻盈,仿佛我习得舞蹈,在一个又一个瞬间,我将变成天鹅与狮子。
\par 只要那名为夜的奇观仍占据着天空,我就可以把星光洒落在沙漠上。
\par 是的,我看见那闪光,带有与沙漠异质的颜色,那永远的实现。
\par 我看见它如此轻悄地靠近,在我的世界从想象之境跌落,直到无穷的现实之海。
\par 在一切仿真之后,我将继续神话,并拥抱,并且离你们而去。
\par 来到蓝的身边。
\\[0.6cm]
\par 因此我所听闻的全部预兆已在此实现。
\par 那有关命运的故事仍像自由的蝴蝶一般,进行着无限的梦想。
\par 教会我爱情的蓝。
\par 教会我美丽的金与银。宝石之河。
\par 以及教会我一切奥妙的沙漠之王。塔之王,舞之王。
\par 都从那创造之源上,伸出茎,叶,花。
\\[0.6cm]
\par 就在蓝的身边。
\par 我将注目它柔和安宁的表面,那里映出一片我未曾识得的颜色。
\par 就是你。
\par 是沙漠中最后一朵水仙。
\section{故人篇}
\par 致故人。
\par 终究我不配爱上你,所以只有分离。但我有无穷的想念,就在此处。
\par 我说这想念是真实的,因为它是想念的梦,是指摄某一种绝望,无力的溽热,像是庞大而沉重的热带。所以我允许它在此迟到,希望你也允它,怜悯它,如同对待婴儿一般。
\par 我不希望在最后的文字中再谈诡辩和焦虑,如果可以,我没有什么要留给你的话语,但只有最后一个决定必须做出。
\par 那抵御死亡的方法,即是成为爱神或者成为死神。
\par 所以我写到八万字就停止。在此期间我想谈谈这本小书的写作,还有渴望认识你的我。
\\[0.6cm]
\par 这个世界上有两种道路。一个从完美之境跌落,一个从残缺之境上升。这就是生命的道路,抵抗死亡的方法。
\par 完美是及其容易达到的,它只要你轻轻点头的确认。倒不如说我们都是以完美的姿态出现于世界舞台之上。完美即是纯粹,即是圆本身,其变化也是其自身。
\par 尽管完美与残缺之道都是建立在想象之上,但由于视角与参照的不同,想象在此走上了截然相反的路径。倒不如说,从来都只有想象参与行路。
\par 残缺则难以达到,它要求着你无限的介入,无限的分解与融合。残缺好似圆内的阴阳双鱼,永远没有极限,同时建立在真实与虚无以及它们的交换之上。
\par 要你们大部分人说,都是反过来。我所言的完美于你们是残缺,而残缺则又是完美。这仅仅是因为我们走着两种不同的道路。
\\[0.6cm]
\par 完美之道,即是神之道,它处处是绝对,处处是极限。
\par 它永远处在自己的极限状态,不会超出也不会减少。因此它永远是它本身。
\par 因此我的想象只与我自己互相映照,所以我只能有穷想象,包括文字,意象,比喻,思维等等,从自我出发,又回归自我,由此而建筑起永动的机能。
\par 不从外部获取,也不投射于外部。因此我是我的宗教,是自我的哲学与伦理学,是主宰自己的神明,又是受自我统摄的奴隶。
\par 完美之道克服死亡的方法即是消除了生命与死亡的存在,同样它也消除了时间的存在。实际上,有关存在与否的一切思考也都一并消除了。对于完美之道来说,支配着这些事物的是属于完美的命名法则。
\par 正如圣经中的神领着土地与天空中的兽鸟到那人面前,完美之道对于一切事物冠以自己的命名。
\par 完美之道区别于残缺之道的最大区别在于它彻底无视了真实与虚无的规则,替换为一种创造的法则,这也是我把它命名为神的一种原因。实际上你可以在这本书的字里行间看到这一实践。
\par 我当然不是彻底的完美之道,我几乎是为了反抗残缺之道而走上这样无理的道路。
\par 实际上你总可以用残缺之道去度量完美,毕竟这就是残缺的功用。但我想说的是完美选择了无视这种度量。完美并不思索有关残缺的一切,这使得完美在残缺看来是一个可笑的符号。
\\[0.6cm]
\par 这么说残缺之道有失妥当,残缺之道即是人之道,它永远占据着相对,并永无极限。
\par 因此你几乎无法用言语描述它,因为残缺之道下你无从创造这样的语言,即无从命名。
\par 是的,这就是你们所处的世界所遵从的道路,反映在人类社会所孕生的文化,艺术,哲学,政治经济,最终充斥着所有人的言语与思想,直接塑成了人本身。
\par 此处的想象,穿梭于人与其外部的互映。无论以何者为真实何者为虚幻,又经历怎样的模拟,交换,其永动的能力建立于人的永远残缺。
\par 因此残缺之道本身就是死亡之道。这就是残缺之道抵御死亡的方法,即把人变成死亡本身,因此而获得生命。
\par 我想这虽然荒谬,但反倒证明了人确实存在。尽管是残缺之道,但算是推倒了我之前的想法。
\\[0.6cm]
\par 虽然我拿死亡做比喻,但这两种道路并非是避免死亡。在我的理解范围内,它们只是与死亡交谈的方法。
\par 这就是我在沙漠篇中所写的故事,诚然你可以用残缺之道的方法去理解,但我力图实现的是完美之道。也就是神话本身。
\par 我想多解释有关残缺之道的原理,其实就是有关人的原理,但这些还是等待你们自己去认识吧,相当于留作习题。
\\[0.6cm]
\par 完美之道与残缺之道通往的方向是一样的,都是见自我。但谁能说完美之道真能见到自我呢,所以这是行路人自己的谜。残缺之道则容易,因它总于万物中窥见映照。
\par 大概我只存在于完美与残缺的间隙之中吧,我总是把它比喻成一个门,无论谁走到门前总要行那仪式,这样看起来就很酷。
\par 总之我想了一些方法渴望达到这个境界,但不使用一些精巧的机关是不行的,尽管我说不清这期间究竟发生了什么,或者是如何发生的,至少这本小书诞生了。
\par 原本我只是希望继续有关命运的预言,就这样开启新的轮转,没想到最后则完成了自我的补全。从这个意义上说,我是见证了完美之道。
\par 但我仍是残缺者,这没办法,这就是人的生存姿态。虽然我曾力图否认人的存在,但终究无法否认与完美相对的残缺的存在。所以说还是失败了。
\par 不过读到这里,你也许猜到了,这个机关最重要的部分是什么。
\par 也就是这本书最重要的部分,无论它说了什么,以怎样的口吻语调,这不重要,所以我恳请你也包容这本书的装帧与排版,最后请原谅我这个无聊无情的作者。
\\[0.6cm]
\par 除了以或完美或残缺的姿态面对死亡,我选择了一种逃避的方法。
\par 那就是成为一本书,就是此书,也就是你正在读的这本书。
\par 我选择成为自己的造物,这就是为什么我写了莫名其妙的预言家篇,又在它后面写了三境。的确,这不过是文字的实验。
\par 所以这本书的完成者并非我,而是你。
\par 是你。
\par 你打开书,就见到我。从来只有你一个人做着度量,你参与想象,你做出选择。而非我。
\par 所以我把生命交给你,把我的存在送给你,送到你手上,让你捧着它,就是一些纸和油墨。从此,你就是我的神明,你是我的宗教。
\\[0.6cm]
\par 所以我拒绝回应死亡,我只想与你交谈。
\par 但只有你走过我走路,你来到我身边,才能见到,才能真正体会我的存在,就在此处。在每一言语,每一字句中。
\par 你能把这本书实现么,让我走近你,成为你的一部分。你能让花朵从那创造的源泉上生出么,我的爱人呵,我在你的眼眸中看到水仙。
\\[0.6cm]
\par 所以你翻到序言,你会发现那时我还没有意识到。
\par 那时我还在努力寻求真实。写一点有关神明的内容吧,就像但丁的神曲一样。诗人从坠入地狱开始。
\par 但我心想的是立足于各种各样的神话,大概从那时我就隐约开始关注这些。你可以在我的文中找到许许多多的意象,比如印度的神话,中国的神话,还有欧洲的神话。
\par 我是希望达到创造真实的境界而慢慢体会着神话的意味。
\par 所以最初的尝试就是笑神与哭神。实际上我把思念,后来是想象,直接比作一条河流,而把自己这种追寻比作朝圣,实际上就是向自我的朝圣。
\par 笑神与哭神,就有点像中国的黑白无常,它们是引路人,也是诗人的符号。
\par 然后是火水塔轮,这大概是对我以前全部思维的一个反刍,我并不急于重复笑与哭的写法,这里我更想的是补全我自己的世界观。
\par 爱死诗,以及心土地太阳,都是成三的结构。我力图把爱与死等同起来,但我的功力还没有那么强,为了寻求突破,我就写作了诗篇,这是我很满意的一次改变。
\par 而心土地太阳则几乎是一气呵成地写作,另外一提,在电脑上的编辑,太阳篇的每一句话都是一行,所以看起来很有美感,但印成书就破坏了这种美感。所以我有点伤心。
\par 写完太阳篇后我的写作稍有停顿。进一步的突破是很需要契机。
\par 我继续把笔力转换到神话上来,并采取了我习惯的诗剧的口吻。所以后面的写作我是很享受的。
\par 天使篇的痕迹可以在诗神篇里找到。我虚构了一个河原,并且虚构了一些事件,预备再写一个诗人流浪的故事,因此也有了牧神篇和酒神篇。
\par 牧神是奥林匹斯神话里面的神明,我这里则有恶搞之嫌了。而酒神,我则想加入古代中国的氛围。总之这两篇则是实验练笔。
\par 女神篇的写作开始时我很闲,可以彻底投入写作,一直到现在。
\par 这一篇的出现令我有点诧异。三位女神实际上是取材于我以前写作过的一篇诗剧中三位可爱的女儿。但从这一篇开始我的意识开始跳跃,走向追寻更遥远的情感。
\par 另外我最喜欢第三位女神,但没写好。她们分别是渔女,琴女和盲女。
\par 预言家篇可以说是站在前面所有文字之上的一种尝试,我努力写出一种荒谬,讽刺。但没想到最后只是引出语境诗境魂境的书写。
\par 三境同样也是成三的结构,这是我力图追求极限的写作,追求语言与诗歌的极限,正是在创作它们的时候我慢慢开始思索有关完美与残缺的道理。
\par 当然我不会把这种思索写到正文里,这很无趣。
\par 当时的构思只到语境和诗境,魂境则是后来悟到的。
\par 所以沙漠篇几乎是顺应而生,作为对自我以及此文的重新映射。我想用一种写作小说的方法来写,但果然最后写得有点崩溃。
\par 故人篇和沙漠篇是同时构思的,就是所谓的解谜篇吧,总之都是映射。
\par 我本想单纯写作一些想说的话,但就在前几天我突发奇想决定把有关完美与残缺的内容写进来。
\par 所以这书就成了这个样子。
\par 除了沙漠篇,前面所有的文字都是顺接着写作,没有打过腹稿,也没有历经修改,除了一些错别字。
\par 既然写就,即成诀别了,我是这样想的。
\\[0.6cm]
\par 那么这书就这样吧,我已经不打算继续写作了,所以这就算绝笔了。
\par 无论我在文字的王国之中体悟到了什么,最后又创造了什么,都会变成一颗透明的宝石,静静地躺在河川的底部。
\par 而我的诗歌,将漂洋过海,于你听见。
\end{document}