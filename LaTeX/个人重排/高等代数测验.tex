\documentclass[11pt]{article}
% 用ctex显示中文并用fandol主题
\usepackage[fontset=fandol]{ctex}
\setmainfont{CMU Serif} % 能显示大量外文字体
\xeCJKsetup{CJKmath=true} % 数学模式中可以输入中文

% AMS全家桶,\DeclareMathOperator依赖之
\usepackage{amsmath,amssymb,amsthm,amsfonts,amscd}
\usepackage{pgfplots,tikz,tikz-cd} % 用来画交换图
\usepackage{bm,mathrsfs} % 粗体字母(含希腊字母)和\mathscr字体
\everymath{\displaystyle} % 全体公式为行间形式
\usepackage[margin=2cm,landscape,a3paper,twocolumn]{geometry}
\usepackage{array}
\renewcommand{\arraystretch}{1.5}
\title{高等代数第五章章节测验}
\date{}
\begin{document}
\vspace{-5em}
\maketitle
\vspace{-5em}
\begin{center}
    学号\underline{\hspace{6em}}\hspace{2em}
    姓名\underline{\hspace{6em}}\vspace{1em}

\begin{tabular}{|c|>{\centering\arraybackslash}p{0.07\textwidth}|>{\centering\arraybackslash}p{0.07\textwidth}|>{\centering\arraybackslash}p{0.07\textwidth}|>{\centering\arraybackslash}p{0.07\textwidth}|>{\centering\arraybackslash}p{0.07\textwidth}|}
    \hline
    题号 & 一 & 二 & 三 & 四 & 总分\\
    \hline
    满分 & 10 & 20 & 40 & 30 & 100 \\
    \hline
    得分 &  &  &  &  & \\
    \hline
\end{tabular}
\end{center}

\paragraph{一、填空题 (本题共 2 小题, 每题 5 分, 共 10 分)}
\paragraph{1.}
多项式$x^3-6x^2+15x-14$的有理根是\underline{\hspace{4em}}.
\paragraph{2.}
若$(x-2)^2\mid Ax^4+Bx^2+16$, 则$4A+B=$\underline{\hspace{4em}}.



\paragraph{1.}
\paragraph{1.}
\paragraph{1.}
\paragraph{1.}
\paragraph{1.}
\paragraph{1.}
\paragraph{1.}
\paragraph{1.}
\paragraph{1.}
\paragraph{1.}
\paragraph{1.}
\paragraph{1.}
\paragraph{1.}

\end{document}