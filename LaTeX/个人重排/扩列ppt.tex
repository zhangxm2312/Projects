\documentclass{beamer}
\usepackage[fontset=fandol]{ctex}
\setsansfont{CMU Sans Serif}
\usefonttheme{professionalfonts}
\usepackage{bm}

\usetheme{default}
% 可选主题:default,AnnArbor,Antibes,Bergen,Berkeley,Berlin,Boadilla,boxes,CambridgeUS,Copenhagen,Darmstadt,Dresden,Frankfurt,Goettingen,Hannover,Ilmenau,JuanLesPins,Luebeck,Madrid,Malmoe,Marburg,Montpellier,PaloAlto,Pittsburgh,Rochester,Singapore,Szeged,Warsaw
% \useoutertheme{sidebar} 使logo和标题作者放在左侧sidebar中

\title{\LaTeX 在扩列条与年度总结中的应用}
\author{章小明}
\date{\today}


\begin{document}
\maketitle
\begin{frame}{目录\footnote{该PPT基于\LaTeX\, beamer默认模板,由毕业论文答辩PPT模板修改而成}}
    \tableofcontents
\end{frame}

\section{基于PPT形式的年终扩列工作开展}
\begin{frame}{请与我扩列!}
    \begin{itemize}
        \item 谁是章小明?
        \begin{itemize}
            \item 一位年轻的理工科经典研究牲
            \item 在生活和精神状态不稳定的痛苦中试图存活下来的ASD\footnote{相信我,其实是学数学学的}
            \item 温和但不够有趣从而时而摆烂的日子人
        \end{itemize}
        \item 对什么感兴趣呢?
        \begin{itemize}
            \item 数学\footnote{准确的说,正在代数组合与代数图论方向.},以及骑车
            \item 魂类和类魂游戏以及解谜游戏, 最近对类银河恶魔城感兴趣,以及该继续玩博德之门了
            \item 在做心理咨询,所以对精神分析(临床)感兴趣,有时会试图给朋友解梦
            \item 学习一下如何社交(嗯)
            \item 其它:一些游戏剧情,一些简单的古典音乐和歌剧
        \end{itemize}
    \end{itemize}
\end{frame}

\begin{frame}{请与我扩列!}
    \begin{itemize}
        \item 会发些什么?
        \begin{itemize}
            \item 一些随口一说的思考和日常的碎片
            \item 一些简单的心情和感受
            \item 偶尔会有较长的回忆和感想作为日记
            \item 一些可能有趣/有梗的东西
            \item //一些你会觉得中文很烂的长难句表述
        \end{itemize}
        \item 会想/不想加的人
        \begin{itemize}
            \item 来者不拒!我希望能观察到更多类型的人,以及与更多的人建立联系
            \item 我不喜欢的:一些过激或刻薄的话或观点
            \item 你可能不喜欢的:男的,而且常常只是个人机从而活性不够
        \end{itemize}
    \end{itemize}
\end{frame}

\section{关于2024年学习工作的系列总结}
\begin{frame}{基于月份回溯的总结}
\begin{itemize}
    \item 一二月:刚考完研自觉上不了国家线该二战了于是一直在自闭,所以还是跑出去旅游了.回家跟我妈对线,打打游戏看看书,享受最后的岁月静好.除此之外准备一下毕业论文,所以弄了个Clifford代数讨论班.
    \item 三月:意料之外的不止过了国家线还进了复试,在焦虑地复习.开始准备毕业论文了,感觉所有压力都很大,但复试考完之后感到这一切都问题不大了.
    \item 四月:上岸了于是感到一阵空虚,一个月写完毕业论文更空虚了.但这时跟女朋友分手了,痛苦.准备一些毕业事宜,发现跟这一届的同学基本不认识,孤独.
    \item 五六月:还在痛苦,但开始重新建立生活了.开始在家看代数书!所以觉得还挺自如的.
\end{itemize}
\end{frame}

\begin{frame}{基于月份回溯的总结}
\begin{itemize}
    \item 七八月:继续(深入)看代数,和我妈对线,平淡无聊的生活,但学代数还挺充实的.和一个很久的好朋友绝交了,所以痛苦.
    \item 九月:准备开学于是在焦虑这一切:导师选择,要学的内容,以及更多的新生活.天天骑车,因此又感觉快乐.
    \item 十十一月:应接不暇的新内容,比如图论以及带助教课.导师放养而且没有什么真正新的内容,因此感觉在学校做一天和尚撞一天钟.好消息是比较有成就感,因为有人会问我学过的内容了.
    \item 十二月:感觉学习压力变大了(点名图论作业),但期末考完就没什么问题了.精神状态在萎靡,即使出去玩了几次.感觉没有表述能力因此茶壶里倒不出饺子,痛苦.
\end{itemize}
\end{frame}

\begin{frame}{一些年度总结}
    感觉生活在持续性钝刀子割肉的痛苦,虽然来源多种多样但其实只是精神障碍在作祟罢了,可能学数学能让我在此之中寻得一丝喘息,所以我如此看待之.\vspace{1em}
    
    一直都在社交上感到不适应和无能为力,但其实好消息是在这方面我越来越清晰了.尽管问题总或多或少存在,但一个很大的进步是我认识到我只是ASD罢了.另一个进步是认识到自己真的挺自恋的,但认识到这一点就已经善莫大焉.\vspace{1em}

    虽然最近状态真的不太好,但其实总该说点什么.可能没有什么对下一年过多的展望,但其实我只希望我能有一个平衡的状态.虽然现在还在一个对时间流逝没有感觉的冷漠的状态,但我想这只是暂时的,冬季的.
\end{frame}

\section{结语}
\begin{frame}
    \begin{center}
        \Huge 谢谢各位老师!
    \end{center}
\end{frame}
\end{document}