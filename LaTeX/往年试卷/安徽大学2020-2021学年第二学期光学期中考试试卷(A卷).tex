\documentclass[UTF8]{article}
\usepackage{ctex,geometry,enumitem}
\usepackage[compact]{titlesec}
\geometry{a4paper,left=1cm,right=1cm,top=1cm,bottom=1cm}
% \pgfplotsset{width=10cm,compat=1.16}
\setenumerate[1]{itemsep=0pt,partopsep=0pt,parsep=\parskip,topsep=10pt}
\setitemize[1]{itemsep=0pt,partopsep=0pt,parsep=\parskip,topsep=5pt}
\setdescription{itemsep=0pt,partopsep=0pt,parsep=\parskip,topsep=5pt}
\titleformat{\section}{\rmfamily}{\thesection. }{0em}{}
\titleformat{\subsection}{\rmfamily}{\thesubsection. }{0em}{}
\newcommand{\m}{\mathrm{m}}
\newcommand{\cm}{\mathrm{cm}}
\newcommand{\mm}{\mathrm{mm}}
\newcommand{\nm}{\mathrm{nm}}
\newcommand\ptsMulti[3]{ \small($#1\!\!$小题$\!\!\times #2\!\!$分$\!\!=\!\!#3\!\!$分)}
\newcommand\pts[1]{ \small{(#1分)}}

\title{安徽大学2020-2021学年第二学期光学期中考试试卷(A卷)}
\author{出卷人:杨群}
\date{}
\begin{document}
    \maketitle
    \section{\large\textbf{简答题} \ptsMulti{4}{5}{20}}
    \subsection{全反射的概念和条件}
    \subsection{Fermat原理}
    \subsection{简述光的三个相干条件和实现方法}
    \subsection{简述薄膜干涉中等倾干涉和等厚干涉的定义和条纹特征}
    \section{为把仪器刻度放大2倍,在仪器上置一平凸透镜,并将平面与刻度紧贴.假设刻度和球面镜顶点距离为30mm,玻璃折射率$n=1.5$,求凸面半径.\pts{15}}
    \section{一薄正透镜将一实物成实像,物像间距112.5cm.(1)当像高为物高4倍时,求正透镜到物的距离;(2)求上问中正透镜的焦距;(3)若所成像高为物高1/4,且物像位置不变,则正透镜应向何方移动,移动多少距离?\pts{15}}
    \section{在Young双缝实验中,两缝间距$d=0.2\mm$,在距离$D=1\m$远的屏上观察干涉条纹.若入射光是波长$\lambda=460\nm\sim 760\nm$的白光,在光屏上离0级明纹20mm处,哪些波长的光被最大限度地加强?\pts{10}}
    \section{波长$\lambda=500\nm$的单色平行光照射在间距$d=0.2\mm$的双缝上,通过其中一个缝的能量是另一个的2倍.在离狭缝$D=50\cm$的光屏上形成干涉图样,求干涉条纹间距$\Delta x$和条纹衬比度$\gamma$.\pts{10}}
    \section{用波长$\lambda=500\nm$的单色光垂直照射到由两块光学平玻璃构成的空气劈形膜上.在观察反射光的现象中,距劈形膜棱边的1.56cm的A处是从棱边算起的第4条暗条纹中心.(1)求此空气劈形膜的劈尖角$\alpha$;(2)改用$\lambda'=600\nm$的单色光垂直照射此劈尖,A处是明纹还是暗纹;(3)在上问条件下,从棱边到A处的范围内共有几条明纹,几条暗纹?\pts{15}}
    \section{用Na光($\lambda=589.3\nm$)观察Michelson干涉条纹,先看到干涉场内有12个亮环,且中心是亮的.移动平面镜$M_1$后,看到中心吞吐了10环,而此时干涉场内还剩下5个亮环.求:(1)$M_1$移动的距离;(2)开始时中心亮斑的干涉级;(3)$M_1$移动后,从中心向外数第5个亮环的干涉级.\pts{15}}
\end{document}