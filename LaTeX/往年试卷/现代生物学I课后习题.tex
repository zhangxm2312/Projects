\documentclass[UTF8]{article}
\usepackage{amssymb,amsmath,amsthm,amscd,latexsym,ctex}
\usepackage{tikz,tikz-cd,pgfplots,geometry,enumitem}
\usepackage{mhchem}
\geometry{a4paper,left=2cm,right=2cm,top=1cm,bottom=1.5cm}
\pgfplotsset{width=10cm,compat=1.16}
\setenumerate[1]{itemsep=0pt,partopsep=0pt,parsep=\parskip,topsep=5pt}
\setitemize[1]{itemsep=0pt,partopsep=0pt,parsep=\parskip,topsep=5pt}
\setdescription{itemsep=0pt,partopsep=0pt,parsep=\parskip,topsep=5pt}

\title{现代生物学I课后习题}
\date{2020.08.19}
\author{zhangxm2312@gmail.com}
\theoremstyle{definition}
\newtheorem{example}{例}[section]

\begin{document}
\addtocounter{section}{-1}
\maketitle

\section{绪论}
\begin{example}\textbf{生命的基本特征是什么?}
    \begin{enumerate}
        \item 生长。生长是生物普遍具有的一种特征。
        \item 繁殖和遗传。生命靠繁殖得以延续,上代特征在下代的重现,通常称为遗传。
        \item 细胞。生物体都以细胞为其基本结构单位和基本功能单位。生长发育的基础就在于细胞的分裂与分化。
        \item 新陈代谢。生物体内维持生命活动的各种化学变化的总称,包括同化和异化。
        \item 应激性。能对由环境变化引起的刺激做出相应的反应。
        \item 病毒是一类特殊的生命。
    \end{enumerate}
\end{example}
\begin{example}\textbf{孟德尔在生物学研究方法上有什么创新?}

    孟德尔的豌豆杂交实验,为遗传学的发展奠定了科学基础。相较于前人有下面显著特点:
    \begin{enumerate}
        \item 他把许多遗传性状分别开来独立研究。
        \item 他进行了连续多代的定量统计分析。
        \item 他应用了假设---推理---验证的科学研究方法。
    \end{enumerate}
\end{example}
\begin{example}\textbf{有人说机械论和活力论是互补关系,你的看法如何?}
    
    个人观点觉得机械论和活力论是相对立的关系。“活力论”观点认识生命,认为生物体具有与物理化学过程不同的生命力,即活力。与活力论相对立的是“机械论”观点,认为生命问题说到底是物理和化学问题,一切生命现象都可以用物理和化学定律做出解释,生物体内没有什么与物理化学不同的生命力。其实个人觉得生物体是不同于物理化学系统,是高级的、非常复杂的生命系统,当把它还原为简单的物理化学系统以后,它所具有的一些特别的性质和功能就会失去。
\end{example}
\begin{example}\textbf{你是否认为21世纪时生命科学的世纪?}

    20世纪下半叶,生物学进入分子生物学时代,研究生物大分子物质的结构、性质和功能,从分子水平上阐述生命现象。20世纪下半叶以来,生命科学文献在科学文献中所占的比例、从事生命科学研究的科学家在自然科学家中所占的比例都在迅速增长,这就是这种趋势的反应。生命系统是地球上最复杂的物质系统,是从非生命系统经过几十亿年进化的结果。现代科学技术的发展对生命科学发展起到重要的作用,生命科学的发展对整个科学技术的发展产生重要影响。生命科学与农业的可持续发展:解决粮食短缺,基因工程将在育种中发挥重要作用。应用基因工程可以改善粮食和畜牧产品品质。实现农业的可持续发展,克服农业化学化的恶果,必须生物防治,降低对农药的依赖等。 生命科学与能源问题的可持续发展:解决能源问题,对生物技术给予厚望。生命科学与人的健康长寿:研制更有效的药物、在基因组的基础上认识人体,理解疾病。生命科学与维持地球生态平衡。
\end{example}
\begin{example}\textbf{举例说明生命科学技术引发了哪些伦理道德问题?}
    
    人类是高度社会化的生物,人类社会有特定的伦理道德,生命科学技术的在人类社会的应用时会引起伦理道德的问题。例如人工授精和试管婴儿技术,可能使子女“只知其母,不知其父”。若供卵者与怀孕的不是一个人,则生母也成了问题。例如克隆技术可以实现人的无性繁殖,那么,人类自身的生产也会批量化吗?例如应用基因工程技术改造人类本身,一些人成就了改造活动的客体,而另一些人是主体,一些认识按照另一些人的设计被创造出来的,这种不平等岂不更甚于财富多寡和政治地位的不同。再例如,人类基因组研究的深入,使获得个人基因信息成为可能,这是不是个人隐私?会不会引致基因歧视。
\end{example}
\section{第一章}
\begin{example}\textbf{从元素周期表上看,生物体的元素组成有何特点?}
    
    参加生物体组成的元素,总数达30余种,在元素周期表中,这30种元素分布在元素周期表的上部和中间部分,即属于相对原子质量较轻的一批元素。
\end{example}
\begin{example}\textbf{什么是微量元素?试设计一组实验,证明某一种微量元素是人体健康所必须的?}
    
    \begin{itemize}
        \item 微量元素是指一些在生物体中含量甚底的元素,一般在百万分之一级。如:铁、氟、锌、硅、硒、锰、碘、钡、锶等等。\\ 微量元素虽然在体内的含量微乎其微,但是能起到重要的生理作用。
        \item 对微量元素功能的研究\\ 哪些元素对人体有益,哪些有害,取决于人们的认识水平和实验。\\ 实验方法:动物实验以缺乏某一种元素的饲料喂养实验动物,观察实验动物是否出现特征的病症。 在饲料中恢复添加这种元素,实验动物的特征病症逐渐消失,恢复健康。应在分子水平或者细胞水平找出该元素的作用机制,是某种酶的必要成分,或参加某个代谢或某项细胞运动。
    \end{itemize}
\end{example}
\begin{example}\textbf{分析水对生命的重要意义。}
    
    \begin{itemize}
        \item 生命起源于水
        \item 物质的溶解、运输和利用需要水
        \item 许多生化反应中水是底物或产物
        \item 关节的润滑
        \item 肺泡的生理功能
        \item 毛细管作用---植物根系吸收水分
        \item 蛋白质、核酸、脂类和多糖是组成生物体最重要的生物大分子,水是生物体内所占比例最大的化学成分。
    \end{itemize}
\end{example}\begin{example}\textbf{糖、脂、氨基酸、核苷酸,这四类生物小分子各有哪些分子结构或者化学性质上的特征?}
    
    \begin{itemize}
        \item 凡是其分子结构具有“多羟基的醛或者酮”的特征的,都称为糖类化合物。不能被水解生成更小糖类分子的糖类物质称为单糖。重要的单糖常见的是葡萄糖和果糖。
        \item 不溶于水,而溶于丙酮、氯仿、乙醚等有机溶剂的分子统称为脂质分子。 许多脂质分子含长碳氢链或环状结构,从而使它们的水溶性很差。 脂类是脂肪酸和醇所形成的酯及其衍生物。
        \item 氨基酸,20种氨基酸中有8种不能由人体合成,必须从外界摄取,称为必需氨基酸。\begin{itemize}
            \item 组成蛋白质的常见氨基酸有20种。
            \item 侧链R不同,组成的氨基酸就不同。
            \item 与羧基相连的碳原子上都有一个氨基。
            \item L---型氨基酸(除甘氨酸Gly外)
            \item 有两种同分异构体,具有旋光性(除甘氨酸Gly外)
            \item 属两性电解质
            \item 属$\alpha$---氨基酸(除脯氨酸Pro为$\alpha$—亚氨基酸)
        \end{itemize}
        \item 核苷酸是组成核酸的基本单位。 由三部分组成:碱基、核糖或脱氧核糖,以及磷酸。 核酸的单体是核苷酸,每一个核苷酸都有一个戊糖(核糖或脱氧核糖)分子、一个磷酸分子和一个含氮碱基。 戊糖与碱基之间形成糖苷键, 戊糖与磷酸之间形成磷酸酯键。
    \end{itemize}
\end{example}\begin{example}\textbf{举例说明4类生物小分子的衍生物在生命过程中的作用。}

    糖、脂、氨基酸、核苷酸是四类生物小分子。\begin{itemize}
        \item 糖的衍生物有多糖,糖原(淀粉、纤维素),在生命过程中起到为生物体提供能量的作用。
        \item 脂质的衍生物有亚油酸和亚麻酸(甘油磷酸和鞘脂、萜类和类固醇等等都是脂类衍生物),它们是人体营养的必须脂肪酸,必须由食物提供,人体不能合成。
        \item 氨基酸的衍生物是蛋白质。构成生物体,如结构蛋白运输作用,如血红蛋白催化作用,如酶调节作用,如胰岛素免疫作用,如抗体运动作用,如肌纤维中的肌球蛋白和肌动蛋白控制作用,如阻遏蛋白。
        \item 核苷酸的衍生物核酸。遗传信息的载体(DNA)及基因表达中介(mRNA),核糖体的结构部分(rRNA),合成多肽时的氨基酸载体(tRNA),基因调控功能(miRNA,siRNA),少数RNA有自催化功能,即酶的作用。
    \end{itemize}
\end{example}\begin{example}\textbf{举例说明生物小分子在工业、农业或者医药上的应用价值。}
    
    在一些亚洲国家,在谷氨酸钠盐在烹饪中用作助鲜剂,从而形成味精生产工业。工业上水解淀粉获得葡萄糖,用于食品工业、医药工业等。
\end{example}\begin{example}\textbf{什么是必需氨基酸,什么是必须脂肪酸?}
    
    \begin{itemize}
        \item 20种氨基酸中有8种不能由人体合成,必须从外界摄取,称为必需氨基酸。
        \item 必需脂肪酸,人体不能合成,必须靠食物提供的不饱和脂肪酸。必需脂肪酸缺乏将影响人体免疫功能、生长发育、皮肤健康以及人类早期生命发育过程中脑及视网膜的发育等功能。
    \end{itemize}
\end{example}\begin{example}\textbf{试举出三种维生素的各方面的资料。}
    
    人体不能合成,必需从食物中取得,需要量极少,但是生命活动所必需的多种有机小分子,统称为维生素。\begin{description}
        \item[维生素A(视黄醇)] 功能:与视觉有关,并能维持粘膜正常功能,调节皮肤状态。帮助人体生长和组织修补,对眼睛保健很重要,能抵御细菌以免感染,保护上皮组织健康,促进骨骼与牙齿发育。\\ 缺乏症:夜盲症、眼球干燥,皮肤干燥及痕痒。\\ 主要食物来源:红萝卜、绿叶蔬菜、蛋黄及肝。
        \item[维生素B1(硫胺素)]功能:强化神经系统,保证心脏正常活动。促进碳水化合物之新陈代谢,能维护神经系统健康,稳定食欲,刺激生长以及保持良好的肌肉状况。\\ 缺乏症:情绪低落、肠胃不适、手脚麻木、脚气病。\\ 主要食物来源:糙米、豆类、牛奶、家禽。 
        \item[维生素C(抗坏血酸)]功能:对抗游离基、有助防癌;降低胆固醇,加强身体免疫力,防止坏血病。\\ 缺乏症:牙龈出血,牙齿脱落;毛细血管脆弱,伤口愈合缓慢,皮下出血等。\\ 主要食物来源:水果(特别是橙类),绿色蔬菜,蕃茄,马铃薯等。
        \item[维生素D]功能:协助钙离子运输,有助小孩牙齿及骨骼发育;补充成人骨骼所需钙质,防止骨质疏松。\\ 缺乏症:小孩软骨病、食欲不振;腹泻等。\\ 主要食物来源:鱼肝油,奶制品,蛋。 
        \item[维生素E(生育酚)]功能:抗氧化剂、有助防癌;生育相关。\\ 缺乏症:红血球受破坏,神经受损害,营养性肌肉萎缩,不育症,月经不调,子宫机能衰退等等。\\ 主要食物来源:植物油、深绿色蔬菜、牛奶、蛋、肝、麦、及果仁。
    \end{description}
\end{example}\begin{example}\textbf{蛋白质、核酸和多糖3类生物大分子中,连接单体的各是什么样的化学键?}
    
    \begin{itemize}
        \item 蛋白质的单体---氨基酸。一个氨基酸的$\alpha$-羧基与另一个氨基酸的$\alpha$-氨基之间脱水缩合形成肽键。
        \item 核酸的基本结构单位是核苷酸。每个核苷酸是由一个五碳糖(核糖或脱氧核糖)、一个含氮碱基(嘌呤或嘧啶)和磷酸结合而成。核酸就是由许多核苷酸聚合而成的生物大分子。磷酸二酯键。
        \item 多糖,淀粉 糖原等同聚多糖,单体为葡萄糖,糖苷键连接。
    \end{itemize}
\end{example}\begin{example}\textbf{什么是生物大分子的高级结构? }
    
    \begin{itemize}
    \item 蛋白质的空间结构
        \begin{itemize}
        \item 一级结构:氨基酸序列
        \item 二级结构:部分肽链发生卷曲和折叠,这种卷曲和折叠是靠肽链中的羧基和氨基间的氢键维持的。包括$\alpha$-螺旋,β-折叠。
        \item 三级结构:指整条肽链盘绕折叠形成一定的空间结构形状。如纤维蛋白和球状蛋白。
        \item 四级结构:蛋白质的四级结构是指各条肽链之间的位置和结构。所以,四级结构只存在于由两条肽链以上组成的蛋白质。
        \end{itemize}
        蛋白质分子的二、三、四级结构统称为蛋白质的高级结构。
    \item 核酸的空间结构\begin{itemize}
        \item 一级结构:线性多核苷酸链中4种不同核苷酸的排列次序,或“核苷酸序列”。
        \item 高级结构:DNA和RNA差别很大。\begin{itemize}
            \item DNA二级结构即双螺旋结构,DNA三级结构是指DNA链进一步扭曲盘旋形成超螺旋结构,DNA的四级结构---DNA与蛋白质形成复合物。
            \item RNA大分子由三种:mRNA,tRNA 和rRNA.tRNA二级结构为三叶草型, tRNA的三级结构为倒L形。
        \end{itemize} 
    \end{itemize}
    \item 多糖的空间结构\begin{itemize}
        \item 一级结构:多糖的单糖残基的组成、排列顺序、相邻单糖残基的连接方式、异头物的构型及糖链有无分支、分支的位置和长短等。
        \item 二级结构:多糖骨架链间以氢键结合所形成的各种聚合体,关系到多糖分子中主链的构象,不涉及侧链的空间排布。
        \item 三级结构:多糖链一级结构的重复序列,由于糖残基中的羟基、羧基、氨基以及硫酸基之间的非共价相互作用,导致有序的二级结构空间形成有规则而粗大的构象。
        \item 四级结构:多糖链间非共价键结合形成的聚集体。
    \end{itemize}
        多糖分子的二、三、四级结构统称为多糖的高级结构。
    \end{itemize}
\end{example}\begin{example}\textbf{简述蛋白质变性和水解的差别。}
    
    \begin{description}
        \item[变性] 蛋白质的正常的物理化学性质发生改变,生物学活性丧失。
        \item[水解] 蛋白质在酸性、碱性、酶等条件下发生水解,蛋白质的水解中间过程,可以生成多肽,但水解的最终产物都是氨基酸。
    \end{description}
\end{example}\begin{example}\textbf{简述DNA双螺旋模型的要点。}
    
    DNA分子是由两条脱氧核糖核酸长链互以碱基配对相连而成的螺旋状双链分子。 DNA主要存在于细胞核的染色质中,线绿体和叶绿体中也有,是遗传信息的携带者。\begin{itemize}
        \item 两条反向平行的核苷酸链共同盘绕形成双螺旋,糖---磷酸---糖构成螺旋主链 \item 两条链的碱基都位于中间,碱基平面与螺旋轴垂直 \item 两条链对应碱基呈配对关系:\ce{A=T C#G} \item 螺旋直径20A,螺距34A,每一螺距中含10bp的DNA\item 双螺旋可以看作是DNA的二级结构,而DNA的三级结构的形成需要蛋白质帮助。
    \end{itemize}
\end{example}\begin{example}\textbf{简述3类生物大分子在生命过程中各有哪些重要功能。}
    
    \begin{itemize}
        \item 糖类:糖类中只有多糖(淀粉、纤维素、糖原)。\begin{itemize}
            \item 淀粉:存在于植物细胞中,是植物细胞的能源物质(或说储能物质)。
            \item 纤维素:存在植物细胞中,是构成植物细胞细胞壁的主要物质。
            \item 糖原:存在于动物细胞,是动物细胞的能源物质。
        \end{itemize}
        \item 蛋白质:均为大分子物质,主要有构成细胞成分(如血红蛋白和细胞膜)、作为酶进行催化、运输物质(如血红蛋白携氧)、免疫作用(抗体)、调节作用(部分激素)。
        \item 核酸:包括脱氧核苷酸和核糖核苷酸,均可作为细胞的遗传物质。
    \end{itemize}
\end{example}
\section{第二章}\begin{example}\textbf{比较原核生物与真核生物的特征,列举它们的主要类群。}
    
\end{example}\begin{example}\textbf{什么是流动镶嵌模型?你是否知道其他有关生物膜的结构模型?}
    
    20世纪70年代提出的流动镶嵌模型概括了生物膜的结构特征。\begin{itemize}
        \item 生物膜的基本框架是甘油磷脂和鞘脂形成的脂双层。甘油磷脂和鞘脂都有“一个具有极性的头”和“两条非极性的尾巴”。在水环境中,分子自发形成脂双层泡:两层这样的脂质分子拼在一起,它们的非极性的尾巴相互靠拢,一层脂分子的“极性头”朝外,朝向周围的水环境,另一层脂分子的“极性头”朝内,朝向泡内的水环境。
        \item 蛋白质镶嵌或者挂靠在脂双层的框架中。
        \item 脂分子和蛋白质分子均具有动态特征。
    \end{itemize}
    对生物膜的深入研究的基础上,又陆续提出一些新的假说,如脂筏模型。
\end{example}\begin{example}\textbf{列表整理集中重要的细胞器的结构特征、功能和相关代谢途径。}

    \begin{description}
        \item[内质网] 靠近细胞核外侧,由单层生物膜折叠而成\begin{itemize}
            \item 基本类型:糙面内质网和光面内质网
            \item 功能:蛋白质的合成、脂质的合成、蛋白质的修饰和新生多肽的折叠与组装
            \item 信号假说
        \end{itemize}
        \item[高尔基体] 远离核的一组小囊和小泡,单层生物膜\begin{itemize}
            \item 功能:蛋白质修饰与加工(糖基化等),蛋白质的分选、蛋白质和脂质的运输和蛋白质分泌等
        \end{itemize}
        \item[溶酶体] 是胞质中一类包着多种水解酶的小泡 ,从高尔基体断裂而来。\\ 溶酶体的功能:\begin{itemize}
            \item 消化细胞内吞的食物,为细胞提供营养\item 清除衰老的细胞器\item 防御功能\item 乳腺和蝌蚪尾巴,靠溶酶体吞噬
        \end{itemize}
        \item[线粒体] 由双层膜的内膜折叠而成\begin{itemize}
            \item 形态结构:有外膜、内膜、脊、基质和膜间隙等组成
            \item 主要功能:线粒体是细胞进行氧化呼吸,产生能量的地方,在线粒体中进行的代谢途径主要有氧化磷酸化和参与脂肪酸代谢
        \end{itemize}
        \item[叶绿体] \begin{itemize}
            \item 形态结构:由双层膜组成,包括:基粒、类囊体、内膜和外膜
            \item 功能:光合作用
        \end{itemize}
    \end{description}
\end{example}\begin{example}\textbf{列举酶的作用特点和酶活性调节种类。}

    酶的催化特点:催化剂可以加快化学反应的速度,酶是生物催化剂,它的突出优点是: 催化效率高、专一性强、可以调节。
    
    酶活性的调节:\begin{itemize}
        \item 酶活性的调节---共价调节\\ 酶蛋白分子和一个基团形成共价结合,结果使酶蛋白分子结构发生改变,使酶活性发生改变。这种调节酶活性的方式称作酶的共价调节。例如,酶与磷酸根的结合。
        \item 酶活性的调节---变构调节\\ 有的调节物结合在酶的其他部位(非活性部位),结合后导致酶蛋白的构象改变,酶的活性变化,或不利于催化,或更有利于反应。这种调节方式称为酶的变构调节。\\ 可以接受变构调节的酶通常是多亚基酶。
    \end{itemize}
\end{example}\begin{example}\textbf{查阅有关光反应和暗反应的早期工作,整理出人们对光合作用逐步形成完整认识的过程。}

    光合作用的探究历程: \begin{enumerate}
        \item 普利斯特利的实验\begin{itemize}
            \item 普利斯特利没有认识到光在植物更新空气中的作用,而将空气的更新归因于植物的生长。\item 由于当时科学发展水平的限制,没有明确更新气体的成分。
        \end{itemize}
        \item 萨克斯的实验\begin{itemize}
            \item 该实验设置了自身对照,自变量为光的有无,因变量是颜色变化(有无淀粉生成)。\item 该实验的关键是:饥饿处理,以使叶片中的营养物质消耗掉,增强了实验的说服力。为了使实验结果更明显,在用碘处理之前应用热酒精对叶片进行脱绿处理。\item 本实验除证明了光合作用的产物有淀粉外,还证明光是光合作用的必要条件。
        \end{itemize}
        \item 恩格尔曼实验\\ 结论:叶绿体是光合作用的场所,光合作用过程能产生氧气。\\ 实验点评:\begin{itemize}
            \item 设置极细光束和黑暗、完全曝光和黑暗两组对照。\item 自变量为光照和黑暗,因变量为好氧菌聚集的部位。
        \end{itemize}
        \item 鲁宾和卡门的实验(\ce{H_{2} ^{18}O + CO2 -> ^{18}O2 \qquad H2O + C ^{18}O2 -> O2})\begin{itemize}
            \item 该实验设置了对照,自变量是标记物质(\ce{H2^{18}O,C^{18}O2}),因变量是\ce{O2}的放射性。\item 鲁宾和卡门的同位素标记法可以追踪CO2和H2O中的C,H,O等元素在光合作用中的转移途径。
        \end{itemize}
    \end{enumerate}
\end{example}\begin{example}\textbf{生物氧化对生命活动的意义?}
    
    生命活动的每个环节都要消耗能量,包括:营养物质进入细胞,废物排出细胞,细胞合成各种大小生物分子,肌肉运动等等。就整个生命世界来看,能量的最初来源是太阳能。绿色植物和光合细菌用太阳能来固定二氧化碳,生成糖类等有机物,然后通过食草食肉动物等环环相扣的食物链获得能量,固定有机化合物中的化学能。寄生腐生的真菌或细菌等也是利用有机物中的化学能。细胞通过生物氧化利用食物分子中的化学能。
\end{example}\begin{example}归纳糖酵解和柠檬酸循环两条代谢途径的要点。
    
    糖酵解途径是体内葡萄糖代谢最主要的途径之一,也是糖、脂肪和氨基酸代谢相联系的途径。由糖酵解途径的中间产物可转变成甘油,以合成脂肪,反之由脂肪分解而来的甘油也可进入糖酵解途径氧化。丙酮酸可与丙氨酸相互转变。\begin{itemize}
        \item 基本途径糖酵解在胞液中进行,其途径可分为两个阶段。第一阶段从葡萄糖生成2个磷酸丙糖。第二阶段由磷酸丙糖转变成丙酮酸,是生成ATP的阶段。\begin{enumerate}
            \item 第一阶段包括4个反应:\begin{itemize}
                \item 葡萄糖被磷酸化成为6-磷酸葡萄糖。此反应由己糖激酶或葡萄糖激酶催化,消耗1分子ATP\item 6-磷酸葡萄糖转变成6-磷酸果糖\item 6-磷酸果糖转变为1,6-二磷酸果糖。此反应由6-磷酸果糖激酶-1催化,消耗1分子ATP\item 1,6-二磷酸果糖分裂成二个磷酸丙糖。
            \end{itemize}
            \item 第二阶段由磷酸丙糖通过多步反应生成丙酮酸。在此阶段每分子磷酸丙糖可生成1分子\ce{NADH+,H+}和2分子ATP,ATP由底物水平磷酸化产生。\begin{itemize}
                \item 1,3-二磷酸甘油酸转变成3-磷酸甘油酸时产生一分子ATP。\item 磷酸烯醇型丙酮酸转变成丙酮酸时又产生1分子ATP,此反应由丙酮酸激酶催化。\item 丙酮酸接收酵解过程产生的1对氢而被还原成乳酸。乳酸是糖酵解的最终产物。
            \end{itemize}
        \end{enumerate}
        \item 关键酶糖酵解途径中大多数反应是可逆的,但有3个反应基本上不可逆,分别由己糖激酶(或葡萄糖激酶),6-磷酸果糖激酶-1和丙酮酸激酶催化,是糖酵解途径流量的3个调节点,所以被称为关键酶。在体内,关键酶的活性受到代谢物(包括ATP,ADP)和激素(如胰岛素和胰高血糖素)等的周密调控。
        \item 生理意义糖酵解最重要的生理意义在于迅速提供能量尤其对肌肉收缩更为重要。此外,红细胞没有线粒体,完全依赖糖酵解供应能量。神经、白细胞、骨髓等代谢极为活跃,即使不缺氧也常有糖酵解提供部分能量。
    \end{itemize}
    柠檬酸循环,是需氧生物体内普遍存在的代谢途径,因为在这个循环中几个主要的中间代谢物是含有三个羧基的柠檬酸,因此得名。三羧酸循环是三大营养素(糖类、脂类、氨基酸)的最终代谢通路,又是糖类、脂类、氨基酸代谢联系的枢纽。
    在三羧酸循环中,反应物葡萄糖或者脂肪酸会变成乙酰辅酶A。这种“活化醋酸”(一分子辅酶和一个乙酰基相连),会在循环中分解生成最终产物二氧化碳并脱氢,质子将传递给辅酶烟酰胺腺嘌呤二核苷酸(\ce{NAD+})和黄素腺嘌呤(FAD),使之成为\ce{NADH+},\ce{H+}和\ce{FADH2}。\ce{NADH+,H+}和\ce{FADH2}会继续在呼吸链中被氧化成\ce{NAD+}和\ce{FAD},并生成水。这种受调节的“燃烧”会生成ATP,提供能量。真核生物的粒线体和原核生物的细胞质是三羧酸循环的场所。
\end{example}\begin{example}\textbf{比较分析物质进出细胞的几种方式。}

细胞必须不断的与周围环境进行物质交换,从环境中获取所需要的营养物质,同时排除代谢产物和废物,才能维持细胞内环境的相对稳定,维持细胞的生命活动。 

细胞膜是具有高选择性的通透屏障,因此物质的跨膜运输与细胞膜的结构及活性有密切相关。

物质跨膜运输方式的分类:
\begin{itemize}
    \item 小分子物质:\begin{itemize}
        \item 被动运输---高浓度一侧向低浓度一侧\item 主动运输---低浓度一侧向高浓度一侧
    \end{itemize}
    \item 大分子或颗粒状物质:胞吞和胞吐。
\end{itemize}
\begin{description}
    \item[被动运输] 被动运输是物质顺浓度梯度(由高浓度到低浓度的方向)的跨膜运输,是不消耗能量的运输。\\ 被动运输包括:简单扩散和易化扩散(也称协助扩散)\begin{itemize}
        \item 简单扩散\begin{itemize}
            \item 简单扩散是物质通过分子随机运动直接通过细胞膜。
            \item 由于膜的脂质双层基本结构和膜脂运动而产生的间隙很小,所以只有脂溶性物质和直径小于1.0 nm且不带电荷的物质,如乙醇、乙醚、\ce{O2}、\ce{CO2}、水等可通过简单扩散而过膜运输。
            \item 渗透作用是指水分子的简单扩散。 
        \end{itemize}
        \item 易化扩散\begin{itemize}
            \item 易化扩散(协助扩散)是葡萄糖、氨基酸、核苷酸等水溶性有机小分子和一些无机离子,在膜转运蛋白介导下顺浓度梯度的被动运输。 \item 膜转运蛋白可分为两类:载体蛋白和通道蛋白。
        \end{itemize}
    \end{itemize}
    \item[主动运输] 主动运输是物质逆浓度梯度的跨膜运输,需要消耗很多能量。 主动运输普遍存在于各类生物的细胞中。
    \item[载体运输] 载体运输是需要载体介导的物质跨膜运输,载体是存在于膜上的膜转运蛋白。上述小分子的跨膜运输方式中,易化扩散和主动运输都属于载体运输。
    \item[胞吞作用和胞吐作用] \begin{itemize}
        \item 胞吞作用通过细胞膜内陷,将胞外固体颗粒或液体包入,并从细胞膜上脱落下来形成胞吞泡,从而将这些物质输入细胞。
        \item 胞吐作用是将细胞内的分泌泡或其它膜泡中的物质排出胞外的过程。在这个过程中分泌泡或膜泡的膜与细胞膜发生融合,破裂后,才能将胞内物质排出。
    \end{itemize}
\end{description}
\end{example}
\begin{example}\textbf{DNA合成应有哪些成分参与?}
    
    \begin{itemize}
        \item DNA的生物合成要求4种脱氧核苷三磷酸作为原料
        \item 需要引物即可以和模板DNA的5’端配对的一小段RNA,新加上去的脱氧核苷酸是加在引物上,使引物链由5向’3’延伸
        \item 需要一条DNA单链做模板,引物链遵循碱基互补配对原则先结合到模板链上去
        \item 反应必须由DNA聚合酶催化
    \end{itemize}
\end{example}\begin{example}\textbf{试述蛋白质合成过程中转录和翻译的主要过程。}
    
    \begin{description}
        \item[转录] 即为mRNA的生物合成。现在细胞核内,以双链DNA中的一条链为模板,指导合成碱基序列与之互补的信使RNA。mRNA生物合成需要一个关键酶---RNA聚合酶。
        \item[翻译] 即为在mRNA指导下合成蛋白质。信使RNA从细胞核出来,进入细胞质,以mRNA为模板,指导合成特定的氨基酸序列的某种蛋白。
    \end{description}
\end{example}
\section{第三章}
\begin{example}\textbf{分析细胞周期、DNA合成和染色体出现之间的关系。}
    
    细胞周期分为分裂期和合成期。合成期细胞活跃的进行DNA合成。染色体只出现在细胞分裂过程中。在染色体出现之前,DNA复制已经完成。
\end{example}\begin{example}\textbf{列表说明有丝分裂各个时期的特征性事件。}
    
    \begin{itemize}
        \item 分裂间期:细胞内进行着大量的生物合成。如DNA、蛋白质的合成期
        \item 前期: 染色质浓缩,折叠,包装,形成光镜下可见的染色体,每条染色体含两条染色单体。
        \item 中期: 核膜消失,染色体排列在赤道板上。
        \item 后期: 姐妹染色单体分开,被分别拉向细胞两侧。
        \item 末期: 重新形成核膜,染色体消失。
        \item 细胞质分裂:胞质形成间隔,最终分开为两个细胞。
    \end{itemize}
\end{example}\begin{example}\textbf{简述减数分裂的几个特征}
    
    \begin{itemize}
        \item 减数分裂发生在产生生殖细胞的过程中。\\ 生殖细胞包括卵细胞和精子细胞。它们的遗传物质总量仅为体细胞的一半,称为n细胞。由2n的体细胞产生n的生殖细胞,需要经过减数分裂。
        \item 减数分裂后,细胞中染色体数目减少一半。
        \item 减数分裂可以分为两个阶段:\begin{itemize}
            \item 第一次减数分裂:DNA复制一次,细胞分裂一次。 \item 第二次减数分裂:DNA不复制,细胞再分裂一次
        \end{itemize}
        \item 结果,子细胞染色体数目减半,遗传物质总量由2n变为n。
    \end{itemize}
    总之,减数分裂就是 DNA 复制一次,细胞连续分裂两次,结果由一个2n细胞分出4个n细胞(生殖细胞)。
\end{example}\begin{example}\textbf{什么是细胞分化?什么是分化决定因子?}
    
    \begin{description}
        \item[细胞分化] 一个或者一种细胞,其分裂增殖产生的后代细胞,在形态结构和功能上相互间不同,并与亲代细胞也不相同
        \item[分化决定因子] 就是那些在细胞分化过程中那个起着关键作用,使细胞向着特定的方向分化的一类蛋白或者核酸。
    \end{description}
\end{example}\begin{example}\textbf{比较两类干细胞的特征和异同。}
    
    干细胞是体内存在的一类具有自我更新和分化潜能的细胞,可区分为胚胎干细胞和成体干细胞。
    \begin{description}
        \item[胚胎干细胞的特征] 形态特征与早起胚胎细胞相似,体积小核大,核质比高,有一个或者多个突出的核仁。有自我更新和无限增殖的能力。具有发育全能性或者多能性。有培养细胞的所有特征,可在体外培养、克隆、冻存及进行遗传操作而不失其多能性。
        \item[成体干细胞特征] 存在于儿童和成人组织中的具有多向分化潜能的一类细胞。成体干细胞与胚胎干胞相比它来源丰富,取材相对容易,避免了伦理方面的问题。
    \end{description}
\end{example}\begin{example}\textbf{说说你对个体衰老的理解。}
    
    生物个体的衰老,意味着身体各部分功能的衰退,渐渐发展到不能执行功能,有以下特征:
    \begin{itemize}
        \item 衰老受遗传控制。
        \item 衰老受中枢神经系统影响,忧虑和烦躁使衰老加快。
        \item 衰老受环境影响,营养不良或者热量摄入过多都会明显加速衰老。
        \item 适当的体力劳动,可使身体衰老减慢,经常的脑力活动,可使脑功能衰退减慢。
        \item 各个人身体衰老的进程快慢相差很大。
    \end{itemize}
\end{example}\begin{example}\textbf{简述细胞衰老的特征。}
    
    \begin{itemize}
        \item 细胞核体积增大,核膜呈现内折,染色质凝集程度增加。 2.线粒体体积膨胀,数量减少。\item 细胞膜结构从液晶变为凝胶状或者固体状,膜的渗透增加,胞内其他生物膜系统也发生变化。\item 细胞骨架系统改变。\item 蛋白质合成改变。
    \end{itemize}
\end{example}\begin{example}\textbf{列表比较细胞坏死和细胞凋亡。}
    
    \begin{tabular}{c|c}
        细胞坏死&细胞凋亡\\\hline  细胞变圆,与周围细胞脱开&细胞外形不规则变化\\ 核染色质凝聚&溶酶体破坏\\ 细胞膜内陷&细胞膜破裂\\ 细胞分为一个一个小体&胞浆外溢\\ 被周围细胞吞噬&引起周围炎症反应
    \end{tabular}
\end{example}\begin{example}\textbf{与正常细胞相比,癌细胞有哪些特点?}
    
    \begin{itemize}
        \item 脱分化\item 无限增殖\item 失去接触抑制\item 对生长因子的需求降低\item 细胞骨架紊乱\item 细胞表面和黏附性质变化
    \end{itemize}
\end{example}\begin{example}\textbf{细胞癌变有哪些可能原因?应如何减少或者防止细胞癌变?分析你所知道的治疗癌症的方法?}
    
    引起细胞癌变的因素称为致癌因子。包括物理致癌因子、化学致癌因子、病毒致癌因子。
    
    减少和防止癌变就应该有合理的饮食,健康的生活作息,积极锻炼。
    
    手术是一项主要的治疗手段,放射治疗和化学治疗。
\end{example}
\section{第四章}
\begin{example}\textbf{多细胞生物如何进行细胞间通讯?细胞通讯的意义是什么?}
    
    \begin{itemize}
        \item 细胞通讯的关键是信号分子与靶细胞信号分子受体相识别,启动细胞一系列反应。信号分子是指细胞产生的能影响其他细胞或者自身的化学物质,如激素、神经递质等。
        \item 细胞能选择性相应细胞信号,主要由于细胞表面或者细胞内存在着信号分子受体。按信号分子是否能进入靶细胞,可将细胞信息传递分为跨膜信号转导和胞内受体信号传递。
        \item 细胞通讯的意义是极大的增强了生物适应复杂环境的能力。
    \end{itemize}
\end{example}\begin{example}\textbf{你认为可以从什么方面寻找证据来推断某一类细胞以何种方式通讯?}
    
    按照信号分子是否能进入靶细胞或者信号分子的性质(水溶性和脂溶性)
\end{example}\begin{example}\textbf{神经元以什么方式传递信息?}
    
    神经冲动在突触的传导,神经冲动是神经元产生的动作电位。包括静息电位和动作电位。神经递质是神经元产生的化学物质。
\end{example}\begin{example}\textbf{内分泌细胞以什么方式传递信息?}

    激素分泌的调节是负反馈调节。
\end{example}\begin{example}\textbf{多细胞动物的整体调节功能是由哪些系统负责的?}
    
    神经系统、激素系统和免疫系统协同完成。
\end{example}\begin{example}\textbf{从离子和电荷角度说明什么是静息电位,神经元如何维持其静息电位?}
    
    神经元在静息状态时,即未接受刺激,未发生神经冲动时,细胞膜内积聚负电荷,细胞膜外积聚着正电荷,膜内外存在着-70 mV 电位差。
    
    造成静息电位的原因很多。其中一个主要原因是细胞膜上存在\ce{Na+,K+}---ATP 泵,这是一个具有 ATP水解酶活性的蛋白质,每水解一个 ATP分子,可将 3个\ce{Na+} 泵向膜外,同时将 2个\ce{K+}泵向膜内。
\end{example}\begin{example}\textbf{解释什么是动作电位?它如何沿神经元传导?}
    
    当神经细胞受到刺激时,细胞膜的透性急剧变化,大量正离子(主要是\ce{Na+})由膜外流向膜内,使膜两侧电位从-70mV, 一下子跳到 +35mV,这就是动作电位。动作电位的产生,意味神经冲动的产生。

    神经传导就是神经冲动沿神经纤维按顺序发生,高等动物体内的神经系统中神经冲动只向一个方向传递,来自感觉神经元的神经冲动总是向着中枢神经系统传导,而源于中枢神经系统的兴奋则向效应器方向传导。
\end{example}\begin{example}\textbf{脊椎动物的中枢神经系统由哪些部分组成?}
    
    脑和脊髓。
\end{example}\begin{example}\textbf{人类大脑皮质的结构大致是什么样的?}
    
    可分为感觉区、运动区和联络区。感觉区分为视觉区、听觉区和体觉区。联络区是最难解释的功能和特征---记忆、语言、学习和个性等的发源地。
\end{example}\begin{example}\textbf{对大脑皮质的功能定位是如何获得的?}
    
    从效应上,整个机体每一处在感觉皮质上都有一个标志,体觉区就可以用身体的各部分表示,这种图称为“小人图”。例如各部位接收的感觉信息按比例标示在顶叶的感觉皮质上。
\end{example}\begin{example}\textbf{大脑皮质的体觉映射图与所代表的躯体部位的比例并不相称,这反映了什么?}
    
    越灵活的躯体部位在映射图上占得比例越大,并不是按比例分配。
\end{example}\begin{example}\textbf{大脑皮质更大的区域是非对映区,这些区域称为什么?它们可能有什么功能?}
    
    称为联络区。可能的功能是记忆、语言、学习和个性等的发源地。大体来说,人的左脑主要负责以严密和次序为特点的逻辑思维,右脑负责以节奏、图形和想象为特点的形象思维。
\end{example}
\begin{example} \textbf{举例说明什么是反射?}
    
    膝跳反应。
    \end{example}
\begin{example} \textbf{损伤下丘脑会影响记忆吗?损伤小脑会影响什么记忆?}
    
    损伤下丘脑不会影响记忆,损伤小脑会影响语言记忆。
    \end{example}
\begin{example} \textbf{在某些动物中观察到的学习模式是否存在于人类中?}

    是存在于人类中。例如“一只鸭子下河,十只鸭子也下河”,生动表现了不受其他过程影响的、习惯化的学习动作。在人类中有“见了草绳就是蛇”,这些都是属于非联合型学习。
\end{example}
\begin{example} \textbf{胰岛素分泌不足会引起机体什么症状?为什么?}

    胰岛素分泌不足引起糖尿病。糖尿病是一种发病率较高的代谢疾病,典型的症状就是持续的高血糖和尿糖以及“三多一少”,即多食、多饮、多尿和体重减轻。胰岛素结合胰岛素受体后,使受体细胞质部分的络氨酸蛋白激酶活性区域激活,催化自身磷酸化,进而催化下游蛋白质磷酸化,最终引起细胞效应。胰岛素分泌不足就会影响这个过程。
\end{example}
\begin{example} \textbf{如果一个人甲状腺素分泌不足,分析病因时要考虑哪些因素?}

    考虑到下丘脑和垂体前叶。下丘脑释放促甲状腺激素释放激素(TRH),TRH促进垂体前叶释放较多的促甲状腺素,促甲状腺素又刺激甲状腺释放甲状腺素。同时甲状腺素对垂体前叶和下丘脑激素分泌有负反馈调节作用。激素的分泌量还受生物体内外环境的改变的影响。
\end{example}
\begin{example} \textbf{什么是抗原?}

    能与抗体结合或者淋巴细胞表面受体结合,进而引起免疫反应的物质。
\end{example}
\begin{example} \textbf{哪类细胞产生抗体?为什么从血清中提取的抗体不是单一类型的(即多克隆抗体)?}
    
    B细胞分泌抗体。抗原通常是由多个抗原决定簇组成的,由一种抗原决定簇刺激机体,由一个B淋巴细胞接受该抗原所产生的抗体称之为单克隆抗体。由多种抗原决定簇刺激机体,相应地就产生各种各样的单克隆抗体,这些单克隆抗体混杂在一起就是多克隆抗体,机体内所产生的抗体就是多克隆抗体;除了抗原决定簇的多样性以外,同样一类抗原决定簇,也可刺激机体产生IgG、IgM、IgA、IgE和IgD等五类抗体。
\end{example}
\begin{example} \textbf{简述T细胞是如何识别抗原的。}

    识别抗原,产生淋巴因子,增殖分化成效应T细胞( 消灭靶器官靶细胞)和记忆细胞。 成熟T细胞表面具有特异性识别抗原并与之结合的分子结构,称为T细胞抗原受体,该受体可以与抗原特异性结合,也就达到了识别抗原的效果。
\end{example}
\begin{example} \textbf{什么是主动免疫、被动免疫、自动免疫?}

    给人体接种经过减毒或者灭活的抗原,人体内就会产生针对抗原的抗体,免疫接种用于预防传染病称为主动免疫。用注射抗体的方法称为被动免疫。机体接受抗原刺激后,因两者相互作用而建立起来的特异性免疫应答称为自动免疫。
\end{example}
\begin{example} \textbf{为什么人们很少有一套相同的主要组织相容性抗原?}

    在不同种属或同种不同系别的个体间进行组织移植时,会出现排斥反应,其本质是细胞表面的同种异型抗原诱导的一种免疫应答。这种代表个体特异性的同种异型抗原称移植抗原或组织相容性抗原,其中能引起强而迅速排斥反应的抗原称主要组织相容性抗原。主要组织相容性复合体就是存在于脊椎动物表面的自我识别标志,控制免疫细胞之间的相互作用。
\end{example}
\end{document}
\section{第五章}
1.孟德尔学说的要点有哪些内容? 1)一对等位基因决定一种性状
2)等位基因可有显性、隐性之分。当一对等位基因处于杂合状态时,表达的基因为显性,不能表达的隐性。
3)在配子形成时各队等位基因彼此分离,独立随机的组合到不同的生殖细胞中去。
4.在杂交产生的第一代(F1)中,其基因型全部为杂合,而呈现显性基因控制的性状。在F2代中显、隐性的分离比为3:1。当考察N对基因时,N对性状分离比为(3:1)n
2.基因在染色体上定位的基本方法是什么?
减数分裂中,同源染色体的行为与等位基因的分离和自由组合存在着平行关系。
3.试述几个关键实验可以证明DNA是遗传的分子基础?
肺炎链球菌转化实验、放射性同位素噬菌体侵染细菌实验、DNA双螺旋模型
4.简要说明真核细胞基因结构的一般模式
5’上游侧翼区,其中有启动子和各种调控元件,是RNA聚合酶识别和起始工作的位置,并有调控基因表达的各是转录因子结合的位置。
转录区,内部又可区分出编码氨基酸序列的外显子和不编码氨基酸序列的内含子,还有前导序列和拖尾序列。 3下游侧翼区。
5.基因表达包含哪些主要步骤 转录过程: 剪接 加帽 加尾
翻译过程:肽链的延长 肽链的终止 翻译后加工
6.简述乳糖操纵子的结构和功能
操纵子(operon):由启动子、操纵基因和结构基因共同构成的基因簇单位,称为原核生物的操纵子。
启动子:RNA聚合酶结合的位点;
操纵基因:转录的开关,决定RNA聚合酶能否转录;
结构基因:编码-半乳糖苷酶(Z)、透性酶(Y)和硫半乳糖苷乙酰转移酶(A)的基因; 调节基因:在操纵子上游,其表达产物为阻遏蛋白; 乳糖+阻遏蛋白,改变阻遏蛋白的形状。
7.比较几种不同的基因突变
基因突变的类型有:点突变、移码突变、缺失突变 涉及一个碱基对的突变称为点突变。涉及插入或缺失1-2个碱基,造成在基因表达中遗传密码阅读时发生错误称为移码突变。若是DNA片段丢失,称为缺失突变。
8.比较几种不同的染色体结构改变
染色体结构的变异:
① 缺失(deletion):染色体丢失了一段。 ② 重复(duplication):染色体多了一个片段。 ③ 易位(translocation):非同源染色体之间的节段转移。 ④ 倒位(invertion):一条染色体上出现断裂,中间片段又旋转后重新连接起
来,造成这一片段上基因序列的颠倒,但是不改变染色体上的基因数目。
9.免疫球蛋白基因的重组有何生物学意义
免疫球蛋白可变区的蛋白质结构变化多端,使得人体的免疫球蛋白结构多种多样,得以应对大自然中成千上万种不同结构的抗原物质的侵袭。决定免疫球蛋白可变区蛋白质结构多样性的谜底,正是在于编码该蛋白的基因在成熟过程中所经历的特殊重组。重组的结果使得VJC区可形成多种格局的DNA序列组合,从而可能编码出许多种蛋白质结构。
10.简述基因工程操作的主要步骤 基因工程一般要经历5个基本步骤: 1)获取目的基因
两条途径:人工合成和从生物基因组DNA中直接分离得到。 2)目的基因和载体DNA在体外连接
用同一种限制性内切酶分别切割载体DNA和目的基因。在合适的条件下,载体质粒的黏性末端与目的基因的黏性末端因碱基互补配对而结合,再经适量的DNA连接酶作用,就形成了重组DNA分子。
3)将重组的DNA分子引入合适的宿主细胞内
重组DNA分子导入受体细胞的方法包括:转化和转导。 4)选择、筛选含有目的基因的克隆
检测的方法很多,一般根据载体的特征或者目的基因的特征。 5)培养、观察目的基因的表达
看目的基因能否在新的宿主细胞中定居下来,自我复制和表达,并稳定传代。
11.关于限制性内切酶、目的基因和载体,你知道哪些?
限制性内切酶是基因工程最常用的工具酶,主要来源于原核生物。能够识别DNA分子上特定的碱基序列,并在特定的位点将DNA分子切割开。限制性内切酶切口分为平末端和粘性末端。
获取目的基因两条途径:人工合成和从生物基因组DNA中直接分离得到。 基因工程载体是基因工程中作为外源基因运载体的DNA片段。特点有:
① 在宿主细胞中能独立自主的复制 ② 携带易于筛选的标记基因
③ 含有多种限制性内切酶的单一识别序列
④ 除必要序列外,载体要尽可能的小,便于导入细胞进行繁殖 ⑤ 使用安全
载体种类有:质粒载体、噬菌体载体、柯斯质粒载体、动物病毒载体、YAC载体。
12.简述人类基因组计划的要点和影响。 计划要点:
1.HGP的提出和实施
2.HGP的工作内容:人的24条染色体,各含一条DNA大分子长链,目前的DNA测序技术,每次读取不到1000bp长度的片段,要先把DNA大分子打断称为几万甚至几十万的片段,分别测序然后再拼接组装。其中,标记作图是前提,测序技术是基础,序列组装是总结。
3.HGP的完成
人类基因组计划带来的深远影响:
对基因组的诠释,推动一系列的基础研究,开拓医学研究和实践的新局面,巨大的商机和利润,对社会伦理的冲击等。
\section{第六章}
1.什么是生物分类的五界系统?
五界系统将生物分成两个总界:原核生物总界和真核生物总界。原核生物总界里只有一个原核生物界。真核生物总界分为四个界:原生生物界、植物界、真菌界和动物界。
2.什么是双名法
每个物种的科学名称由两部分组成,第一部分是属名,属名是名词性质,且第一个字母大写,第二部分是种名,种名是形容词,带有修饰限定属名的意思,无需大写,种名后面还可以有定名者的姓名,有时定名者姓名可以省略。双名法的生物学名均应为拉丁文。在书面格式上,生物物种的学名都应用斜体字,表明是拉丁文,再不能用斜体字的情况下,则在学名下面划一道线,以示区别。
3.植物生活史有哪几种主要类型
植物生活史指种子植物的种子或非种子植物的孢子经过营养生长和生殖生长又形成新一代种子和孢子的整个生活历程。被子植物的生活史,展示了植物从种子萌发开始经幼苗、植株、开花、授精、形成合子直到发育成新的种子的过程,可分为两个阶段:
1.从合子到胚囊母细胞或花粉母细胞减数分裂前,细胞分裂为2n,可称为二倍体世代或无性世代。
2.从减数分裂开始到成熟胚囊或2-3个花粉细胞形成为止,仅含单倍染色体n,可称为单倍体世代或有性世代。
4.植物组织有哪些主要的类型
一般把植物组织分为两大类:分生组织和成熟组织。成熟组织又可以分为:薄壁组织、保护组织、机械组织、输导组织、分泌组织等不同组织。
5.简述植物各大类群的特点和分类
一般将自然界中植物分为低等植物和高等植物两大类。低等植物是指无根、茎、叶分化,通常生活于水中或潮湿地方的植物。生殖器官常是单细胞的,有性生殖形成的合子不经过胚直接萌发成新植物体。低等植物可分为藻类和地衣。高等植物一般有根、茎、叶分化,有性生殖形成的合子经过胚的阶段再发育成植物体,包括苔藓、蕨类、裸子植物和被子植物。
6.植物在地球生态系统中和在人类生活中起什么样的作用 植物给地表大气层带来氧气;
为地球上一切生命提供能源和食物来源
植物参与土壤的形成,为一切生物创造栖息地 植物是自然界物质循环的重要环节 植物是人类亲密的朋友。
7.动物界分为哪些主要门类
通常把众多的动物门类划分为无脊椎动物和脊椎动物两大类。
8.以人体为例,简述消化系统、呼吸系统、循环系统、排泄系统、激素系统和神经系统的结构和功能。
消化系统:人的消化管全长可达9米,分为口腔、咽、食管、胃、小肠、大肠和直肠等,最后经肛门通向体外。口腔里的牙齿是最坚硬的,有力的咀嚼器官,舌为味觉器官。咽位于消化管和呼吸道交叉处,前接口腔,后接食管和喉。食物经咽进入食管。食管是食物从口腔进入胃的通道。胃位于腹腔上方,其前端以僨门与食管相连,中间为胃底和胃体,后端以幽门与小肠相连。小肠分为十二指肠、空肠和回肠,全长可达6米,主要是消化和吸收器官。大肠可分为盲肠、阑尾、升结肠、横结肠、降结肠。乙状结肠和直肠,主要功能为吸收水分、电解质及形成粪便。
呼吸系统:呼吸系统包括鼻、咽、喉、气管、支气管和肺。前五者在呼吸过程中仅为气体进出肺的通道,因此称为呼吸道。肺是气体交换中最重要的部位。肺位于密闭的胸腔中。右肺分为上、中、下三叶,左肺分为上、下两叶,海绵状构造,其表面包以一层光滑而湿润的胸膜。支气管入肺后一再分支,最后成终末支气管和呼吸细支气管。呼吸细支气管末端膨大为囊状,称为肺泡管,肺泡管壁向外凸出形成半球形的盲囊,即为肺泡。肺泡由一层扁平上层细胞和若干弹性组织构成,外面与丰富的微血管网紧紧相贴,吸入的空气在此处与微血管内的血液进行气体交换,所以肺泡是气体交换的结构和功能单位。
循环系统:血液循环系统由心脏、动脉、毛细血管、静脉和血液组成,血液流动主要靠心脏的搏动。人的血液循环分为体循环、肺循环和冠状动脉循环。体循环是从左心室泵出动脉血,经各级主动脉,再到全身各器官组织的毛细血管,进行物质交换后变成静脉血,回流到右心房。肺循环从右心室输出静脉血,经肺动脉到肺泡毛细血管,经气体交换后,变成动脉血,再由肺静脉回到左心房。血液在毛细血管中与心肌壁组织实行物质交换和气体交换,然后流入小静脉、冠状静脉,最后流入右心房,这就是冠状动脉循环。
排泄系统:排泄过程通常依赖于排泄器官来执行。肾是脊椎动物主要的渗透调节和排泄器官。以人为例,排泄器官由肾、输尿管、膀胱和尿道所组成。肾单位是肾的功能单位,位于皮质和髓质内,每一肾单位均由肾小体和肾小管组成。
激素系统:内分泌系统由内分泌腺和内分泌细胞组成。人体主要的内分泌腺有垂体、甲状
腺、甲状旁腺、肾上腺、胰岛和性腺等。人体重要的功能调节系统,在体液调节中起主要作用,它与神经系统紧密联系,相互配合,共同调节体内各种生理功能。
神经系统:脑和脊髓为中枢神经系统,从脑发出的脑神经和从脊髓发出的脊神经属于周围神经系统。能感受体内外环境的变化,相应的调节人的多方面活动,对内能协调各器官、系统的活动,使它们相互配合形成一个整体,对外使人和动物能适应外部环境的各种变化。
9.简述卵子形成过程和月经周期
卵原细胞分化为初级卵母细胞,然后完成第一次减数分裂,形成两个细胞,即次级卵母细胞和第一极体。极体不能受精发育,但是可以分裂。从卵巢排出的卵,其实就是次级卵母细胞,它进入输卵管之后,若在24小时内与精子相遇,便完成了第二次减数分裂,分裂的结果与第一次相同,产生一个有效的卵子和一个不能受精的极体。
女性在性成熟之后,大约28天有一个次级卵母细胞发育成熟,此期间子宫内膜浅层发生周期性的剥脱和出血,由于这一时间与月球绕地球1周的时间相似,故称为月经。
10.微生物有哪些特点?它包括哪些类型? 个体小,表面积大。 吸收多,代谢力强。 繁殖快,生长旺盛。 分布广,容易变异。
微生物包括原核细胞型、真核细胞型和非细胞型。
11.试比较细菌、真菌的异同点 一、细胞结构
细菌和真菌都具有细胞结构属于细胞型生物,在它们的细胞结构中都具有细胞壁、
细胞膜、细胞质,但却存在诸多不同,具体表现在:一是细胞壁的成分不同:细菌细胞壁的主要成分是肽聚糖,而真菌细胞壁的主要成分是几丁质。二是细胞质中的细胞器组成不同:细菌只有核糖体一种细胞器;而真菌除具有核糖体外,还有内质网、高尔基体、线粒体、中心体等多种细胞器。三是细菌没有成形的细胞核,只有拟核;真菌具有。 二、生物类型
一是就有无成形的细胞核来看:真菌有核膜包围形成的细胞核,属于真核生物;细菌没 有核膜包围形成的细胞核,属于原核生物。
二是就组成生物的细胞数目来看:真菌既有由单个细胞构成的单细胞型生物(如酵母
菌),也有由多个细胞构成的多细胞型生物(如食用菌、霉菌等);细菌全部是由单个细胞构成,为单细胞型生物。四是真菌细胞核中的DNA与蛋白质结合在一起形成染色体(染色质);细菌没有染色体,其DNA分子单独存在。 三、细胞大小
真核细胞较大,直径一般为10μm~100μm;而原核细胞较小,直径一般为1μm~10μm。 四、增殖方式
真菌为真核生物,细胞的增殖主要通过有丝分裂进行,因真菌种类的不同其个体增殖方式主要有出芽生殖(如酵母菌)和孢子生殖(食用菌)等方式;细菌是原核生物,为单细胞
型生物,通过细胞分裂而增殖,具有原核生物增殖的特有方式---二分裂。 五、名称组成
尽管在真菌和细菌的名称中都有一个菌字,但细菌的名称中一般含有:球、杆、弧、螺 旋等描述细菌形态的字眼,只有乳酸菌例外(乳酸杆菌),而真菌名称中则不含有。
13.什么是病毒?它有哪些特点
病毒由一个核酸分子(DNA或RNA)与蛋白质(Protein)构成或仅由蛋白质构成(如朊病毒)。特点是:1.个体极小,只能通过细菌滤器,只有在电子显微镜下才能够看到。2.无细胞结构,仅含一种类型的核酸,其主要成分是蛋白质和核酸。3.无完整的酶系,不能进行独立的代谢活动。4.严格的活细胞内寄生,以复制的方式增殖。5.在离体条件下,以无生命的化学大分子状态存在,并可形成结晶。6.对抗生素不敏感,但是对干扰素敏感。
14.什么是朊病毒?其发现有何理论意义和实践意义
是一类不含核酸而仅由蛋白质构成的可自我复制并具感染性的因子。朊病毒可引起人和其他哺乳动物的多种神经系统疾病。人利用微生物的潜力是无穷的,它们在解决人类面临的各种危机中发挥不可替代的独特作用。
15.生物多样性包括哪些方面内容
生物多样性包括遗传多样性、物种多样性和生态系统多样性。
16.什么是生物资源的直接价值和间接价值
直接价值就是指人们直接收获和使用生物资源所形成的价值。间接价值包括非消费使用价值、选择价值、存在价值和科学价值。
17.分析世界生物多样性锐减的原因
过度的采集、砍伐和捕捞。生境破坏。水体污染。
18.分析《生物多样性公约》中规定对保护生物多样性的作用
《生物多样性公约》中生物多样性被定义为“来自陆地、海洋和其他水生生态系统及其所构成生态综合体等所有来源的、活的生物体中的变异性;这就包括了物种内、物种间和生态系统等多层次的多样性”。保护生物多样性可以从遗传多样性、物种多样性以及生态系统多样性3个层次。同时体现了保护生物多样性的迫切性。保护生物多样性的基本途径就是:就地保护和迁地保护。
\section{第七章}
1.生态学是一门研究什么问题的科学
生态学作为一门研究生物及其环境之间关系的科学。
2.什么叫生态因子
生态因子指对生物有影响的各种环境因子。一般将生态因子分为非生物因子和生物因子两大类。非生物因子包括温度、湿度、风、日照等理化因素;生物因子包括同种和异种的生物个体。
3.生态系统有哪些基本特征
组成特征---群落时间是生态系统的核心。不同生态系统由不同的生物群落与其生存的环境共同组成。
地区特征---系统具有一定的地区特点和空间结构。
时间特征---生态系统都随着时间的推移,从形成到稳定,到衰落的演变过程。
开放特征---生态系统都是不同程度的开放系统,不断地从外界输入能量和物质,经过转换变为输出,从而维持系统的有序状态。
平衡特征---处于稳定阶段的生态系统,系统内各生物种内、种间及生物与环境之间在结构和功能方面具有复杂的动态平衡特征。
4.什么是食物链
生物成员之间以食物营养关系彼此联系起来的序列,称为食物链。
5.简述生态系统中能量流动的“十分之一定律”及其意义
在自然生态系统中,营养级之间能量的转化,大致只有1/10转移到下一个营养级以组成生物量,9/10被生物体代谢损耗掉,因为称为“十分之一定律”。
6.从碳、氮、磷等元素的循环特点看人类活动对全球变化的作用
农业生态系统中碳的同化随工业辅助能的投入而增加,但是动植物残体和残余物中的碳是以有机物的形式返回土壤,还是以二氧化碳的形式返回大气,则会显著影响着系统的碳素循环。如果这些非经济产品的有机物不能通过秸秆还田和沤制使用有机肥的返回到土壤,则土壤微生物的碳源将会减少,土壤有机质含量可能降低,长期处于这种状况,将会造成地力的持续衰退。
工业固氮是一种具有明显的生态负效应的产业,消耗能源,生产和运输成本高,并且可能造成环境污染。
人类开采磷矿石,制造和使用磷肥、农药和洗涤剂,以及排放含磷的工业废水和生活污水,都对自然界的磷循环发生影响。
7.造成生态平衡失调的原因包括哪些 1.生物种类成分的改变 2.森林和植被的破坏 3.环境破坏
8.如何正确认识人类与环境的关系
自然环境为人类提供了丰富多彩的物质基础和活动舞台。人类在诞生以后很长的时间里,只是自然事物的采集者和捕食者,那时候尚未对环境造成污染。所谓的环境问题都是人口增长、滥采滥捕造成的。随着人类的生产活动和消费活动领域扩大,人类改造环境的作用越来越明显,影响越来越大。产生比较严重的环境问题如水土流失、水旱灾害和沙漠化等。人类和环境的关系主要是通过人类的生产和消费活动而表现出来的。人类与环境是对立统一的关系。
9.人类社会如何应对日益恶化的环境问题
保护生态与环境从改变观念起步。同时需要不懈的协调的努力:控制人口数量,控制人口素质。深入研究,掌握规律。制定并严格执行法律法规。参加国际协作。
\section{第八章}
1.你怎么样理解生物技术的定义
生物技术是应用自然科学及工程学的原理,依靠微生物、动物、植物体作为反应器将物料进行加工以提供产品来为社会服务的技术。
2.生物技术的发展有哪几件标志性事件
1953年,沃森和克里克提出DNA双螺旋模型 1973年,完成第一例大肠杆菌基因工程 1978年,胰岛素在大肠杆菌表达成功 1982年,第一例转基因鼠成功 1990年,人类基因组计划启动 1997年,第一只克隆羊诞生
3.简述三大类生物材料的主要特征
天然生物材料:组成基本就是生物体的元素组成。天然生物材料是复合材料,有活性的生物材料往往是细胞与细胞外物质组成的复合材料。它的生物活性赋予它比任何传统人工材料都高超的自组装分级结构和优异性能。根据天然生物材料组分不同可分为五大类:蛋白结构、结构多糖、生物软组织、生物复合纤维和天然生物矿物。
生物医用材料:它是指用于医疗、能植入生物体或能与生物组织相结合的材料。它首先必须具有与其使用部位相适应的机械、生化性能。医用材料的种类有:金属材料、陶瓷材料、高分子材料、生物复合材料和生物衍生材料。
仿生和组织工程材料:仿生智能材料不仅能感知外界环境或内部状态的变化,而且能通过材料自身或外界的某种反馈机制,及时将材料的一种或多种性能改变,作出恰当的响应,这就是仿生智能材料研究的內容。组织工程材料是以工程学和生命科学的原理和方法研制出的生物材料,用来代替缺损组织、器官,并行驶一定的生理功能。它提供了器官及组织再建的一种真实的可能性。
4.什么是仿生学
仿生学是一门建立在多学科边缘上的综合性科学。
5.仿生学研究中有哪几种必不可少的模型 数学模型、生物模型、技术模型
6.举例说明你所知道的仿生学在生物工程技术中的应用 信息仿生:电子蛙眼、电子鼻
控制仿生:根据蝙蝠的“回声探测器”制造出供盲人使用的“探路仪”和“盲人眼镜” 拟态仿生:坦克的迷彩伪装
建筑仿生:模仿植物的设计的筒形叶桥
化学仿生:昆虫信息激素及其类似物在农业生产中的应用
整体仿生:机器人
7.分析生物传感器的特点,以及与别种传感器的关联
传感器就是能够感受各种被测的物理的、化学的和生物的信息,并按照一定规律转换成可用的信号输出的器件或装置。而生物传感器就是运用固定化的生物成分或者生物体本身作为分子识别元件,选择性作用于目标物,通过物理或者化学信号转换元件捕捉分子识别元件与目标物之间相互作用的产物或效应。并最终以电信号的形式显示出来,从而实现对生物化学物质的分析检测。生物传感器的选择性、灵敏度、响应特性等性能与整体组成相关。具有极高的灵敏度和特异性。
8.以葡萄糖传感器为例分析酶传感器的特点
酶传感器是由固定化酶膜与信号转换元件结合而成,常用的酶传感器有酶电极、酶场效应管、光纤光学酶传感器、热敏电阻酶传感器等。 葡萄糖传感器:
现在测定血糖的葡萄糖传感器为例,葡萄糖氧化酶电极是研究最早的酶电极。它利用葡萄糖氧化酶(GOD)催化葡萄糖反应为基础,结合电化学测量技术实现对葡萄糖的检测。 \ce{C6H_{12}O6 + H2O+ O2 -> C6H_{12}O7 + H2O2}
反应中消耗的O2量或者生成H2O2的量都与被测物质浓度呈正比,这样出现以氧或者过氧化氢为电子传递的酶电极。
用氧电极作为基础电极的生物传感器采用铂作为阴极,在阴极施加负电压,氧气透过酶膜和透气膜,在阴极发生还原反应,还原电流与氧的分压呈正比,阴极上发生的电极如下:
\ce{O2 + 2H2O + 4e- -> 4OH-}
氧电极为基础电极的优点在于其良好的特异性,酶膜和透气膜起到选择透过作用,只有氧气能有透过到达阴极并发生反应,干扰物质必须透过这两层膜,并能在所能施加的电压条件下发生反应才能干扰。同时由于透气膜和酶膜分开,使得酶的特异性不会因为氧电极电解产物受到损害。缺点是容易受到周围环境中氧的分压变化影响,起始电流比较大,故灵敏度较低。
9.试述免疫传感器和DNA生物传感器的工作原理
免疫传感器就是利用生物体内抗原-抗体之间特异性反应来实施微量物质的测定,是一种高灵敏度、高选择性生物传感器。
DNA生物传感器是对特定DNA序列及其变异的识别,严格遵守碱基配对的原则,并将这种相互作用转换为可检测信号。
10.比较几种生物能源,你认为哪种更有优势
植物能源、生物气体燃料和微生物能源中,微生物���源的优势更大。
11.分析海洋生物资源的重要性
海洋中蕴藏着丰富的资源,具有为人类提供食物、医药和能源的巨大潜力。海洋生物中还
具有许多能够改善人类生存质量和政府人类所面临的各种疾病的生物活性物质,对提高人类健康有重要的作用。海洋植物资源种类繁多,深入开发利用,海洋有可能成为第二粮仓。
12.简述海洋生物工程的几个重要领域
海水养殖:海水养殖农牧化、生物技术育种育苗产业、生物技术饲料产业、海水养殖病害综合防治产业。
海洋活性物质与海洋药物产业:海洋微生物是生理活性物质的重要源泉
海洋生态和环境的生物修复:海洋污染的微生物技术处理、海区富营养化的生物修复技术、滨海滩涂的生态修复技术、赤潮生物清除技术。
13.什么是发酵工程
发酵工程,是指采用现代工程技术手段,利用微生物的某些特定功能,为人类生产有用的产品,或直接把微生物应用于工业生产过程的一种新技术。
14.分析现代和近现代发酵工程的主要特征
1.微生物具备的特征:微生物具有种类多、繁殖速度快、分布广、容易培养、代谢能力强以及容易变异的特点。
2.原料无需精制:通常以糖质或淀粉质为主,加入少量无机或有机氮源,只要不含对细胞有毒的物质,一般无需精制。
3.发酵设备的选择及工艺条件的控制:通常在常温常压下进行,反应比较平稳。
15.举例说明代谢调控在发酵中的应用 黄色短杆菌生物合成赖氨酸的发酵调控。
在黄色短杆菌中,天冬氨酸激酶是赖氨酸和苏氨酸合成途径中的关键酶。它是一个变构酶,具有两个变构部位,可以分别与终产物赖氨酸和苏氨酸结合,其活性受赖氨酸和苏氨酸的影响。当只有一种终产物过量,并与酶结合时,酶活性只受部分影响。当两者终产物同时过量,同时与酶的两个变构部位结合时,酶活性被完全抑制。
16.举例说明发酵工艺的操作和控制对发酵效果会带来什么样的影响。
温度对发酵的影响及控制:温度升高可增加微生物生长繁殖速度,缩短生长周期。另一方面由于温度的升高,导致微生物酶变性而杀死微生物。温度影响微生物的代谢方向,培养基的物理性质。适当的温度是指在该温度下最适于菌的生长或发酵产物的生成。
Ph对发酵的影响及控制:发酵过程中的ph是微生物在一定环境下生命代谢活动的综合指标。
CO2对发酵的影响:CO2是微生物的代谢产物。增加搅拌可减低发酵液中CO2浓度。增加通气量可增加容氧,同时排出发酵液中的CO2。
\section{第九章}
1.什么是生命伦理学
伦理学是研究人们的道德思想,道德行为,道德规范的学问,也称道德哲学。顾名思义,生命伦理学所研究的是生命科学技术提示或涉及的伦理道德问题,包括道德见解,决定,行为,政策等诸多方面。
2.生命伦理学有哪些核心精神和基本原则 行善原则 自主原则 不伤害原则 公正原则
3.举例说明生命伦理学几个热点问题以及你自己对这些问题的看法 (1)人的克隆
(2) 胚胎干细胞研究 (3) 人类基因组计划 (4) 转基因食品和药物
(5) 辅助生殖和 “安乐死” (6) 文化差异和利益冲突
4.你认为应如何推动生命伦理学的研究和应用。
\item 优化知识结构、变革伦理观念。生命伦理工作需要努力学习有关知识,适应科技的 发
展。尤其要向科学学习,向科学家学习。不懂科学,空谈伦理,无济于事,害人害己。 \item 与科学家、法学家密切合作。这样不仅可以充分吸收科学家和法学家的知识和智慧,
使生命伦理更好地适应和促进科学的发展,而且可以化为法律和法规,依靠国家政权的力量来实施。
\item 加强与公众的沟通、听取公众的意见。随着健康和生命价值的上升,以及自主意识 的
增强,公众对生命伦理问题更加敏感,更加关注。没有公众的理解、支持和参与,生命伦理难以发挥作用。
\item 建立和健全伦理审查委员会,生命伦理审查委员会,包括医院伦理委员会,生命科学
研究机构及各国家的伦理委员会,是生命论理学建制化和体制化的重要表现,也是生命伦理发挥作用的组织保证。伦理委员会必须结构合理和规范,是有代表性、公正性和权威性。有职有权。


\section{思考题}
\begin{enumerate}
    \item 阐述生命具有的基本特征是什么?
    \item 如何理解21世纪是生命科学的世纪?
    \item 生命科学研究的基本方法有哪些?
    \item 何谓常量元素、微量元素?常见的常量元素与哪些?
    \item 生物小分子包括哪些类别?什么是必需脂肪酸?什么是必需氨基酸?
    \item 天然油脂中脂肪酸的结构有什么特点?
    \item 何谓维生素?根据溶解性质的不同,维生素分为哪几类?举例。 
    \item 生物大分子包含哪几类?它们所具有的共同特征是什么?维持生物大分子空间结构的主要作用力有哪些?
    \item 什么是蛋白质的一、二、三、四级结构?何谓蛋白质的变性、变构和复性?
    \item 请阐述DNA分子双螺旋结构模型的主要特征及其生物学功能。
    \item RNA的类别及其作用。核酸的变性、复性和分子杂交。
    \item 简述多糖链的结构、特性和功能。
    \item 比较细胞的两大类型和结构特点
    \item 何谓生物膜?简述生物膜的流动镶嵌模型的结构特点及其生物学功能。
    \item 列举酶的作用特点和酶活性的调节类型。
    \item 归纳葡萄糖氧化分解产生能量的过程。
    \item 简述生物大分子和小分子进入细胞的方式。
    \item 何谓DNA 的半保留复制?
    \item 简述蛋白质在生物体内的合成(转录和翻译)过程。
    \item 简述细胞内代谢途径区室分布的意义。
    \item 真核细胞分裂有哪两种方式?分别出现在那一类细胞?
    \item 何谓细胞周期?细胞周期分为哪几个阶段?
    \item 试阐述有丝分裂各个时期的特征性事件。
    \item 简述减数分裂的几个主要特征。
    \item 什么是细胞分化?什么是干细胞?细胞分化的潜能有哪几种情况?
    \item 阐述细胞衰老具有的特征及其原因。
    \item 什么是细胞坏死、细胞凋亡?
    \item 与正常细胞相比,癌细胞具有哪些主要特征?细胞癌变有哪些可能的原因?
    \item 什么是细胞信号分子?它包括哪些种类?
    \item 细胞信息传递有那两种方式?膜受体有哪几类?
    \item 何谓反射、反射弧?反射弧包括那几部分?
    \item 何谓突触?它有哪几种类型?
    \item 何谓学习和记忆?它们有哪几类?学习和记忆的细胞与分子机制是什么?
    \item 何谓激素?人体激素有哪几类?简述激素信息传递到大细胞后的级联信号放大效应。
    \item 何谓免疫、抗原、抗体、免疫应答?
\end{enumerate}
\end{document}