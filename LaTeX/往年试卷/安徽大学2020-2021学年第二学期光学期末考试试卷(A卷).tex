\documentclass[UTF8]{article}
\usepackage{ctex,geometry,enumitem,color}
\usepackage[compact]{titlesec}
\everymath{\displaystyle}
\geometry{a4paper,left=1cm,right=1cm,top=0.8cm,bottom=1.2cm}
% \pgfplotsset{width=10cm,compat=1.16}
\setenumerate[1]{itemsep=0pt,partopsep=0pt,parsep=\parskip,topsep=10pt}
\setitemize[1]{itemsep=0pt,partopsep=0pt,parsep=\parskip,topsep=5pt}
\setdescription{itemsep=0pt,partopsep=0pt,parsep=\parskip,topsep=5pt}
\titleformat{\section}{\rmfamily}{\thesection. }{0em}{}
\titleformat{\subsection}{\rmfamily}{\thesubsection. }{0em}{}
\newcommand{\m}{\mathrm{m}}
\newcommand{\cm}{\mathrm{cm}}
\newcommand{\mm}{\mathrm{mm}}
\newcommand{\nm}{\mathrm{nm}}
\newcommand\ptsMulti[3]{ \small($#1\!\!$小题$\!\!\times #2\!\!$分$\!\!=\!\!#3\!\!$分)}
\newcommand\pts[1]{ \small{(#1分)}}

\title{安徽大学2020-2021学年第二学期光学期末考试试卷(A卷)}
\author{出卷人:杨群}
\date{}
\begin{document}
    \maketitle

    \section{\large\textbf{简答题}\ptsMulti{4}{5}{20}}

    \subsection{简述Huygens-Fresnel原理}
    \noindent \textcolor{red}{解: 波前$\Sigma$上的每个面元$\mathrm{d}\Sigma$都可以被看作新的发出次波的振动中心.空间某点P的振动是所有这些次波的相干叠加.}

    \subsection{简要说明双折射现象及形成原因}
    \noindent \textcolor{red}{解: 双折射是光束入射到各向异性晶体中,被分解为两束光而沿不同方向折射的现象. 形成原因是两束折射光在晶体内的传播速度不同.}

    \subsection{简述Malus定律}
    \noindent \textcolor{red}{解: 强度为$I_0$的线偏振光通过检偏器后,出射光的强度为$I=I_0 \cos^2 \theta$, 其中$\theta$是检偏器与偏振方向的夹角.}

    \subsection{如何区分圆偏振光和自然光}
    \noindent \textcolor{red}{解: 在入射光前依次放置$\frac{\lambda}{4}$波晶片和偏振片,旋转偏振片一周. 若出射光有消光位置,则该入射光为圆偏振光,否则为自然光.}

    \noindent \textcolor{red}{\small 问: 如何区分入射光的五种偏振态?\\ 解:将偏振片放入光路并慢慢旋转一周. (1)若出射光强变化且有消光位置,则入射光是线偏振光.\\ (2)若出射光强不变,则入射光为自然光或圆偏振光. •在入射光前依次放置$\frac{\lambda}{4}$波晶片和偏振片,旋转偏振片一周.\\ •若出射光有消光位置,则该入射光为圆偏振光,否则为自然光.\\ (3)若出射光强变化但无消光位置,则入射光为部分偏振光或椭圆偏振光.\\ •将偏振片旋至光强最强位置,在偏振片后放置$\frac{\lambda}{4}$波晶片,将光轴旋至与偏振片透振方向平行.\\ •将偏振片由前面移至后面,旋转偏振片一周. •若出射光有消光位置,则入射光为椭圆偏振光,否则为部分偏振光.}

    \section{一玻璃半球曲率半径为$R$,折射率$n=1.5$,半球平面边镀银.一物高$h$,置于曲面顶点前$2R$处.求此光具组所成的最后的像在何处. \pts{10}}
    \noindent \textcolor{red}{解: 在折射情况下有$\frac{n'}{s'}+\frac{n}{s}=\frac{n'-n}{r}$,横向放大率$V=-\frac{ns'}{n's}$.在反射情况下$\frac{1}{s'}+\frac{1}{s}=-\frac{2}{r}$,横向放大率$V=-\frac{s'}{s}$.\\ 其中$s',s,r$分别为像距,物距和球面曲率半径,$n',n$分别为像方和物方折射率.\\ 
    第一次(折射):代入$n'=1.5,n=1,s=2R,r=R$,得$s'=\infty$,即成无穷远处的正立实像;\\ 第二次(反射):显然反射为反方向(从玻璃向空气)的平行正立实像;\\ 第三次(折射):代入$n=1.5,n'=1,s=\infty,r=-R$,有$s'=2R$,即映回原处的倒立等大实像.}

    \section{设平凸透镜和平板玻璃良好接触,两者间空气间隙形成Newton环.用波长$\lambda=589\nm$的光照射,测得从中心算起的第$k$个暗纹直径$r_k=0.70\mm$,第$k+10$个$r_{k+10}=1.70\mm$.求:(1)平凸透镜凸面的曲率半径$R$;\\ (2)若形成Newton环的空气间隙中充满折射率$n=1.33$的水,则上述两暗纹直径各变为多大?\pts{10}}
    \noindent \textcolor{red}{解: (1)$R=\frac{r^2_{k+10}-r^2_k}{10\lambda}=407.47\mm $; (2)$r'_k=\frac{r_k}{\sqrt{n}}=1.47\mm , r'_{k+10}=\frac{r_{k+10}}{\sqrt{n}}=0.61\mm $}

    \section{在Fresnel圆孔衍射实验中,光源距圆孔$R=1.5\m$,波长$\lambda=630\nm$,接受屏距圆孔$b=6.0\m$,圆孔半径$\rho$从$0.5\mm$开始扩大.求最先两次出现亮斑和暗斑时圆孔的半径$\rho_{l1},\rho_{l2}$和$\rho_{d1},\rho_{d2}$.\pts{15}}
    \noindent \textcolor{red}{解: $\rho_k=\sqrt{\frac{Rb}{R+b} k\lambda}, k$为奇数时为亮斑,为偶数时为暗斑. $\rho_1=0.870\mm>\rho$,因此$\rho_{l1}=\rho_1=0.870\mm,\\ \rho_{l2}=\rho_3=1.506\mm,\rho_{d1}=\rho_2=1.230\mm,\rho_{d2}=\rho_4=1.740\mm$.}

    \section{单缝Fraunhofer衍射实验中,垂直入射有波长$\!\!\lambda_1=400\nm\!\!$和$\!\!\lambda_2=760\nm$.已知单缝宽$\!\!a=1.0\times 10^{-2}\cm$,透镜焦距$f=50\cm$.(1)求两种光的一级衍射明纹中心间距;(2)若用光栅常数$d=1.0\times 10^{-3}\cm$的光栅替换单缝,其他条件同上,求两种光的一级主极大间距.\pts{15}}
    \noindent \textcolor{red}{解: (1)一级衍射班的位置对应$\alpha=\tan \alpha$的第一个根,其中$\alpha=\frac{\pi a}{\lambda}\sin \theta$,解得$\alpha=1.43\pi, \theta=\arcsin(1.43\frac{\lambda}{a})$.\\ 分别代入$a=1.0\times 10^{-2}\cm,\lambda_1=400\nm,\lambda_2=760\nm$,即有$\theta_1=0.0057,\theta_2=0.0109,l_1=f\theta_1=2.86\mm,l_2=f\theta_2=5.43\mm,\Delta l=2.57\mm$.\\ (2)一级主极大的位置对应$\beta=\pi$,其中$\beta=\frac{\pi d}{\lambda}\sin \theta$,即对应$\theta=\arcsin \frac{\lambda}{d} $.\\ 分别代入$d=1.0\times 10^{-3}\cm,\lambda_1=400\nm,\lambda_2=760\nm$,即有$\theta_1=0.0400,\theta_2=0.0760,l_1=f\theta_1=20.00\mm,l_2=f\theta_2=38.04\mm,\Delta l=18.03\mm$.}

    \section{通过一理想偏振光片观察部分线偏振光(由自然光和线偏振光混合而成)的强度,当从最大光强方位转过$30^\circ$时,光强变成$7/8$.求:(1)此部分偏振光种线偏振光和自然光强之比;(2)入射光的偏振度;(3)旋转偏振片时最小透射光强和最大透射光强之比;(4)当偏振光从最大光强方位转过$60^\circ$时的透射光强和最大光强之比.\pts{15}}
    \noindent \textcolor{red}{解: (1)由Malus定律,设自然光强和线偏振光强分别为$I_1,I_2$,则有$\frac{I_1}{2}+I_2 \cos^2 \frac{\pi}{6}=\frac{7}{8}(\frac{I_1}{2}+I_2)$,解得$\frac{I_1}{I_2}=2$. (2)$P=\frac{I_{\mathrm{max}}-I_{\mathrm{min}}}{I_{\mathrm{max}}+I_{\mathrm{min}}}=\frac{1}{3}$. (3)$\frac{I_{\mathrm{min}}}{\mathrm{max}}=\frac{I_1/2}{I_1/2+I_2}=\frac{1}{2}$. (4)$\frac{I_1/2+I_2 \cos^2 \frac{\pi}{3}}{I_1/2+I_2}=\frac{5}{8}$.}

    \section{在两块主截面夹角为$\frac{\pi}{3}$的Nicol棱镜中插入一块主截面平分上述夹角的$\frac{\lambda}{4}$波片,光强为$I_0$的自然光入射之.\\ 求(1)通过$\frac{\lambda}{4}$波片后光的偏振态;(2)通过第二个Nicol波片的光强.\pts{15}}
    \noindent \textcolor{red}{解: (1)沿第一个Nicol棱镜透振方向振动的线偏振光.\\ (2)$I_2=A^2_{e2}+A^2_{o2}+2A_{e2}A_{o2}\cos \delta$.代入$\!A_{e2}\!=\!A_1\cos \alpha\cos \beta,A_{o2}=A_1\sin \alpha\sin \beta, \delta=\pm \frac{\pi}{2}, A_1^2=\frac{I_0}{2}$,得$I_2=\frac{5}{8}A_1^2=\frac{5}{16}I_0$.}

\end{document}