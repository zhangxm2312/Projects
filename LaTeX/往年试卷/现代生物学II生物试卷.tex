\documentclass[UTF8]{article}
\usepackage{ctex,mhchem,geometry,enumitem}
\geometry{a4paper,left=2cm,right=2cm,top=2cm,bottom=2cm}
% \pgfplotsset{width=10cm,compat=1.16}
\setenumerate[1]{itemsep=0pt,partopsep=0pt,parsep=\parskip,topsep=10pt}
\setitemize[1]{itemsep=0pt,partopsep=0pt,parsep=\parskip,topsep=5pt}
\setdescription{itemsep=0pt,partopsep=0pt,parsep=\parskip,topsep=5pt}
\title{现代生物学II期中试卷}
\begin{document}
    \maketitle
    \section{名词解释题}
    \begin{enumerate}
        \item \textbf{相对性状}:同种生物的同一种性状的不同表现类型。
        \item \textbf{等位基因}:指位于一对同源染色体相同位置上控制同一性状不同形态的基因。
        \item \textbf{伴性遗传}:性染色体上基因所控制的性状的遗传常常与性别相关联的现象。
        \item \textbf{基因}:可以表达产生多肽链或RNA链的一段DNA序列
        \item \textbf{基因重组}:指在生物体进行有性生殖的过程中,控制不同性状的基因重新组合。
        \item \textbf{突变}:除基因重组以外,遗传物质的任何可检测出的并能遗传的改变。
        \item \textbf{启动子}:RNA聚合酶识别、结合和开始转录的一段DNA序列。
        \item \textbf{载体}:基因工程中,作为外源基因运载体的DNA片段,能够携带外源基因转运到宿主细胞中去,进行复制和表达。
        \item \textbf{基因组}:携带遗传信息的遗传物质的总和。
        \item \textbf{基因工程}:按照人们的意愿把一种生物的某种基因提取出来,加以修饰改造。然后放到另一种生物的细胞里,定向地改造生物的遗传性状。
    \end{enumerate}
    \section{填空题}
    \begin{enumerate}
        \item 孟德尔学说包括\underline{分离定律}和\underline{自由组合定律}两条定律。
        \item 真核细胞的基因结构包括\underline{5'上游侧翼区}、\underline{编码区}、和\underline{3'下游侧翼区}三部分。
        \item 操纵子包括三个组分,即\underline{启动子}、\underline{操纵基因}和\underline{结构基因}。
        \item 突变涉及\underline{基因序列的改变}称为基因突变;涉及\underline{染色体结构和数目的改变}称为染色体畸变。从突变的诱因来看,基因突变区分为\underline{自发突变}和\underline{诱发突变}两大类;基因突变有多种类型,主要有\underline{点突变}、\underline{移码突变} \\ 和\underline{缺失突变}。染色体的畸变分为\underline{染色体结构变异}和\underline{染色体数目变异}两大类,前者包括\underline{缺失}、\underline{重复}、\underline{倒位}、\underline{易位}4种类型;后者包括\underline{整倍体改变}和\underline{非整倍体改变}。
        \item 基因工程载体主要有\underline{质粒载体}、\underline{噬菌体载体}、\underline{柯斯质粒载体}、\underline{动物病毒载体}和\underline{YAC载体}。
        \item 目的基因的获得主要有两条途径:\underline{人工合成}和\underline{从生物基因组DNA中直接分离得到}。
    \end{enumerate}
    \section{简答题}
    \subsection{简述哪几个关键实验可以证明DNA是遗传的分子基础?\protect\footnote{广为流传的生物思考题答案有误,按那个写会被扣分。}}
    \begin{enumerate}
        \item 肺炎双球菌转化实验
        \item 艾弗里证明DNA是遗传物质的实验
        \item 噬菌体侵染细菌实验
    \end{enumerate}
    \subsection{简述基因工程的基本操作步骤}\begin{enumerate}
        \item 目的基因的获得
        \item 目的基因与载体DNA在体外连接——构建重组DNA
        \item 将重组DNA分子导入受体(宿主)细胞
        \item 目的基因的表达和检测
        \item 培养、观察
    \end{enumerate}
    \subsection{简述人类基因组计划(HGP)的目标和工作内容\protect\footnote{此小题答案不是满分}}
    目标:完成人类核基因组的全部DNA测序。

    工作内容:先把DNA大分子打断为片段,分别测序,再组装起来。
    \begin{enumerate}
        \item 标记作图:作图方法有遗传作图和物理作图。
        \item 测序:DNA测序技术有化学降解法和链终止法。
        \item 组装
    \end{enumerate}
    \subsection{简述DNA重组的生物学意义}
    DNA重组作为正常的生理状况而大量存在于生命活动中。使得有性生殖产生的后代,基因组合得以多样化,从而对环境的适应能力大为增强。
    \section{论述题}
    \subsection{孟德尔学说的要点有哪些内容?}
    \begin{enumerate}
        \item 一对等位基因决定一种性状。
        \item 等位基因可有显性、隐性之分。当一对等位基因处于杂合状态时,表达的基因为显性,不能表达的基因为隐性。
        \item 在配子形成时各对等位基因彼此分离,独立随机地组合到不同的生殖细胞中去。
        \item 在杂交产生的第一代$F_1$中,其基因型全部为杂合,而呈现显性基因控制的性状,在$F_2$代中显、隐性的分离比为3:1。当考察$n$对基因时,$n$对性状之分离比则为$(3:1)^n$。
    \end{enumerate}
    \subsection{试论述真核生物基因表达的调控方式。}
    \begin{enumerate}
        \item 染色质水平和DNA水平的调控。真核生物的染色质具有复杂结构,且处于动态中,染色质结构的变化起着基因表达的调控作用。
        \item 转录水平的调控。在真核生物基因的结构中,有各种特定的DNA序列片段,其作用是专一地与各种调节蛋白相结合,称为调节元件。
        \item mRNA水平的调控。刚刚转录完成的mRNA通常要经过多种剪接,才变成作为翻译模板的成熟mRNA。整个 mRNA加工剪切的过程是mRNA水平基因表达调控的重要内容。
    \end{enumerate}
\end{document}