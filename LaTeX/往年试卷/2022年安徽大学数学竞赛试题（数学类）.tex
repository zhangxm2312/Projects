\documentclass[UTF8]{article}
\usepackage{ctex,amsmath,amssymb,amsthm,geometry,enumitem,bm,color}
\geometry{a4paper,left=1cm,right=1cm,top=0.5cm,bottom=1.5cm}
\setenumerate[1]{itemsep=0pt,partopsep=0pt,parsep=\parskip,topsep=10pt}
\setitemize[1]{itemsep=0pt,partopsep=0pt,parsep=\parskip,topsep=5pt}
\setdescription{itemsep=0pt,partopsep=0pt,parsep=\parskip,topsep=5pt}
\begin{document}
\title{2022年安徽大学数学竞赛试题(数学类)}
% \author{}
\date{2022.9.24}
\maketitle
\paragraph*{1.}有直线$$l_1:\frac{x-4}{1}=\frac{y-3}{-2}=\frac{z-8}{1};\qquad l_2:\frac{x+1}{7}=\frac{y+1}{-6}=\frac{z+1}{1}$$
(1)$l_1$与$l_2$是否异面;(2)求两直线任一点连线线段的中点轨迹的一般方程.(10分)

\paragraph*{2.}$A\in M_4(\mathbb{C}), \mathrm{tr}(A^k)=k (k=1,2,3,4)$,求$\det A$.(15分)

\paragraph*{3.}$f,g\in \mathbb{F}[x]$且$(f(x),g(x))=1$.设$M\in M_n(\mathbb{F}),A=f(M),B=g(M)$.\\ 证明:$ABX=0$的任一解能被表为$AX=0$和$BX=0$解的和.(10分)

\paragraph*{4.}有半正定矩阵$A\neq O$与正定矩阵$B$.\\ 证明:(1)存在可逆矩阵$P$使得$P^TAP$为对角矩阵且$P^TBP$为单位矩阵;(2)$\det A+B\geq \det B$.(15分)

\paragraph*{5.}证明:(1)$\displaystyle \lim_{n\to \infty}(n+1)!^{\frac{1}{n+1}}-n!^{\frac{1}{n}}$;(2)$\displaystyle \lim_{n\to \infty}\int_0^{\pi/2}\sin^{\sqrt{n}}x \mathrm{d}x$.(每小题5分,共10分)

\paragraph*{6.}$(a,+\infty)$上一致连续函数$f$的值域$R_f\subset (A,+\infty)$,以及$(A,+\infty)$上一致连续函数$g$.求证$g\circ f$是$(a,+\infty)$上一致连续函数.(10分)

\paragraph*{7.}求证$\displaystyle \sum_{n=2}^\infty \frac{\cos(nx)}{n\ln n}$在$(0,2\pi)$上内闭一致收敛,但不一致收敛.(10分)

\paragraph*{8.}$f$在$[0,+\infty)$上单调增且二阶可导,$F$在$\mathbb{R}$上非负,$\displaystyle \int_0^{+\infty}F(x) \mathrm{d}x$发散,且在$[0,+\infty)$上恒成立$f''(x)+F(f(x))\leq 0$.\\ 证明:$\displaystyle \lim_{x\to +\infty}f'(x)=0$.(10分)

\paragraph*{9.}$f,g$是在$[0,1]$上的非负连续函数,$h$是在$[0,1]^2$上的非负连续函数,且在$[0,1]$上恒成立$$g(x)=\int_0^1 h(x,y)f(y)\mathrm{d}y\qquad f(x)=\int_0^1 h(x,y)g(y)\mathrm{d}y$$
证明:(1)$\forall x\in [0,1]\exists \xi\in [0,1]:\dfrac{f(x)}{g(x)}=\dfrac{g(\xi)}{f(\xi)}$;(2)$f(x)\equiv g(x)$.(10分)
\end{document}
